%%
%% This is file `phd-pkgmanager.tex',
%% generated with the docstrip utility.
%%
%% The original source files were:
%%
%% phd-pkgmanager.dtx  (with options: `PKG')
%% ----------------------------------------------------------------
%% phd --- A package to beautify documents.
%% E-mail: yannislaz@gmail.com
%% Released under the LaTeX Project Public License v1.3c or later
%% See http://www.latex-project.org/lppl.txt
%% ----------------------------------------------------------------


\NeedsTeXFormat{LaTeX2e}[1994/12/01]%
\ProvidesFile{phd-packagemanager}[2015/1/13 v1.0 less preamble (YL)]%
\let\ltxtoday\today

\RequirePackage{fixltx2e}[2006/03/24]
\@ifundefined{c@chapter}{%
      \newcounter{chapter}
      \def\thechapter{\@arabic\c@chapter}
}{}
\ExplSyntaxOn
\clist_new:N \g_phd_packages_loaded_clist
\ExplSyntaxOff
\ExplSyntaxOn
\clist_new:N \g_phd_packages_not_found
\ExplSyntaxOff
\ExplSyntaxOn
\clist_new:N \g_phd_packages_loaded_by_others
\ExplSyntaxOff
\ExplSyntaxOn
\cs_new:Npn \save_symbol: #1
  {
    \cs_gset_eq:cc {orig#1} {#1}
    \cs_undefine:c {#1}
  }
\ExplSyntaxOff
\ExplSyntaxOn
\cs_new:Npn \restore_symbol: #1 #2
  {
    \cs_gset_eq:cc {#1#2} {#2}
    \cs_gset_eq:cc {#2} {orig#2}
 }
\ExplSyntaxOff

\newcommand*{\renamerobustsymbol}[2]{%
  \expandafter\let\expandafter\origrealcommand
    \csname #2\space\endcsname
    \expandafter\global\expandafter\let\csname#1#2\endcsname=\origrealcommand
}
\def\ifnotsavedsym@helper#1#2!{\expandafter\ifx\csname orig#2\endcsname\relax}
\newcommand*{\ifnotsavedsym}[1]{%
  \expandafter\ifnotsavedsym@helper\string#1!%
}
\let\oldcontentsline\contentsline
\newif\ifcomplete
\ExplSyntaxOn
\global\let\origRequirePackage\RequirePackage
\DeclareDocumentCommand\RequirePackage {o m o}
  {
     \IfValueTF{#3}
       {\IfValueTF {#1}
           { \origRequirePackage [{#1}] {#2} [{#3}] }
           { \origRequirePackage {#2} [{#3}]      }
       }
       {
        \IfValueTF{#1}
           {
             \origRequirePackage  [{#1}] {#2}
           }
           {
             \origRequirePackage {#2}
           }
      }
  }

\ExplSyntaxOff
\newif\ifloadpackages
\loadpackagestrue
\newcommand{\missingpkgs}{}
\newcommand{\foundpkgs}{}
\newcommand{\if@sty@file@exists@star}[3]{%
  \ifloadpackages
    \IfFileExists{#1.sty}{#2}{#3}%
  \else
    #3%
  \fi
}
\newcommand{\if@sty@file@exists}[3]{%
  \ifloadpackages
    \IfFileExists{#1.sty}%
                 {#2\@cons\foundpkgs{{#1}}}%
                 {#3\completefalse\@cons\missingpkgs{{#1}}}%
  \else
    #3\completefalse\@cons\missingpkgs{{#1}}%
  \fi
}
\newcommand{\IfStyFileExists}{%
  \@ifstar{\if@sty@file@exists@star}{\if@sty@file@exists}%
}
\newcommand\symbols{\flushleft}
\def\endsymbols{\endflushleft}

\def\dosymbol#1{%
   \leavevmode\hbox to .33\textwidth{%
    \hbox to 1.2em%
    {\hss$#1$\hfil}%
   \footnotesize\texttt{\string#1}\hss}%
   \penalty10}
\ExplSyntaxOn
\IfStyFileExists{calligra}
  {\save_symbol:{filename}
   \RequirePackage{calligra}
   \restore_symbol:{CAL}{filename}
   \DeclareMathAlphabet{\mathcalligra}{T1}{calligra}{m}{n}
   \DeclareFontShape{T1}{calligra}{m}{n}{<->s*[2.2]callig15}{}
  }
  {}
\ExplSyntaxOff

\IfStyFileExists{chancery}
  {\newcommand{\mathpzc}[1]{\mbox{\usefont{OT1}{pzc}{m}{it}##1}}}
  {}
\ifxetex
   \else
     \ifluatex
        \RequirePackage{etex}
     \else
        \RequirePackage{etex}
  \fi
\fi

\cxset{nag keys/.store in =\nagkeys@cx,
       onlyamsmath keys/.store in=\onlyamsmathkeys@cx,
       xcolor keys/.store in=\xcolorkeys@cx}
\cxset{xcolor keys={fixpdftex,usenames,dvipsnames,
                  svgnames,x11names,table}}
%% |\PassOptionsToPackage{\onlyamsmathkeys@cx}{onlyamsmath}|
\ifMICROTYPE
\ifengine%
  {\RequirePackage[tracking=true]{microtype}}%
  {\RequirePackage[tracking=true]{microtype}}%
  {\RequirePackage[tracking=true]{microtype}}%
\fi
\newif\ifRAGGEDTWOE
\newif\ifEVERYSEL
\newif\ifFOOTMISC
\PassOptionsToPackage{ragged2e}{footnotes,raggedrightboxes}
\RequirePackage{ragged2e}
\newif\ifSOUL
\IfStyFileExists{soul}
{\SOULtrue\RequirePackage{soul}
    \sethlcolor{thehighlight}}
{}

\RequirePackage{lettrine}
\ifx\dropcap\undefined
  \def\dropcap#1#2{%
    \lettrine[lines=3, lraise=0.1, nindent=0em, slope=.1em]{#1}{#2}
  }%
\fi
\IfStyFileExists{siunitx}{
   \RequirePackage{siunitx}
   \sisetup{fixed-exponent =0,
            scientific-notation = false}}
{}
\cxset{acronym keys/.store in = \acronymkeys@cx}
\cxset{acronym keys={smaller,printonlyused,withpage}}
\PassOptionsToPackage{\acronymkeys@cx}{acronym}
\RequirePackage{acronym}
\RequirePackage{mdframed}
\RequirePackage{adjustbox}
\RequirePackage{fancybox}
\RequirePackage{graphicx}[1999/02/16]
\DeclareGraphicsExtensions{.jpg, .JPG, .jpeg, .png, .eps}
\graphicspath{{graphics/}{graphics//}{../images/}{images//}{./images/}{./graphics/}%
   {../graphics/}{./pic/}{../pic}}
\RequirePackage{wrapfig}
\RequirePackage[quiet]{rotating}

\DeclareRobustCommand\thickrule{%
    \leavevmode \leaders \hrule height 2pt \hfill \kern \z@}
\DeclareRobustCommand\thinrule{\vrule width\textwidth height0.4pt depth0pt\relax}%
\DeclareRobustCommand\mediumrule{\rule{\textwidth}{0.8pt}}
\DeclareRobustCommand\Rule{{\color{\tocchapternumberfill@cx}\rule[-4.1pt]{13cm}{0.4pt}}}
\DeclareRobustCommand\bottomline{\medskip
   \noindent\rule{\linewidth}{0.4pt}\medskip}
\DeclareRobustCommand\topline{\par\medskip
   \noindent\rule{\linewidth}{0.4pt}\medskip}
\cxset{chapter rule color/.store in={\chapter@rule@color}}%
\cxset{chapter rule color=spot!30}
\DeclareRobustCommand\tikzrule{%
  \tikz [color=\chapter@rule@color, very thick, inner sep=0pt, outer sep=0pt]%
        \draw(0,0)--(\the\linewidth,0);
}%
\newcommand\drawrule[3][]{%
    \offinterlineskip
          \tikz [ name=s,trim left,
                   anchor=base,
                   draw=black,
                 % double distance=.2pt,
                  line width=#3,
                  %very thick,
                  inner sep=0pt,
                  outer sep=0pt,#1]   \draw(0,0)--(#2,0);
}
 \def\drawdoublerule#1#2{%
    \drawrule{#1}{#2}%
    \vskip2.5pt
    \drawrule{#1}{#2}%
 }
\newif\ifLIPSUM
\RequirePackage{lipsum}
\RequirePackage{kantlipsum}
\RequirePackage{blindtext}
\DeclareDocumentCommand\lorem{ s }{Fusce adipiscing justo nec ante. Nullam in enim.
 Pellentesque felis orci, sagittis ac, malesuada et, facilisis in,
 ligula. Nunc non magna sit amet mi aliquam dictum. In mi. Curabitur
 sollicitudin justo sed quam et quadd.
 \IfBooleanTF{#1}%
 {}%
 {\par}}
\newcommand{\fox}{``The quick brown fox jumps over the lazy dog''}
\newcommand\frogking{%
\leavevmode
\hskip1em In olden times when wishing
still helped one, there lived a
king whose daughters were all
beautiful, but the youngest was so
beautiful that the sun itself,
which has seen so much, was
astonished whenever it shone in
her face. Close by the king's
castle lay a great dark forest,
and under an old lime-tree in the
forest was a well, and when
the day was very warm, the
king's child went out into the
forest and sat down by the side
of the cool fountain, and when she was bored she
took a golden ball, and threw it up on a high and caught it, and this
ball was her favorite plaything. \par}%
\newcommand\onepar{In olden times when wishing
still helped one, there lived a
king whose daughters were all
beautiful, but the youngest was so
beautiful that the sun itself,
which has seen so much, was
astonished whenever it shone in
her face. Close by the king's
castle lay a great dark forest,
and under an old lime-tree in the
forest was a well, and when
the day was very warm, the
king's child went out into the
forest and sat down by the side
of the cool fountain, and when she was bored she
took a golden ball, and threw it up on a high and caught it, and this
ball was her favorite plaything.}%

\newcommand\alicei{%
  The King and Queen of Hearts were seated on their throne
  when they arrived, with a great crowd assembled about them
  ---all sorts of little birds and beasts, as well as the
  whole pack of cards: the Knave was standing before them,
  in chains, with a soldier on each side to guard him; and
  near the King was the White Rabbit, with a trumpet in one
  hand, and a scroll of parchment in the other.  In the very
  middle of the court was a table, with a large dish of
  tarts upon it: they looked so good, that it made Alice
  quite hungry to look at them---``I wish they'd get the
  trial done,'' she thought, ``and hand round the
  refreshments!''.  But there seemed to be no chance of this,
  so she began looking at everything about her to pass away
  the time.}%

\newcommand\aliceii{%
  Alice had never been in a court of justice before, but she
  had read about them in books, and she was quite pleased to
  find that she knew the name of nearly everything there.
  ``That's the judge,'' she said to herself, ``because of his
  great wig.''.

  The judge, by the way, was the King, and as he wore his
  crown over the wig, (look at the frontispiece if you want
  to see how he did it,) he did not look at all comfortable,
  and it was certainly not becoming.
}

 \newcommand\aliceiii{``And that's the jury-box,'' thought Alice, ``and those
  twelve creatures,'' (she was obliged to say ``creatures,''
  you see, because some of them were animals, and some were
  birds) ``I suppose they are the jurors.''.  She said this
  last word two or three times over to herself being rather
  proud of it: for she thought, and rightly too, that very
  few little girls of her age knew the meaning of it at all.
  However, ``jurymen'' would have done just as well.}

 \newcommand\aliceiv{The twelve jurors were all writing very busily on slates.
  ``What are they doing?'' Alice whispered to the Gryphon.
  ``They can't have anything to put down yet, before the
  trial's begun.''.}

\newcommand\alicev{``They're putting down their names,'' the Gryphon
  whispered in reply, ``for fear they should forget them
  before the end of the trial.''.}

\newcommand\alicevi{``Stupid things!'' Alice began in a loud indignant voice,
  but she stopped herself hastily, for the White Rabbit
  cried out, ``Silence in the court!''; and the King put on
  his spectacles and looked anxiously round, to make out who
  was talking.\par}
\def\ALPHABET {A B C D E F G H I J K L M N O P Q R S T U V W X Y Z}
\def\alphabet {a b c d e f g h i j k l m n o p q r s t u v w x y z}
\newcommand{\punctuation}{! ? . / , : }
\RequirePackage{fonttable}

\RequirePackage{booktabs}
\newcounter{step}
\newcommand\resetinc{\setcounter{step}{0}}
\newcommand\inc{\stepcounter{step}\thestep}
\RequirePackage{tabularx}
\RequirePackage{array}
\RequirePackage{dcolumn}
\RequirePackage{rccol}
\RequirePackage{longtable}
\let\origLT@array=\LT@array
\let\origLT@start=\LT@start
\newenvironment{longsymtable}[2][true]{%
  \expandafter\global\expandafter\let
  \expandafter\ifshowsymtable\csname if#1\endcsname
  \ifshowsymtable
    \mbox{}%
    \Needspace*{1\baselineskip}%
    \mbox{}%
    \begin{center}%
    \phantomsection
    \refstepcounter{table}%
    \let\refstepcounter=\@gobble
    \let\LT@array=\origLT@array
    \let\LT@start=\origLT@start

    \addcontentsline{toc}{subsection}{%
     \protect\numberline{\tablename~\thetable:}{#2}}%
    \@makecaption{\fnum@table}{#2}%
    \gdef\lt@indexed{}%
    \let\next=\relax
  \else
    % The following was taken verbatim from verbatim.sty.
    \let\do\@makeother\dospecials\catcode`\^^M\active
    \let\verbatim@startline\relax
    \let\verbatim@addtoline\@gobble
    \let\verbatim@processline\relax
    \let\verbatim@finish\relax
    \let\next=\verbatim@
  \fi
  \next
}{%
  \ifshowsymtable
    \end{center}
    \let\@elt=\index\lt@indexed  % Close our index ranges.
    \gdef\lt@indexed{}%
    \vskip 8ex minus 2ex
  \fi
}


\newcommand{\ltindex}[1]{%
  \index{#1|(}%
  \@cons{\lt@indexed}{{#1|)}}%
}
\newcommand{\ltidxboth}[2]{\mbox{}\ltindex{#1 #2}\ltindex{#2>#1}}

\let\LT@array=\origLT@array
\let\LT@start=\origLT@start

\IfStyFileExists*{type1cm}
  {\usepackage{type1cm}}
  {}
\newif\ifhavemultirow
\IfStyFileExists{multirow}
  {\havemultirowtrue
  \RequirePackage{multirow}
  }
  {}
\newif\ifhavecolortbl
\IfStyFileExists{colortbl}
  {\havecolortbltrue\RequirePackage{colortbl}}
  {}
\RequirePackage{threeparttable}
\RequirePackage{threeparttablex}

\IfStyFileExists {pdflscape}
  {\RequirePackage{pdflscape}}
  {}

\ifUNICODE
\else
\ifxetex
  \def\new@mathgroup{\alloc@8\mathgroup\mathchardef\@cclvi}
  \patchcmd{\document@select@group}{\sixt@@n}{\@cclvi}{}{}
  \patchcmd{\select@group}{\sixt@@n}{\@cclvi}{}{}
\fi
\ifluatex
  \def\new@mathgroup{\alloc@8\mathgroup\mathchardef\@cclvi}
  \patchcmd{\document@select@group}{\sixt@@n}{\@cclvi}{}{}
  \patchcmd{\select@group}{\sixt@@n}{\@cclvi}{}{}
\fi
\fi

\newif\ifAMS
\AMStrue
\ExplSyntaxOn
\newif\ifMTOOLS
\newcommand\MTOOLS{\pkgname{mathtools}}
 \RequirePackage{mathtools}
 \RequirePackage{suffix}
\IfStyFileExists{mathtools}
  {\MTOOLStrue
   \save_symbol:{xleftrightarrow} \save_symbol:{xLeftarrow}
   \save_symbol:{xRightarrow} \save_symbol:{xLeftrightarrow}
   \save_symbol:{xrightharpoondown} \save_symbol:{xrightharpoonup}
   \save_symbol:{xleftharpoondown} \save_symbol:{xleftharpoonup}
   \save_symbol:{xleftrightharpoons} \save_symbol:{xrightleftharpoons}
   \save_symbol:{xhookleftarrow} \save_symbol:{xhookrightarrow}
   \save_symbol:{xmapsto} \save_symbol:{underbracket}
   \save_symbol:{overbracket} \save_symbol:{lparen} \save_symbol:{rparen}
   \save_symbol:{dblcolon} \save_symbol:{coloneqq} \save_symbol:{Coloneqq}
   \save_symbol:{coloneq} \save_symbol:{Coloneq} \save_symbol:{eqqcolon}
   \save_symbol:{Eqqcolon} \save_symbol:{eqcolon} \save_symbol:{Eqcolon}
   \save_symbol:{colonapprox} \save_symbol:{Colonapprox}
   \save_symbol:{colonsim} \save_symbol:{Colonsim} \save_symbol:{overbrace}
   \save_symbol:{underbrace}%NEW
   \save_symbol:{underbrace}
   \save_symbol:{overleftrightarrow}%NEW
   \save_symbol:{mathscr}
   \save_symbol:{ulcorner}
   \save_symbol:{urcorner}
   \save_symbol:{llcorner}
   \save_symbol:{lrcorner}
   \save_symbol:{backepsilon}
   \save_symbol:{digamma}
   \save_symbol:{underrightarrow}
   \save_symbol:{underleftarrow}
   \save_symbol:{underleftrightarrow}
   \save_symbol:{eth}
   \save_symbol:{underbracket}
   % The mathtools package delays the definitions of some of its symbols
   % to the \begin{document}.  We redefine \AtBeginDocument to force
   % mathtools to define everything immediately.
   \let\origAtBeginDocument=\AtBeginDocument
   \def\AtBeginDocument##1{##1}
  % \let\RequirePackage\origRequirePackage
  \PassOptionsToPackage{donotfixmathbugs}{mathtools}
   \RequirePackage{mathtools}

   \let\AtBeginDocument=\origAtBeginDocument

   \restore_symbol:{MTOOLS}{xleftrightarrow}
   \restore_symbol:{MTOOLS}{xLeftarrow}
   \restore_symbol:{MTOOLS}{xRightarrow}
   \restore_symbol:{MTOOLS}{xLeftrightarrow}
   \restore_symbol:{MTOOLS}{xrightharpoondown}
   \restore_symbol:{MTOOLS}{xrightharpoonup}
   \restore_symbol:{MTOOLS}{xleftharpoondown}
   \restore_symbol:{MTOOLS}{xleftharpoonup}
   \restore_symbol:{MTOOLS}{xleftrightharpoons}
   \restore_symbol:{MTOOLS}{xrightleftharpoons}
   \restore_symbol:{MTOOLS}{xhookleftarrow}
   \restore_symbol:{MTOOLS}{xhookrightarrow}
   \restore_symbol:{MTOOLS}{xmapsto}
   \restore_symbol:{MTOOLS}{underbracket}
   \restore_symbol:{MTOOLS}{overbracket} \restore_symbol:{MTOOLS}{lparen}
   \restore_symbol:{MTOOLS}{rparen} \restore_symbol:{MTOOLS}{dblcolon}
   \restore_symbol:{MTOOLS}{coloneqq} \restore_symbol:{MTOOLS}{Coloneqq}
   \restore_symbol:{MTOOLS}{coloneq} \restore_symbol:{MTOOLS}{Coloneq}
   \restore_symbol:{MTOOLS}{eqqcolon} \restore_symbol:{MTOOLS}{Eqqcolon}
   \restore_symbol:{MTOOLS}{eqcolon} \restore_symbol:{MTOOLS}{Eqcolon}
   \restore_symbol:{MTOOLS}{colonapprox}
   \restore_symbol:{MTOOLS}{Colonapprox}
   \restore_symbol:{MTOOLS}{colonsim} \restore_symbol:{MTOOLS}{Colonsim}
   \restore_symbol:{MTOOLS}{overbrace} \restore_symbol:{MTOOLS}{underbrace}
   \restore_symbol:{MTOOLS}{underbracket}

   % Some of the above are defined in terms of \dblcolon.  At the time
   % of this writing it doesn't seem like any other package uses the
   % name \dblcolon so it should be safe to retain its mathtools
   % definition.
   \let\dblcolon=\MTOOLSdblcolon
  }
  {}
\ExplSyntaxOff
%%
\newif\ifFontIsAvailable
\def\testFontAvailability#1{%
  \count255=\interactionmode
  \batchmode
  \let\preload=\nullfont
  \font\preload="#1" at 10pt
  \ifx\preload\nullfont \FontIsAvailablefalse
  \else \FontIsAvailabletrue \fi
  \interactionmode=\count255
}

\def\FindAnInstalledFont#1#2{
  \expandafter\getFirstFontName#1/\end
  \let\next\gobbleTwo
  \ifx\trialFontName\empty
    \def#2{<No suitable font found>}%
  \else
    \testFontAvailability{\trialFontName}
    \ifFontIsAvailable
      \edef#2{\trialFontName}%
    \else
      \let\next\FindAnInstalledFont
    \fi
  \fi
  \expandafter\next\expandafter{\remainingNames}{#2}
}
\def\getFirstFontName#1/#2\end{\def\trialFontName{#1}\def\remainingNames{#2}}
\def\gobbleTwo#1#2{}
\newcommand\ligatures[2][Old Standard-Regular]{%
  \bgroup
  \fontspec[Ligatures = Common]{#1}%
  \textit{#2}%
  \egroup
}
\renewcommand\U[1]{{\texttt{U+#1}}(\char"#1)\xspace}

\newif\ifYH
\newcommand\YH{yhmaths}
\IfStyFileExists{yhmaths}
  {\YHtrue
   \let\origRequirePackage=\RequirePackage    % We don't want amsmath loaded.
   \def\RequirePackage##1{}
   \RequirePackage{yhmath}
   \let\RequirePackage=\origRequirePackage
  }
  { \RequirePackage{yhmath}}

\RequirePackage{multienum}
\ExplSyntaxOn
\newif\ifACCENTS
\IfStyFileExists{accents}
  {\ACCENTStrue
   \save_symbol:{undertilde}
   \save_symbol:{dddot}
   \save_symbol:{ddddot}
   \RequirePackage{accents}
   \restore_symbol:{ACCENTS}{undertilde}
   \restore_symbol:{ACCENTS}{dddot}
   \restore_symbol:{ACCENTS}{ddddot}
  }
  {}
\ExplSyntaxOff
 %\RequirePackage {nath} DANGEROUS
\IfStyFileExists{mathrsfs}
  {\newcommand{\mathscr}[1]{\mbox{\usefont{U}{rsfs}{m}{n}##1}}}
  {}
\def\TX{txfonts}

  % Redefine \DeclareMathSymbol to stick "ABX" in front of each symbol name.
  % Do the same for \DeclareMathDelimiter.
  % Define only those accents that are not defined elsewhere.
%%
\RequirePackage{nicefrac}
\RequirePackage{xfrac}
\RequirePackage{amssymb}[2002/01/22]
\RequirePackage{amsthm}[2002/01/22]
\RequirePackage{amsopn}
\RequirePackage{amscd}
\setcounter{MaxMatrixCols}{20}
\ifengine{}{}{%
  \IfStyFileExists{dsfont}%
    { \newcommand{\mathds}[1]{\mbox{\usefont{U}{dsrom}{m}{n}##1}}
      \newcommand{\mathdsss}[1]{\mbox{\usefont{U}{dsss}{m}{n}##1}}}
    {}
}

\ExplSyntaxOn
\newif\ifFEYN
\newcommand\FEYN{\pkgname{feyn}}
\IfStyFileExists{feyn}
  {\FEYNtrue
   \let\origProvidesPackage=\ProvidesPackage
   \def\ProvidesPackage##1[##2]{\origProvidesPackage{##1}[##2]\endinput}
   \save_symbol:{filename}
   \usepackage{feyn}
   \restore_symbol:{FEYN}{filename}
   \let\ProvidesPackage=\origProvidesPackage
   \DeclareFontFamily{OMS}{textfeyn}{\skewchar\font'000}
   \DeclareFontShape{OMS}{textfeyn}{m}{n}{%
     <-10.5>feyntext10%
     <10.5-11.5>feyntext11%
     <11.5->feyntext12%
   }{}
   \DeclareRobustCommand{\feyn}[1]{{\usefont{OMS}{textfeyn}{m}{n}##1}}
   \DeclareRobustCommand{\wfermion}{\feyn{\char"64}}
   \DeclareRobustCommand{\hfermion}{\feyn{\char"6B}}
   \DeclareRobustCommand{\shfermion}{\feyn{\char"6C}}
   \DeclareRobustCommand{\whfermion}{\feyn{\char"6D}}
   \DeclareRobustCommand{\gvcropped}{\feyn{\char"07}}
   \DeclareRobustCommand{\bigbosonloop}{\feyn{\char"7B}}
   \DeclareRobustCommand{\smallbosonloop}{\feyn{\char"7C}}
   \DeclareRobustCommand{\bigbosonloopA}{\feyn{\char"5B}}
   \DeclareRobustCommand{\smallbosonloopA}{\feyn{\char"5C}}
   \DeclareRobustCommand{\bigbosonloopV}{\feyn{\char"1B}}
   \DeclareRobustCommand{\smallbosonloopV}{\feyn{\char"1C}}
  }
  {}
 \DeclareRobustCommand\FIRE{{\large\color{red}\Fire}}
\ExplSyntaxOff
%%
%%   \save_symbol:{white}
%%   \save_symbol:{repeat}
%%   % Don't let igo redefine all of the font-size commands.
%%   \save_symbol:{scriptsize}\newcommand{\scriptsize}{}
%%   \save_symbol:{tiny}\newcommand{\tiny}{}
%%   \save_symbol:{large}\newcommand{\large}{}
%%   \save_symbol:{Large}\newcommand{\Large}{}
%%   \save_symbol:{LARGE}\newcommand{\LARGE}{}
%%   \save_symbol:{huge}\newcommand{\huge}{}
%%   \save_symbol:{Huge}\newcommand{\Huge}{}
%%   \restore_symbol:{IGO}{black}
%%   \restore_symbol:{IGO}{white}
%%   %\restore_symbol:{IGO}{repeat}
%%   \restore_symbol:{IGO}{tiny}
%%   \restore_symbol:{IGO}{large}
%%   \restore_symbol:{IGO}{Large}
%%   \restore_symbol:{IGO}{LARGE}
%%   \restore_symbol:{IGO}{huge}
%%   \restore_symbol:{IGO}{Huge}
\newif\ifULSY
\newcommand\ULSY{\pkgname{ulsy}}
\IfStyFileExists{ulsy}
  {\ULSYtrue\usepackage{ulsy}}
  {}
\ExplSyntaxOn
\newif\ifCEQ
\newcommand\CEQ{\pkgname{colonequals}}
\IfStyFileExists{colonequals}
  {\save_symbol:{colonapprox}
   \save_symbol:{colonsim}
   \CEQtrue
   \usepackage{colonequals}
   \restore_symbol:{CEQ}{colonapprox}
   \restore_symbol:{CEQ}{colonsim}
  }
  {}
\ExplSyntaxOff
\newif\ifCMLL
\newcommand\CMLL{\pkgname{cmll}}
\IfStyFileExists{cmll}
  {\CMLLtrue
   \newcommand*{\textCMLL}[1]{{\usefont{U}{cmllr}{m}{n}##1}}
   \DeclareRobustCommand{\CMLLparr}{\textCMLL{\char0}}
   \DeclareRobustCommand{\CMLLshpos}{\textCMLL{\char1}}
   \DeclareRobustCommand{\CMLLshneg}{\textCMLL{\char2}}
   \DeclareRobustCommand{\CMLLshift}{\textCMLL{\char3}}
   \DeclareRobustCommand{\CMLLcoh}{\textCMLL{\char4}}
   \DeclareRobustCommand{\CMLLscoh}{\textCMLL{\char5}}
   \DeclareRobustCommand{\CMLLincoh}{\textCMLL{\char6}}
   \DeclareRobustCommand{\CMLLsincoh}{\textCMLL{\char7}}
   \DeclareRobustCommand{\CMLLbigwith}{\raisebox{2ex}{\textCMLL{\char8}}}
   \DeclareRobustCommand{\CMLLbigparr}{\raisebox{2ex}{\textCMLL{\char10}}}
  }
  {}
\ExplSyntaxOn
 \newif\ifST
 \newcommand\ST{\pkgname{stmaryrd}}
 \IfStyFileExists{stmaryrd}
  {\STtrue
   \save_symbol:{lightning}
   \save_symbol:{bigtriangleup} \save_symbol:{bigtriangledown}
   \RequirePackage{stmaryrd}
   \restore_symbol:{ST}{lightning}
   \restore_symbol:{ST}{bigtriangleup} \restore_symbol:{ST}{bigtriangledown}
  }
  {}
\ExplSyntaxOff
\ExplSyntaxOn
\newif\ifXPFEIL
\newcommand\XPFEIL{\pkgname{extpfeil}}
\IfStyFileExists{extpfeils}
  {\XPFEILtrue
   % extpfeil tries to do a \RequirePackage of stmaryrd with
   % conflicting options from what we used to load stmaryd.  We
   % therefore temporarily make \RequirePackage a no-op to prevent LaTeX
   % from complaining.
   \let\origRequirePackage=\RequirePackage
   \renewcommand*{\RequirePackage}[2][]{}
   \save_symbol:{xlongequal}
   \save_symbol:{xmapsto}
   \RequirePackage{extpfeil}
   \restore_symbol:{XPFEIL}{xlongequal}
   \restore_symbol:{XPFEIL}{xmapsto}
   \let\RequirePackage=\origRequirePackage
  }
  {}
\ExplSyntaxOff
\iffalse
\newif\ifEU
\IfStyFileExists{euscript}
  {\EUtrue\RequirePackage[mathcal]{euscript}
   \renewcommand{\mathcal}[1]{\mbox{\usefont{U}{eus}{m}{n}##1}}
  }
  {\let\CMcal\mathcal}
\fi
  \newif\ifBM
  \IfStyFileExists{bm}
    {\BMtrue
      \RequirePackage{bm}
    }
   {}
\ifUNICODE
 \else
\IfStyFileExists{bbm}
  {\newcommand{\mathbbm}[1]{\mbox{\usefont{U}{bbm}{m}{n}##1}}
   \newcommand{\mathbbmss}[1]{\mbox{\usefont{U}{bbmss}{m}{n}##1}}
   \newcommand{\mathbbmtt}[1]{\mbox{\usefont{U}{bbmtt}{m}{n}##1}}}
  {}
\fi
\ifUNICODE
\else
\IfStyFileExists{bbold}
  {
  %</fontdef
  \newcommand{\BBmathbb}[1]{\mbox{\usefont{U}{bbold}{m}{n}##1}}
   % We have to manually define all of the symbols we care about.
   \newcommand{\BBsym}[1]{\ensuremath{\BBmathbb{\char##1}}}
   \newcommand{\Langle}{\BBsym{`<}}
   \newcommand{\Lbrack}{\BBsym{`[}}
   \newcommand{\Lparen}{\BBsym{`(}}
   \newcommand{\bbalpha}{\BBsym{"0B}}
   \newcommand{\bbbeta}{\BBsym{"0C}}
   \newcommand{\bbgamma}{\BBsym{"0D}}
   \newcommand{\Rparen}{\BBsym{`)}}
   \newcommand{\Rbrack}{\BBsym{`]}}
   \newcommand{\Rangle}{\BBsym{"3E}}
  }
  {}
\fi
\IfStyFileExists{mbboard}
  {\newcommand{\MBBmathbb}[1]{\mbox{\usefont{OT1}{mbb}{m}{n}##1}}}
  {}
\ifx\MBBmathbb\undefined
\else
  % Define only the symbols we actually use.
  \newcommand{\bbnabla}{\MBBmathbb{\char"9A}}
  \newcommand{\bbdollar}{\MBBmathbb{\char"24}}
  \newcommand{\bbeuro}{\MBBmathbb{\char"FB}}
  \newcommand{\bbpe}{\MBBmathbb{\char"D4}}
  \newcommand{\bbqof}{\MBBmathbb{\char"D7}}
  \newcommand{\bbyod}{\MBBmathbb{\char"C9}}
  \newcommand{\bbfinalnun}{\MBBmathbb{\char"CF}}

  % The following was copied from mbboard.sty.
  \DeclareFontFamily{OT1}{mbb}{\hyphenchar\font45}
  \DeclareFontShape{OT1}{mbb}{m}{n}{
        <5> <6> <7> <8> <9> <10> gen * mbb
        <10.95> mbb10 <12> <14.4> mbb12 <17.28> <20.74> <24.88> mbb17
        }{}
\fi

\ifx\mathfrak\undefined
\else
  \renewcommand{\mathfrak}[1]{\mbox{\fontencoding{U}\fontfamily{euf}\selectfont#1}}
\fi
\newif\ifUPGR
    \RequirePackage[Symbol]{upgreek}
\ExplSyntaxOn
\newif\ifMDOTS
\newcommand\MDOTS{\pkgname{mathdots}}
\ifUNICODE
\else
\IfStyFileExists{mathdots}
  {\MDOTStrue
   \save_symbol:{ddots}
   \save_symbol:{vdots}
   \save_symbol:{iddots}
   \save_symbol:{dddot}
   \save_symbol:{ddddot}
   \RequirePackage{mathdots}
   \restore_symbol:{MDOTS}{ddots}
   \restore_symbol:{MDOTS}{vdots}
   \restore_symbol:{MDOTS}{iddots}
   \restore_symbol:{MDOTS}{dddot}
   \restore_symbol:{MDOTS}{ddddot}
  }
  {}
\fi
\ExplSyntaxOff
\ExplSyntaxOn
\let\oldSI\SI
\let\SI\undefined
\newif\ifASCII
\newcommand\ASCII{\pkgname{ascii}}
\IfStyFileExists{ascii}
  {\ASCIItrue
   \save_symbol:{HT}
   \RequirePackage{ascii}
   \restore_symbol:{ascii}{HT}
   \let\SI\undefined
  }
  {}
\let\SI\oldSI
\ExplSyntaxOff
\ExplSyntaxOn
\newif\ifCHINA
\newcommand\CHINA{%
  \Chinasym
  \index{china2e=\textsf{china2e} (package)}%
  \index{packages>china2e=\textsf{china2e}}}
\IfStyFileExists{china2e}
  {\CHINAtrue
   \save_symbol:{Info}
   \save_symbol:{Earth}
   \save_symbol:{Telephone}
   \save_symbol:{Fire}
   \save_symbol:{vdots}
   \let\origDeclareSymbolFont=\DeclareSymbolFont
   \let\origDeclareMathSymbol=\DeclareMathSymbol
   \renewcommand{\DeclareSymbolFont}[5]{}
   \renewcommand{\DeclareMathSymbol}[4]{%
     \DeclareRobustCommand{##1}{{\uchr##4}}}
   \usepackage{china2e}
   \let\DeclareSymbolFont=\origDeclareSymbolFont
   \let\DeclareMathSymbol=\origDeclareMathSymbol
   \restore_symbol:{china}{Info}
   \restore_symbol:{china}{Earth}
   \restore_symbol:{china}{Telephone}
   \restore_symbol:{china}{Fire}
   \restore_symbol:{CHINA}{vdots}
  }
  {}
\ExplSyntaxOff
\newif\ifHARP
\newcommand\HARP{\pkgname{harpoon}}
\IfStyFileExists{harpoon}
  {\HARPtrue\usepackage{harpoon}}
  {}

\DeclareTextCommandDefault{\ltextcopyright}{\textcircled{c}}
\DeclareTextCommandDefault{\ltextregistered}{\textcircled{\scshape r}}
\DeclareTextCommandDefault{\ltexttrademark}{\textsuperscript{TM}}
\DeclareTextCommandDefault{\ltextordfeminine}{\textsuperscript{a}}
\DeclareTextCommandDefault{\ltextordmasculine}{\textsuperscript{o}}
\DeclareTextSymbol{\textcentoldstyle}{TS1}{'213}
\DeclareTextSymbolDefault{\textcentoldstyle}{TS1}
\DeclareTextSymbol{\textdollaroldstyle}{TS1}{'212}
\DeclareTextSymbolDefault{\textdollaroldstyle}{TS1}
\DeclareTextSymbol{\textguarani}{TS1}{'220}
\DeclareTextSymbolDefault{\textguarani}{TS1}

\def\UTFDeclarations{%
  \DeclareUTFcharacter[\UTFencname]{x3008}{\textlangle}
  \DeclareUTFcharacter[\UTFencname]{x3009}{\textrangle}
  \DeclareUTFcharacter[\UTFencname]{x301A}{\textlbrackdbl}
  \DeclareUTFcharacter[\UTFencname]{x301B}{\textrbrackdbl}


  \DeclareUTFcharacter[\UTFencname]{xFF10}{\textzerooldstyle}
  \DeclareUTFcharacter[\UTFencname]{xFF11}{\textoneoldstyle}
  \DeclareUTFcharacter[\UTFencname]{xFF12}{\texttwooldstyle}
  \DeclareUTFcharacter[\UTFencname]{xFF13}{\textthreeoldstyle}
  \DeclareUTFcharacter[\UTFencname]{xFF14}{\textfouroldstyle}
  \DeclareUTFcharacter[\UTFencname]{xFF15}{\textfiveoldstyle}
  \DeclareUTFcharacter[\UTFencname]{xFF16}{\textsixoldstyle}
  \DeclareUTFcharacter[\UTFencname]{xFF17}{\textsevenoldstyle}
  \DeclareUTFcharacter[\UTFencname]{xFF18}{\texteightoldstyle}
  \DeclareUTFcharacter[\UTFencname]{xFF19}{\textnineoldstyle}


  \DeclareEncodedCompositeCharacter{\UTFencname}{\textcircled}{20DD}{25EF}
  \DeclareUTFcomposite[\UTFencname]{x2460}{\textcircled}{1}
  \DeclareUTFcomposite[\UTFencname]{x2461}{\textcircled}{2}
  \DeclareUTFcomposite[\UTFencname]{x2462}{\textcircled}{3}
  \DeclareUTFcomposite[\UTFencname]{x2463}{\textcircled}{4}
  \DeclareUTFcomposite[\UTFencname]{x2464}{\textcircled}{5}
  \DeclareUTFcomposite[\UTFencname]{x2465}{\textcircled}{6}
  \DeclareUTFcomposite[\UTFencname]{x2466}{\textcircled}{7}
  \DeclareUTFcomposite[\UTFencname]{x2467}{\textcircled}{8}
  \DeclareUTFcomposite[\UTFencname]{x2468}{\textcircled}{9}
  \DeclareUTFcomposite[\UTFencname]{x2469}{\textcircled}{10}
  \DeclareUTFcomposite[\UTFencname]{x246A}{\textcircled}{11}
  \DeclareUTFcomposite[\UTFencname]{x246B}{\textcircled}{12}
  \DeclareUTFcomposite[\UTFencname]{x246C}{\textcircled}{13}
  \DeclareUTFcomposite[\UTFencname]{x246D}{\textcircled}{14}
  \DeclareUTFcomposite[\UTFencname]{x246E}{\textcircled}{15}
  \DeclareUTFcomposite[\UTFencname]{x246F}{\textcircled}{16}
  \DeclareUTFcomposite[\UTFencname]{x2470}{\textcircled}{17}
  \DeclareUTFcomposite[\UTFencname]{x2471}{\textcircled}{18}
  \DeclareUTFcomposite[\UTFencname]{x2472}{\textcircled}{19}
  \DeclareUTFcomposite[\UTFencname]{x2473}{\textcircled}{20}
  \DeclareUTFcomposite[\UTFencname]{x24B6}{\textcircled}{A}
  \DeclareUTFcomposite[\UTFencname]{x24B7}{\textcircled}{B}
  \DeclareUTFcomposite[\UTFencname]{x24B8}{\textcircled}{C}
  \DeclareUTFcomposite[\UTFencname]{x24B9}{\textcircled}{D}
  \DeclareUTFcomposite[\UTFencname]{x24BA}{\textcircled}{E}
  \DeclareUTFcomposite[\UTFencname]{x24BB}{\textcircled}{F}
  \DeclareUTFcomposite[\UTFencname]{x24BC}{\textcircled}{G}
  \DeclareUTFcomposite[\UTFencname]{x24BD}{\textcircled}{H}
  \DeclareUTFcomposite[\UTFencname]{x24BE}{\textcircled}{I}
  \DeclareUTFcomposite[\UTFencname]{x24BF}{\textcircled}{J}
  \DeclareUTFcomposite[\UTFencname]{x24C0}{\textcircled}{K}
  \DeclareUTFcomposite[\UTFencname]{x24C1}{\textcircled}{L}
  \DeclareUTFcomposite[\UTFencname]{x24C2}{\textcircled}{M}
  \DeclareUTFcomposite[\UTFencname]{x24C3}{\textcircled}{N}
  \DeclareUTFcomposite[\UTFencname]{x24C4}{\textcircled}{O}
  \DeclareUTFcomposite[\UTFencname]{x24C5}{\textcircled}{P}
  \DeclareUTFcomposite[\UTFencname]{x24C6}{\textcircled}{Q}
  \DeclareUTFcomposite[\UTFencname]{x24C7}{\textcircled}{R}
  \DeclareUTFcomposite[\UTFencname]{x24C8}{\textcircled}{S}
  \DeclareUTFcomposite[\UTFencname]{x24C9}{\textcircled}{T}
  \DeclareUTFcomposite[\UTFencname]{x24CA}{\textcircled}{U}
  \DeclareUTFcomposite[\UTFencname]{x24CB}{\textcircled}{V}
  \DeclareUTFcomposite[\UTFencname]{x24CC}{\textcircled}{W}
  \DeclareUTFcomposite[\UTFencname]{x24CD}{\textcircled}{X}
  \DeclareUTFcomposite[\UTFencname]{x24CE}{\textcircled}{Y}
  \DeclareUTFcomposite[\UTFencname]{x24CF}{\textcircled}{Z}
  \DeclareUTFcomposite[\UTFencname]{x24D0}{\textcircled}{a}
  \DeclareUTFcomposite[\UTFencname]{x24D1}{\textcircled}{b}
  \DeclareUTFcomposite[\UTFencname]{x24D2}{\textcircled}{c}
  \DeclareUTFcomposite[\UTFencname]{x24D3}{\textcircled}{d}
  \DeclareUTFcomposite[\UTFencname]{x24D4}{\textcircled}{e}
  \DeclareUTFcomposite[\UTFencname]{x24D5}{\textcircled}{f}
  \DeclareUTFcomposite[\UTFencname]{x24D6}{\textcircled}{g}
  \DeclareUTFcomposite[\UTFencname]{x24D7}{\textcircled}{h}
  \DeclareUTFcomposite[\UTFencname]{x24D8}{\textcircled}{i}
  \DeclareUTFcomposite[\UTFencname]{x24D9}{\textcircled}{j}
  \DeclareUTFcomposite[\UTFencname]{x24DA}{\textcircled}{k}
  \DeclareUTFcomposite[\UTFencname]{x24DB}{\textcircled}{l}
  \DeclareUTFcomposite[\UTFencname]{x24DC}{\textcircled}{m}
  \DeclareUTFcomposite[\UTFencname]{x24DD}{\textcircled}{n}
  \DeclareUTFcomposite[\UTFencname]{x24DE}{\textcircled}{o}
  \DeclareUTFcomposite[\UTFencname]{x24DF}{\textcircled}{p}
  \DeclareUTFcomposite[\UTFencname]{x24E0}{\textcircled}{q}
  \DeclareUTFcomposite[\UTFencname]{x24E1}{\textcircled}{r}
  \DeclareUTFcomposite[\UTFencname]{x24E2}{\textcircled}{s}
  \DeclareUTFcomposite[\UTFencname]{x24E3}{\textcircled}{t}
  \DeclareUTFcomposite[\UTFencname]{x24E4}{\textcircled}{u}
  \DeclareUTFcomposite[\UTFencname]{x24E5}{\textcircled}{v}
  \DeclareUTFcomposite[\UTFencname]{x24E6}{\textcircled}{w}
  \DeclareUTFcomposite[\UTFencname]{x24E7}{\textcircled}{x}
  \DeclareUTFcomposite[\UTFencname]{x24E8}{\textcircled}{y}
  \DeclareUTFcomposite[\UTFencname]{x24E9}{\textcircled}{z}
  \DeclareUTFcomposite[\UTFencname]{x24EA}{\textcircled}{0}
  \DeclareUTFcharacter[\UTFencname]{x25EF}{\textbigcircle}
  \DeclareUTFcomposite[\UTFencname]{x3251}{\textcircled}{21}
  \DeclareUTFcomposite[\UTFencname]{x3252}{\textcircled}{22}
  \DeclareUTFcomposite[\UTFencname]{x3253}{\textcircled}{23}
  \DeclareUTFcomposite[\UTFencname]{x3254}{\textcircled}{24}
  \DeclareUTFcomposite[\UTFencname]{x3255}{\textcircled}{25}
  \DeclareUTFcomposite[\UTFencname]{x3256}{\textcircled}{26}
  \DeclareUTFcomposite[\UTFencname]{x3257}{\textcircled}{27}
  \DeclareUTFcomposite[\UTFencname]{x3258}{\textcircled}{28}
  \DeclareUTFcomposite[\UTFencname]{x3259}{\textcircled}{29}
  \DeclareUTFcomposite[\UTFencname]{x325A}{\textcircled}{30}
  \DeclareUTFcomposite[\UTFencname]{x325B}{\textcircled}{31}
  \DeclareUTFcomposite[\UTFencname]{x325C}{\textcircled}{32}
  \DeclareUTFcomposite[\UTFencname]{x325D}{\textcircled}{33}
  \DeclareUTFcomposite[\UTFencname]{x325E}{\textcircled}{34}
  \DeclareUTFcomposite[\UTFencname]{x325F}{\textcircled}{35}
  \DeclareUTFcomposite[\UTFencname]{x32B1}{\textcircled}{36}
  \DeclareUTFcomposite[\UTFencname]{x32B2}{\textcircled}{37}
  \DeclareUTFcomposite[\UTFencname]{x32B3}{\textcircled}{38}
  \DeclareUTFcomposite[\UTFencname]{x32B4}{\textcircled}{39}
  \DeclareUTFcomposite[\UTFencname]{x32B5}{\textcircled}{40}
  \DeclareUTFcomposite[\UTFencname]{x32B6}{\textcircled}{41}
  \DeclareUTFcomposite[\UTFencname]{x32B7}{\textcircled}{42}
  \DeclareUTFcomposite[\UTFencname]{x32B8}{\textcircled}{43}
  \DeclareUTFcomposite[\UTFencname]{x32B9}{\textcircled}{44}
  \DeclareUTFcomposite[\UTFencname]{x32BA}{\textcircled}{45}
  \DeclareUTFcomposite[\UTFencname]{x32BB}{\textcircled}{46}
  \DeclareUTFcomposite[\UTFencname]{x32BC}{\textcircled}{47}
  \DeclareUTFcomposite[\UTFencname]{x32BD}{\textcircled}{48}
  \DeclareUTFcomposite[\UTFencname]{x32BE}{\textcircled}{49}
  \DeclareUTFcomposite[\UTFencname]{x32BF}{\textcircled}{50}
}
\ifengine{\UTFDeclarations}{\UTFDeclarations}{}
\ifxetex\else\ifluatex\else
  \RequirePackage{textcomp}
  \PassOptionsToPackage{mathcomp}{rmdefault}
  \RequirePackage{mathcomp}
  \fi
\fi
\ifxetex
    \else
     \ifluatex
     \else
       %\RequirePackage{exscale}
       %\RequirePackage{relsize}
     \fi
\fi
\newcommand{\tabitem}[2]{%
  \texttt{\symbol{`\\}#1} & \@nameuse{#1}
   & \bfseries\@nameuse{#1}& \itshape\@nameuse{#1}
   \ifthenelse{\equal{#2}{}}
    {}
    {& \texttt{\symbol{`\\}#2} & \@nameuse{#2}
     & \bfseries\@nameuse{#2}
     & \itshape\@nameuse{#2} \\}
}
%%  \tabitem{textcapitalcompwordmark}{textascendercompwordmark}
\ExplSyntaxOn
\newif\ifWASY
\newcommand\WASY{\pkgname{wasysym}}
\IfStyFileExists{wasysym}
  {\WASYtrue
   \save_symbol:{lightning}
   \save_symbol:{Box}
   \save_symbol:{Diamond}
   \save_symbol:{clock}
   \RequirePackage{wasysym}
   \restore_symbol:{WASY}{lightning}
   \restore_symbol:{WASY}{Box}
   \restore_symbol:{WASY}{Diamond}
   \restore_symbol:{WASY}{clock}
  }
  {}
\ExplSyntaxOff
\newif\ifPI
\newcommand\PI{\pkgname{pifont}}
\IfStyFileExists{pifont}
  {\PItrue\RequirePackage{pifont}}
  {}
\ExplSyntaxOn
\newif\ifMARV
\newcommand\MARV{\pkgname{marvosym}}
\IfStyFileExists*{marvosym}
  {\save_symbol:{CheckedBox}
   \RequirePackage{marvosym}[2000/05/01]% Major rewrite at this version.
   \global\MARVtrue
   \restore_symbol:{CheckedBox}{CheckedBox}
   \@ifundefined{Denarius} % \Denarius is a newer symbol.
     {\global\MARVfalse}
     {}
   \@ifundefined{MVRightarrow}% \Mvrightarrow is an even newer symbol.
     {\global\MARVfalse}
     {}
  }
  {}
\ExplSyntaxOff

\ExplSyntaxOn
\newif\ifDING
\newcommand\DING{\pkgname{bbding}}
\IfStyFileExists{bbding}
  {\DINGtrue
   \save_symbol:{Cross}
   \save_symbol:{Square}
   \RequirePackage{bbding}
   \restore_symbol:{ding}{Cross}
   \restore_symbol:{ding}{Square}
  }
  {}

\newcount\c@lumnsleft
\newcount\t@talcolumns
\newdimen\c@lumnwidth
\newenvironment{commandsInColumns}[1]{%
  \t@talcolumns=#1\advance\t@talcolumns-1\c@lumnsleft=\t@talcolumns%
  \c@lumnwidth=-2em\multiply\c@lumnwidth by \t@talcolumns%
  \advance\c@lumnwidth by\hsize \divide\c@lumnwidth by #1%
  \vskip\z@     % Ensures vertical mode
  \catcode`\^^M=12%
  \hbox\bgroup%
  \st@rtenv%
}
{\ifnum\c@lumnsleft=\t@talcolumns \egroup
 \else \egroup \fi}
{\catcode`\^^M=12%
 \gdef\st@rtenv{\@ifnextchar^^M{\dr@pnext\doNextComm@nd}{\doNextComm@nd}}%
 \gdef\setComm@nd#1#2^^M{%
   \hbox to \c@lumnwidth%
     {\hbox to .5cm{#1\hss}\hbox{\expandafter\setn@me\string#1.}\hss}%
   \advance\c@lumnsleft-1%
   \ifnum\c@lumnsleft>0%
     \hskip2em%
   \else%
     \egroup%
     \hbox\bgroup%
     \c@lumnsleft\t@talcolumns%
   \fi%
   \doNextComm@nd%
  }}
\def\dr@pnext#1#2{#1}
\def\doNextComm@nd{\@ifnextchar\end{}{\setComm@nd}}%
\def\setn@me#1#2.{\CSname{#2}}
\newcommand{\CSname}[1]{\texttt{\protect\bslash#1}}
\ExplSyntaxOff
\ExplSyntaxOn
\newif\ifEUSYM\EUSYMfalse
\newcommand\EUSYM{\pkgname{eurosym}}
\IfStyFileExists{eurosym}
  {\EUSYMtrue
   \save_symbol:{EUR}
   \usepackage{eurosym}
   \restore_symbol:{MARV}{EUR}
  }
  {}
\newif\ifESV\ESVfalse
\newcommand\ESV{\pkgname{esvect}}
\ExplSyntaxOff
\IfStyFileExists{esvect}
  {\ESVtrue
   \RequirePackage{esvect}
   \DeclareMathSymbol{\fldra}{\mathrel}{esvector}{'021}
   \DeclareMathSymbol{\fldrb}{\mathrel}{esvector}{'022}
   \DeclareMathSymbol{\fldrc}{\mathrel}{esvector}{'023}
   \DeclareMathSymbol{\fldrd}{\mathrel}{esvector}{'024}
   \DeclareMathSymbol{\fldre}{\mathrel}{esvector}{'025}
   \DeclareMathSymbol{\fldrf}{\mathrel}{esvector}{'026}
   \DeclareMathSymbol{\fldrg}{\mathrel}{esvector}{'027}
   \DeclareMathSymbol{\fldrh}{\mathrel}{esvector}{'030}
  }
  {}

  \RequirePackage{mhchem}
  \RequirePackage{chemfig}

\newif\ifMAN
\newcommand\MAN{\pkgname{manfnt}}
\IfStyFileExists{manfnt}
  {\MANtrue\RequirePackage{manfnt}}
  {}

\ExplSyntaxOn
  \newenvironment {ddanger}
 {
  \begin{trivlist}\item[]\noindent
  \begingroup\hangindent=3.5pc\hangafter=-2
  \cs_set_nopar:Npn \par{\endgraf\endgroup}
  \hbox to0pt{\hskip-\hangindent\dbend\kern2pt\dbend\hfill}\ignorespaces
 }{
  \par\end{trivlist}
 }
\ExplSyntaxOff

\ExplSyntaxOn
\newif\ifIFS
\newcommand\IFS{\pkgname{ifsym}}
\IfStyFileExists{ifsym}
  {\IFStrue
   \save_symbol:{Letter}
   \save_symbol:{Square}
   \save_symbol:{Cross}
   \save_symbol:{Sun}
   \save_symbol:{TriangleUp} \save_symbol:{TriangleDown} \save_symbol:{Circle}
   \save_symbol:{Lightning}
   \RequirePackage[alpine,clock,electronic,geometry,misc,weather]{ifsym}[2000/04/18]
   \restore_symbol:{ifs}{Letter} \restore_symbol:{ifs}{Square}
   \restore_symbol:{ifs}{Cross} \restore_symbol:{ifs}{Sun}
   \restore_symbol:{ifs}{TriangleUp} \restore_symbol:{ifs}{TriangleDown}
   \restore_symbol:{ifs}{Circle} \restore_symbol:{ifs}{Lightning}
   \DeclareRobustCommand{\allCubes}{%
     \Cube{1}~%
     \Cube{2}~%
     \Cube{3}~%
     \Cube{4}~%
     \Cube{5}~%
     \Cube{6}%
   }
  }
  {}
\ExplSyntaxOff
\newif\ifUTILD
\newcommand\UTILD{\pkgname{undertilde}}
\IfStyFileExists{undertilde}
  {\UTILDtrue\RequirePackage{undertilde}}
  {}
\RequirePackage{phdfilecontents}
\RequirePackage{changepage}

\RequirePackage{keyval}
\usepackage{xkvview}
\RequirePackage{ifmtarg}
\RequirePackage{fp}
\RequirePackage{ifthenx}
\RequirePackage{xspace}
\RequirePackage{xstring}
\RequirePackage{multido}
\RequirePackage{etoolbox}
\RequirePackage{parselines}
\def\TRUE{ \meta{true code} }
\def\FALSE{ \meta{false code} }
\def\PASS{\par{\bfseries\textcolor{green!50!blue}{PASS}}\ ~}
\def\FAIL{\par{\bfseries\textcolor{red!70!black}{FAIL}}\ ~}
  \RequirePackage{calc}
\RequirePackage{pict2e}
\RequirePackage{picture}
\RequirePackage{tikz}
\usetikzlibrary{%
  arrows,%
  calc,%
  fit,%
  patterns,%
  plotmarks,%
  shapes.geometric,%
  shapes.misc,%
  shapes.symbols,%
  shapes.arrows,%
  shapes.callouts,%
  shapes.multipart,%
  shapes.gates.logic.US,%
  shapes.gates.logic.IEC,%
  er,%
  automata,%
  backgrounds,%
  chains,%
  topaths,%
  trees,%
  petri,%
  mindmap,%
  matrix,%
  calendar,%
  folding,%
  fadings,%
  through,%
  positioning,%
  scopes,%
  decorations.fractals,%
  decorations.shapes,%
  decorations.text,%
  decorations.pathmorphing,%
  decorations.pathreplacing,%
  decorations.footprints,%
  decorations.markings,%
  shadows}
\usetikzlibrary{tikzmark}
\RequirePackage{pgfplots}
\pgfplotsset{compat=1.11}
\RequirePackage{pgfplotstable}
\IfStyFileExists{forest}
  {\RequirePackage {forest}}
  {}
\RequirePackage{drawstack}
\ExplSyntaxOn
\tl_new:N \beforehyperhook
\cs_gset:Npn \putbeforehyperhook #1
  {
    \tl_gput_left:Nn \beforehyperhook {#1}
  }
\ExplSyntaxOff

\newcommand*{\BeforeHyperrefHook}
  {%
 % \putbeforehyperhook
  \RequirePackage{float}%
  \RequirePackage{newfloat}
  }

   % \RequirePackage{verse}} TO FIX

\newcommand*{\AfterHyperrefHook}{%
  \RequirePackage{algorithm2e}%
  \RequirePackage{fancyhdr}

  \RequirePackage{datetime} %%scrtime
  \RequirePackage{scrtime}
  \RequirePackage{datenumber}
  \RequirePackage{natbib}
   \bibliographystyle{abbrvnat}
  \usepackage{bibentry} % needs checking
  %\bibpunct{(}{)}{;}{a}{,}{,}
  %%%%%%%%%%%%%%%%%%%%%%%%%%%%%%%%%%%%%%%%%%%%%%%%%%%%%%%%%%%%%%%
\@ifpackageloaded{natbib}{%
    \providecommand\refname{References} % internationalize?
    \providecommand\bibname{Bibliography}


\setlength\bibhang{1em}
\renewenvironment{thebibliography}[1]{%
  \bibsection\parindent \z@\bibpreamble\bibliosize\list
   {\@biblabel{\arabic{NAT@ctr}}}{\@bibsetup{##1}%changed
    \setcounter{NAT@ctr}{0}}%
    \ifNAT@openbib
      \renewcommand\newblock{\par}
    \else
      \renewcommand\newblock{\hskip .11em \@plus.33em \@minus.07em}%
    \fi
    \sloppy\clubpenalty4000\widowpenalty4000
    \sfcode`\.=1000\relax
    \let\citeN\cite \let\shortcite\cite
    \let\citeasnoun\cite
 }{\def\@noitemerr{%
  \PackageWarning{natbib}
     {Empty `thebibliography' environment}}%
  \endlist\vskip-\lastskip}

}{}
}
\def\sethyperref{%
  \BeforeHyperrefHook
  \RequirePackage{hyperref}
%% hyperdoc has a problem with tcolorboc documentation
%% macros.
%%\usepackage{hypdoc}
\hypersetup{
  bookmarks,
  raiselinks,
  pageanchor,
  hyperindex=true,
  colorlinks,
  allcolors=theblue,
  linktocpage,
  hyperfootnotes=true,
  breaklinks=true,
  anchorcolor= theblue,
  filecolor=blue,
  hypertexnames=true, %useguessable names for links
  urlcolor= theblue,
  linkcolor= theblue,
  pdftitle={My Title},
  pdfauthor={Yiannis Lazarides},
  pdfsubject={The phd LaTeX package},
  pdfkeywords={LaTeX package management, document design},
  plainpages=true%do page number anchors as plain Arabic
 }
\AfterHyperrefHook
}
\newif\ifALGORITHM
\@ifpackageloaded{hyperref}{%
    %%\RequirePackage{algorithms}
 }
 {\typeout{Algorithm loaded}}
  \RequirePackage{algorithm2e}
\RequirePackage{multicol}


\newif\ifULEM
\IfStyFileExists{ulem}
{\ULEMtrue\RequirePackage[normalem]{ulem}}
{}
\newif\ifhaveslashed
\IfStyFileExists*{slashed}
  {\haveslashedtrue\RequirePackage{slashed}}
  {}

\newif\ifhavecancel
\IfStyFileExists*{cancel}
  {\havecanceltrue\RequirePackage{cancel}}
  {}

\newcommand\hcancel[2][red]{\setbox0=\hbox{#2}%
\rlap{\raisebox{.45\ht0}{\textcolor{#1}{\rule{\wd0}{1pt}}}}#2}
\newif\ifSTAVE
\newcommand\STAVE{\pkgname{staves}}
\IfStyFileExists{staves}
  {\STAVEtrue\usepackage{staves}}
  {}

\newif\ifSHUF
\newcommand\SHUF{\pkgname{shuffle}}
\IfStyFileExists{shuffle}
  {\let\origDeclareSymbolFont=\DeclareSymbolFont
   \let\origDeclareMathSymbol=\DeclareMathSymbol
   \renewcommand{\DeclareSymbolFont}[5]{}
   \renewcommand{\DeclareMathSymbol}[4]{%
     \DeclareRobustCommand{##1}{{\usefont{U}{shuffle}{m}{n}\char##4\relax}}
   }
   \SHUFtrue
   \RequirePackage{shuffle}
   \let\DeclareSymbolFont=\origDeclareSymbolFont
   \let\DeclareMathSymbol=\origDeclareMathSymbol
  }
  {}

\RequirePackage{framed}
\RequirePackage{varioref}
\RequirePackage{setspace}

\providecommand*\switch[2]{{\fontfamily{#1}\selectfont #2}}

\newif\ifhavecenternot
\IfStyFileExists*{centernot}%
  {\havecenternottrue\RequirePackage{centernot}}
  {}
\RequirePackage{genealogytree}
\RequirePackage{chngcntr}
\RequirePackage{fourier-orns}
\RequirePackage{eso-pic}
\RequirePackage{alphalph}
\RequirePackage{fmtcount}
\RequirePackage{varwidth}
\RequirePackage{comment}
\RequirePackage{textcase}
\RequirePackage[autostyle=false]{csquotes}
\RequirePackage{alltt}[1997/06/16]

\RequirePackage{caption} % check
\RequirePackage{subcaption}

\RequirePackage[final]{pdfpages}
\newif\ifCCLIC
\newcommand\CCLIC{\pkgname{cclicenses}}
\IfStyFileExists{cclicenses}
  {\CCLICtrue
   \RequirePackage{cclicenses}
   \DeclareTextAccentDefault{\textcircled}{OMS}
  }
  {}
\ifxetex\else
\newif\ifFOUR
\newcommand\FOUR{\pkgname{fourier}}
\IfStyFileExists{fourier}
  {\FOURtrue
   \RequirePackage{fourier-orns}
   % Define single-glyph symbols.
   \DeclareFontEncoding{FMS}{}{}
   \DeclareFontSubstitution{FMS}{futm}{m}{n}
   \DeclareFontEncoding{FML}{}{}
   \DeclareFontSubstitution{FML}{futmi}{m}{it}
   \newcommand{\fourierdef}[6]{%
     \DeclareRobustCommand{##1}{{\usefont{##2}{##3}{##4}{##5}\char##6}}}
   \fourierdef{\parallelslant}{FMS}{futm}{m}{n}{134}
   \fourierdef{\nparallelslant}{FMS}{futm}{m}{n}{143}
   \fourierdef{\FOURrho}{FML}{futmi}{m}{it}{26}
   \fourierdef{\FOURvarrho}{FML}{futmi}{m}{it}{37}
   \fourierdef{\varvarrho}{FML}{futmi}{m}{it}{129}
   \fourierdef{\FOURpi}{FML}{futmi}{m}{it}{25}
   \fourierdef{\FOURvarpi}{FML}{futmi}{m}{it}{36}
   \fourierdef{\varvarpi}{FML}{futmi}{m}{it}{131}
   \fourierdef{\FOURpartial}{FML}{futmi}{m}{it}{64}
   \fourierdef{\varpartialdiff}{FML}{futmi}{m}{it}{130}
   \fourierdef{\FOURtexteuro}{TS1}{futx}{m}{n}{191}
   % Fake a math accent with text-mode commands.
   \DeclareRobustCommand{\FOURfakewidetopaccent}[5]{%
     \setbox0=\hbox{\ensuremath{##1}}%
     \setbox1=\hbox{\ensuremath{abc}}%
     \leavevmode
     \ifdim\wd0<\wd1
       \kern1pt
       \rlap{\raisebox{##2}{\makebox[\wd0]{\usefont{FMX}{futm}{m}{n}\char##3}}}%
       \kern-0.1em
       \box0
     \else
       \rlap{\raisebox{##4}{\makebox[\wd0]{\usefont{FMX}{futm}{m}{n}\char##5}}}%
       \box0
     \fi
   }

   % Manually define Fourier's extensible accents.  Note that we don't
   % bother trying to use Fourier's \mathring to construct the
   % \FOURwidering symbol.
   \DeclareFontEncoding{FMX}{}{}
   \DeclareFontSubstitution{FMX}{futm}{m}{n}
   \DeclareRobustCommand{\FOURwidearc}[1]{%
     \FOURfakewidetopaccent{##1}{0ex}{216}{0.5ex}{217}}
   \DeclareRobustCommand{\FOURwideOarc}[1]{%
     \FOURfakewidetopaccent{##1}{0ex}{228}{0.5ex}{229}}
   \DeclareRobustCommand{\FOURwideparen}[1]{%
     \FOURfakewidetopaccent{##1}{0ex}{148}{0.5ex}{150}}
   \DeclareRobustCommand{\FOURwidering}[1]{\overset{\smash{\vbox to .2ex{%
     \hbox{$\mathring{}$}}}}{\FOURwideparen{##1}}}

   % Manually define Fourier's variable-sized delimiters.
   \newcommand{\fouriercdef}[6]{%
     \DeclareRobustCommand{##1}{%
       \textvcenter{\usefont{##2}{##3}{##4}{##5}\char##6}}}
   \fouriercdef{\FOURtllbracket}{FMX}{futm}{m}{n}{133}
   \fouriercdef{\FOURdllbracket}{FMX}{futm}{m}{n}{139}
   \fouriercdef{\FOURtrrbracket}{FMX}{futm}{m}{n}{134}
   \fouriercdef{\FOURdrrbracket}{FMX}{futm}{m}{n}{140}
   \newcommand*{\FOURverticals}[1]{%
     \vbox{%
       \baselineskip=-\maxdimen
       \lineskiplimit=\maxdimen
       \lineskip=0pt%
       \usefont{FMX}{futm}{m}{n}%
       \ialign{####\cr##1}%
     }%
   }
   \DeclareRobustCommand{\FOURtVERT}{%
     \raisebox{0.5ex}{\textvcenter{\FOURverticals{\char147\cr\char147\cr}}}}
   \DeclareRobustCommand{\FOURdVERT}{%
     \raisebox{0.5ex}{\textvcenter{\FOURverticals{\char147\cr\char147\cr\char147\cr\char147\cr}}}}
  }
  {}
\fi
\IfStyFileExists{dirtree}
{
  \RequirePackage{dirtree}}
{}
\IfStyFileExists*{needspace}
  {\RequirePackage{needspace}}
  {\newcommand{\Needspace}[2]{\par \penalty-100\begingroup
     \setlength{\dimen@}{##2}%
     \dimen@ii\pagegoal \advance\dimen@ii-\pagetotal
     \ifdim \dimen@>\dimen@ii
       \break
     \fi\endgroup}
  }
\RequirePackage{uncial}
\newif\ifarchaic
  \archaictrue
\ifarchaic

\newif\ifLINA
\newcommand\LINA{\pkgname{lineara}}
\IfStyFileExists{lineara}
  {\LINAtrue\RequirePackage{lineara}}
  {}
\newif\ifLINB
\newcommand\LINB{\pkgname{linearb}}
\IfStyFileExists{linearb}
  {\LINBtrue\RequirePackage{linearb}}
  {}
\newif\ifCYPR
\newcommand\CYPR{\pkgname{cypriot}}
\IfStyFileExists{cypriot}
  {\CYPRtrue\RequirePackage{cypriot}}
  {}
\newif\ifSARAB
\newcommand\SARAB{\pkgname{sarabian}}
\IfStyFileExists{sarabian}
  {\SARABtrue\RequirePackage{sarabian}}
  {}
\newif\ifPRSN
\newcommand\PRSN{\pkgname{oldprsn}}
\IfStyFileExists{oldprsn}
  {\PRSNtrue\RequirePackage{oldprsn}}
  {}
\RequirePackage{hieroglf}
\newif\ifUGAR
\newcommand\UGAR{\pkgname{ugarite}}
\RequirePackage{ugarite}
\IfStyFileExists{ugarite}
  {\UGARtrue\RequirePackage{ugarite}}
  {}
\newif\ifOLMEC
\newif\ifscriptolmec \scriptolmectrue
\cxset{olmec/.is if=scriptolmec}
\cxset{olmec=true}
\ifscriptolmec
\RequirePackage{epiolmec}
\IfStyFileExists{epiolmec}
  {\OLMECtrue\RequirePackage{epiolmec}}
  {}
\fi

\newif\ifPHILOKALIA
\def\loadphilokalia{%
  \@namedef{ver@xltxtra.sty}{}% a fake for a "xlextra" package
  \RequirePackage{philokalia}
  \IfStyFileExists{philokalia}
    {\PHILOKALIAtrue\RequirePackage{philokalia}}
    {}
}%
\ifengine{\loadphilokalia}{\loadphilokalia}{}
\ifPHILOKALIA
  \newfontfamily\plk{Philokalia-Regular}
  \newfontfamily\PHtitl[Script=Greek,RawFeature=+titl;grek]{Philokalia-Regular}
\fi
\def\diacritic#1{{#1\LARGE ῾◌◌\char"0375}}
\newfontfamily\greek[Script=Greek,Scale=1.0]{Arial Unicode MS}
\def\greektext#1{\greek{#1}}
 \newsavebox{\philobox}
 \savebox{\philobox}{\PHtitl Π}
\endinput
%%
%% End of file `phd-pkgmanager.tex'.
