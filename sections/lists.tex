%FIX LIST DIAGRAM
\begingroup


%
%\cxset{steward,
%  chapter name=chapter,
%  numbering=arabic,
%  custom= stewart,
%   image={sweepers.jpg},
%  texti={Lists are essential elements of any document style and perhaps the most troublesome to get right.
%         In this chapter we discuss the construction of lists and offer a key value interface.},
%  textii={The Chapter discusses in detail the construction of lists. It reviews the mechanisms offered
%          by LaTeX and outlines a key value approach to building lists. We define a standard interface that does not
%          interfere with the original commands. The three standard list styles \textit{enumerate, itemize} and \textit{description} are redesigned to accept a key value interface. The photograph is Lewis Hine's which noted: ``Ivey Mill Company, Hickory, N.C. Some doffers and sweepers. Plenty of them.'' Location: Hickory, Catawba County Date: November 1908. Photographs like this were used by Hine to campaign against child labour.
%         }
%}

\chapter{Standard \LaTeX\ Lists}

\tcbset{width=\linewidth,arc=1mm,before=\bigskip,after=\medskip,left=8mm}

The general parameters affecting a general list is shown in the  diagram  below\footnote{Produced using the \texttt{layouts} package.}. LaTeX offers three general list structures, enumerate, itemize and description.
%\begin{figure}[hp]
%\listdiagram
%\caption{Layout of an \texttt{enumerate} list} \label{fig:lstenum}
%\end{figure}

\section{Package usage}

List are set using setenumerate, setitemize, setdescription. It is also possible to create new list structures, which will be explained a bit later on.

\section{The list geometry}

You can draw a list diagram as shown below, using the function \docAuxCmd{drawlistdiagram} from
the \pkgname{xlayouts} package, which is bundled with the \pkgname{phd} package.

\medskip
{\centering
\drawlistdiagram

\captionof{figure}{\latexe list diagram.}
}
\medskip
%
\section{The description list environment}

You can use the description list environment as you would normally use it with \latexe.

\begin{docEnv} {description} {}
\end{docEnv}

However, a number of settings are available to modify the styling of the environment. These 
include settings for fonts and color, as well as spacing and margins.

\begin{docKey} {description label font-face} { = \meta{font shape} } {initial, default=inherit}
\end{docKey}

\begin{docKey} {description label font-family} { = \meta{font shape} } {initial, default=inherit}
\end{docKey}

\begin{docKey} {description label font-size} { = \meta{font size} } {initial, default=inherit}
\end{docKey}

\begin{docKey} {description label font-weight} { = \meta{font weight} } {initial, default=inherit}
\end{docKey}

\begin{docKey} {description label font-shape} { = \meta{font shape} } {initial, default=inherit}
\end{docKey}

\begin{docKey} {description label color} { = \meta{color name} } {initial, default=thedescriptionlabelcolor}
\end{docKey}

\begin{docKey} {description label sep} { = \meta{dim} } {initial, default = 1em}
\end{docKey}

\begin{docKey} {description label width} { = \meta{dim} } {initial, default = 1em}
\end{docKey}

\begin{docKey} {description margin left} { = \meta{dim} } {initial, default = 0em}
\end{docKey}

\begin{docKey} {description margin right} { = \meta{dim} } {initial, default = 0em}
\end{docKey}

\begin{docKey} {description item indent} { = \meta{dim} } {initial, default = 0em}
\end{docKey}

Unlike the enumerate and itemize environment, the description list environment is defined in the book class.
The environment is defined as:
\begin{teXXX}
\newenvironment{description}
  {\list{}{\labelwidth\z@ \itemindent-\leftmargin
    \let\makelabel\descriptionlabel}}
  {\endlist}
\newcommand*\descriptionlabel[1]{\hspace\labelsep\labelcolor@cx
  \normalfont\bfseries #1}
\end{teXXX}


What is important to notice here is that all the standard list parameters are left essentially unchanged. The only item that is affected is \lstinline{\makelabel}, which is redefined in \lstinline{description} label.


We can define a number of keys for ease of formatting such descriptions lists rather than each time redefining them.


%
%\begin{tcolorbox}[title=Basic description list keys]
%\begin{lstlisting}
%\cxset{
% description label font-size/.store in=\descriptionlabelfontsize@cx,
% description label font-weight/.store in=\descriptionlabelfontweight@cx,
% description label font-family/.store in=\descriptionlabelfontfamily@cx,
% description label font-shape/.store in=\descriptionfontshape@cx,
% description label color/.store in=\descriptionlabelcolor@cx,
% description label sep/.store in=\descriptionlabelsep@cx,
% description label width/.store in=\descriptionlabelwidth@cx,
% description margin left/.store in=\descriptionmarginleft@cx,
% description margin right/.store in=\descriptionmarginright@cx,
% description item indent/.store in=\descriptionitemindent@cx,
% list parindent/.store in=\descriptionlistparindent@cx,
%}
%\end{lstlisting}
%\end{tcolorbox}




%\begin{tcolorbox}[title=Basic description list keys]
%\begin{lstlisting}
%\def\setdescription#1{%
%\cxset{#1}%
%\renewenvironment{description}%
%{\list{}{\listparindent\descriptionlistparindent@cx%
%                       \leftmargin=\descriptionmarginleft@cx%
%                       \rightmargin=\descriptionmarginright@cx%
%                       \itemindent\descriptionitemindent@cx%
%                       \labelwidth\descriptionlabelwidth@cx%
%                       \labelsep=\descriptionlabelsep@cx%
%                       \let\makelabel\descriptionlabel}}%
%               {\endlist}%
%%
%\renewcommand\descriptionlabel[1]{%
%  \fboxrule0pt\fboxsep0pt%
%  \hspace\descriptionlabelsep@cx%
%  \fbox{\color{\descriptionlabelcolor@cx}%
%  \normalfont\bfseries\raggedleft##1\thickspace%
%}}%
%}
%\end{lstlisting}
%\end{tcolorbox}



\begin{description}
\item [Special] \lorem
\item [Another] \lorem
\end{description}



\section{Creating new description like environments}

The macro \docAuxCmd*{NewDescriptionEnvironment} can be used to redefine new description like environments.

\begin{texexample}{Define a new description list environment}{ex:newdesclist}
% define the orange description environment
\NewDescriptionEnvironment
  [
    description label font-size     = small,
    description label font-weight   = bfseries,
    description label font-family   = serif,
    description label font-shape    = italic,
    description label color         = bgsexy,
    description label sep           = 3.5pt\relax,
    description label width         = 40pt,
    description margin left         = 50pt,
    description margin right        = 20pt,
    description item indent         = -2.5pt,
    list parindent=1em,
  ]
  {orangedescription}

% Sample
The \texttt{orangedescription} environment in action.
\begin{orangedescription}
 \item[One] \lorem
 \item[Two] \lorem
 \item[Three] \lorem
\end{orangedescription}

\lorem

\makeatother
\end{texexample}



\section{Example: redefining a description list}
We will now develop a description environment, that can be useful for the documentation of packages to describe options. We will use a description list as the basis of the environment. We define the following key values.
|\itemindent-\leftmargin|
%%% reset some of the values
\cxset{description label color=black, 
         description margin left=60pt, 
         description item indent=70pt,
         description margin right=50pt}

\begin{description}
\item[The description label font-size] This is the first item. \the\itemindent (itemindent) \the\leftmargin (leftmargin)
\item[The description label font-weight] This is the second item. \lipsum*[1]
\end{description}

\cxset{
         description margin left=60pt, 
         description item indent=30pt,
         }
         
\begin{description}
\item[The description label font-size] This is the first item. \the\itemindent (itemindent) \the\leftmargin (leftmargin)
\item[The description label font-weight] This is the second item. \lipsum*[1]
\end{description}

\cxset{
         description margin left=10pt, 
         description margin right=20pt,
         description item indent=80pt,
         }
         
\begin{description}
\item[The description label font-size] This is the first item. \the\itemindent (itemindent) \the\leftmargin (leftmargin)
\item[The description label font-weight] This is the second item. \lipsum[1]
\end{description}
\endinput

\section{Enumerated lists}


\begin{enumerate}
\item one
\item two
\item three
\end{enumerate}

Enumerated (numbered) list environments are characterized by numbering. They use a variety of fields and counters as shown in table.

\subsection{Vertical skips}

By default LaTeX adds vertical skips, as shown in figure 1. The definition of these skips is influenced by the font size and are defined in the \texttt{bk10.clo} files, hence hard to find and change. Each level of the list has its own definition as \lstinline{\@listi}.

\bigskip
\tcbset{width=\linewidth,arc=1mm,before=\bigskip,left=8mm}

\begin{tcolorbox}[title=Extract from bk10.clo]
\begin{lstlisting}
\def\@listi{\leftmargin\leftmargini
            \parsep 4\p@ \@plus2\p@ \@minus\p@
            \topsep 8\p@ \@plus2\p@ \@minus4\p@
            \itemsep4\p@ \@plus2\p@ \@minus\p@}
\let\@listI\@listi
\@listi

\def\@listii {\leftmargin\leftmarginii
              \labelwidth\leftmarginii
              \advance\labelwidth-\labelsep
              \topsep    4\p@ \@plus2\p@ \@minus\p@
              \parsep    2\p@ \@plus\p@  \@minus\p@
              \itemsep   \parsep}
\def\@listiii{\leftmargin\leftmarginiii
              \labelwidth\leftmarginiii
              \advance\labelwidth-\labelsep
              \topsep    2\p@ \@plus\p@\@minus\p@
              \parsep    \z@
              \partopsep \p@ \@plus\z@ \@minus\p@
              \itemsep   \topsep}
\def\@listiv {\leftmargin\leftmarginiv
              \labelwidth\leftmarginiv
              \advance\labelwidth-\labelsep}
\def\@listv  {\leftmargin\leftmarginv
              \labelwidth\leftmarginv
              \advance\labelwidth-\labelsep}
\def\@listvi {\leftmargin\leftmarginvi
              \labelwidth\leftmarginvi
              \advance\labelwidth-\labelsep}
\end{lstlisting}
\end{tcolorbox}

\endinput
%\cxset{enumerate numberingi/.is choice,
%  enumerate numberingi/.code={\renewcommand\theenumi {\csname#1\endcsname{enumi}}},
%  enumerate numberingii/.code={\renewcommand\theenumii {\csname#1\endcsname{enumii}}},
%  enumerate numberingiii/.code={\renewcommand\theenumiii {\csname#1\endcsname{enumiii}}},
%  enumerate numberingiv/.code={\renewcommand\theenumiv {\csname#1\endcsname{enumiv}}},
%  enumerate labeli punctuation/.store in=\enumeratepunctuationi@cx,
%  enumerate labeli/.is choice,
%  enumerate labeli/brackets/.code={\renewcommand\labelenumi{(\theenumi\enumeratepunctuationi@cx)}},
%  enumerate labeli/square brackets/.code={\renewcommand\labelenumi{[\theenumi\enumeratepunctuationi@cx]}},
%  enumerate labeli/right bracket/.code={\renewcommand\labelenumi{\theenumi\enumeratepunctuationi@cx)}},
%  enumerate label left/.store in=\enumeratelabelleft@cx,
%  enumerate label right/.code=\renewcommand\labelenumi{\enumeratelabelleft@cx\theenumi\enumeratepunctuationi@cx#1},
%  enumerate leftmargini/.code={\setlength\leftmargini{#1}},
%  enumerate leftmarginii/.code={\setlength\leftmarginii{#1}},
%  enumerate leftmarginiii/.code={\setlength\leftmarginiii{#1}},
%  enumerate leftmarginiv/.code={\setlength\leftmarginiv{#1}},
%  listi topsep/.store in=\listitopsep@cx,
%  listi partopsep/.store in=\listipartopsep@cx,
%  listi itemsep/.store in=\listiitemsep@cx,
%  listi parsep/.store in=\listiparsep@cx,
%  listii topsep/.store in=\listiitopsep@cx,
%  listii partopsep/.store in=\listiipartopsep@cx,
%  listii itemsep/.store in=\listiiitemsep@cx,
%  listii parsep/.store in=\listiiparsep@cx,
%  listiii topsep/.store in=\listiiitopsep@cx,
%  listiii partopsep/.store in=\listiiipartopsep@cx,
%  listiii itemsep/.store in=\listiiiitemsep@cx,
%  listiii parsep/.store in=\listiiiparsep@cx,
%}
%
%\cxset{compact1/.style={%
%  enumerate numberingi=arabic,
%  enumerate numberingii=alph,
%  enumerate numberingiii=alph,
%  enumerate numberingiv=roman,
%  enumerate labeli punctuation=.,
%  enumerate label left=,
%  enumerate label right=,
%  enumerate leftmargini=2.2em,
%  enumerate leftmarginii=2.1em,
%  enumerate leftmarginiii=1.5em,
%  enumerate leftmarginiv=2em,
%  listi topsep=8\p@ \@plus2\p@ \@minus\p@,
%  listi itemsep=0\p@ \@plus2\p@ \@minus\p@,
%  listi parsep=0\p@ \@plus2\p@ \@minus\p@,
%  listii topsep=0\p@ \@plus2\p@ \@minus\p@,
%  listii itemsep=0\p@ \@plus2\p@ \@minus\p@,
%  listii parsep=0\p@ \@plus2\p@ \@minus\p@,
%  listiii topsep=0\p@ \@plus2\p@ \@minus\p@,
%  listiii itemsep=0\p@ \@plus2\p@ \@minus\p@,
%  listiii parsep=0\p@ \@plus2\p@ \@minus\p@,
%}}

%\cxset{compact2/.style={%
%  enumerate numberingi=alph,
%  enumerate numberingii=roman,
%  enumerate numberingiii=alph,
%  enumerate numberingiv=roman,
%  enumerate labeli punctuation=,
%  enumerate label left=(,
%  enumerate label right=),
%  enumerate leftmargini=2.2em,
%  enumerate leftmarginii=2.1em,
%  enumerate leftmarginiii=1.5em,
%  enumerate leftmarginiv=2em,
%  listi topsep=8\p@ \@plus2\p@ \@minus\p@,
%  listi itemsep=0\p@ \@plus2\p@ \@minus\p@,
%  listi parsep=0\p@ \@plus2\p@ \@minus\p@,
%  listii topsep=0\p@ \@plus2\p@ \@minus\p@,
%  listii itemsep=0\p@ \@plus2\p@ \@minus\p@,
%  listii parsep=0\p@ \@plus2\p@ \@minus\p@,
%  listiii topsep=0\p@ \@plus2\p@ \@minus\p@,
%  listiii itemsep=0\p@ \@plus2\p@ \@minus\p@,
%  listiii parsep=0\p@ \@plus2\p@ \@minus\p@,
%}}


%\def\setenumerate#1{
%\cxset{#1}
%\def\@listi{\leftmargin\leftmargini
%            \parsep\listiparsep@cx
%            \topsep\listitopsep@cx\relax
%            \itemsep\listiitemsep@cx}
%\def\@listii{\leftmargin\leftmarginii
%            \parsep\listiiparsep@cx
%            \topsep\listiitopsep@cx\relax
%            \itemsep\listiiitemsep@cx}
%\def\@listiii{\leftmargin\leftmarginiii
%            \parsep\listiiiparsep@cx
%            \topsep\listiiitopsep@cx\relax
%            \itemsep\listiiiitemsep@cx}
%}

\setenumerate{compact1}


The list can be viewed here:

\begin{enumerate}
\item Level i
      \begin{enumerate}
       \item Level ii
          \begin{enumerate}
            \item Level iii
              \begin{enumerate}
                \item Level iv. \lipsum*[1]
              \end{enumerate}
          \end{enumerate}
      \end{enumerate}
\end{enumerate}


\begin{tcblisting}{title=Example with style \textit{compact2}}

\cxset{compact2/.style={%
  enumerate numberingi=alpha,
  enumerate numberingii=roman,
  enumerate numberingiii=alpha,
  enumerate numberingiv=roman,
  enumerate labeli punctuation=,
  enumerate label left=(,
  enumerate label right=),
  enumerate leftmargini=2.2em,
  enumerate leftmarginii=2.1em,
  enumerate leftmarginiii=1.5em,
  enumerate leftmarginiv=2em,
  listi topsep=8\p@ \@plus2\p@ \@minus\p@,
  listi itemsep=0\p@ \@plus2\p@ \@minus\p@,
  listi parsep=0\p@ \@plus2\p@ \@minus\p@,
  listii topsep=0\p@ \@plus2\p@ \@minus\p@,
  listii itemsep=0\p@ \@plus2\p@ \@minus\p@,
  listii parsep=0\p@ \@plus2\p@ \@minus\p@,
  listiii topsep=0\p@ \@plus2\p@ \@minus\p@,
  listiii itemsep=0\p@ \@plus2\p@ \@minus\p@,
  listiii parsep=0\p@ \@plus2\p@ \@minus\p@,
}}
\setenumerate{compact2}
\begin{enumerate}
\item Does this project actually merit the use of the Minor Works Form or Intermediate Form instead of their `grown up' relatives?
\item Do the number of PC or prime cost items mean that it would be more desirable to use a re-measurable form?
\item Is this a contract which merits the production of full scale bills
of quantities or is something more standardised going to suffice?
\end{enumerate}
\end{tcblisting}

As you will observe the numbering in the above example has been enclosed in round brackets, using:

\begin{tcolorbox}
\begin{lstlisting}
  enumerate label left=(,
  enumerate label right=),
\end{lstlisting}
\end{tcolorbox}

The next example is from the \textit{LaTeX Companion}. In example~\ref{ex:companion}, the first-level list elements are decorated with the section sign (\S) as a prefix and a period as a suffix (omitted in references). We will
define this as a style named \textit{paragraphsymbol} for the lack of any better name. This style can sometimes be found in legal texts.

\begin{texexample}{Paragraph symbols in enumerate}{ex:companion}
\cxset{paragraphsymbol/.style={%
  enumerate numberingi=arabic,
  enumerate labeli punctuation=.,
  enumerate label left=\S,
  enumerate label right=,
}}
\setenumerate{paragraphsymbol}
\begin{enumerate}
\item \lorem
\item \lorem
\item \lorem
\end{enumerate}
\end{texexample}
\
\section{Creating enumerated environments}

New enumerated environments cab be created by using the macro \lstinline{\newenumeratedenvironment}. Keys are set as either styles or individually.

\def\newenumeratedenvironment#1#2{%
 \expandafter\def\csname#1\endcsname{%
 \cxset{#2}
 \ifnum \@enumdepth >\thr@@\@toodeep\else
 \advance\@enumdepth\@ne
 \edef\@enumctr{enum\romannumeral\the\@enumdepth}%
 \expandafter
 \list
 \csname label\@enumctr\endcsname
 {\usecounter\@enumctr\def\makelabel####1{\hss\llap{####1}}}%
 \fi}
 \expandafter\let\csname end#1\endcsname=\endlist
}


\begin{texexample}{An enumerated list factory}{}
\newenumeratedenvironment{paragraphsymbol}{
  enumerate numberingi=roman,
  enumerate labeli punctuation=.,
  enumerate label left={\textcolor{purple}{\P}},
  enumerate label right=,
}
\begin{paragraphsymbol}
\item \lorem
\item \lorem
\item \lorem
      \begin{itemize}
        \item This is bullets
      \end{itemize}
\end{paragraphsymbol}
\end{texexample}


\newpage
\section{Setup keys for enumerate lists}
\begin{description}
\item [enumerate numberingi] Sets the numbering style of the list at level $n$. Valid values are \textit{Alph, alph, arabic, Roman, roman, WORDS, words}.
\end{description}

\clearpage

\section{Itemized lists}

The itemized \LaTeX\ lists are similar to those for the enumerated lists. However they are somehow simpler as there is no need for counters.

\bigskip
\begin{tcolorbox}[width=\linewidth,arc=2mm,title=Default \LaTeX\ parameters for itemized lists]
\begin{lstlisting}
\newcommand\labelitemi{\textbullet}
\newcommand\labelitemii{\normalfont\bfseries \textendash}
\newcommand\labelitemiii{\textasteriskcentered}
\newcommand\labelitemiv{\textperiodcentered}
\end{lstlisting}
\end{tcolorbox}





\begin{itemize}
\item Level i
      \begin{itemize}
       \item Level ii
          \begin{itemize}
            \item Level iii
              \begin{itemize}
                \item Level iv. \lipsum*[1]
              \end{itemize}
          \end{itemize}
      \end{itemize}
\end{itemize}

\cxset{red/.style={
 labelitemi={{\color{green}\ding{'64}}},
 labelitemii=\color{red}\textendash,
 labelitemiii=\textasteriskcentered,
 labelitemiv=\textperiodcentered,
}}

Now that we have managed to abstract the itemized environment we can generate a new environment factory.

\def\newitemizedenvironment#1#2{
\expandafter\def\csname#1\endcsname{%
 \cxset{#2}%
 \ifnum \@itemdepth >\thr@@\@toodeep\else
 \advance\@itemdepth\@ne
 \edef\@itemitem{labelitem\romannumeral\the\@itemdepth}%
 \expandafter
 \list
 \csname\@itemitem\endcsname
 {\def\makelabel####1{\hss\llap{####1}}}%
 \fi}
 \expandafter\let\csname end#1\endcsname=\endlist
}

%\newitemizedenvironment{reditemize}{black}
%
%
%\begin{reditemize}
%\item Test.
%   \begin{reditemize}
%    \item test.
%   \end{reditemize}
%\end{reditemize}
%
%\begin{itemize}
%\item Level i
%      \begin{itemize}
%       \item Level ii
%          \begin{itemize}
%            \item Level iii
%              \begin{itemize}
%                \item Level iv. \lipsum*[1]
%              \end{itemize}
%          \end{itemize}
%      \end{itemize}
%\end{itemize}


\section{Itemized lists with ding symbols}

So far we have used both standard symbols as well as those provided by the pifont that offers numerous,
dingbang symbols. The pifont package also offers environments to do that more easily.


\begin{texexample}{dinglist}{}
\begin{dinglist}{"E4}
\item The first item. \item The second
item in the list.
\end{dinglist}
\end{texexample}

\begin{dingautolist}{'300}
\item The first item in the list.\label{lst:a}
\item The second item in the list.\label{lst:b}
\item The third item in the list.\label{lst:c}
\item The fourth item in the list.\label{lst:d}
\end{dingautolist}

\newenvironment{steps}{\dingautolist{'300}}{\enddingautolist}

\begin{steps}
\item The first item in the list.\label{lst:a}
\item The second item in the list.\label{lst:b}
\item The third item in the list.\label{lst:c}
\item The fourth item in the list.\label{lst:d}
\end{steps}

\makeatother
\endgroup
