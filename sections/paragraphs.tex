\parindent1em

\chapter{Paragraphs and Lists}

\epigraph{The paragraph is essentially a unit of thought, not a length}{H.W.Fowler (1858-1933)}

\noindent Paragraphs represent a distinct logical step within the whole argument expounded in section of a document. The \texttt{Tufte-book} class has a control sequence that is named \cmd{\newthought} to reinforce the idea that a paragraph must start with a new thought or argument. How does this particular paragraph contribute to the argument? 
What logical step does it make? Where does it fit in the overall chain?

\section{Historical Notes}

The oldest mark of punctuation in Greek manuscripts is the paragraph. It first occurs as a horizontal stroke (sometimes with a dot over it), placed at the beginning of a line, just beneath the first two or three letters.

This was followed by the paragph mark the pilcrow\footnote{See also \protect\url{http://www.smithsonianmag.com/arts-culture/the-origin-of-the-pilcrow-aka-the-strange-paragraph-symbol-8610683/?no-ist}} (\S). 

After the establishment of indentation the method of marking paragraphs becomes essentially what we find today. At first the old mark was still use for emphasis. But this custom was short-lived.

In the eighteenth century it was a printer’s custom to print the first word of each paragraph in capitals. 

It remains to consider the origin of the so-called section 
mark [\S], called on the continent, \emph{paragraphe}. The genesis of 
this mark has been explained in two different ways. The first 
of these is equally ingenious and ingenuous. It is thus 
expressed in an American treatise on composition and rhetoric . 
" The Section [\S], the mark for which seems to be a combina- 
tion of two s's, standing for \emph{signum sectionis}, the sign of the 
section." The theory is still more definitely expounded  in 
'Quackenbos, Course of Composition and Rhetoric, p. 145. 


\section{Typesetting paragraphs}

Typesetting paragraphs with \tex does not require any particular effort from the user, other than leaving a blank line to separate a paragraph from other page elements.

\begin{texexample}{Paragraph marking}{} 

In olden times when wishing
still helped one, there lived a
king whose daughters were all
beautiful, but the youngest was so
beautiful that the sun itself,
which has seen so much, was
astonished whenever it shone in
her face. 

Close by the king's
castle lay a great dark forest,
and under an old lime-tree in the
forest was a well, and when
the day was very warm, the
king's child went out into the 
forest and sat down by the side
of the cool fountain, and when she was bored she
took a golden ball, and threw it up on a high and caught it, and this
ball was her favorite plaything.

 This is a paragraph with Maths,
 \[d=a+b+c\]
 where $d=sum$.
\end{texexample}



\section{First line indentation and paragraph separation}

In most books good typography dictates that the first line of a paragraph is indented. \tex provides two commands that can be used for first line indentation. The first one is \cs{parindent} which is a length expressed normally in |ems|. The \cs{noindent} does what its name implies. The paragraph indentation is sometimes resisted by newcomers to \tex, however most professionally printed material in English typically does not indent the first paragraph, but indents those that follow. For example, Robert Bringhurst states that we should "Set opening paragraphs flush left."\footnote{Bringhurst, Robert (2005). \textit{The Elements of Typographic Style}. Vancouver: Hartley and Marks. p. 39. ISBN 0-88179-206-3.} Bringhurst explains as follows.

"The function of a paragraph is to mark a pause, setting the paragraph apart from what precedes it. If a paragraph is preceded by a title or subhead, the indent is superfluous and can therefore be omitted."[4]

The Elements of Typographic Style states that "at least one en [space]" should be used to indent paragraphs after the first,[4] noting that that is the "practical minimum".[5] An em space is the most commonly used paragraph indent.[5] Miles Tinker, in his book Legibility of Print, concluded that indenting the first line of paragraphs increases readability by 7\%, on the average.[6] Where longer lines of text are used it is not uncommon to indent the first paragraph line by at least two ems.

\begin{docCommand}{parindent} { \meta{dim} }
\begin{docCommand}{parskip} {\meta{dim}}
\begin{docCommand}{noindent}{}
\LaTeXe has basic parameters that control the appearance of normal paragraphs,
\cs{parindent} and  \cs{parskip}.
The length \cs{parindent}  is the indentation of the first line of a paragraph and the length
parskip is the vertical spacing between paragraphs, as illustrated in \ref{fig:paragaraph}. The
value of \cs{parskip} is usually 0pt, and \texttt{parindent} is usually defined in terms of \textit{ems}
so that the actual indentation depends on the font being used. If \texttt{parindent} is set to a
negative length, then the first line of the paragraphs will be \textit{outdented} into the lefthand
margin.
\end{docCommand}
\end{docCommand}
\end{docCommand}



\subsection{Block paragraph}

A block paragraph is obtained by setting \cs{parindent} to |0em|; \cs{parskip} should be set to
some positive value so that there is some space between paragraphs to enable them to be
identified. Most typographers heartily dislike block paragraphs, not only on aesthetical
grounds but also on practical considerations. Consider what happens if the last line of a
block paragraph is full and also is the last line on the page. The following block paragraph

It is important to know that \latex typesets paragraph by paragraph. For example, the
\cs{baselineskip} that is used for a paragraph is the value that is in effect at the end of the
paragraph, and the font size used for a paragraph is according to the size declaration (e.g.,
large or normalsize or small) at the end of the paragraph, and the raggedness or
otherwise of the whole paragraph depends on the declaration (e.g., \texttt{centering}) in effect
at the end of the paragraph. If a pagebreak occurs in the middle of a paragraph TeX will
not reset the part of the paragraph that goes onto the following page, even if the textwidths
on the two pages are different.


\subsection{Hanging paragraphs}
 
 \begin{docCommand}{hangafter}{\meta{number of lines}}
\begin{docCommand}{hangindent}{\meta{dim}}
A hanging paragraph is one where the length of the first few lines differs from the length
of the remaining lines. A normal indented paragraph may be considered to be a special case of a hanging paragraph where few is one. There are two commands controlling this \cs{hangafter} and \cs{hangindent}, which are both provided by \tex.
\end{docCommand}
\end{docCommand}


These commands can be used - very carefully to arrange wrapping figures
and drop capitals. 

\begin{texexample}{Hanging paragraphs}{}
\hangindent 8em  \hangafter 3  \footnotesize
Adeste hendecasyllabi. quot estis 
omnes. undique quotquot estis omnes. 
iocum me putat esse moecha turpis. 
et negat mihi nostra reddituram 
pugillaria si pati potestis. 
persequamur eam. et reflagitemus. 
quae sit quaeritis. illa quam uidetis 
turpe incedere mimice ac moleste 
ridentem catuli ore Gallicani. 
circumsistite eam. et reflagitate. 
moecha putida. redde codicillos. 
redde putida moecha codicillos. 
non assis facis. o lutum. lupanar, 
aut si perditius potest quid esse. 
sed non est tamen hoc satis putandum 
quod si non aliud potest ruborem 
ferreo canis exprimamus ore. 
conclamate iterum altiore uoce. 
moecha putide. redde codicillos. 
redde putida moecha moecha codicillos. 
sed nil proficimus. nihil mouetur. 
mutanda est ratio modusque uobis 
siquid proficere amplius potestis. 
pudica et proba. redde codicillos.

\end{texexample}


As you probably have guessed, this can be used to wrap figures into the text, although this is hardly necessary. 

Using \cs{hangindent} at the start of a paragraph will cause the paragraph to be hung.
If the length \meta{indent} is positive the lefthand end of the lines will be indented but
if it is negative the righthand ends will be indented by the specified amount. If the
number $num$, say $N$, is negative the first $N$ lines of the paragraph will be indented while
if $N$ is positive the $N+1$ the lines onwards will be indented. 

There should be no space between the  command and
the start of the paragraph. 


\section{Centering lines}

Lines can be centered using the \cs{centerline} command. We can use it to center a small phrase commonly found
in typography \footnote{"The quick brown fox jumps over the lazy dog" is an English-language pangram (a phrase that contains all of the letters of the alphabet). It has been used to test typewriters and computer keyboards, and in other applications involving all of the letters in the English alphabet. Owing to its shortness and coherence, it has become widely known and is often used in visual arts.}.

\noindent\centerline{\small\fox}

We can achieve the same effect using \TeX\  primitives \cs{hfil} and writing \verb+\hfil\small\fox\hfil+
\medskip

{\hfil\small\fox\hfil}


This will give a slightly different center?

\section{Flush and Rugged}

Flushleft text has the lefthand end of the lines aligned vertically at the lefthand margin
and flushright text has the righthand end of the lines aligned vertically at the righthand
margin. The opposites of these are raggedleft text where the lefthand ends are not aligned
and raggedright where the righthand end of lines are not aligned. LaTeX normally typesets
flushleft and flushright.

\topline

{\small \begin{flushleft} \lorem \end{flushleft}}

{\small \begin{flushright} \lorem \end{flushright}}

\bottomline

\section{Centered text}
\latex provides an environment for centering blocks of text, as well as a single command \cmd{centering}. 

\begin{texexample}{Centering Text}{}
\begin{center}
In the beginning

Then God created Newton,

And objects at rest tended to remain at rest,

And objects in motion tended to remain in motion,

And energy was conserved and momentum was conserved and

matter was conserved

And God saw that it was conservative.
\end{center}
\end{texexample}


Text in a flushleft environment is typeset flushleft and raggedright, while in a
flushright environment is typeset raggedleft and flushright. In a center environment
the text is set raggedleft and raggedright, and each line is centered. A small vertical space
is put before and after each of these environments.


\section{Other paragraph styles}

\tex's paragraph builder can be accessed in \tex or \latex derived formats by boxing and unboxing the text and using the command \cmd{\lastbox} to manipulate the contents as an example consider the following problem:

\def\weirdtitle#1{%
       \bgroup
       \setbox0=\vbox{\bf\noindent #1}%
       \setbox1=\vbox{%
            \unvbox0
            \setbox2=\lastbox
            \hbox to \linewidth{\hfill\unhbox2 \hfill}%
       }%
       \unvbox1
      \egroup
  }%

\def\wavelast#1{%
       \bgroup
       \setbox0=\vbox{\bf\noindent #1}%
       \setbox1=\vbox{%
            \unvbox0
            \setbox2=\lastbox
            \hbox to \linewidth{\hfill\uwave{\unhbox2}\hfill}%
       }%
       \unvbox1
      \egroup
  }%
  
\begin{scriptexample}{example}{}
\weirdtitle{A Dialogue between the Landlady, and Susan the Chambermaid, proper to be
read by all Innkeepers, and their Servants; with the Arrival, and
affable Behaviour of a beautiful young Lady; which may teach Persons of
Condition how they may acquire the Love of the whole World.}
\end{scriptexample}

The example was from an old question at \tex{}MAG.\footnote{\url{http://dante.ctan.org/tex-archive/info/digests/tex-mag/v2.n2}.} The solutions offered varied but the one used here, is what was considered to be the most elegant. 

\begin{teXXX}
\def\weirdtitle#1{%
       \bgroup
       \setbox0=\vbox{\bf\noindent #1}%
       \setbox1=\vbox{%
            \unvbox0
            \setbox2=\lastbox
            \hbox to \linewidth{\hfill\unhbox2 \hfill}%
       }%
       \unvbox1
      \egroup
  }%
\end{teXXX}

The solution is to put the paragraph in a box |\box0| and then manipulate the contents in a second box |\box1|. In |box 1| we unvbox the box (causing it to be typeset) and then in yet a third box we pick the last line (\cmd{\lastbox}). This is then placed in a horizontal list and using appropriate glue we center the text. 

\subsection{Underlining the last line of the text}

The next example appears in \autoref{ch:tulipmania}. The last line of the caption is centered and a wavy line drawn underneath it.

\begin{scriptexample}{wavelast}{}
\wavelast{A Dialogue between the Landlady, and Susan the Chambermaid, proper to be
read by all Innkeepers, and their Servants; with the Arrival, and
affable Behaviour of a beautiful young Lady; which may teach Persons of
Condition how they may acquire the Love of the whole World.}
\end{scriptexample}

\emphasis{uwave}
\begin{teX}
\def\wavelast#1{%
       \bgroup
       \setbox0=\vbox{\bf\noindent #1}%
       \setbox1=\vbox{%
            \unvbox0
            \setbox2=\lastbox
            \hbox to \linewidth{\hfill\uwave{\unhbox2}\hfill}(*@\label{lin:uwave}@*)%
       }%
       \unvbox1
      \egroup
  }%
\end{teX}

What just happened is that in line [\ref{lin:uwave}] we unboxed the last line and then centered it, by using |\hfill| glue on each side. We then undelined it using a modified version of the command \cmd{\uwave} from the \pkgname{ulem} package.

\def\weirdtitlei#1{%
       \bgroup
       \setbox0=\vbox{\bf\noindent #1}%
       \setbox1=\vbox{%
            \unvbox0
            \setbox2=\lastbox
            \hbox to \linewidth{\hfill\unhbox2}%
       }%
       \unvbox1
      \egroup
  }%

The |\hbox| can easily be changed to use "Russian style" last lines in paragraphs.

\begin{scriptexample}{example}{}
\weirdtitlei{A Dialogue between the Landlady, and Susan the Chambermaid, proper to be
read by all Innkeepers, and their Servants; with the Arrival, and
affable Behaviour of a beautiful young Lady; which may teach Persons of
Condition how they may acquire the Love of the whole World.}
\end{scriptexample}

\section{everypar}

\tex performs another action when it starts a paragraph:
it inserts whatever is currently the contents of the \emph{token
list} \cs{everypar}. Usually you will no™t notice this, because
the token list is empty in plain TEX (the TEX book [3]
gives only a simple example, and the exhortation  €˜if you
let your imagination run you will think of better applications €™).
\latex, however, makes regular use of
\cs{everypar}. Some mega-trickery with \cs{everypar}
can be found in \cite{Lamport1994}. 

When \tex enters horizontal mode, it will interrupt its normal scanning to read
tokens that were predefined by the command everypar={token list}. For
example, suppose you have said `everypar={A}'. If you type `B' in vertical mode, TEX
will shift to horizontal mode (after contributing parskip glue to the current page),
and a horizontal list will be initiated by inserting an empty box of width |parindent|.

Then \tex will read `AB', since it reads the everypar tokens before getting back to the
`B' that triggered the new paragraph. Of course, this is not a very useful illustration of
\cs{everypar}; but if you let your imagination run you will think of better applications.

\begin{texexample}{everypar}{}
\def\makefirstwordbold#1.{\textbf{#1 }}
\everypar{\makefirstwordbold}
This is the first paragraph.\par
This is the second paragraph.\par
\everypar{}
\end{texexample}


We can use \cs{everypar} to add bullets to all paragraphs or a symbol such as the paragraph symbol.
\medskip

\verb+\everypar={$\bullet\quad$}+

\begin{texexample}{everypar add bullets}{}
\everypar={$\bullet\quad$}

This is a test

This is a test

\everypar={}
\end{texexample}

\section{Double spacing}

Some people---especially those of control of formatting Theses---like documents to be \textit{double spaced}, such Gestapo type imposition of one's own taste of design normally result in making these documents harder to read but perhaps that is the intention or as \cite{Abrahams2003, Wilson2009} they have `\ldots shares in papermills and lumber companies'. As an Engineer I had countless encounters with overzealous Consultants which actually specified in Construction Specifications that arial had to be used, text had to be doublespacedgin etc. 

\begin{docCommand}{onehalfspacing}{}
\begin{docCommand}{doublespacing}{}
The package \texttt{setspace} \cite{setspace} can be used to make life easier, just include the package and use \cs{onehalfspacing} or \cs{doublespacing}.
\end{docCommand}
\end{docCommand}


\section{Controlling the width of a paragraph}
\subsection{Minipages}


\begin{minipage}{6.7cm}
\parindent=0pt 
{\obeylines

adeste hendecasyllabi. quot estis 
omnes. undique quotquot estis omnes. 
iocum me putat esse moecha turpis. 
et negat mihi nostra reddituram 
pugillaria si pati potestis. 
persequamur eam. et reflagitemus. 
quae sit quaeritis. illa quam uidetis 
turpe incedere mimice ac moleste 
ridentem catuli ore Gallicani. 
circumsistite eam. et reflagitate. 
moecha putida. redde codicillos. 
redde putida moecha codicillos. 
non assis facis. o lutum. lupanar, 
aut si perditius potest quid esse. 
sed non est tamen hoc satis putandum 
quod si non aliud potest ruborem 
ferreo canis exprimamus ore. 
conclamate iterum altiore uoce. 
moecha putide. redde codicillos. 
redde putida moecha moecha codicillos. 
sed nil proficimus. nihil mouetur. 
mutanda est ratio modusque uobis 
siquid proficere amplius potestis. 
pudica et proba. redde codicillos.

\hfil Catullus\par}
\end{minipage}
\hspace{0.8em}
\begin{minipage}{8cm}
{\obeylines
Come here, nasty words, so many I can hardly 
tell where you all came from. 
That ugly slut thinks I'm a joke 
and refuses to give us back 
the poems, can you believe this shit? 
Lets hunt her down , and demand them back! 
Who is she, you ask? That one, who you see 
strutting around, with ugly clown lips, 
laughing like a pesky French poodle. 
Surround her, ask for them again! 
"Rotten slut, give my poems back! 
Give 'em back, rotten slut, the poems!" 
Doesn't give a shit? Oh, crap. Whorehouse. 
Or if anything's worse, you're it. 
But I've not had enough thinking about this. 
If nothing else, lets make that 
pinched bitch turn red-faced. 
All together shout, once more, louder: 
"Rotten slut, give my poems back! 
Give 'em back, rotten slut, the poems!" 
But nothing helps, nothing moves her. 
A change in your methods is cool, 
if you can get anything more done. 
"Sweet thing, give my poems back!"\par

\hfil Catullus\par}
\end{minipage}


\section{obeylines}

\begin{docCommand}{obeylines}{}
You may have several consecutive lines of input for which you want the output
to appear line-for-line in the same way. One solution is to type \cs{par} at the
end of each input line; but that's somewhat of a nuisance, so plain TEX provides the
abbreviation `obeylines', which causes each end-of-line in the input to be like \cs{par}.
After you say obeylines you will get one line of output per line of input, unless an
input line ends with `\%' or unless it is so long that it must be broken. For example, you
probably want to use obeylines if you are typesetting a poem. 
\end{docCommand}

Be sure to enclose
obeylines in a group, unless you want this \textit{poetry} mode to continue to the end of
your document.  You can also use \cs{break} to break a paragraph at a specific point.  \footnote{but why would you want to do so?}\footnote{See source2e File b: ltplain.dtx Date: 2005/09/27 Version v1.1y 17 for the definition of \cs{obeylines}}

\begin{texexample}{obeylines}{ex:obeylines}
\obeylines
Roses are red, 
\quad Violets are blue; 
Rhymes can be typeset
\quad With boxes and glue. \footnote{From page 94 of the TeXBook} 

\end{texexample}

If you are familiar with with |HTML|, you can redefine the obeylines command to \cs{pre}, I find it easier to remember. Strictly speaking it should be the verbatim enevironment.

{\small
\begin{verbatim}
\newcommand{\pre}{\obeylines}
{\pre \small \em \smallskip
Roses are red,
\quad Violets are blue;
Rhymes can be typeset
\quad With boxes and glue.
\smallskip}
\end{verbatim}
}



{\obeylines
{\Large\bf  Catullus 42 \footnote{For a translation of the poem see \url{http://www.obscure.org/obscene-latin/catullus-42.html}}}

adeste hendecasyllabi. quot estis 
omnes. undique quotquot estis omnes. 
iocum me putat esse moecha turpis. 
et negat mihi nostra reddituram 
pugillaria si pati potestis. 
persequamur eam. et reflagitemus. 
quae sit quaeritis. illa quam uidetis 
turpe incedere mimice ac moleste 
ridentem catuli ore Gallicani. 
circumsistite eam. et reflagitate. 
moecha putida. redde codicillos. 
redde putida moecha codicillos. 
non assis facis. o lutum. lupanar, 
aut si perditius potest quid esse. 
sed non est tamen hoc satis putandum 
quod si non aliud potest ruborem 
ferreo canis exprimamus ore. 
conclamate iterum altiore uoce. 
moecha putide. redde codicillos. 
redde putida moecha moecha codicillos. 
sed nil proficimus. nihil mouetur. 
mutanda est ratio modusque uobis 
siquid proficere amplius potestis. 
pudica et proba. redde codicillos.


\hfil Catullus\par}


\bigskip
Another way to use |\obeylines| is in combination with |\everypar|

\begin{texexample}{everypar and obeylines}{}
{\obeylines\everypar{\hfill}\parindent=0pt
Mademoiselle from Armentires, Parlez-vous,
Mademoiselle from Armentires, Parlez-vous,
Mademoiselle from Armentires,
She hasn't been kissed for forty years.
Hinky-dinky parlez-vous.

Oh Mademoiselle from Armentires, Parlez-vous,
Mademoiselle from Armentires, Parlez-vous,
She got the palm and the croix de guerre,
For washin' soldiers' underwear,

Hinky-dinky parlez-vous.
\hfil World War I Army Song\par}
\end{texexample}

Roughly speaking, \TeX breaks paragraphs into lines in the following
way: Breakpoints are inserted between words or after hyphens so as to produce
lines whose badnesses do not exceed the current \cs{tolerance}. If there's no way
to insert such breakpoints, an overfull box is set. Otherwise the breakpoints are
chosen so that the paragraph is mathematically optimal, i.e., best possible, in
the sense that it has no more \cs{demerits} than you could obtain by any other
sequence of breakpoints. Demerits are based on the badnesses of individual lines
and on the existence of such things as consecutive lines that end with hyphens,
or tight lines that occur next to loose ones.  \footnote{Perhaps a still unsurpassed algorithm, by other software.}

In the TeXBook, Knuth gives this exercises for the reader. 

\begin{latexquotation}
Since \tex reads an entire paragraph before it makes any decisions about
line breaks, the computer's memory capacity might be exceeded if you are typesetting
the works of some philosopher or modernistic novelist who writes 200-line paragraphs.
Suggest a way to cope with such authors. \footnote{Assuming that the author is deceased and/or set in his or her ways, the remedy
is to insert {\cs{parfillskip=0pt} \cs{par} \cs{parskip=0pt} \cs{noindent}} in random places, after
each 50 lines or so of text. (Every space between words is usually a feasible breakpoint,
when you get sufficiently far from the beginning of a paragraph.)}
\end{latexquotation}

This brings almost to the end the discussion on paragraphs. A simple paragraph and so much to experiment with. If you writing for e-readers, perhaps we also need to redefine how often we use paragraphs. They should be much shorter to cater for shorter attention spans and scanning of text by users, but this is a discussion for some-where else.


\section{Narrowing paragraphs}

\begin{docCommand}{narrower}{}
You can use the command \cs{narrower} to indent paragraphs both sides by  an amount equal to the
\cs{parindent} value.
\end{docCommand}

\startlineat{214}
\begin{teXXX}
 \def\narrower{%
   \advance\leftskip\parindent
   \advance\rightskip\parindent}
\end{teXXX}

\begin{texexample}{narrower text}{}
\onepar

\narrower\small
\onepar\par
\end{texexample}

We can even make paragraphs doubly narrow by using \cs{narrower} \cs{narrower} in example \refCom{narrow}.

\begin{texexample}{narrowing both sides}{narrow}
The sentence \fox. has been typeset with normal paragraph settings.

\parindent3em
\narrower \narrower\small 
The sentence \fox has been typeset with larger left skips.
\medskip
\end{texexample}

\begin{docCommand}{leftskip}{}
\begin{docCommand}{rightskip}{}

In this paragraph we are back to the normal margins, as you can see for yourself. We will let it run on a little longer so that the margins are clearly visible.
\end{docCommand}
\end{docCommand}




\section{Shaping paragraphs with shapepar}


By using \cs{parshape}, you could literally make your paragraph any shape you want.
But if you want your paragraph to be shaped a heart, there €™s a package, \docpkg{shapepar}, that
could ease your work. The package provides a few predefined shapes that you could call
up by using \cs{diamondpar}, \cs{squarepar}, and \cs{heartpar}

The size is adjusted automatically so that the entire shape is filled with text. There may not be displayed maths or \verb+ €˜\vadjust €™+  material (no \verb+\vspace+) in the argument of shapepar. The macros work for both LaTeX and plain TeX. For LaTeX, specify usepackage{shapepar}; for Plain, input shapepar.sty.
shapepar works in terms of user-defined shapes, though the package does provide some predefined shapes: so you can form any paragraph into the form of a heart by putting heartpar{sometext...} inside your document. The tedium of creating these polygon definitions may be alleviated by using the shapepatch extension to transfig which will convert xfig output to shapepar polygon form.
The author is Donald Arseneau. The package is Copyright  © 1993,2002,2006 Donald Arseneau.



\newcommand{\abc}{abcdefghijklmnopqrstuvwxyz}

\fbox{\begin{minipage}{2cm}%
 \smallskip \baselineskip=7pt\tiny
\noindent \hfuzz 0.1pt
\parshape 30 0pt 120pt 1pt 118pt 2pt 116pt 3pt 112pt 6pt
108pt 9pt 102pt 12pt 96pt 15pt 90pt 19pt 84pt 23pt 77pt
27pt 68pt 30.5pt 60pt 35pt 52pt 39pt 45pt 43pt 36pt 48pt
27pt 51.5pt 21pt 53pt 16.75pt 53pt 16.75pt 53pt 16.75pt 53pt
16.75pt 53pt 16.75pt 53pt 16.75pt 53pt 16.75pt 53pt 16.75pt
53pt 14.6pt 48pt 28pt 45pt 30.67pt 36.5pt 51pt 23pt 76.3pt
The wines of France and California may be the best
known, but they are not the only fine wines. Spanish
wines are often underestimated, and quite old ones may
be available at reasonable prices. For Spanish wines
the vintage is not so critical, but the climate of the
Bordeaux region varies greatly from year to year. Some
vintages are not as good as others,
so these years ought to be
s\kern -.1pt p\kern -.1pt e\kern -.1pt c\hfil ially
n\kern .1pt o\kern .1pt t\kern .1pt e\kern .1pt d\hfil:
1962, 1964, 1966. 1958, 1959, 1960, 1961, 1964,
1966 are also good California vintages.
Good luck finding them!
\label{fig:parshape}
\end{minipage}}

\section{Summary}
\tex's main blocks are paragraphs. It treats all words as tokens and applies an algorithm of using glue and boxes to typeset it. Commands are available  to modify the display of all elements of paragraphs. We have not discussed {\em boxes} and {\em glue} yet. This is still yet to come once we delve a bit more in the programming side of things.

   
\section{Linenumbers}

In many cases especially those that involve scholarly critical editions we may want to number paragraphs. This seemingly easy task, is extremely difficult to achieve with \tex and \latex, unless you use a pre-existing package. In this case we can use the \docpkg{lineno} package, that can produce numbered paragraphs as shown below.\TODO{clashes with fp}



\section{Dangerous bends}
\gdef\tstory{There are cries, sobs, confusion among the people, and
     at that moment the cardinal himself, the Grand Inquisitor, passes by the
     cathedral. He is an old man, almost ninety, tall and erect, with a
     withered face and sunken eyes, in which there is still a gleam of light.
     He is not dressed in his brilliant cardinal's robes, as he was the day
     before, when he was burning the enemies of the Roman Church~\char144
     \kern2em\hfill---Fyodor Dostoyevsky}
     % The example for several primitives uses \tstory.

\begingroup
     \hsize=2.5in
     \setbox0=\vbox{\adjdemerits=0
     \doublehyphendemerits=100000
     \finalhyphendemerits=900000
     \tstory\par}
     \setbox1=\vbox{\adjdemerits=1000000
     \doublehyphendemerits=100000
     \finalhyphendemerits=900000
     \tstory\par}
     \hbox{\box0\kern0.25in\box1}

\endgroup

The command \cs{doublehyphendemerits} is used by \tex when is breaking a paragraph into lines, this value is added to the demerits calculated for a line if the line and the previous one end with discretionary breaks [98].

\bgroup
\hsize=2.5in%                
   
  \setbox0=\vbox{\tstory\par}%  holds the definition of \tstory

     \setbox1=\vbox{\adjdemerits=0
        \doublehyphendemerits=200000
     \tstory\par}

     \hbox{\box0\kern0.25in\box1}






\egroup
\bigskip
\begingroup
\hsize=2.5in
     \setbox0=\vbox{\adjdemerits=0
     \doublehyphendemerits=100000
     \tstory\par}% 
     \setbox1=\vbox{\adjdemerits=0
     \doublehyphendemerits=100000
     \finalhyphendemerits=900000
     \tstory\par}
     \hbox{\box0\kern0.25in\box1}
\endgroup
\bigskip

You can adjust the \cs{looseness} of a paragraph by adjusting the looseness value. The command |\looseness=l| tells \tex to try and make the current paragraph l lines longer (if loosenessl is $> 0$) or l lines shorter (if $ l < 0$) while maintaining the general tolerances used to typeset the ms. IF $ l is > 0$, a tie `\~' is often placed between the last two words in the paragraph to prevent a short last line [103-104]. The parameter is reset to zero at the end of every paragraph or when internal vertical mode is started [349].

{
\hsize=4.5in
     \tstory\par
     \vskip6pt
     {\looseness=-1
     \tstory\par}
}


\section{French spacing}

Before we conclude this chapter it remains to discuss, spacing after punctuation. The \frenchspacing declaration tells \latex not to insert extra space at the end of sentences. This style is common in non-English languages, such as French and hence its name.

\index{frenchspacing}\index{junkspacing}\index{nonfrenchspacing}
Each character in a font has a space factor \index{ space factor} code that is an integer between 0 and 32767. The code is used to adjust the space factor in a horizontal list. The uppercase letters A-Z have space factor code 999. Most other characters have code 1000 [76]. However, Plain TeX makes `)', `'', and `]' have space factor code 0. Also, the \cs{frenchspacing} and \cs{nonfrenchspacing} modes in Plain \tex work by changing the \cs{sfcode} for: `.', `?', `!', `:', `;', and `,' [351].

\begingroup
\def\junkspacing{\sfcode`\.32767 \sfcode`\?6000 \sfcode`\!3000
    \sfcode`\:2500 \sfcode`\;2000 \sfcode`\,1500}

\def\nonfrenchspacing{\sfcode`\.3000 \sfcode`\?3000 \sfcode`\!3000
   \sfcode`\:2000 \sfcode`\;1500 \sfcode`\,1250}

\def\frenchspacing{\sfcode`\.1000 \sfcode`\?1000 \sfcode`\!1000
   \sfcode`\:1000 \sfcode`\;1000 \sfcode`\,1000}

 % Quotes are intentionally omitted in the following story:

 \let\tstory\frogking

\bigskip

\noindent\rule{\linewidth}{0.4pt}

\medskip
\narrower
{\hfill\hfill\small \texttt{\textbackslash junkspacing}}
\medskip


 \junkspacing Once upon a time, there was a naughty squirrel. Where shall I eat
     today? it asked. There were three options: a distant oak tree; a nearby 
    walnut tree; and a freshly-stocked bird feeder. I think\par

\bigskip

\smallskip
{\hfill\hfill\small \texttt{\textbackslash nonfrenchspacing}}

\medskip
\raggedright
     \nonfrenchspacing Once upon a time, there was a naughty squirrel. Where shall I eat
     today? it asked. There were three options: a distant oak tree; a nearby 
    walnut tree; and a freshly-stocked bird feeder. I think\par \par

\bigskip

\smallskip
{\hfill\hfill\noindent\small \texttt{\textbackslash frenchspacing}}

\medskip
     \frenchspacing Once upon a time, there was a naughty squirrel. Where shall I eat
     today? it asked. There were three options: a distant oak tree; a nearby 
    walnut tree; and a freshly-stocked bird feeder. I think\par \par

\medskip
\noindent\rule{\linewidth}{0.4pt}
\endgroup


One other item we need to examine is what happens at the end of an abbreviation or if you end a sentence with a capital letter? Probably not much, especially if you are using the microtype package. Example~\ref{bs} demonstrates its use. The last example is from Barbara Beeton's example at |TX.SX|.\footnote{\url{http://tex.stackexchange.com/questions/22561/what-is-the-proper-use-of-i-e-backslash-at}.}

\begin{texexample}{spacing after abbreviations}{bsat}
\makeatother
The name of the corporation is A.B.C.What happens to spacing after the last stop? 

The name of the corporation is A.B.C. What happens to spacing after the last stop?

The name of the corporation is A.B.C\@. What happens to spacing after the last stop?

Languages like JS, HTML, etc.\ were not used by King Henry III\@. This is Barbara Beeton's example.
\end{texexample}

\section{Code tables}

Table~\ref{tab:coded} 
shows the `numeric' coded commands and the corresponding
glyphs. 

Table~\ref{tab:alpha} 
shows the alphabetic coding (in both single
character and command form) and the corresponding glyphs together with their
transliterations. Note that not every glyph has a transliteration.

\begin{comment}

\begin{center}
  \Large\cartouche{\pmglyph{K:l-i-o-p-a-d:r-a}}
\end{center}
\end{comment}

\chapter{Hyphenation}
\label{ch:hyphenation}

\epigraph{Although it does not find all possible division points in a word, it very rarely makes an error. Tests on a pocket dictionary word list indicate that about 40\% of the allowable hyphen points are found, with 1\% error (relative to the total number of hyphen points). The algorithm requires 4K 36-bit words of code, including the exception dictionary.}{--Franklin Mark Liang \citep{liang83}}

\label{ch:hyphenation} \index{hyphenation}
It is said that George Bernard Shaw would examine galley proofs of his work and recast sentences, or even whole pages, in order to avoid unsightly word breaks, excessive white space caused by justification, and other typographical difficulties. Of course, he was published at a time when typesetters were sent to the block for committing the abominations above.\cite{Major1991} 

\section{A war over hyphens}

\index{Hyphen War}
In 1989-1990 there was a conflict called The Hyphen War (in Czech, Pomlčkov\'a v\' alka; in Slovak, Pomlkov¡ vojna €”literally `'Dash War'') was the tongue-in-cheek name given to the conflict over what to call Czechoslovakia after the fall of the Communist government. The Communist system in Czechoslovakia fell in November 1989. But in 1990, the official name of the country was still the "Czechoslovak Socialist Republic" (in Czech and in Slovak Československá socialistick¡ republika, or ČSSR). President clav Havel proposed merely dropping the word "Socialist" from the name, but Slovak politicians wanted a second change. They demanded that the country's name be spelled with a hyphen (e.g. "Republic of Czecho-Slovakia" or "Federation of Czecho-Slovakia"), as it was spelled from Czechoslovak independence in 1918 until 1920, and again in 1938 and 1939. President Havel then changed his proposal to "Republic of Czecho-Slovakia" \footnote{See discussion at \url{http://en.wikipedia.org/wiki/Hyphen_War}}. 

The use of the hyphen in English compound nouns and verbs has, in general, been steadily declining. Compounds that might once have been hyphenated are increasingly left with spaces or are combined into one word. In 2007, the sixth edition of the \textit{Shorter Oxford English Dictionary} removed the hyphens from 16,000 entries, such as fig-leaf (now fig leaf), pot-belly (now pot belly) and pigeon-hole (now pigeonhole). The advent of the Internet and the increasing prevalence of computer technology have given rise to a subset of common nouns that may in the past have been hyphenated (e.g. \textit{toolbar}, \textit{hyperlink}, \textit{pastebin}).

Despite decreased usage, hyphenation remains the norm in certain compound modifier constructions and, amongst some authors, with certain prefixes. Hyphenation is also routinely used to avoid unsightly spacing in justified texts (for example, in newspaper columns). 

\section{Hyphenation of common words}
With the advent of computers hyphenating justified text automatically became a challenge.


In \TeX78 a rule-driven algorithm for English   \index{TEX78}
was built-in by Liang and Knuth. Their algorithm
found 40\% of the allowable hyphens, with
about 1\% error. Although authors
claimed that these results are quite good, Liang
continued working on the generalization of the idea
of rules expressed by hyphenating and inhibiting
patterns. In his dissertation \citep{liang83} he describes
a method, which is used in TEX82, based
on the generalization of the prefix, suffix and the
vowel-consonant-consonant-vowel rules. He wrote
(in \texttt{WEB}) the program \texttt{PATGEN} (Liang and Breitenlohner,
1991) to automate the process of pattern \index{hyphenation!patterns}
generation from a set of already hyphenated words.

He started with the 1966 edition of Webster's Pocket\cite{websters1961}
Dictionary that included hyphenated words and inflections
 (about 50 000 entries in total). In the early
stages, testing the algorithm on a 115 000 word dictionary
from the publisher, 10 000 errors in words
not occurring in the pocket dictionary were found.

Most of these were specialized technical terms that
we decided not to worry about, but a few hundred
were embarrasing enough that we decided to add
them to the word list. (Liang, 1983, p. 30). He
reports the following figures: 89,3\% permissible hyphens
found in the input word-list with 4447 patterns
with 14 exceptions.

Liang's method is described by Knuth (1986b,
Appendix H) and was later adopted in many programs
such as troff (Emerson and Paulsell, 1987)
and Lout, and in localizations of today's WYSIWYG
DTP systems such as QuarkXPress, Ventura,
etc. Although specialized dictionaries such
as Allen's (1990) by Oxford University Press separate
possible word-division points into at least two
categories (preferred and less recommended), we
have not seen any program that incorporates the
possibility of taking into account these classes of
hyphenation points so far.



\section{Liang's hyphenation algorithm}

Franklin M. Liang's hyphenation algorithm is based
on what is termed \emph{competing hyphenation patterns}.\index{hyphenation>competing hyphenation patterns} 

Liang experimented with hyphenation and came with the idea of \textit{hyphenation patterns}.
These are simply strings of letters that, when they match a word, tell us how to hyphenate at some point in the pattern.
For example, the pattern |tion| tell us that we can hyphenate between the |t|. Or when the pattern 'cc' appears in a word, we can ususally hyphenate between the c's. Liang gives some good hyphenating patterns\cite{Liang1981}.

\begin{teX}
.in-d  .in-t  .un-d  b-s -cia
\end{teX}

\noindent (The character '.' matches the beginning or end of a word).


The patterns can
give excellent compression for a hyphenation dictionary,
and using these patterns the fast hyphenator algorithm
can also correctly hyphenate unknown (non-dictionary)
words most of the time. Liang's work
covers also the machine learning of the hyphenation
patterns and exceptions by |PatGen|  pattern generator.
The hyphenation patterns can allow and prohibit
hyphenation breaks on multiple levels. Figure \ref{fig:patterns} shows
the pattern matching on the word |algorithm|. 

\begin{figure}

\mbox{\fbox{\strut.}\fbox{\strut a}\fbox{\strut  l}\fbox{\strut g}\fbox{\strut o}\fbox{\strut  r}\fbox{ \strut i}\fbox{\strut t}\fbox{\strut h}\fbox{\strut m}\fbox{\strut .}}
   4l1g4
     l g o3
    1g o
            2i t h
               4h1m

\mbox{\fbox{\strut 4}\fbox{\strut 1}\fbox{\strut 4}\fbox{\strut 3}\fbox{\strut 2}\fbox{\strut 0}\fbox{\strut 4}\fbox{\strut 1}}

\mbox{\strut\fbox{a}\fbox{l}\fbox{-}\fbox{g}\fbox{o}\fbox{-}\fbox{r}\fbox{i}\fbox{t}\fbox{h}\fbox{-}\fbox{m}}

\caption{\tex hyphenation of the word algorithm.}
\label{fig:patterns}
\end{figure}


The patterns consist of strings of letters and digits. Digits
indicates a `hyphenation value'\index{hyphenation!hyphenation value} for some intercharacter position.  For
example, the pattern \texttt{\.{3t2ion}} specifies that if the string \texttt{\.{tion}}
occurs in a word, we should assign a hyphenation value of 3 to the
position immediately before the \.{t}, and a value of 2 to the position
between the \.{t} and the \.{i}.

To hyphenate a word, we find all patterns that match within the word and
determine the hyphenation values for each intercharacter position.  If
more than one pattern applies to a given position, we take the maximum of
the values specified (i.e., the higher value takes priority).  If the
resulting hyphenation value is odd, this position is a feasible
breakpoint; if the value is even or if no value has been specified, we are
not allowed to break at this position.

In order to find quickly the patterns that match in a given word and to
compute the associated hyphenation values, the patterns generated by this
program are compiled by \.{INITEX} into a compact version of a finite
state machine.  For further details, see the \TeX 82 source.


The
\tex English hyphenation patterns 4l1g4, lgo3, 1go,
2ith and 4h1m match this word and determine its
hyphenation. Only odd numbers mean hyphenation
breaks. If two (or more) patterns have numbers in
the same place, the highest number wins. The \texttt{algo-
rith-m} hyphenation is bad, but the last one-letter
hyphenation is suppressed by \tex, so we end up with
the correct \texttt{al-go-rithm}.(See also the Section on |\hyphenminleft| and |hyphenminright| for more details
how this parameters are adjusted in \tex.

One of the most notable features of this pattern based
hyphenation is the human-readable format of
the knowledge database, in contrast to an equivalent
finite state machine or a similarly good artificial neural
network. This format is good for manual checking and
corrections.



\section{How to tinker hyphenation}

TeX will not insert a hyphen before the number of letters specified by \docAuxCmd{lefthyphenmin},
nor after the number of letters specified by \docAuxCmd{righthyphenmin}. For U.S. English,
|\lefthyphenmin=2| and |\righthyphenmin=3|. 

\index{hyphenation>penalty>\textbackslash hyphenpenalty}
\begin{docCommand}{hyphenpenalty}{}
The best way to examine the effects of the various hyphenation parameters
is to put the words in a narrow minipage (much quicker and visual rather than examining TeX's output. The first parameter setting command is we will examine is \cs{hyphenpenalty}.
\end{docCommand}

\begin{texexample}{}{}
\fbox{
\begin{minipage}{1.3cm}
\hyphenpenalty=-2000
photographer and hyphenation. \par
potographer and hyphenation. \par
unhelpful\par
\end{minipage}}
\end{texexample}



The \cs{hyphenpenalty} can be used to adjust the hyphenation of paragraphs. 
This example typesets a paragraph with three different values of \cs{hyphenpenalty}. 

There is no difference in using the Plain TeX value of 50 and using 0. 
Increasing |\hyphenpenalty| to 200 eliminates all hyphenated words in the paragraph. 
Decreasing |\hyphenpenalty|  to -2000 results in two addition hyphenated words.

\begin{texexample}{}{}
\hsize2.5in
\long\def\testhyphenpenalty#1%
    {\par\leavevmode
       \hyphenpenalty=#1 %
        \tstory% 
        \par
        \vskip2\baselineskip
     }

\testhyphenpenalty{50} 
\testhyphenpenalty{200}
\testhyphenpenalty{-2000}
\end{texexample}





\section*{\textbackslash lefthyphenmin}
\newthought{This parameter holds} the minimum number of characters that must appear at the beginning of a hyphenated word (i.e., before the `-'). In particular, \tex will not hyphenate words with fewer than the sum of \cs{lefthyphenmin} and \cs{righthyphenmin} characters [454]. The |whatsit\mkindex{whatsit`5`language}|  made by a change to \cs{language} includes the current value of |\lefthyphenmin|.

Changes made to |\lefthyphenmin| are \textit{local} to the group containing the change.

\begin{teX}
\def\tstoryA{There are cries, sobs, confusion among the people, and at
   that moment the cardinal himself, the Grand Inquisitor, passes by the
   cathedral. He is an old man \ldots\par}

   {\language255\hyphenation{m-oment}}
   \count0=\lefthyphenmin
   \setbox0=\vbox{\hsize=4.3in\language255 \tstoryA}
   \setbox1=\vbox{\hsize=4.3in\language255\lefthyphenmin=1\righthyphenmin=2 \tstoryA}
   \medskip\par
   \hbox to \hsize{\box0}
   \medskip
   \hbox to \hsize{\box1}
\end{teX}

This will produce:

\noindent{\color{orange}\rule{5cm}{1pt}\hfill\hfill\par}

\begingroup
\overfullrule=0.5pt
\def\tstoryA{There are cries, sobs, confusion among the people, and at
   that moment the cardinal himself, the Grand Inquisitor, passes by the
   cathedral.\par}


   {\language255\hyphenation{m-oment}}
   \count0=\lefthyphenmin
   \setbox0=\vbox{\hsize=4.3in\language255 \tstoryA}
   \setbox1=\vbox{\hsize=4.3in\language255\lefthyphenmin=1\righthyphenmin=2 \tstoryA}
   \medskip\par
   \hbox to \hsize{\box0}
   \medskip
   \hbox to \hsize{\box1}
\medskip
\endgroup


{\hfill\hfill\color{orange}\rule{5cm}{1pt}\par}
\hfill\hfill{\raise6pt\hbox{\small}


\section{Hyphenation exceptions}

The command \cs{-}  inserts a discretionary hyphen into a word. This also becomes the only point where hyphenation is allowed in this word. This command is especially useful for words containing special characters (e.g., accented characters), because LaTeX does not automatically hyphenate words containing special characters. A list of hyphenation exceptions has been kept and updated
by Barbara Beeton for many years.\footcite{beeton2015} 

\begin{teX}
\begin{minipage}{2in}
I think this is: su\-per\-cal\-%
i\-frag\-i\-lis\-tic\-ex\-pi\-%
al\-i\-do\-cious
\end{minipage}
\end{teX}
\bigskip



\noindent{\color{orange}\rule{5cm}{1pt}\hfill\hfill\par}
\begin{center}
\par
\begin{minipage}{2in}
I think this is: su\-per\-cal\-%
i\-frag\-i\-lis\-tic\-ex\-pi\-%
al\-i\-do\-cious
\par
\end{minipage}
\end{center}
{\hfill\hfill\color{orange}\rule{5cm}{1pt}\par}
\hfill\hfill{\raise6pt\hbox{\small}
\bigskip

This can be quite cumbersome if one has many words that contain a dash like electromagnetic-endioscopy. One alternative to this is using the \cs{hyp} command of the \docpkg{hyphenat} package. This command typesets a hyphen and allows full automatic hyphenation of the other words forming the compound word. One would thus write

\begin{teX}
electromagnetic\hyp{}endioscopy
\end{teX}


Several words can be kept together on one line with the command

\begin{teX}
\mbox{text}
\end{teX}

It causes its argument to be kept together under all circumstances. 
For example when we are typesetting phone numbers,

\begin{teX}
My phone number will change soon to be |\mbox{0116 291 2319}|.
\end{teX}

\noindent |\fbox| is similar to |\mbox|, but in addition there will be a visible box drawn around the content.

To avoid hyphenation altogether, the penalty for hyphenation can be set to an extreme value:

\begin{teX}
\hyphenpenalty=100000
\end{teX}



The following sample texts from \textit{The frog king}\cite{frogking} have been traditionally used for testing hyphenation algorithms as they
include both short as well as long words. We have varied the |hyphenpenalty| as shown. It is a tribute to \tex that even with no hyphenation present (the last column) the text still looks very presentable with virtually no visible large spaces.

\long\def\sampletext{%
\hskip1em In olden times when wishing
still helped one, there lived a
king whose daughters were all
beautiful, but the youngest was so
beautiful that the sun\hl{ itself},
which has seen so much, was
astonished whenever it shone in
her face. Close by the king's
castle lay a great dark forest,
and under an old lime-tree in the
forest was a well, and when
the day was very warm, the
king's child went out into the 
forest and sat down by the side
of the cool fountain, and when she was bored she
took a golden ball, and threw it up on a high and caught it, and this
ball was her favorite plaything. \par}

\overfullrule=0.1pt

\begin{minipage}{1.9in}
 \hyphenpenalty=0\sampletext
\end{minipage}\hspace{.8cm}
\begin{minipage}{1.9in}
 \hyphenpenalty=100\sampletext
\end{minipage}\hspace{.8cm}
\begin{minipage}{1.9in}
 \hyphenpenalty=100000 \sampletext
\end{minipage}


\def\samplerivers{%
\hskip1em Repeated repeated repeated repeated
repeated repeated repeated repeated
repeated repeated repeated repeated
repeated repeated repeated repeated
repeated repeated repeated repeated
repeated repeated repeated repeated
repeated repeated repeated repeated
repeated repeated repeated repeated
repeated repeated repeated repeated
repeated repeated repeated repeated
repeated repeated repeated repeated
repeated repeated repeated repeated
repeated.}

\overfullrule=0.1pt

\begin{minipage}{1.9in}
 \looseness=-1 \hyphenpenalty=0\samplerivers
\end{minipage}\hspace{.8cm}
\begin{minipage}{1.9in}
  \hyphenpenalty=100\samplerivers
\end{minipage}\hspace{.8cm}
\begin{minipage}{1.9in}
 \hyphenpenalty=100000 \samplerivers
\end{minipage}




%%% Code from GIT posted by Wilson

\frenchspacing
\fussy

\makeatletter

\newbox\trialbox
\newbox\linebox
\newcount\maxbad
\newcount\linebad
\newcount\bestbad
\newcount\worstbad
\newcount\overfulls
\newcount\currenthbadness


\def\trypar#1\par{%
  \showtrybox{\linewidth}{#1\par}%
}

\newcommand\showtrybox[2]{%
  \currenthbadness=\hbadness
  \maxbad=0\relax
  \setbox\trialbox=\vbox{%
    \hsize#1\relax#2%
    \hbadness=10000000\relax
    \eatlines
  }%
  \hbadness=10000000\relax
  \setbox\trialbox=\vbox{%
    \hsize#1\relax#2%
    \printlines
  }%
  \noindent\usebox\trialbox\par
  \hbadness=\currenthbadness
}

\newcommand\trybox[2]{%
  \currenthbadness=\hbadness
  \maxbad=0\relax
  \setbox\trialbox=\vbox{%
    \hsize#1\relax#2\par
    \hbadness=10000000\relax
    \eatlines
  }%
  \hbadness=\currenthbadness
}

\def\eatlines{%
  \begingroup
  \setbox\linebox=\lastbox
  \setbox0=\hbox to \hsize{\unhcopy\linebox\hss}%
  \linebad=\the\badness\relax
  \ifnum\linebad>\maxbad\relax \global\maxbad=\linebad\relax \fi
  \ifvoid\linebox\else
    \unskip\unpenalty\eatlines
  \fi
  \endgroup
}

\def\printlines{%
  \begingroup
  \setbox\linebox=\lastbox
  \setbox0=\hbox to \hsize{\unhcopy\linebox}%
  \linebad=\the\badness\relax
  \ifvoid\linebox\else
    \unskip\unpenalty\printlines
    \ifhmode\newline\fi\noindent\box\linebox\showbadness
  \fi
  \endgroup
}

\def\showbadness{%
  \makebox[0pt][l]{%
    \ifnum\currenthbadness<\linebad\relax
      \ifnum\linebad=1000000\relax\expandafter\@gobble\fi
      {\quad\color{red}\rule{\overfullrule}{\overfullrule}~{\footnotesize\sffamily(\the\linebad)}}%
    \fi
  }%
}

\makeatother



\begin{minipage}{5cm}

\trypar
There is no just ground, therefore, for the charge brought against me by~
certain ignoramuses---that I have never written a moral tale, or, in more
precise words, a tale with a moral. They are not the critics predestined
to bring me out, and \emph{develop} my morals:---that is the secret. By and by
the ``North American Quarterly Humdrum'' will make them ashamed of their
stupidity. In the meantime, by way of staying execution---by way of
mitigating the accusations against me---I offer the sad history appended,---
a history about whose obvious moral there can be no question whatever,
since he who runs may read it in the large capitals which form the title
of the tale. I should have credit for this arrangement---a far wiser one
than that of La Fontaine and others, who reserve the impression to be
conveyed until the last moment, and thus sneak it in at the fag end of
their fables.\par
\end{minipage}

\the\hbadness


\hbadness=2000 
\begin{minipage}{5cm}
\trypar \hyphenpenalty=-50
There is no just ground, therefore, for the charge brought against me by~
certain ignoramuses---that I have never written a moral tale, or, in more
precise words, a tale with a moral. They are not the critics predestined
to bring me out, and \emph{develop} my morals:---that is the secret. By and by
the ``North American Quarterly Humdrum'' will make them ashamed of their
stupidity. In the meantime, by way of staying execution---by way of
mitigating the accusations against me---I offer the sad history appended,---
a history about whose obvious moral there can be no question whatever,
since he who runs may read it in the large capitals which form the title
of the tale. I should have credit for this arrangement---a far wiser one
than that of La Fontaine and others, who reserve the impression to be
conveyed until the last moment, and thus sneak it in at the fag end of
their fables.\par
\end{minipage}
\bigskip
\clearpage

\section{Testing badness}
The following text displays the badness as calculated by the linebreaking algorithm.
\begin{figure*}[htb]
\fussy
\hbadness=-1 
\begin{minipage}[t]{4.5cm}
\mbox{}
\trypar\hyphenpenalty=-500\looseness=1
In olden times when wishing
still helped one, there lived a
king whose daughters were all
beautiful, but the youngest was so
beautiful that the sun itself,
which has seen so much, was
astonished whenever it shone in
her face. Close by the king's
castle lay a great dark forest,
and under an old lime-tree in the
forest was a well, and when
the day was very warm, the
king's child went out into the 
forest and sat down by the side
of the cool fountain, and when she was bored she
took a golden ball, and threw it up on a high and caught it, and this
ball was her favorite plaything. \par
\end{minipage}
\hspace{2cm}
\begin{minipage}[t]{4.5cm}
\mbox{}
\trypar\hyphenpenalty=10000
In olden times when wishing
still helped one, there lived a
king whose daughters were all
beautiful, but the youngest was so
beautiful that the sun itself,
which has seen so much, was
astonished whenever it shone in
her face. Close by the king's
castle lay a great dark forest,
and under an old lime-tree in the
forest was a well, and when
the day was very warm, the
king's child went out into the 
forest and sat down by the side
of the cool fountain, and when she was bored she
took a golden ball, and threw it up on a high and caught it, and this
ball was her favorite plaything. \par
\end{minipage}
\caption{Comparison of two sample texts. The left has a hyphenpenalty=-500 and the right has a hyphenpenenalty=10000. Both look acceptable. The text is set at 4.5cm textwidth}
\end{figure*}

\lorem

\lorem

\lorem



\chapter{The line breaking problem}

\epigraph{Psychologically bad breaks are not easy to define; we just know they are bad. When
the eye journeys from the end of one line to the beginning of another, in the presence
of a bad break, the second word often seems like an anticlimax, or isolated from
its context. Imagine turning the page between the words ‘Chapter’ and ‘8’ in some
sentence; you might well think that the compositor of the book you are reading should
not have broken the text at such an illogical place}{Donald Knuth}

In the days of typewriters once the end of line was reached a bell rung to tell the author that the end of the line was reached. The author then had the choice to press the carriage return to start a new line or to extend the line by a couple of characters.

Consider the following short text, 

\begin{scriptexample}{}{}
In olden times when wishing
\end{scriptexample}

 \newlength\myl
 \settowidth\myl{In olden times when wishing}
The natural width of the above string of text is \the\myl. Consider again the same string but this time all white space removed.
\begin{scriptexample}{}{}
Inoldentimeswhenwishing
\end{scriptexample}
\settowidth\myl{Inoldentimeswhenwishing}
\the\myl 


If the paragraph was to be constrained at a width $<\the\myl$ one could distribute less white space between the words of the line. If the width of the paragraph was less than limit there would be no choice but to move a word to the line below it. TeX optimizes the justification of a full paragraph rather than optimize the looks of a single line in order to produce high quality typesetting.


Breaking a paragraph into lines consists of selecting the break points in a paragraph. To determine how much space a line takes, each character (or rather glyph) in the paragraph is modelled as a box with a specific width, height and elevation from the baseline. These properties are determined by the font being used and the font size.


Sometimes a broken line may not completely fit into the available space between the left and the right margin. To make it fit, space may be distributed among the spaces in the line, or some whitespace may be taken out, if the text exceeds the 
line width. We assume that there is a single target width for the complete paragraph and the paragraph is adjusted. 
An indent that applies to the line of the paragraph can be modeled using a fixed width unbreakable space (possibly with a negative value).

\subsection{Formalizing the problem}

Now that we have a good idea of the problem we can formalize it. A paragraph $p$ is a sequence of $n>0$ characters $c_i$,  $i\leq i_i \leq n$.
A \textit{breakpoint candidate}\index{paragraph!breakpoint candidate} in $p$ is an index of a character in $p$ for which it is allowed to break a line. Typical break point candidates are space and hyphen characters and hyphenation points in words.

The \textit{line breaking problem} for a paragraph $p$ for a desired \textit{text width} and \textit{indent} is finding a set of break points of $p$ that look `nice'. We will consider a fixed \textit{text width}, $t_w$, with the exception of the first line which can possibly have a negative indent $in$. What looks nice is obviously a subjective term. Another typography factor is the grayness of the paragraph. 

\section{Greedy algorithm}

An easy way to break a paragraph into lines is to use the greedy algorithm. This algorithm basically puts as many 
words on the line as it can contain, repeating the process for each line until there are no more words in the paragraph. The greedy algorithm is a line-by-line process. During the execution of the algorithm each line is handled independently. (\citep{elyaakoubi})



\section{Knuth Pratt algorithm}
\tex's line breaking algorithm optimizes line breaks on the level of a paragraph. \tex determines character widths taking kerning and ligatures in account and uses a three phased process. 

\begin{description}
\item[Phase 1] In this phase no hyphenation takes place. Only white spaces are considered  for line breaking. For a paragraph broken in $k$ lines, for each line $j=k\ldots k$  a \textit{badness} $b_i$ is calculated using:

$$b_j=100\left(\frac{nlw_{sj}-t_w}{nsp_{sj}\dot f}\right)^{3}$$

where 

$nlw{sj} = \text{the natural width}$

Why this penaly function is a power of three was never explained properly, obviously is to use a penalty function which is non-linear. 

Depending on the value of $b_j$, the line is classified as \textit{tight}, \textit{loose} or  \textit{very loose}. If none of the line's badness exceed a \textit{pretolerance}  limit the paragraph is accepted and no further processing takes place.

\item[Phase 2] In phase 2 the hyphenation points are calculated for the paragraph and a new set of breakpoints  determined. If all of the line's badness is below a a tolerance level  (which differs from the pretolerance level) a \textit{demerit} for the paragraph is calculated as a combination of line badness, roughly as

\begin{equation}
\sum_{j=1}^{n}  \left(c+b_j \right)^{2} + p \cdot \vert p\vert 
\end{equation}

where $c$ is a constant line penalty and $p$ a penalty term. a tight or loose line increases the penalty $p$. If two consecutive lines are hyphenated or if the last line is hyphenated, the penalty is increased. A paragraph with a demerit below a threshold is accepted.

\item[Phase 3] In phase 3 the steps in phase 2 are repeated but now with an \textit{emergency stretch} that allows lines to shrink or expand more. If this last phase does not succeed, \tex outputs overly long lines, due to the thresholds in the algorithm.
\end{description}

The description above is very broad, as Knuth introduced more variables that control for example, what makes a good hyphenation point and what it doesn’t.

There are also rules as to spacing after punctuation, how kerning and similar aspects are considered.  For mathematical typesetting a different algorithm is used. 


\section{Patents and other research}

A very similar algorithm to that provided by the Knuth-Plaas algorithm was filed as a patent
by Adobe \cite{adobepatent}. Another more recent attempt is by Holkner\cite{Holkner2006}. Holkner attempts to optimize over multiple objectives but as he writes:

\begin{quote}
Also surprising is the change in performance as more objective functions are added. When
$\sigma$Looseness is added to $\mu$Looseness and $\Sigma\text{hyphen}$ performance improves --- by more than 10 times
for the wider two columns. On the other hand, when Looseness is added to Looseness and $\Sigma\text{River}$
performance degrades so badly that some of the tests had to be stopped when they took more than six
hours to complete.

\end{quote}

What is interesting though is the author's conclusions towards the end of his thesis. Having defined some new metrics, which we will discuss soon, he writes:

\begin{quote}
It has become apparent through our research that while \tex returns the optimal paragraph according
to its weighted sum measurement, this measurement is not optimal with respect to our metrics.
Having shown that our metrics represent real-world typographic qualities, we can say that \tex can be
improved upon, and that our method does this.

\end{quote}

You can get more info at \url{http://yallara.cs.rmit.edu.au/~aholkner/presentation2.pdf}


\section{Typesetting with varying letter widths}
A totally different approach to hyphenation and line breaking was the work of early typographers including
Gutenburg, where varying width of glyphs were used to fit lines into exact widths. Such an approach was
also used by the \so{hz} software\cite{hz}\index{line breaking!hz-program}\index{hz-program}

An attempt to apply the techniques used by Gutenberg and the |hz| program to 
was done by Miroslava Mis\'akov\'a \cite{Miroslava1998}. The author demonstrated the potential of a simple
method that allows typesetting with varying letter widths implemented by font expansion
to attain better interword spacing of composition. The results were very interesting,
even though the method was not flexible enough for practical use. (this is different than \so{letterspacing}, which
is just a typographic way favoured by some for emphasizing text without affecting the grayness of the page.

These and other methods are discussed under the Chapter for microtypography. 
\section{The strangeness of TeX}

{
 \everypar{\def\indent{1}}
\indent 3 i s a prime number.
and

 \everypar{\def\vrule{1}}
\vrule 3 is  a prime number.
}




\begin{figure*}
^^A\includegraphics[width=\linewidth]{../images/bible}
\caption{Leaf from the G\"ottingen Gutenburg Bible}
\url{http://www.gutenbergdigital.de/gudi/eframes/bibelsei/frmlms/frms.htm}
\label{fig:bible}
\end{figure*}

\section{The Gutenberg Bible}

The Gutenberg Bible (also known as the 42-line Bible, the Mazarin Bible or the B42) was the first major book printed with a movable type printing press, marking the start of the "Gutenberg Revolution" and the age of the printed book. Widely hailed for its high aesthetic and artistic qualities,\cite{Martin1996} the book has iconic status in the West. It is an edition of the Vulgate, printed by Johannes Gutenberg, in Mainz, Germany in the 1450s. Only twenty-one complete copies survive, and they are considered by many sources to be the most valuable books in the world, even though a completed copy has not been sold since 1978.

The 36-line Bible is also sometimes referred to as a Gutenberg Bible, but is possibly the work of another printer.

In appearance the Gutenberg Bible closely resembles the large manuscript Bibles that were being produced at the time. The Giant Bible of Mainz, probably produced in Mainz in 1452-3, has been suggested as the particular model Gutenberg used.[4] Around this time large Bibles, designed to be read from a lectern, were returning to popularity for the first time since the twelfth century. In the intervening period, small hand-held Bibles had been usual.[5] The text of the Gutenberg Bible is traditional, falling within the Paris Vulgate group of texts.[6] Manuscript Bibles all had texts that differed slightly, and the copy used by Gutenberg as the exemplar for his Bible has not been discovered.[7]

The Bible was not Gutenberg's first work.\cite{Man2002} Preparation of it probably began soon after 1450, and the first finished copies were available in 1454 or 1455.[10] However, it is not known exactly how long the Bible took to print.

Gutenberg made three significant changes during the printing process.[11] The first sheets were rubricated by being passed twice through the printing press, using black and then red ink. This was soon abandoned, with spaces being left for rubrication to be added by hand.

Some time later, after more sheets had been printed, the number of lines per page was increased from 40 to 42, presumably to save paper. Therefore, pages 1 to 9 and pages 256 to 265, presumably the first ones printed, have 40 lines each. Page 10 has 41, and from there on the 42 lines appear. The increase in line number was achieved by decreasing the interline spacing, rather than increasing the printed area of the page.
Finally, the print run was increased, probably to 180 copies, necessitating resetting those pages which had already been printed. The new sheets were all reset to 42 lines per page. Consequently, there are two distinct settings in folios 1-32 and 129-158 of volume I and folios 1-16 and 162 of volume II.[12][13]

The most reliable information about the Bible's date comes from a letter. In March 1455, future Pope Pius II wrote that he had seen pages from the Gutenberg Bible, being displayed to promote the edition, in Frankfurt.[14].
It is believed that in total 180 copies of the Bible were produced, 135 on paper and 45 on vellum.[15]

The production process: 'Das Werk der B\"acher'

In a legal paper, written after completion of the Bible, Gutenberg refers to the process as 'Das Werk der Bücher': The work of the books. He had invented the printing press and was the first European to print with movable type[16]. But his greatest achievement was arguably demonstrating that the whole process of printing actually produced books.

Many book-lovers have commented on the high standards achieved in the production of the Gutenberg Bible, some describing it as one of the most beautiful volumes ever printed. The quality of both the ink and other materials and the printing itself have been noted. [1]

Paper and vellum

A single complete copy of the Gutenberg Bible has 1,272 pages; with 4 pages per folio-sheet, 318 sheets of paper are required per copy. The 45 copies printed on vellum required 11,130 sheets. The 135 copies on paper required 49,290 sheets of paper. The handmade paper used by Gutenberg was of fine quality and was imported from Italy. Each sheet contains a watermark, which may be seen when the paper is held up to the light, left by the papermold.

The paper size is 'double folio', with two pages printed on each side (making a total of four pages per sheet). After printing the paper is folded once to the size of a single page. Typically, five of these folded sheets (carrying 10 leaves, or 20 printed pages) were combined to a single physical section, called a quinternion, that could then be bound into a book. Some sections, however, carried as few as 4 leaves or as many as 12 leaves.[17] It is possible that some sections were printed in a larger number, especially those printed later in the publishing process, and sold unbound. The pages were not numbered. This whole technique of course was not new, since it was used already to make white-paper books to be written afterwards. New was the necessity to determine beforehand the right place and orientation of each page on the five sheets, so as to end up in the right reading sequence. Also new was the technique of getting the printed area correctly located on each page.

The folio size, 307 x 445 mm, has the ratio of 1.45. The printed area had the same ratio, and was shifted out of the middle to leave a 2:1 white margin, both horizontally and vertically. Historian John Man writes that the ratio was chosen because of being close to the golden ratio of 1.61.[9] To reach this ratio more closely the vertical size should be 338 mm, but there is no reason why Gutenberg would leave this non-trivial difference of 8 mm go by in such a detailed work in other aspects.

Ink

Gutenberg had to develop a new kind of ink, an oil-based one (as compared with the traditional water-based ink used in manuscripts), so that it would stick better to the metal types. His ink was based on carbon, with high metallic content, including copper, lead, and titanium.

Type style

The Gutenberg Bible is printed in the blackletter type styles that would become known as Textualis (Textura) and Schwabacher. The name texture refers to the texture of the printed page: straight vertical strokes combined with horizontal lines, giving the impression of a woven structure. Gutenberg already used the technique of justification, that is, creating a vertical, not indented, alignment at the left and right-hand sides of the column. To do this, he used various methods, including using characters of narrower widths, adding extra spaces around punctuation, and varying the widths of spaces around words.[19][20] On top of this, he subsequently let punctuation marks go beyond that vertical line, thereby using the massive black characters to make this justification stronger to the eye.


Rubrication, illumination and binding

Copies left the Gutenberg workshop unbound, without decoration, and for the most part without rubrication.
Initially the rubrics -- the headings before each book of the Bible -- were printed, but this experiment was quickly abandoned, and gaps were left for rubrication to be added by hand. A guide of the text to be added to each page, printed for use by rubricators, survives.\cite{Kapr1996}

The spacious margin allowed for illuminated decoration to be added by hand. The amount of decoration presumably depended on how much each buyer could or would pay for. Some copies were never decorated.[22] The place of decoration can be known or inferred for about 30 of the surviving copies. Perhaps 13 of these received their decoration in Mainz, but others were worked on as far away as London.[4] The vellum Bibles were more expensive and perhaps for this reason tend to be more highly decorated, although the vellum copy in the British Library is completely undecorated.[23] There has been speculation that the Master of the Playing Cards was partly responsible for the illumination of the Princeton copy, though all that can be said for certain is that the same model book was used for some of the illustrations in this copy and for some of the Master's playing cards.\cite{Buren1974}

Although many Gutenberg Bibles have been rebound over the years, 9 copies retain fifteenth-century bindings. Most of these copies were bound in either Mainz or Erfurt.[4] Most copies were divided into two volumes, the first volume ending with The Book of Psalms. Copies on vellum were heavier and for this reason were sometimes bound in three or four volumes.\cite{Martin1996}

\section{Life is not always simple}
 This document provides some samples of archaic fonts. They are
available from CTAN in the \texttt{fonts/archaic} directory. The fonts
form a set that display how the Latin alphabet and script evolved from the
initial Proto-Semitic script until Roman times.

    The fonts tend to consist of letters only --- punctuation had not 
been invented during this period except for a word-divider in some cases.
Some of the scripts had signs for numbers but in others either letters
doubled as numbers or the numbers were spelt out. The fonts are all
single-cased. Upper- and lower-case letters were again only invented after
the end of this period.

    Other fonts are also available for some scripts that were not on the
main alphabetic tree. The period covered by the scripts is from about 
3000~BC to the Middle Ages.

    For some of the scripts transliterations into the Latin alphabet can
be automatically generated by the accompanying LaTeX packages.

  The vowels (a, e, i, o, u) are: \textcypr{\Ca{} \Ce{} \Ci{} \Co{} \Cu}.


\section{Remaining Limitations}

Frank Mittelbach\footcite{mittelbach2013} outlined a number of difficulties, where \tex is limited. One of them is that \tex's hyphenation algorithm knows only two staes: a place in a word can or cannot act as a 
hyphenation point. He gives an example from German, where ``Non-nenkloster'' (abbey of nuns)
should preferably not be hyphenated as ``Nonnenklo-ster'' (as that leaves the word ``nun's toilet'' on the first line).

Liang’s pattern-based approach works very well for
languages for which the hyphenation rules can be
expressed as patterns of adjacent characters next to
hyphenation points. Such patterns may not be easy
to detect but once determined they will hyphenate
reasonably well. For the approach to be usable, the
necessary set of patterns should be be reasonably
small, as each discrepancy needs one or more exception
patterns with the result that the pattern set
would either become very large or the hyphenation
results would have many errors.

To improve the situation for the latter type of
languages one would need to implement and potentially
first develop other types of approaches. For
now Liang’s algorithm is hardwired in all engines,
though in theory LuaTEX offers possibilities of dropping
in some replacement.

























