\chapter{Iteration}
\precis{A discussion as to how to program simple for loops, in
TeX and LaTeX.}

\section{\TeX's simple \protect\texttt{loop}}

\newthought{Knuth in the TeXBook} provided a simple loop macro that can be used for iteration. It must be pointed out that there are no real looping structures in \tex other than pure recursion (including tail recursion). All looping mechanisms are build on top of these.

\begin{docCmd}{loop}{}
The |\loop...\repeat| construction is defined in Plain TeX and works like this:
You say `|\loop| $\alpha$ |\if|\dots $\beta$  |\repeat|', where $\alpha$ and $\beta$ are any sequences of
commands, and where |\if...| is any conditional test (without a matching |\fi|). 
\end{docCmd}

\tex
will first do $\alpha$; then if the condition is true, \tex will do $\beta$ and repeat the whole process
again starting with $\alpha$. If the condition ever turns out to be false, the loop will stop.

\emphasis{loop,repeat}
\begin{texexample}{Knuth's Loop}{}
\bgroup
\newcount\n
\n=0
\loop
   \advance\n by1
     \number\n,   
   \ifnum\n<30
\repeat
\egroup
\end{texexample}




The \cmd{\loop} macro that does all these wonderful things is actually quite simple.
It puts the code that's supposed to be repeated into a control sequence called
\cmd{\body}, and then another control sequence iterates until the condition is false:

\begin{teX}
\def\loop#1\repeat{\def\body{#1}\iterate}
\def\iterate{\body\let\next=\iterate\else\let\next=\relax\fi\next}
\end{teX}


\noindent\colorbox{gray!10}{\parbox{10cm}{\noindent
\newcount\n
\n=0
\loop
   \advance\n by1
     \texttt{\number\n, } 
   \ifnum\n<30
\repeat
}}
\bigskip

Just observe that the |\loop| arguments are delimited by |\repeat|. We could as well named it |\endloop| (a repeat at the end of a loop somehow sounds wrong!)

\emphasis{for,endfor}
\begin{teX}
\def\for#1\endfor{\def\body{#1}\iterate}
\def\iterate{\body\let\next=\iterate\else\let\next=\relax\fi\next}
\newcount\n
\n=0
% Example usage
\for
   \advance\n by1
     \texttt{\number\n, }  
   \ifnum\n<30
\endfor
\end{teX}  

\begin{texexample}{For}{}
\bgroup
\def\for#1\endfor{\def\body{#1}\iterate}
\def\iterates{\body\let\next=\iterates\else\let\next=\relax\fi\next}

\noindent\colorbox{gray!10}{\parbox{10cm}{\noindent
\newcount\n
\n=0
\for%
   \advance\n by1
     \texttt{\number\n, }%  
   \ifnum\n<30%
\endfor%
}}
\egroup
\end{texexample}

\subsection{Breaking out of a loop}

\index{Iteration!break}\index{\protect\textbackslash break}
Although the loop macros are fairly simple, breaking out of them or using conditionals needs some work.

\emphasis{let,iterate,relax}
\begin{teXXX}
\newcount\mycount
\mycount=0
\loop\ifnum\mycount<13
A 
\ifnum\mycount>5
    \let\iterate\relax
 \fi
 \advance\mycount by1\relax
\repeat
\end{teXXX}
\medskip
producing:
\medskip

\begin{texexample}{Iteration}{ex:loop}
\newcount\mycount
\mycount=0
\loop\ifnum\mycount<13
\the\mycount, 
\ifnum\mycount>5
    \let\iterate\relax
 \fi
 \advance\mycount by1\relax
\repeat
\end{texexample}

One can define a command \cmd{\break}, so as to have a better structure and make the code more readable:

\emphasis{def,break,let,iterate,relax}
\begin{teXXX}
\def\break{\let\iterate\relax}
\newcount\mycount
\mycount=0
\loop\ifnum\mycount<13
\the\mycount, 
\ifnum\mycount>5
    \break    
\fi
\advance\mycount by1\relax
\repeat
\end{teXXX}





\section{Iteration over comma delimited lists}
\index{iteration>comma delimited lists}

The comma delimited list is one of the most common programming datastructure. A list is simply defined using a macro:

\begin{verbatim}
\def\mylist{John,Mary,Mathew,George,Maria}
\end{verbatim}

Although, lists can be defined as shown in the |\mylist| macro, this is not very useful. In most cases the \textit{elements} of the list would be added programmatically. Such list are used for example by \latex to keep track of input files.

Unlike many other programming languages, lists can be delimited by any character or even macros. Many package authors use a semicolon. This a perfectly legal in \tex.

\begin{teX}
\mylist{John;Mary;Mathew;George;Maria}
\end{teX}
as well as this:

\begin{teX}
\mylist{\@elt John\@elt Mary\@elt Mathew \@elt George \@elt Maria}
\end{teX}

\begin{docCommand}{@elt}{}
Using a macro to delimit the list elements, has the advantage that when we invoke the |mylist| list macro the |@elt| macro can map a function over the elements. We will see that a bit later in more detail but for the time being we will demonstrate this with an example:
\end{docCommand}

\begin{texexample}{Elt Lists}{}
\def\mylist{\@elt John,\@elt Mary,\@elt Mathew, \@elt George, \@elt Maria,}
\def\@elt#1,{\textit{#1} }
\mylist
\end{texexample}

Note that the macro |\@elt| when it is defined is delimited with a comma. 

\newthought{Adding Elements}

\begin{docCommand}{g@addto@macro}{}
There are many ways to add an element to the list, but perhaps the easiest is to use the \latex \cmd{\g@addto@macro}. 
\end{docCommand}

\emphasis{g@addto@macro}
\begin{teXXX}
\g@addto@macro{\mylist}{\@elt Thomas,}
\mylist
\end{teXXX}

One disadvantage of this approach is that the last item on the list will have a comma. A better approach would be to check if
the list is empty and to insert an elemen

\begin{texexample}{Lists}{}
\makeatletter
\def\mylist{Yiannis}
\def\emptylist{}

\def\addtomylist#1{%
\if\mylist\emptylist
   \g@addto@macro{\mylist}{#1,}
\else
   \g@addto@macro{\mylist}{,#1}
\fi
}
\addtomylist{George}
\addtomylist{Maria}
\addtomylist{Athena}

\mylist
\makeatother
\end{texexample}
 



\subsection{How to Use LaTeX’s kernel looping constructs}

\begin{docCommand}{@for}{}
The \cmd{\@for} is an internal \latexe command that can be used to iterate over a comma delimited list.
\end{docCommand}

\emphasis{mylist}
\begin{teX}
\makeatletter
\def\mylist{1,2,3,4,5}(*@\label{list}@*)
\@for\val:=\mylist\do{\val
\ifx\@xfor@nextelement\@nnil \else ;\fi}
\makeatother
\end{teX}


\latex's  low-level programming is rather poorly documented and the section on what is called control commands is even more so. The current \latex team are trying to provide some proper looping structures in \latex3. 

If you want to loop over comma-lists, \latex provides the \cmd{\@for} macro. This works by repeatedly assigning list items to a temporary variable:

To use it we need to define a list:

\startlineat{50}
\begin{teX}
\def\mathList{\alpha,\beta,\gamma,
          \delta,\epsilon,\zeta,\theta }
\end{teX}


\def\mathList{\alpha;\beta;\gamma;\delta;\epsilon;\zeta;\theta}

Using the |@for| loop we can iterate over the list as follows:

\begin{teX}
\@for\i:=\mathList\do{%
  \ensuremath \i  \space 
 }
\end{teX}

\parindent0pt

Running the example we simply get the list but now without the comma

\begin{teX}
\makeatletter
\def\mathList{\alpha, \beta, \gamma, \delta, \epsilon, \zeta, \theta }
\@for\i:=\mathList\do{%
  \ensuremath \i  \space 
 }
\makeatother
\end{teX}




\begin{teX}
\makeatletter
\def\atestiii{}
\def\alist{a,b,c,d,v,e,f,g,h}
Test 1 \@removeelement{v}{a,b,c,d,v,e,f,g,h}{\atestiii} 
returns \atestiii
\alist
\gdef\blist{1,2,3,4,5,v,6,7}%
Test 2 \@removeelement{v}{\expand\blist}{\atestii} prints \atestii
\meaning\atestii
\meaning\@removeelement

\def\remove#1#2{
 \@removeelement #2{#1}\atestiii \atestiii
}

removes an element \atestiii ~~and \alist

\remove c\alist 


the variable holding the list \atestiii
\end{teX}


The iteration does not have the proper meaning that you would normally expect in other programming languages, it is defined as follows:


\begin{verbatim}
\@for(*@\textsubscript{all~elements of the list to }@*)\i:=\mathList\do{%
  \ensuremath \i  \space 
 }
\end{verbatim}

The other interesting thing to note as well as watch out is `:=', which is just a delimiter. I have used |\i| for simplicity, but \cmd{\i}, is a reserved word, meaning a \textit{dotless} i, which is found in some language like Turkish. If you going to use it in your writings you will need to save it and restore it afterwards. A lot of macro writers also use |\ii| or |\@i| or other similar variables. It is simply a temporary variable that at the end of the iteration gets the value |\@nil| and since |@nil| is undefined it essentially destroys it. 

We need to remind ourselves again about |##|
20.5. The |##| feature is indispensable when the replacement text of a definition
contains other definitions. For example, consider


\begin{teX}
\def\a#1{\def\b##1{##1#1}}
after which `\a!' will expand to `\def\b#1{#1!}'. We will see later that ## is also
important for alignments; see, for example, the definition of \matrix in Appendix B.
\end{teX}

\begin{teX}
\long\def\@for#1:=#2\do#3{%
\expandafter\def\expandafter\@fortmp\expandafter{#2}%
\ifx\@fortmp\@empty \else
\expandafter\@forloop#2 ;\@nil;\@nil\@@#1{#3}\fi}

\long\def\@iforloop#1;#2\@@#3#4{\def#3{#1}\ifx #3\@nnil
\expandafter\@fornoop \else
#4\relax\expandafter\@iforloop\fi#2\@@#3{#4}}


\@for\i:=\mathList\do{%
  \ensuremath \i --  
 }


\meaning\loop 
\def\a#1{\textcolor{blue}{\uppercase{#1}}}
\def\b{test}
\expandafter\a\b

\a\b
\end{teX}


\section{Recursion}
Recursion is a difficult subject to grasp, although we experience it daily in our actions, language and thoughts. 
The main characteristic of recursion, is that it can take its own output as the next input, a loop that can be extended indefinitely to create sequences of structures of unbounded length or complexity. In language we understand that a sentence can in principle be extended indefinitely, even though in practice it cannot be---although the novelist Henry James had a damn good try in the \emph{The Figure in the Carpet}. Of course what we are interested here is to study how we can write recursive macros in \tex rather than the more interesting aspects of recursion as it applies to thoughts and language. 

When we write:

\begin{verbatim}
\def\mymacro{\mymacro}
\end{verbatim}

The macro will expand itself indefinitely. As it does not save anything in memory it does not exceed the capacity of any data structure, it will just cause your computer to hang.

If we slightly modify the above macro to |\def\mymacro{a \mymacro}| during expansion the macro will typeset and then call itself again, typeset a and call itself again forever. After a while, it overflows the computer’s memory. The reason for this is that we never finish a paragraph. The letters accumulate in main memory as part of the same paragraph.


\begin{teXXX}
\def\parselist#1;{\pickup#1,;,}
\def\pickup#1,{% Note that #1 may be \null
\if;#1\let\next=\relax(*@\label{test}@*)
\else\let\next=\pickup
    #1% use #1 in any way
\fi\next}
\parselist $a_1$, $a_2$, $a_3$, $a_4$, $a_5$;
\end{teXXX}

The macro \cmd{\pickup} expects its argument to be delimited by `,’, so it ends up
getting the first component of the original argument. It uses it in any desired way and
then expands itself recursively. The process ends when the current argument becomes the `;’. The compound argument may have any number of components (even zero).\ref{test}

\section{LaTeX3 Recursion}



%%endinput iteration.tex









