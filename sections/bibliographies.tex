\chapter{Bibliography Management} 
\begin{refsection}
%\begin{figure}[p]
%\includegraphics[width=\textwidth]{./images/ammar.jpg}
%\caption{Wilson, Digital Collage, L. Ammar \protect\url{http://daliahammar.com/post/49217473452/wilson-digital-collage}}
%\end{figure}
 
%\precis{In this chapter we outline a number of experimental keys that been defined to handle Table of Contents (ToC) formatting. These keys are currently experimental.}
%\addtocimage{-12pt}{-20pt}{./images/tocblock-man-02.jpg}


For any academic/research writing, incorporating references into a document is an important task. Fortunately, \latex provides  a variety of features that make dealing with references much simpler, including built-in support for citing references. However, a much more powerful and flexible solution is achieved thanks to an auxiliary tool called \bibtex and if your \latex  distribution does not include it is obtainable from \url{http://www.bibtex.org}.


The style of this book places all citations to the side margin. For example, the command  \verb+\cite{Abrahams2003}+, will produce the citation \cite{Abrahams2003}. I find this type of style (suggested by \cite{Tufte1997}) more clear and relevant.

Notes in text for many centuries, before printed books were a common feature. The author picking up a different thread and not wishing to divert immediate attention away from the main body of his work. With printing, the costs of books were high and printers started placing citations and footnotes at the bottom of the page. You are not limited though to use only this style, by using |\cite{Bringhurst2005}|, \citet{Bringhurst2005}.

You can also use, the following code to get a within the text full citation:


|\bibentry{Bringhurst2005}|



\bibtex provides for the storage of all references in an external, flat-file database. This database can be linked to any \latex document, and citations made to any reference that is contained within the file. This is often more convenient than embedding them at the end of every document written. There is now a centralized bibliography source that can be linked to as many documents as desired (write once, read many!). 

Of course, bibliographies can be split over as many files as one wishes, so there can be a file containing references concerning General Relativity and another about Quantum Mechanics. When writing about Quantum Gravity (QG), which tries to bridge the gap between these two theories, both of these files can be linked into the document, in addition to references specific to QG.

\section{Markup for Citations and bibliography}

Classical \latex's environment for generating a list of references or a bibliography is called
the \docAuxEnv{thebibliography}. In its default configuration it automatically
generates an appropriate heading and implements a vertical list structure in which every publication is represented as a separate item.

\begin{teX}
\begin{thebibliography}{widest label}
\bibitem[label1] {cite-key1} bibliographic information
\bibitem[label2] {cite-key2} bibliographic information
      ...
\end{thebibliography}
\end{teX}

To actually cite a given document is very easy. Go to the point where you want the citation to appear, and use the following: cite cite key, where the cite key is that of the bibitem you wish to cite. When LaTeX processes the document, the citation will be cross-referenced with the bibitems and replaced with the appropriate number citation. The advantage here, once again, is that LaTeX looks after the numbering for you. If it were totally manual, then adding or removing a reference would be a real chore, as you would have to re-number all the citations by hand.

Instead of WYSIWYG editors, typesetting systems like TeX or LaTeX \citep{lamport2004} can be used. cite{Abut1990}

\section{Referring to specific pages}

Sometimes you want to refer to a certain page, figure or theorem in a text book. For that you can use the arguments to the 

\begin{texexample}{Citation Example}{}
\cs{cite} command:
\cite[p. 215]{Mittelbach2004}
\end{texexample}

The argument, "p. 215", will show up inside the same brackets

\section{BibTeX}

I have previously introduced the idea of embedding references at the end of the document, and then using the \cs{cite} command to cite them within the text. In this tutorial, I want to do a little better than this method, as it's not as flexible as it could be. Which is why I wish to concentrate on using BibTeX.

A BibTeX database is stored as a .bib file. It is a plain text file, and so can be viewed and edited easily. The structure of the file is also quite simple. An example of a BibTeX entry:

\begin{verbatim}
@article{greenwade93,
    author  = "George D. Greenwade",
    title   = "The {C}omprehensive {T}ex {A}rchive {N}etwork ({CTAN})",
    year    = "1993",
    journal = "TUGBoat",
    volume  = "14",
    number  = "3",
    pages   = "342--351"
}
\end{verbatim}

Each entry begins with the declaration of the reference type, in the form of @type. BibTeX knows of practically all types you can think of, common ones are: book, article, and for papers presented at conferences, there is inproceedings. In this example, I have referred to an article within a journal.\sidenote{\obeylines 
book,
article,
conference
}

After the type, you must have a left curly brace '\{' to signify the beginning of the reference attributes. The first one follows immediately after the brace, which is the citation key. This key must be unique for all entries in your bibliography. It is this identifier that you will use within your document to cross-reference it to this entry. It is up to you as to how you wish to label each reference, but there is a loose standard in which you use the author's surname, followed by the year of publication. This is the scheme that I use in this tutorial.

Next, it should be clear that what follows are the relevant fields and data for that particular reference. The field names on the left are BibTeX keywords. They are followed by an equals sign (=) where the value for that field is then placed. BibTeX expects you to explicitly label the beginning and end of each value. I personally use quotation marks ("), however, you also have the option of using curly braces \verb+('{', '}')+. But as you will soon see, curly braces have other roles, within attributes, so I prefer not to use them for this job as they can get more confusing. 

A notable exception is when you want to use characters with umlauts (ü, ö, etc), since their notation is in the format \verb+\"{o}+, and the quotation mark will close the one opening the field, causing an error in the parsing of the reference.

Remember that each attribute must be followed by a comma to delimit one from another. You do not need to add a comma to the last attribute, since the closing brace will tell BibTeX that there are no more attributes for this entry, although you won't get an error if you do.

It can take a while to learn what the reference types are, and what fields each type has available (and which ones are required or optional, etc). So, look at this entry type reference and also this field reference for descriptions of all the fields. It may be worth bookmarking or printing these pages so that they are easily at hand when you need them.

\section{Authors}

BibTeX can be quite clever with names of authors. It can accept names in forename surname or surname, forename. I personally use the former, but remember that the order you input them (or any data within an entry for that matter) is customizable and so you can get BibTeX to manipulate the input and then output it however you like. If you use the forename surname method, then you must be careful with a few special names, where there are compound surnames, for example "John von Neumann". In this form, BibTeX assumes that the last word is the surname, and everything before is the forename, plus any middle names. You must therefore manually tell BibTeX to keep the 'von' and 'Neumann' together. This is achieved easily using curly braces. So the final result would be "John {von Neumann}". This is easily avoided with the surname, forename, since you have a comma to separate the surname from the forename.

Secondly, there is the issue of how to tell BibTeX when a reference has more than one author. This is very simply done by putting the keyword |and| in between every author. As we can see from another example:


\section{The natbib package}

Using the standard \latex bibliography support, you will see that each reference is numbered and each citation corresponds to the numbers. The numeric style of citation is quite common in scientific writing. In other disciplines, the author-year style, e.g., (Roberts, 2003), such as Harvard is preferred, and is in fact becoming increasingly common within scientific publications. A discussion about which is best will not occur here, but a possible way to get such an output is by the natbib package. In fact, it can supersede LaTeX's own citation commands, as |natbib| allows the user to easily switch between Harvard or numeric \docpkg{natbib} \citep{natbib2009}.


The first job is to add the following to your preamble:

\begin{verbatim}
\usepackage{natbib}
\end{verbatim}


The bibliography |.bib| file is still typed using the normal format as for example:---

\begin{verbatim}
@book{goossens93,
    author    = "Michel Goossens and Frank Mittlebach and Alexander Samarin",
    title     = "The LaTeX Companion",
    year      = "1993",
    publisher = "Addison-Wesley",
    address   = "Reading, Massachusetts"
}
\end{verbatim}



Also, you need to change the bibliography style file to be used, so edit the appropriate line at the bottom of the file so that it reads: |\bibliographystyle{plainnat}|. Once done, it is basically a matter of altering the existing \texttt{cite} commands to display the type of citation you want.


The main commands simply add a (t)  for 'textual' or (p) for 'parenthesized', to the basic \cs{cite} command. You will also notice how Natbib by default will compress references with three or more authors to the more concise 1st surname et al version. By adding an asterisk (*), you can override this default and list all authors associated with that citation. There are some other less common commands that Natbib supports, listed in the table here.

Using |natbib|, can satisfy every style required by a stern and difficult editor.

\begin{table}[htp]
\begin{tabular}{lp{8cm}}
\toprule
Citation command	&Output\\
\midrule
\verb+ \citet{goossens93}+	&\citet{goossens93}\\
\verb+ \citep{goossens93}+	&\citep{goossens93}\\
\verb+ \citet*{goossens93}+	&\citet*{goossens93}\\
\verb+ \citep*{goossens93}+	&\citep*{goossens93}\\
\verb+ \citeauthor{goossens93}+	&\citeauthor{goossens93} \\
\verb+ \citeauthor*{goossens93}+	&\citeauthor*{goossens93}\\
\verb+ \citeyear{goossens93}+	&\citeyear{goossens93}\\
\verb+ \citeyearpar{goossens93}+	&\citeyearpar{goossens93}\\
\verb+ \citealt{goossens93}+	&\citealt{goossens93}\\
\verb+ \citealp{goossens93}+	&\citealp{goossens93}\\
\bottomrule
\end{tabular}
\caption{Natbib package commands}
\end{table}

When changing the bibliography style, sometimes natbib is upset because it can't interpret the data correctly.

In any case, after changing the argument to |\bibliographystyle| a run of LaTeX and one of BibTeX are necessary to get back in sync. Removing the |.bbl| and |.aux| files before those run is recommended, in order to avoid spurious error messages that might corrupt the .aux file currently being generated.\footnote{\url{http://tex.stackexchange.com/questions/54480/package-natbib-error-bibliography-not-compatible-with-author-year-citations}}

\section{Including URLs in bibliography}

As you can see, there is no field for URLs. One possibility is to include Internet addresses in howpublished field of @misc or note field of |@techreport|, |@article|,|@book|:

\begin{lstlisting}[language={[common]TeX},% 
                           alsolanguage={[LaTeX]TeX},% 
                           alsolanguage={[primitive]TeX},%
                           ]
howpublished = "\url{http://www.example.com}"
\end{lstlisting}

Note the usage of \cs{url} command to ensure proper appearance of URLs.
Another way is to use special field url and make bibliography style recognise it.

\begin{lstlisting}[language={[common]TeX},% 
                           alsolanguage={[LaTeX]TeX},% 
                           alsolanguage={[primitive]TeX},%
                           ]
URL = "http://www.example.com"
\end{lstlisting}

You need to use \texttt{usepackage{url}} in the first case or \texttt{usepackage{hyperref}} in the second case.
Styles provided by Natbib (see below) handle this field, other styles can be modified using |urlbst| program. Modifications of three standard styles (|plain|, |abbrv| and |alpha|) are provided with |urlbst|.


\section{changing punctuation}

When I started using natbib I kept getting square barackets. Use
\begin{lstlisting}[language={[common]TeX},% 
                           alsolanguage={[LaTeX]TeX},% 
                           alsolanguage={[primitive]TeX},%
                           ]
    \bibpunct{(}{)}{;}{a}{,}{,}
    \bibliographystyle{plainnat}
\end{lstlisting}

\section{Error Checking}

You can check the file for errors by runing it through |bibTeX|. This will point database errors etc. 


\subsection{Entry Types}

Bibliography entries included in a .bib file are split by types. The following types are understood by virtually all |BibTeX| styles:

\subsubsection*{article}
  An article from a journal or magazine.

  Required fields: author, title, journal, year

  Optional fields: volume, number, pages, month, note, key

\emph{book}
   A book with an explicit publisher.
   Required fields: author/editor, title, publisher, year
   Optional fields: volume, series, address, edition, month, note, key

\emph{booklet}
   A work that is printed and bound, but without a named publisher or sponsoring institution.
   Required fields: title
   Optional fields: author, howpublished, address, month, year, note, key

\emph{conference}
   The same as inproceedings, included for Scribe compatibility.
   Required fields: author, title, booktitle, year
   Optional fields: editor, pages, organization, publisher, address, month, note, key

\emph{inbook}

    A part of a book, usually untitled. May be a chapter (or section or whatever) and/or a range of pages.
    Required fields: author/editor, title, chapter/pages, publisher, year
    Optional fields: volume, series, address, edition, month, note, key

\emph{incollection}

    A part of a book having its own title.
    Required fields: author, title, booktitle, year
    Optional fields: editor, pages, organization, publisher, address, month, note, key

\emph{inproceedings}

An article in a conference proceedings.
Required fields: author, title, booktitle, year
Optional fields: editor, series, pages, organization, publisher, address, month, note, key

\emph{manual}

Technical documentation.
Required fields: title
Optional fields: author, organization, address, edition, month, year, note, key

\emph{mastersthesis}

A Master's thesis.
Required fields: author, title, school, year
Optional fields: address, month, note, key

\emph{misc}

For use when nothing else fits.

Required fields: none
Optional fields: author, title, howpublished, month, year, note, key

\emph{phdthesis}

A Ph.D. thesis.

Required fields: |author|, |title|, |school|, |year|\\
Optional fields: |address|, |month|, |note|, |key|
proceedings
The proceedings of a conference.
Required fields: title, year
Optional fields: editor, publisher, organization, address, month, note, key
techreport
A report published by a school or other institution, usually numbered within a series.
Required fields: author, title, institution, year
Optional fields: type, number, address, month, note, key
unpublished
A document having an author and title, but not formally published.
Required fields: author, title, note
Optional fields: month, year, key

\section{The bibentry package}

 This package allows one to be able to place bibliographic entries anywhere
 in the text. It is to be used to produce annotated bibliographies, such as
 \begin{quote}
   For an intoduction to this topic, see Jones, J.~R., Basics on this topic,
   {\it J.\ Last Resorts}, \textbf{13}, 234--254, 1994. For more advanced
   information, see \dots.
 \end{quote}

 The idea is that the full reference is used, not just the citation Jones
 [1994].

 \section{Invoking the Package}
 The macros in this package are included in the main document
 with the |\usepackage| command of \LaTeXe,
 \begin{quote}
 |\documentclass[..]{...}|\\
 |\usepackage{|\texttt{\filename}|}|
 \end{quote}

 \section{Usage}

 \newcommand\btx{\textsc{Bib}\TeX}
 This package must be used with \btx, not with a hand-written
 \texttt{thebibliography} environment.

 More precisely, there must be a \texttt{.bbl} file external to the \LaTeX\
 file; whether this is written by hand or by BibTeX is unimportant.

| \nobibliography|
 The bibliography entries are stored with the command
 |\nobibliography| |\marg{bibfiles}|, which is like the usual
 |\bibliography| |\marg{bibfiles}| except no bibliography is printed. The
 \texttt{.bbl} file is read in as usual but the \texttt{thebibliography} is
 redefined so that all the entries are stored, not printed.


 The text of the entries may be printed with the command
 \begin{quote}
    |\bibentry| |\marg{key}|
 \end{quote}

 These commands may only be issued after |\nobibliography|, for otherwise
 the reference texts are not known.

 The final period of the original text will be missing, so that one can add
 punctuation as one pleases.

 Regular |\cite| (or the \texttt{natbib} versions) may be issued anywhere as
 usual.

|\nobibliography*|
 If a regular list of references is to be given too, with the
 |\bibliography|\sidenote{bibfiles} command, issue the starred version
 |\nobibliography*| (without argument) in order to store the bib entry texts.
 This will load the same \texttt{.bbl} file as |\bibliography|, but will avoid
 messages from BibTeX about multiple |\bibdata| commands and warnings from
 \LaTeX\ about multiply defined citations.

 The processing procedure is as usual:
 \begin{enumerate}
  \item \LaTeX\ the file;
  \item Run \btx;
  \item \LaTeX\ the file twice.
 \end{enumerate}

 \noindent
 \textbf{Note:} it is highly recommended to make use of the \docpkg{url}
 package, which will nicely format both |url| and |doi| addresses; in particular,
 they will break at convenient locations without a hyphen.\index{bibliography>doi}
\index{bibliography>url}




Here are some useful references about \LaTeX. They are
available in every worthy bookshop. Many other good documentations
might be found on the web (the FAQ of \textsf{comp.text.tex} for
instance).


\begin{verbatim}
\bibitem[GMS93]{companion} Michel Goossens, Franck Mittelbach and Alexander
Samarin, \emph{The \LaTeX{} Companion}, Addison Wesley, 1993.
\bibitem[Lam97]{lamport} Leslie Lamport, \emph{\LaTeX: A Document Preparation
System}, Addison Wesley, 1997.
\end{verbatim}

This is the main matter of the document, mentioning
[\ref{doc1}] and [\ref{doc2}], for instance.

\section{The Biblatex package}

Perhaps the most comprehensive and flexible package that has been developed so far is Biblatex.
This package provides advanced bibliographic facilities for use with LaTeX in conjunction
with BibTeX. 

The package completely reimplemented  the bibliographic
facilities provided by LaTeX. It also provides a  custom backend |Biber| by default is used which processes
the BibTeX format data files and them performs all sorting, label generation
(and a great deal more). 

Legacy BibTeX is also supported as a backend, albeit with a
reduced feature set. Biblatex does not use the backend to format the bibliography
information as with traditional BibTeX: instead of being implemented in BibTeX
style files, the formatting of the bibliography is entirely controlled by TeX macros.

Good working knowledge in LaTeX should be sufficient to design new bibliography
and citation styles. There is no need to learn BibTeX’s postfix stack language. This
package also supports subdivided bibliographies, multiple bibliographies within
one document, and separate lists of bibliographic information such as abbreviations
of various fields. 

Bibliographies may be subdivided into parts and/or segmented
by topics. Just like the bibliography styles, all citation commands may be freely
defined. With Biber as the backend, features such as customisable sorting, multiple
bibliographies with different sorting, customisable labels, dynamic data modification
are available. The package is completely localized and can interface with the babel
package. 

\subsection{Citation commands}

The biblatex package offers a full repertoire of both traditional 

\cite{Bringhurst2005}

\parencite{Bringhurst2005}

\footcite{Bringhurst2005}

\footcitetext{Lamport1994}

But youcan also use \autocite{Lamport1994}

\subsection{Text Commands}

\citeauthor{Lamport1994} described in detail the usage of \latex. All citation text commands
are excluded from the citation tracking. In \citetitle{Lamport1994} you can read the background
thinking that logically described.

\footfullcite[See, pg52 of ][Tufte wrote a number of other books, after this original work and went on to become famous.]{Tufte1990}

\subsection{Compatibility with natbib package commands}


\printbibliography[heading=subbibliography]
\end{refsection}




