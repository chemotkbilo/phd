\cxset{chapter numbering=arabic}


\chapter{Futurelet}
\precis{A discussion on one of the most esoteric commands of \protect\tex, with examples as to how to write macros with optional arguments.}
\addtocimage{-12pt}{-20pt}{../images/tocblock-futurelet.jpg}
\epigraph{Life can only be understood backwards; but it must be lived forwards.}{
---S Kierkegaard}

The \docAuxCmd{futurelet} primitive deserves its own chapter, as most people have difficulty in understanding the command. The instruction allows the user to \textit{look ahead}. By look ahead we mean that \tex will look at a future token\footnote{remember that a token is either a single character or a macro command} without absorbing it, i.e, without removing that token from the token list. This operation allows the programmer to perform a test to check what token follows. There are a couple of articles about \cs{futurelet} in TUGBoat; for example \citetitle{Eijkhout2001} by \citeauthor{Eijkhout2001}\footcite{Eijkhout2001} and \citetitle{bechto88} by \citeauthor{bechto88}\footcite{bechto88}. Both articles are generally difficult to follow. 

There are many applications of |\futurelet| but the most common one is that of scanning forward
to find if a `[' is found to build \latexe macros with optional arguments. Here we will  present three examples in detail, including one from the |PGF| module |parser| that can be used to build Deterministic Finite Automata.


\section{The mechanics of futurelet}

The |\futurelet|
instruction has the following format:


\begin{commands}[]{}
\cmd{\futurelet} \meta{token$_1$}\meta{token$_2$}\meta{token$_3$}
\end{commands}


\begin{enumerate}
\item  \tex will execute a \cmd{\let}\meta{token$_1$}=\meta{token$_3$}.
We therefore have generated a copy of \meta{token$_3$}
stored under the name of (token$_1$).\label{lettoken}

\item  removes (token$_1$) from the main token list.

\item \tex expands (token2). This token is for all
practical purposes a macro with the following
properties:

\begin{enumerate}
\item The macro will use (tokenl), which is a
copy of (token3), to find out what (token3)
is, in other words what token is to be
expected later.
\item It will cause another macro to be expanded
which will ultimately absorb (token3).

This other macro ordinarily depends on
what $<token_l>$ is.
\end{enumerate}
\end{enumerate}

The description above, is a bit of a mouthful and it is better to describe it with an example. In Example~\ref{futurelet} we will try and find if the next token is the opening square bracket `['. We then according to the definition in \ref{lettoken} this should be stored in \cs{tokenone}. We verify this by peeking at its meaning.

\begin{texexample}{futurelet}{futurelet}
\def\tokentwo#1{}
\futurelet\tokenone\tokentwo[
\meaning\tokenone
\end{texexample}

The second token \cs{tokentwo} we have defined it, so that it justs absorbs its next argument and does nothing for the time being. As you can see its meaning is \texttt{the character [}. Now what happens if there was a space between the \cs{tokentwo} and the `['?

\begin{texexample}{futurelet second}{futurelet2}
\def\tokentwo#1{}
\futurelet\tokenone\tokentwo     [
\meaning\tokenone
\end{texexample}

As you can see so far the spaces have been absorbed, but let us now change the definition of \cs{tokentwo}.

\begin{texexample}{futurelet second}{futurelet2}
\def\tokentwo#1{}
\futurelet\tokenone\tokentwo     
\meaning\tokenone
\end{texexample}



\begin{texexample}{futurelet}{futurelet}
\def\tokentwo#1{%
   \ifx\tokenone[ true [\else false\fi
}
\futurelet\tokenone\tokentwo[
\meaning\tokenone
\end{texexample}

We try again with spaces,

\emphasis{tokentwo,[}
\begin{texexample}{futurelet}{futurelet}
\def\tokentwo#1{%
   \ifx\tokenone[ true [\else false\fi
}
\futurelet\tokenone\tokentwo     [
\meaning\tokenone
\end{texexample}

As you can see from the examples we cannot capture the spaces. This might present a problem, if we enclose everything in other macros as \tex might leave extra spaces in the stream. Better to absorb them. We will see how later, using LaTeX. 




\subsection{Using \textbackslash futurelet in Macros with Optional
Arguments}

A typical application of |\futurelet| is the handling
of macros with optional arguments\footcite{bechto1988} as they are used,
for instance, in \latex. By ``optional argument" we
mean an argument which in most cases is omitted,
and is provided only occasionally in macro calls.\footnote{See also the discussion at \url{http://tex.stackexchange.com/questions/4557/how-to-use-futurelet-to-define-optional-parameters}}

\textbf{Defining the Problem}

Let us give a specific example: we would like to
define a macro \cmd{xx}, which can be called in two
different ways:

\begin{enumerate}
\item With optional argument as in |\xx [opt]{arg}|
where opt is the optional argument enclosed
in square brackets and \meta{arg} is the mandatory argument
argument.

\item Without optional argument as in |\xx{arg}|
where \meta{arg} is again the regular argument.

\end{enumerate}


Before we discuss how this can be done in \tex,
observe that we do not really have to use an
optional argument. We could simply define two
different macros \cmd{xxwithoptions} for the case where an
optional argument is given, and \cmd{xxnooptions} for the
case where no optional argument is given:


\begin{texexample}{two macros}{ex:twomacros}
\def\xxWithOpt [#1]#2{...}
\def\xxNoOpt #1{...}
\def\xxWithOpt (#1)#2{\fbox{#2}}
\xxWithOpt (box){Testing}
\end{texexample}

How we can use |\futurelet| to find out
whether an optional argument was given or not?

We will define a macro |\xx| whose only function is
to check whether there is an opening square bracket
(optional argument is present) or not (no optional
argument). The |\xx| macro will, after this has been
determined, cause the |\xxWithOpt| macro to be invoked
when there is an optional argument, and the
|\xxNoOpt| macro to be called if there is no opening
bracket. In other words the macros |\xxWithOpt|
and |\xxNoOpt| do the "real work while the only
purpose of the |\xx| macro is to decide which of the
two macros should be invoked.


Here is the completely worked out example.


\begin{teX}
\def \xxWithOpt [#1] #2{...}
\def\xxNoOpt #2{...}

\def\xx {%
\futurelet\xxLookedAtToken
    \xxDecide
}

% (3) The \xxDecide macro, based on
% the lookahead of \xx, calls
% either \xxWithOpt or \xxNoOpt .
\def\xxDecide {%
 \ifx\xxLookedAtToken [%
\let\next = \xxWithOpt
\else
 \let\next = \xxNoOpt
 \fi
\next
}
\end{teX}

\section{Other Applications in the LaTeX kernel}

\begin{teX}
\def\elidebefore[#1]#2{[$\ldots$] #2}
\def\elideafter#1{#1$\ldots$}

\def\elide {%
\futurelet\ifoptions
    \choosemacro
}

\elide{Lorem Ipsum}

\elide[b]{Lorem ipsum}
\end{teX}

\begin{comment}
% The \choosemacro, based on
% the lookahead of \elide, calls
% either \elidebefore or \elideafter 
\end{comment}

\begin{teX}
\def\choosemacro{%
 \ifx\ifoptions [%
     \let\choice = \elidebefore 
 \else
    \let\choice = \elideafter
 \fi
\choice
}
\end{teX}



\begin{teX}
\elide{Lorem Ipsum}

\elide[b]{Lorem ipsum}

\end{teX}

\begin{teX}
\def \xxWithOpt [#1] #2{...}
\def\xxNoOpt #2{...}

\def\xx {%
\futurelet\xxLookedAtToken
    \xxDecide
}

% (3) The \xxDecide macro, based on
% the lookahead of \xx, calls
% either \xxWithOpt or \xxNoOpt .
\def\xxDecide {%
 \ifx\xxLookedAtToken [%
\let\next = \xxWithOpt
\else
 \let\next = \xxNoOpt
 \fi
\next
}
\end{teX}



To build a command with any optional parameter, as you find in many of LaTeX's commands, you will need two things:

\begin{itemize}
\item a macro with delimited parameters

\item a way to grab the first non-space token that follows the command
\end{itemize}


The first part is fairly easy using delimited argument macros, for example we can say

\begin{verbatim}
\def\test(#1)#2#3{#1, #2, #3}
\end{verbatim}

We can then call this macro as:

\begin{verbatim}
\test(a){b}{c}
\end{verbatim}


resulting in a,b,c

To define the |()| as an optional parameter, we effectively need to define the macro as a conditional a sort of a "yes-no" switch. If \tex finds the "(" bracket the "yes-code" will be called and if it finds only the normal arguments the "no-code" will be executed.

For this we can use the |\@ifnextchar| macro from the LaTeX kernel.
You can say |@ifnextchar{char}{yes-code}{no-code}| to test for |(|. The result then will depend on the token that follows. If this token is the same as the first argument, then the "yes-code" is executed, otherwise the "no-code" is executed. The first argument should be a single token (for instance a character). Spaces are ignored. 

As for example we can redefine the LaTeX code for `rule` to accept an optional parameter in round brackets, rather than the traditional square brackets.

\begin{texexample}{Using ifnextchar}{ex:ifnextchar}
\makeatletter
\def\Rule{\@ifnextchar(\@Rule%
        {\@Rule(\z@)}}
\def\@Rule(#1)#2#3{%
 \leavevmode
 \hbox{%
 \setlength\@tempdima{#1}%
 \setlength\@tempdimb{#2}%
 \setlength\@tempdimc{#3}%
 \advance\@tempdimc\@tempdima
 \vrule\@width\@tempdimb\@height\@tempdimc\@depth-\@tempdima}}
\makeatother

A test \Rule(6.5pt){100pt}{1pt}

Another test \Rule{100pt}{3pt}

Not that difficult but you will need to , but why on earth do you need this?
\end{texexample}




\begin{comment}
\def\elidebefore[#1]#2{[$\ldots$] #2}
\def\elideafter#1{#1$\ldots$}

\def\elide {%
\futurelet\ifoptions
    \choosemacro
}

% The \choosemacro, based on
% the lookahead of \elide, calls
% either \elidebefore or \elideafter 


\def\choosemacro{%
 \ifx\ifoptions [%
     \let\choice = \elidebefore 
 \else
    \let\choice = \elideafter
 \fi
\choice
}

Testing \elide[b]{Lorem ipsum}

\elide{Lorem Ipsum}

\elide[b]{Lorem ipsum}

\end{comment}


\section{Using LaTeX \protect\textbackslash @ifnextchar}

\latex defines the |\@ifnextchar| kernel command that is used effectively to
determine the token that follows the command. It is used in the definitions
of macros with optional arguments amongst other things.

\begin{teXXX}
\@ifnextchar]{true}{false}] 
\@ifnextchar[{true}{false}[
\end{teXXX}
The result would both be true,

\begin{texexample}{Example ifnextchar}{ifnextchar}
\makeatletter
\@ifnextchar]{true ]}{false} ] %notice ]
\@ifnextchar[{true [}{false} [ %notice [
\makeatother
\end{texexample}


\section{Building Deterministic Finite Automata (DFA)}
\label{section-module-parser}

The package PGF has a module |parser| that can be used to be a DFA engine for parsing strings of single letter sequences and based on the letter perform some action. The manual is somehow cryptic on its usage, but as it is
used in the |svg| library you can examine and study the code more carefully there.
The module defines some commands for creating a simple
  letter-by-letter parser.

This module provides commands for defining a parser that scans some
given text letter-by-letter. For each letter, some code is executed
and, possibly a state-switch occurs. The parsing process ends when a
final state has been reached.

\begin{docCmd}{pgfparserparse}{ \marg{parser name}\meta{text}}
  This command is used to parse the \meta{text} using the (previously
  defined) parser named \meta{parser name}.

  The \meta{text} is not contained in curly braces, rather it is all
  the text that follows. The end of the text is determined implicitly,
  namely when the final state of the parser has been reached.

  The parser works as follows: At any moment, it is in a certain
  \emph{state}, initially this state is called |initial|. Then, the
  first letter of the \meta{text} is examined (using the |\futurlet|
  command). For each possible state and each possible letter, some
  action code is stored in the parser in a table. This code is then
  executed. This code may, but need not, trigger a \emph{state
    switch}, causing a new state to be set. The parser then moves on
  to the next character of the text and repeats the whole
  procedure, unless it is in the state |final|, which causes the
  parsing process to stop immediately.

  In the following example, the parser counts the number of |a|'s
  in the \text{text}, ignoring any |b|'s. The \meta{text} ends with
  the first~|c|.
\begin{texexample}{An FDA}{ex:fda}
\newcount\mycount
\pgfparserdef{myparser}{initial}{the letter a}
{\advance\mycount by 1\relax}
\pgfparserdef{myparser}{initial}{the letter b}
{} % do nothing
\pgfparserdef{myparser}{initial}{the letter c}
{\pgfparserswitch{final}}% done!

\pgfparserparse{myparser}aabaabababbbbbabaabcccc
There are \the\mycount\ a's.
\end{texexample}
\end{docCmd}

\begin{docCmd}{pgfparserdef}{ \marg{parser name}\marg{state}\marg{symbol meaning}\marg{action}}
  This command should be used repeatedly to define a parser named
  \meta{parser name}. With a call to this command you specify that the
  \meta{parser name} should do the following: When it is in state
  \meta{state} and reads the letter \meta{symbol meaning}, perform the
  code stored in \meta{action}.

  The \meta{symbol meaning} must be the text that results from
  applying the \TeX\ command |\meaning| to the given character. For
  instance, |\meaning a| yields |the letter a|, while |\meaning 1|
  yields |the character 1|. A space yields |blank space|. 

  Inside the \meta{action} you can perform almost any kind of
  code. This code will not be surrounded by a scope, so its effect
  persists after the parsing is done. However, each time after the
  \meta{action} is executed, control goes back to the parser. You
  should not launch a parser inside the \meta{action} code, unless you
  put it in a scope.

  When you set the \meta{state} to |all|, the state \meta{action} is
  performed in all states as a fallback, whenever \meta{symbol
    meaning} is encountered. This means that when you do not specify
  anything explicitly for a state and a letter, but you do specify
  something for |all| and this letter, then the specified
  \meta{action} will be used.

  When the parser encounters a letter for which nothing is specified
  in the current state (neither directly nor indirectly via |all|), an
  error occurs.
\end{docCmd}

\begin{docCmd}{pgfparserswitch}{ \marg{state}}
  This command can be called inside the action code of a parser to
  cause a state switch to \meta{state}.
\end{docCmd}

Let us first use an example from the PGF manual (page 752) to understand better what we will try and do later with
or parser that we intend to use.

\begin{texexample}{SVG Paths}{ex:svgpaths}
\begin{pgfpicture}
\pgfpathsvg{M10 80 Q 52.5 10, 95 80 T 180 80}
\pgfusepath{stroke}
\end{pgfpicture}
\end{texexample}

The |SVG| syntax consists of numerous commands staring from a letter and followed such as \textbf{M 20 20}. In the
example M stands for move and Q stands for Quadratic. The letter T just strings multiple quadratic Beziers. After the first control point is defined the shortcut looks at the previous control point and infers a new one from it. This means that after yur first control point, you can make fairly complex shapes by specifyig only end points. The mozilla website has a good tutorial if you want to use svg\footnote{\protect\url{https://developer.mozilla.org/en-US/docs/Web/SVG/Tutorial/Paths}}. In general it will be easier to use a graphics program and then transfer the code to PGF. Anyway this chapter is about something else.

\begin{texexample}{Parser example}{ex:headingsparser}
\bgroup
\pgfparserdef{headparser}{initial}{the letter C}
{\boxanelement{chapter}}
\pgfparserdef{headparser}{initial}{the letter N}
{\boxanelement{number}} 
\pgfparserdef{headparser}{initial}{the letter T}
{\boxanelement{title}} 
\pgfparserdef{headparser}{initial}{the letter p}
{\breakline} 
\pgfparserdef{headparser}{initial}{the character ;}
{\pgfparserswitch{final}}

\def\boxanelement#1{\fbox{\LARGE#1 }}%
\def\breakline{\par\nointerlineskip}%
\pgfparserparse{headparser}CpNpTp;
\medskip
\pgfparserparse{headparser}CNT;
\egroup
\end{texexample}

Since we did not define any action for the case the user types |C N T;| rather than |CNT| our parser will lead to an error message. We do this next by adding another definition this time for all the states.

\begin{verbatim}
\pgfparserdef{headparser}{all}{blank space \space}{}
\end{verbatim}

\begin{texexample}{Parser example}{ex:headingsparser2}
\bgroup
\makeatletter
\cxset{chapter font-family=sffamily, chapter name=Section,
          chapter numbering=arabic, chapter font-weight=bold}
          
\pgfparserdef{headparser}{initial}{the letter C}
{\boxanelement{\LARGE\chaptername}}
\pgfparserdef{headparser}{initial}{the letter N}
{\boxanelement{\LARGE\thechapter}} 
\pgfparserdef{headparser}{initial}{the letter T}
{\boxanelement{\LARGE title}} 
\pgfparserdef{headparser}{initial}{the letter p}
{\breakline} 
\pgfparserdef{headparser}{all}{the letter b}
{before } 
\pgfparserdef{headparser}{all}{the letter a}
{after } 

% meaning of letter A
\pgfparserdef{headparser}{initial}{the letter A}
{\boxanelement{Authors}}

% meaning of letter E
\pgfparserdef{headparser}{initial}{the letter E}
{\boxanelement{Epigraph}}

\pgfparserdef{headparser}{all}{blank space \space}{\ifvmode\vskip3pt\else\space\fi}
\pgfparserdef{headparser}{initial}{the character ;}
{\pgfparserswitch{final}}

\def\boxanelement#1{\fbox{\LARGE#1 }}
\def\breakline{\par\nointerlineskip}
\pgfparserparse{headparser}Cp Np Tp Ep;
\medskip
\pgfparserparse{headparser}C N Tp Ap Ep;
\makeatother
\egroup
\end{texexample}

It will be nice to know, what particular element we are typesetting in order to perhaps detect if we have additional elements and instructions for the typesetting engine. This brings us to the notion of state. So what we can do is extend our parser to define different states for elements. 

We need also to handle the code before and code after settings. We can think of an element as an object and for the properties to be |C.a| |C.b| so we will limit our pattern DFA to the essential elements. 

Introducing 2-letter combinations or more will create a more complex situation. If we limit the letters to two a
all we need is to store them in a macro. If it is empty we are at the beginning and if we have entered it twice we are at the end. 

\section{How does the pgf parser module works?}

This fine piece of code is defined in about fifty lines of code, so it is short and brief and it leverages \tex’s two commands \cmd{\futurelet} and \cmd{\csname}\ldots\cmd{\endcsname}.

% Defines an action to be taken in a state for some symbol
% 
% \#1 = parser
% \#2 = state
% \#3 = symbol text (result of saying \meaning on the symbol)
% \#4 = action
% 
% Description:
% 
% When the parser encounters #3 while being in state #2, then code #4
% is executed.
% 
% When #2 is set to the special text "all", then the symbol is handled
% in all states by #4, except when a state defines something special
% for this symbol
\startlineat{10}

Recall from the previous section that the user has to provide the parameters for each action. This is done in a series of calls to the \cmd{\pgfparserdef} which takes four parameters. The macro is defined next using a |csname| construct. The name of the macro combines 3 arguments. Parameter \texttt{\#1} is the parser name, \texttt{\#2} the state, \texttt{\#3} the symbol text (as meaning) and \texttt{\#4} the action required. 

\begin{teX}
\def\pgfparserdef#1#2#3#4{%
  \expandafter\def\csname pgf@parser@a@#1@#2@#3\endcsname{#4}
}
\end{teX}



\begin{teXXX}
\def\pgfparserparse#1{%
  \def\pgf@parser{#1}%
  \def\pgf@parser@state{initial}%
  \pgf@parser@parse%
}
\end{teXXX}

Line 10 defines the main parsing function, which in its turn defines the name of the parser and sets the initial state to initial. It then calls the internal macro |\pgf@parser@parse|, which is the interesting part, as it uses \cmd{\futurelet}. 

Recall that futurelet uses three tokens. The first two are defined in the code whereas the third one is picked from the stream. In our case the symbol will be captured in |\pgf@parser@symbol| which will then be passed on to the second token which is a macro to continue reading the stream.

\begin{teXXX}
\def\pgf@parser@parse{%
  \futurelet\pgf@parser@symbol\pgf@parser@cont%
}
\end{teXXX}

Essentially what we are doing here is saying |\let\pgf@parser@symbol=A|. This is then used to define the action
of the DFA.  

\begin{teXXX}
\def\pgf@parser@cont{%
  % Ok, defined?
  \expandafter\let\expandafter\pgf@parser@action%
  \csname pgf@parser@a@\pgf@parser @\pgf@parser@state @\meaning\pgf@parser@symbol\endcsname%
  \ifx\pgf@parser@action\relax%
    \expandafter\let\expandafter\pgf@parser@action%
    \csname pgf@parser@a@\pgf@parser @all@\meaning\pgf@parser@symbol\endcsname%
    \ifx\pgf@parser@action\relax%
      \pgferror{Unexpected character
        '\meaning\pgf@parser@symbol' in parser '\pgf@parser' in state
        '\pgf@parser@state'}%
    \fi%
  \fi%
  \pgf@parser@action%
  \ifx\pgf@parser@symbol\pgfutil@sptoken(*@\label{riddedstart}@*)%
    \expandafter\pgf@parser@rid@space%
  \else%
    \expandafter\pgf@parser@rid@other%  
  \fi(*@\label{ridded}@*)%
}
\end{teXXX}


When the user defines an action for a parser the code defines a token with the form

\texttt{pgf@parser@a@myparser@initial@\meaning A}

The continue function above checks if this function is defined, and then executes the action. 

After executing the action the function needs to remove the current token from the stream of letters of the DFA and then call the parsing macro again. In Lines~\ref{riddedstart}--\ref{ridded} the code checks if the token is a space or not and calls an auxiliary macro to remove it. \tex when parsing a macro ignore spaces after it and some special trickery is required to remove the space token. In the \latexe kernel this is normally done using \cmd{\sp@token}. PGF has its own token \cmd{\pgfutil@sptoken}.

\begin{teXXX}
{%
  \def\:{\pgf@parser@rid@space} \expandafter\gdef\: {\futurelet\pgf@parser@ignore\pgf@parser@ridded}
  \gdef\pgf@parser@rid@other{\afterassignment\pgf@parser@ridded\let\pgf@parser@ignore=}  
}
\end{teXXX}

Lastly the parser is called again by the command |\pgf@parser@ridded| or terminated if it has reached the final state, see \cmd{\pgf@parser@ridded} at Line~\ref{parserrrided}.

\begin{teXXX}
\def\pgf@parser@ridded{(*@\label{parserrrided}@*)%
  \ifx\pgf@parser@state\pgf@parser@final@text%
    % done
  \else%
    \expandafter\pgf@parser@parse%
  \fi%
}
\end{teXXX}















