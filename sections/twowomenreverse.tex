
\parindent0pt

\begin{minipage}{1.05\textwidth}
\vspace{\baselineskip}
\parindent0pt
\fboxrule0pt
{
\centering
\fbox{\centering
\begin{minipage}[t]{0.89\textwidth}
\centering
\begin{minipage}[t]{0.41\textwidth}
\includegraphics[width=1\textwidth]{./images/threewomen01.png}\vspace*{-8pt}%
\captionof*{figure}{\noindent\footnotesize\textbf{WALDO PEIRCE}, a famous painting in his own right,
turned model for Bellows, posed for this impressive portrait in New York studio in 1920.}
\end{minipage}\hspace{0.5cm}
\begin{minipage}[t]{0.4\textwidth}
   \includegraphics[width=1\textwidth]{./images/threewomen02.png}\vspace*{-8pt}
    \captionof*{figure}{\noindent\footnotesize\textbf{MRS KATHERINE ROSEN,}
                 the daughter of Charles Rosen, he was an artist and neighbor of bellows, 
                 posed for this  meditative study in 1921.}
\end{minipage}
\end{minipage}
}}

\medskip

\fbox{\hskip-0.3cm\includegraphics[width=1.03\textwidth]{./images/twowomen-03.png}}\\[-27.5pt]
\setlength{\linewidth}{.95\textwidth}
\setlength{\columnsep}{8pt}
\begin{multicols}{2}
\noindent \footnotesize\textbf{TWO WOMEN,} portrays a professional model dressed and undressed. The range and richness of colors is unusual among Bellows' pictures. Bellows always had a horror of studio pictures and ``pretty nudes,'' rarely worked from professional models and never painted a still life.
\end{multicols}
\vfill

\captionof{figure}{Balancing three images on a page. Should the larger image be at the top or at the bottom?}
\end{minipage}

\newcommand\articleheading[1]{%
    \par
    \vspace*{2\baselineskip}
    \bgroup
    \LARGE\bf\textsf{\noindent #1}
    \egroup
   \vskip2\baselineskip
}
\clearpage

\begin{minipage}{\textwidth}
\includegraphics[width=\textwidth]{./images/yaleartschool.png}

\articleheading{TRADITION AND TECHNIQUE AT YALE'S SCHOOL OF  FINE ARTS}

\end{minipage}
\begin{multicols}{3}
        \lettrine{A}{t Yale}\lorem \lipsum[1-3]
        \par
\end{multicols}

\newgeometry{top=0pt, left=0pt, right=0pt, top=0pt, bottom=2cm}
\pagebreak

\begin{minipage}{\textwidth}
\includegraphics[width=\textwidth]{./images/sculpture-lesson.jpg}\par
\vspace{\baselineskip}

\centerline{\HUGE\bfseries SCULPTURE LESSON}
\vspace{0.5\baselineskip}

\centerline{\LARGE\bfseries Noted arist shows how adventurous amateurs can model with clay }

\end{minipage}

{
\leftskip1cm\rightskip1cm\columnsep-1.3cm\par\leavevmode

\begin{multicols}{3}
        \lettrine{A}{t Yale} \lorem \lorem \lorem \lorem
        
\end{multicols}
}

\newgeometry{top=1.5cm,left=2cm,right=2cm,bottom=2cm}

\pagebreak





\lipsum[1]
\includegraphics[height=0.8\textheight, width=\textwidth\relax]{./images/nino.png}

This is a short caption test and this one is a long caption test.

\includegraphics[width=\textheight, width=\textwidth]{./images/woman.png}
Donna Velata.

\clearpage
\raggedbottom

%% Odalisque  template
\thispagestyle{plain}
\noindent\includegraphics[width=\textwidth]{./images/odalisque.png} \vskip0pt%
This is a short caption test and this one is a long caption test.

\vspace*{2\baselineskip}

%%% Ginerva?
\begin{minipage}[t]{0.3\textwidth}
\vbox to -6cm{\noindent\includegraphics[width=\linewidth]{./images/ginerva.png}\vskip0pt%
This is a short caption test and this one is a long caption test.}
\end{minipage}%
% header
\begin{minipage}[t]{.7\textwidth}%
\noindent\textbf{\Huge \hfill Kathleen Gilje\hskip0.1em\hfill}\\[2\baselineskip]
\end{minipage}


\leftskip0.41\textwidth

\lettrine{T}{he Kathleen Gilje} template is another common style found in many books and magazines. They are comparatively difficult to achieve with \tex as you will need to control the amount of text that you provide in the front page.


Vestibulum ut mollis odio. Vivamus ut risus eu dolor laoreet viverra. Nullam elit erat, congue at placerat ut, posuere non diam. Suspendisse eget dui et mi varius bibendum at non orci. Morbi justo arcu, posuere non tempus at, vestibulum sit amet lorem. Class aptent taciti sociosqu ad litora torquent per conubia nostra, per inceptos himenaeos. Donec tempor dignissim tellus, vitae vestibulum tellus hendrerit tempus. Nullam varius justo sit amet risus semper non semper eros placerat. Integer eleifend ligula in est gravida ornare tincidunt velit tristique.


Donec vel erat a ipsum condimentum volutpat vel non odio. Vivamus non justo orci. Pellentesque ligula ipsum, vestibulum at molestie vel, mollis sed odio. Donec rhoncus, sem in auctor tincidunt, libero quam scelerisque urna, et volutpat purus magna ac nulla. Cras vel quam nec urna viverra ornare eu et nibh. Pellentesque tincidunt leo non odio varius vitae sollicitudin neque adipiscing. 

\section{Full Page Images}

\leftskip0pt\parindent1em

In euismod, enim a dictum pharetra, libero nibh tempor enim, vel fermentum justo justo eget sem. Integer convallis massa nec turpis volutpat tristique. Quisque fringilla volutpat sem porta elementum. Donec vel metus quis nisl venenatis vehicula ac quis est. Maecenas vulputate lacinia lacus quis porttitor. Aliquam consectetur consectetur metus eu bibendum. Lorem ipsum dolor sit amet, consectetur adipiscing elit. In sem mauris, mollis nec pulvinar posuere, facilisis quis turpis. Quisque vel laoreet mauris.

\subsection{Images that are painted fully on the page}

Many opening styles required that an image is typeset fully on the page with no margins at all. Although
it is possible to place such images using skips or by simply adjusting the geometry of the page, it is much
easier to achieve it using the \texttt{remember picture, overlay} settings of a TikZ picture environment.

\begin{teX}
\newpage
\mbox{}
\begin{tikzpicture}[remember picture,overlay]
% draw image
\node[inner sep=0] at (current page.center)
{\includegraphics[width=\paperwidth,height=\paperheight]{./images/napoleon}};
\end{tikzpicture}

\newpage
\end{teX}

Note that the image must have proportions that suit the aspect ratio of the page geometry. If you are using
LuaLaTeX or another pdf capable engine, you should use \CMDI{\pdfpagewidth} and \CMDI{\pdfpageheight}. If
the image is not fitting properly it will be cropped by the pdf driver. 


\noindent\includegraphics[width=\textwidth]{./images/napoleon.jpg}

\newpage
\begin{tikzpicture}[remember picture,overlay]
% draw image
\node[inner sep=0] at (current page.center)
{\includegraphics[width=\paperwidth,height=\paperheight]{./images/napoleon}};
\end{tikzpicture}

\newpage

\newenvironment{kathleen}[1][b]{\def\placement{#1}\parindent0pt
}{}

\cxset{kathleen align/.is choice,
       kathleen align/top/.code=\xdef\kathleenplacement@cx{t},
       kathleen align/bottom/.code=\xdef\kathleenplacement@cx{b},
       kathleen align/center/.code=\xdef\kathleenplacement@cx{c},
       kathleen imagei/.code=\def\imagei{\includegraphics[width=\textwidth]{#1}\par},
 kathleen imageii/.code=\def\imageii{\includegraphics[width=\textwidth]{#1}\par},
kathleen imageiii/.code=\def\imageiii{\includegraphics[width=\textwidth]{#1}\par},
kathleen imageiv/.code=\def\imageiv{\includegraphics[width=\textwidth]{#1}\par},
kathleen imagev/.code=\def\imagev{\includegraphics[width=\textwidth]{#1}\par},
kathleen captioni/.code=\def\captioni{\captionof{figure}{#1}},
kathleen captionii/.code=\def\captionii{\captionof{figure}{#1}},
kathleen captioniii/.code=\def\captioniii{\captionof{figure}{#1}},
kathleen scale/.store in=\kathleenscale@cx
}

\long\def\printkathleen{\begin{kathleen}[t]
\begin{minipage}{\kathleenscale@cx\textwidth}
\begin{minipage}[\kathleenplacement@cx]{0.3\textwidth}
\vbox{}
\imagei
\captioni
\imageii
\captionii
\imageiii
\captioniii
\end{minipage}\hspace{1cm}
\begin{minipage}[\kathleenplacement@cx]{0.46\textwidth}
\vbox{}
\imageiv
\captionof{figure}{This is a short caption test and this one is a long caption test.}\par
\imagev
\captionof{figure}{This is a short caption test and this one is a long caption test.}
\end{minipage}
\end{minipage}
\end{kathleen}}

\begin{figure}
\cxset{kathleen align = top,
       kathleen imagei = {./images/ladyagnew.png},
       kathleen imageii = {./images/etta.png},
       kathleen imageiii = {./images/etta.png},
       kathleen imageiv = {./images/ladyagnew.png},
       kathleen imagev  = {./images/etta.png},
       kathleen captioni = {Al contrario di quanto si pensi, Lorem Ipsum non \`e semplicemente una sequenza casuale di caratteri. Risale ad un classico della letteratura latina del 45 AC.}, 
       kathleen captionii = {Finibus Bonorum et Malorum di Cicerone. Questo testo un trattato su teorie di etica, molto popolare nel Rinascimento. La prima riga del Lorem Ipsum.},
       kathleen captioniii= This is a short caption.,
       kathleen scale = 1.1,
} 

\printkathleen

\caption{The Kathleen template page. It consists of five images and their caption text. Parameters can be set via a key value interface.}
\end{figure}
\clearpage

\cxset{kathleen align = top,
       kathleen imagei = {./images/ladyagnew.png},
       kathleen imageii = {./images/etta.png},
       kathleen imageiii = {./images/etta.png},
       kathleen imageiv = {./images/ladyagnew.png},
       kathleen imagev  = {./images/etta.png},
       kathleen captioni = {Al contrario di quanto si pensi, Lorem Ipsum non \`e semplicemente.}, 
       kathleen captionii = {Finibus Bonorum et Malorum di Cicerone. Questo testo  un trattato.},
       kathleen captioniii= This is a short caption.,
       kathleen scale = 0.7
} 

\cxset{kathleen align=bottom}




\begin{center}\printkathleen\par\label{kathleen}\end{center}

\newpage

\section{The Kathleen template} 

A lot of pages in image rich books have complicated settings for images.
These are difficult to manipulate and we provide here what we hope is
a better method. For example the Figure~\ref{kathleen} shows such a complex layout. This can be achieved by only filling in the template
values as shown below.

\begin{tcolorbox}
\begin{lstlisting}
\cxset{kathleen align = top,
       kathleen imagei = ladyagnew,
       kathleen imageii = etta,
       kathleen imageiii = etta,
       kathleen imageiv = ladyagnew,
       kathleen imagev  = etta,
       kathleen captioni = {Al contrario di quanto si pensi.}, 
       kathleen captionii = {Finibus Bonorum et Malorum di.},
       kathleen captioniii= This is a short caption.,} 
\cxset{kathleen align=bottom,
       kathleen scale=.5}

\printkathleen

\end{lstlisting}
\end{tcolorbox}


\newgeometry{top=0pt,left=1cm,right=1cm,marginparsep=0pt}

\clearpage


\parindent0pt
\pagestyle{empty}

\fboxsep0pt
\fboxrule0pt

\vspace*{-1cm}
\begin{minipage}{1.05\textwidth}
\hskip-0.9cm\includegraphics[width=1.03\textwidth]{./images/parasol-05.jpg}\\[-27.5pt]
\setlength{\linewidth}{0.95\textwidth}
\setlength{\columnsep}{10pt}
\begin{multicols}{2}
\noindent \footnotesize\textbf{DESIGNED FOR CONTRAST} with the wearer's ensemble, these plaid  tafetta and green rayon parasols, are best sellers at Maey's in New York. Set of matching parasol and shoes, or
gloves, scarves or bags, are also available to give simple dresses
a custom appearance.
\end{multicols}
\vspace{-0.25cm}
\rule{1.5cm}{0pt}\fbox{
\begin{minipage}[t]{0.87\textwidth}
\begin{minipage}[t]{0.41\textwidth}
\includegraphics[width=1.03\textwidth]{./images/parasol-06.jpg}\par%
\noindent \footnotesize\textbf{CHERRY ORNAMENTS} adorn handle and tip of this parasol, made by Jane Derby to go with the afternoon dress. Straight handles are very popular.
\end{minipage}\hspace{0.5cm}
\begin{minipage}[t]{0.4\textwidth}
   \includegraphics[width=1\textwidth]{./images/parasol-07.jpg}\par
\noindent \footnotesize\textbf{MATCHING SETS} of afternoon dress
and parasol, and four-piece polka dot weekend dress and parasol,
both designed by Briganne.
\end{minipage}
\end{minipage}
}

\vfill

\captionof{figure}{Balancing three images on a page. Should the larger image be at the top or at the bottom?}
\end{minipage}




\begin{minipage}{\textwidth}
\begin{minipage}[b][\textheight][b]{.47\linewidth}
\vspace*{2cm}

\includegraphics[width=\linewidth]{./images/parasol-03.jpg}\par
\vspace{2\baselineskip}

\centerline{\bfseries\Huge Parasols}
\vspace{2\baselineskip}

\begin{quote}
\lipsum[2]
\end{quote}

\vfill

\textbf{SHOES AND PARASOL SET} in pink are here combined with a dress, one of whose skirts is pink. Parasol is from New York's ``Uncle Sam'' parasol shop.
\end{minipage}\hspace*{1cm}
\begin{minipage}[b]{.53\linewidth}
\mbox{}
\includegraphics[width=\linewidth]{./images/parasol-01.jpg}\par
\end{minipage}
\end{minipage}

\newgeometry{top=1.5cm,bottom=3cm,left=3.5cm,right=3.5cm}

\clearpage