\documentclass[pagestyleset=scrheadings]{scrbook}
\usepackage{scrlayer-scrpage}
\usepackage{phd,afterpage}
\usepackage{hyperref}
\begin{document}
\makeatletter
\def \newpage {%
\everypar{AAA\hsize13cm
  \everypar{BBB \everypar{\hsize10cm\everypar{CCC\hsize14.5cm}}}
}
 \if@noskipsec
 \ifx \@nodocument\relax
 \leavevmode
 \global \@noskipsecfalse
 \fi
 \fi
 \if@inlabel
 \leavevmode
 \global \@inlabelfalse
 \fi
 \if@nobreak \@nobreakfalse \fi
 \par
 \vfil
 \penalty -\@M}

\newcommand\photo[2][]{%
\bgroup
\pagecolor{black!20}

\gdef\chapstory{\hspace*{-11cm}\hbox to 0pt{\colorbox{black!70}{\rightskip10pt \parbox{9cm}{%
\vskip15pt

\color{black!20}
\Huge\bfseries\RaggedLeft \chaptername@cx~13\\[30pt]
Photography\\
Layouts\\
and\\
LaTeX\\[15pt]\par
}\hskip15pt}}}
\newbox\pic
\setbox\pic=\hbox{%
  \includegraphics[height=\paperheight]{./images/amato-01.jpg}
}
\begin{tikzpicture}[help lines]
\path (0,0) rectangle (\paperwidth,\paperheight);
\node[scope fading=west,inner sep=0pt,outer sep=0pt,anchor=north east, opacity=1] at(\paperwidth,\paperheight) {\box\pic};
\node at ++(-2,23)[text=thegray] {\chapstory};
\end{tikzpicture}
\egroup
\pagestyle{empty}%
%\afterpage{\pagecolor{white}}%


\mbox{.}
 }
 
\makeatletter\@specialtrue
\setdefaults

\cxset{custom=photo,
          chaptername=\chaptername,
          chapter opening=any}

\chapter{Test}{}
\lorem

\newpage


\begin{abstract}
Unlike 1he Trans-Adanlic Slave Trade the transportation of slaves from
Mrica to Asia and the Mediterranean was of great araliquily, bul the
intense historical interest in the Trans-Atlantic Trade for the past two
hundred years has overshadowed the stud)• of the Asian slave trade
which, unlil 1his past clec::ade, has been largely ignored despite 1he fac1
that the total number of Mricans exported to Asia was spread out ovc:r
thousand years (belween 800 AD and 1900 AD) bul has been estimated
at approximate!)• lhe same as the number of Mricans sent 10 the Americas
in four and a half centuries i.e. 12,580,000. This paper describes the
Mrican slave trade to Asia across the Sahara Desert, ovc:r the lled Sea,
and from lhe coast of East Mrica, and how this trade was conducted
in each of lhese regions. HiSIOf)' is not a social science, but a member
of the humanities famil)'. It is the search of every available source using
any discipline to narrate a story and n01 bound by any rigid theoretical
or methodological concepts. In the compilation or this essay, I have
employed 1he latest information and interpretations on 1he Mrican sla\'C!
trade lo Asia to wrile the history of that institution as lO what happened,
where, when, how and why.
\end{abstract}

\lipsum
\newpage

\pagecolor{white}
\cxset{section color=spot!50}
\renewsection
\everypar{}

In order to understand the complexities of programming TeX (I am using the word TeX to denote all its related engines and formats)  we need to keep in mind that people have some thousand of years of accumulated tradition about how documents should be printed to satisfy esthetic and utilitarian criteria.  With that much of history, there is a rich collection of examples, and people are just as prolific in inventing new styles now as they ever were, so there seems to be no limit to the expectations for document preparation systems. What defines a document is also rapidly evolving.

Furthermore, every user of LaTeX is an “expert” in what his or her document should look like. As the output is visible, it can readily be compared  against some physical or mental image. To an occassional user this immediate feedback, is different to the reaction of an occassional user to a new programming language, for example Haskel, its monads and other concepts that might be unfamiliar with. 

A basic misconception as to what TeX is stems from the fact that most users of LaTeX do not spend an adequate amount of time to understand the requirements for a Typographical Engine and the solution that TeX solved using, glue penalties and boxes. To any programmer the 250 basic commands of TeX and at least 60 parameters would make no sense, until they spent the necessary time to understand what typography is and TeX’s basic glue model.

\section{User Programming Interfaces: What does the user have to face?}

One of the rivals of TeX as a typesetting engine---and perhaps its only only rival---is Adobe’s InDesign, which of course comes with a Graphical User Interface and bells and whistles, a price tag and no guarantees for backward compatibilities. Scripting it is another story. Here is an extract in JavaScript. These settings take more than two pages in the example and I am just providing a fragment of the code to give you a taste of what is involved.

\begin{verbatim}
var myDocument = app.documents.item(0);
var myPage = myDocument.pages.item(0);
var myTextFrame = myPage.textFrames.add();
myTextFrame.contents = "x";
var myTextObject = myTextFrame.parentStory.characters.item(0);
myTextObject.alignToBaseline = false;
myTextObject.appliedCharacterStyle = myDocument.characterStyles.item("[None]");
myTextObject.appliedFont = app.fonts.item("Minion ProRegular");
myTextObject.appliedLanguage = app.languagesWithVendors.item("English: USA");
.
.
.
myTextObject.capitalization = Capitalization.normal;
myTextObject.composer = "Adobe Paragraph Composer";
myTextObject.desiredGlyphScaling = 100;
myTextObject.desiredLetterSpacing = 0;
myTextObject.desiredWordSpacing = 100;
myTextObject.dropCapCharacters = 0;
\end{verbatim}

If typing:

\begin{verbatim}
myTextObject.appliedFont = app.fonts.item("Minion ProRegular");
\setmainfont{Minion Pro Regular}
\end{verbatim}

Complexities HarfBuzz, ICU Engine, 

\section{Document Settings}

One has to differentiate about settings and programming. For example a setting can modify a documents main font, where programming LaTeX to .


\section{Mark-up Language}

Having your document as a User Interface presents difficulties. It needs to be semantically marked to make it easy to write and edit and also for other computer programs to parse it, classify it and perhaps transform it into other presentation formats.

\section{Presentation Language}

Presentation languages, themeselves need in my opinion to be Turing complete and as sucth TeX is a good proposition or a TeX-like language. 

\section{Scripting or Writing Extensions}

The OPs question dealt mostly with this aspect, and the valid in many respects criticism picked up on the name and synatx on the more esoteric commands of TeX’s language. For example @makeatletter and @makeatother.  Of course (since you are writing in  a document) you need to tell the Typesetter that some of the letters in your program are special. This is not unusual in computer languages. 

Here TeX’s way of saying catcode`@=11 or catcode`@=12 is perhaps a better choice? or you can mould it into begin{script} end{script}? I am not sure. One when learning to program tends to adopt to the idiosyncrancies of the computer language rather the other way around. Programming in Haskell for example is much more different than C, and even the machine you are programming in can leave traces of ...



\section{Current Needs}

Before delving into the topic deeper, I would like to suggest that the “persona” of a LaTeX user needs to be defined. Who are the likely users? Of course suggestions in other posts that your grandmother cannot use it, or the typical user needs to open it and start typing are irrelevant, as in my opinion the typical “persona” of a LaTeX user is a person with a good background with a University Degree or equivalent experinece or a student studying for higher degrees, computer scientists and mathematicians. 

As I mentioned earlier, I do not think the syntax is a hindrance, but some of the limitations of TeX are showing. 

-- Improvements in mark-up commands for ...
-- Tables
-- Graphics
-- Output routines
-- Flowing text from one text block to another
-- Pluggable models for the page builder etc..

As a final note the OP asked does it have to be like this? If one could design a typesetting system from scratch, in 2015, couldn’t we make something much simpler and more intuitive? 


 
\end{document} 

http://tex.stackexchange.com/questions/147180/how-to-add-a-gradient-fade-out-effect-to-an-image

http://wwwimages.adobe.com/content/dam/Adobe/en/devnet/indesign/sdk/cs6/scripting/InDesign_ScriptingGuide_JS.pdf















