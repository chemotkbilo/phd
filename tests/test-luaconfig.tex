%!TEX TS-program = lualatex
\documentclass[A3]{article}
\usepackage{luatexbase,filecontents}
\usepackage{luacode}
\begin{filecontents*}{test.config}
# test.config
# Read timeout in seconds
read.timeout=10
# Write timeout in seconds
write.timeout=5
#acceptable ports
ports = 1002,1003,1004
\end{filecontents*}
\begin{document}

We first write a small configuration file using the filecontents package and then
we load the config package of penlight
\begin{luacode}
  -- readconfig.lua
  
local config       = require 'pl.config'
local t              = config.read 'test.config'

tex.print(t.write_timeout)
\end{luacode}



\end{document}