\documentclass[]{article}
\pagestyle{empty}
\setlength\textwidth{352.81416pt}
\setlength\parindent{0pt}
%StartShownPreambleCommands
\usepackage{luacode}
\usepackage{listings}
\usepackage{fontspec}
\setmonofont[Scale=0.81]{DejaVu Sans Mono}
%StopShownPreambleCommands
\begin{document}
\begin{luacode*}
    local http=require("socket.http")
    local page = http.request( 'http://jabref.sourceforge.net/journals/journal_abbreviations_general.txt' )
    tex.print("\\begin{lstlisting}[basicstyle=\\ttfamily\\scriptsize,")
    tex.print("language={[5.2]Lua}]")
    tex.print(page:gsub("\n","\r"))
    tex.print("\\command \\end{lstlisting}")
    
 local socket = require( "socket" )

-- Connect to the client
local client = socket.connect( "www.apple.com", 80 )
-- Get IP and port from client
local ip, port = client:getsockname()

-- Print the IP address and port to the terminal
tex.print( "IP Address:", ip )
tex.print( "Port:", port )   


-- Requests information about a document, without downloading it.
-- Useful, for example, if you want to display a download gauge and need
-- to know the size of the document in advance
r, c, h = http.request {
  method = "HEAD",
  url = "http://www.tecgraf.puc-rio.br/~diego"
}
-- r is 1, c is 200, and h would return the following headers:
-- h = {
--   date = "Tue, 18 Sep 2001 20:42:21 GMT",
--   server = "Apache/1.3.12 (Unix)  (Red Hat/Linux)",
--   ["last-modified"] = "Wed, 05 Sep 2001 06:11:20 GMT",
--   ["content-length"] = 15652,
--   ["connection"] = "close",
--   ["content-Type"] = "text/html"
-- }

tex.print(h.server, c)



\end{luacode*}
\end{document}