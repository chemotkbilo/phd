\documentclass[oneside, a4paper,12pt]{ltxdoc}
\usepackage{geometry}
\usepackage{hyperref}
\usepackage{phd}
\tcbuselibrary{documentation}
 
\makeatletter\makeatother

%to break urls
\expandafter\def\expandafter\UrlBreaks\expandafter{\UrlBreaks%  save the current one
  \do\a\do\b\do\c\do\d\do\e\do\f\do\g\do\h\do\i\do\j%
  \do\k\do\l\do\m\do\n\do\o\do\p\do\q\do\r\do\s\do\t%
  \do\u\do\v\do\w\do\x\do\y\do\z\do\A\do\B\do\C\do\D%
  \do\E\do\F\do\G\do\H\do\I\do\J\do\K\do\L\do\M\do\N%
  \do\O\do\P\do\Q\do\R\do\S\do\T\do\U\do\V\do\W\do\X%
  \do\Y\do\Z}
\Urlmuskip = 2mu plus 1mu 
\input{defaultstyle}

\definecolor{niagra}{rgb}{0.7,0.027,0.22}
\definecolor{tiber}{rgb}{0.0039,0.09,0.27}

\cxset{%
 chapter color=tiber,
 title font-color=tiber,
 section afterindent = false, 
 section color= tiber,     
 section beforeskip=15pt,
 section afterskip=15pt,
 section indent=-3cm,
 section font-family= sffamily,
 section font-size= LARGE,
 section font-weight= bfseries,
 section font-shape=,
 section align= left,
 section numbering prefix =\thechapter.,
 section numbering= arabic,
 section spaceout=soul,
  section numbering suffix=,
 }
\cxset{title font-color=tiber,
chapter rule color=thegray}
\makeatletter
\begingroup \catcode `|=0 \catcode `[= 1
\catcode`]=2 \catcode `\{=12 \catcode `\}=12
\catcode`\\=12 |gdef|@xsmallverbatim#1\end{smallverbatim}[#1|end[smallverbatim]]
|gdef|@sxsmallverbatim#1\end{smallverbatim*}[#1|end[smallverbatim*]]
|endgroup
\def\@smallverbatim{\trivlist \item\relax
  \if@minipage\else\vskip\parskip\fi
  \leftskip\@totalleftmargin\rightskip\z@skip
  \parindent\z@\parfillskip\@flushglue\parskip\z@skip
  \@@par
  \@tempswafalse
  \def\par{%
    \if@tempswa
      \leavevmode \null \@@par\penalty\interlinepenalty
    \else
      \@tempswatrue
      \ifhmode\@@par\penalty\interlinepenalty\fi
    \fi}%
  \let\do\@makeother \dospecials
  \obeylines \smallverbatim@font \@noligs
  \hyphenchar\font\m@ne
  \everypar \expandafter{\the\everypar \unpenalty}%
}
\def\smallverbatim{\@smallverbatim \frenchspacing\@vobeyspaces \@xsmallverbatim}
\def\endsmallverbatim{\if@newlist \leavevmode\fi\endtrivlist}

\def\smallverbatim@font{\normalfont\smallverbatimsize\ttfamily}


\def\@sect#1#2#3#4#5#6[#7]#8{%
  \ifnum #2>\c@secnumdepth
   \let\@svsec\@empty
  \else
  \refstepcounter{#1}%
  \protected@edef\@svsec{\@seccntformat{#1}\relax}%
 \fi
 \@tempskipa #5\relax
 \ifdim \@tempskipa>\z@
 \begingroup
#6{%
 \@hangfrom{\hskip #3\relax\@svsec}%
 \interlinepenalty \@M #8\@@par}%
 \endgroup
 \csname #1mark\endcsname{#7...}%
 \addcontentsline{toc}{#1}{%
 \ifnum #2>\c@secnumdepth \else
   \protect\numberline{\csname the#1\endcsname}%
 \fi
 #7}%
 \else
 \def\@svsechd{%
 #6{\hskip #3\relax
  \@svsec #8}%
 \csname #1mark\endcsname{#7}%
 \addcontentsline{toc}{#1}{%
 \ifnum #2>\c@secnumdepth \else
      \protect\numberline{\csname the#1\endcsname}%
 \fi
 #7}}%
 \fi 
 \@xsect{#5}}
 
  
\cxset{geometry/.is choice,
          geometry/elements/.code = \geometry{ heightrounded,
          includeheadfoot=false,
          textwidth=      251pt,
          textheight=     502pt,
          paperwidth=     374pt,
          paperheight=    648pt,
          vmarginratio=   1:2,
          marginparwidth= 60pt,
          marginparsep=   18pt,
          outer=          90pt}}   
          
\cxset{geometry/aureo/.code=\geometry{ heightrounded,
         includeheadfoot=true, 
         textwidth=      120mm,
         textheight=     194mm,
         paperwidth=     17cm,
         paperheight=    24cm,
         marginratio=    2:3,
         marginparwidth= 62pt,
         marginparsep=   10pt}}            
    
\cxset{geometry/compact aureo/.code= \geometry{
    heightrounded,
    includeheadfoot=false,
    textheight=     175mm,
    textwidth=      108mm,
    paperwidth=     140mm,
    paperheight=    210mm,
    marginratio=    1:1,
    marginparwidth= 11mm,
    marginparsep=   7pt}
}    

\cxset{geometry/super compact/.code= 
\geometry{
    heightrounded,
    includeheadfoot=false,
    textheight=     175mm,
    textwidth=      108mm,
    paperwidth=     140mm,
    paperheight=    210mm,
    marginratio=    1:1,
    marginparwidth= 11mm,
    marginparsep=   7pt}    
}
 
\cxset{geometry/standard/.code=                   
 \geometry{%standard
    heightrounded,
    a4paper,
    includeheadfoot=true,
    textwidth=      110mm,
    textheight=     220mm,
    marginratio=    1:2,
    marginparwidth= 30mm,
    marginparsep=   12pt}}   
    
\cxset{geometry/periodical/.code=                  
 \geometry{%periodical
    heightrounded,
    includeheadfoot=false,
    textheight=     165mm,
    textwidth=      110mm,
    paperwidth=     170mm,
    paperheight=    240mm,
    marginratio=    2:3,
    marginparwidth= 26mm,
    marginparsep=   10pt}}   
    
\cxset{geometry/periodical aureo/.code= \geometry{%periodical aureo
    heightrounded,
    includeheadfoot=true,
    textwidth=      120mm,
    textheight=     194mm,
    paperwidth=     17cm,
    paperheight=    24cm,
    marginratio=    2:3,
    marginparwidth= 62pt,
    marginparsep=   10pt}}
    
\cxset{geometry/tufte a4paper/.code=    
      \geometry{a4paper,heightrounded,
        left=24.8mm,top=27.4mm,headsep=2\baselineskip,textwidth=107mm,
       marginparsep=8.2mm,marginparwidth=49.4mm,
       textheight=661.5pt,headheight=\baselineskip,
       }}  %error on asymetr
       
\cxset{geometry/tufte letterpaper/.code=\geometry{%  
        heightrounded,     
        letterpaper,left=1in,top=1in,headsep=2\baselineskip,textwidth=26pc,marginparsep=2pc,
       marginparwidth=12pc,textheight=594pt,
       headheight=\baselineskip\relax,reversemp=true,
       twoside=false
       }}         
%  

\cxset{newgeometry/.is choice,
        newgeometry/tufte letterpaper/.code=\newgeometry{%  
        heightrounded,     
        letterpaper,
        left=1in,
        top=1in,
        headsep=2\baselineskip,textwidth=26pc,marginparsep=2pc,
       marginparwidth=12pc,textheight=594pt,
       headheight=\baselineskip\relax,
       reversemp=false,
       twoside=true}}  



%\cxset{geometry=standard}    

\makeatother
\cxset{geometry=standard}    
\begin{document}

%               \makeatletter
%              \@twosidetrue\@mparswitchfalse

                    
%\tableofcontents
\cxset{mainmatter numbering=arabic}
\def\innermarginname{Inner}
\mainmatter

%\lipsum[1-10]
%
%\cxset{geometry units=mm}
%\pagestyle{grid}

\input{./sections/marginpar}
\input{./sections/pages}
\makeatother

\chapter{Introduction}

\cxset{section indent=1sp}
\cxset{section afterskip=20pt plus1pt minus 1pt}
\cxset{section numbering=none}
\cxset{section align=centering}
\cxset{section afterindent=true}
\cxset{section color=black!60}
\cxset{section align=RaggedRight}

\section*{Tufte and Page Layouts.}



\section*{\language-1 This is some very long heading to see if we can manipulate it, even if the hang from form is present that complicates things quite a bit.}

Test\marginpar[]{\footnotesize Some testing \lorem.}
\lorem



\printreadability



\the\baselineskip

\the\normalbaselineskip

\lipsum[1-5]\marginnote{\lorem}
\lipsum[1-5]\marginnote{\lorem}



\end{document}
\makeatletter
% For style 22 need 

\def\ps@verticalrule{%
    \leftskip\z@\let\@mkboth\@gobbletwo\vfuzz=5\p@
    \def\@oddhead{}%
    \def\@evenhead{}%
     \def\@evenfoot{}%
      \def\@oddfoot{}%
  \def\@oddhead{\verbatimsize
    \vbox to 0pt{\vspace{\the\headsep}%
      \noindent\hbox to \dimexpr\the\textwidth+.75cm\relax{%
            \hfill\mbox{%
                \color{thegray}\rule{1pt}{\dimexpr\the\textheight-\footnotesep\relax}\hspace*{1pt}\color{black}\thepage%
                }%
      \makebox[\z@][l]{\@c@pyrightline}%
%     \noindent\hspace*{-9pc}\rule{37pc}{0.25pt}%
    }}%
  }%
 
  \def\sectionmark##1{}%
  \def\subsectionmark##1{}%
 }
\makeatother

 \pagestyle{verticalrule} 
 
\input{./styles/style22}

\end{document}

\cxset{chapter opening=any}
\chapter{Introduction}

The development of a style template for a statistics book, following the style of Statistics for Engineers and Scientists.
The book can easily be developed by the \pkgname{phd} package.  Remember that the best way to start is to use an existing template and to modify the parameters.

\begin{docEnvironment}[doclang/environment content=key description,doc updated=2015-01-29]{docKey}{\oarg{key path}\oarg{options}\marg{name}\marg{parameters}\marg{description}}
  Documents a key with given \meta{name} and an optional \meta{key path}.
  The given \meta{options} are set with \refCom{tcbset}.
  This key takes mandatory or optional \meta{parameters} as value
  with a short \meta{description}.
  It is automatically indexed and can be referenced with
  \refCom{refKey}\marg{name}.
\begin{dispExample}
\begin{docKey}[foo]{footitle}{=\meta{text}}{no default, initially empty}
  Creates a heading line with \meta{text} as content.
\end{docKey}
\end{dispExample}
\end{docEnvironment}


\begin{docCommand}{docValue}{\marg{name}}
  Documents a value with given \meta{name}. Typically, this is a value for a key.
  This value is automatically indexed.
\begin{dispExample}
A feasible value for \refKey{/foo/footitle} is \docValue{foovalue}.
\end{dispExample}
\end{docCommand}

\begin{docCommand}[doc new=2015-01-08]{docCounter}{\marg{c@chapter}}
  Documents a counter with given \meta{name}. The counter is automatically indexed.
\begin{dispExample}
The counter \docCounter{foocounter} can be used for computation.
\end{dispExample}
\end{docCommand}


\begin{figure}[htbp]
\centering
\includegraphics[width=0.8\textwidth]{./images/statistics.jpg}
\end{figure}

I am not too very sure if the choice of color and sky blue for the sections is such a good combination, so we will try both and see which one you like.

\section{Set the Sections} 
 \arial

 The sections can easily be set using the package. We need to define some colors, the font parameters and the
 indentation into the margins. We will revisit some of the settings but for the time being the following settings will do.
 
 \begin{teX}
 \cxset{%
   section afterindent = false, 
   section color= sweet,     
   section beforeskip=15pt,
   section afterskip=15pt,
   section indent=-3cm,
   section font-family= sffamily,
   section font-size= LARGE,
   section font-weight= bfseries,
   section font-shape=,
   section align= left,
   section numbering prefix =\color{black!55}\thechapter.,
   section numbering= arabic,
   section spaceout=soul,
}
 \end{teX}



 \cxset{% 
  section numbering prefix =\thechapter.,}

\section{Examples and Solutions}

\newcounter{example}[chapter]

Obviously any book addressed to a student audience will have special sections for examples, solutions problems and answers and this was the main reason why I have introduced this particular style early in the documentation.

The examples are framed with a `T’ shaped ruler and are numbered in arabic numerals, with the chapter prefix. they are reset at every chapter.
\newlength\rulerlength
\cxset{rule color=sweet}
\color{black!85}
\makeatletter
\def\example{%
\stepcounter{example}
\settowidth{\rulerlength}{\textbf{Example 10.1}}
\addtolength{\rulerlength}{1cm}
\parindent0pt
\bgroup

\color{sweet}
\parindent-\rulerlength\relax
\leavevmode\par
\rule{\dimexpr(2\rulerlength)}{1.5pt}%
\vskip-12pt
\moveright-4pt\vbox to 0pt{\llap{\sffamily\textbf{Example \thechapter.\theexample :} \hbox to 0pt{\color{sweet}\rule{1.5pt}{\baselineskip}}}}
\egroup%
\everypar{\parindent2em}
}%

\example Suppose that an engineer encounters data from a manufacturing process in which
100 items are sampled and 10 are found to be defective. It is expected and anticipated
that occasionally there will be defective items. Obviously these 100 items
represent the sample. However, it has been determined that in the long run, the
company can only tolerate 5\% defective in the process. Now, the elements of probability
allow the engineer to determine how conclusive the sample information is
regarding the nature of the process. In this case, the population conceptually
represents all possible items from the process. Suppose we learn that if the process
is acceptable, that is, if it does produce items no more than 5\% of which are defective,
there is a probability of 0.0282 of obtaining 10 or more defective items in
a random sample of 100 items from the process. This small probability suggests
that the process does, indeed, have a long-run rate of defective items that exceeds
5\%. In other words, under the condition of an acceptable process, the sample information
obtained would rarely occur. However, it did occur! Clearly, though, it
would occur with a much higher probability if the process defective rate exceeded
5\% by a significant amount.

From this example it becomes clear that the elements of probability aid in the
translation of sample information into something conclusive or inconclusive about
the scientific system. In fact, what was learned likely is alarming information to
the engineer or manager. Statistical methods, which we will actually detail in
Chapter 10, produced a P-value of 0.0282. The result suggests that the process
very likely is not acceptable. The concept of a P-value is dealt with at length
in succeeding chapters. The example that follows provides a second illustration.

\def\solution{%
   \everypar{}
   \parindent0pt
  \leavevmode\par
  \makebox{\llap{\bfseries\textit{Solution }:}\thinspace}%
  \parindent2em
  }
 
\solution Let $X$ represent the number of good components in the sample. The probability
distribution of $X$ is\par

\lipsum[1]\par

\section{Figures}

Since the book is a mathematical textbook, it does not contain too many pictures. Captions are styled easily
with normal phd settings---as a matter of fact---the default style is ok and the looks of the page are
improved by the usage of the same sky blue color as the headings.

\begin{figure}[htbp]
\centering
\includegraphics[width=0.8\textwidth]{./images/statistics-figures.jpg}
\caption{Figures are styled after common LaTeX styles and do not need any particular treatment.}
\end{figure}

\section{Tables}

Tables are always very visual and book designers tend to spend considerable artistic effort to style them. This particular book style, just styles the captions at the primary color of the book (sky blue). Smaller tables are set
in black, including the heading.

The tables in the Appendix all have nice blue captions and some of them have also a small figure at the edge. 
Of course any TeXnician would generate these tables, nowdays using Lua. The figures can be absolute positioned
using pgf or similar methods.

\begin{figure}[htbp]
\example Reproduce the pages shown below to cater for figures.\par
\vskip\abovedisplayskip

\centering
\includegraphics[width=0.8\textwidth]{./images/statistics-tables.jpg}
\caption{Figures are styled after common LaTeX styles and do not need any particular treatment.}
\end{figure}

\lipsum[1-3]
\def\doublerule{%
\toprule
\addlinespace[0pt]
\midrule}

\begin{table}[htbp]
\example Develop a table with a double rule on top of the table head.\par
\centering 
\caption{Absorption of Moisture in Concrete Aggregates}
\label{tbl:stats}
\begin{tabular}{rrrrrr}
\doublerule
Aggregate    &1  &2 &3 &4 &5\\
\midrule
                   &551 & 595 & 639 & 417 & 563\\
\bottomrule
\end{tabular}
\end{table}

\noindent In addition, we may be interested in making individual comparisons among these
5~population means

\solution The only item we needed to define in addition to standard macros provided is the \string\doublerule.
\begin{verbatim}
\begin{table}[htbp]
\centering 
\caption{Absorption of Moisture in Concrete Aggregates}
\label{tbl:stats}
\begin{tabular}{rrrrrr}
\doublerule
Aggregate    &1  &2 &3 &4 &5\\
\midrule
                   &551 & 595 & 639 & 417 & 563\\
\bottomrule
\end{tabular}
\end{table}
\end{verbatim}



\makeatletter\@specialfalse\makeatother

\cxset{
 %toc image =,
 chapter format = block,,
 chapter name=Chapter,
 chapter numbering=arabic,
% number font-size= LARGE,
% number font-family= rmfamily,
% number font-weight= bfseries,
% number before=,
% number dot=,
% number after=\par\offinterlineskip,
% number position=leftname,
% chapter font-family= sffamily,
% chapter font-weight= normalfont,
% chapter font-size= Large,
% chapter before={\vspace*{15pt}\par},
% chapter after=,
% number color=black!90,
 %
 chapter title margin-top=30pt,
% title margin bottom=20pt,
 chapter align=left,
 chapter title align=RaggedRight,
 chapter title width=\textwidth,
%
% chapter title before={},
 chapter title font-family= sffamily,
 chapter title color= black,
 chapter title font-weight= bfseries,
 chapter title font-size= LARGE,
% title spaceout=none,
 header style= plain,
 section font-size=Large,
 section color=black,
 section number prefix=\thechapter.,
 section indent=0pt,
 }


\chapter{Introduction to Style One}
\addcontentsline{toc}{section}{Template 1 (style01)}
\cxset{headings ruled-01}

\begin{summary}
This design is simple and its distinguishing characteristic is a short summary at the beginning of the chapter. This is almost like an abstract typeset in italic font without setting the margins in. We provide a \lstinline{summary} environment for convenience. Note the very simple line in the running head to the left of the page number.
\end{summary}

\medskip
\begin{figure}[ht]
\centering
\includegraphics[width=\textwidth]{./chapters/chapter01.jpg}
\end{figure}

The summary after the chapter head can easily be incorporated using \textit{summary}. You
can also use \string\begin\meta{abstract}. The latter will produce a heading with the word, abstract.
Both the summary as well as the abstract can take parameters to be set, for internationalization and to typeset
the words abstract, or summary. If you use \textit{precis}, the summary will be added into the Table of Contents as
well.

I originally picked this style, as a boringly easy style to develop, but it proved a hard nut to crack when it came to sections. Both the section numbering as well as the caption of figures, proved to be difficult to style using
the build-in \latexe commands. 

\section{Images and Figures}

Images and figures are using traditional captions with the exception they are restricted in a certain portion of the textwidth. The word Fig. is abbreviated to Fig. and uses a dot.

\captionsetup[figure]{format = plain,
                                 width=.67\textwidth,
                                 justification=justified,
                                 singlelinecheck=false,
                                 name=Fig.,
                                 labelsep=space,
                                 oneside,
                                 margin=0pt
                                 }

\begin{figure}[ht]
\includegraphics[width=\textwidth]{./images/elementary-images.jpg}
\caption{Note the abbreviation and the restriction of the caption to\\
 a minipage. This combined with the width option manages\\
  the typesetting well.}
\end{figure}

I had to make a special style to capture this. It is unbelievable that a piece of textblock can get so complicated and this particular style, let me to re-think some of the concepts in the phd design. the complication that arises here
is that with most images measuring the image is necessary. The narrower image in the figure would of course not work on the same settings, and the caption is at the full width of the figure, as shown below.













\makeatother

\clearpage
\makeatletter\@debugtrue\makeatother
\cxset{
% chapter toc=true,
 name=CHAPTER,
 chapter numbering=ORDINALS,
% number font-size=Large,
% number font-family=rmfamily,
% number font-weight=bfseries,
% number before=\kern0.5em,
% number dot=,
% number after=\hfill\hfill\par,
% number position=rightname,
% number color=black,
 chapter font-family=rmfamily,
 chapter font-weight=bold,
 chapter font-size=Large,
 chapter before={\vspace*{20pt}\par\hfill},
 chapter after=,
 chapter color=black,
 %
 title margin top=10pt,
% title before=\par\nointerlineskip\hfill,
% title after=\hfill\hfill\par\nointerlineskip,
 chapter title font-family=rmfamily,
 chapter title font-color= black,
 chapter title font-weight=bfseries,
 chapter title font-size=LARGE,
 chapter title width=0.8\textwidth,
 chapter title align=centering,
% title margin-left=0pt,
 author block=false}

\debugtitle
\debugchapter
\chapter[Template 2]{Mondino, the Restorer of Anatomy}

The archive.org is an extraordinary hunting ground  for typographical surprises. On a recent excursion to find some books on Versalius I stubled on a book titled \emph{Andreas Vesalius, the reformer of anatomy} by  Ball, James Moores. It is an old book published in 1910 and has a couple of unusual features. Check the figure below and see if you can identify the challenging feature.

\begin{figure}[ht]
\centering
\includegraphics[width=0.8\textwidth]{versalius}
\caption{J.B. Moore’s \emph{Andreas Versalius, the Reformer of Anatomy} has many unusual features, including chapter numbers using ordinals. }
\end{figure}

\cxset{chapter toc=true,
          chapter opening=anywhere}
          
\chapter{The Template}          
The template is called \emph{Versalius} and is stored under style02. It can be loaded in the normal way using:
\begin{verbatim}
\usepackage[style02]{phd}
\end{verbatim}

I have not reproduced the full extend of the book’s requirements, as some details are quite cumbersome to be automated through \tex. These though can easily be incorporated in a manual way. More about this later.


\section{The Table of Contents}
Another interesting aspect of this book, which is common with many books of its period is the ToC. The ToC shows the full range of the chapter pages, i.e., it is marked as Page 1-16 rather than the common practice nowdays that indicates only the starting page of the chapter. It also has “TABLE OF CONTENTS”  as a heading and not just contents as you would expect from today’s books.

\begin{figure}[ht]
\centering
\includegraphics[width=0.8\textwidth]{versalius-01}
\caption{J.B. Moore’s \emph{Andreas Versalius, the Reformer of Anatomy} has many unusual features, including chapter numbers using ordinals. }
\end{figure}

\section{List of Illustrations}

\begin{figure}[ht]
\centering
\includegraphics[width=0.8\textwidth]{versalius-02}
\caption{J.B. Moore’s \emph{Andreas Versalius, the Reformer of Anatomy} has many unusual features, including chapter numbers using ordinals. }
\end{figure}

\section{The Frontmatter}
As a foreward there is an unumbered chapter called ``Introduction’’. The chapter heading also has a head band.
\begin{figure}[ht]
\centering
\includegraphics[width=0.8\textwidth]{versalius-03}
\caption{J.B. Moore’s \emph{Andreas Versalius, the Reformer of Anatomy} has many unusual features, including chapter numbers using ordinals. }
\label{lettrine}
\end{figure}

\bgroup
\centering
\includegraphics[width=0.7\textwidth]{versalius-headband}

\LARGE\bfseries INTRODUCTION\par
\egroup
\def\dropcapversalius{%
\vbox to 0pt{\vskip6pt\leavevmode\noindent\includegraphics[width=2.39cm]{versalius-dropcap}%
}%
}
\parindent0pt

\hangindent2.6cm \hangafter0
\dropcapversalius \textsc{he dropcap will have to be inserted}, either using the lettrine package or do be achieved via a parshape command and manual entry. You can also write your own macro command using the details we provide under the Paragraphs chapter. On this page I have manually inserted it, as I used an image from the book for the dropcap. If you were to use the template for a full book, it will be then preferable to use
the lettrine package to set the dropcaps. If you observe Figure~\ref{lettrine} carefully, you will notice the first line of theopening paragraph is in small caps. As \tex typesets the full paragraph this is almost an impossible task to achieve through normal \tex commands and in order not to overcomplicate the discussion it can be achieved manually via trial and error. 

\section{Figures}

Most of the figures are wrapped illustrations. A couple are full page figures and bear no caption numbering. One such illustration is shown on page~\pageref{fig:vesalius}. Do note that the List of Illustrations does have the illustrations listed with additional information to that shown in the captions. 

\begin{figure}[p]
\centering
\includegraphics[width=\textwidth]{vesalius}
\centering
ANDREAS VESALIUS\par
(From an old copperplate engraving)\par
\label{fig:vesalius}
\end{figure}







%\input{./styles/style03}
\input{./styles/style04}

\cxset{style05/.style={
 chapter opening=any,
% chapter toc=true,
 chapter name=Chapter,
 chapter color=black!90,
 chapter numbering=arabic,
% number font-size=Large,
% number font-family=rmfamily,
% number font-weight=\normalfont\itshape,
% number color= black,
% number before=\kern0.5em,
% number dot=,
% number after=\hfill\hfill\par,
 number position= rightname,
 chapter shape=none,
 chapter display=block,
 chapter float=center,
 chapter font-family=rmfamily,
 chapter font-weight= itshape,
 chapter font-size=Large,
 chapter before=\hrule width \columnwidth \kern12.6pt \par\hfill,
 chapter after=,
 chapter color=black!90,
 chapter spaceout=none,
% handle margins and padding
% title margin top=10pt,
% title margin bottom=10pt,
% title before=,
% title after=\vskip12.6pt\hrule width \columnwidth,
% title font-family=rmfamily,
% title font-color=black!90,
% title font-weight=bfseries,
% title font-size=huge,
 chapter title align=centering,
 title font-shape = normal,
 header style= headings}}

\cxset{style05}
\chapter{Introduction to Style Five}\index{styles!style5}

\tcbset{width=\textwidth}
I think this style can be improved with a bit of color. You can experiment with it quite easily. The spacing on top of this style can also be adjusted to suit your typographical taste.
\medskip
\begin{figure}[ht]
\centering
\includegraphics[width=0.6\textwidth]{./chapters/chapter05.png}
\end{figure}

%\section{General notes on rules}

LaTeX's default rules would normally give problems. Best is to use TeX's primitives to built them.

\index{rules!example color}

\begin{texexample}{}{}
\makeatletter
\hrule width 5cm \kern2.6\p@
AAAAAAAAAAAAAAAAAAAAA
\vskip2.6pt\hrule width 5cm
\medskip

Problem with LaTeX rules.

\rule{5cm}{0.4pt}\par
AAAAAAAAAAAAAAAAAAAAA\par%
\rule[6.5pt]{5cm}{0.4pt}

\def\rule{\@ifnextchar[\@rule{\@rule[\z@]}}
\def\@rule[#1]#2#3{%
 \leavevmode
 \hbox{%
 \setlength\@tempdima{#1}%
 \setlength\@tempdimb{#2}%
 \setlength\@tempdimc{#3}%
 \advance\@tempdimc\@tempdima%
 \vrule\@width\@tempdimb\@height\@tempdimc\@depth-\@tempdima}}

\def\thickrule{\leavevmode \leaders \hrule height 3pt \hfill \kern \z@}

{\color{teal}\hrule width 10.5cm height3pt \kern2.6\p@
    {{\color{black!80}\HUGE CHAPTER TITLE}}\vskip3pt
\hrule width 10.5cm height3pt}
\makeatother
\end{texexample}


\section{Images}

\begin{figure}[htbp]
\centering
\includegraphics[width=0.8\textwidth]{telegraphy}
\caption{Spread from the \textit{History of Telegraphy}, the caption is set left and the image is centered.}
\end{figure}

%\input{./styles/style06}
\input{./styles/style07}
\input{./styles/style08}
\input{./styles/style09}
\input{./styles/style10}
\input{./styles/style11}
\input{./styles/style12}
\input{./styles/style15}
\input{./styles/style16}
%\input{./styles/style17}
\input{./styles/style18}
\input{./styles/style19}
\input{./styles/style21}
\input{./styles/style22}
\input{./styles/versochapter}
\input{./styles/style26}
\input{./styles/style27}
\input{./styles/style28}
\input{./styles/style29}
\input{./styles/style30}
\input{./styles/style31}
\input{./styles/style32}
\input{./styles/style33}
\input{./styles/style34}
\input{./styles/style35}
\input{./styles/style36}
\input{./styles/style37}
\input{./styles/style38}
\input{./styles/style39}
\input{./styles/style40}
%\input{./styles/style41}
\input{./styles/style42}
\input{./styles/style43}
%\input{./styles/style44}
%\input{./styles/style45}
\input{./styles/style49}
\input{./styles/style50}
\input{./styles/style51}
\input{./styles/style52}
\input{./styles/style53}


\begin{docCommand}{docColor}{\marg{name}}
  Documents a color with given \meta{name}. The color is automatically indexed.
\begin{dispExample}
The color \docColor{spot} is available.
\end{dispExample}
\end{docCommand}


\begin{docCommand}{docColor*}{\marg{name}}
  Identical to \refCom{docColor}, but without index entry.
\end{docCommand}






\end{document}

0, .68 , .93