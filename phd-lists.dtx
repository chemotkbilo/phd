% \iffalse meta-comment
%<*internal>
\iffalse
%</internal>
%<*readme>
----------------------------------------------------------------
phd-pkgmanager --- a package to shorten preambles
E-mail: yannislaz@gmail.com
Released under the LaTeX Project Public License v1.3c or later
See http://www.latex-project.org/lppl.txt
----------------------------------------------------------------
This file provides a phd for defining a class.
%</readme>
%<*readmemd>
###The `phd-lists` LaTeX2e package (version 0.08.0)

The `phd-lists` latex package forms part of a suite of packages
that are bundled with the `phd` package and the class with the 
same name which provide convenient methods to create new styles 
for books, reports and articles. It also loads the most commonly used packages
and resolves conflicts.

This work consists of the file  
   `phd-lists.dtx`,
   
and the derived files   

   `phd-lists.ins`,  
   `phd-lists.pdf`, 
   `phd-lists.sty`.

###Installation

run
    
    `phd-lua  phd-lists.dtx`   on windows

If you have any difficulties with the package come and join us at
http://tex.stackexchange.com and post a new question or
add a comment at http://tex.stackexchange.com/a/45023/963.
or send me a message at  yannislaz at gmail.com

### Documentation

The package was written using the `doc` and `docscript` packages,
so that it is self documented in a literary programming style.
The .pdf is a fat document, providing over fifty book styles (the
equivalent of classes) plus there is a lot of write-up on the inner
workings of TeX and LaTeX2e. However, you don't need to know much
to use it.

      \usepackage{phd}
      \input{style13}

All choices, are made via an extended key-value interface.
Although not a compliment, it resembles CSS and the keys are a bit verbose but
attributes are easy to change and have a consistent and easy to remember interface.

To set or add a key we only use one command:

      \cxset{chapter name font-size   = Huge,
             chapter number font-size = HUGE}

### Future Development

This is still an experimental version, but I will retain the
interface in future releases. There is a large amount of
work still to be carried out to improve the template styles
provided, to test it more thoroughly and to add a number of
improvements in the special designs. At present I estimate
that I have completed about 70% of the work that needs
to be done.

__The package as it stands is not production stable.__

%</readmemd>
%
%<*TODO>
1. On final round add pkg options. This was left as last in order not to solve problems by adding
    options. Too many options are not a good User Interface.
2.  Finish symbol management, both text and math. Math already 80% incorporated.
3.  Better integration of indexing commands.   
4.  Revisit layout manager for Chapters. Broke again in tests.
5.  Docs. Add all references.
6.  Incorporate phd class for more flexibility.
7.  Improve package manager.
8.  Group script loading for better font management.
9.  General font management to relook it again.
10. Add all style sections (about 100 already prepared). Once they
     are all working issue beta version.
%</TODO>
%<*internal>
\fi
\def\nameofplainTeX{plain}
\ifx\fmtname\nameofplainTeX\else
  \expandafter\begingroup
\fi
%</internal>
%<*install>
\input docstrip.tex
\keepsilent
\askforoverwritefalse
\preamble
----------------------------------------------------------------
phd --- A package to beautify documents.
E-mail: yannislaz@gmail.com
Released under the LaTeX Project Public License v1.3c or later
See http://www.latex-project.org/lppl.txt
----------------------------------------------------------------
\endpreamble

%\BaseDirectory{C:/users/admin/my documents/github/phd}
%\usedir{MWE}
\generate{\file{\jobname.sty}{
  \from{\jobname.dtx}{LISTS}}
  }

%\nopreamble\nopostamble

%</install>

%<install>\endbatchfile
%<*internal>
%\usedir{tex/latex/phd}
\generate{
  \file{\jobname.ins}{\from{\jobname.dtx}{install}}
}
\nopreamble\nopostamble

\generate{
	\file{README.txt}{\from{\jobname.dtx}{readme}}
  }

\generate{
  \file{\jobname.md}{\from{\jobname.dtx}{readmemd}}
}
\generate{
  \file{\jobname-todo.tex}{\from{\jobname.dtx}{TODO}}
}

\ifx\fmtname\nameofplainTeX
  \expandafter\endbatchfile
\else
  \expandafter\endgroup
\fi
%</internal>
%<*driver>

%\listfiles
%gdef\@onlypreamble{} % TO BE REMOVED NEEDED FOR TUTS
\documentclass[twoside,11pt,a4paper]{ltxdoc}
\usepackage[bottom=2cm]{geometry}
\savegeometry{std}
% \usepackage[style=mla]{biblatex}
\usepackage{phd}
\usepackage{phd-documentation}
\usepackage{phd-toc}
\usepackage{phd-runningheads}
\usepackage{phd-lowersections}
\usepackage{makeidx}
\usepackage{phd-lists}
\pagestyle{headings}
\sethyperref
\cxset{palette bbc}
\makeindex
\begin{filecontents}{defaults-chapters}
%%    General Defaults for Chapters
\cxset{%    
    chapter title margin-top-width    =  0cm,
    chapter title margin-right-width  =  1cm,
    chapter title margin-bottom-width = 10pt,
    chapter title margin-left-width   = 0pt,
    chapter align                     = left,
    chapter title align               = left, %checked
    chapter name                      = hang,
    chapter format                    = fashion,
    chapter font-size                 = Huge,
    chapter font-weight               = bold,
    chapter font-family               = sffamily,
    chapter font-shape                = upshape,
    chapter color                     = black,
    chapter number prefix             = ,
    chapter number suffix             = ,
    chapter numbering                 = arabic,
    chapter indent                    = 0pt,
    chapter beforeskip                = -3cm,
    chapter afterskip                 = 30pt,
    chapter afterindent               = off,
    chapter number after              = ,
    chapter arc                       = 0mm,
    chapter background-color          = bgsexy,
    chapter afterindent               = off,
    chapter grow left                 = 0mm,
    chapter grow right                = 0mm, 
    chapter rounded corners           = northeast,
    chapter shadow                    = fuzzy halo,
    chapter border-left-width         = 0pt,
    chapter border-right-width        = 0pt,
    chapter border-top-width          = 0pt,
    chapter border-bottom-width       = 0pt,
    chapter padding-left-width        = 0pt,
    chapter padding-right-width       = 10pt,
    chapter padding-top-width         = 10pt,
    chapter padding-bottom-width      = 10pt,
    chapter number color              = white,
    chapter label color               = white,    
    }
 \cxset{    
    chapter number font-size        = huge,
    chapter number font-weight      = bfseries,
    chapter number font-family      = sffamily,
    chapter number font-shape       = upshape,
    chapter number align            = Centering,
    }
\cxset{%    
     chapter title font-size        = Huge,
     chapter title font-weight      = bold,
     chapter title font-family      = calligra,
     chapter title font-shape       = upshape,
     chapter title color            = black,
     }    
\end{filecontents}
%% LaTeX2e file `defaults-chapters'
%% generated by the `filecontents' environment
%% from source `phd-scriptsmanager' on 2015/08/25.
%%
%%    General Defaults for Chapters
\cxset{%
    chapter title margin-top-width    =  0cm,
    chapter title margin-right-width  =  1cm,
    chapter title margin-bottom-width = 10pt,
    chapter title margin-left-width   = 0pt,
    chapter align                     = left,
    chapter title align               = left, %checked
    chapter name                      = hang,
    chapter format                    = fashion,
    chapter font-size                 = Huge,
    chapter font-weight               = bold,
    chapter font-family               = sffamily,
    chapter font-shape                = upshape,
    chapter color                     = black,
    chapter number prefix             = ,
    chapter number suffix             = ,
    chapter numbering                 = arabic,
    chapter indent                    = 0pt,
    chapter beforeskip                = -3cm,
    chapter afterskip                 = 30pt,
    chapter afterindent               = off,
    chapter number after              = ,
    chapter arc                       = 0mm,
    chapter background-color          = bgsexy,
    chapter afterindent               = off,
    chapter grow left                 = 0mm,
    chapter grow right                = 0mm,
    chapter rounded corners           = northeast,
    chapter shadow                    = fuzzy halo,
    chapter border-left-width         = 0pt,
    chapter border-right-width        = 0pt,
    chapter border-top-width          = 0pt,
    chapter border-bottom-width       = 0pt,
    chapter padding-left-width        = 0pt,
    chapter padding-right-width       = 10pt,
    chapter padding-top-width         = 10pt,
    chapter padding-bottom-width      = 10pt,
    chapter number color              = white,
    chapter label color               = white,
    }
 \cxset{
    chapter number font-size        = huge,
    chapter number font-weight      = bfseries,
    chapter number font-family      = sffamily,
    chapter number font-shape       = upshape,
    chapter number align            = Centering,
    }
\cxset{%
     chapter title font-size        = Huge,
     chapter title font-weight      = bold,
     chapter title font-family      = calligra,
     chapter title font-shape       = upshape,
     chapter title color            = black,
     }
  
%\definecolor{bgsexy}{HTML}{FF6927}
%
%\definecolor{creamy}{HTML}{FDEBD7}
\cxset{chapter title color= creamy,
       chapter label color = creamy,
       chapter number color = creamy,
       chapter number font-size = Huge,
       subsection title color = creamy,
       chapter name = CHAPTER,
       chapter label case = upper,
       chapter number align=left,
       part format = traditional,
       part background-color=spot,
       part beforeskip                = -3cm,
       part afterskip                 = 30pt,
       }
\begin{document}
\parindent1em
\coverpage{asia}{Book Design Monographs}{Camel Press}{LISTS}{DESIGN} 
\pagestyle{empty}
%\coverpage{habtoor-city}{Delay Claim}{HLS-DSE/JV}{HABTOOR CITY}{MEP CLAIM} 
\secondpage
\pagestyle{empty}
\clearpage

\tableofcontents

\pagestyle{empty}
\setcounter{secnumdepth}{6}
\parskip0pt plus.1ex minus.1ex
\mainmatter
\pagenumbering{arabic}
\pagestyle{headings}        
\makeatletter
%\@debugtrue

\makeatother
%FIX LIST DIAGRAM
\begingroup


%
%\cxset{steward,
%  chapter name=chapter,
%  numbering=arabic,
%  custom= stewart,
%   image={sweepers.jpg},
%  texti={Lists are essential elements of any document style and perhaps the most troublesome to get right.
%         In this chapter we discuss the construction of lists and offer a key value interface.},
%  textii={The Chapter discusses in detail the construction of lists. It reviews the mechanisms offered
%          by LaTeX and outlines a key value approach to building lists. We define a standard interface that does not
%          interfere with the original commands. The three standard list styles \textit{enumerate, itemize} and \textit{description} are redesigned to accept a key value interface. The photograph is Lewis Hine's which noted: ``Ivey Mill Company, Hickory, N.C. Some doffers and sweepers. Plenty of them.'' Location: Hickory, Catawba County Date: November 1908. Photographs like this were used by Hine to campaign against child labour.
%         }
%}

\chapter{Standard \LaTeX\ Lists}

\tcbset{width=\linewidth,arc=1mm,before=\bigskip,after=\medskip,left=8mm}

The general parameters affecting a general list is shown in the  diagram  below\footnote{Produced using the \texttt{layouts} package.}. LaTeX offers three general list structures, enumerate, itemize and description.
%\begin{figure}[hp]
%\listdiagram
%\caption{Layout of an \texttt{enumerate} list} \label{fig:lstenum}
%\end{figure}

\section{Package usage}

List are set using setenumerate, setitemize, setdescription. It is also possible to create new list structures, which will be explained a bit later on.

\section{The list geometry}

You can draw a list diagram as shown below, using the function \docAuxCmd{drawlistdiagram} from
the \pkgname{xlayouts} package, which is bundled with the \pkgname{phd} package.

\medskip
{\centering
\drawlistdiagram

\captionof{figure}{\latexe list diagram.}
}
\medskip
%
\section{The description list environment}

You can use the description list environment as you would normally use it with \latexe.

\begin{docEnv} {description} {}
\end{docEnv}

However, a number of settings are available to modify the styling of the environment. These 
include settings for fonts and color, as well as spacing and margins.

\begin{docKey} {description label font-face} { = \meta{font shape} } {initial, default=inherit}
\end{docKey}

\begin{docKey} {description label font-family} { = \meta{font shape} } {initial, default=inherit}
\end{docKey}

\begin{docKey} {description label font-size} { = \meta{font size} } {initial, default=inherit}
\end{docKey}

\begin{docKey} {description label font-weight} { = \meta{font weight} } {initial, default=inherit}
\end{docKey}

\begin{docKey} {description label font-shape} { = \meta{font shape} } {initial, default=inherit}
\end{docKey}

\begin{docKey} {description label color} { = \meta{color name} } {initial, default=thedescriptionlabelcolor}
\end{docKey}

\begin{docKey} {description label sep} { = \meta{dim} } {initial, default = 1em}
\end{docKey}

\begin{docKey} {description label width} { = \meta{dim} } {initial, default = 1em}
\end{docKey}

\begin{docKey} {description margin left} { = \meta{dim} } {initial, default = 0em}
\end{docKey}

\begin{docKey} {description margin right} { = \meta{dim} } {initial, default = 0em}
\end{docKey}

\begin{docKey} {description item indent} { = \meta{dim} } {initial, default = 0em}
\end{docKey}

Unlike the enumerate and itemize environment, the description list environment is defined in the book class.
The environment is defined as:
\begin{teXXX}
\newenvironment{description}
  {\list{}{\labelwidth\z@ \itemindent-\leftmargin
    \let\makelabel\descriptionlabel}}
  {\endlist}
\newcommand*\descriptionlabel[1]{\hspace\labelsep\labelcolor@cx
  \normalfont\bfseries #1}
\end{teXXX}


What is important to notice here is that all the standard list parameters are left essentially unchanged. The only item that is affected is \lstinline{\makelabel}, which is redefined in \lstinline{description} label.


We can define a number of keys for ease of formatting such descriptions lists rather than each time redefining them.


%
%\begin{tcolorbox}[title=Basic description list keys]
%\begin{lstlisting}
%\cxset{
% description label font-size/.store in=\descriptionlabelfontsize@cx,
% description label font-weight/.store in=\descriptionlabelfontweight@cx,
% description label font-family/.store in=\descriptionlabelfontfamily@cx,
% description label font-shape/.store in=\descriptionfontshape@cx,
% description label color/.store in=\descriptionlabelcolor@cx,
% description label sep/.store in=\descriptionlabelsep@cx,
% description label width/.store in=\descriptionlabelwidth@cx,
% description margin left/.store in=\descriptionmarginleft@cx,
% description margin right/.store in=\descriptionmarginright@cx,
% description item indent/.store in=\descriptionitemindent@cx,
% list parindent/.store in=\descriptionlistparindent@cx,
%}
%\end{lstlisting}
%\end{tcolorbox}




%\begin{tcolorbox}[title=Basic description list keys]
%\begin{lstlisting}
%\def\setdescription#1{%
%\cxset{#1}%
%\renewenvironment{description}%
%{\list{}{\listparindent\descriptionlistparindent@cx%
%                       \leftmargin=\descriptionmarginleft@cx%
%                       \rightmargin=\descriptionmarginright@cx%
%                       \itemindent\descriptionitemindent@cx%
%                       \labelwidth\descriptionlabelwidth@cx%
%                       \labelsep=\descriptionlabelsep@cx%
%                       \let\makelabel\descriptionlabel}}%
%               {\endlist}%
%%
%\renewcommand\descriptionlabel[1]{%
%  \fboxrule0pt\fboxsep0pt%
%  \hspace\descriptionlabelsep@cx%
%  \fbox{\color{\descriptionlabelcolor@cx}%
%  \normalfont\bfseries\raggedleft##1\thickspace%
%}}%
%}
%\end{lstlisting}
%\end{tcolorbox}



\begin{description}
\item [Special] \lorem
\item [Another] \lorem
\end{description}



\section{Creating new description like environments}

The macro \docAuxCmd*{NewDescriptionEnvironment} can be used to redefine new description like environments.

\begin{texexample}{Define a new description list environment}{ex:newdesclist}
% define the orange description environment
\NewDescriptionEnvironment
  [
    description label font-size     = small,
    description label font-weight   = bfseries,
    description label font-family   = serif,
    description label font-shape    = italic,
    description label color         = bgsexy,
    description label sep           = 3.5pt\relax,
    description label width         = 40pt,
    description margin left         = 50pt,
    description margin right        = 20pt,
    description item indent         = -2.5pt,
    list parindent=1em,
  ]
  {orangedescription}

% Sample
The \texttt{orangedescription} environment in action.
\begin{orangedescription}
 \item[One] \lorem
 \item[Two] \lorem
 \item[Three] \lorem
\end{orangedescription}

\lorem

\makeatother
\end{texexample}



\section{Example: redefining a description list}
We will now develop a description environment, that can be useful for the documentation of packages to describe options. We will use a description list as the basis of the environment. We define the following key values.
|\itemindent-\leftmargin|
%%% reset some of the values
\cxset{description label color=black, 
         description margin left=60pt, 
         description item indent=70pt,
         description margin right=50pt}

\begin{description}
\item[The description label font-size] This is the first item. \the\itemindent (itemindent) \the\leftmargin (leftmargin)
\item[The description label font-weight] This is the second item. \lipsum*[1]
\end{description}

\cxset{
         description margin left=60pt, 
         description item indent=30pt,
         }
         
\begin{description}
\item[The description label font-size] This is the first item. \the\itemindent (itemindent) \the\leftmargin (leftmargin)
\item[The description label font-weight] This is the second item. \lipsum*[1]
\end{description}

\cxset{
         description margin left=10pt, 
         description margin right=20pt,
         description item indent=80pt,
         }
         
\begin{description}
\item[The description label font-size] This is the first item. \the\itemindent (itemindent) \the\leftmargin (leftmargin)
\item[The description label font-weight] This is the second item. \lipsum[1]
\end{description}
\endinput

\section{Enumerated lists}


\begin{enumerate}
\item one
\item two
\item three
\end{enumerate}

Enumerated (numbered) list environments are characterized by numbering. They use a variety of fields and counters as shown in table.

\subsection{Vertical skips}

By default LaTeX adds vertical skips, as shown in figure 1. The definition of these skips is influenced by the font size and are defined in the \texttt{bk10.clo} files, hence hard to find and change. Each level of the list has its own definition as \lstinline{\@listi}.

\bigskip
\tcbset{width=\linewidth,arc=1mm,before=\bigskip,left=8mm}

\begin{tcolorbox}[title=Extract from bk10.clo]
\begin{lstlisting}
\def\@listi{\leftmargin\leftmargini
            \parsep 4\p@ \@plus2\p@ \@minus\p@
            \topsep 8\p@ \@plus2\p@ \@minus4\p@
            \itemsep4\p@ \@plus2\p@ \@minus\p@}
\let\@listI\@listi
\@listi

\def\@listii {\leftmargin\leftmarginii
              \labelwidth\leftmarginii
              \advance\labelwidth-\labelsep
              \topsep    4\p@ \@plus2\p@ \@minus\p@
              \parsep    2\p@ \@plus\p@  \@minus\p@
              \itemsep   \parsep}
\def\@listiii{\leftmargin\leftmarginiii
              \labelwidth\leftmarginiii
              \advance\labelwidth-\labelsep
              \topsep    2\p@ \@plus\p@\@minus\p@
              \parsep    \z@
              \partopsep \p@ \@plus\z@ \@minus\p@
              \itemsep   \topsep}
\def\@listiv {\leftmargin\leftmarginiv
              \labelwidth\leftmarginiv
              \advance\labelwidth-\labelsep}
\def\@listv  {\leftmargin\leftmarginv
              \labelwidth\leftmarginv
              \advance\labelwidth-\labelsep}
\def\@listvi {\leftmargin\leftmarginvi
              \labelwidth\leftmarginvi
              \advance\labelwidth-\labelsep}
\end{lstlisting}
\end{tcolorbox}

\endinput
%\cxset{enumerate numberingi/.is choice,
%  enumerate numberingi/.code={\renewcommand\theenumi {\csname#1\endcsname{enumi}}},
%  enumerate numberingii/.code={\renewcommand\theenumii {\csname#1\endcsname{enumii}}},
%  enumerate numberingiii/.code={\renewcommand\theenumiii {\csname#1\endcsname{enumiii}}},
%  enumerate numberingiv/.code={\renewcommand\theenumiv {\csname#1\endcsname{enumiv}}},
%  enumerate labeli punctuation/.store in=\enumeratepunctuationi@cx,
%  enumerate labeli/.is choice,
%  enumerate labeli/brackets/.code={\renewcommand\labelenumi{(\theenumi\enumeratepunctuationi@cx)}},
%  enumerate labeli/square brackets/.code={\renewcommand\labelenumi{[\theenumi\enumeratepunctuationi@cx]}},
%  enumerate labeli/right bracket/.code={\renewcommand\labelenumi{\theenumi\enumeratepunctuationi@cx)}},
%  enumerate label left/.store in=\enumeratelabelleft@cx,
%  enumerate label right/.code=\renewcommand\labelenumi{\enumeratelabelleft@cx\theenumi\enumeratepunctuationi@cx#1},
%  enumerate leftmargini/.code={\setlength\leftmargini{#1}},
%  enumerate leftmarginii/.code={\setlength\leftmarginii{#1}},
%  enumerate leftmarginiii/.code={\setlength\leftmarginiii{#1}},
%  enumerate leftmarginiv/.code={\setlength\leftmarginiv{#1}},
%  listi topsep/.store in=\listitopsep@cx,
%  listi partopsep/.store in=\listipartopsep@cx,
%  listi itemsep/.store in=\listiitemsep@cx,
%  listi parsep/.store in=\listiparsep@cx,
%  listii topsep/.store in=\listiitopsep@cx,
%  listii partopsep/.store in=\listiipartopsep@cx,
%  listii itemsep/.store in=\listiiitemsep@cx,
%  listii parsep/.store in=\listiiparsep@cx,
%  listiii topsep/.store in=\listiiitopsep@cx,
%  listiii partopsep/.store in=\listiiipartopsep@cx,
%  listiii itemsep/.store in=\listiiiitemsep@cx,
%  listiii parsep/.store in=\listiiiparsep@cx,
%}
%
%\cxset{compact1/.style={%
%  enumerate numberingi=arabic,
%  enumerate numberingii=alph,
%  enumerate numberingiii=alph,
%  enumerate numberingiv=roman,
%  enumerate labeli punctuation=.,
%  enumerate label left=,
%  enumerate label right=,
%  enumerate leftmargini=2.2em,
%  enumerate leftmarginii=2.1em,
%  enumerate leftmarginiii=1.5em,
%  enumerate leftmarginiv=2em,
%  listi topsep=8\p@ \@plus2\p@ \@minus\p@,
%  listi itemsep=0\p@ \@plus2\p@ \@minus\p@,
%  listi parsep=0\p@ \@plus2\p@ \@minus\p@,
%  listii topsep=0\p@ \@plus2\p@ \@minus\p@,
%  listii itemsep=0\p@ \@plus2\p@ \@minus\p@,
%  listii parsep=0\p@ \@plus2\p@ \@minus\p@,
%  listiii topsep=0\p@ \@plus2\p@ \@minus\p@,
%  listiii itemsep=0\p@ \@plus2\p@ \@minus\p@,
%  listiii parsep=0\p@ \@plus2\p@ \@minus\p@,
%}}

%\cxset{compact2/.style={%
%  enumerate numberingi=alph,
%  enumerate numberingii=roman,
%  enumerate numberingiii=alph,
%  enumerate numberingiv=roman,
%  enumerate labeli punctuation=,
%  enumerate label left=(,
%  enumerate label right=),
%  enumerate leftmargini=2.2em,
%  enumerate leftmarginii=2.1em,
%  enumerate leftmarginiii=1.5em,
%  enumerate leftmarginiv=2em,
%  listi topsep=8\p@ \@plus2\p@ \@minus\p@,
%  listi itemsep=0\p@ \@plus2\p@ \@minus\p@,
%  listi parsep=0\p@ \@plus2\p@ \@minus\p@,
%  listii topsep=0\p@ \@plus2\p@ \@minus\p@,
%  listii itemsep=0\p@ \@plus2\p@ \@minus\p@,
%  listii parsep=0\p@ \@plus2\p@ \@minus\p@,
%  listiii topsep=0\p@ \@plus2\p@ \@minus\p@,
%  listiii itemsep=0\p@ \@plus2\p@ \@minus\p@,
%  listiii parsep=0\p@ \@plus2\p@ \@minus\p@,
%}}


%\def\setenumerate#1{
%\cxset{#1}
%\def\@listi{\leftmargin\leftmargini
%            \parsep\listiparsep@cx
%            \topsep\listitopsep@cx\relax
%            \itemsep\listiitemsep@cx}
%\def\@listii{\leftmargin\leftmarginii
%            \parsep\listiiparsep@cx
%            \topsep\listiitopsep@cx\relax
%            \itemsep\listiiitemsep@cx}
%\def\@listiii{\leftmargin\leftmarginiii
%            \parsep\listiiiparsep@cx
%            \topsep\listiiitopsep@cx\relax
%            \itemsep\listiiiitemsep@cx}
%}

\setenumerate{compact1}


The list can be viewed here:

\begin{enumerate}
\item Level i
      \begin{enumerate}
       \item Level ii
          \begin{enumerate}
            \item Level iii
              \begin{enumerate}
                \item Level iv. \lipsum*[1]
              \end{enumerate}
          \end{enumerate}
      \end{enumerate}
\end{enumerate}


\begin{tcblisting}{title=Example with style \textit{compact2}}

\cxset{compact2/.style={%
  enumerate numberingi=alpha,
  enumerate numberingii=roman,
  enumerate numberingiii=alpha,
  enumerate numberingiv=roman,
  enumerate labeli punctuation=,
  enumerate label left=(,
  enumerate label right=),
  enumerate leftmargini=2.2em,
  enumerate leftmarginii=2.1em,
  enumerate leftmarginiii=1.5em,
  enumerate leftmarginiv=2em,
  listi topsep=8\p@ \@plus2\p@ \@minus\p@,
  listi itemsep=0\p@ \@plus2\p@ \@minus\p@,
  listi parsep=0\p@ \@plus2\p@ \@minus\p@,
  listii topsep=0\p@ \@plus2\p@ \@minus\p@,
  listii itemsep=0\p@ \@plus2\p@ \@minus\p@,
  listii parsep=0\p@ \@plus2\p@ \@minus\p@,
  listiii topsep=0\p@ \@plus2\p@ \@minus\p@,
  listiii itemsep=0\p@ \@plus2\p@ \@minus\p@,
  listiii parsep=0\p@ \@plus2\p@ \@minus\p@,
}}
\setenumerate{compact2}
\begin{enumerate}
\item Does this project actually merit the use of the Minor Works Form or Intermediate Form instead of their `grown up' relatives?
\item Do the number of PC or prime cost items mean that it would be more desirable to use a re-measurable form?
\item Is this a contract which merits the production of full scale bills
of quantities or is something more standardised going to suffice?
\end{enumerate}
\end{tcblisting}

As you will observe the numbering in the above example has been enclosed in round brackets, using:

\begin{tcolorbox}
\begin{lstlisting}
  enumerate label left=(,
  enumerate label right=),
\end{lstlisting}
\end{tcolorbox}

The next example is from the \textit{LaTeX Companion}. In example~\ref{ex:companion}, the first-level list elements are decorated with the section sign (\S) as a prefix and a period as a suffix (omitted in references). We will
define this as a style named \textit{paragraphsymbol} for the lack of any better name. This style can sometimes be found in legal texts.

\begin{texexample}{Paragraph symbols in enumerate}{ex:companion}
\cxset{paragraphsymbol/.style={%
  enumerate numberingi=arabic,
  enumerate labeli punctuation=.,
  enumerate label left=\S,
  enumerate label right=,
}}
\setenumerate{paragraphsymbol}
\begin{enumerate}
\item \lorem
\item \lorem
\item \lorem
\end{enumerate}
\end{texexample}
\
\section{Creating enumerated environments}

New enumerated environments cab be created by using the macro \lstinline{\newenumeratedenvironment}. Keys are set as either styles or individually.

\def\newenumeratedenvironment#1#2{%
 \expandafter\def\csname#1\endcsname{%
 \cxset{#2}
 \ifnum \@enumdepth >\thr@@\@toodeep\else
 \advance\@enumdepth\@ne
 \edef\@enumctr{enum\romannumeral\the\@enumdepth}%
 \expandafter
 \list
 \csname label\@enumctr\endcsname
 {\usecounter\@enumctr\def\makelabel####1{\hss\llap{####1}}}%
 \fi}
 \expandafter\let\csname end#1\endcsname=\endlist
}


\begin{texexample}{An enumerated list factory}{}
\newenumeratedenvironment{paragraphsymbol}{
  enumerate numberingi=roman,
  enumerate labeli punctuation=.,
  enumerate label left={\textcolor{purple}{\P}},
  enumerate label right=,
}
\begin{paragraphsymbol}
\item \lorem
\item \lorem
\item \lorem
      \begin{itemize}
        \item This is bullets
      \end{itemize}
\end{paragraphsymbol}
\end{texexample}


\newpage
\section{Setup keys for enumerate lists}
\begin{description}
\item [enumerate numberingi] Sets the numbering style of the list at level $n$. Valid values are \textit{Alph, alph, arabic, Roman, roman, WORDS, words}.
\end{description}

\clearpage

\section{Itemized lists}

The itemized \LaTeX\ lists are similar to those for the enumerated lists. However they are somehow simpler as there is no need for counters.

\bigskip
\begin{tcolorbox}[width=\linewidth,arc=2mm,title=Default \LaTeX\ parameters for itemized lists]
\begin{lstlisting}
\newcommand\labelitemi{\textbullet}
\newcommand\labelitemii{\normalfont\bfseries \textendash}
\newcommand\labelitemiii{\textasteriskcentered}
\newcommand\labelitemiv{\textperiodcentered}
\end{lstlisting}
\end{tcolorbox}





\begin{itemize}
\item Level i
      \begin{itemize}
       \item Level ii
          \begin{itemize}
            \item Level iii
              \begin{itemize}
                \item Level iv. \lipsum*[1]
              \end{itemize}
          \end{itemize}
      \end{itemize}
\end{itemize}

\cxset{red/.style={
 labelitemi={{\color{green}\ding{'64}}},
 labelitemii=\color{red}\textendash,
 labelitemiii=\textasteriskcentered,
 labelitemiv=\textperiodcentered,
}}

Now that we have managed to abstract the itemized environment we can generate a new environment factory.

\def\newitemizedenvironment#1#2{
\expandafter\def\csname#1\endcsname{%
 \cxset{#2}%
 \ifnum \@itemdepth >\thr@@\@toodeep\else
 \advance\@itemdepth\@ne
 \edef\@itemitem{labelitem\romannumeral\the\@itemdepth}%
 \expandafter
 \list
 \csname\@itemitem\endcsname
 {\def\makelabel####1{\hss\llap{####1}}}%
 \fi}
 \expandafter\let\csname end#1\endcsname=\endlist
}

%\newitemizedenvironment{reditemize}{black}
%
%
%\begin{reditemize}
%\item Test.
%   \begin{reditemize}
%    \item test.
%   \end{reditemize}
%\end{reditemize}
%
%\begin{itemize}
%\item Level i
%      \begin{itemize}
%       \item Level ii
%          \begin{itemize}
%            \item Level iii
%              \begin{itemize}
%                \item Level iv. \lipsum*[1]
%              \end{itemize}
%          \end{itemize}
%      \end{itemize}
%\end{itemize}


\section{Itemized lists with ding symbols}

So far we have used both standard symbols as well as those provided by the pifont that offers numerous,
dingbang symbols. The pifont package also offers environments to do that more easily.


\begin{texexample}{dinglist}{}
\begin{dinglist}{"E4}
\item The first item. \item The second
item in the list.
\end{dinglist}
\end{texexample}

\begin{dingautolist}{'300}
\item The first item in the list.\label{lst:a}
\item The second item in the list.\label{lst:b}
\item The third item in the list.\label{lst:c}
\item The fourth item in the list.\label{lst:d}
\end{dingautolist}

\newenvironment{steps}{\dingautolist{'300}}{\enddingautolist}

\begin{steps}
\item The first item in the list.\label{lst:a}
\item The second item in the list.\label{lst:b}
\item The third item in the list.\label{lst:c}
\item The fourth item in the list.\label{lst:d}
\end{steps}

\makeatother
\endgroup

\DocInput{\jobname.dtx}
\printindex
 %
% 
\end{document}
%</driver>
% \fi
% 
%  \CheckSum{0}
%  \CharacterTable
%  {Upper-case    \A\B\C\D\E\F\G\H\I\J\K\L\M\N\O\P\Q\R\S\T\U\V\W\X\Y\Z
%   Lower-case    \a\b\c\d\e\f\g\h\i\j\k\l\m\n\o\p\q\r\s\t\u\v\w\x\y\z
%   Digits        \0\1\2\3\4\5\6\7\8\9
%   Exclamation   \!     Double quote  \"     Hash (number) \#
%   Dollar        \$     Percent       \%     Ampersand     \&
%   Acute accent  \'     Left paren    \(     Right paren   \)
%   Asterisk      \*     Plus          \+     Comma         \,
%   Minus         \-     Point         \.     Solidus       \/
%   Colon         \:     Semicolon     \;     Less than     \<
%   Equals        \=     Greater than  \>     Question mark \?
%   Commercial at \@     Left bracket  \[     Backslash     \\
%   Right bracket \]     Circumflex    \^     Underscore    \_
%   Grave accent  \`     Left brace    \{     Vertical bar  \|
%   Right brace   \}     Tilde         \~}
%
%
%
% \changes{1.0}{2013/01/26}{Converted to DTX file}
%
% \DoNotIndex{\newcommand,\newenvironment}
% \GetFileInfo{phd.dtx}
% 
%  \def\fileversion{v1.0}          
%  \def\filedate{2012/03/06}
% \title{The \textsf{phd} package.
% \thanks{This
%        file (\texttt{phd.dtx}) has version number \fileversion, last revised
%        \filedate.}
% }
% \author{Dr. Yiannis Lazarides \\ \url{yannislaz@gmail.com}}
% \date{\filedate}
%
%
% 
% ^^A\maketitle
% 
% ^^A\frontmatter
%  ^^A\coverpage{./images/hine02.jpg}{Book Design }{Camel Press}{}{}
%  \newpage
% ^^A\secondpage
% \pagestyle{empty}
%
%
% 
%
%
% \pagestyle{headings}
% \raggedbottom
%  \OnlyDescription
%
%  ^^A\StopEventually{\printindex}

% \CodelineNumbered
% \pagestyle{headings}
% 
% 
% ^^A\part{IMPLEMENTATION AND FRIENDS}
% 
%
% \chapter{Lists Package Code Implementation Objectives and Strategy}
% 
% \epigraph{
% I was reflecting on the convoluted Java frameworks widely adopted at work. Those hefty frameworks brought coding structures and conventions to large engineering teams; meanwhile, they also sucked the fun of programming like a Pastafarian monster slurping all the tomato sauce on a plate of spaghetti.
%}{\href{http://blog.zmxv.com/2015/07/code-golf-at-google.html}{Zhen Wang}}
%
% We start by outlining what we are trying to achieve with this package:
%
% \begin{enumerate}
% \item To provide a declarative interface to enable users to modify headings by
%       setting keys, rather than writing macros.  
% \item The interface must be able to manupulate properties of headings down to
%       the last detail.
% \item To provide a compatibility mode, where documents wishing to test the package
% can have an easy switch to switch in and out. This is also important for the testing of the package.
% \item To provide a number of templates that cover most of the typical use case.
% \item To provide means for a plug-in architecture for extensions.

% \end{enumerate}
% 
% \section{Terminology}
%
%  \begin{description}
%  \item [document] Any written item, as a book, article, or letter, especially 
%                  of a factual or informative nature.
%  \item [heading] A division of a document or document series. For a normal
%        book headings are chapters, sections etc. However we allow for
%        specifying a more complex document divided into books, volumes
%        parts etc. For example the Bible has Books, chapters and verses,
%        where a legal document might require divisions such as clauses.
%        In general these divisions are numbered. These document divisions
%        are stored in the comma list \refCom{phd_book_divisions_clist}.
%  \item [head] A typeset heading, such as chapter head, or section head.
%        This can include a counter, label and title for example, 
%        \emph{Chapter 1 Introduction}.
%  \item [dom] This is a programming interface that provides a structured
%        representation of the document (a tree) and it defines a way
%        that the structure can be accessed. Although \latexe does not
%        offer a standard way to build such a tree (mainly because
%        \tex does not require the marking of paragraphs, it is 
%        useful to think of the document as a tree structure. We also
%        allow for a semi-automated way to build such a tree (with the 
%        exception that paragraphs are not included).
% \item [element] A part of the document tree that can be styled on
%       its own. For example the chapter label, or the section number.
%
% \end{description}
%
% \section{Users}
%  We classify users according to the \LaTeX3 terminology as a) programmers b) template designers
%  and c) authors.
% \subsection{Author}
%  We assume that the author has an exising template which she is using but might want to do
%  some minor modifications, for example use an italic shape for the font of the mark, but an 
%  upright font for the page numbers. 
%
% {\obeylines 
%~~ |\cxset|
%~~~~~|{|
%~~~~~~~~\textit{chapter number color}~~|format          = apa,|
%~~~~~~~~\textit{section title font-size} |font-size   = Large,|
%~~~~~|}|
%}  
%
% We follow the idea of representing the basic elements of documents
% as elements, each one having a parent in order to specify
% the element we need to style as accurate as possible. One can think of
% this approach being congruent with objects in other languages.
% As a matter fact nothing stops us from defining a key value
% interface as shown below.
%
% {\obeylines 
%~~ |\cxset|
%~~~~~|{| 
%~~~~~~~~\textit{header.even.mark.font.size}   = |Large,|
%~~~~~~~~\textit{header.even.mark.font.family} = |serif,|
%~~~~~|}|
%}  
%
% This would pehaps make it easier for the template designer, but I have rejected
% the idea as my aim is to make it easy for the author, who can search the template
% and just enter a couple of new proerty values.
%
% \subsection{Template designer}
% \pagestyle{headings}
% The template designer in the example above would have selected the format style
% from a number of predefined formats (templates) or would have created a style
% called \textit{apa} from an existing template and modified it using declarative
% key style.
%
% \subsection{The programmer}
%
% The programmer in the example above could have created the basic format
% \textit{apa} by using both declarative as well as defining or using existing
% macros. To the programmer we offer an extension mechanism, where the contents
% of a |ps@| command are defined. For example the programmer can define a new
% style using \tikzname, but without having to worry about defining full |ps@|
% and their interface.
%
% \section{Preliminaries}
%
%  Standard file identification. We first announce the package 
%	 and require that it be used with \LaTeX2e. 
% \iffalse
%<*LISTS>
% \fi
%  
%
%    \begin{macrocode}
\NeedsTeXFormat{LaTeX2e}[1994/12/01]%
\RequirePackage[2014/05/01]{latexrelease}
\ProvidesFile{phd-lists}[2015/1/13 v1.0 less preamble (YL)]%
%    \end{macrocode}
%
% 
% \section{Source2e Interface}
% 
% I am not very fond of mixing expl3 control sequences with source2e commands. Here
% we provide an interface for all these commands we might use. 
% This section can be revisited once expl3 code becomes available.
%
%    \begin{macrocode}
\ExplSyntaxOn
\let\ltxtoday\today
\let\phd_hang_from:nn \@hangfrom
\newif\if@ltxcompat \@ltxcompatfalse
\ExplSyntaxOff
%    \end{macrocode}
%
%
% \chapter{List Management}
% 
% Lists like tables, have always been difficult to devise a syntax for setting them. 
% We first start from enumerated lists.
%
% \section{Current status}
%
% Most journals develped their own lists and hard-wired them. Current packages are:
% \pkgname{enumitem} , \pkgname{enumerate}, \pkgname{paralist}.
%
% \subsection{enumerate}
% This package gives the \refEnv{enumerate} an optional argument which determines
% the style in which the counter is printed. An occurence of one of the tokens 
% \textbf{A a I i } or \textbf{1} produces the value of the counter printed with
% (respectively) \cmd{\Alph} \cmd{\alph} \cmd{\Roman} \cmd{\roman} or \cmd{\arabic}. 
% These letters may be surrounded by any string involving any other \tex expressions, 
% however the tokens \textbf{A a I i 1} must be inside a \{\} group if they are
% not to be taken as special.
%
%    \begin{macrocode}
% \begin{enumerate}[EX i.]
%   \item one one one one one one one
%         one one one one\label{LA}
%   \item two
%      \begin{enumerate}[{example} a)]
%        \item one of two one of two
%          one of two\label{LB}
%        \item two of two
%       \end{enumerate}
%   \item two of two
% \end{enumerate}
% 

% \begin{enumerate}[{A}-1]
% \item one\label{LC}
% \item two
% \end{enumerate}
%    \end{macrocode}
%
% This package minimally changes the original \latexe definitions. It is very convenient when you
% want now and then to change labels in a document.
%
% The enumerate environment uses four counters: \docCounter{enumi}, \docCounter{enumii}, 
% \docCounter{enumiii}, \docCounter{enumiv}. These control the numbering of the $n^{th}$ level
% enumeration.
%
% \section{User Level}
%
% At the user level we want to simplify things as much as possible. Since we are going to provide
% styles, we can provide some very flexible environments, including a |newlist| environment.
%
% \subsection{Requirements}
%
% The keys will be set at two levels. At the User Level we will provide an optional argument to
% the standard or non-standard environments to enable them to be redefined on the fly. Once they
% are redefined they will stay in force until another directive is used.
%
% \subsubsection {Vertical spacing}
%
%  The options |topsep, partopsep, parsep, itemsep| will be offered.\footnote{This will be inline with
%   the enumitem, so users familiar with its syntax can continue using it.}
%
% \subsubsection {Horizontal spacing}
%
% The options |leftmargin|, |rightmargin|, |listparindent|, |labelwidth|, |labelsep|, |itemindent| will
% also be offered.   
% 
% \subsection{Labels}
%
% The labels in the enumerate environment are mostly the numbering labels. Bezos's package allows for
% a set of commands to go here.
%
% |label=\emph{\alph*})|
%
% It also offers |label*=\meta{commands}| that emulates the enumerate package style.
%
% Options we need to offer |font|, |format| |align| |before|
%
% \subsubsection{Commands to start and resume the numbering of the list}
%
% Bezos offers this also as a key. This is very useful and has its uses.
%
% The |CSS| model that we try to follow, is more suitable for exensions for languages and others.
% This uses the |list-style-property|, which is common both for |<ul>| or |<ol>|. The
% |<ul>| and |<ol>| can be thought of as environments. 
%
% \section{Key definitions}
%
%  We first set the enumerate keys, since this involves a number of levels, we
%  automate it by mapping the counters to a function.
%
%    \begin{macrocode}
\RequirePackage{enumerate}
%    \end{macrocode}
%
% \begin{docCmd}{phd_enumerate_list_levels_clist} { \meta{clist} }
%   A clist containing the number of levels, used for the automatic generation
%   of keys.
% \end{docCmd}
%
%    \begin{macrocode}
\ExplSyntaxOn
\clist_gset:Nn \phd_enumerate_list_levels_clist {i,ii,iii,iv,v,vi}
%    \end{macrocode}
% 
%  \begin{docCmd}{phd_create_enumerate_list_keys} { \marg{level} }
%   Creates a set of keys for a level, such as ``i,ii'' etc.
%  \end{docCmd}
%    \begin{macrocode}
\cs_new:Npn  \phd_create_enumerate_list_keys #1
  {
  \cxset 
    {
      enumerate~numbering#1/.is~choice,
      
      enumerate~numbering#1/arabic/.code                        = 
        \cs_gset:cpn {theenum#1} 
          {
            \@arabic \cs:w c@enum#1 \cs_end:\relax
          },
      
      enumerate~numbering#1/alpha/.code                         = 
        \cs_gset:cpn {theenum#1} 
          {
             \exp_after:wN \alphalph \cs:w c@enum#1 \cs_end: \relax
          },
        
      enumerate~numbering#1/alph/.code                          =  
        \cs_gset:cpn {theenum#1} 
          { 
            \exp_after:wN \alphalph {\cs:w c@enum#1 \cs_end: \relax}
          },                                              
      
      enumerate~numbering#1/Alpha/.code                         = 
        \cs_gset:cpn {theenum#1} {
          \exp_after:wN \AlphAlph{\cs:w c@enum#1 \cs_end:}
          
           \relax
        },
      
      enumerate~numbering#1/roman/.code                         = 
        \cs_gset:cpn {theenum#1} {\@roman \cs:w c@enum#1 \cs_end:\relax},
      
      enumerate~numbering#1/Roman/.code                         = 
        \cs_gset:cpn {theenum#1} {\@Roman {\cs:w c@enum#1 \cs_end:\relax}},
      
      enumerate~numbering#1/none/.code                          =  
        \cs_gset:cpn {theenum#1} {},
      
%      enumerate~list-style-type/.is~choice,
%      enumerate~list-style-type/decimal/.code                  = 
%      \cs_gset:Npn \theenum#1 {\@arabic\c@enum#1},                                                   
%      enumerate~list-style-type/upper-roman/.code              = 
%      \cs_gset:Npn \theenum#1 {\@Roman\c@enum#1}\relax,
%      enumerate~list-style-type/lower-roman/.code              = 
%      \cs_gset:Npn \theenum#1 {\@roman\c@enum#1}\relax,                                             
%      enumerate~list-style-type/lower-alpha/.code              = 
%      \cs_gset:Npn \theenum#1 {\@alph\c@enum#1}\relax,                                                                                          
%      enumerate~list-style-type/uper-alpha/.code               = 
%      \cs_gset:Npn \theenum#1 {\@Alph\c@enum#1}\relax,                                                                                                                                       
%      enumerate~list-style-type/Diamond/.code                  = 
%      \cs_gset:Npn \theenum#1 {\Diamond},                                                                                                                                                                                        
%      enumerate~list-style-type/.unknown/.code                 =  
%      \cs_gset:Npn \theenum#1 {\@Alph\c@enum#1},                                                                                                                                                                                                                                             
      
      enumerate~leftmargin#1/.code                             = {\global\setlength\leftmargini{##1}},
      list#1~topsep/.store                                     = list#1topsep@cx,
      list#1~partopsep/.store                                  = list#1partopsep@cx,
      list#1~itemsep/.store                                    = list#1itemsep@cx,
      list#1~parsep/.store                                     = list#1parsep@cx,
      list#1~parindent/.store                                  = list#1parindent@cx,
  }  
}


\clist_map_inline:Nn \phd_enumerate_list_levels_clist
  {
    \phd_create_enumerate_list_keys {#1} 
  }
                                            
\ExplSyntaxOff
%    \end{macrocode}
%
%  The left and right margin are \docAuxCmd {leftskip} and \docAuxCmd{rightskip} and is 
%  the distance from the current margin.
% 
%
% \cxset{
%        enumerate numberingi   = Alpha,
%        enumerate numberingii  = alpha,
%        enumerate numberingiii = Roman,
%        enumerate numberingiv  = roman}
%
%   
% \lorem 
% \begin{enumerate}
%  \item This is the first level
%   \begin{enumerate} 
%     \item This is the second level
%         \begin{enumerate}
%            \item{third list}
%               \begin{enumerate}
%                 \item {fourth list}
%               \end{enumerate}
%          \end{enumerate}
%   \end{enumerate}
%  \end{enumerate}
% 
%    \begin{macrocode}
\cxset{  
  enumerate labeli punctuation/.store in=\enumeratepunctuationi@cx,
  enumerate labeli/.is choice,
  enumerate labeli/brackets/.code={\renewcommand\labelenumi{(\theenumi\enumeratepunctuationi@cx)}},
  enumerate labeli/square brackets/.code={\renewcommand\labelenumi{[\theenumi\enumeratepunctuationi@cx]}},
  enumerate labeli/right bracket/.code={\renewcommand\labelenumi{\theenumi\enumeratepunctuationi@cx)}},
  enumerate label left/.store in=\enumeratelabelleft@cx,
  enumerate label right/.code=\renewcommand\labelenumi{\enumeratelabelleft@cx\theenumi\enumeratepunctuationi@cx#1},
%  
}
\cxset{compact1/.style={%
  enumerate numberingi=alpha,
  enumerate numberingii=Roman,
  enumerate numberingiii=alpha,
  enumerate numberingiv=roman,
  enumerate labeli punctuation=.,
  enumerate label left=,
  enumerate label right=,
  enumerate leftmargini=2.2em,
  enumerate leftmarginii=2.1em,
  enumerate leftmarginiii=1.5em,
  enumerate leftmarginiv=2em,
  listi topsep=10\p@ \@plus2\p@ \@minus\p@,
  listi itemsep=0\p@ \@plus2\p@ \@minus\p@,
  listi parsep=0\p@ \@plus2\p@ \@minus\p@,
  listi partopsep=0pt plus1pt minus0pt,
%  
  listii topsep=0\p@ \@plus2\p@ \@minus\p@,
  listii itemsep=0\p@ \@plus2\p@ \@minus\p@,
  listii parsep=0\p@ \@plus2\p@ \@minus\p@,
%  
  listiii topsep=0\p@ \@plus2\p@ \@minus\p@,
  listiii itemsep=0\p@ \@plus2\p@ \@minus\p@,
  listiii parsep=0\p@ \@plus2\p@ \@minus\p@,
%
  listiv topsep=0\p@ \@plus2\p@ \@minus\p@,
  listiv itemsep=0\p@ \@plus2\p@ \@minus\p@,
  listiv parsep=0\p@ \@plus2\p@ \@minus\p@,
}}

\cxset{compact2/.style={%
  enumerate numberingi=Alpha,
  enumerate numberingii=roman,
  enumerate numberingiii=alph,
  enumerate numberingiv=roman,
  enumerate labeli punctuation=,
  enumerate label left=(,
  enumerate label right=),
  enumerate leftmargini=2.2em,
  enumerate leftmarginii=2.1em,
  enumerate leftmarginiii=1.5em,
  enumerate leftmarginiv=0em,
  listi parindent=1em,
  listii parindent=1em,
  listiii parindent=1em,
  listiv parindent=1em,
  listi topsep   = 8\p@ \@plus2\p@ \@minus\p@,
  listi itemsep = 0\p@ \@plus2\p@ \@minus\p@,
  listi parsep   = 0\p@ \@plus2\p@ \@minus\p@,
  listii topsep  = 0\p@ \@plus2\p@ \@minus\p@,
  listii itemsep= 0\p@ \@plus2\p@ \@minus\p@,
  listii parsep  = 0\p@ \@plus2\p@ \@minus\p@,
  listiii topsep = 0\p@ \@plus2\p@ \@minus\p@,
  listiii itemsep= 0\p@ \@plus2\p@ \@minus\p@,
  listiii parsep  = 0\p@ \@plus2\p@ \@minus\p@,
}}


\ExplSyntaxOn
\gdef\setenumerate#1
  {
    \cxset{#1}
      \gdef\@listi{%
            \listparindent\listiparindent@cx
            \leftmargin\leftmargini
            \parsep\listiparsep@cx
            \topsep\listitopsep@cx\relax
            \itemsep\listiitemsep@cx}
            
      \gdef\@listii{
           \listparindent\listiiparindent@cx
            \leftmargin\leftmarginii
            \parsep\listiiparsep@cx
            \topsep\listiitopsep@cx\relax
            \itemsep\listiiitemsep@cx}
            
      \gdef\@listiii{
            \listparindent\listiiiparindent@cx
            \leftmargin\leftmarginiii
            \parsep\listiiiparsep@cx
            \topsep\listiiitopsep@cx\relax
            \itemsep\listiiiitemsep@cx}
     \gdef\@listiv{
            \listparindent\listivparindent@cx
            \leftmargin\leftmarginiv
            \parsep\listivparsep@cx
            \topsep\listivtopsep@cx\relax
            \itemsep\listivitemsep@cx}       
}

\ExplSyntaxOff
%    \end{macrocode}
%
% \setenumerate{compact2}
% 
% \lorem
% \begin{enumerate}
%  \item This is the first level\par
%         \begin{enumerate}
%            \item{third list}
%            \item {fourth list}
%          \end{enumerate}
%  \end{enumerate}
% 
% \setenumerate{compact1}
%
% \begin{enumerate}
%  \item This is the first level\par
%         \begin{enumerate}
%            \item{third list}
%            \item {fourth list}
%          \end{enumerate}
%  \end{enumerate}

% \section{The Itemize environment}
%
% The standard \latexe defined itemize environment, is much easier to 
% to parameterize, since there are no counters to worry about. However,
% we still need to worry about syntactic sugar for a better user interface.
% 
%    \begin{macrocode}
\ExplSyntaxOn
\cxset
  {
     labelitemi/.code     = \cs_set:Npn \labelitemi   { \color{theitemicolor} #1   },
     labelitemii/.code    = \cs_set:Npn \labelitemii  { \color{theitemiicolor} #1  },
     labelitemiii/.code   = \cs_set:Npn \labelitemiii { \color{theitemiiicolor} #1 },
     labelitemiv/.code    = \cs_set:Npn \labelitemiv  { \color{theitemivcolor} #1  },
  }
\ExplSyntaxOff
%    \end{macrocode}
%    \begin{macrocode}
\def\imgtest{
   \tikz[remember picture,overlay]\node[xshift=0cm, yshift=10pt, below left] (0,0) {\includegraphics[width=\labelwidth]{amato}}; 
   }
\def\dingpoint{\ding{217}}
\def\dingpointi{\ding{226}}
\def\dingpointii{\ding{229}}
\def\dingpointiii{\ding{110}}


\cxset
  {
    labelitemi    = \dingpointi,%\imgtest,%\dingpointi,
    labelitemii   = \dingpointii,
    labelitemiii  = $\iiint$,
    labelitemiv   = \textbullet,
  }
%    \end{macrocode}

% 
% \listparindent1em
%  \begin{itemize}
%   \item \lorem
%   \item \lorem\par
%          
%         This is the second paragraph.
          
%      \begin{itemize}
%         \item Second level
%         \item Second level
%           \begin{itemize}
%              \item Third Level
%              \item Third Level
%                \begin{itemize}
%                  \item Fourth Level
%                    \lorem
%                    \lorem
%                \end{itemize}    
%           \end{itemize}
%      \end{itemize} 
%  \end{itemize}
%
%  \section{The description environment}
%
%    \begin{macrocode}
\ExplSyntaxOn
\cxset
  {
    description~label~font-size/.fontsize     = l_phd_description_label_fontsize,
    description~label~font-weight/.fontweight = l_phd_description_label_fontweight,
    description~label~font-family/.fontfamily = l_phd_description_label_fontfamily,
    description~label~font-shape/.fontstyle   = l_phd_description_label_fontshape,
    description~label~color/.store            = l_phd_description_label_color,
    description~label~sep/.store              = l_phd_description_label_sep,
    description~label~width/.store            = l_phd_description_label_width,
    description~margin~left/.store            = l_phd_description_margin_left,
    description~margin~right/.store           = l_phd_description_margin_right,
    description~item~indent/.store            = l_phd_description_item_indent,
    list~parindent/.store                     = l_phd_description_list_parindent,
  }
\ExplSyntaxOff  
%    \end{macrocode}
%
% We also define a macro \docAuxCmd {setdescription} as a helper macro to assist in changing 
% settings at any point in a document.
%    \begin{macrocode}
\ExplSyntaxOn
\cs_set:Npn \setdescription #1
  {
    \cxset{#1}%
    \renewenvironment{description}
      {
        \list{}
          {
            \listparindent = \l_phd_description_list_parindent
            \leftmargin    = \l_phd_description_margin_left
            \rightmargin   = \l_phd_description_margin_right
            \itemindent    = \l_phd_description_item_indent
            \labelwidth    = \l_phd_description_label_width
            \labelsep      = \l_phd_description_label_sep
            \cs_set_eq:NN \makelabel \description_label:n
          }
      }
      { \endlist }
%    \end{macrocode}
%
% \begin{docCmd}{descriptionlabel} { \marg{label text} }
%   This control sequence is similar to the one defined in the book classes
%   which we overwrite. It is responsible for typesetting the label contents.
% \end{docCmd}
%    \begin{macrocode}
    \cs_set:Npn \description_label:n ##1
    {
      \hspace\l_phd_description_label_sep
    {
      \color{\l_phd_description_label_color}%
      \normalfont\bfseries\raggedleft##1\thickspace%
    }
   }
}%
\ExplSyntaxOff
%    \end{macrocode}
%    \begin{macrocode} 
\setdescription{ 
 description label font-size=normal,
 description label font-weight=bfseries,
 description label font-family=sffamily,
 description label font-shape=upshape,
 description label color=bgsexy,
 description label sep=0sp\relax,
 description label width=100pt,
 description margin left=20pt,
 description margin right=20pt,
 description item indent=90pt,
 list parindent=5em,
}
%    \end{macrocode}
%    \begin{macrocode}
\ExplSyntaxOn
\NewDocumentCommand \NewDescriptionEnvironment { o m }
  {
    \cxset{#1}
    \newenvironment{#2}
     {\list{}{\listparindent\l_phd_description_list_parindent
                       \leftmargin=\l_phd_description_margin_left
                       \rightmargin=\l_phd_description_margin_right
                       \itemindent\l_phd_description_item_indent
                       \labelwidth\l_phd_description_label_width
                       \labelsep=\l_phd_description_label_sep
                       \let\makelabel\description_label:n}}%
               {\endlist}%
%
     \cs_set:Npn \description_label:n ##1
       {

         \hspace\l_phd_description_label_sep
         \color{\l_phd_description_label_color}
         \cs_if_exist_use:cTF {l_phd_description_label_fontfamily}{}{family}
         \cs_if_exist_use:cTF {l_phd_description_label_fontweight}{}{weight}
         \cs_if_exist_use:cTF {l_phd_description_label_fontshape}{}{shape}
         \cs_if_exist_use:cTF {l_phd_description_label_fontsize}{}{size}
        \hbox_to_wd:nn \l_phd_description_label_width {\hss##1\hskip0.5em}%
}}%

\ExplSyntaxOff
%    \end{macrocode}

% \section{In paragraph lists}
%
% Next we handle lists that are inline. That is they are placed within paragraph text.
% This is similar to the \pkgname{paralist} package.

%    \begin{macrocode}
\def\phd@item[#1]{%
  \if@noitemarg
    \@noitemargfalse
    \if@nmbrlist
      \refstepcounter{\@listctr}%
    \fi
  \fi
  \settowidth{\@tempdima}{#1}%
  \ifdim\@tempdima>\z@
    \makelabel{{#1}}
    \nobreakspace
  \fi
  \ignorespaces
  }
%    \end{macrocode}
% 
%    \begin{macrocode}
\ExplSyntaxOn
\cs_set:Npn \inparaenum 
  {
    \ifnum\@enumdepth>3
      \@toodeep
    \else
      \int_incr:N \@enumdepth
      \cs_set:Npx \@enumctr 
         {
           enum\romannumeral\the\@enumdepth
         }
    \fi
    \@ifnextchar[{\@enumlabel@}{\list_in_paragraph_aux:}
}
  
\cs_set:Npn \list_in_paragraph_aux: 
  {
    \usecounter{\@enumctr}
    \cs_set:Npn \@itemlabel 
      {
        \cs:w label\@enumctr \cs_end:
      }
    \cs_set_eq:NN \@item \phd@item
    \cs_set:Npn \makelabel ##1 {##1}
    \ignorespaces
  }
  
\let\endinparaenum\ignorespacesafterend
\ExplSyntaxOff
%    \end{macrocode}

% This paragraph has an enumerated list \begin{inparaenum} \item[a)] test \item test \item test. \end{inparaenum}

%</LISTS>
\endinput