\newacro{Sass}{Syntactically Awesome StyleSheets}
\chapter{Sass}

\section{Introduction to CSS}

\ac{Sass} is an extension of CSS that adds power and elegance to the basic language. It allows you to use variables, nested rules, mixins, inline imports, and more, all with a fully CSS-compatible syntax. Sass helps keep large stylesheets well-organized, and get small stylesheets up and running quickly, particularly with the help of the Compass style library.

\section{Installation}

If you are using windows you may need to install Ruby first.\index{Sass>installation}

\begin{minted}{bash}
gem install sass
\end{minted}

To run Sass from the command line, just use

\begin{minted}{bash}
sass input.scss output.css
\end{minted}

There are also numerous GUI's for \ac{Sass} that they can process a file and change it to normal css.

Installation is normally without much of a problem on all operating systems.

On windows as I mentioned earlier we need to have ruby installed and to install this as a 
|gem|. 

To verify everything is in order type:

\begin{minted}{bash}
sass -version
>Sass 3.4.17 (selective Steve)
\end{minted}
\noindent you should see the current version of your installation.


Sass is a scripting language that is interpreted into Cascading Style Sheets (CSS). SassScript is the scripting language itself. Sass consists of two syntaxes. The original syntax, called "the indented syntax", uses a syntax similar to Haml. It uses indentation to separate code blocks and newline characters to separate rules. The newer syntax, "SCSS", uses block formatting like that of CSS. It uses braces to denote code blocks and semicolons to separate lines within a block. The indented syntax and SCSS files are traditionally given the extensions \docfileextension{.sass} and \docfileextension{.scss} respectively.

\section{Using Sass}

Sass allows variables to be defined. Variables begin with a dollar sign (|$|). Variable assignment is done with a colon (|:|).

SassScript supports four data types:

Numbers (including units)


Strings (with quotes or without)

Colors (name, or names)

Booleans

Variables can be arguments to or results from one of several available functions. During translation, the values of the variables are inserted into the output CSS document.

In SCSS style

\begin{minted}{scss}
$blue: #3bbfce;
$margin: 16px;

.content-navigation {
  border-color: $blue;
  color: darken($blue, 10%);
}

.border {
  padding: $margin / 2;
  margin: $margin / 2;
  border-color: $blue;
}
\end{minted}

Or SASS style

\begin{minted}{sass}
$blue: #3bbfce
$margin: 16px

.content-navigation
  border-color: $blue
  color: darken($blue, 10%)

.border
  padding: $margin/2
  margin:  $margin/2
  border-color: $blue
\end{minted}


\subsection{Processing}

Processing the code then produces the \docfileextension{.css} file, ready for deployment. 

\begin{minted}{css}
.content-navigation {
  border-color: #3bbfce;
  color: #2b9eab;
}

.border {
  padding: 8px;
  margin: 8px;
  border-color: #3bbfce;
}
\end{minted}


Personally I am not too sure if a preprocessor is always a good idea. There was a lot of criticism when the tools came out.\footnote{\protect\url{http://blog.millermedeiros.com/the-problem-with-css-pre-processors/}}. My advice if you are new to web development and css, using a pre-processor
will actually slow down your learning not accelerate it. Having said that, there is no escape from
it as most css frameworks probably need a preprocessor.









