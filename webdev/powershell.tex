\chapter{Powershell}
\setminted{fontsize=\footnotesize,
 breaklines,
 breakautoindent=false,
% breaksymbolleft=\raisebox{0.8ex}{
% \small\reflectbox{\cr}
 %},
 breaksymbolindentleft=0pt,
 breaksymbolsepleft=0pt,
%breaksymbolright=\small\cr,
 breaksymbolindentright=0pt,
 breaksymbolsepright=0pt,
 mathescape,
 linenos,
 numbersep=5pt,
 %frame=single,
 numbersep=5pt,
 xleftmargin=0pt,}
 
\setminted[PowerShell]{fontsize=\footnotesize,
 breaklines,
 breakautoindent=false,
% breaksymbolleft=\raisebox{0.8ex}{
% \small\reflectbox{\cr}
 %},
 breaksymbolindentleft=0pt,
 breaksymbolsepleft=0pt,
%breaksymbolright=\small\cr,
 breaksymbolindentright=0pt,
 breaksymbolsepright=0pt,
 mathescape,
 linenos,
 numbersep=5pt,
 %frame=single,
 numbersep=5pt,
 xleftmargin=0pt,}
 
Poweshell is a crazy idea about scripting windows. It has a cumbersome syntax, does not come with any good guide
and is full of traps and errors and definitely not free software.

A full list of all guides on the internet can be found \href{http://stackoverflow.com/questions/9956109/looking-for-quick-and-complete-powershell-tutorial}{powershelltuts}.

\begin{minted}{PowerShell}
$string = "TESTING123"
$string = $string.substring("6")
$string
G123
\end{minted}

This gives |G123|.

\begin{minted}{PowerShell}
$string = "TESTING123"
$string = $string.substring("2","6")
$string
STING1
\end{minted}

\section{A full program}
\makeatletter
\expandafter\gdef\csname PY@tok@err\endcsname{}

\expandafter\def\csname PYGdefault@tok@err\endcsname{\def\PYGdefault@bc##1{{\strut ##1}}}
\makeatother
\begin{minted}[fontsize=\footnotesize]{PowerShell}
# npm-windows-upgrade
# This script updates npm on Windows, making sure that the npm installed
# with `npm install -g npm@latest` is actually the first npm found in PATH
# (C) Copyright 2015 Microsoft, developed by Felix Rieseberg
# felix.rieseberg@microsoft.com

# ----------------------------------------------------------------------
# Usage: ./upgrade-npm.ps1
# ----------------------------------------------------------------------
[CmdletBinding()]
Param(
    [Parameter(Mandatory=$True)]
    [string]$version,
    [string]$NodePath=(Join-Path $env:ProgramFiles nodejs)
)

$ErrorActionPreference = "Stop"

#
# Self-Elevate
# ---------------------------------------------------------------------------------------------------------
function IsAdministrator
{
    $Identity = [System.Security.Principal.WindowsIdentity]::GetCurrent()
    $Principal = New-Object System.Security.Principal.WindowsPrincipal($Identity)
    $Principal.IsInRole([System.Security.Principal.WindowsBuiltInRole]::Administrator)
}

function IsUacEnabled
{
    (Get-ItemProperty HKLM:\Software\Microsoft\Windows\CurrentVersion\Policies\System).EnableLua -ne 0
}

function UpdateNpm($PassedNodePath)
{
    $NpmPath = (Join-Path $PassedNodePath "node_modules\npm")
    Write-Debug "Assuming npm in $NpmPath"

    if (Test-Path $PassedNodePath)
    {
        # Create tmp directory, delete files if they exist
        $TempPath = "$env:temp\npm_upgrade"
        if ((Test-Path $TempPath) -ne $True)
        {
            New-Item -ItemType Directory -Force -Path $TempPath
        }

        # Copy away .npmrc
        $Npmrc = $False
        if (Test-Path $NpmPath)
        {
            cd $NpmPath

            if (Test-Path .npmrc)
            {
                $Npmrc = $True
                Write-Debug "Saving .npmrc"
                Copy-Item .npmrc $TempPath
            }
        }

        # Upgrade npm
        cd $PassedNodePath
        Write-Debug "Upgrading npm in $PassedNodePath"
        .\npm install npm@$version

        # Copy .npmrc back
        if ($Npmrc)
        {
            Write-Debug "Restoring .npmrc"
            $TempFile = "$TempPath\.npmrc"
            Copy-Item $TempFile $NpmPath -Force
        }

        "All done!"
    } else
    {
        "Could not find installation location (assumed in $PassedNodePath) - aborting upgrade"
    }
}

if (!(IsAdministrator))
{
    "Please restart this script from an administrative PowerShell!"
    "NPM cannot be upgraded without administrative rights. To run PowerShell as Administrator,"
    "right-click PowerShell and select 'Run as Administrator'"
    return
}

#
# Upgrade
# ---------------------------------------------------------------------------------------------------------
$AssumedNpmPath = (Join-Path $NodePath "node_modules\npm")

if ((Test-Path $AssumedNpmPath) -ne $True)
{
    $NodePath = (Join-Path $env:ProgramFiles nodejs)
    # If the user installed an x86 version of NodeJS on an x64 system, the NodeJS installation will be found
    # in env:ProgramFiles(x86)
    if ((Test-Path $NodePath) -ne $True) {
        if (Test-Path "Env:ProgramFiles(x86)")
        {
            $NodePath = (Join-Path ${env:ProgramFiles(x86)} nodejs)
        }
        else
        {
            "We could not find npm - aborting upgrade."
            "To manually tell npm-version-upgrade where to install npm,"
            'run this script with the parameter --npmPath:"C:\MyNPMLocation\"'
            return
        }
    }
}

UpdateNpm($NodePath)
\end{minted}

\section{Snippets}

\subsection{What version is installed}

You can find the version of the PowerShell by using \mintinline{PowerShell}{$PSVersionTable}

\begin{minted}[fontsize=\footnotesize]{PowerShell}
$PSVersionTable

Name                           Value                                                                                   
----                           -----                                                                                   
CLRVersion                     2.0.50727.5485                                                                          
BuildVersion                   6.1.7601.17514                                                                          
PSVersion                      2.0                                                                                     
WSManStackVersion              2.0                                                                                     
PSCompatibleVersions           {1.0, 2.0}                                                                              
SerializationVersion           1.1.0.1                                                                                 
PSRemotingProtocolVersion      2.1                                                                                     

\end{minted}

\subsection{Where is it installed}

It is always |C:\Windows\System32\WindowsPowershell\v1.0|. It was left like that for backward compability is what I heard or read somewhere

\begin{minted}[fontsize=\footnotesize]{PowerShell}
$PShome
C:\Windows\System32\WindowsPowerShell\v1.0
\end{minted}



\begin{Verbatim}[gobble=2,fontsize=\footnotesize,fontfamily=tt]
   Verbatim line.
\end{Verbatim}

\VerbatimFootnotes
We can put verbatim\footnote{\verb+_Yes!_+} text in footnotes.

