
\clearpage
\section{Sectioning Commands - File F ltsect.dtx}

This file defines the declarations such as \cmd{author} which are used by \cmd{maketitle}.
|\maketitle| itself is defined by each class, not in the \latex kernel.
The second part of the file defines the generic commands used for defining sectioning
commands such as |\chapter|. Again the actual document level commands
are defined in the class files, in terms of these commands


The \cmd{@startsection} as the manual says is the mother of all sectioning commands.

The |\@startsection{name}{level}{indent}{beforeskip}|
|{afterskip}{hstyle}*[altheading]{heading}| command is the mother of all
the user level sectioning commands. The part after the *, including the * is
optional

\paragraph{name:} \eg subsection
\paragraph{level}: a number, denoting depth of section - eg., chapter=1, section=2, etc.
\paragraph{indent:} Indentation of heading from left margin

