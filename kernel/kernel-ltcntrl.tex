\chapter{Control Structures}

\label{ch:ltcntrl}

Looping structures are an absolute necessity to handle lists and other constructions.
As \tex is so different from traditional programming languages, macro writers propose
to harnass TEX into a more familiar system, by imposing
syntaxes borrowed from various successful highlevel
programming languages. 

\citeauthor{laan1992}  claims that this sugaring does injustice
to TEX’s nature might result, and users might become
intimidated, because of the difficult—at least unusual—
encoding used to achieve the aim. The more so when
functional equivalents are already there, although perhaps
hidden, and not tagged by familiar names.\footcite{laan1992} 

Leslie Lamport provided a somewhat limited amount of macros, but adequate for
\latex. It must be emphasized that \pkgname{expl3} did a tremendous job in providing
tens of alternatives in a more organized manner. If you need to program any serious amount of code
you are advised to move over to these. The \pkgname{etoolbox} package\footcite{etoolbox} also provides many
alternatives. However it is advisable to go over these older methods in order to understand
the rest of the kernel.\footcite{Kabelschacht1987}
%
% \begin{oldcomments}
\begin{teX}
%<*2ekernel>
\message{control,}
\end{teX}
%
% \@whilenum TEST \do {BODY}
% \@whiledim TEST \do {BODY}  : These implement the loop
%           while  TEST  do  BODY  od
%     where  TEST  is a TeX \ifnum or \ifdim test, respectively.
%     They are optimized for the normal case of TEST initially false.
%
% \@whilesw SWITCH \fi {BODY} : Implements the loop
%               while SWITCH do BODY od
%     Optimized for normal case of SWITCH initially false.
%
% \@for NAME := LIST \do {BODY} : Assumes that LIST expands to A1,A2,
%      ... ,An .
%      Executes  BODY  n  times, with  NAME = Ai  on the i-th iteration.
%      Optimized for the normal case of n = 1.  Works for n=0.
%
% \@tfor NAME := LIST \do {BODY}
%      if, before expansion, LIST = T1 ... Tn  where each Ti is a
%      token or {...}, then executes  BODY  n  times, with  NAME = Ti
%      on the i-th iteration.  Works for n=0.
%
%  NOTES: 1. These macros use no \@temp sequences.
%         2. These macros do not work if the body contains anything that
%         looks syntactically to TeX like an improperly balanced \if
%         \else \fi.
%
% \@whilenum TEST \do {BODY} ==
%  BEGIN
%    if  TEST
%      then  BODY
%            \@iwhilenum{TEST \relax BODY}
%  END
%
% \@iwhilenum {TEST BODY} ==
%  BEGIN
%    if  TEST
%      then  BODY
%            \@nextwhile = def(\@iwhilenum)
%      else  \@nextwhile = def(\@whilenoop)
%    fi
%    \@nextwhile {TEST BODY}
%  END
%
% \@whilesw SWITCH \fi {BODY} ==
%  BEGIN
%    if SWITCH
%      then BODY
%           \@iwhilesw {SWITCH BODY}\fi
%    fi
%  END
%
% \@iwhilesw {SWITCH BODY} \fi ==
%  BEGIN
%    if SWITCH
%      then BODY
%           \@nextwhile = def(\@iwhilesw)
%      else \@nextwhile = def(\@whileswnoop)
%    fi
%    \@nextwhile {SWITCH BODY} \fi
%  END
%
% \end{oldcomments}
%
\begin{docCmd}{@whilenoop}{}
\begin{docCmd}{@whilenum}{}
\begin{docCmd}{@iwhilenum}{}
% \changes{v1.0f}{1995/07/09}{Reimplemented using Kabelschacht method}
% \changes{v1.0g}{1995/08/16}{Removed \cs{@whilenoop}}
% \changes{v1.0g}{1995/08/16}{Made defs long}
\begin{teX}
\long\def\@whilenum#1\do #2{\ifnum #1\relax #2\relax\@iwhilenum{#1\relax
     #2\relax}\fi}
\long\def\@iwhilenum#1{\ifnum #1\expandafter\@iwhilenum
         \else\expandafter\@gobble\fi{#1}}
\end{teX}
\end{docCmd}
\end{docCmd}
\end{docCmd}
%
\begin{docCmd}{@whiledim}{}
\begin{teX}
\long\def\@whiledim#1\do #2{\ifdim #1\relax#2\@iwhiledim{#1\relax#2}\fi}
\end{teX}
\end{docCmd}

\begin{docCmd}{@iwhiledim}{}
\begin{teX}
\long\def\@iwhiledim#1{\ifdim #1\expandafter\@iwhiledim
        \else\expandafter\@gobble\fi{#1}}
\end{teX}
\end{docCmd}
%
\begin{docCmd}{@whileswnoop}{}
\begin{docCmd}{@whilesw}{}
\begin{docCmd}{@iwhilesw}{}
% \changes{v1.0f}{1995/07/09}{Reimplemented using Kabelschacht method}
% \changes{v1.0g}{1995/08/16}{Removed \cs{@whileswnoop}}
\begin{teX}
\long\def\@whilesw#1\fi#2{#1#2\@iwhilesw{#1#2}\fi\fi}
\long\def\@iwhilesw#1\fi{#1\expandafter\@iwhilesw
         \else\@gobbletwo\fi{#1}\fi}
\end{teX}
\end{docCmd}
\end{docCmd}
\end{docCmd}
%
% \begin{oldcomments}
%
% \@for NAME := LIST \do {BODY} ==
%    BEGIN \@forloop expand(LIST),\@nil,\@nil \@@ NAME {BODY} END
%
% \@forloop CAR, CARCDR, CDRCDR \@@ NAME {BODY} ==
%   BEGIN
%     NAME = CAR
%     if def(NAME) = def(\@nnil)
%       else BODY;
%            NAME = CARCDR
%            if def(NAME) = def(\@nnil)
%              else BODY
%                   \@iforloop CDRCDR \@@ NAME \do {BODY}
%            fi
%     fi
%   END
%
% \@iforloop CAR, CDR \@@ NAME {BODY} =
%     NAME = CAR
%     if def(NAME) = def(\@nnil)
%        then  \@nextwhile = def(\@fornoop)
%        else  BODY ;
%              \@nextwhile = def(\@iforloop)
%     fi
%     \@nextwhile name cdr {body}
%
% \@tfor NAME := LIST \do {BODY}
%    =  \@tforloop LIST \@nil \@@ NAME {BODY}
%
% \@tforloop car cdr \@@ name {body} =
%     name = car
%     if def(name) = def(\@nnil)
%        then  \@nextwhile == \@fornoop
%        else  body ;
%              \@nextwhile == \@forloop
%     fi
%     \@nextwhile name cdr {body}
% \end{oldcomments}
%
\begin{docCmd}{@nnil}{}
\begin{teX}
\def\@nnil{\@nil}
\end{teX}
\end{docCmd}
%
\begin{docCmd}{@empty}{}
\begin{teX}
\def\@empty{}
\end{teX}
\end{docCmd}
%
\begin{docCmd}{@fornoop}{}
% \changes{v1.0g}{1995/08/16}{Made defs long}
% \changes{v1.0h}{2007/08/06}{Really make defs long}
\begin{teX}
\long\def\@fornoop#1\@@#2#3{}
\end{teX}
\end{docCmd}
%
\begin{docCmd}{@for}{}
% \changes{v1.0d}{1995/04/24}
%      {Dont expand second argument with \cs{edef}: /1317 (DPC)}
\begin{teX}
\long\def\@for#1:=#2\do#3%
  {%
    \expandafter\def\expandafter\@fortmp\expandafter{#2}%
      \ifx\@fortmp\@empty%
    \else
      \expandafter\@forloop#2,\@nil,\@nil\@@#1{#3}%
    \fi
  }
\end{teX}
\end{docCmd}
%
\begin{docCmd}{@forloop}{}
\begin{teX}
\long\def\@forloop#1,#2,#3\@@#4#5{\def#4{#1}\ifx #4\@nnil \else
       #5\def#4{#2}\ifx #4\@nnil \else#5\@iforloop #3\@@#4{#5}\fi\fi}
\end{teX}
\end{docCmd}
%
\begin{docCmd}{@iforloop}{}
% \changes{v1.0f}{1995/07/09}{Reimplemented using Kabelschacht method}
% \changes{v1.0g}{1995/08/16}{Made defs long}
\begin{teX}
\long\def\@iforloop#1,#2\@@#3#4{\def#3{#1}\ifx #3\@nnil
       \expandafter\@fornoop \else
      #4\relax\expandafter\@iforloop\fi#2\@@#3{#4}}
\end{teX}
\end{docCmd}
%
\begin{docCmd}{@tfor}{ \marg{next token}\marg{list}\marg{body code}}
The looping structure iterates over each token in a list. It uses
the Kabelschacht method. Note that you may have to expand the list
if in a macro, using an appropriate number of |\expandafter| commands.

\begin{texexample}{tfor}{ex:tfor}
\makeatletter
\def\alist{ABCDE FGH}
\expandafter\@tfor\expandafter\next%
 \expandafter:\expandafter=\alist\do{%
  [\next]%
}
\makeatother
\end{texexample}

\begin{teX}
\def\@tfor#1:={\@tf@r#1 }

\long\def\@tf@r#1#2\do#3{\def\@fortmp{#2}\ifx\@fortmp\space\else
    \@tforloop#2\@nil\@nil\@@#1{#3}\fi}
    
\long\def\@tforloop#1#2\@@#3#4{\def#3{#1}\ifx #3\@nnil
       \expandafter\@fornoop \else
      #4\relax\expandafter\@tforloop\fi#2\@@#3{#4}}
\end{teX}
Note that it ignore any spaces in the list
\end{docCmd}
%
\begin{docCmd}{@break@tfor}{}
 Break out of a |\@tfor| loop. This should be called \emph{inside}
 the scope of an |\if|. See |\@iffileonpath| for an example.
\begin{teX}
\long\def\@break@tfor#1\@@#2#3{\fi\fi}
\end{teX}
\end{docCmd}
%
\begin{docCmd}{@removeelement}{}
    Removes an element from a comma-separated list and puts it into
    a control sequence, called as
    |\@removeelement{|\meta{element}|}{|\meta{list}|}{|\meta{cs}|}|.
    Due to the implementation method the \meta{element} is not allowed
    to contain braces.
\begin{teX}
\def\@removeelement#1#2#3{%
  \def\reserved@a##1,#1,##2\reserved@a{##1,##2\reserved@b}%
  \def\reserved@b##1,\reserved@b##2\reserved@b{%
    \ifx,##1\@empty\else##1\fi}%
  \edef#3{%
    \expandafter\reserved@b\reserved@a,#2,\reserved@b,#1,\reserved@a}}
\end{teX}
\end{docCmd}

%\Finale
\endinput
