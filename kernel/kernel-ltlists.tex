\chapter{ltlists.dtx}
         
 \section{List, and related environments}

This section of the kernel creates generic factory commands for the creation of list environments. 
These list environments \refEnv{enumerate}, \refEnv{itemize} are defined in the kernel, whereas others are normally
defined in the class files (see \vref{ch:bookclass}, \vref{sec:clo} and \vref{sec:booklists}).
To understand the properly we need to note that lists are shaped paragraphs, which Lamport
decided to shape with \cs{parshape}. By shaping the paragraph the geometry of the
paragraphs can be achieved. The complexity of the code is necessary to make sure
that at every item the |\parshape| is used and also in any following paragraphs. Remember
that |\parshape| will only act on one paragraph. As Knuth said the \latexe Team did some
nifty tricks with |\parshape|.

 The generic commands for creating an indented environment --
 |enumerate|, |itemize|, |quote|, etc -- are:
 \begin{quote}
        |\list|\marg{label}\marg{commands} ... |\endlist|
 \end{quote}

 which can be invoked by the user as the |list| environment.  The \meta{label}
 argument specifies item labeling.  \meta{commands} contains commands for
 changing the horizontal and vertical spacing parameters.

 Each item of the environment is begun by the command
 |\item[|item label|]|
 which produces an item labeled by |itemlabel|.  If the argument is
 missing, then the |label| argument of the |\list| command is used as the
 item label. This enables the user to type |item|  or |item[some label]|.

 The label is formed by putting |\makelabel|\marg{ITEMLABEL} in an hbox
 whose width is either its natural width or else |\labelwidth|,
 whichever is larger.  The |\list| command defines \refCom{makelabel} to have
 the default  definition:
 \begin{quote}
     |\makelabel|\marg{arg} == BEGIN |\hfil| arg END
 \end{quote}
 which, for a label of width less than \refCom{labelwidth}, puts the label
 flushright, \refCom{labelsep} to the left of the item's text.  However,
 |\makelabel| can be |\let| to another command by the \refCom{list}'s
\meta{commands} argument.

 A \refCom{usecounter}\marg{foo} command in the second argument causes the
 counter \emph{foo} to be initialized to zero, and stepped by every
 |\item| command without an argument.  (|\label| commands within the
 list refer to this counter.)

 When you leave a list environment, returning either to an enclosing
 list or normal text mode, \latexe begins a new paragraph if and only if
 you leave a blank line after the |\end| command.  This is accomplished
 by the \refCom{@endparenv} command.

 Blank lines are ignored every other reasonable place--i.e.:
 \begin{itemize}
  \item  Between the |\begin{list}| and the first |\item|,
  \item  Between the |\item| and the text of that item.
  \item Between the end of the last item and the |\end{list}|.
 \end{itemize}

 For an environment like \docAuxEnvironment{quotation}, in which items are not labeled,
 the entire environment is a single item.  It is defined by
 letting |\quotation| == |\list{}{...}\item\relax|.  (Note the
 |\relax|, there in case the first character in the environment is a
 '['.)  The spacing parameters provide a great deal of flexability in
 designing the format, including the ability to let the indentation of
 the first paragraph be different from that of the subsequent ones.

 The trivlist environment is equivalent to a list environment
 whose second argument sets the following parameter values:
 \begin{description}
 \item[\refCom{leftmargin} = 0:] causes no indentation of left margin
 \item[\refCom{labelwidth} = 0:] see below for precise effect this has.
 \item[\refCom{itemindent} = 0:] with a null label, makes first paragraph
        have no indentation.  Succeeding paragraphs have |\parindent|
        indentation.  To give first paragraph same indentation, set
        |\itemindent| = |\parindent| before the |\item[]|.
 \end{description}

 Every |\item| in a trivlist environment must have an argument---in
 many cases, this will be the null argument (|\item[]|).  The trivlist
 environment is mainly used for paragraphing environments, like
 verbatim, in which there is no margin change.  It provides the same
 vertical spacing as the list environment, and works reasonably well
 when it occurs immediately after an |\item| command in an enclosing
 list.



 \subsection{List and Trivlist}

 The following variables are used inside a list environment:
 \begin{description}
 \item[] \refCom{@totalleftmargin} The distance that the prevailing left
     margin is indented from the outermost left margin,
 \item[] \refCom{linewidth} The width of the current line.  Must be
     initialized to |\hsize|.
 \item[] \refCom{@listdepth} A count for holding current list nesting depth.
 
 \item[] \refCom{makelabel} A macro with a single argument, used to
   generate the label from the argument (given or implied)
   of the |\item| command. Initialized to |\@mklab| by the |\list|
   command.  This command must produce  some stretch---i.e., an
   |\hfil|.
   
 \item[] \refCom{if@inlabel} A switch that is false except between the time
   an |\item| is encountered and the time that \TeX{}
   actually enters horizontal mode.  Should be tested by commands
   that can be messed up by the list environment's use of |\everypar|.
   
 \item[] \refCom{@labels} When |@inlabel = true|, it holds the labels
   to be put out by |\everypar|.
   
 \item[] \refCom{if@noparitem} A switch set by |\list| when
   |@inlabel = true|.
      Handles the case of a |\list| being the first thing in an item.
      
 \item[] \refCom{if@noparlist} A switch set true for a list that begins an
   item.  No |\topsep| space is added before or after |\item|'s such a
   list.
   
 \item[] \refCom{if@newlist} Set true by |\list|, set false by the first
   text (by |\everypar|).
   
 \item[]  \refCom{if@noitemarg} Set true when executing an |\item| with no
   explicit argument.  Used to save space. To save time, make two
   separate  \refCom{@item} commands.
   
 \item[] \refCom{if@nmbrlist} Set true by |\usecounter| command, causes
   list to be numbered.
   
 \item[] |noskipsec| A switch set true by a sectioning command when
    it is creating an in-text heading with |\everypar|.
 \end{description}


 Throughout a list environment, |\hsize| is the width of the current
 line, measured from the outermost left margin to the outermost right
 margin.  Environments like tabbing should use \refCom{linewidth} instead of
 |\hsize|.

 Here are the parameters of a list that can be set by commands in
 the |\list|'s \meta{commands} argument.  These parameters are all TeX
 skips or dimensions (defined by |\newskip| or |\newdimen|), so the
 usual \TeX\ or \LaTeX\ commands can be used to set them.  The
 commands will be executed in vmode if and only if the |\list| was
 preceded by a |\par| (or something like an |\end{list}|), so the
 spacing parameters can be set according to whether the list is
 inside a paragraph or is its own paragraph.


 \subsection{Vertical Spacing (skips)}
 \begin{description}
 \item[]  \refCom{topsep} Space between first item and preceding paragraph.
 \item[] \refCom{partopsep} Extra space added to \refCom{topsep} when
        environment starts a new paragraph (is called in vmode).
 \item[] \refCom{itemsep} Space between successive items.
 \item[] \refCom{parsep} Space between paragraphs within an item -- the
                 \refCom{parskip} for this environment.
 \end{description}

 \subsection{Penalties}
 \begin{description}

 \item[] \refCom{@beginparpenalty} put at the beginning of a list
 \item[] \refCom{@endparpenalty} put at end of list
  \item[] \refCom{@itempenalty} put between items.
  \end{description}

 \subsection{Horizontal Spacing (dimens)}
 \begin{description}
 \item[]  \refCom{leftmargin} space between left margin of enclosing
   environment (or of page if top level list) and left margin of
                     this list.  Must be nonnegative.
  \item[] \refCom{rightmargin} analogous.
  \item[] \refCom{listparindent} extra indentation at beginning of every
     paragraph of a list except the one started by the \refCom{item}
                      command.  May be negative!  Usually, labeled
                       lists have \refCom{listparindent} equal to zero.
   \item[]  \refCom{itemindent} extra indentation added right BEFORE an item
                      label.
  \item[] \refCom{labelwidth} nominal width of box that contains the label.
                      If the natural width of the
                         label $< =$ \refCom{labelwidth},
                      then the label is flushed right inside a box
                      of width \refCom{labelwidth} (with an |\hfil|).
                      Otherwise,
                      a box of the natural width is employed, which
                       causes an indentation of the text on that line.
     \item[] \refCom{labelsep} space between end of label box and text of
                      first item.
  \end{description}


 \subsection{Default Values}
 
 Defaults for the list environment are set as follows.
 First, \refCom{rightmargin}, \refCom{listparindent} and \refCom{itemindent}
 are set
      to 0pt.  Then, one of the commands
      \refCom{@listi}, \refCom{@listii}, ... , \refCom{@listvi}
      is called, depending upon the current level of the list.
      The \refCom{@list} \ldots commands should be defined by the document
      style.  A convention that the document style should follow is
      to set \refCom{leftmargin} to
      \refCom{leftmargini},\ldots, \refCom{leftmarginiv} for
      the appropriate level.  Items that aren't changed may be left
      alone, but everything that could possibly be changed must be
      reset.


\LinesNumbered
\begin{algorithm}
\caption{The \refCom{list} environment}
  \refCom{list}\marg{LABEL}\marg{COMMANDS} ==\\
   \Begin{
     \eIf{\refCom{@listdepth} > 5}{
        LaTeX error: 'Too deeply nested'}{
        \refCom{@listdepth} :=G \refCom{@listdepth} + 1\\
     }
     \refCom{rightmargin}     := 0pt\\
     \refCom{listparindent}   := 0pt\\
     \refCom{itemindent}      := 0pt\\
     eval(@list \refCom{romannumeral}\refCom{the}\refCom{@listdepth})\\  
     \refCom{@itemlabel}      :=L LABEL\\
     \refCom{makelabel}       == \refCom{@mklab}\\
     @nmbrlist        :=L false\\
     COMMANDS\\
     \refCom{@trivlist}\\  
     parskip          :=L \refCom{parsep}\\
     parindent        :=L \refCom{listparindent}\\
     \refCom{linewidth}        :=L \refCom{linewidth} - \refCom{rightmargin} -\refCom{leftmargin}\\
     \refCom{@totalleftmargin} :=L \refCom{@totalleftmargin} + \refCom{leftmargin}\\
     parshape 1 \refCom{@totalleftmargin} \refCom{linewidth}\\
     \cs{ignorespaces} 
   }
\end{algorithm}

The \refCom{endlist} simply adjusts the listdepth counter and ends the \refCom{trivlist}.

\begin{algorithm}
 \refCom{endlist} == \\
  \Begin{
    \refCom{@listdepth} :=G \refCom{@listdepth} -1\\
    \refCom{endtrivlist}\\
  }
\end{algorithm}

The \refCom{@trivlist} is define as,

\begin{algorithm}
 \refCom{@trivlist} ==\\
  \Begin{
    \If{@newlist = T}{\refCom{@noitemerr}}
     This command removed for some forgotten reason.\\
     \refCom{@topsepadd} :=L \refCom{topsep}\\
     \If{@noskipsec}{leave vertical mode}
     \eIf{vertical mode}{
        \refCom{@topsepadd} :=L \refCom{@topsepadd} + \refCom{partopsep}}{
        \cs{unskip} \cs{par}}
     \eIf{@inlabel = true}{
         @noparitem :=L true
         @noparlist :=L true}{
         @noparlist :=L false
             \refCom{@topsep}   :=L \refCom{@topsepadd}}
     \refCom{@topsep}      :=L \refCom{@topsep} + \refCom{parskip}\\
      Restore paragraphing parameters\\
     \cs{leftskip}     :=L 0pt\\  
     \cs{rightskip}    :=L \refCom{@rightskip}\\
     \cs{parfillskip}     :=L 0pt + 1fil\\
   NOTE: \cs{@setpar} called on every \refCom{list} in case \cs{par} has been\\
   temporarily  munged before the \refCom{list} command.\\
     \cs{@setpar}{if @newlist = false then {\cs{@@par}} fi}\\
     \refCom{@newlist}         :=G T\\
     \refCom{@outerparskip}    :=L \cs{parskip}\\
 }
\end{algorithm}

\begin{algorithm}
 \refCom{trivlist}  ==\\
 \Begin{
  \refCom{parsep} := \refCom{parskip}\\
   @nmbrlist := F\\
  \refCom{@trivlist}\\
  \refCom{labelwidth} := 0\\
  \refCom{leftmargin} := 0\\
  \refCom{itemindent} := \refCom{parindent}\\
  \refCom{@itemlabel} :=L "empty"\\ 
  \refCom{makelabel}\marg{LABEL} == LABEL\\
 }
\end{algorithm}



\begin{algorithm}
 \refCom{endtrivlist} ==\\
 \Begin{
     \If{@inlabel = T}{\refCom{indent}}
     \If{horizontal mode}{\refCom{unskip} \refCom{par}}
     \eIf{@noparlist = true}{}{
        \If{\refCom{lastskip} > 0}{
              \refCom{@tempskipa} := \refCom{lastskip}
              \refCom{vskip} - \refCom{lastskip}
              \refCom{vskip} \refCom{@tempskipa} -\refCom{@outerparskip} + \refCom{parskip}
             }
           \refCom{@endparenv}
     }
   }
\end{algorithm}


\begin{algorithm}
 \refCom{@endparenv} ==
   \Begin{
    \refCom{addpenalty}\marg{@endparpenalty}\\
    \refCom{addvspace}\marg{\refCom{@topsepadd}}\\
     ends the \refCom{begin} command's \refCom{begingroup}\\
    \refCom{endgroup}\\
    \refCom{par}  ==  \Begin{%
                  \refCom{@restorepar}\\
                  \refCom{everypar}|{}|
                  \refCom{par}
                  }
    \refCom{everypar} == \Begin{remove \refCom{lastbox} \refCom{everypar}|{}|}
   to match the \refCom{end} commands \refCom{endgroup}\\
    \refCom{begingroup}  
   }
\end{algorithm}

\index{kernel>lists>\textbackslash item}
The definition of item, is fairly simple deferring the complexity
to \refCom{@item} which follows.

\begin{algorithm}[htbp]
 \refCom{item} == \Begin{
    \If{math mode}{issue warning}
    \eIf{ next char = {\tt [}}{
             \refCom{@item}}{
              \refCom{@noitemarg} := true\\
             \refCom{@item}[\refCom{@itemlabel}]}
  }
\caption{The algorithm for \textbackslash item}
\end{algorithm}


\begin{algorithm}
 \refCom{@item}[LAB] ==
    \Begin{
     \eIf{@noparitem = true}{
        @noparitem := false
             % NOTE: then clause  hardly every taken,\\
             %  so made a macro \refCom{@donoparitem}\\
            \refCom{box}\refCom{@labels} :=G\\
             \refCom{hbox} \Begin{\refCom{hskip} -\refCom{leftmargin}\\
                                   \refCom{box}\\refCom{@labels}\\
                                   \refCom{hskip} \refCom{leftmargin}}
            \If{@minipage = false}{
               \refCom{@tempskipa} := \refCom{lastskip}\\
               \refCom{vskip} -\refCom{lastskip}\\
               \refCom{vskip} \refCom{@tempskipa} + \refCom{@outerparskip} - \refCom{parskip}}
            }{
          \If{@inlabel = true}{
              then \refCom{indent} \refCom{par}   % previous item empty.
           }
           \If{hmode}{then 2 \refCom{unskip}'s\\
                           % To remove any space at end of prev.\\
                           % paragraph that could cause a blank line.\\
                     \refCom{par}\\
           }
           \eIf{if @newlist = T}{
                \eIf{@nobreak = T}{ 
                      % Kludge if list follows \refCom{section}\\
                      \refCom{addvspace}\marg{\refCom{@outerparskip} - \refCom{parskip}}}{
                       \refCom{addpenalty}\marg{\refCom{@beginparpenalty}}\\
                       \refCom{addvspace}\marg{\refCom{@topsep}}\\
                       \refCom{addvspace}\marg{-\refCom{parskip}}\\  
                    }}{
                \refCom{addpenalty}\marg{\refCom{@itempenalty}}\\
                \refCom{addvspace}\marg{\refCom{itemsep}}\\
            }

            @inlabel :=G true\\
     }

     \refCom{everypar}\{ @minipage :=G F\\
                @newlist :=G F\\
                \If{@inlabel = true}{
                    @inlabel :=G false
                       \refCom{hskip} -\refCom{parindent}\\
                       \refCom{box}\refCom{@labels}\\
                       \refCom{penalty} 0\\
                       \refCom{box}\refCom{@labels} :=G null\\
                }
                \refCom{everypar}\{\} \}\\
     @nobreak :=G false\\
     \If{@noitemarg = true}{
        @noitemarg := false\\
            \If{@nmbrlist}{
               \refCom{refstepcounter}\{\refCom{@listctr}\}
            }
     }
     \refCom{@tempboxa}   :=L \refCom{hbox}\marg{\refCom{makelabel}\marg{LAB}}\\
     \refCom{box}\refCom{@labels} :=G \refCom{@labels}\refCom{hskip}\refCom{itemindent}\\
                       \refCom{hskip} - (\refCom{labelwidth} + \refCom{labelsep})\\
                 \eIf{\refCom{wd}\refCom{@tempboxa} > \refCom{labelwidth}}{
                          \refCom{box}\refCom{@tempboxa}}{
                          \refCom{hbox} to \refCom{labelwidth}\{\refCom{unhbox}\refCom{@tempboxa}\}\\
                 } 
      \refCom{hskip}\refCom{labelsep}\\
     \refCom{ignorespaces} %gobble space up to text
  }
\end{algorithm}




\begin{algorithm}
 \refCom{makelabel}\marg{LABEL} == ERROR\\
    default to catch lonely \refCom{item}

 \refCom{usecounter}\marg{CTR} == \Begin{
                              \refCom{@nmbrlist} :=L true\\
                              \refCom{@listctr} == CTR\\
                              \refCom{setcounter}\marg{CTR}\marg{0}
                             }
\end{algorithm}



 

\begin{docCommand}{topsep} {\meta{dim}}
The space between the first item and the preceding paragraph.
\end{docCommand}

\begin{docCommand}{partopsep} {\meta{dim}}
The space between the first item and the preceding paragraph.
\end{docCommand}

\begin{docCommand}{itemsep} {\meta{dim}}
The space between the first item and the preceding paragraph.
\end{docCommand}

\begin{docCommand}{@itemsep} {\meta{dim}}
The space between the first item and the preceding paragraph.
\end{docCommand}

\begin{docCommand}{@topsepadd} {\meta{dim}}
The space between the first item and the preceding paragraph.
\end{docCommand}

\begin{docCommand}{outerparskip} {\meta{dim}}
The space between the first item and the preceding paragraph.
\end{docCommand}

\begin{docCommand}{parsep} {\meta{dim}}
The space between the first item and the preceding paragraph.
\end{docCommand}

The skip registers \docAuxCommand{topskip}, \docAuxCommand*{partopsep}, 
\docAuxCommand{itemsep}, \docAuxCommand{parsep}, \docAuxCommand{@topsep},
\docAuxCommand{@topsepadd}, \docAuxCommand{outerparskip} are created.

    \begin{teX}
\newskip\partopsep
\newskip\itemsep
\newskip\parsep
\newskip\@topsep
\newskip\@topsepadd
\newskip\@outerparskip
    \end{teX}
 
 
 \begin{docCommand}{leftmargin} {\meta{dim}}
\end{docCommand}
\begin{docCommand}{rightmargin} {\meta{dim}}
\end{docCommand}
\begin{docCommand}{listparindent} {\meta{dim}}
\end{docCommand}
\begin{docCommand}{labelwidth} {\meta{dim}}
\end{docCommand}
\begin{docCommand}{@totlaleftmargin} {\meta{dim}}
\end{docCommand}
\begin{docCommand}{labelsep} {\meta{dim}}
\end{docCommand}
\begin{docCommand}{linewidth} {\meta{dim}}
\end{docCommand}
\begin{docCommand}{itemindent} {\meta{dim}}
\end{docCommand}
 \begin{docCommand}{@totalleftmargin} {}
 \end{docCommand}


 
    \begin{teX}
\newdimen\leftmargin
\newdimen\rightmargin
\newdimen\listparindent
\newdimen\itemindent
\newdimen\labelwidth
\newdimen\labelsep
\newdimen\linewidth
\newdimen\@totalleftmargin \@totalleftmargin=\z@
    \end{teX}
 
\begin{docCommand} {leftmargini} { \meta{dim}}
\end{docCommand}
\begin{teX}
\newdimen\leftmargini
\end{teX}
\begin{docCommand} {leftmarginii} { \meta{dim}}
\end{docCommand}
\begin{teX}
\newdimen\leftmarginii
\end{teX}

\begin{docCommand} {leftmarginiii} { \meta{dim}}
\end{docCommand}
\begin{teX}
\newdimen\leftmarginiii
\end{teX}

\begin{docCommand} {leftmarginiv} { \meta{dim}}
\end{docCommand}
\begin{teX}
\newdimen\leftmarginiv
\end{teX}
    \begin{teX}
\newdimen\leftmarginv
    \end{teX}


\begin{docCommand}{@listdepth}{}
\end{docCommand}
\begin{docCommand}{@itempenalty}{}
\end{docCommand}
\begin{docCommand}{@beginparpenalty}{}
\end{docCommand}
\begin{docCommand}{@endparpenalty}{}
\end{docCommand}
    \begin{teX}
\newcount\@listdepth \@listdepth=0
\newcount\@itempenalty
\newcount\@beginparpenalty
\newcount\@endparpenalty
    \end{teX}


 \begin{docCommand}{@labels} { \meta{box} }
 \end{docCommand}
    \begin{teX}
\newbox\@labels
    \end{teX}
 


 \begin{docCommand}{if@inlabel}{}
  \end{docCommand}
    \begin{teX}
\newif\if@inlabel \@inlabelfalse
    \end{teX}



\begin{docCommand}{ if@newlist} { }
\end{docCommand}
  \begin{teX}
\newif\if@newlist   \@newlistfalse
    \end{teX}
  
  \begin{docCommand}{if@newlistfalse }{}
  \end{docCommand}
  \begin{docCommand}{if@newlisttrue }{}
  \end{docCommand}
 
  
 
\begin{docCommand}{if@noparitem}{}
 \end{docCommand}
 \begin{docCommand}{ if@noparitemtrue}{}
 \end{docCommand}
 \begin{docCommand}{ if@noparitemfalse}{}
 \end{docCommand}
 
\begin{teX}
\newif\if@noparitem \@noparitemfalse
\end{teX}
 

\begin{docCommand} {if@noparlist} { }
  \end{docCommand}
  
\begin{docCommand} {if@noparlistfalse}{}
\end{docCommand}

\begin{docCommand} {if@noparlisttrue}{}
\end{docCommand}  
  
\begin{teX}
\newif\if@noparlist \@noparlistfalse
\end{teX}


\begin{docCommand} {if@noitemarg} { }
\end{docCommand}

\begin{docCommand}{if@noitemargtrue}{}
\end{docCommand}  

\begin{docCommand}{if@noitemargfalse}{}
\end{docCommand}  

    \begin{teX}
\newif\if@noitemarg \@noitemargfalse
    \end{teX}


\begin{docCommand}{if@newlist}{}
\end{docCommand}

\begin{docCommand}{if@newlisttrue}{}
\end{docCommand}  
\begin{docCommand}{if@newlistfalse}{}
\end{docCommand}  
\begin{docCommand}{if@nmbrlist}{}
\end{docCommand}
  
\begin{teX}
\newif\if@nmbrlist  \@nmbrlistfalse
\end{teX}




\begin{docCommand} {@itemlabel} { }
  Temporary definition to store the label.
\end{docCommand}

\index{kernel>lists>\textbackslash list}
\begin{docCommand} {list} {\marg{item label } \marg {commands} }
 List takes two arguments and is an author command for building
other lists.
\end{docCommand}

    \begin{teX}
\def\list#1#2{%
  \ifnum \@listdepth >5\relax
    \@toodeep
  \else
    \global\advance\@listdepth\@ne
  \fi
   \end{teX}
Initialize all margins to 0pt in case the user did not specify anything.   
   \begin{teX}  
  \rightmargin\z@
  \listparindent\z@
  \itemindent\z@
  \csname @list\romannumeral\the\@listdepth\endcsname
  \def\@itemlabel{#1}%
  \let\makelabel\@mklab
  \@nmbrlistfalse
  #2\relax
  \@trivlist
  \parskip\parsep
  \parindent\listparindent
  \advance\linewidth -\rightmargin
  \advance\linewidth -\leftmargin
  \advance\@totalleftmargin \leftmargin
  \end{teX}
Now set parshape for the rest of the paragraphs.   
  \begin{teX}  
  \parshape \@ne \@totalleftmargin \linewidth
  \ignorespaces}
    \end{teX}


\begin{docCommand}{par@deathcycles}{}
\end{docCommand}
    \begin{teX}
\newcount\par@deathcycles
    \end{teX}
 

 \begin{docCommand}{@trivlist}{}
 \end{docCommand}
  Because |\par| is sometimes made a no-op it is possible for a missing
 |\item| to produce a loop that does not fill memory and so never gets
 trapped by \TeX.  We thus need to trap this here by seting |\par| to
 count the number of times a paragraph ii is called with no progress
 being made started.
    \begin{teX}
\def\@trivlist{%
  \if@noskipsec \leavevmode \fi
  \@topsepadd \topsep
  \ifvmode
    \advance\@topsepadd \partopsep
  \else
    \unskip \par
  \fi
  \if@inlabel
    \@noparitemtrue
    \@noparlisttrue
  \else
    \if@newlist \@noitemerr \fi
    \@noparlistfalse
    \@topsep \@topsepadd
  \fi
  \advance\@topsep \parskip
  \leftskip \z@skip
  \rightskip \@rightskip
  \parfillskip \@flushglue
  \par@deathcycles \z@
  \@setpar{\if@newlist
             \advance\par@deathcycles \@ne
             \ifnum \par@deathcycles >\@m
               \@noitemerr
               {\@@par}%
             \fi
           \else
             {\@@par}%
           \fi}%
  \global \@newlisttrue
  \@outerparskip \parskip}
    \end{teX}

The function of a |trivlist| is very simple, it sets numbering to false and initializes all  widths to zero, thus at this stage it will default to a normal paragraph then it calls the auxiliary function \refCom{@trivlist}, which will set
the top and bottom skips and other paragraphing parameters, if set..

 \begin{docCommand}{trivlist} { \meta{void} }
 \end{docCommand} 
     \begin{teX}
\def\trivlist{%
  \parsep\parskip
  \@nmbrlistfalse
  \@trivlist
  \labelwidth\z@
  \leftmargin\z@
  \itemindent\z@
    \end{teX}

    We initialise |\@itemlabel| so that a \texttt{trivlist} with
    an |\item| not having an optional argument doesn't produce an
    error message.

 \begin{docCommand} {makelabel} {\marg{item label}}
 \end{docCommand}
    \begin{teX}
  \let\@itemlabel\@empty
  \def\makelabel##1{##1}}
    \end{teX}


 \begin{docCommand}{endlist} {  \meta{void}}
  ends the list environment
 \end{docCommand}
     \begin{teX}
\def\endlist{%
  \global\advance\@listdepth\m@ne
  \endtrivlist}
    \end{teX}
 

    The definition of \refCom{trivlist} used to be in ltspace.dtx 
    so that other commands could be `let to it'.  
    They now use \refCom{def}.
 \begin{docCommand}{endtrivlist}{}
 \end{docCommand}
 
% \changes{v1.2b ltspace}{1994/11/12}{Changed order of tests to make
% \refCom{@noitemerror} correct: end of an era.}
% \changes{v1.0i}{1995/05/25}{Macros moved from ltspace.dtx}
% \changes{v1.0n}{1996/10/25}{Change \refCom{indent} to \refCom{leavevmode}}
% \changes{v1.0n}{1996/10/25}{Reset flags explicitly}
% \changes{v1.0o}{1996/10/26}{Correct typo}
    \begin{teX}
\def\endtrivlist{%
  \if@inlabel
    \leavevmode
    \global \@inlabelfalse
  \fi
  \if@newlist
    \@noitemerr
    \global \@newlistfalse
  \fi
  \ifhmode\unskip \par
    \end{teX}
    We also check if we are in math mode and issue an error message
    if so (hoping that |\@currenvir| resolves suitably). Otherwise
    the usual ``perhaps a missing item'' error will get triggered
    later which is confusing.
 \changes{v1.0s}{2002/10/28}{Check for math mode (pr/3437)}
    \begin{teX}
  \else
    \@inmatherr{\end{\@currenvir}}%
  \fi
  \if@noparlist \else
    \ifdim\lastskip >\z@
      \@tempskipa\lastskip \vskip -\lastskip
      \advance\@tempskipa\parskip \advance\@tempskipa -\@outerparskip
      \vskip\@tempskipa
    \fi
    \@endparenv
  \fi
}
    \end{teX}
 

  \begin{docCommand}{@endparenv}{}
 \end{docCommand}
 
 

 
\begin{docCommand}{@doendpe}{}
   To suppress the paragraph indentation in text immediately following
   a paragraph-making environment, \refCom{everypar} is changed to remove the
   space, and \refCom{par} is redefined to restore \refCom{everypar}.  Instead of
   redefining \refCom{par} and \refCom{everypar}, \refCom{@endparenv} was changed to 
   set the \refCom{@endpe switch}, letting |end| redefine |par| and 
   |everypar|.   This allows paragraph-making environments to work right when called 
 by other environments. )
\end{docCommand}
 
    \begin{teX}
\def\@endparenv{%
  \addpenalty\@endparpenalty\addvspace\@topsepadd\@endpetrue}
    \end{teX}

    \begin{teX}
\def\@doendpe{\@endpetrue
     \def\par{\@restorepar\everypar{}
          \par\@endpefalse}\everypar
    \end{teX}
    
    Use |\setbox0=\lastbox| instead of   |\hskip -\parindent|   
    so that a |\noindent| becomes a no-op when used before 
    a line immediately following a list environment(23 Oct 86).
 \changes{v1.0k}{1995/11/07}{Enclosed \refCom{setbox0} assignment by a
 group so that it leaves the contents of box $0$ intact.
    } 
    \begin{teX}
               {{\setbox\z@\lastbox}\everypar{}\@endpefalse}}
    \end{teX}
 

  \begin{docCommand}{if@endpe}{}
 \end{docCommand}
  \begin{docCommand}{@endpelttrue}{}
 \end{docCommand}
% \begin{macro}{\if@endpe}
% \begin{macro}{\@endpefalse}
% \begin{macro}{\@endpeltrue}
    \begin{teX}
\newif\if@endpe
\@endpefalse
    \end{teX}
% \end{macro}\end{macro}\end{macro}

 \begin{docCommand}{@mklab}{}
 \end{docCommand}
    \begin{teX}
\def\@mklab#1{\hfil #1}
    \end{teX}

 \changes{LaTeX2.09}{1992/09/18}
     {(RmS) Added warning if \refCom{item} is used in math mode}
 \changes{v1.0c}{1994/04/28}
     {Replaced \refCom{@ltxnomath} by \refCom{@inmatherr}}
 \changes{v1.0d}{1994/05/03}
     {Removed superfluous braces}
     
 \begin{docCommand}{item}{}
 \end{docCommand}     
     \begin{teX}
\def\item{%
  \@inmatherr\item
  \@ifnextchar [\@item{\@noitemargtrue \@item[\@itemlabel]}}
    \end{teX}
 
 \begin{docCommand} {@donoparitem} { }
 \end{docCommand}
    \begin{teX}
\def\@donoparitem{%
  \@noparitemfalse
  \global\setbox\@labels\hbox{\hskip -\leftmargin
                               \unhbox\@labels
                                \hskip \leftmargin}%
  \if@minipage
    \else
      \@tempskipa\lastskip
      \vskip -\lastskip
      \advance\@tempskipa\@outerparskip
      \advance\@tempskipa -\parskip
      \vskip\@tempskipa
  \fi}
    \end{teX}

 
 \begin{docCommand}{@item}{}
 \end{docCommand}
     \begin{teX}
\def\@item[#1]{%
  \if@noparitem
    \@donoparitem
  \else
    \if@inlabel
      \indent \par
    \fi
    \ifhmode
      \unskip\unskip \par
    \fi
    \if@newlist
      \if@nobreak
        \@nbitem
      \else
        \addpenalty\@beginparpenalty
        \addvspace\@topsep
        \addvspace{-\parskip}%
      \fi
    \else
      \addpenalty\@itempenalty
      \addvspace\itemsep
    \fi
    \global\@inlabeltrue
  \fi
  \everypar{%
    \@minipagefalse
    \global\@newlistfalse
    \end{teX}
    This |\if@inlabel| check is needed in case an item starts of
    inside a group so that |\everypar| does not become empty
    outside that group. 
 \@nobreakfalse, etc etc.
    \begin{teX}
    \if@inlabel
      \global\@inlabelfalse
    \end{teX}
    The paragraph indent is now removed by using |\setbox...| since
    this makes |\noindent| a no-op here, as it should be. Thus the
    following comment is redundant but is left here for the sake of
    future historians:
    this next command was changed from an hskip to a kern to avoid
    a break point after the parindent box: the skip could cause a
    line-break if a very long label occurs in raggedright setting.
 \changes{v1.0d}{1994/05/03}{\refCom{hskip} changed to \refCom{kern}}
 \changes{v1.0m}{1996/10/23}{\refCom{kern...} changed to \refCom{setbox...}}
 \changes{v1.0r}{1997/02/21}
    {\refCom{ifvoid} check added for \refCom{noindent}. latex/2414}
 If |\noindent| was used after |\item| want to cancel the |\itemindent|
 skip. This case can be detected as the indentation box will be void.
    \begin{teX}
      {\setbox\z@\lastbox
       \ifvoid\z@
         \kern-\itemindent
       \fi}%
    \end{teX}

    \begin{teX}
      \box\@labels
      \penalty\z@
    \fi
    \end{teX}
    This code is intended to prevent a page break after the first
    line of an item that comes immediately after a section title. It
    may be sensible to always forbid a page break after one line of
    an item?  As with all such settings of |\clubpenalty| it is local
    so will have no effect if the item starts in a group.

    Only resetting |\@nobreak| when it is true is now
    essential since now it is sometimes set locally.
 \changes{v1.0m}{1996/10/23}{Added setting of \refCom{clubpenalty} and
    set \refCom{@nobreakfalse} only when necessary}
    \begin{teX}
    \if@nobreak
      \@nobreakfalse
      \clubpenalty \@M
    \else
      \clubpenalty \@clubpenalty
      \everypar{}%
    \fi}%
    \end{teX}
 \changes{v1.0l}{1996/07/26}{Remove unecessary \refCom{global} before
                 \refCom{@nobreak...}}
 \changes{v1.0m}{1996/10/23}{\refCom{@nobreak...} moved into the
          \refCom{everypar} and not executed unconditionally, see above} 
    \begin{teX}
  \if@noitemarg
    \@noitemargfalse
    \if@nmbrlist
    \end{teX}
 \changes{v1.0g}{1995/05/17}{Removed surplus braces}
    \begin{teX}
      \refstepcounter\@listctr
    \fi
  \fi
    \end{teX}
    
We use |\sbox| to support colour commands.
\begin{teX}
  \sbox\@tempboxa{\makelabel{#1}}%
  \global\setbox\@labels\hbox{%
    \unhbox\@labels
    \hskip \itemindent
    \hskip -\labelwidth 
    \hskip -\labelsep
    \ifdim \wd\@tempboxa >\labelwidth
      \box\@tempboxa
\end{teX}
 \changes{LaTeX2.09}{1991/11/22}
         {(RmS) Changed second call to \refCom{makelabel} to
           \refCom{unhbox}\refCom{@tempboxa}.
          Avoids problems with side effects in \refCom{makelabel} and is
               more efficient.}
    \begin{teX}
    \else
      \hbox to\labelwidth {\unhbox\@tempboxa}%
    \fi
    \hskip \labelsep}%
  \ignorespaces}
    \end{teX}
 
 \begin{docCommand}{makelabel}{}
 Initial definition of \cs{makelabel} set to produce the infamous and often encountered error, if it
 has not been defined by one of the paragraphing environments.
 \end{docCommand}
 
    \begin{teX}
\def\makelabel#1{%
  \@latex@error{Lonely \string\item--perhaps a missing
        list environment}\@ehc}
    \end{teX}
 

 \begin{docCommand}{nbitem}{}
 \end{docCommand}
% \begin{macro}{\@nbitem}
 \changes{v1.0g}{1995/05/17}{Removed surplus braces}
    \begin{teX}
\def\@nbitem{%
  \@tempskipa\@outerparskip
  \advance\@tempskipa -\parskip
  \addvspace\@tempskipa}
    \end{teX}

 \begin{docCommand}{usecounter}{}
 \end{docCommand}
    \begin{teX}
\def\usecounter#1{\@nmbrlisttrue\def\@listctr{#1}\setcounter{#1}\z@}
    \end{teX}


 \subsection{Itemize and Enumerate}

  Enumeration is done with four counters: |enumi|, |enumii|, |enumiii|
  and |enumiv|, where |enum|N controls the numbering of the Nth level
  enumeration.  The label is generated by the commands
  \refCom{labelenumi} \ldots{} \cs{labelenumiv}, which should be defined
  by the document style.
  
  Note that \cs{p@enum}N \cs{theenum}N defines the output
  of a \refCom{ref} command.  A typical definition might be:
  
 \begin{verbatim}
     \def\theenumii{\alph{enumii}}
     \def\p@enumii{\theenumi}
     \def\labelenumii{(\theenumii)}
 \end{verbatim}
 which will print the labels as `(a)', `(b)', \ldots
 and print a \refCom{ref} as `3a'.

 The item numbers are moved to the right of the label box, so they are
 always a distance of \refCom{labelsep} from the item.

 \refCom{@enumdepth} holds the current enumeration nesting depth.

 Itemization is controlled by four commands: \docAuxCommand{labelitemi},
 \docAuxCommand{labelitemii},
 \docAuxCommand{labelitemiii}, and \docAuxCommand{labelitemiv}.
 To cause the second-level list to be
 bulleted, you just define \docAuxCommand{labelitemii}
 to be |$\bullet$|.  |@itemspacing|
 and \refCom{@itemdepth} are the analogs of |@enumspacing| and
 \refCom{@enumdepth}.


 \begin{docCommand}{@enumdepth}{}
 \end{docCommand}
     \begin{teX}
\newcount\@enumdepth \@enumdepth = 0
    \end{teX}
 

Next the counters for the \refEnv{enumerate} environment, \docCounter{c@enumi}, \docCounter{c@enumii}, \docCounter{c@enumiii} and \docCounter{c@enumiv} are created.

    \begin{teX}
\@definecounter{enumi}
\@definecounter{enumii}
\@definecounter{enumiii}
\@definecounter{enumiv}
    \end{teX}


 \begin{docEnvironment}{enumerate}{}{}
 \end{docEnvironment}  
   The enumerate environment enumerates the list. The macro
  is written very efficiently and defines the basic structure. The
  typesetting parameters are left for the class files such as book
  to define them.

     \begin{teX}
\def\enumerate{%
  \ifnum \@enumdepth >\thr@@\@toodeep\else
    \advance\@enumdepth\@ne
    \edef\@enumctr{enum\romannumeral\the\@enumdepth}%
      \expandafter
      \list
        \refComname label\@enumctr\endcsname
        {\usecounter\@enumctr\def\makelabel##1{\hss\llap{##1}}}%
  \fi}
    \end{teX}

    \begin{teX}
\let\endenumerate =\endlist
    \end{teX}

 \begin{docCommand}{@itemdepth}{}
 \end{docCommand}
    \begin{teX}
\newcount\@itemdepth \@itemdepth = 0
    \end{teX}

Finally the \refEnv{itemize} environment is defined. This is defined in the kernel to have a limit of 4 levels only (which as a matter
of fact is more than adequate. \footnote{changes v1.0j 1995/07/09 Use \textbackslash expandafter. 
Hard to believe the team missed it!}

 \begin{docEnvironment}{itemize}{}
 \end{docEnvironment}
     \begin{teX}
\def\itemize{%
  \ifnum \@itemdepth >\thr@@\@toodeep\else
    \advance\@itemdepth\@ne
    \edef\@itemitem{labelitem\romannumeral\the\@itemdepth}%
    \end{teX}
    
Call the \refCom{list} sets all parameters and defines a  \refCom{makelabel} for this particular
environment. This is a good place to also hook other items to decorate the label, if necessary.
Remember also that the \docAuxCommand{@itemitem}, defined above is just there to facilitate  
calling |labelitemi| \ldots |labelitemn| depending on the level.

    \begin{teX}
    \expandafter
    \list
      \csname\@itemitem\endcsname
      {\def\makelabel##1{\hss\llap{##1}}}%
  \fi}
    \end{teX}

    \begin{teX}
\let\enditemize =\endlist
    \end{teX}




