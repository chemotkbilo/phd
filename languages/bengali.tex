\section{Bengali}
\label{sec:bengali}
\idxlanguage{Bengali}
\index{Bengali fonts>Shonar Bangla}
\index{Bengali fonts>Vrinda}
\index{Bengali fonts>code2000}
\newfontfamily\bengali[Script=Bengali,Scale=1.3]{Shonar Bangla}

There are two Windows fonts that can be used with Windows \textit{Shonar Bangla} and \textit{Vrinda}. For open source fonts one can use, \texttt{code2000}.

\docAuxCommand{bengali} Once the key is set the command \cmd{\bengali} is available for use in typesetting Bengali text.

\bigskip

\bgroup



\bengali
\centering

  অ  আ ই  ঈ  উ  ঊ  ঋ  এ  ঐ\par

\fontspec[Script=Bengali,Scale=3.2]{Vrinda}

\centering

  অ  আ ই  ঈ  উ  ঊ  ঋ  এ  ঐ\par


\fontspec[Script=Bengali,Scale=3.2]{code2000.ttf}

\centering

  অ  আ ই  ঈ  উ  ঊ  ঋ  এ  ঐ\par

\captionof{table}{The consonant{\protect\bengal{} ক (kô)} along with the diacritic form of the vowels {\protect\bengal{} অ, আ, ই, ঈ, উ, ঊ, ঋ, এ, ঐ, ও and ঔ} \textit{from Wikipedia}.}
\egroup


Bengali is a Unicode block containing characters for the Bangla, Assamese, Bishnupriya Manipuri, Daphla, Garo, Hallam, Khasi, Mizo, Munda, Naga, Rian, and Santali languages. In its original incarnation, the code points U+0981..U+09CD were a direct copy of the Bengali characters A1-ED from the 1988 ISCII standard, as well as several Assamese ISCII characters in the U+09F0 column. The Devanagari, Gurmukhi, Gujarati, Oriya, Tamil, Telugu, Kannada, and Malayalam blocks were similarly all based on ISCII encodings.

\begin{scriptexample}[]{Bengal}
\unicodetable{bengal}{"0980,"0990,"09A0,"09B0,"09C0,"09D0,"09E0,"09F0}
\end{scriptexample}


\printunicodeblock{./languages/bengali.txt}{\bengal}



\bgroup
\bengali\LARGE
\char"0995 + \color{blue} \char"09BC + \color{red}\char"09AF  = \char"0995\char"09CD \char"09AF
\egroup



See also \url{http://www.nongnu.org/freebangfont/downloads.html} for additional fonts.









