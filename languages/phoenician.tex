\section{Phoenician}
\label{s:phoenician}
\arial

The Phoenician alphabet and its successors were widely used over a broad area surrounding the Mediterranean Sea.

\let\phoenician\lycian

\begin{scriptexample}[]{Phoenician}

\unicodetable{phoenician}{"10900,"10910}

\end{scriptexample}

The Phoenician alphabet, called by convention the Proto-Canaanite alphabet for inscriptions older than around 1200 BCE, is the oldest verified consonantal alphabet, or abjad.[1] It was used for the writing of Phoenician, a Northern Semitic language, used by the civilization of Phoenicia. It is classified as an abjad because it records only consonantal sounds (matres lectionis were used for some vowels in certain late varieties).

Phoenician became one of the most widely used writing systems, spread by Phoenician merchants across the Mediterranean world, where it evolved and was assimilated by many other cultures. The Aramaic alphabet, a modified form of Phoenician, was the ancestor of modern Arabic script, while Hebrew script is a stylistic variant of the Aramaic script. The Greek alphabet (and by extension its descendants such as the Latin, the Cyrillic, and the Coptic) was a direct successor of Phoenician, though certain letter values were changed to represent vowels.

\begin{figure}[ht]
\includegraphics[width=\textwidth]{./images/phoenician.jpg}
\captionof{figure}{
Phoenician votive inscription from Idalion (Cyprus), 390 BC. BM 125315 from The Early Alphabet by John F. Healy.}
\end{figure}

As the letters were originally incised with a stylus, most of the shapes are angular and straight, although more cursive versions are increasingly attested in later times, culminating in the Neo-Punic alphabet of Roman-era North Africa. Phoenician was usually written from right to left, although there are some texts written in boustrophedon.


\PrintUnicodeBlock{./languages/phoenician.txt}{\phoenician}


\input{./languages/palmyrene}


