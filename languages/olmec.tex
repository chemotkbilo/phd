\section{Epi-Olmec}
\label{s:olmec}
Epi-Olmec is an ancient Mesoamerican logosyllabic script which has been deciphered by Terrence Kaufman and John Justeson. A complete description of the script has been described by \cite{kaufman}. The most famous inscription is on the Tuxtla Statuette. The Tuxtla Statuette is a small 6.3 inch (16 cm) rounded greenstone figurine, carved to resemble a squat, bullet-shaped human with a duck-like bill and wings. Most researchers believe the statuette represents a shaman wearing a bird mask and bird cloak.[1] It is incised with 75 glyphs of the Epi-Olmec or Isthmian script, one of the few extant examples of this very early Mesoamerican writing system. The Tuxtla Statuette is particularly notable in that its glyphs include the Mesoamerican Long Count calendar date of March 162 CE, which in 1902 was the oldest Long Count date discovered. A product of the final century of the Epi-Olmec culture, the statuette is from the same region and period as La Mojarra Stela 1 and may refer to the same events or persons.[3] Similarities between the Tuxtla Statuette and Cerro de las Mesas Monument 5, a boulder carved to represent a semi-nude figure with a duckbill-like buccal mask, have also been noted.[4]

\begin{figure}[ht]
\centering
\includegraphics[height=0.35\textheight]{./images/tuxtla-statuette.png}\hspace{1em} 
\includegraphics[height=0.35\textheight]{./images/tuxtla-statuette-01.jpg}
\caption{Frontal view of the Tuxtla Statuette. Note the Mesoamerican Long Count calendar date of March 162 CE (8.6.2.4.17) down the front of the statuette. The left figure is from wikipedia and the right from the original \protect\href{http://www.readcube.com/articles/10.1525/aa.1907.9.4.02a00030}{Holmes} paper.\citep{holmes1907}}
\end{figure}

\subsection{The epiolmec package}

The script has not been as yet encoded as by the Unicode consortium. Syropoulos \citep{syropoulos} created a font for the script and also wrote an article for TUGboat. Interestingly the paper describes the procedure used to develop the font. The package \pkgname{epiolmec} which is available both in \TeX live and Mik\tex, provides commands to access the glyphs. It is also possibly easier to typeset the script using traditional \latexe techniques, as they provide transcription commands rather than using a unicode font with the glyphs allocated in the private area directly.

\begin{verbatim}
\documentclass{article}
\usepackage{epiolmec,multicol}
\begin{document}
  \begin{center}
      \begin{minipage}{80pt}
      \begin{multicols}{3}
         \EOku\\ \EOji\\  
         \EOtze\\ \EOstep \\
       \end{multicols}    
     \end{minipage}       
  \end{center}
\end{document}
\end{verbatim}

Since the Epi-Olmec script is a logosyllagraphy we
need some practical way to access the symbols of the
script. Originally Syropoulos used the Ω translation
process that mapped words and “syllables” to the
corresponding glyphs of the font. In this way one obtains
a natural way for typing in Epi-Olmec texts. In addition,
in order to avoid the problem mentioned above,
he used a wrapper that typesets the text vertically.
For short texts \cmd{\shortstacks} is adequate, while
for longer texts, he used a |multicols| environment
inside a relatively narrow minipage. 

\begin{scriptexample}{Epi-Olmec}
\bgroup
\HUGE
\centering
\EOpi   \EOofficerI \EOofficerII \EOofficerIII

\captionof{figure}{The output of \string\EOpi, \string\EOofficerI, \string\EOofficerII\ and \string\EOofficerIII\ commands. }
\egroup
\end{scriptexample}

\subsection{Numbering System}\index{Epi-Olmec>vigesimal system}

The Epi-Olmec people used the same numbering system  
 as the Maya. Their numbering system was a vigesimal system and
 the digits were written in a top-down fashion. Thus, we need a macro
 that will typeset numbers in this fashion when it is used with \LaTeX\
 (actually $\epsilon$-\LaTeX). In addition, we need a macro that will
 just output the vigesimal digits. Such a macro could be used with
 $\Lambda$ with the |LTL| text and paragraph directions. To recapitulate,
 we need to define two macros that will basically typeset vigesimal numbers
 in either horizontal or vertical mode.

 For the various calculations that are performed, we need at least three
 counter variables. The fourth is needed for the macro that typesets the
 vigesimal numbers vertically and its usage is explained below. 

\begin{scriptexample}{EpicOtmec}
\def\textb#1{\text{\makebox[6em]{\hss#1~~   \protect\string#1\hfill}}}
\begin{multicols}{3}
\bgroup
\parindent0pt
$\textb{\EOzero}=0$\\
$\textb{\EOi} = 1$\\
$\textb{\EOii} = 2$\\
$\textb{\EOiii} =3$\\
$\textb{\EOiv}  =4$\\
$\textb{\EOv}   =5$\\
$\textb{\EOvi}  =6$\\
$\textb{\EOvii} =7$\\
$\textb{\EOviii} =8$\\
$\textb{\EOix} =9$\\ 
$\textb{\EOx} =10$\\
$\textb{\EOxi} =11$\\
$\textb\EOxii =12$\\
$\textb{\EOxiii} =13$\\
$\textb{\EOxiv} =14$\\
$\textb{\EOxv} =15$\\
$\textb{\EOxvi} =16$\\
$\textb\EOxvii =17$\\
$\textb{\EOxviii} =18$\\
$\textb{\EOxix} =19$\\
$\textb{\EOxx} =20$\\
\egroup
\end{multicols}
\end{scriptexample}


%% TODO add to index all symbols

\begin{multicols}{4}
\bgroup
\def\K#1{\makebox[3em]{{\color{theunicodesymbolcolor}\hss#1\hfill}} \string#1}
\parindent0pt
\K\EOSpan\\ 
\K\EOJI \\
\K\EOvarji\\ 
\K\EOvarki \\
\K\EOpi \\
\K\EOpe \\
\K\EOpuu \\
\K\EOpa \\
\K\EOvarpa\\ 
\K\EOpu \\
\K\EOpo \\
\K\EOti \\
\K\EOte \\
\K\EOtuu \\
\K\EOta \\
\K\EOtu \\
\K\EOto \\
\K\EOtzi \\
\K\EOtze \\
\K\EOtzuu \\
\K\EOtza \\
\K\EOvartza\\ 
\K\EOtzu \\
\K\EOki \\
\K\EOke \\
\K\EOkuu \\
\K\EOvarkuu\\ 
\K\EOku\\ 
\K\EOko \\
\K\EOSi \\
\K\EOvarSi\\ 
\K\EOSuu \\
\K\EOSa \\
\K\EOSu \\
\K\EOSo \\
\K\EOsi \\
\K\EOvarsi\\ 
\K\EOsuu \\
\K\EOsa \\
\K\EOsu \\
\K\EOji \\
\K\EOje \\
\K\EOja \\
\K\EOvarja\\ 
\K\EOju \\
\K\EOjo \\
\K\EOmi \\
\K\EOme \\
\K\EOmuu \\
\K\EOma \\
\K\EOni \\
\K\EOvarni\\
\K\EOne \\
\K\EOnuu \\
\K\EOna \\
\K\EOnu \\
\K\EOwi \\
\K\EOwe \\
\K\EOwuu \\
\K\EOvarwuu\\
\K \EOwa\\
\K\EOwo \\
\K\EOye \\
\K\EOyuu \\
\K\EOya \\
\K\EOkak \\
\K\EOpak \\
\K\EOpuuk\\
\K\EOyaj \\
\K\EOScorpius\\
\K\EODealWith\\
\K\EOYear \\
\K\EOBeardMask \\
\K\EOBlood \\
\K\EOBundle \\
\K\EOChop \\
\K\EOCloth \\
\K\EOSaw \\
\K\EOGuise \\
\K\EOofficerI\\
\K\EOofficerII \\
\K\EOofficerIII \\
\K\EOofficerIV \\
\K\EOKing \\
\K\EOloinCloth \\
\K\EOlongLipII \\
\K\EOLose \\
\K\EOmexNew \\
\K\EOMiddle \\
\K\EOPlant \\
\K\EOPlay \\
\K\EOPrince \\
\K\EOSky \\
\K\EOskyPillar \\
\K\EOSprinkle \\
\K\EOstarWarrior\\
\K\EOTitleII \\
\K\EOtuki \\
\K\EOtzetze\\
\K\EOChronI \\
\K\EOPatron \\
\K\EOandThen\\
\K\EOAppear \\
\K\EODeer \\
\K\EOeat \\
\K\EOPatronII \\
\K\EOPierce \\
\K\EOkij \\
\K\EOstar  \\
\K\EOsnake \\
\K\EOtime \\
\K\EOtukpa  \\
\K\EOflint \\
\K\EOafter \\
\K\EOvarBeardMask \\
\K\EOBedeck \\
\K\EObrace \\
\K\EOflower  \\
\K\EOGod \\
\K\EOMountain \\
\K\EOgovernor \\
\K\EOHallow \\
\K\EOjaguar \\
\K\EOSini \\
\K\EOknottedCloth \\
\K\EOknottedClothStraps \\
\K\EOLord \\
\K\EOmacaw \\
\K\EOmonster \\
\K\EOmacawI \\
\K\EOskyAnimal\\
\K\EOnow \\
\K\EOTitleIV \\
\K\EOpenis \\
\K\EOpriest  \\
\K\EOstep\\
\K\EOsing \\
\K\EOskin \\
\K\EOStarWarrior \\
\K\EOsun \\
\K\EOthrone\\
\K\EOTime \\
\K\EOHallow \\
\K\EOTitle \\
\K\EOturtle \\
\K\EOundef \\
\K\EOGoUp \\
\K\EOLetBlood \\
\K\EORain \\
\K\EOset \\
\K\EOvarYear\\
\K\EOFold \\
\K\EOsacrifice \\
\K\EObuilding \\
\egroup
\end{multicols} 

\subsection{Technical}

The font is defined with the local encoding \texttt{LEO}. 

\begin{verbatim}
\DeclareFontEncoding{LEO}{}{}
\DeclareFontSubstitution{LEO}{cmr}{m}{n}
\DeclareFontFamily{LEO}{cmr}{\hyphenchar\font=-1}
\end{verbatim}

Note the |\hyphenchar\font=-1| that disables hyphenation in the |\DeclareFontFamily|  declaration. You cannot behead the \EOofficerII\ in order to hyphenate the text!


