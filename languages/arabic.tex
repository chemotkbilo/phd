\section{Arabic}
\label{s:arabic}

The Arabic script is a writing system used for writing several languages of Asia and Africa, such as Arabic, Sorani and Luri Dialects of Kurdish language, Persian, Pashto and Urdu.[1] Even until the 16th century, it was used to write some texts in Spanish.[2] After the Latin script, Chinese characters, and Devanagari, it is the fourth-most widely used writing system in the world.[3]
The Arabic script is written from right to left in a cursive style. In most cases the letters transcribe consonants, or consonants and a few vowels, so most Arabic alphabets are abjads.

The Arabic script has its roots in the Aramaic language and the Nabataen Arabs who wrote in the Aramaic script between the first century \BC{} and third centuries \AD{}. The Nabataens were a gathering of nomadic Arab tribes living
in a region stretching from the Sinai Peninsula to northern
Arabia and eastern Jordan. In the Hellenistic era following
Alexander the Great’s conquests, they formed a kingdom that
lasted from around 150 BC until conquest by the Romans in 105
AD; their capital was the peerless rock city of Petra. Their
Nabatæn form of Aramaic writing became the immediate
parent of Arabic writing.   

The script was first used to write texts in Arabic, most notably the Qurʼān, the holy book of Islam. With the spread of Islam, it came to be used to write languages of many language families, leading to the addition of new letters and other symbols, with some versions, such as Kurdish, Uyghur, and old Bosnian being abugidas or true alphabets. It is also the basis for a rich tradition of Arabic calligraphy. Like Hebrew, Arabic is an important religious script whose
significance, longevity and expansion are owed to its veneration as a vehicle of faith. Once it was chosen to convey the Koran in the seventh
century, its hegemony in the region, and beyond, was assured.
Today, the Arabic consonantal alphabet is read and written on
the Arabian Peninsula, throughout the Near East, in western,
Central and South-East Asia, in parts of Africa and in all areas of
Europe influenced by Islam (illus. 66). 

The Arabic script has
been adapted to more languages belonging to more families
than any other Semitic script, including Berber, Somali, Swahili
(illus. 67), Urdu, Turkish, Uighur, Kazakh, Farsi (Persian),
Kashmiri, Malay, even Spanish and Slavonic in Europe.37 When
borrowed, Arabic letters were never dropped, but new or
derived letters frequently were added to reproduce sounds not
included in the Arabic inventory. Arabic facilitates this process
by distinguishing between some letters only by varying the
number of dots written with each; this function can then easily
be extended by foreign tongues needing new letters compatible
with Arabic’s fundamental appearance.38 Arabic is one of the
world’s great scripts, and will doubtless survive for many more
centuries.


\begin{Arabic}


ّ هو إذ الغاية؛ شريف الفوائد، جم المذهب، عزيز فنّ التاريخ فنّ أنّ اعلم
والملوك سيرهم، في والأنبياء أخلاقهم، في الأمم من الماضين أحوال على يوقفنا
ّ أحوال في يرومه لمن ذلك في الإقتداء فائدة تتم حتّى وسياستهم؛ دولهم في
والدنيا. الدين


\end{Arabic}

Like all Semitic scripts, Arabic uses a consonantal alphabet
commonly indicating word roots, but with a richer inventory of
28 basic letters and additional augmentations, some created by
adding a dot under existing letters (illus. 68). (A ‘29th’ letter is
the ligature of la¯m and ’a¯lif.) Arabic also inherited the long vowel
use of some consonants and the special diacritics to signal
other vowels. However, vowels in Arabic are consistently indicated
only in the Koran and in poetry. All other texts use only
consonantal writing, with diacritics assisting occasionally in
ambiguous readings. The use of ’a¯lif for long /a:/ is an Arabic
innovation. Short /a/, /i/ and /u/ make use of derived forms of
simplified consonants: for /a/, a horizontal bar over the consonant;
for /i/, a similar bar under the consonant; and for /u/, a
small hook over the consonant. If a tiny circle is written above a
consonant, this means no vowel accompanies the consonant. All
but six Arabic letters occur in four different shapes, each determined
by the letter’s position in a word: independent (the neutral
or standard shape), initial, medial or final (illus. 69).39

The oldest Islamic inscription was found in 1999 and described by ‘{}Ali ibn Ibrahim Ghamman in Zuhayr in 
Saudi Arabia and is dated \AD{644-645}\footnote{ 
The inscription of Zuhayr, the oldest Islamicinscription (24 AH/AD 644–645), the rise of theArabic script and the nature of the early Islamic state.} Hoyland\cite{hoyland2010} gives a good review of the development of Arabic as
a written language during the late Roman period in Palestine and Arabia. 




As of Unicode 7.0, the Arabic script is contained in the following blocks:
Arabic (0600—06FF, 255 characters)
Arabic Supplement (0750—077F, 48 characters)
Arabic Extended-A (08A0—08FF, 39 characters)
Arabic Presentation Forms-A (FB50—FDFF, 608 characters)
Arabic Presentation Forms-B (FE70—FEFF, 140 characters)
Rumi Numeral Symbols (10E60—10E7F, 31 characters)
Arabic Mathematical Alphabetic Symbols (1EE00—1EEFF, 143 characters)[1][2]

The basic Arabic range encodes the standard letters and diacritics, but does not encode contextual forms (U+0621–U+0652 being directly based on ISO 8859-6); and also includes the most common diacritics and Arabic-Indic digits. The Arabic Supplement range encodes letter variants mostly used for writing African (non-Arabic) languages. The Arabic Extended-A range encodes additional Qur'anic annotations and letter variants used for various non-Arabic languages. The Arabic Presentation Forms-A range encodes contextual forms and ligatures of letter variants needed for Persian, Urdu, Sindhi and Central Asian languages. The Arabic Presentation Forms-B range encodes spacing forms of Arabic diacritics, and more contextual letter forms. The presentation forms are present only for compatibility with older standards, and are not currently needed for coding text.[3] 

The Arabic Mathematical Alphabetical Symbols block encodes characters used in Arabic mathematical expressions.


Position in word:	Isolated	Final	Medial	Initial
Glyph form:\scalebox{3}[3]{ب}{ـب}‎	ـبـ‎	 \scalebox{3}{بـ}


\printunicodeblock[2]{./languages/arabic.txt}{\arabicfont}



