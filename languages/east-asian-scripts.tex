\index{Katakana}\index{Hiragana}
\index{Bopomofo}\index{Hangul}\index{Yi}
\index{East Asian Scripts>Katakana}
\index{East Asian Scripts>Hiragana}
\index{East Asian Scripts>Hangul}
\index{East Asian Scripts>Bopomofo}
\index{East Asian Scripts>Yi}
\index{scripts>cjk}
\pagestyle{headings}
\index{Yi fonts>Microsoft Yi Baiti}
\chapter{East Asian Scripts}
\epigraph{

For writing is the foundation of the classics and the arts, the beginning of
royal government. It is the means by which people of the past reach posterity,
by which people of the future know the past. 

{\cjk 蓋文字者,經藝之本,王政之始。前人所以垂後,後人所以識古。}
}{ Xu Shen  in the ``Postface'' of the \emph{Shuowen}}

\bigskip

\noindent This chapter presents the most common scripts currently in use in East Asia. This includes Chinese, Japanese and Korean. It also discusses several scripts for minority languages spoken in southern China. The scripts discussed are as follows:


\begin{center}
\begin{tabular}{lll}
\nameref{s:han} &Hiragana &Hangul\\
\nameref{s:bopomofo} &Katakana &\nameref{s:yi}\\
\end{tabular}
\end{center}
\bigskip

\parindent1em

Settings for |cjk| languages and scripts follow:

\begin{docKey}[phd]{cjk font}{\meta{font name}}{default none, initial code2000.ttf}
This key when set produces all necessary command to set the font for cjk typesetting.
\end{docKey}

\parindent1em
\section{Han CJK Unified Ideographs}
\label{s:han}
\index{CJK}
The Chinese, Japanese and Korean (CJK) scripts share a common background. In the process called Han unification the common (shared) characters were identified, and named "CJK Unified Ideographs". Unicode defines a total of 74,617 CJK Unified Ideographs.[1]\footnote{\protect\url{http://shahon.org/wp-content/uploads/2010/02/Galambos-2006-Orthography-of-early-Chinese-writing.pdf}}

The terms ideographs or ideograms may be misleading, since the Chinese script is not strictly a picture writing system.
Historically, Vietnam used Chinese ideographs too, so sometimes the abbreviation "CJKV" is used. This system was replaced by the Latin-based Vietnamese alphabet in the 1920s.


\unicodetable{cjk}{"4E00,"4E10,"4E20,"4E30,"4E40,"4}


\input{./languages/bopomofo}


\section{Yi}
\label{s:yi}

The Yi script (Yi: {\yi ꆈꌠꁱꂷ} nuosu bburma [nɔ̄sū bū̠mā]; Chinese: {\cjk 彝文}; pinyin: Yí wén) is an umbrella term for two scripts used to write the Yi language; Classical Yi, an ideogram script, the later Yi Syllabary. The script is also historically known in Chinese as Cuan Wen (Chinese: {\cjk 爨文}; pinyin: Cuàn wén) or Wei Shu (simplified Chinese: {\cjk韪书}; traditional Chinese: {\cjk 違書}; pinyin: Wéi shū) and various other names ({\cjk夷字、倮語、倮倮文、毕摩文}), among them "tadpole writing" ({\cjk蝌蚪文}).[1]

This is to be distinguished from romanized Yi ({\yi 彝文罗马拼音} Yiwen Luoma pinyin) which was a system (or systems) invented by missionaries and intermittently used afterwards by some government institutions.[2][3] There was also a Yi abugida or alphasyllabary devised by Sam Pollard, the Pollard script for the Miao language, which he adapted into "Nasu" as well.[4][5] Present day traditional Yi writing can be sub-divided into five main varieties (Huáng Jiànmíng 1993); Nuosu (the prestige form of the Yi language centred on the Liangshan area), Nasu (including the Wusa), Nisu (Southern Yi), Sani (撒尼) and Azhe (阿哲).[6][7]

The Unicode block for Modern Yi is Yi syllables (U+A000 to U+A48C), and comprises 1,164 syllables (syllables with a diacritic mark are encoded separately, and are not decomposable into syllable plus combining diacritical mark) and one syllable iteration mark (U+A015, incorrectly named YI SYLLABLE WU). In addition, a set of 55 radicals for use in dictionary classification are encoded at U+A490 to U+A4C6 (Yi Radicals).[11] Yi syllables and Yi radicals were added as new blocks to Unicode Standard Version 3.0.[12]

Classical Yi - which is an ideographic script like the Chinese characters - has not yet been encoded in Unicode, but a proposal to encode 88,613 Classical Yi characters was made in 2007.[13]

\bgroup
\yi \char"A000: Yi Syllable It\\

\yi \char"A001: Yi Syllable Ix\\

\yi \char"A002: Yi Syllable I\\
\egroup

\begin{scriptexample}[]{Yi}
\unicodetable{yi}{"A000,"A010,"A020,"A030,"A040,"A050,"A060,"A070,"A080,"A090,"A0A0,"A0B0,"A0C0}
\end{scriptexample}


