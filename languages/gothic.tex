\def\codex#1{\emph{Codex #1}\index{codex>#1}}
%\newfontfamily{\gothicfamily}{Noto Sans Gothic}
\newfontfamily{\gothicfamily}{code2001.ttf}
\section{Gothic}

\label{s:gothic}

\subsection{Introduction}

East Germanic Goths rose to prominence during the Great
Migrations of the fourth and fifth centuries AD 31 Their Gothic
languages are primarily known to us today through a few surviving
fragments of Bible translations. It was the Visigothic bishop
Wulfila († AD 383), according to three ecclesiastical historians
writing a century later, who created ‘Gothic letters’ in order to
translate the Bible into the Visigothic language. The fourth century
Greek alphabet was Wulfila’s only apparent source.

Though the bishop’s original Visigothic hand has not survived,
closely related derivative scripts preserved in two later Gothic
manuscripts no older than the sixth century have been preserved
(illus. 116).

‘Wulfila’s script’, as it perhaps should properly be designated,
is an alphabetic script written from left to right without word
separation. Spaces indicate sentences or passages, as does a
colon or a centred dot (as with the Iberian scripts). Nasal suspension
– that is, marking where an /m/ or /n/ should be – is
sometimes indicated by a macron (a topping stroke) above the
preceding letter. Ligatures are even rarer than macrons. There
are frequent contractions: for example, ius is often used to spell
‘Jesus’. Apart from rare profane relics – witness the sixth-century
Latin-Gothic Deed of Naples – Wulfila’s script, measured
by those few inscriptions that have survived, appears to have
conveyed exclusively ecclesiastical texts.

\begin{figure}[htb]
\includegraphics[width=.45\textwidth]{gothic}
\caption{Codex Carolinus}
\end{figure}

The Gothic script that Wulfila devised from the Greek
alphabet did not engender daughter scripts. After the sixth century
AD, it was replaced almost everywhere by related descendants
of Greek and Latin alphabets. Gothic’s last sentinel, the
ninth-century \codex{Vindobonensis} 795, was perhaps by then
only an antiquarian curiosity. The \emph{Codex Carolinus} preserves papal correspondence
with Frankish rulers, including letters exchanged by popes from Gregory III (731-741) to Hadrian I (772-795). the Codex was written in 791 on the orders of Charlemagne in order to rescue papyrus copies threatened with decay. It contains 99 letters, almost exclusively papal, and survives today in Vienna, \"Osterreichische Nationalbibliotek 449, in a copy probably made at Colone during the pontificate of Archbishop Willibert (870-889). The preface of the \codex{Carolinus} appears to refer to a second part that may have contained letters to byzantine rulers, now lost. Parallel copies of the Codex have not turned up. \citep{jasper2001papal}. 

\subsection{Unicode}

The Gothic alphabet was added to the Unicode Standard in March, 2001 with the release of version 3.1.

The Unicode block for Gothic is U+10330–U+1034F in the Supplementary Multilingual Plane. As older software that uses UCS-2 (the predecessor of UTF-16) assumes that all Unicode codepoints can be expressed as 16 bit numbers (U+FFFF or lower, the Basic Multilingual Plane), problems may be encountered using the Gothic alphabet Unicode range and others outside of the Basic Multilingual Plane.

\begin{scriptexample}[]{Gothic}
\unicodetable{gothicfamily}{"10330,"10340}
\end{scriptexample}
{\gothicfamily
𐍀	𐍁	𐍂	𐍃	𐍄	𐍅	𐍆	𐍇	𐍈	𐍉	𐍊}
%http://www.gotica.de/carolinus.html

%\begin{thebibliography}
%\bibitem[Fitzmyer(1995)]{fitzmyer}
%J.~A. Fitzmyer.
%\newblock \emph{The Aramaic inscriptions of Sefīre}, volume~19 of
%  \emph{Biblica et orientalia Sacra Scriptura antiquitatibus orientalibus
%  illustrata}.
%\newblock Pontificial Biblical Institute, Rome, 1995.
%\newblock URL
%  \url{http://web.archive.org/web/20051104215025/http://www.nelc.ucla.edu/Faculty/Schniedewind_files/NWSemitic/Aramaic_ABD.pdf}.
%\end{thebibliography}  




