\section{Inscriptional Pahlavi}
\label{s:inscriptionalpahlavi}

Pahlavi or Pahlevi denotes a particular and exclusively written form of various Middle Iranian languages. The essential characteristics of Pahlavi are[1]
the use of a specific Aramaic-derived script, the Pahlavi script;
the high incidence of Aramaic words used as heterograms (called hozwārishn, "archaisms").

Pahlavi compositions have been found for the dialects/ethnolects of Parthia, Parsa, Sogdiana, Scythia, and Khotan.[2] Independent of the variant for which the Pahlavi system was used, the written form of that language only qualifies as Pahlavi when it has the characteristics noted above.


Pahlavi is then an admixture of
written Imperial Aramaic, from which Pahlavi derives its script, logograms, and some of its vocabulary.

spoken Middle Iranian, from which Pahlavi derives its terminations, symbol rules, and most of its vocabulary.
Pahlavi may thus be defined as a system of writing applied to (but not unique for) a specific language group, but with critical features alien to that language group. It has the characteristics of a distinct language, but is not one. It is an exclusively written system, but much Pahlavi literature remains essentially an oral literature committed to writing and so retains many of the characteristics of oral composition.

\begin{scriptexample}[]{Pahlavi}
\unicodetable{inscriptionalpahlavi}{"10B60,"10B70}
\end{scriptexample}