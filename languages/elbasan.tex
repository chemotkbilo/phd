\section{Elbasan}

\newfontfamily\elbasan{Albanian.otf}
The Elbasan script is a mid 18th-century alphabetic script used for the Albanian language. It was named after the city of Elbasan where it was invented. It was mainly used in the area of Elbasan and Berat. It is widely considered to be the first original alphabet developed for transcribing the Albanian language.

The primary document associated with the alphabet is the Elbasan Gospel Manuscript, known in Albanian as the Anonimi i Elbasanit (The Anonymous of Elbasan).[1] The document was created at St. Jovan Vladimir's Church in central Albania, but is preserved today at the National Archives of Albania in Tirana. Its 59 pages contain Biblical content written in an alphabet of 40 letters,[1] of which 35 frequently recur and 5 are rare. Dots are used on three characters as inherent features of them to indicate varied pronunciation (pre-nasalization and gemination) found in Albanian. The script generally uses Greek letters as numerals with a line on top.

Another original script used for Albanian, was Beitha Kukju's script of the 19th century. This script did not have much influence either.

Elbasan is a simple alphabetic script written from left to right horizontally. The alphabet consists of forty letters.

\subsection{Accents and Other Marks}

The Elbasan manuscript contains breathing accents, similar to
those used in Greek. Those accents do not appear regularly in the orthography and have
not been fully analyzed yet. Raised vertical marks also appear in the manuscript, but are
not specific to the script. Generic combining characters from the Combining Diacritical
Marks block can be used to render these accents and other marks.

\subsection{Names}

The names used for the characters in the Elbasan block are based on those of the
modern Albanian alphabet.

\subsection{Numerals and punctuation}

There are no script-specific numerals or punctuation marks.
A separating dot and spaces appear in the Elbasan manuscript, and may be rendered with
U+00B7 middle dot and U+0020 space, respectively. For numerals, a Greek-like system
of letter and combining overline is in use. Overlines also appear above certain letters in
abbreviations, such as $\overline{\text{\elbasan\char "10507\char"1051D}}$ to indicate Zot (Lord). The overlines in numerals and abbreviations
can be represented with U+0305 combining overline. (See also \href{charts}{http://www.unicode.org/charts/}.)

\subsection{unicode}

Elbasan is a Unicode block containing the historic Elbasan characters for writing the Albanian language. Free fonts for personal use can be found at \href{http://www.fontspace.com/category/unicode\%20font\%20for\%20elbasan}{fontspace}, which I have used here. Commercial fonts can be found at Evertype.

\begin{scriptexample}[]{Elbasan}
\unicodetable{elbasan}{"10500,"10510,"10520}
\end{scriptexample}