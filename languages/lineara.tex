\section{Linear A}
\label{s:lineara}

\section{Aegean and Cypriote Syllabaries}

The Greeks had evidently already occupied the mainland and islands of the
Ægean, including Crete, by the middle of the third millennium
BC. Around 2000 BC, following their consolidation of power on
Crete, new wealth from trade with cosmopolitan Canaan
allowed the creation of a complex palace economy, with major
centres at Knossos, Phaistos and other Cretan sites – Europe’s
first high civilization, the Minoan. Trade with Canaan had evidently
also brought Greeks into contact with Byblos’ pictorial
syllabic writing, whose underlying principle the Minoans borrowed.
Now, Cretans could also write their Minoan Greek language
using a small corpus of syllabo-logographic signs
representing \textit{in-di-vi-du-al} syllables. The signs themselves and
their phonetic values – nearly all V (e) or CV (te) – were wholly
indigenous: what the rebus signs, all originating from the
Cretan world, depicted, one pronounced in Minoan Greek, not
in a Semitic language. (Minoan Greek appears to have been an
archaic sister tongue of the mainland’s Mycenæan Greek.\footnote{A History of Writing. })

Three separate but related forms of syllabo-logographic
writing emerged in the Ægean between c. 2000 and 1200 BC: the
Minoan Greeks’ ‘hieroglyphic’ script and Linear A, and the
later Mycenæan Greeks’ Linear B. Minoan Greeks apparently
also took their writing at an early date to Cyprus, where it experienced
two stages: Cypro-Minoan (evidently derived from
Linear A is one of two currently undeciphered writing systems used in ancient Greece. Cretan hieroglyphic is the other. Linear A was the primary script used in palace and religious writings of the Minoan civilization. It was discovered by archaeologist Arthur Evans. It is the origin of the Linear B script, which was later used by the Mycenaean civilization.

Linear A and its daughter Linear C, the ‘Cypriote Syllabic
Script’. All Ægean and Cypriote scripts are clearly syllabologographic,
as the objective identity of each rebus sign would
have been immediately recognizable to each learner and user. It
seems that determinatives were never employed in any of the
Ægean or Cypriote scripts; however, logograms additionally
depicted most spelt-out items on accounting tablets. All Ægean
and Cypriote scripts, but for these separate logograms, were
completely phonetic.
\medskip

\includegraphics[width=0.8\textwidth]{./images/cretan-hieroglyphs.png}

\medskip
Crete’s `hieroglyphic’ script is the patriarch of this robust
family, its inspiration perhaps derived from Byblos via Cyprus
around 2000 BC. As its name implies, this script used
pictorial signs to reproduce the syllabic inventory of the Minoan
Greek language, here used in rebus fashion as at Byblos. This
writing occurs on seal stones (and their clay impressions), baked
clay, and metal and stone objects, most of these discovered at
Knossos and dating from 2000– 1400 BC (the script was concurrent
with Linear A). There exist about 140 different signs in all –
that is, 70 to 80 syllabic signs and their alloglyphs (different signs
with the same sound value), as well as logograms: human figures,
parts of the body, flora, fauna, boats and geometrical shapes.
Writing direction was open: from left to right, from right to left,
with every other line reversed, even spiral. That this script also
included logograms and numerals suggests that it was initially
used for book-keeping, among other things, until its replacement
in this function with its simplification, Linear A. Thereafter, like
Anatolian hieroglyphs, the Cretan hieroglyphic script appears to
have assumed a ceremonial role in Minoan Greek society,
reserved for sacred inscriptions, dedications and royal proclamations
on round clay disks.

In the 1950s, Linear B was largely udeciphered and found to encode an early form of Greek. Although the two systems share many symbols, this did not lead to a subsequent decipherment of Linear A. Using the values associated with Linear B in Linear A mainly produces unintelligible words. If it uses the same or similar syllabic values as Linear B, then its underlying language appears unrelated to any known language. This has been dubbed the Minoan language.\footnote{\url{http://www.people.ku.edu/~jyounger/LinearA/LinAIdeograms/}}

\begin{scriptexample}[]{Linear A}
\unicodetable{lineara}{  
\number"10600,"10610,"10620,"10630,"10640,"10650,"10660,"10670,
"10680,"10690,"106A0,"106B0,"106C0,"106D0,"106E0,"106F0,"10710,"10720,"10730,"10740,"10750,"10760,"10770}
\end{scriptexample}

Many of the characters form group and specialists name them such as vases in transliterations.

\begin{scriptexample}[]{Vases}
\begin{center}
\scalebox{3}{{\lineara \char"106A6}}
\scalebox{3}{{\lineara \char"106A5}}
\scalebox{3}{{\lineara \char"106A7}}
\scalebox{3}{{\lineara \char"106A9}}
\end{center}
\end{scriptexample}

Linear A contains more than 90 signs (open vowels and consonants+vowels) in regular use and a host of
logograms, many of which are ligatured with syllabograms and/or fractions; about 80\% of these
logograms do not appear in Linear B. While many of Linear A’s signs are also found in Linear B, some
signs are unique to A (e.g., A *301 and following), while some signs found in Linear B are not yet found
in Linear A (e.g., B 12, 14-15, 18-19, 25, 32-33, 36, 42-43, 52, 62-64, 68, 71-72, 75, 83-84, 89-91).

The Unicode Linear A encoding is broadly based on the GORILA ([{\arial ɡɔɹɪˈlɑː}]) catalogue
(Godart and Olivier 1976–1985)\citep{gorila}, which is the basic set of characters used in decipherment efforts.However, “ligatures” which consist of simple horizontal juxtapositions are not uniquely encoded here, as
these may be composed of their constituent parts. On the other hand, “ligatures” which consist of stacked
or touching elements have been encoded.\footnote{An online resource for ancient writing systems in the mediterranean\protect\url{http://lila.sns.it/mnamon/index.php?page=Risorse&id=19&lang=en}. } 






