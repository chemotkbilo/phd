% \iffalse meta-comment
%<*internal>
\iffalse
%</internal>
%<*readme>
----------------------------------------------------------------
phd-pkgmanager --- a package to shorten preambles
E-mail: yannislaz@gmail.com
Released under the LaTeX Project Public License v1.3c or later
See http://www.latex-project.org/lppl.txt
----------------------------------------------------------------
This file provides a phd for defining a class.
%</readme>
%<*readmemd>
###The `phd` LaTeX2e package

The `phd` latex package and the class with the same name provide
convenient methods to create new styles for books, reports
and articles. It also loads the most commonly used packages 
and resolves conflicts.

This work consists of the file  `phd.dtx`,
and the derived files   `phd.ins`,  `phd.pdf`, and `phd.sty`.

###Installation

run
          phd-lua.bat on windows
           pdflatex phd.dtx
           makeindex -s gind.ist -g phd 

If you have any difficulties with the package come and join us at
http://tex.stackexchange.com and post a new question or
add a comment at http://tex.stackexchange.com/a/45023/963.
or send me a message at  yannislaz at gmail.com

### Documentation

The package was written using the `doc` and `docscript` packages,
so that it is self documented in a literary programming style. 
The .pdf is a fat document, providing over fifty book styles (the
equivalent of classes) plus there is a lot of write-up on the inner
workings of TeX and LaTeX2e. However, you don't need to know much
to use it.

      \usepackage{phd}
      \input{style13}

All choices, are made via an extended key-value interface. 
Although not a compliment, it resembles CSS and the keys are a bit verbose but
attributes are easy to change and have a consistent and easy to remember interface.

To set or add a key we only use one command:

      \cxset{chapter name font-size = Huge,
             chapter number font-size = HUGE} 

### Future Development

This is still an experimental version, but I will retain the
interface in future releases. There is a large amount of
work still to be carried out to improve the template styles
provided, to test it more thoroughly and to add a number of
improvements in the special designs. At present I estimate
that I have completed about 70% of the work that needs
to be done.

__The package as it stands is not production stable.__ 


%</readmemd>
%
%<*TODO>
1. On final round add pkg options. This was left as last in order not to solve problems by adding
    options. Too many options are not a good User Interface.
2.  Finish symbol management, both text and math. Math already 80% incorporated.
3.  Better integration of indexing commands.   
4.  Revisit layout manager for Chapters. Broke again in tests.
5.  Docs. Add all references.
6.  Incorporate phd class for more flexibility.
7.  Improve package manager.
8.  Group script loading for better font management.
9.  General font management to relook it again.
10. Add all style sections (about 100 already prepared). Once they
     are all working issue beta version.
%</TODO>
%<*internal>
\fi
\def\nameofplainTeX{plain}
\ifx\fmtname\nameofplainTeX\else
  \expandafter\begingroup
\fi
%</internal>
%<*install>
\input docstrip.tex
\keepsilent
\askforoverwritefalse
\preamble
----------------------------------------------------------------
phd --- A package to beautify documents.
E-mail: yannislaz@gmail.com
Released under the LaTeX Project Public License v1.3c or later
See http://www.latex-project.org/lppl.txt
----------------------------------------------------------------
\endpreamble

%\BaseDirectory{C:/users/admin/my documents/github/phd}
%\usedir{MWE}
\generate{\file{\jobname.sty}{
  \from{\jobname.dtx}{COUNTERS}}
  }

%\nopreamble\nopostamble

%</install>

%<install>\endbatchfile
%<*internal>
%\usedir{tex/latex/phd}
\generate{
  \file{\jobname.ins}{\from{\jobname.dtx}{install}}
}
\nopreamble\nopostamble

\generate{
	\file{README.txt}{\from{\jobname.dtx}{readme}}
  }

\generate{
  \file{README.md}{\from{\jobname.dtx}{readmemd}}
}
\generate{
  \file{TODO.tex}{\from{\jobname.dtx}{TODO}}
}

\ifx\fmtname\nameofplainTeX
  \expandafter\endbatchfile
\else
  \expandafter\endgroup
\fi
%</internal>
%<*driver>

%\listfiles
%gdef\@onlypreamble{} % TO BE REMOVED NEEDED FOR TUTS
\documentclass[twoside,11pt,a4paper]{ltxdoc}
\usepackage[bottom=2cm]{geometry}
\savegeometry{std}
% \usepackage[style=mla]{biblatex}
\usepackage{phd}
\usepackage{phd-documentation}
\usepackage{phd-counters}
\usepackage{phd-toc}
\usepackage{phd-runningheads}
\usepackage{phd-lowersections}
\usepackage{makeidx}
\usepackage{phd-lists}
\pagestyle{headings}
\sethyperref
\cxset{palette bbc}
\makeindex
\begin{filecontents}{defaults-chapters}
%%    General Defaults for Chapters
\cxset{%    
    chapter title margin-top-width    =  0cm,
    chapter title margin-right-width  =  1cm,
    chapter title margin-bottom-width = 10pt,
    chapter title margin-left-width   = 0pt,
    chapter align                     = left,
    chapter title align               = left, %checked
    chapter name                      = hang,
    chapter format                    = fashion,
    chapter font-size                 = Huge,
    chapter font-weight               = bold,
    chapter font-family               = sffamily,
    chapter font-shape                = upshape,
    chapter color                     = black,
    chapter number prefix             = ,
    chapter number suffix             = ,
    chapter numbering                 = arabic,
    chapter indent                    = 0pt,
    chapter beforeskip                = -3cm,
    chapter afterskip                 = 30pt,
    chapter afterindent               = off,
    chapter number after              = ,
    chapter arc                       = 0mm,
    chapter background-color          = bgsexy,
    chapter afterindent               = off,
    chapter grow left                 = 0mm,
    chapter grow right                = 0mm, 
    chapter rounded corners           = northeast,
    chapter shadow                    = fuzzy halo,
    chapter border-left-width         = 0pt,
    chapter border-right-width        = 0pt,
    chapter border-top-width          = 0pt,
    chapter border-bottom-width       = 0pt,
    chapter padding-left-width        = 0pt,
    chapter padding-right-width       = 10pt,
    chapter padding-top-width         = 10pt,
    chapter padding-bottom-width      = 10pt,
    chapter number color              = white,
    chapter label color               = white,    
    }
 \cxset{    
    chapter number font-size        = huge,
    chapter number font-weight      = bfseries,
    chapter number font-family      = sffamily,
    chapter number font-shape       = upshape,
    chapter number align            = Centering,
    }
\cxset{%    
     chapter title font-size        = Huge,
     chapter title font-weight      = bold,
     chapter title font-family      = calligra,
     chapter title font-shape       = upshape,
     chapter title color            = black,
     }    
\end{filecontents}
%% LaTeX2e file `defaults-chapters'
%% generated by the `filecontents' environment
%% from source `phd-scriptsmanager' on 2015/08/25.
%%
%%    General Defaults for Chapters
\cxset{%
    chapter title margin-top-width    =  0cm,
    chapter title margin-right-width  =  1cm,
    chapter title margin-bottom-width = 10pt,
    chapter title margin-left-width   = 0pt,
    chapter align                     = left,
    chapter title align               = left, %checked
    chapter name                      = hang,
    chapter format                    = fashion,
    chapter font-size                 = Huge,
    chapter font-weight               = bold,
    chapter font-family               = sffamily,
    chapter font-shape                = upshape,
    chapter color                     = black,
    chapter number prefix             = ,
    chapter number suffix             = ,
    chapter numbering                 = arabic,
    chapter indent                    = 0pt,
    chapter beforeskip                = -3cm,
    chapter afterskip                 = 30pt,
    chapter afterindent               = off,
    chapter number after              = ,
    chapter arc                       = 0mm,
    chapter background-color          = bgsexy,
    chapter afterindent               = off,
    chapter grow left                 = 0mm,
    chapter grow right                = 0mm,
    chapter rounded corners           = northeast,
    chapter shadow                    = fuzzy halo,
    chapter border-left-width         = 0pt,
    chapter border-right-width        = 0pt,
    chapter border-top-width          = 0pt,
    chapter border-bottom-width       = 0pt,
    chapter padding-left-width        = 0pt,
    chapter padding-right-width       = 10pt,
    chapter padding-top-width         = 10pt,
    chapter padding-bottom-width      = 10pt,
    chapter number color              = white,
    chapter label color               = white,
    }
 \cxset{
    chapter number font-size        = huge,
    chapter number font-weight      = bfseries,
    chapter number font-family      = sffamily,
    chapter number font-shape       = upshape,
    chapter number align            = Centering,
    }
\cxset{%
     chapter title font-size        = Huge,
     chapter title font-weight      = bold,
     chapter title font-family      = calligra,
     chapter title font-shape       = upshape,
     chapter title color            = black,
     }
  
%\definecolor{bgsexy}{HTML}{FF6927}
%
%\definecolor{creamy}{HTML}{FDEBD7}
\cxset{chapter title color= creamy,
       chapter label color = creamy,
       chapter number color = creamy,
       chapter number font-size = Huge,
       subsection title color = creamy,
       chapter name = CHAPTER,
       chapter label case = upper,
       chapter number align=left,
       part format = traditional,
       part background-color=spot,
       part beforeskip                = -3cm,
       part afterskip                 = 30pt,
       }
\ExplSyntaxOn
\makeatletter
\tl_map_inline:nn
 {
  \@ifpackageloaded
  \@ifpackagewith
  \RequirePackage
  \@ifl@aded
  \@ifpackagelater
  \@ifclasslater
  \@ifclassloaded
  \@ifl@t@r
  \@parse@version
 }
 {
  \tl_remove_once:Nn \@preamblecmds {\do#1}
 }
 \makeatother
\ExplSyntaxOff  
\usepackage[perpage]{footmisc}        
\begin{document}
\parindent1em
\def\secondpageimgdescription {Failaka Islad, Kuwait, building wall showing bullet
  holes, from the Iraqi invasion a quarter-century ago. (Megan O'Toole Al Jazeera }
\coverpage{bullets}{Book Design Monographs}{Camel Press}{COUNTERS}{DESIGN} 
\pagestyle{empty}
%\coverpage{habtoor-city}{Delay Claim}{HLS-DSE/JV}{HABTOOR CITY}{MEP CLAIM} 
\secondpage
\pagestyle{empty}
\clearpage

\tableofcontents

\pagestyle{empty}
\setcounter{secnumdepth}{6}
\parskip0pt plus.1ex minus.1ex
\mainmatter
\pagenumbering{arabic}
\pagestyle{headings}
      
\makeatletter
%\@debugtrue

\makeatother
\makeatletter\@specialfalse\makeatother
\parindent1em
\chapter{The Basic LaTeX3 Syntax and Approach}
 \label{ch:l3intro}
 \cxset{epigraph width=0.7\textwidth}
 \epigraph{A final hint: listen carefully to what language users say they
want, until you have an understanding of what they really want. Then
find some way of achieving the latter at a small fraction of the cost
of the former. This is the test of success in language design, and
of progress in programming methodology. Perhaps these two are the same
subject anyway.}{C.A.R. Hoare, 1973}

		
\epigraph{Frank, in case you needed encouragement, please bear this in mind: I'm very much down at the blunt end of (La)TeX -- almost a total end-user. Following an earlier recommendation in this Q\&A, I visited the expl3 manual and was scared witless... Hope you can understand that---it's not a complaint, just an indication of the intellectual/experience distance from here to there.}{---Brent.Longborough Mar 2 '12 at 9:02 at \href{https://tex.stackexchange.com/questions/45838/what-can-i-do-to-help-the-latex3-project/46427\#46427}{SX.TX}}
 
Niklaus Wirth, the developer of the Pascal language long back in the 70’s wrote a paper titled \emph{On the Design of Programming Languages}. In his paper Wirth advocated that an important aspect of language design is \emph{simplicity}. He later on described the lessons learnt from his own works as:\footnote{\protect\url{http://chrisposkitt.com/tag/wirth/}}:

\begin{enumerate}
\item Writing a program is difficult.
\item Writing a correct program is even more so.
\item Writing a publishable program is exacting.
\item Programs are not written. They grow!
\item Controlling growth needs much discipline.
\item Reducing size and complexity is the triumph.
\item Programs must not be regarded as code for computers, but as literature for humans.
\end{enumerate}

The LaTeX3 syntax can only be described with some awe as `different’, although it retains some remnants of 
\tex’s syntax retaining the backslash, it is so different that many developers and package writers have resisted its adoption irrespective of the fact that it offers some solid code. 

Resistance to the language is understandable and noticed early by Computer Science pioneers. Hoare wrote:

\cxset{quotation font-size=\normalsize}
\begin{quotation}
A necessary condition for the achievement of any of these objectives
is the utmost simplicity in the design of the language. Without simplicity,
even the language designer himself cannot evaluate the consequences of his
design decisions. Without simplicity, the compiler writer cannot achieve
even reliability, and certainly cannot construct compact, fast and
- efficient compilers. But the main beneficiary of simplicity is the user
of the language. In all spheres of human intellectual and practical
activity, from carpentry to golf, from sculpture to space travel, the
true craftsman is the one who thoroughly understands his tools. And this
applies to programmers too. A programmer who fully understands his
language can tackle more complex tasks, and complete them quicker and
more satisfactorily than if he did not. In fact, a programmer's need
for an understanding of his language is so great, that it is almost
impossible to persuade him to change to a new one. No matter what the
deficiencies of his current language, he has learned to live with them;
he has learned how to mitigate their effects by discipline and documentation,
and even to take advantage of them in ways which would be impossible
in a new and cleaner language which avoided the deficiency.

It therefore seems especially necessary in the design of a new
programming language, intended to attract programmers away from their
current high level language, to pursue the goal of simplicity to an
extreme, so that a programmer can readily learn and remember all its
features, can select the best facility for each of his purposes, can
fully understand the effects and consequences of each decision, and can
then concentrate the major part of his intellectual effort to understanding
his problem and his programs rather than his tool.
\end{quotation}

I have been programming for many years and have a disdain for languages that---as Hoare
put it--- I cannot remember ``all its features’’.  LaTeX3 has not achieved the level of simplicity required in its core. As a tool it fails the simplicity test and effortful learning is necessary to use it effectively. Currently there are probably less than twenty developers that understand it fully. 

Where, \latex3 excels is its architecture, overall plan and direction and modularizing the code to an extend that the required tools reside in logically set modules or classes in \latex’s terminology. What I can promise you, once you master it, there is no looking back. 

\section{Is it stable?}

One question that often arises is the stability of the current \latex~3 code base. Of course the degree to which software are “stable enough” depends on the requirements. Joseph Wright, answering a question on the SX.TX Q\&A site wrote:

\begin{latexquotation}
If you want 'will never change again', then plain TeX is probably your best bet. Knuth does still fix bugs periodically, but most things are now likely to be regarded as 'features' rather than bugs and so it's extremely likely that a document written in plain today will still work totally unchanged in tens of years (assuming TeX systems continue to be available).

The LaTeX2e kernel is also very unlikely to change further, and so is almost if not quite as stable as TeX itself. The team do fix bugs and do allow a bit more leeway than Knuth does, but even so it's extremely unlikely anything will change with LaTeX2e at the kernel level in a way that would require changes in documents.

There are some LaTeX packages one could reasonably decide to use which are also very stable and unlikely to see changes, either because they are no longer being actively developed or because the authors are careful to only change code related to genuine bugs or new, non-breaking, features. Obvious candidates are keyval, graphicx, etc.: probably there is actually quite a decent list, depending on your requirements.

In the case of the LaTeX3 packages l3kernel and l3packages, 'stable' does not extend as far as 'you will never have to make a change to a document using them', at least at this stage. What it means is that the team will not be making 'arbitrary' changes and will document/announce when this happens. Most of l3kernel is 'done', with the plans primarily focused on addition of new functionality rather than altering existing code. However there are a few places where we know some change may be required, and that will be announced on the LaTeX-L mailing list and documented. Even within these changes, 'breaking' (non-back-compatible) alterations will be small in number, but there is at least one of them we still need to do.

In the case of xparse, \docAuxCommand*{DeclareDocumentCommand} and so on are 'stable' in the sense that they will only be augmented, not removed, but there could be some changes on the more esoteric functions (for example, there are questions centred on the \textbf{g} argument type).

Thus 'stable enough' depends on your use case. If you can live with 'will have to make very occasional changes based on documented and scheduled updates' then expl3 is entirely usable. (I and others use if routinely in packages.) On the other hand, if you want 'this code must work with no changes with all future releases of support code' then we are not quite there yet.
\end{latexquotation}

\section{Getting started}

Other than the obvious of making sure you have the latest distribution from the LaTeX3 repository, 
the first step is to understand the conventions used by the \LaTeX3 developers. Macros are termed \meta{functions} and \meta{variables}. Macro names in general use the underscore and the colon in their names.
This is by design and to be honest is part of what many developers are unhappy about. It does cut down on the readability of the code and the longer names are more difficult to remember. This type of naming convention is similar to Hungarian notation, in which the name of a variable or function indicates its type  or its intended use and it does not have a lot of friends.


Consider the \tex primitive \docAuxCommand*{meaning}. In \latex3 it has been remapped to \docAuxCommand*{token_to_meaning:N}. Similarly \docAuxCommand*{scan_stop:} has been let to \docAuxCommand*{relax}.

\begin{texexample}{Getting started}{ex:meaning}
\ExplSyntaxOn
\def\somevar{one}
\token_to_meaning:N \scan_stop:  \\
\meaning\somevar \\
\token_to_meaning:N \somevar \\
\token_to_meaning:N \token_to_meaning:N
\ExplSyntaxOff
\end{texexample}

The part that comes after the colon is termed the \emph{function signature}. For example in |token_to_meaning:N|, the function signature is the \textbf{N}. The individual letter “N” is termed the argument specifier. Another important part is the prefix of the functions. There are some exceptions but the prefix normally indicates the module where the macro has been defined. So |\token_to_meaning:N|  can be found in the |l3token| package.\footnote{The term module and package are used interchangeably by the \latex3 Team.}

Consider the definition of a simple function  |\phd_print_xy:nn| that accepts two values $x,y$ and prints them. This can be defined by one of the |cs_| type functions.

One way we could have defined the macro using  \tex would be:

\begin{teXXX}
\def\phdprint #1#2{x#1 y#2}
\end{teXXX}

Using \latexe we would have probably used |\newcommand| and if the definition was internal to a package used an |@|. 

\begin{teXXX}
\makeatletter
\newcommand\phd@print [2] {x#1 y#2}
\makeatother
\end{teXXX}

In \latex3 we would use |\cs_set_no_par:Npn|.

\begin{teXXX}
\cs_set_nopar:Npn \phd_print_xy:nn #1#2 { x #1 y #2 }
\end{teXXX}

So what is this mysterious |\cs_set_nopar:Npn|? We can find out by peeking at its meaning. This is shown in Example~\ref{ex:somemeaning}. As you can see behind the new dress is Knuth’s same old |\def|.

\begin{texexample}{The meaning of a command}{ex:somemeaning}
\ExplSyntaxOn
\token_to_meaning:N  \cs_set_nopar:Npn
\ExplSyntaxOff
\end{texexample}

But first let us examine the |:Npn| part of the |\cs_set_no_par:Npn| more carefully. What this means is the macro has three arguments. The first one is N-type which is a \tex token. The second one is p-type, which denotes normal \tex parameters such as |#1#2|. Lastly the n-type can be either a single token or a bracketted parameter. 

There are many more argument specifiers. Functions can be found with different argument specifiers and these are termed \emph{variants}. Recall that a macro can be defined using |\def|, |\edef| or a |\csname| construct. The argument specifier to the |\cs_setnopar| can be varied to achieve it. 

\begin{texexample}{The meaning of a command}{ex:somemeaning}
\ExplSyntaxOn
\token_to_meaning:N  \cs_set_nopar:Npx\\
\token_to_meaning:N  \cs_set_nopar:cpx\\
\ExplSyntaxOff
\end{texexample}

Consider the use  of a |\csname| construct to define our |\phd_print_xy:nn| macro. The example that follows

\begin{texexample}{ex:csname}{ex:csname}

\ExplSyntaxOn
\expandafter\def\csname phd_print_xy:nn\endcsname #1 #2{x#1 y#2}

\token_to_meaning:N \phd_print_xy:nn\\

\cs_set_nopar:cpx {phd_print_xy:nn} #1 #2 {x#1 y#2}
\token_to_meaning:N \phd_print_xy:nn\\
\ExplSyntaxOff
\end{texexample} 

By using \latex3 functions, we do not need to use the |\expandafter| macro. The macros are generally longer but the overall code is shorter.

So far we have used the |\token_to_meaning:N|. \latex3 offers similar commands to get the argument specification, the prefix and the replacement specification. When we specify a macro in \latex3 we can capture all its constituent parts and handle them individually if we want.

\begin{texexample}{Dissecting a macro}{}
\ExplSyntaxOn
\cs_set_nopar:Npn \phd_print_xy:nn #1#2! { x #1 y #2 }
\token_to_meaning:N \phd_print_xy:nn \\
\token_get_arg_spec:N \phd_print_xy:nn  \\
\token_get_prefix_spec:N \phd_print_xy:nn\\
\token_get_replacement_spec:N \phd_print_xy:nn\\
\ExplSyntaxOff
\end{texexample}


Some argue that the syntax is not syntactic sugar but syntactic cyanide that changes the look and feel both of \latexe and \tex command macros. You should think of |expl3| as a new computer language. It does introduce consistency and offers a full repertoire of tools. The syntactic strangeness of the language does introduce barriers to mastering it, but the advantages far outweigh the difficulties of the language.


The eye tends to miss the argument specifier, it is important to note that the macro
name is \cmd{\test\_something:nn} and not \cmd{\test\_something} and the factory command is |\cs_new:Npn| and not |\cs_new|. If you have been programming using traditional macros this is a common mistake that you will accidentally make and you will get an |error unknown| message.

\section{Where from here}

The chapters of this book follow a logical sequence for learning the language, although most of them can be read as stand alone. 

The steps in learning any computer language require a logical sequence of study:

\begin{enumerate}
\item Understanding the syntax
\item Variables and datatypes
\item Numbers and assignments
\item Control Structures
\item Functions
\item Data structures
\item Ecosystem
\end{enumerate}

In the next chapter we would study the creation of functions in more detail. This is the most important skill to master before you proceed with the rest of the programming constructs, such as iteration, arithmetic operations etc.



\chapter{Defining Functions and Variables}

\section{Defining functions}
There are two main methods to define functions. In the first method you are required to use parameter tex, whereas in the second this can be left out, as it can be inferred from the argument specification of the function being defined. The functions used to create other functions can be found in both forms. For example:

\begin{texexample}{Using parameter text}{}
\ExplSyntaxOn
\cs_set_nopar:Npn \phd_print:n #1 {#1}
\token_to_meaning:N \phd_print:n\\

\cs_set_nopar:Nn  \phd_print:n  {#1}


\token_to_meaning:N \phd_print:n\\
\ExplSyntaxOff
\end{texexample}



 Functions can be created with no requirement that they are declared
 first (in contrast to variables, which must always be declared).\footnote{This primarily refers to variables that require a \tex register.}
 Declaring a function before setting up the code means that the name
 chosen will be checked and an error raised if it is already in use.
 The name of a function can be checked at the point of definition using
 the \docAuxCommand*{cs_new}\ldots functions: this is recommended for all
 functions which are defined for the first time.

 There are three primary ways to define new functions, using |new|, |set| or |gset| variations.  The first one is similar to the \latexe |\newcommand|, and produces macros that will generate an error if there is an attempt to redefine them. The other two are variations of the |\def or \edef| and |\gdef or \xdef| \tex commands.
 
 All classes define a function to expand to the substitution text.
 Within the substitution text the actual parameters are substituted
 for the formal parameters (|#1|, |#2|, \ldots).
 
 \begin{description}
   \item[\texttt{new}]
     Create a new function with the \texttt{new} scope,
     such as \docAuxCommand* {cs_new:Npn}.  The definition is global and will result in
     an error if it is already defined.
   \item[\texttt{set}]
     Create a new function with the \texttt{set} scope,
     such as \docAuxCommand* {cs_set:Npn}. The definition is restricted to the current
     \TeX{} group and will not result in an error if the function is already
     defined.
   \item[\texttt{gset}]
     Create a new function with the \texttt{gset} scope,
     such as \docAuxCommand* {cs_gset:Npn}. The definition is global and
     will not result in an error if the function is already defined.
 \end{description}

  Finally, the functions in
 Subsections~\ref{sec:l3basics:defining-new-function-1}~and
 \ref{sec:l3basics:defining-new-function-2} are primarily meant to define
 \emph{base functions} only. Base functions can only have the following
 argument specifiers:
 \begin{description}
   \item[|N| and |n|] No manipulation.
   \item[|T| and |F|] Functionally equivalent to |n| (you are actually
     encouraged to use the family of |\prg_new_conditional:| functions
     described in Section~\ref{sec:l3prg:new-conditional-functions}).
   \item[|p| and |w|] These are special cases.
 \end{description}



 Within each set of scope there are different ways to define a function.
 The differences depend on restrictions on the actual parameters and
 the expandability of the resulting function.
 \begin{description}
   \item[\texttt{nopar}]
      Create a new function with the \texttt{nopar} restriction,
      such as \docAuxCommand*{cs_set_nopar:Npn}. The parameter may not contain
      \docAuxCommand*{par} tokens.
   \item[\texttt{protected}]
      Create a new function with the \texttt{protected} restriction,
      such as \docAuxCommand*{cs_set_protected:Npn}. The parameter may contain
      \docAuxCommand*{par} tokens but the function will not expand within an
      \texttt{x}-type expansion.
 \end{description}
 
 
\subsection{Defining new functions using parameter text}

Theses function are \TeX ish in style, as compared to those functions that use the signature to automatically detect the number of parameters and are more \LaTeX-like. They are mainly used with the |:Npn| signature specification.

\begin{texexample}{Using parameter text}{}
\ExplSyntaxOn
\cs_new:Npn \phd_print:n #1 {#1}

\token_to_meaning:N \cs_new:Npn\\
\token_to_meaning:N \phd_print:n\\
\ExplSyntaxOff
\end{texexample}

\begin{docCommand}{cs_new:Npn} {\meta{function} \meta{parameters} \marg{code}}
Creates \meta{function} to expand to \meta{code} as replacement text. Within the \meta{code}, the
\meta{parameters} (\#1, \#2, etc.) will be replaced by those absorbed by the function. The
definition is \textbf{global} and an error will result if the \meta{function} is already defined.
Variants with |cpn,Npx,cpx| are predefined by the kernel.
\end{docCommand}

The |:Npn| form can also be used even if there is no parameter text. However this is considered a constant variable and is preferred to be coded as a |tl| such.

\begin{texexample}{Usage of the macro}{ex:csnew}
\ExplSyntaxOn
  \cs_new:Npn \copyrightfootnote: 
    {
      \footnotetext{Copyright~(2014-2015)~of~Yiannis~Lazarides,~distributed~
      under~the~\LaTeX{}~Project~Public~License~(LPPL).}
    }
  \copyrightfootnote:
\ExplSyntaxOff
\end{texexample}

An important point to note is if you use the function signature type you will get an error if the trailing |:| is not used in the macro name. 

\begin{teXXX}
\cs_new:Nn \copyrightafootnote 
  {
    ...
  }
\copyrightafootnote
\end{teXXX}

This will produce an error:\ExplSyntaxOn\copyrightfootnote:\ExplSyntaxOff

\begin{verbatim}
! LaTeX error: "kernel/missing-colon"
! Function '\copyrightafootnote' contains no ':'.
! See the LaTeX3 documentation for further information.
! For immediate help type H <return>.
\end{verbatim}

If the function is redefined, it will produce an error, similar to \latexe |\newcommand|. However, do note that the |set| family of commands can silently overwrite it. 

\begin{texexample}{Usage of the macro \protect\string\cs\_gset:Npn}{ex:csnew}
\ExplSyntaxOn
\cs_gset:Npn \copyrightfootnote: {\footnotetext{Copyright~(2014-2015)~of~Yiannis~Lazarides,~distributed~
under~the~\LaTeX{}~Project~Public~License~(LPPL).}}
\copyrightfootnote:
\ExplSyntaxOff
\end{texexample}

\begin{docCommand}{cs_new_nopar:Npn} {\meta{function} \meta{parameters} \marg{code}}
Creates \meta{function} to expand to \meta{code} as replacement text. Within the \meta{code}, the
\meta{parameters} (\#1, \#2, etc.) will be replaced by those absorbed by the function. When the
\meta{function} is used the hparametersi absorbed cannot contain \par tokens. The definition
is global and an error will result if the \meta{function} is already defined.
\end{docCommand}

\begin{texexample}{Meaning}{}
\ExplSyntaxOn
\token_to_meaning:N \cs_new_nopar:Npn
\ExplSyntaxOff
\end{texexample}

\begin{docCommand}{cs_new_protected:Npn}{\meta{function} \meta{parameters} \marg{code}}
Creates \meta{function} to expand to \meta{code} as replacement text. Within the hcodei, the
hparametersi (\#1, \#2, etc.) will be replaced by those absorbed by the function. The
\meta{function} will not expand within an x-type argument. The definition is global and an
error will result if the hfunctioni is already defined.
\end{docCommand}

\begin{docCommand}{cs_new_protected_nopar:Npn}{\meta{function} \meta{parameters} \marg{code}}
Creates \meta{function} to expand to \meta{code} as replacement text. 
When the \meta{function} is used the \meta{parameters} absorbed cannot contain \docAuxCommand*{par} tokens. The hfunctioni
will not expand within an x-type argument. The definition is global and an error will
result if the \meta{function} is already defined.
\end{docCommand}

This brings us to the end of the |new| type functions that can be used for function definitions. They all have variants of the form |cpn| and |cpx| and the base function for edef also is available. You can consult the manual for more definitions.

\subsubsection{The set type functions}

The rest of the commands are variations using the |set| form of function creating macros. These do not issue a 
warning if redefined.

 \begin{docCommand}{cs_set:Npn} {\meta{function} \meta{parameters} \marg{code}}
   Sets \meta{function} to expand to \meta{code} as replacement text.
   Within the \meta{code}, the \meta{parameters} (|#1|, |#2|,
   \emph{etc.}) will be replaced by those absorbed by the function.
   The assignment of a meaning to the \meta{function} is restricted to
   the current \TeX{} group level.
\end{docCommand}

\begin{texexample}{Meaning}{}
\ExplSyntaxOn
\token_to_meaning:N \cs_set:Npn
\ExplSyntaxOff
\end{texexample}

As can be seen from the example this is |\protected \long \def|. The |\cs_set_nopar:Npn| in the maeaning in the example is described next and is simply an equivalent function to |\def|.

 \begin{docCommand} {cs_set_nopar:Npn}{\meta{function} \meta{parameters} \marg{code}}
   Sets \meta{function} to expand to \meta{code} as replacement text.
   Within the \meta{code}, the \meta{parameters} (|#1|, |#2|,
   \emph{etc.}) will be replaced by those absorbed by the function.
   When the \meta{function} is used the \meta{parameters} absorbed
   cannot contain \cs{par} tokens. The assignment of a meaning
   to the \meta{function} is restricted to the current \TeX{} group
   level.
 \end{docCommand}
 
 \begin{texexample}{Meaning \textbackslash cs\_set\_nopar:Npn}{}
\ExplSyntaxOn
\token_to_meaning:N \cs_set_nopar:Npn
\ExplSyntaxOff
\end{texexample}
 

\begin{docCommand}{cs_set_protected:Npn} {\meta{function} \meta{parameters} \marg{code}}
   Sets \meta{function} to expand to \meta{code} as replacement text.
   Within the \meta{code}, the \meta{parameters} (|#1|, |#2|,
   \emph{etc.}) will be replaced by those absorbed by the function.
   The assignment of a meaning to the \meta{function} is restricted to
   the current \TeX{} group level. The \meta{function} will
   not expand within an \texttt{x}-type argument.
 \end{docCommand}
 \begin{texexample}{Meaning \textbackslash cs\_set\_protected:Npn}{}
 \ExplSyntaxOn
 \token_to_meaning:N \cs_set_protected:Npn
\ExplSyntaxOff
\end{texexample}
 


\begin{docCommand}{cs_set_protected_nopar:Npn}{\meta{function} \meta{parameters} \marg{code}}
   Sets \meta{function} to expand to \meta{code} as replacement text.
   Within the \meta{code}, the \meta{parameters} (|#1|, |#2|,
   \emph{etc.}) will be replaced by those absorbed by the function.
   When the \meta{function} is used the \meta{parameters} absorbed
   cannot contain \cs{par} tokens. The assignment of a meaning
   to the \meta{function} is restricted to the current \TeX{} group
   level. The \meta{function} will not expand within an
   \texttt{x}-type argument.
\end{docCommand}
\begin{texexample}{Meaning \textbackslash cs\_set\_protected\_nopar:Npn}{}
\ExplSyntaxOn
\token_to_meaning:N \cs_set_protected_nopar:Npn
\ExplSyntaxOff
\end{texexample}
 
Next the above are made available by the \latex3 kernel but all in the |global| form of the command. The syntax is identical except they use |cs_gset|.


\begin{docCommand} {cs_gset:Npn}{\meta{function} \meta{parameters} \marg{code}}
   Globally sets \meta{function} to expand to \meta{code} as replacement
   text. Within the \meta{code}, the \meta{parameters} (|#1|, |#2|,
  \emph{etc.}) will be replaced by those absorbed by the function.
  The assignment of a meaning to the \meta{function} is \emph{not}
   restricted to the current \TeX{} group level: the assignment is
   global.
\end{docCommand}
\begin{texexample}{Meaning \textbackslash cs\_gset:Npn}{}
\ExplSyntaxOn
\token_to_meaning:N \cs_gset:Npn
\ExplSyntaxOff
\end{texexample}

\begin{docCommand}{cs_gset_nopar:Npn} {\meta{function} \meta{parameters} \marg{code}}
   Globally sets \meta{function} to expand to \meta{code} as replacement
   text. Within the \meta{code}, the \meta{parameters} (|#1|, |#2|,
   \emph{etc.}) will be replaced by those absorbed by the function.
   When the \meta{function} is used the \meta{parameters} absorbed
   cannot contain \cs{par} tokens. The assignment of a meaning to the
   \meta{function} is \emph{not} restricted to the current \TeX{}
   group level: the assignment is global.
\end{docCommand}
\begin{texexample}{Meaning \textbackslash cs\_gset\_nopar:Npn}{}
\ExplSyntaxOn
\token_to_meaning:N \cs_gset_nopar:Npn
\ExplSyntaxOff
\end{texexample}


\begin{docCommand} {cs_gset_protected:Npn} {\meta{function} \meta{parameters} \marg{code}}
   Globally sets \meta{function} to expand to \meta{code} as replacement
   text. Within the \meta{code}, the \meta{parameters} (|#1|, |#2|,
   \emph{etc.}) will be replaced by those absorbed by the function.
   The assignment of a meaning to the \meta{function} is \emph{not}
   restricted to the current \TeX{} group level: the assignment is
   global. The \meta{function} will not expand within an
   \texttt{x}-type argument.
\end{docCommand}
\begin{texexample}{Meaning \textbackslash cs\_gset\_protected:Npn}{}
\ExplSyntaxOn
\token_to_meaning:N \cs_gset_protected:Npn
\ExplSyntaxOff
\end{texexample}

\begin{docCommand}{cs_gset_protected_nopar:Npn} {\meta{function} \meta{parameters} \marg{code}}
   Globally sets \meta{function} to expand to \meta{code} as replacement
   text. Within the \meta{code}, the \meta{parameters} (|#1|, |#2|,
   \emph{etc.}) will be replaced by those absorbed by the function.
   When the \meta{function} is used the \meta{parameters} absorbed
   cannot contain \cs{par} tokens. The assignment of a meaning to the
   \meta{function} is \emph{not} restricted to the current \TeX{}
   group level: the assignment is global. The \meta{function} will
   not expand within an \texttt{x}-type argument.
\end{docCommand}
\begin{texexample}{Meaning \textbackslash cs\_gset\_protected\_nopar:Npn}{}
\ExplSyntaxOn
\token_to_meaning:N \cs_gset_protected_nopar:Npn
\ExplSyntaxOff
\end{texexample}

This brings us to the end of the functions available to the developer for defining macros. It’s a lot of them. In the next section some more functions are defined, this time using the signature of the function the function are created automatically without the need to type in the parameter text.


\subsection{Defining new functions using the signature}

The functions outlined below have a simpler form in that they create other commands without the need to specify their arguments. The number of parameters is detected automatically from the function signature. Which method is the best is obvious up to the user preferences.\footnote{See discussion at SX.TX \protect{\url{http://tex.stackexchange.com/questions/240675/differences-in-latex3-function-generation-methods}}} 


\begin{docCommand}{cs_new:Nn}{\meta{function}\marg{code}}
Creates \meta{function} to expand to \meta{code} as replacement text. A nice feature is that within the \meta{code}
the number of parameters is detected automatically from the function signature. These \meta{parameters} (\#1, \#2, etc.) will be replaced by those absorbed by the function. The definition is global and an error will result if the \meta{function} is already defined.\footnote{The definitions of the commands have been taken mostly verbatim from the documentation of the package.}


\begin{texexample}{Signature}{ex:signature}
\ExplSyntaxOn
\cs_new:Nn \exampleone:nn {}
\cs_new:Nn \exampletwo:nn{#1 #2}
\exampleone:nn {one}{two}

\exampletwo:nn{one }{two}

\texttt\textbackslash\cs_to_str:N\exampleone:nn
\ExplSyntaxOff
\end{texexample}
\end{docCommand}

 
 
 
\begin{docCommand}{cs_new_nopar:Nn}{\meta{function} \marg{code}}
   Creates \meta{function} to expand to \meta{code} as replacement text.
   Within the \meta{code}, the number of \meta{parameters} is detected
   automatically from the function signature. These \meta{parameters}
   (|#1|, |#2|, \emph{etc.}) will be replaced by those absorbed by the
   function.  When the \meta{function} is used the \meta{parameters}
   absorbed cannot contain \docAuxCommand*{par} tokens. The definition is global and
   an error will result if the \meta{function} is already defined.
 \end{docCommand}

\begin{docCommand}{cs_new_protected:Nn}{\meta{function} \marg{code}}
   Creates \meta{function} to expand to \meta{code} as replacement text.
   Within the \meta{code}, the number of \meta{parameters} is detected
   automatically from the function signature. These \meta{parameters}
   (|#1|, |#2|, \emph{etc.}) will be replaced by those absorbed by the
   function. The \meta{function} will not expand within an \texttt{x}-type
   argument. The definition is global and
   an error will result if the \meta{function} is already defined.
\end{docCommand}


%
% \begin{function}
%   {
%     \docAuxCommand*_new_protected_nopar:Nn, \docAuxCommand*_new_protected_nopar:cn,
%     \docAuxCommand*_new_protected_nopar:Nx, \docAuxCommand*_new_protected_nopar:cx
%   }
%   \begin{syntax}
%     \docAuxCommand*{cs_new_protected_nopar:Nn} \meta{function} \Arg{code}
%   \end{syntax}
%   Creates \meta{function} to expand to \meta{code} as replacement text.
%   Within the \meta{code}, the number of \meta{parameters} is detected
%   automatically from the function signature. These \meta{parameters}
%   (|#1|, |#2|, \emph{etc.}) will be replaced by those absorbed by the
%   function.  When the \meta{function} is used the \meta{parameters}
%   absorbed cannot contain \docAuxCommand*{par} tokens. The \meta{function} will not
%   expand within an \texttt{x}-type argument. The definition is global and
%   an error will result if the \meta{function} is already defined.
% \end{function}

Similarly to the |cs_new| commands the |cs_set| functions create other commands, this time
with a local scope. This pattern is followed right through the kernel.

 \begin{docCommand}{cs_set:Nn}{\meta{function}\marg{code}}
   Sets \meta{function} to expand to \meta{code} as replacement text.
   Within the \meta{code}, the number of \meta{parameters} is detected
   automatically from the function signature. These \meta{parameters}
   (|#1|, |#2|, \emph{etc.}) will be replaced by those absorbed by the
   function.
   The assignment of a meaning to the \meta{function} is restricted to
   the current \TeX{} group level.
 \end{docCommand}

\begin{docCommand}{cs_set_nopar:Nn}{\meta{function}\marg{code}}
   Sets \meta{function} to expand to \meta{code} as replacement text.
   Within the \meta{code}, the number of \meta{parameters} is detected
   automatically from the function signature. These \meta{parameters}
   (|#1|, |#2|, \emph{etc.}) will be replaced by those absorbed by the
   function.  When the \meta{function} is used the \meta{parameters}
   absorbed cannot contain \docAuxCommand*{par} tokens.
   The assignment of a meaning to the \meta{function} is restricted to
   the current \TeX{} group level. This is the \tex primitive \docAuxCommand*{def}
\end{docCommand}

\begin{teXXX}
\tex_let:D \cs_set_nopar:Npn \tex_def:D
748 \tex_let:D \cs_set_nopar:Npx \tex_edef:D
749 \etex_protected:D \cs_set_nopar:Npn \cs_set:Npn
750                     { \tex_long:D \cs_set_nopar:Npn }
751 \etex_protected:D \cs_set_nopar:Npn \cs_set:Npx
752                   { \tex_long:D \cs_set_nopar:Npx }
753 \etex_protected:D \cs_set_nopar:Npn \cs_set_protected_nopar:Npn
754 { \etex_protected:D \cs_set_nopar:Npn }
755 \etex_protected:D \cs_set_nopar:Npn \cs_set_protected_nopar:Npx
756 { \etex_protected:D \cs_set_nopar:Npx }
757 \cs_set_protected_nopar:Npn \cs_set_protected:Npn
758 { \etex_protected:D \tex_long:D \cs_set_nopar:Npn }
759 \cs_set_protected_nopar:Npn \cs_set_protected:Npx
760 { \etex_protected:D \tex_long:D \cs_set_nopar:Npx }
\end{teXXX}
\ExplSyntaxOn
\meaning\cs_new:Npn
\ExplSyntaxOff


\begin{docCommand}{cs_set_protected:Nn}{\meta{function}\marg{code}}
   Sets \meta{function} to expand to \meta{code} as replacement text.
   Within the \meta{code}, the number of \meta{parameters} is detected
   automatically from the function signature. These \meta{parameters}
   (|#1|, |#2|, \emph{etc.}) will be replaced by those absorbed by the
   function. The \meta{function} will not expand within an \texttt{x}-type
   argument.
   The assignment of a meaning to the \meta{function} is restricted to
   the current \TeX{} group level.
 \end{docCommand}

\begin{docCommand}{cs_set_protected_nopar:Nn}{ \meta{function} \marg{code}}
   Sets \meta{function} to expand to \meta{code} as replacement text.
   Within the \meta{code}, the number of \meta{parameters} is detected
   automatically from the function signature. These \meta{parameters}
   (|#1|, |#2|, \emph{etc.}) will be replaced by those absorbed by the
   function.  When the \meta{function} is used the \meta{parameters}
   absorbed cannot contain \docAuxCommand*{par} tokens. The \meta{function} will not
   expand within an \texttt{x}-type argument.
   The assignment of a meaning to the \meta{function} is restricted to
   the current \TeX{} group level.
 \end{docCommand}

The next commands create functions with global scope.

 \begin{docCommand}{cs_gset:Nn}{ \meta{function} \marg{code}}
   Sets \meta{function} to expand to \meta{code} as replacement text.
   Within the \meta{code}, the number of \meta{parameters} is detected
   automatically from the function signature. These \meta{parameters}
   (|#1|, |#2|, \emph{etc.}) will be replaced by those absorbed by the
   function.
   The assignment of a meaning to the \meta{function} is  global.
 \end{docCommand}

 \begin{docCommand}{cs_gset_nopar:Nn}{ \meta{function} \marg{code}}
   Sets \meta{function} to expand to \meta{code} as replacement text.
   Within the \meta{code}, the number of \meta{parameters} is detected
   automatically from the function signature. These \meta{parameters}
   (|#1|, |#2|, \emph{etc.}) will be replaced by those absorbed by the
   function.  When the \meta{function} is used the \meta{parameters}
   absorbed cannot contain \docAuxCommand*{par} tokens.
   The assignment of a meaning to the \meta{function} is global.
 \end{docCommand}
 

\section{Copying control sequences}

Control sequences (not just functions as defined above) can be set to have the same
meaning using the functions described here. Making two control sequences equivalent
means that the second control sequence is a copy of the first (rather than a pointer to
it). Thus the old and new control sequence are not tied together: changes to one are not
reflected in the other. These are syntactic replacements for |\let|.

\begin{texexample}{Let}{}
\ExplSyntaxOn
\cs_set_nopar:Nn \testa: {AAA}
\cs_set_eq:NN\testb: \testa:
\token_to_meaning:N \testa:  \\
\cs_set_nopar:Nn \testa: {BBBB}
\testb:  \\
\token_to_meaning:N \testb:  \\
\token_to_meaning:N \testa:  \\
\testa:\\
\testb: \\
\meaning\cs_set_eq:NN

% check if equal to \let
\token_to_meaning:N \let\\
\token_to_meaning:N \cs_set_equal:NN
\ExplSyntaxOff
\end{texexample}

 In the following text \enquote{cs} is used as an abbreviation for
 \enquote{control sequence}.

 \begin{docCommand}{cs_new_eq:NN} {\meta{cs1} \meta{cs2}}
   Globally creates \meta{control sequence 1} and sets it to have the same
   meaning as \meta{control sequence 2} or |<token>|.
   The second control sequence may
   subsequently be altered without affecting the copy.
\end{docCommand}


\begin{docCommand}{cs_set_eq:NN} {\meta{cs1} \meta{cs2}}
   Sets \meta{control sequence1} to have the same meaning as
   \meta{control sequence2} (or |<token>|).
   The second control sequence may subsequently be
   altered without affecting the copy. The assignment of a meaning
   to the \meta{control sequence1} is restricted to the current
   \TeX{} group level.
 \end{docCommand}


\begin{docCommand} {cs_gset_eq:NN} {\meta{cs1} \meta{cs2}}
   Globally sets \meta{control sequence1} to have the same meaning as
   \meta{control sequence2} (or |<token>|).
   The second control sequence may subsequently be
   altered without affecting the copy. The assignment of a meaning to
   the \meta{control sequence1} is \emph{not} restricted to the current
   \TeX{} group level: the assignment is global.
\end{docCommand}

\section{Undefining control sequences}

There are occasions where control sequences need to be deleted. This is handled in a
very simple manner by the use of 
|\cs_undefine:N| \meta{control sequence},
which sets \meta{control sequence} to be globally |undefined|.

\begin{texexample}{Undefining control sequences}{ex:undefine}
\ExplSyntaxOn
\cs_set_nopar:Npn \testa: {AAA}
\cs_set_nopar:cpn {testb} {AAA}

\cs_undefine:N \testa:
\cs_undefine:c {testb}
\token_to_meaning:N \cs_undefine:N\\

\token_to_meaning:N \testa:\\
\token_to_meaning:c {testb}\\
\token_to_meaning:N \token_to_meaning:c
\ExplSyntaxOff
\end{texexample}

The function would simply set the command to the \tex primitive |undefine|, as can be seen from the example.
There is another group of commands associated with constructor functions.

\section{Converting to and from control sequences}

\begin{docCommand}{cs_if_exist_use:N} {\meta{control sequence}}
Tests whether the \meta{control sequence} is currently defined (whether as a function or another
control sequence type), and if it does inserts the \meta{control sequence} into the input stream.
\end{docCommand}

\begin{docCommand}{cs_if_exist_use:NTF} {\meta{control sequence}}
Tests whether the \meta{control sequence} is currently defined (whether as a function or another
control sequence type), and if it does inserts the \meta{control sequence} into the input stream
followed by the \meta{true code}.
\end{docCommand}

\begin{texexample}{Converting to and from control sequences}{ex:ifexists}
\ExplSyntaxOn
\cs_if_exist_use:NTF \test {}{\FALSE}
\ExplSyntaxOff
\end{texexample}

Note that numerous times, I have typed |\cs_if_exists_use:NTF| rather than the more grammatical  |\cs_if_exist_use:NTF| with consequent errors. Grammar is hardwired in the brain and it requires mental effort to write ungrammatical commands. This is an issue that needs to be addressed by the \latex3 developers. 

The famous |\csname| is mapped in this section of the module as well. Unpredictably, it got a shorter name, but a weird suffix |w|! It deserves both as it is the workhorse of \tex. The remapped commands are formally described in the manual as shown below:

\begin{docCommand}{cs:w} {\meta{control sequence name} \texttt{cs\_end:}}
Converts the given \meta{control sequence name} into a single control sequence token. This
process requires one expansion. The content for \meta{control sequence name} may be literal
material or from other expandable functions. The \meta{control sequence name} must, when
fully expanded, consist of character tokens which are not active: typically, they will be
of category code 10 (space), 11 (letter) or 12 (other), or a mixture of these.
\end{docCommand}


\section{User Commands}

All the commands above are at the programming level. For the development of user commands the \pkgname{xparse} package provides some extremely useful commands. These are dealt under \nameref{ch:xparse}
on page \pageref{ch:xparse}.

\begin{teXXX}
\NewDocumentCommand{\kant}{s>{\SplitArgument{1}{-}}O{1-7}}
  {
   \group_begin:
   \IfBooleanTF{#1} (*@\label{starargument}@*)
     { \cs_set_eq:NN \kgl_par: \kgl_star: }
     { \cs_set_eq:NN \kgl_par: \kgl_nostar: }
     \kgl_process:nn #2
    \kgl_print:
   \group_end:
  }
\end{teXXX}

In Line~\ref{starargument} we test for the star version of the command and then we continue examining the optional argument |O{1-7}|, but first and here is the magic, we have passed the argument through a pre-processing macro named |\SplitArgument|, which has captured the splitted argument and placed it, into two braced macros. It then passes it to a second macro |\getwords| that expects two mandatory aruguments and which handles the typesetting of the two words.
    
\begin{texexample}{Split Argument}{}    
\NewDocumentCommand{\separatewords}{>{\SplitArgument{1}{-}}m}{\getwords#1}
\NewDocumentCommand{\getwords}{ m m }{First word:#1  Second~Word:#2}
\separatewords{mail-coach}

\separatewords{mail}

\end{texexample}    

A similar example see TX.SX.\footnote{\protect{\url{http://tex.stackexchange.com/questions/154941/new-command-in-tex-for-fraction/154950\#154950}}}


\chapter{LaTeX3 Control Structures}
 \section{The boolean data type}

 This section describes a boolean data type which is closely
 connected to conditional processing as sometimes you want to
 execute some code depending on the value of a switch
 (\emph{e.g.},~draft/final) and other times you perhaps want to use it as a
 predicate function in an |if_predicate:w| test. The problem of the
 primitive \docAuxCommand*{if_false:} and \docAuxCommand*{if_true:} tokens is that it is not
 always safe to pass them around as they may interfere with scanning
 for termination of primitive conditional processing. In \latex3
 two canonical booleans ar employed: \docAuxCommand*{c_true_bool} or
\docAuxCommand{c_false_bool}. Besides preventing problems as described above. This also let
to the implementation of  a simple boolean parser supporting the
 logical operations And, Or, Not, \emph{etc.}\ which can then be used on
 both the boolean type and predicate functions.

 All conditional |\bool_| functions except assignments are expandable
 and expect the input to also be fully expandable (which will generally
 mean being constructed from predicate functions, possibly nested).
 
Before a boolean can be used it needs to be created with \docAuxCommand{bool_new:N}, but first let us make sure we understand what a boolean is. A Boolean data type is a data type, having two values (usually denoted \emph{true} and \emph{false}), intended to represent the truth values of logic and Boolean algebra. It is named after George Boole, who first defined an algebraic system of logic in the mid 19th century. 

So how does \latex3 construct a boolean? If we examine the code, which we will in a small example, we can see that a boolean variable is just another macro that either stores 0 or 1. If the value is odd then the boolean is \emph{true} else the boolean is \emph{false}. 

\begin{teXXX}
\tex_chardef:D \c_true_bool = 1 ~
\tex_chardef:D \c_false_bool = 0 ~
\end{teXXX}

\begin{teXXX}
 \cs_new_protected:Npn \bool_new:N #1 { \cs_new_eq:NN #1 \c_false_bool }
 \cs_generate_variant:Nn \bool_new:N { c }
\end{teXXX}

When a new boolean is constructed it is always set to false, as is evident from its code. 

Here is the formal syntax of the |\bool_new:N| function.

 \begin{docCommand}{bool_new:N}{\meta{boolean}}
   Creates a new \meta{boolean} or raises an error if the
   name is already taken. The declaration is global. The
   \meta{boolean} will initially be \texttt{false}. Once the boolean is created
   it can be set to logical true or false using \docAuxCommand*{bool_set_false:N} and \docAuxCommand*{bool_set_true:N}.
 \end{docCommand}
 
\begin{texexample}{Booleans}{}
\ExplSyntaxOn
\bool_new:N \mybool
\bool_set_false:N \mybool
\bool_if:NTF\mybool { \PASS } { \FAIL }
\ExplSyntaxOff
\end{texexample}
 

 
The real strength of the \latex~3 macros are the convenience of providing for |Or| and |And|
operations, negation etc.  and for its ability to evaluate fully boolean expressions. 

\begin{docCommand}{bool_if:nTF}{\marg{boolean expression} \marg{true code} \marg{false code}}
   Tests the current truth of \meta{boolean expression}, and
   continues expansion based on this result. The
   \meta{boolean expression} should consist of a series of predicates
   or boolean variables with the logical relationship between these
   defined using |&&| (\enquote{And}), \verb"||" (\enquote{Or}),
   |!| (\enquote{Not}) and parentheses. Minimal evaluation is used
   in the processing, so that once a result is defined there is
   not further expansion of the tests. 
\end{docCommand}   



\begin{texexample}{Booleans}{}
\ExplSyntaxOn
\bool_new:N\chapterfloat
\bool_new:N\numberfloat
\bool_set_false:N\chapterfloat
\bool_set_true:N\numberfloat

\bool_if:nTF {\chapterfloat || \numberfloat}  { \TRUE }{ \FALSE }

\bool_if:nTF {\chapterfloat && \numberfloat}  { \TRUE }{ \FALSE }

\ExplSyntaxOff
\end{texexample}

\subsection{\textbackslash if\_meaning}

The primitive |ifx| conditional has an equivalent in \latex3. This is called more semantically \docAuxCommand*{if_meaning:w}. This compares two tokens based on their meaning.



\begin{texexample}{Test ifx}{}
\ExplSyntaxOn
\group_begin:
  \cs_set_nopar:Npn \a: {BBB}
  \cs_set_nopar:Npn \b: {BBB~}
  \cs_set_nopar:Npn \c: {B~BB}
  
  \if_meaning:w \a:\b: \PASS \else: \FAIL \fi:
  \if_meaning:w \a:\c: \PASS \else: \FAIL \fi:
  
  \token_to_meaning:N \b:\\
  \token_to_meaning:N \a:  
\group_end:  
\ExplSyntaxOff
\end{texexample}

\begin{texexample}{LaTeX2e booleans}{}
\makeatletter
\ExplSyntaxOn
\if@mainmatter
     in~main~text
   \else
    not~in~main~text  
\fi    

 \meaning\@mainmattertrue\\
\bool_new:N \phd_mainmatter_bool 
\meaning\phd_mainmatter_bool
\ExplSyntaxOff
\makeatother  
\end{texexample}

\section{Predicate functions}

Predicate functions are one of the more powerful features of |expl3|. What are predicate functions? They are macros that test a predicate (\emph{true} or \meta{false}) and branch to either a true or false branch or just a single branch depending on the signature of the function. The |expl3| package has numerous such functions for example:

\begin{teXXX}
 \str_if_eq:nnT {}{}{}
\end{teXXX}

accepts two strings and if true does something. The expl3 package, provides a function that can generate such predicate functions fairly easily.

\begin{docCommand}{prg_set_conditional:Npnn}{\meta {function name}: \meta{arg spec} \meta{parameters} \marg{conditions code}}

These functions create a family of conditionals using the same \meta{code} to perform the
test created. Those conditionals are expandable if \meta{code} is. The new versions will
check for existing definitions and perform assignments globally (cf. |\cs_new:Npn|) whereas
the set versions do no check and perform assignments locally (cf. |\cs_set:Npn|). The
conditionals created are dependent on the comma-separated list of \meta{conditions}, which
should be one or more of p, T, F and TF.
\end{docCommand}

\begin{teXXX}
\prg_set_conditional:Npnn \cs_if_exist:N #1 { p , T , F , TF }
 {
 \if_meaning:w #1 \scan_stop:
   \prg_return_false:
     \else:
        \if_cs_exist:N #1
           \prg_return_true:
        \else:
          \prg_return_false:
      \fi:
 \fi:
}
2556 \prg_new_conditional:Npnn \token_if_eq_meaning:NN #1#2 { p , T , F , TF }
2557 {
2558 \if_meaning:w #1 #2
2559 \prg_return_true: \else: \prg_return_false: \fi:
2560 }

2201 \prg_new_conditional:Npnn \mode_if_math: { p , T , F , TF }
2202 { \if_mode_math: \prg_return_true: \else: \prg_return_false: \fi: }

\prg_new_conditional:Npnn \int_if_even:n #1 { p , T , F , TF}
3321 {
3322 \if_int_odd:w \__int_eval:w #1 \__int_eval_end:
3323 \prg_return_false:
3324 \else:
3325 \prg_return_true:
3326 \fi:
3327 }
\end{teXXX}

\begin{texexample}{isEven}{}
\ExplSyntaxOn
\prg_new_conditional:Npnn \isEven:n #1 { p, T, F, TF}
{
 \if_int_odd:w \__int_eval:w #1 \__int_eval_end:
    \prg_return_false:
 \else:
    \prg_return_true:
 \fi:
}

\isEven:nTF {2045679}{\PASS}{\FAIL}
\isEven:nTF {1000000}{\PASS}{\FAIL}
\ExplSyntaxOff
\end{texexample}

A common need for programmers is the testing of an integer or real for positiveness  with expl3 we can use predicate functions. In Example~\ref{ex:positive} we define predicate functions \docAuxCommand*{isPositive:nTF} to test an integer expression and feed the results to a true or false branch or according to the function signature. 

\begin{texexample}{isPositive} {ex:positive}
\ExplSyntaxOn
\prg_new_conditional:Npnn \isPositive:n #1 { p, T, F, TF}
{
\if_int_compare:w  \__int_eval:w #1 \__int_eval_end: >\__int_eval:w 0 \__int_eval_end:
     \prg_return_true:
\else:
    \prg_return_false:
\fi:          
}

\prg_new_conditional:Npnn \isNegative:n #1 { p, T, F, TF}
{
\if_int_compare:w  \__int_eval:w #1 \__int_eval_end: >\__int_eval:w 0 \__int_eval_end:
     \prg_return_false:
\else:
    \prg_return_true:
\fi:          
}
\cs_new:Npn \assert_is_positive:n #1 
   {
     \isPositive:nTF {#1} {\PASS #1} {\FAIL #1}
   }  
\cs_new:Npn \assert_is_negative:n #1 
   {
     \isNegative:nTF {#1} {\PASS #1} {\FAIL #1}
   } 
\assert_is_positive:n {2059+23-1245}
\assert_is_positive:n {-2059+23-1245}
\assert_is_negative:n {2059+23-1245}
\assert_is_negative:n {-2059+23-1245}
\ExplSyntaxOff
\end{texexample}

In the next example  we will use a common \tex trick to determine if a number is an integer or not. When \tex tries to convert a number to roman it will not scan past a minus sign .

\begin{texexample}{isInteger} {ex:isinteger}
\ExplSyntaxOn
\prg_set_conditional:Npnn \isInteger:n #1 { p, T, F, TF}
{
   \tl_if_blank:oTF {#1}{\prg_return_false:}
    {
     \tl_if_blank:oTF {  \__int_to_roman:w -\__int_eval:w #1 \__int_eval_end: }
		   {
		     \prg_return_true:
		   }
		   {
		     % not a number, but can be a negative number
		     \prg_return_false:
	         }
   }   
}

\cs_new:Npn \assert_is_integer:n #1 
   {
     \isInteger:nTF {#1} {\PASS\ ~~ #1} {\FAIL\ ~~ #1}\par
   }  
\assert_is_integer:n { }   
\assert_is_integer:n { 12}
\assert_is_integer:n {2059+1}
\assert_is_integer:n {-2059}
\assert_is_integer:n {2059}
%\assert_is_integer:n {ABC-1245}
\ExplSyntaxOff
\end{texexample}

The tests will pass provided even if we pass a  |numexpr|, but the assertion will fails if the number is negative. 
What we should have done was to test first if the head of the string was a (-) and then send it for further processing. 

\begin{texexample}{Testing the head of a string for the minus sign}{ex:string}
\ExplSyntaxOn
\cs_set:Npn \test:#1#2;{
   \str_if_eq:nnTF {-}{#1}{\PASS\par }{\FAIL\par }
   \str_if_eq:nnTF {-}{#1#2}{\PASS\par }{\FAIL\par }
}
\test:-;
\test:-12356;
\test:1234;
\ExplSyntaxOff
\end{texexample}

This passes all the comparison correctly, so we will have to re-write our function to test for the minus sign, before we send it to the main function. The reason I wrote the two tests above, is that a minus sign cannot be considered a number. 




\chapter{LaTeX3 String Manipulation and other Goodies}

 \TeX{} associates each character with a category code: as such, there is no
 concept of a \enquote{string} as commonly understood in many other
 programming languages. However, there are places where we wish to manipulate
 token lists while in some sense \enquote{ignoring} category codes: this is
 done by treating token lists as strings in a \TeX{} sense.

 A \TeX{} string (and thus an \pkg{expl3} string) is a series of characters
 which have category code $12$ (\enquote{other}) with the exception of
 space characters which have category code $10$ (\enquote{space}). Thus
 at a technical level, a \TeX{} string is a token list with the appropriate
 category codes. In this documentation, these will simply be referred to as
 strings: note that they can be stored in token lists as normal.

 The functions documented here take literal token lists,
 convert to strings and then carry out manipulations. Thus they may
 informally be described as \enquote{ignoring} category code. Note that
 the functions \docAuxCommand*{cs_to_str:N}, \docAuxCommand*{tl_to_str:n}, \docAuxCommand*{tl_to_str:N} and
 \docAuxCommand*{token_to_str:N} (and variants) will generate strings from the appropriate
 input: these are documented in \pkg{l3basics}, \pkg{l3tl} and \pkg{l3token},
 respectively.

 \section{The first character from a string}

 \begin{docCommand}{str_head:n}{\docAuxCommand*{str_head:n} \marg{token list}}
   Converts the \meta{token list} into a string, as described for
   \docAuxCommand*{tl_to_str:n}. The \docAuxCommand*{str_head:n} function then leaves
   the first character of this string in the input stream.
   The \docAuxCommand*{str_tail:n} function leaves all characters except
   the first in the input stream. The first character may be
   a space. If the \meta{token list} argument is entirely empty,
   nothing is left in the input stream.
 \end{docCommand}

\begin{texexample}{Strings}{ex:strings}
\ExplSyntaxOn
\DeclareDocumentCommand\asentence{ m }{
  \str_head:n {#1}\par}
  
\asentence{This is something}  

\str_head:n{\This~is~something}\par
\str_tail:n{\This~is~something}
\ExplSyntaxOff


\end{texexample}

 \subsection{Tests on strings}

The package provides some very powerful commands that can be used in string comparisons. Internally the comparisons are carried out using |\pdfstrcmp|. This has some complications in LuaTeX. 

 \begin{docCommand}{str_if_eq_x:nnTF}{\docAuxCommand*{str_if_eq_p:nn} \marg{tl1} \marg{tl2}}
%     \docAuxCommand*{str_if_eq:nnTF} \Arg{tl_1} \Arg{tl_2} \Arg{true code} \Arg{false code}
%   \end{syntax}
   Compares the two \meta{token lists} on a character by character
   basis, and is \texttt{true} if the two lists contain the same
   characters in the same order. Thus for example
   \begin{verbatim}
     \str_if_eq_p:no { abc } { \tl_to_str:n { abc } }
   \end{verbatim}
   is logically \texttt{true}.
\end{docCommand}


\begin{texexample}{String comparisons}{ex:test}
\ExplSyntaxOn
\let\abc\empty
\str_if_eq_x:nnTF{abc}{abc}{\TRUE}{\FALSE}\par
\str_if_eq_x:nnTF{\abc}{\abc}{\TRUE}{\FALSE}
\ExplSyntaxOff
\end{texexample}

 \section{String manipulation}

 \begin{docCommand}{str_lower_case:n}{\marg{tokens}}
%      \str_lower_case:n, \str_lower_case:f, 
%      \str_upper_case:n, \str_upper_case:f
%   }
%   \begin{syntax}
%     \docAuxCommand*{str_lower_case:n} \Arg{tokens}
%     \docAuxCommand*{str_upper_case:n} \Arg{tokens}
%   \end{syntax}
   Converts the input \meta{tokens} to their string representation, as
   described for \docAuxCommand*{tl_to_str:n}, and then to the lower or upper
   case representation using a one-to-one mapping as described by the
   Unicode Consortium file |UnicodeData.txt|.
   
   These functions are intended for case changing programmatic data in
   places where upper/lower case distinctions are meaningful. One example
   would be automatically generating a function name from user input where
   some case changing is needed. In this situation the input is programmatic,
   not textual, case does have meaning and a language-independent one-to-one
   mapping is appropriate. For example
%   \begin{verbatim}
%     \docAuxCommand*_new_protected:Npn \myfunc:nn #1#2
%       {
%         \docAuxCommand*_set_protected:cpn
%           {
%             user
%             \str_upper_case:f { \tl_head:n {#1} }
%             \str_lower_case:f { \tl_tail:n {#1} }
%           }
%           { #2 }
%       }
%   \end{verbatim}
%   would be used to generate a function with an auto-generated name consisting
%   of the upper case equivalent of the supplied name followed by the lower
%   case equivalent of the rest of the input.
%   
%   These functions should \emph{not} be used for
%   \begin{itemize}
%     \item Caseless comparisons: use \docAuxCommand*{str_fold_case:n} for this
%       situation (case folding is district from lower casing).
%     \item Case changing text for typesetting: see the \docAuxCommand*{tl_lower_case:n(n)},
%       \docAuxCommand*{tl_upper_case:n(n)} and \docAuxCommand*{tl_mixed_case:n(n)} functions which
%       correctly deal with context-dependence and other factors appropriate
%       to text case changing.
%   \end{itemize}
%
%   \begin{texnote}
%     As with all \pkg{expl3} functions, the input supported by
%     \docAuxCommand*{str_fold_case:n} is \emph{engine-native} characters which are or
%     interoperate with \textsc{utf-8}. As such, when used with \pdfTeX{}
%     \emph{only} the Latin alphabet characters A--Z will be case-folded
%     (\emph{i.e.}~the \textsc{ascii} range which coincides with
%     \textsc{utf-8}). Full \textsc{utf-8} support is available with both
%     \XeTeX{} and \LuaTeX{}, subject only to the fact that \XeTeX{} in
%     particular has issues with characters of code above hexadecimal
%     $0\mathrm{xFFF}$ when interacting with \docAuxCommand*{tl_to_str:n}.
%   \end{texnote}
 \end{docCommand}
 
 A common programming task is to convert strings to either uppercase or lowercase equivalents.v
 \begin{texexample}{Converting strings to lower and uppercase}{ex:cases}%TOFIX 
 \ExplSyntaxOn
    \tl_tail:n {TEST} 
   
      \cs_new_protected:Npn \myfunc:nn #1#2
       {
         \cs_set_protected:cpn
           {
             user
             \str_upper_case:f { \tl_head:n {#1} }
             \str_lower_case:f { \tl_tail:n {#1} }
           }
           { #2 }
       }
\docAuxCommand*_new_protected:cpn {yiannis}{Lazarides}
 \ExplSyntaxOff
 \end{texexample}



\chapter{Expansion and LaTeX3}

Expansion and variants are central to the concept of \latex3. The module| l3expan| provides generic methods for expanding \tex arguments in a systematic manner. The functions in the module all have prefix |exp|.

The module provides functions to produce \enquote{variants}. This is one of the most fundamental concepts of \latex3 and is good before we proceed further to recap on some of the \latex3 concepts.

\begin{description}
\item [naming conventions] The naming convention for command in \latex3 (expl3)  structures for command names is:

\textbackslash \meta{module}\textunderscore \meta{description}: \meta{arg-specifiers}

\textit{module} identifies the (main) type of data the function manipulates or use (for example, int (integers), prop (property lists), etc., or it might be the name of a package or some specific concept. 

\textit{description} says what is being done, e.g., |put_left|, |get|, |clear|, |count|, etc. If it makes sense the same descriptions are reused, but for special tasks there can, of course, be some that are used only once.

\textit{arg-specifiers} finally describe what arguments the function has and how they should be treated (more on this below).


\item[Base functions] \lorem 

\item[Variant functions]  Any command that uses one or more of these \emph{arg-specifiers} is called a \emph{variant} of the corresponding \emph{base function}. What these functions do is that they modify the argument one way or another and \textbf{only} then pass it to the underlying base function. For example:

\begin{teXXX}
\foo_bar:cVno {cmd} \VAR {text} \CMD
\end{teXXX}

would

\begin{itemize}
\item generate from the string |cmd| the command name \cs{cmd}
\item look up the value of the variable \cs{VAR}
\item leave text alone
\item expand \cs{CMD} once and surround the result with braces
\end{itemize}

It is important to stress that the variants do not produce aliases for the functions, they are also not overloading them. They just expand the base function arguments in a different way. 

\begin{teXXX}
 \foo_bar:Nnnn \cmd {<value-of-\VAR>} {text} {<one-level-expansion-of-\CMD>}
\end{teXXX}

Now ideally we want any possible variant of a base function automatically available for a programmer. Unfortunately, this can only be reliably done if all variants have all been predefined (as TeX doesn't offer you to trap the \enquote{undefined csname} error and do something on the fly).

Given the number of arg-specifier and the possible permutations predefining all variants, of which 90\% would never be needed, is not realistic. As Fank Mittelbach wrote on TEX.SX Q\&A site the \latex3 Team adopted the following strategy:

\begin{enumerate}
\item conceptually all variants are available and everybody can assume this is the case

\item in reality the kernel only defines a small subset that is often needed

\item any variant not defined by the kernel needs to be defined by the programmer using \docAuxCommand*{cs_generate_variant:Nn}

\item \docAuxCommand*{cs_generate_variant:Nn} has been designed in such a way that it doesn't matter if it is called several times: if the variant already exists it will do nothing. So if two programmers define the same variant in their packages it doesn't hurt, the first one executed will define the variant the second one will simply be ignored (with very little overhead).
\end{enumerate}

If some variants are used fairly often they may eventually get defined already in the kernel. Because of the last point it doesn't hurt if some packages still define the variant, i.e., there is no need for programmers to modify their packages in that case.

So in summary: Whenever you need a variant that is not predefined, define it at the beginning of your code. This is even sensible if you need the variant only once, because the code using the variant will be much more readable than any manual preprocessing of the argument and the speed difference is close to zero.

\item[The exp\_args:N.. functions]

Technicically speaking a variant defined via |\cs_generate_variant:Nn| has a very simple definition: |\foo_bar:cVno| above would simply expand to
\end{description}

\section{How to define variants}

The workhorse function used to define variants is:

\begin{docCommand}{cs_generate_variant:Nn} { \meta{parent control sequence} \marg{variant arg-spec}}

\end{docCommand}




 
\chapter{A more flexible and robust method of defining functions with LaTeX3 and xparse}
\label{ch:xparse}

The \LaTeX3 Team developed the package \pkg{xparse} to provide document level 
authors with some powerful commands that extend those such as \cs{newcommand}
of \latexe. The code is been stable and the interface is not expected to change. 
Although targetted at document level, the commands offered can be used effectively to produce code used in packages.\footnote{\protect\url{http://tex.stackexchange.com/questions/98152/always-use-newdocumentcommand-instead-of-newcommand}} The functions offered by the package enable commands with star, or optional arguments to be produced easily. 

\begin{docCommand}{DeclareDocumentCommand}{\marg{function}\marg{argument specification}\marg{code}}
This family of commands are used to create a document-level \emph{function}. The argument
specification for the function is given by \textit{arg spec}, and expanding to be replaced by the
\textit{code}. Unlike \latex's definition commands, all xparse commands take two arguments.
The first one is the \textit{argument specifier}, and the second is the \textit{code.}
\end{docCommand}

\begin{texexample}{DeclareDocumentCommand}{l3:1}
\DeclareDocumentCommand \foo { m o m } { 
    arg 1 = #1, arg 2 = #2,  arg 3 = #3 }
\foo{A}[B]{C}  

\foo{A}{B}    
\end{texexample}

In the example above |{m o m}| is the argument specifier. It tells the function  to expect, two mandatory arguments and one optional denoted by the letter \textbf{o}. There are many more specifiers. For example \textbf{O} takes an parameter as a default value.\index{argument specifier}

\begin{texexample}{DeclareDocumentCommand}{l3:1}
\DeclareDocumentCommand \foo { m O{\ldots} m } { 
    arg 1 = #1, arg 2 = #2,  arg 3 = #3 }
\foo{A}[B]{C}  

\foo{A}{B}    
\end{texexample}

The argument markers can be entered in any order. In the following example we will also add an optional argument in a curly bracket. Although this is frowned upon in certain contexts it is useful. Consider the case of a chapter title that also has a subtitle. \docAuxCommand*{Chapter}, it maybe more natural and useful to have input of the form, as shown in Example~\ref{l3:g}. 

\begin{texexample}{DeclareDocumentCommand}{l3:g}
\DeclareDocumentCommand \MyChapter { o m g } { 
\centering #2\par #3\par }
    
\MyChapter{THIS IS THE MAIN TITLE}{This is a subtitle}  

\MyChapter[]{THIS IS THE MAIN TITLE}{This is a subtitle}        
\end{texexample}

\begin{texexample}{DeclareDocumentCommand}{l4:g}
\DeclareDocumentCommand \MyChapter {s o m g } { 
\IfBooleanTF {#1} {\gdef\fonta{\bfseries\selectfont}}{\gdef\fonta{}}
\IfNoValueTF {#2} {No option\par}{#2}

\centering {\fonta #3}\par #4\par 
  }   
\MyChapter{ THIS IS THE MAIN TITLE}  

\MyChapter*[short title]{THIS IS THE MAIN TITLE}{This is a subtitle}        
\end{texexample}

As you can see, it is fairly easy to produce starred and unstarred versions of commands as well as as any form of optional arguments. Let us now see some of the other command definition functions, before we continue with other specifiers.

\begin{docCommand}{NewDocumentCommand}{\marg{function}\marg{argument specification}\marg{code}}
will issue an error if \meta{function} has already been defined
\end{docCommand}

\begin{docCommand}{RenewDocumentCommand}{\marg{function}\marg{argument specification}\marg{code}}
For changing a definition,
issuing an error message if the macro does
not already exist.
\end{docCommand}


As the \cmd{\DeclareDocumentCommand} always updates a definition, it is used for the examples in this chapter to avoid any errors.

What sets the above commands apart from \latexe \cmd{\newcommand} is the argument specification.



\begin{texexample}{DeclareDocumentCommand}{l3:1}
\DeclareDocumentCommand \teststar {s o m } { 
\IfBooleanTF {#1}
  { \typesetnormalchapter {#2} {#3} }
  { \typesetstarchapter {#3} }
}  
\newcommand\typesetnormalchapter[2][]{
  normal chapter
}
\newcommand\typesetstarchapter[1]{
  #1
}
\teststar{Test}

\teststar*{test}
\end{texexample}    

    
The argument specification \textbf{m o m} in the example enables the function to accept three arguments, two mandatory and one optional. 


\section{Argument specifications}

The basic idea of an argument specification is that each argument is listed as a single letter. 
As the argument specification is a mandatory argument, a function with no arguments still needs an arg spec.

\begin{texexample}{Empty arg spec}{}
\DeclareDocumentCommand\atest{}{some text}
\atest
\end{texexample}

Manadatory arguments are created using the letter \textbf{m}.

\begin{marglist}
\item [m] Mandatory. This is a standard mandatory argument, which can either be a single token alone or multiple tokens surrounded by curly braces. Regardless of the input, the argument will
be passed to the internal code surrounded by a brace pair. This is the \pkgname{xparse} type
specifier for a normal \tex argument.
\end{marglist}

\begin{texexample}{Mandatory Values, verbatim}{}
\DeclareDocumentCommand\testverbatim{ v }{
    \ttfamily#1
}
\testverbatim+ \this is a test +

\testverbatim * &^%$#\test *

\testverbatim{\ttfamily \bfseries\normalfont test}
\end{texexample}

The \textbf{l} specifier reads its argument, until it encounters a left brace. It is equivalent to \tex \# argument. Can be used basically for |\hbox| type comands.

\begin{texexample}{Mandatory arguments l-specifier}{}
\DeclareDocumentCommand\myhbox{ l }{
   \hbox to \dimexpr(#1)\relax
}
\fbox{\myhbox 12pt+1em+13ex  {test}}
\end{texexample}



\begin{marglist}
\item [o] Optional argument in  []. Returns |-NoValue-| if not present.
\item [O] As for \textbf{o}, but returns \meta{default}, if no value is given. Should be given as |O{default}|.

\item [s] Starred version
\item [v] Verbatim. Reads an argument “verbatim”, between the following character and its next occurrence,
in a way similar to the argument of the LATEX2" command \cmd{\verb}. Thus
a v-type argument is read between two matching tokens, which cannot be any of
\%, \#, \{, \}, \^ or  . The verbatim argument can also be enclosed between braces,
\{ and \}. A command with a verbatim argument will not work when it appears
within an argument of another function.
\item [l] An argument which reads everything up to the first open group token: in standard
\latex this is a left brace.
\item [u] Reads an argument “until \meta{tokens} are encountered, where the desired \meta{tokens}
are given as an argument to the specifier: |u|\meta{tokens}.
\item [d] An optional argument that is delimited. 
\item [D] As for d, but returns \meta{default} if no value is given: D\meta{token1} \meta{token2}\marg{default}.
Internally, the o, d and O types are short-cuts to an appropriated-constructed D
type argument.
\item [t]  An optional \meta{token}, which will result in a value \cs{BooleanTrue} if \meta{token} is 
            present and \cs{BooleanFalse} otherwise. Given as \meta{token}.
\item [g] An optional argument given inside a pair of \tex group tokens (in standard \latex,
              \{ . . . \}, which returns |-NoValue-| if not present.
\item [G] As  for \textbf{g} but returns \meta{default} if no value is given: |G|\marg{default}.
\end{marglist}

\begin{texexample}{Default Values}{}
\DeclareDocumentCommand\testcolor{ O{red} m }{
    \textcolor{#1}{#2}
}
\testcolor{This is typeset in red}
\testcolor[blue]{This is typeset in blue.}
\end{texexample}

\section{Testing special values}

The optional arguments of a function defined using |xparse| use dedicated variables to return
information about the naure of the argument received.

\begin{docCommand}{IfNoValueTF}{\marg{argument}\marg{true code}\marg{false code}}
The function tests if the argument has a value and executes the true of false code, by means
of a |-NoValue-| marker. 
\begin{texexample}{special values}{}
\DeclareDocumentCommand\doccmd{O{red} m}
    {
        \IfNoValueTF{#1}
            {\doccmdnocolor{#1}}
            {\doccmdcolor{\textcolor{#1}{#2}}}
     }
\newcommand\doccmdnocolor[1]{#1}
\newcommand\doccmdcolor[2]{#1 #2}     
This is \doccmd[blue]{text}  and this is \doccmd{text}.   
\end{texexample}
\end{docCommand}

\begin{texexample}{special values}{}
\DeclareDocumentEnvironment{allbold}{o}
    {
        \bfseries 
        \IfNoValueTF{#1}
            {\color{red}}
            {\color{#1}}
    }
    {                 }
\begin{allbold}[magenta]
\lorem
\end{allbold}
\end{texexample}

\begin{texexample}{variants}{ex:variants}
\ExplSyntaxOn
\cs_set:Npn \foo_something:Nn #1#2 {
   \csname\expandafter#1\endcsname{blue}{a a a} 
   { #2}
  }
\cs_generate_variant:Nn \foo_something:Nn { c }
%\meaning\foo_something:cn
\ExplSyntaxOff
\lorem


\end{texexample}



\parindent1em
\chapter{LaTeX3 counters and registers}

 \section{Introduction}
 
 This \latex3 module is dealing with integer arithmetic as well as utilizing these to create counter data type structures. It can be used to develop counters in a similar fashion to \latexe, although the user level functions are missing.  All control squences in this module are prefixed with |\int|. Some of the examples are prefixed as |phd_counter| and they emulate \latex’s counter commands. In this chapter also we describe a rather long example to typeset a table of numbers in various notations, such as roman, literal, arabic etc.
 
 \section{Integer expressions}

 Calculation and comparison of integer values can be carried out
 using literal numbers, \texttt{int} registers, constants and
 integers stored in token list variables. The standard operators
 \texttt{+}, \texttt{-}, \texttt{/} and \texttt{*} and
 parentheses can be used within such expressions to carry
 arithmetic operations. This module carries out these functions
 on \emph{integer expressions} (\enquote{\texttt{intexpr}}).

 \begin{docCommand}{int_eval:n} {\marg{integer expression}}
    Evaluates the \meta{integer expression}, expanding any
   integer and token list variables within the \meta{expression}
   to their content (without requiring \docAuxCommand*{int_use:N}/\docAuxCommand*{tl_use:N})
   and applying the standard mathematical rules. For example both
 \end{docCommand}
   
   \begin{verbatim}
     \int_eval:n { 5 +  4 * 3 - ( 3 + 4 * 5 ) }
   \end{verbatim}
   and
   \begin{verbatim}
     \tl_new:N  \l_my_tl
     \tl_set:Nn \l_my_tl { 5 }
     \int_new:N  \l_my_int
     \int_set:Nn \l_my_int { 4 }
    \int_eval:n { \l_my_tl +  \l_my_int * 3 - ( 3 + 4 * 5 ) }
   \end{verbatim}
   both evaluate to \( -6 \). The  \marg{integer expression} may
   contain the operators \texttt{+}, \texttt{-}, \texttt{*} and
   \texttt{/}, along with parenthesis \texttt{(} and \texttt{)}.
   Any functions within the expressions should expand to an
   \meta{integer denotation}: a sequence of a sign and digits matching
   the regex |\-?[0-9]+|).
   After expansion \docAuxCommand*{int_eval:n} yields an  \meta{integer denotation}
   which is left in the input stream.
 
  
     Exactly two expansions are needed to evaluate \docAuxCommand*{int eval:n}.
     The result is \emph{not} an \meta{internal integer}, and therefore
     requires suitable termination if used in a \TeX{}-style integer
     assignment.
   
 
  
  If you familiar with e-tex’s |numexpr|, this is equivalent code. 
 
  \begin{texexample}{Integer Evaluation}{ex:numexpr}
  \ExplSyntaxOn
   \int_eval:n { 5 +  4 * 3 - ( 3 + 4 * 5 )+2*2 }\par
   \int_to_arabic:n { ( 2+7 ) / 3 }\par
   \int_to_alph:n { 2+7 } \par
   \int_to_Alph:n { 6 * 2 }\par
   \int_to_roman:n { 9 } \par
   \int_to_Roman:n { 21 } \par
  \ExplSyntaxOff 
  \end{texexample} 
  

 
 \section{Creating and initializing integers} 
  
These are \latex’s equivalents of counters. Numerous commands are provided by the \latex3 kernel and these in my opinion offer a much better interface to lower level commands. A common question is that what one does if it is required to access a \latexe counter. The more-or-less “official” answer was provided by Joseph Wright at the Stack Exchange\footnote{\protect\url{http://tex.stackexchange.com/questions/167094/manipulate-a-latex2e-counter-with-latex3}} Q\&A site with the recommendation that: `mixing up the two interaces is asking for trouble, and while we are working on several areas we’ve not got a “user level” counter approach at yet. (Indeed, the entire question of how variables at the document-level should be handled is open.)’ 
  
  So you have it you keep on using commands such as \docAuxCommand*{setcounter} here is that |\c@..|. is an internal for LaTeX2e, and the entire point of expl3 is to have clear interfaces and internals. There's no reason to abuse the interfaces here (no functionality gain), so stick with them. In my opinion also it is not a good idea to mix |\@| notation with |expl3| notation. When there is a need to use both it is best to clearly separate the two and create an interface if they must somehow share information.

 

  Before we go further with the code is instructive to peek at the \latex3 kernel and understand what is an integer. An integer is simply a \tex \docAuxCommand*{newcount}, as shown from the code below.
  
  \begin{teXXX}
  \cs_new_protected:Npn \int_new:N #1
  {
    \__chk_if_free_cs:N #1(*@\label{allocation}@*)
    \cs:w newcount \cs_end: #1
  }
 \end{teXXX} 
  

   
 \begin{texexample}{allocations}{ex:counter allocations} 
 \ExplSyntaxOn
 \newcount\somecounter
 \meaning\somecounter\par
 
 \int_new:N\someothercounter
 \meaning\someothercounter\par
 
 \tl_map_inline:nn {
    \somecounter
    \someothercounter
  }
  { \cs_undefine:N #1 }
  
  \meaning\somecounter
 \ExplSyntaxOff
 \end{texexample}
 
 As you can observe from the example, using |\int_new:N| to create a counter is identical to the \latexe |\newcount|. It is instructive to keep this in mind later on and in your code, during debugging. Line~\ref{allocation} checks if the |\count| is available and then allocates the counter to the command sequence.
 Once we typeset the example, I have used |\cs_undefined| in a sequence to free the register. This is always good practice.  You must be careful not to get confused here with the terminology, we are dealing with \tex’s primitive |\count| registers.\footnote{To be more specific e-\tex.} Although the original \tex came only with 256 registers the new engines allow up to 65535 count registers. 
  
 \begin{docCommand}{int_new:N}{\meta{integer}}
   Creates a new \meta{integer} or raises an error if the name is
   already taken. The declaration is global. The \meta{integer} will
   initially be equal to $0$.
 \end{docCommand}
 
 In most instances counters involve a three step operation:
 
 \begin{enumerate}
 \item Creating the counter
 \item Adding values
 \item Typesetting the value or using it in another expression
 \end{enumerate}
 

 
 \begin{texexample}{Counters}{ex:l3counters}
 \ExplSyntaxOn
 \int_new:N \exercise
 \int_add:Nn \exercise {12+15}
 \int_to_roman:n \exercise \\
 
 \int_use:N \c_max_register_int
 \ExplSyntaxOff
 \end{texexample}

The \docAuxCommand*{int_use:N} is the \tex primitive \docAuxCommand*{the}. This is one of several \latex3 names of the primitive.


 \begin{docCommand}{int_const:Nn}{\meta{integer} \marg{integer expression}}
   Creates a new constant \meta{integer} or raises an error if the name
   is already taken. The value of the \meta{integer} will be set
   globally to the \meta{integer expression}.
 \end{docCommand}
 
The next commands can be used for round off, absolute functions etc.

 \begin{docCommand}{int_abs:n}{\marg{integer expression}}
   Evaluates the \meta{integer expression} as described for
   \docAuxCommand*{int_eval:n} and leaves the absolute value of the result in
   the input stream as an \meta{integer denotation} after two
   expansions.
 \end{docCommand}
 
 \begin{docCommand}{int_div_round:nn}{\marg{intexpr1} \marg{intexpr2}}
   Evaluates the two \meta{integer expressions} as described earlier,
   then divides the first value by the second, and rounds the result
   to the closest integer.  Ties are rounded away from zero.
   Note that this is identical to using
   |/| directly in an \meta{integer expression}. The result is left in
   the input stream as an \meta{integer denotation} after two expansions.
 \end{docCommand}
 

 \begin{docCommand}{int_div_truncate:nn}{ \marg{intexpr1} \marg{intexpr2}}
   Evaluates the two \meta{integer expressions} as described earlier,
   then divides the first value by the second, and rounds the result
   towards zero.  Note that division using |/|
   rounds the result. The result is left in the input stream as an
   \meta{integer denotation} after two expansions.
 \end{docCommand}
 
 \begin{texexample}{Truncating}{ex:truncate}
 \ExplSyntaxOn
 
 \int_div_round:nn  {10}{3}~~
 \int_div_truncate:nn  {13}{3}
 
 \ExplSyntaxOff
 \end{texexample}




\begin{docCommand}{int_max:nn}{ \marg{intexpr1} \marg{intexpr2}}
   Evaluates the \meta{integer expressions} as described for
   \docAuxCommand*{int_eval:n} and leaves either the larger or smaller value
   in the input stream as an \meta{integer denotation} after two
   expansions. The minimum of two numbers an be fund using |\int_min:nn|
\end{docCommand}

 \begin{texexample}{Finding minima and maxima}{ex:maxima}
 \ExplSyntaxOn
 
 \int_max:nn  {10}{3}~~
 \int_min:nn  {13}{3}
 
 \ExplSyntaxOff
 \end{texexample}


  \begin{docCommand}{int_mod:nn}{ \marg{intexpr1} \marg{intexpr2}}
   Evaluates the two \meta{integer expressions} as described earlier,
   then calculates the integer remainder of dividing the first
   expression by the second.  This is obtained by subtracting
   \docAuxCommand*{int_div_truncate:nn} \marg{intexpr1} \marg{intexpr2} times
   \meta{intexpr2} from \meta{intexpr1}.  Thus, the result has the
   same sign as \meta{intexpr1} and its absolute value is strictly
   less than that of \meta{intexpr2}.  The result is left in the input
   stream as an \meta{integer denotation} after two expansions.
   (See example~\ref{ex:mod}).
   
 \end{docCommand}
  
 \begin{texexample}{Modulus}{ex:mod}
 \ExplSyntaxOn
     \int_mod:nn  {10+13}{3+3}~~
 \ExplSyntaxOff
 \end{texexample}

\subsection{Setting and incrementing integers}

\begin{docCommand}{int_add:Nn}{\meta{integer} \marg{integer expression}}
   Adds the result of the \meta{integer expression} to the current
   content of the \meta{integer}.
 \end{docCommand}

\begin{docCommand}{int_incr:Nn}{\meta{integer} \marg{integer expression}}
 Increases the value stored in \meta{integer} by $1$.
 \end{docCommand}


\begin{docCommand}{int_decr:Nn}{\meta{integer} \marg{integer expression}}
  Decreases the value stored in \meta{integer} by $1$. 
 \end{docCommand}
 
 \begin{docCommand}{int_set:Nn}{ \meta{integer} \marg{integer expression}}
   Sets \meta{integer} to the value of \meta{integer expression},
   which must evaluate to an integer.
 \end{docCommand}
 
 \begin{docCommand}{int_sub:Nn} {\meta{integer} \marg{integer expression}}
   Subtracts the result of the \meta{integer expression} from the
   current content of the \meta{integer}.
 \end{docCommand}  

  \section{Using integers}
  
 Although we have already used \docAuxCommand*{int_use:N} to recover and typeset the value of a counter
 we now give its formal definition and an example of usage. As this is the primitive |\the| some care needs to 
 be  taken with expansion.
 

 \begin{docCommand}{int_use:N}{ \meta{integer}}
   Recovers the content of an \meta{integer} and places it directly
   in the input stream. An error will be raised if the variable does
   not exist or if it is invalid. Can be omitted in places where an
   \meta{integer} is required (such as in the first and third arguments
   of \docAuxCommand*{int_compare:nNnTF}).
 \end{docCommand}
 
 If we are to follow \latexe’s paradigm counters are names and not command sequences at the user level.
 With \latex3 of course we can define them as both. In Example~\ref{ex:intuse}, we use the |:c| variant of the commands to define a counter \meta{somecounter}, add globally an integer expression and then retrieve and typeset its value. 
 
 \begin{texexample}{More on retrieving values}{ex:intuse}
 \ExplSyntaxOn
   \int_new:c   {somecounter}
   \int_gadd:cn {somecounter} {(263+223)/23}
   \int_use:c   {somecounter}
 \ExplSyntaxOff
 \end{texexample}
 
\subsection{Longer example}

We wish to create a table that would list numbers and their literal equivalents.

 \begin{texexample}{Creating a numbered table}{ex:inc}
 \ExplSyntaxOn
 \int_gzero_new:N \phd_step_int
 \cs_new:Nn \g_phd_step_counter:n {
     \int_gincr:N\phd_step_int 
     \int_use:N \phd_step_int
 }
 
 \DeclareDocumentCommand\Inc{}{
    \g_phd_step_counter:n
 }
 \begin{tabular}{ll}
 \Inc  & One \\
 \Inc  & Two\\
 \Inc  & Three\\
 \end{tabular}
 
  \ExplSyntaxOff
 \end{texexample}
 
 With all these syntactic changes to the \latex code and conventions, perhaps we should retrace our steps to Knuth’s original terminology, Lamport’s structural documents concepts and a more simplistic language as expounded by my own philosophy in the phd package.

So what do we need, first we need to think that all these functions and programming are to produce documents, so our higher level macros should be document focused. 

The intention of the design layer is to provide interfaces that allow specifying layout and typography styles in a declarative way. On the implementation side there are a number of prototype implementations (most notably xtemplate and the recent reimplementation of the ldb). Those need to get unified into a common model which requires some more experimentation and probably also some further thoughts.

But the real importance of this layer is not the implementation of its interfaces but the conceptual view of it: provisioning a rich declarative method (or methods) to describe design and layout. I.e., enabling a designer to think not in programs but in visual representations and relationships.

 \begin{texexample}{Counter concepts}{ex:inc2}
 \makeatletter
 \ExplSyntaxOn

  \DeclareDocumentCommand\NewCounter{ m } {
     \int_gzero_new:c {#1}
     % create auxiliary functions
     \cs_set:Nn \g_phd_stepcounter:n {
         \int_gincr:c {#1} 
         \int_use:c {#1}
      }
  }
  
 \DeclareDocumentCommand\StepCounter{ m } { 
    \g_phd_stepcounter:n {#1} 
}  
  
 \DeclareDocumentCommand\SetCounter { m m } {
    \int_gset:cn {#1}{#2}
 }

    
\NewCounter{phd_temp_counter} 

 \DeclareDocumentCommand\Inc{}{
    \StepCounter{phd_temp_counter}
 }
 
 \SetCounter{phd_temp_counter}{12}
 
 \DeclareDocumentCommand\CounterToAlpha{ m }{
     \edef\x{\int_use:c{#1}}
     \int_to_Alph:n {\x}
}   
 \DeclareDocumentCommand\CounterToalpha{ m }{
     \edef\x{\int_use:c{#1}}
     \int_to_alph:n {\x}
}   

\DeclareDocumentCommand\CounterToRoman{ m }{
     \edef\x{\int_use:c{#1}}
     \int_to_Roman:n {\x}
}
\DeclareDocumentCommand\CounterToroman{ m }{
     \edef\x{\int_use:c{#1}}
     \int_to_roman:n {\x}
}
\DeclareDocumentCommand\IncA{}{
    \CounterToAlpha{phd_temp_counter}
}
\DeclareDocumentCommand\Inca{}{
    \CounterToalpha{phd_temp_counter}
}

\DeclareDocumentCommand\IncR{}{
    \CounterToRoman{phd_temp_counter}
}
\DeclareDocumentCommand\Incr{}{
    \CounterToroman{phd_temp_counter}
}
\DeclareDocumentCommand\IncW{}{
    \edef\x{\int_use:c{phd_temp_counter}}
    \expandafter\Words@cx{\x}
}
\DeclareDocumentCommand\Incw{}{
    \edef\x{\int_use:c{phd_temp_counter}}
    \expandafter\words@cx{\x}
}

 \begin{tabular}{c c c c c c c}
 \toprule
 Number & Literal & literal &Alpha &alpha &Roman &roman\\
 \midrule
 \Inc  & \IncW &\Incw &\IncA &\Inca &\IncR &\Incr\\
 \Inc  & \IncW &\Incw &\IncA &\Inca &\IncR &\Incr\\
 \Inc  & \IncW &\Incw &\IncA &\Inca &\IncR &\Incr\\
 \Inc  & \IncW &\Incw &\IncA &\Inca &\IncR &\Incr\\
 \bottomrule
 \end{tabular}
 
 \ExplSyntaxOff
 \makeatother
 \end{texexample}

What have just happened in our example, we have emulated most of \latexe counter macros in a kind of a different way, but there is an important part of the counter macros that is missing. This is the ability to reset counters when a master counter is changed. For example when a chapter counter is incremented the section counters in \latexe will reset and start counting from one again.



\docAuxCommand*{cl@foo} List of counters to be reset when foo stepped. This has a  format
\begin{verbatim}
   \@elt{countera}\@elt{counterb}\@elt{counterc}
\end{verbatim}

\textbf{Adding a prefix to the counters} So what we need to do first decide on some prefixes for counters or another similar convention. I will use a prefix |__counter_| to make the code readable. Although tempting to use |c@|  to make our counters compatible with \latexe counters, as stated earlier this would be against the central philosophy of \latex3 of keeping a onsistent syntax and function aiming conventions. All we have to do is add the prefixes and create the new counter to hold the rest counters and provide helper commands. We can also do some error capturing. 


\begin{phdverbatim}
\DeclareDocumentCommand\NewCounter{ o m } {
% create new counter if it does not exist and also its resets counters
  \int_if_exist:NTF {__counter_#2}
    {
      \msg_error:nn { counters } { counter exists and cannot be created }
    }
    { 
      \int_gzero_new:c {__counter_#2}
      \int_gzero_new:c {__counter_resets_#2}
    }
 
% handle the reset   
  \IfNoValueF{#1}
    {
      \int_if_exist:NTF {__counter_resets_#1} 
                        {add to reset} {false code}
    }    
% create auxiliary functions  to be added later   
       
  }
\end{phdverbatim}
    
This is a rough outline of the code we need to develop. It is considered bad practice to mix too many low level commands with high level commands and we should replace these with auxiliary functions. The auxiliary functions will create automated functions such as |\thechapter| in \latexe.

\begin{teXXX}
\cs_new:Nn \phd_create_new_counter:n {
    \int_if_exist:NTF {__counter_#2}{error}
         { 
             \int_gzero_new:c {__counter_#2}
             \int_gzero_new:c {__counter_resets_#2}
         }
}
\end{teXXX}

We continue our code development, this time we will add a prefix before the counter name. The code is shown
in Example~\ref{ex:createcounters}



\begin{texexample}{Auxiliary constructor function}{ex:createcounters}
\ExplSyntaxOn
\makeatletter
\cs_gset:Nx  \counter_prefix: {c@}
\cs_gset:Nx  \counter_resets_prefix: {__counter_resets_prefix_}

\cs_gset:Nn  \phd_create_new_counter:n {
    \int_if_exist:cTF {\counter_prefix:#1}{ERROR~counter~exists}
        { 
            \int_gzero_new:c {\counter_prefix:#1}
            \seq_new:c {\counter_resets_prefix:#1}
        }
}

\phd_create_new_counter:n {test}

\phd_create_new_counter:n {test2}

\cs_gset:Nn\stepacounter:n {
  \int_gincr:c{\counter_prefix: #1}
}

\cs_gset:Nn\setacounter:cn {
  \int_set:cn {\counter_prefix: #1}{#2}
}

\cs_gset:Nn  \countervalue:n {
    \the\cs:w\counter_prefix: #1\cs_end:\relax
}

\def\makeauxiliaries#1 {\mycommandaux#1\relax}

\def\mycommandaux#1#2\relax{%
       \uppercase{
       \expandafter\gdef\csname #1}
       #2Value
       \endcsname
       {\the\cs:w\counter_prefix: #1#2\cs_end:\relax}
}

\makeauxiliaries {test}

\setacounter:cn {test}{18}

\countervalue:n {test}\par


\makeauxiliaries {section}
first~test~\TestValue\par 
\stepacounter:n {test}
\stepacounter:n {test}

second~test~\TestValue\par 

section~counter~first~test:~\SectionValue\par
section~counter~with~\docAuxCommand*{thesection}:~\thesection\par
\stepacounter:n {section}
section~counter~second~test:~\SectionValue
\ExplSyntaxOff
\makeatother
\end{texexample}



\ExplSyntaxOn
\newcommand{\makeauxiliaryfunctions}[1]{\mycommandaux#1\relax}
\def\mycommandaux#1#2\relax{%
       \uppercase{\expandafter\gdef\cs:w #1}#2Value\cs_end:
       {\tex_the:D\cs:w\counter_prefix: #1#2\cs_end:\relax}%
    }
\ExplSyntaxOff
    
The example above creates a function |\phd_create_new_counter:n| and then tests it. The function uses a conditional to test if the counter exists  and then if it has not been defined earlier it creates the two counters and sets them to zero. If it exists it will typeset an error message. This is of course so that we can view the error in the document, normally this would display an error on the screen. I have not covered the messaging part of the code so far for displaying errors and warnings. These are created with the \pkgname{l3msg} package, which is bundled with |expl3|. We will cover this later on in the book. 

\textbf{Adding the counter to the reset list of another} We now go on to develop a function to add to the reset list of another.

\begin{texexample}{Adding to the reset}{ex:addtoreset}
\ExplSyntaxOn
\cs_gset:Nn \addtoreset:nn {
    \exp_args:Nf\seq_put_left:cn {\counter_resets_prefix:#1}{#2}
    Added~ to~the~ #1 ~ resets~ #2.~The~resets~list~is~now~\seq_use:cn {\counter_resets_prefix:#1}{,}
 }

\ExplSyntaxOff
\end{texexample}

The |\stepcounter:c| will be used to step a counter and to reset all subsidiary counters. 


\begin{texexample}{Adding to the reset}{ex:addtoreset}
\ExplSyntaxOn
\makeatletter
\cs_gset:Nn \resetcounter:c 
  {
    \int_gset:cn {\counter_prefix: #1}{0}
  }
  
\cs_gset:Nn \stepcounter:c {
  \int_gincr:c {\counter_prefix: #1}
  \seq_set_eq:Nc \tempa {__counter_resets_prefix_#1}
  \seq_map_inline:Nn \tempa {\resetcounter:c{##1}}
}      

\stepcounter:c {test}

% Test that the value is captured
\int_use:c {\counter_prefix: test}

\makeatother
\ExplSyntaxOff
\end{texexample}

Now that we have developed the code for |\addtoreset:nn| we are ready to modify and finalize our counter creation macro to the following:

\begin{texexample}{Refactor creation macro}{ex:refactor}
\ExplSyntaxOn
 \DeclareDocumentCommand \NewCounter{ o m } {
    \phd_create_new_counter:n {#2}
    \IfNoValueF{#1}
      {
         \int_if_exist:cT {\counter_resets_prefix:#1} 
             {\addtoreset:nn{#1}{#2}} 
      }    
    \makeauxiliaryfunctions {#1}
 }
%\NewCounter{Chapter}
%\NewCounter[Chapter]{Section}\par
%\NewCounter[Chapter]{Figure}
%\stepcounter:c{Chapter}
%\int_use:c {\counter_prefix: Chapter} 
%\int_use:c {\counter_prefix: Section}
%\countervalue:n{Chapter}
%\ChapterValue
\ExplSyntaxOff
\end{texexample}

%% Rewrite as the examples are in a group and they cannot leak out

\ExplSyntaxOn
\NewDocumentCommand\NewCounter{ o m } {
  \phd_create_new_counter:n {#2}
    \IfNoValueF{#1}
      {
        \int_if_exist:cT {\counter_resets_prefix:#1} 
             {\addtoreset:nn{#1}{#2}} 
      }    
  \makeauxiliaryfunctions {#1}
}
\ExplSyntaxOff

All we have to now do is to write some tests. The \latex3 Team provide testing routines with the tests being run using a Lua script. In our case we can run all our tests within the documentation here. These tests are shown in Example~\ref{ex:countertests}. 


\begin{texexample}{Counter module tests }{ex:countertests}
\ExplSyntaxOn
\NewCounter{Chapter}
\NewCounter[Chapter]{Section}\par
\NewCounter[Chapter]{Figure}\par
\NewCounter[Chapter]{Problems}\par
\stepcounter:c{Chapter}
\stepcounter:c{Chapter}
Value~ of~ Chapte~ counter:~ \int_use:c {\counter_prefix: Chapter}\par 
Value~of~Section~ counter:~\int_use:c {\counter_prefix: Section}\par
Value~of~Chapter~ counter~ using \docAuxCommand*{countervalue:n}:~ \countervalue:n{Chapter}\par
Value~of~Chapter~ counter~ using \docAuxCommand*{ChapterValue}:~ \ChapterValue\par
\ExplSyntaxOff
\end{texexample}


\begin{docCommand}{ChapterCounter}{}
   Typesets the Chapter counter. This is equivalent to |\thechapter|. Decorating the actual counter value, should
   all be based on a key-value system and the author has no access to this function. It is the template designer’s 
   job to define it.
   
  \begin{verbatim}
   chapter numbering = arabic
   chapter font-size = huge
  \end{verbatim} 
  
\end{docCommand}

\begin{docCommand}{ChapterCounterValue}{}
   Typesets the Chapter counter in arabic. This form can also be used in other expressions This is equivalent to \latexe’s \docAuxCommand*{c@chapter}, but not equal, i.e, there is no interface to \latexe counters.
\end{docCommand}
  
  
 \subsection{Integer conditionals}

Comparing the values of two counters can be achieved with the use of conditional expressions. There are numerous commands provided for this purpose and we outline some of the most important ones. Do consult the manual to view others. The first one we will examine is \docAuxCommand*{int_compare:nNnTF}. This function evaluates each of two expressions and branches to the true or false code. 

\begin{docCommand}{int_compare:nNnTF}{\marg{intexpr1} \meta{relation} \marg{intexpr2}
\marg{true code} \marg{false code}}
   This function first evaluates each of the \meta{integer expressions}
   as described for \docAuxCommand*{int_eval:n}. The two results are then
   compared using the \meta{relation}:
   \begin{center}
     \begin{tabular}{ll}
       Equal                 & |=| \\
       Greater than      & |>| \\
       Less than           & |<| \\
     \end{tabular}
   \end{center}
 \end{docCommand}
 
 Consider two counters \docAuxCommand*{counteri} and \docAuxCommand*{counterii} that we need to compare their values and branch to either false or true code.
 
 \begin{texexample}{Integer conditionals}{ex:intconditionals}
 \ExplSyntaxOn
 \int_new:N  \counteri
 \int_new:N  \counterii
 
 \int_compare:nNnTF {\counteri} = {\counterii}
     {true~code\par}{false~ code\par}
     
 \int_gadd:Nn \counterii {15+12}    
 
  \int_compare:nNnTF {\counteri} = {\counterii}
     {true~code~\par}{false~ code\par}
     
\ifnum\counteri<\counterii primitive~ifnum~true\else primitive~false\fi\par
   
 \ExplSyntaxOff
 \end{texexample}
 
 A common error is to include the \meta{relation} code in curly brackets. This leads to errors during parsing. 
 
   
 
 
  \begin{docCommand}{int_case:nnTF} {\marg{test integer expression}
      \{ 
     \marg{intexpr case1} \marg{code case1} 
     \marg{intexpr case2} \marg{code case2} 
     \ldots 
     \marg{intexpr casen} \marg{code casen} 
     \} 
     \marg{true code}
     \marg{false code}}
     
   This function evaluates the \meta{test integer expression} and
   compares this in turn to each of the
   \meta{integer expression cases}. If the two are equal then the
   associated \meta{code} is left in the input stream. If any of the
   cases are matched, the \meta{true code} is also inserted into the
   input stream (after the code for the appropriate case), while if none
   match then the \meta{false code} is inserted. The function
   \docAuxCommand*{int_case:nn}, which does nothing if there is no match, is also
   available. For example
\end{docCommand}   
   \begin{texexample}{Case example}{ex:case}
   \makeatletter
   \ExplSyntaxOn
     \int_case:nnF
       { 2 * 5 }
       {
         { 5 }       { Small }
         { 4 + 6 }   { Medium }
         { -2 * 10 } { Negative }
       }
       { No idea! }
       
   
  \cs_set:Npn \@fnsymbol #1
   {
    \int_case:nnF {#1}
     {
      {0} {}
      {1} { \TextOrMath \textasteriskcentered* }
      {2} { \TextOrMath \textdagger\dagger }
      {3} { \TextOrMath \textdaggerdbl\ddagger }
      {4} { \TextOrMath \textsection\mathsection }
      {5} { \TextOrMath \textparagraph\mathparagraph }
      {6} { \TextOrMath \textbardbl\| }
      {7} { \TextOrMath {\textasteriskcentered\textasteriskcentered}{**} }
      {8} { \TextOrMath {\textdagger\textdagger}{\dagger\dagger} }
      {9} { \TextOrMath {\textdaggerdbl\textdaggerdbl}{\ddagger\ddagger} }
     }
     { \@ctrerr }
   }
   
   
   \@fnsymbol {3}
     \ExplSyntaxOff  
     \makeatother
    \end{texexample}
    
 
 
 The next case example is from \href{https://github.com/wspr/fontspec/blob/master/fontspec.dtx}{fontspec.dtx}. It just wraps the official \latexe definition from \pkgname{fixltx2e} into the |expl| language. If you are going to utilize code from \latexe verbatim, it is always best to use it as is and only using a wrapper.

 \begin{texexample}{Case example}{ex:case2}
   \makeatletter
   \ExplSyntaxOn
   
  \cs_set:Npn \@fnsymbol #1
   {
    \int_case:nnF {#1}
     {
      {0} {}
      {1} { \TextOrMath \textasteriskcentered* }
      {2} { \TextOrMath \textdagger\dagger }
      {3} { \TextOrMath \textdaggerdbl\ddagger }
      {4} { \TextOrMath \textsection\mathsection }
      {5} { \TextOrMath \textparagraph\mathparagraph }
      {6} { \TextOrMath \textbardbl\| }
      {7} { \TextOrMath {\textasteriskcentered\textasteriskcentered}{**} }
      {8} { \TextOrMath {\textdagger\textdagger}{\dagger\dagger} }
      {9} { \TextOrMath {\textdaggerdbl\textdaggerdbl}{\ddagger\ddagger} }
     }
     { \@ctrerr }
 }
 
   
 \@fnsymbol {3}
 \ExplSyntaxOff  
 \makeatother
 \end{texexample}
    

%
% \begin{function}[updated = 2013-01-13, EXP, pTF]{\int_compare:n}
%   \begin{syntax} 
%     \docAuxCommand*{int_compare_p:n} \\
%     ~~\{ \\
%     ~~~~\meta{intexpr_1} \meta{relation_1} \\
%     ~~~~\ldots{} \\
%     ~~~~\meta{intexpr_N} \meta{relation_N} \\
%     ~~~~\meta{intexpr_{N+1}} \\
%     ~~\} \\
%     \docAuxCommand*{int_compare:nTF}
%     ~~\{ \\
%     ~~~~\meta{intexpr_1} \meta{relation_1} \\
%     ~~~~\ldots{} \\
%     ~~~~\meta{intexpr_N} \meta{relation_N} \\
%     ~~~~\meta{intexpr_{N+1}} \\
%     ~~\} \\
%     ~~\Arg{true code} \Arg{false code}
%   \end{syntax}
%   This function evaluates the \meta{integer expressions} as described
%   for \docAuxCommand*{int_eval:n} and compares consecutive result using the
%   corresponding \meta{relation}, namely it compares \meta{intexpr_1}
%   and \meta{intexpr_2} using the \meta{relation_1}, then
%   \meta{intexpr_2} and \meta{intexpr_3} using the \meta{relation_2},
%   until finally comparing \meta{intexpr_N} and \meta{intexpr_{N+1}}
%   using the \meta{relation_N}.  The test yields \texttt{true} if all
%   comparisons are \texttt{true}.  Each \meta{integer expression} is
%   evaluated only once, and the evaluation is lazy, in the sense that
%   if one comparison is \texttt{false}, then no other \meta{integer
%     expression} is evaluated and no other comparison is performed.
%   The \meta{relations} can be any of the following:
%   \begin{center}
%     \begin{tabular}{ll}
%       Equal                    & |=| or |==| \\
%       Greater than or equal to & |>=|        \\
%       Greater than             & |>|         \\
%       Less than or equal to    & |<=|        \\
%       Less than                & |<|         \\
%       Not equal                & |!=|        \\
%     \end{tabular}
%   \end{center}
% \end{function}
%

%
% \begin{function}[EXP,pTF]{\int_if_even:n, \int_if_odd:n}
%   \begin{syntax}
%     \docAuxCommand*{int_if_odd_p:n} \Arg{integer expression}
%     \docAuxCommand*{int_if_odd:nTF} \Arg{integer expression}
%     ~~\Arg{true code} \Arg{false code}
%   \end{syntax}
%   This function first evaluates the \meta{integer expression}
%   as described for \docAuxCommand*{int_eval:n}. It then evaluates if this
%   is odd or even, as appropriate.
% \end{function}
 \subsection{Integer expression loops}
 
 Integer expression loops, bring \latex nearer to the functionality of other computer languages. In Example~\ref{ex:dowhile} a \emph{do}\ldots \emph{while} loop is constructed to print all even numbers from |0..16|. Many variations to looping structures are also provided and these are discussed after the example.
 
 \begin{texexample}{Integer Expression loops}{ex:dowhile}
 \ExplSyntaxOn
 \int_new:N \l_tempa_int
 \int_zero:N \l_tempa_int 
 \int_do_while:nn {\l_tempa_int <= 10 + 6 } {
     \int_use:N \l_tempa_int,~ 
     \int_add:Nn \l_tempa_int {2}
 }
 \ExplSyntaxOff
 \end{texexample}
 
 In the next example we will use \refCom{int_step_function} and call a function at each iteration. 
 
\begin{texexample}{Step function} {ex:stepfunction}
\ExplSyntaxOn
\cs_set:Npn \my_func:n #1 {test~#1}
\int_step_function:nnnN {1} {1} {5} {\my_func:n}
\ExplSyntaxOff
\end{texexample}
 
\section{Summing up} 

This has been a long chapter, and we both deserve some coffee and a break. We have discussed the creation of integer expressions, their use as counters and typesetting commands for counters. We have also examined in depth conditionals associated with integers and also some looping structures that are very robust. I have spent more time than I expected on this module, as it wraps up a lot of the concepts we have been discussing in other chapters and I thought a thorough review and some longer examples would be beneficial. 

By now, if you have been running the examples on your own, you should be more or less start thinking in
\latex3 speak. It takes a while for the syntax and the concepts to sink in. From my own experience, you need to spend at least 2-3 weeks just programming in the |expl| language and you should avoid the temptation to use \latexe macros or \tex primitives. Easier said that done.
 

 
\cxset{section align=left,
       section font-weight=bold}
       
\chapter{The LaTeX3 l3msg Module and how to use it for Error, Warning and other Messages}
\index{messaging>error}\index{messaging>warning}
\section{Introduction}

Messages need to be passed to the user by modules, either when errors occur or to indicate
how the code is proceeding. The l3msg module provides a consistent method for doing
this (as opposed to writing directly to the terminal or log).

The system used by l3msg to create messages divides the process into two distinct
parts. Named messages are created in the first part of the process; at this stage, no
decision is made about the type of output that the message will produce. The second
part of the process is actually producing a message. At this stage a choice of message
class has to be made, for example error, warning or info.

By separating out the creation and use of messages, several benefits are available.
First, the messages can be altered later without needing details of where they are used
in the code. This makes it possible to alter the language used, the detail level and so
on. Secondly, the output which results from a given message can be altered. This can be
done on a message class, module or message name basis. In this way, message behaviour
can be altered and messages can be entirely suppressed.

\section{Creating messages}

Messages \emph{must} be created before they can be used. This has the advantage that they can be used
over and over and also one could start thinking of internationalizing the package.

Messages can be created as either new or set and there is also a TF to check if the message exists:

\begin{docCommand}{msg_new:nnnn}{ \marg{module} \marg{message} \marg{text} \marg{more text} }
Creates a hmessagei for a given hmodulei. The message will be defined to first give htexti
and then \meta{more text} if the user requests it. If no \meta{more text} is available then a standard
text is given instead. Within htexti and more text four parameters (\#1 to \#4) can be
used: these will be supplied at the time the message is used. An error will be raised if
the \meta{message} already exists.
\end{docCommand}

\section{Messaging classes}

Messages are divided into categories termed message classes. 
These are according to \emph{severity}, \emph{fatal}, \emph{critical}, \emph{error}, \emph{warning} and \emph{info}. Each one has its own set of creation functions.

\begin{docCommand*}{msg_fatal:nnnnnn}{\marg{module} \marg{message} \marg{arg one} \marg{arg two} \marg{arg three}
\meta{arg four}}
Issues \meta{module} error \meta{message}, passing \meta{arg one} to \meta{arg four} to the text-creating
functions. After issuing a fatal error the \tex run will halt.
\end{docCommand*}



\begin{texexample}{Typical package error setup}{ex:errors}
% Error message example
%
% simulate LaTeX2e \fmversion
\def\fmversion{2000/11/12}
\makeatletter
\ExplSyntaxOn 

% create error boolean
\bool_new:N \l_mypackage_error_bool
 
% redirect package errors here  %(*@\label{l:warning}@*)
\cs_new_protected:Npn \mypackage_warning:nxx #1 #2 #3 
  {
    \bool_set_true:N \l_mypackage_error_bool
    \msg_info:nnxx { mypackage } { #1 } { #2 } { #3 }
  }
 
% define some error messages
\msg_set:nnnn { mypackage } { old-version }
  { LaTeX~source~files~more~than~5~year~old.~ Is~dated~(year:#1~date:#2-#3) }
  { Please~update~your~distribution~visit~ctan } %(*@\label{test}@*)
   

% check version number   	   	
\cs_new:Npn \mypackage_check_version:n #1 
  {
    \exp_after:wN \l_mypackage_check_version_aux:w #1\q_stop
  }

% check version number auxiliary 
% #1 relation
% #2 true code
% #3 false code  
\cs_new:Npn \l_mypackage_check_version_aux:w #1/#2/#3\q_stop 
  {
    \int_compare:nNnTF { ( \tex_year:D-#1 )*12 + (\tex_month:D-#2) } > { 65 }
      { \FAIL \mypackage_warning:nxx { old-version } { #1 } { #1 / #2 / #3 } }
      { \PASS } 
  }
  
\mypackage_check_version:n \fmversion 
\mypackage_check_version:n \fmtversion 
  
\ExplSyntaxOff
\end{texexample}

In Example~\ref{ex:errors} Line \ref{test} we define our own package command for issuing an error message, rather than typing |\msg_error:nnxx| directly. This is considered good practice and we avoid typing in \enquote{mypackage} all the time. 

\section{How to translate strings}
\index{internationalization>expl3}

I posted similar code to the |TX.SX| Q\&A site to elicit comments from other users, as to recommended best practices.
%http://tex.stackexchange.com/questions/246810/latex3-l3msg-best-practices 
At this moment in time I am not too sure if a LaTeX3 approach to internationalization is appropriate. If one looks at the complexities, it is preferable to use Lua. 






\chapter{Expl3 File Operations}
\label{ch:l3files}


 
\tex provides only some basic primitive control sequences for dealing with files. \tex is also limited to 16 input streams and 16 output streams making it considerably difficult to manipulate too many files. In most of the examples here we have used, streams allocated by the \latexe kernel for temporary operations and hence we can re-use them, but with extreme care. 

Checking for the existence of a file is simple and we can use the |\file_if_exist:nTF| function. 

\section{Creating streams}

Before you can use a file in \tex you need to allocate a stream. With \latex3 you can use \docAuxCommand*{ior_new:N} or \docAuxCommand*{iow_new:N} depending if is a stream for write or read. All I/O operations are global: streams are declared with global names and treated accordingly.   

As one can run out of handles very quickly, the \pkgname{phd} package loads the \pkgname{morewrites} and also patches the \pkgname{filecontents} package to enable us to use it. This was renamed to |phdfilecontents|. In Example~\ref{ex:createstreams} 

\begin{texexample}{File streams}{ex:createstreams}
\ExplSyntaxOn
\iow_new:N \scratch_filea
\iow_new:N \scratch_fileb
\iow_new:N \scratch_filec
\iow_new:N \scratch_filed
\ExplSyntaxOff
\end{texexample}

Another way used widely by \latexe and package authors is to use \latexe's |\@inputcheck| file handle. This is a file handle used by |filecontents| as well as internally by the \latexe kernel for  and for building the |\IfFileExists| control sequence and hence its name. 


\begin{texexample}{LaTeXe examples}{ex:inputcheck}
\makeatletter
\bgroup
\ttfamily \meaning\@inputcheck\\
\number\@inputcheck %
\egroup
\makeatother
\end{texexample}

Stream management goes back to \tex and Plain\tex which use an allocation meachanism to assign the the names to numbers and then . This mechanism was then moved onto \latex2.  \latex3 uses a slightly different mechanism but the basic logic is still the same. With \latex3 the allocations are kept in a property list. 

\section{File Operations}
\begin{texexample}{File operations}{ex:fileops}
\ExplSyntaxOn
% Check if a file exists 
  \file_if_exist:nTF { filetest.txt } { \PASS } { \FAIL }
% Check for another one
  \file_if_exist:nTF { filetest }     { \PASS } { \FAIL }
  \g_file_current_name_tl
\ExplSyntaxOff
\end{texexample}

\subsection{Handling paths}

The |l3files| module provides a mechanism to add paths to the search path used to search for 
a file. This is pretty much similar to |\graphicspath|.

\begin{docCommand*}{file_path_include:n} {\marg{path}}
Adds \meta{path} to the list of those used to search when reading files. The assignment is local.
The \meta{path} is processed in the same way as a \meta{file name}, i.e., with x-type expansion
except active characters. Spaces are not allowed in the \meta{path}.
\end{docCommand*}

In Example~\ref{ex:corpora1} we add to the search path the directory, where our corpora data is residing. Then we check to see if the file |female.txt| exist. This is a large text file (extension |.txt|), containing common female names
in the US. We will use this file later on for some for examples.

\begin{texexample}{File operations}{ex:corpora1}
\ExplSyntaxOn
% Add path
\file_path_include:n {./corpora/}
\file_if_exist:nTF {female.txt} {\PASS}{\FAIL}
\ExplSyntaxOff
\end{texexample}
\index{path}\index{file operations>path}

\subsection{Loading files}

Loading a full file can of course be achieved with the |\input| command. Under the experimental section of the |\expl3| there is also a function that can load a file on condition that it exists. 

\begin{docCommand} {file_if_exist_input:n} { \marg{file name} }
Searches for \meta{file name} using the current TEX search path and the additional paths
controlled by |\file_path_include:n|). If found, inserts the \meta{true code} then reads in
the file as additional \latex source as described for |\file_input:n|. Note that 
|\file_if_exist_input:n| does not raise an error if the file is not found, in contrast to |\file-input:n|.
\end{docCommand}

\begin{texexample}{Loading a file only if it exists}{}
\ExplSyntaxOn
\file_if_exist_input:n {filetest.txt}
\ExplSyntaxOff
\end{texexample}

Since the file is loaded within the expl3 block, it will cause all spaces to be removed from the output, we can overcome this by following the |expl3| design pattern of declaring a function with |xparse| and calling it outside the block.

\begin{texexample}{Loading a file only if it exists}{}
\ExplSyntaxOn
\DeclareDocumentCommand \CorporaInput{ m }
  {
    \file_if_exist_input:n { #1 }
  }
\ExplSyntaxOff
\CorporaInput {filetest.txt}  
\end{texexample}

As is usual with |expl3| functions the |nTF| signature form of the |\file_if_exist_input:| is also available. 

\begin{texexample}{Loading a file only if it exists}{}
\ExplSyntaxOn
\DeclareDocumentCommand \CorporaInput{ m }
  {
     \file_if_exist_input:nTF {#1}
  }
\ExplSyntaxOff
\CorporaInput {filetest.txt}{ \PASS } { \FAIL }
\CorporaInput { filetest.txt }{ \PASS } { \FAIL }
\end{texexample}

The above code fails if we leave spaces between the \{\verb*+ filetest.txt +\}, we can easily remove them using the  |>{ \TrimSpaces }| argument processor.

\begin{texexample}{Loading a file only if it exists}{ex:trimspaces}
\ExplSyntaxOn
\DeclareDocumentCommand \CorporaInput{  >{  \TrimSpaces } m }
  {
     \file_if_exist_input:nTF {#1}
  }
\ExplSyntaxOff
\CorporaInput {filetest.txt}{ \PASS } { \FAIL }
\CorporaInput { filetest.txt }{ \PASS } { \FAIL }
\makeatletter
%\@ifpackageloaded{test}{\PASS test loaded}{\FAIL}
\makeatother
\end{texexample}

In the next example, we will try and consolidate some of the skills we have been developing so far.
In the \pkgname{phd} package, we are loading over 70 packages through the package manager. We wanted
to automatically keep track of which packages we loaded and which we did not (as they might
not have been in our distribution). \latexe provides a useful macro \docAuxCommand{@ifpackageloaded}
that can check if the package has been loaded or not. 

\begin{texexample}{Loading a Package}{ex:loadpackage}
\ExplSyntaxOn

\clist_new:N \g_packages_loaded_clist
\clist_new:N \g_packages_failed_clists
\clist_new:N \g_packages_loaded_by_others_clist

\makeatletter
%\DeclareDocumentCommand \requirepackage{  >{  \TrimSpaces } m m m }
%  {
%     \file_if_exist:nTF {#1.sty} 
%       { 
%         \@ifpackageloaded{#1} 
%              {
%                   \clist_put_left:Nn \g_packages_loaded_by_others_clist  {#1} 
%               }
%               {
%                   \clist_put_left:Nn \g_packages_loaded_clist  {#1}  
%                 #2
%              }
%        } 
%        { 
%          \clist_put_left:Nn \g_packages_failed_clist  {#1}
%          #3 
%        }
%  }
 

%\@ifpackageloaded{xcolor}{true}{false}
%\@ifpackageloaded{lettrine}{\PASS}{\FAIL}
%\@ifpackagewith{ragged2e}{}{\PASS}{\FAIL}
%\@ifpackagewith{soul}{}{\PASS}{\FAIL}
%\@ifpackagewith{siunitx}{fixed-exponent,scientific-notation}{\PASS}{\FAIL siunitx}
% \makeatother
% 
%\par
%\requirepackage {xcolor}    {\PASS } { \FAIL }
%\requirepackage {soul}      {\PASS } { \FAIL }
%\requirepackage {calligra}  {\PASS } { \FAIL }
%\requirepackage {hyperref}  {\PASS } { \FAIL }
%\requirepackage {layouts}   {\PASS } { \FAIL }
%\par
%
%\clist_map_inline:Nn \g_packages_loaded_by_others_clist 
%  {
%    loaded~by~others~\ldots~ #1,~
%  }
\makeatother
\ExplSyntaxOff
\end{texexample}



\section{Reading and writing to streams}

Typical file operations are reading, writing and appending. Common file management operations are creating, deleting, opening, closing, copying and renaming.

\begin{texexample}{File operations}{ex:fileops}
\edef\someheading{Another test}
\ExplSyntaxOn
\iow_open:Nn \tempstream { filetest.txt }
\iow_now:Nx \tempstream {\someheading}
\iow_close:N \tempstream
\let\getfile\file_input:n
%\file_input:n {filetest.txt}
\ExplSyntaxOff
\getfile {filetest.txt}
%\getfile{./corpora/female.txt}
\end{texexample}   

\section{Input-output streams}

Reading one line at a time from a file, uses the \tex primitive |\read|. One important item to watch is that there are different commands for read and write you need to use the |\io|\textcolor{thered}{\texttt{r}}, rather than |iow_|

Streams  are precious in \tex  as we only have 16 available , so when reading from a file, when we use \LaTeXe, we can use some of \LaTeX build-in streams, so we will be using |\@inputcheck|

\begin{texexample}{File operations}{ex:fileops}
\ExplSyntaxOn

\makeatletter
\global\let\ltx_scratch_stream \@inputcheck
\makeatother
\ior_open:Nn \ltx_scratch_stream {male-a.txt}
\ior_get:NN \ltx_scratch_stream \l_tmpa_tl
\tl_use:N \l_tmpa_tl\par
\ior_get:NN \ltx_scratch_stream \l_tmpa_tl
\tl_use:N \l_tmpa_tl
\ior_close:N \ltx_scratch_stream

\ExplSyntaxOff
\end{texexample}   


Reading line by line is not very useful. What we need is the ability to read all he lines
recursively until the end of the file. 

\begin{texexample}{File operations}{ex:fileops}
\ExplSyntaxOn
% add the file path to the name
\file_path_include:n {./corpora/}

% open the stream
\ior_open:Nn \ltx_scratch_stream {male-a.txt}

% define a macro so we can do recursion
\cs_set:Npn \read_loop {
  \if_eof:w \ltx_scratch_stream
    \ior_close:N \ltx_scratch_stream
    \let\next\relax
 \else:
   \ior_get:NN \ltx_scratch_stream \tmpa
   \tl_use:N \tmpa,~
   \let\next\read_loop
 \fi:      
 \next 
}

% read the file and typeset the words with a comma
\read_loop
\ExplSyntaxOff
\RaggedRight
\end{texexample}   

In the following example we do some more changes to the example. This time instead of typesetting the names,
we will add them to a clist. We will also read a different file that contains an alphabetical list of male names. Then we will check if the name Zacharias is in the list. 

\begin{texexample}{File operations}{ex:fileops}
\ExplSyntaxOn
% \clist
\clist_new:N \males
% open the stream
\file_path_include:n {./corpora/}
\ior_open:Nn \ltx_scratch_stream {male.txt}

% define a macro so we can do recursion
\cs_set:Npn \read_loop {
  \ior_if_eof:NTF \ltx_scratch_stream
    {
      \ior_close:N \ltx_scratch_stream
      \cs_set_eq:NN \next\relax
    }
    {  
      \ior_get:NN \ltx_scratch_stream \tmpa
      \clist_put_right:Nx \males {\tl_use:N \tmpa}
      \cs_set_eq:NN \next\read_loop
   } 
 \next 
}

% read the file and parse the words as a clist \males
\read_loop

% check if Zacharias or Mary are in the list
\clist_if_in:NnTF\males {Zacharias} {\PASS} {\FAIL}
\clist_if_in:NnTF\males {Mary} {\PASS} {\FAIL}
\ExplSyntaxOff
\end{texexample}   

\section{Appending to a file}

To append to a file, first we need to read the contents of the file into a |tl_var| then apend the material and close the file. We then open it again in write mode and write the contents of the |tl_var|. Let us try it out.


    
\section{Writing to the log or aux files}    

There are some constant input-output streams. There is a somewhat different programming philosophy here are these are normally via messages and not directly as shown in the example here.  These are handled in the chapter dealing with messages.

\begin{texexample}{Writing to log and terminal}{ex:log}
\ExplSyntaxOn
\iow_term:x {Something}
\ExplSyntaxOff
\end{texexample}
      
Armed with all these it maybe time to review again our database functions that we have created in the earlier chapter on clists.

                
          
            
              
                
                  
                      




\chapter{LaTeX3 Key value system}
\label{l3:keys}
The key-value system has been discussed earlier but avoided to cover the |l3keys| module of \latex3 until such time as the basics of the expl3 syntax was discussed. 


The l3keys modules provides general purpose keyval processing for expl3 code. However, it does not interact with LaTeX2e's package or class option system. For that, you need to load some additional code, which is available in the package l3keys2e. This provides the \docAuxCommand*{ProcessKeysOptionscommand} to parse class/package options and process them using keyvals defined by l3keys.

The reason for this separation is that l3keys is intended to form part of a LaTeX3 kernel, while l3keys2e is tied to the LaTeX2e model for processing options. It seems extremely likely that a stand-alone LaTeX3 kernel will use keyval options 'natively' but with a different underlying implementation.

 The high level functions here are intended as a method to create
 key--value controls. Keys are themselves created using a key--value
 interface, minimising the number of functions and arguments
 required. Each key is created by setting one or more \emph{properties}
 of the key:
 \begin{verbatim}
   \keys_define:nn { mymodule }
     {
       key-one .code:n   = code including parameter #1,
       key-two .tl_set:N = \l_mymodule_store_tl
     }
 \end{verbatim}
 
  At a document level, |\keys_set:nn| will be used within a
 document function, for example
 \begin{verbatim}
   \DeclareDocumentCommand \MyModuleSetup { m }
     { \keys_set:nn { mymodule } { #1 }  }
   \DeclareDocumentCommand \MyModuleMacro { o m }
     {
       \group_begin:
         \keys_set:nn { mymodule } { #1 }
        ... Main code for the macro
       \group_end:
     }
 \end{verbatim}
 
 The process of incorporating a key value system into a macro or a package involves three steps. First the keys are defined then processed to set them to some values and lastly incorporated into a function or package.
 
 It is best to illustrate the process with a small example. Example\ref{ex:keyval1} defines two keys that affect the typesetting of paragraphs |parindent| and |parskip|. These are defined using the |.code|, pretty much the same way that |pgfkeys| that we discussed earlier defines keys. 
 
 \begin{texexample}{Key value}{ex:keyval1}
 \ExplSyntaxOn
 \keys_define:nn {scratch}
   {
      parindent .code:n = \parindent#1,
      parskip     .code:n = \parskip#1
   }
   
\DeclareDocumentCommand \MyModuleSetup { m }
     { \keys_set:nn { scratch } { #1 }  }
     
\DeclareDocumentCommand \MyModuleMacro { o }
     {
       \group_begin:
         \keys_set:nn { scratch } { #1 }
         % Main code for \MyModuleMacro
         \lorem\par
         \lorem\par
       \group_end:
     }
 \ExplSyntaxOff   
 \MyModuleSetup{parindent=1em, parskip=1pt}
 \MyModuleMacro [parindent=10pt, parskip=10pt]
 \end{texexample}
 
 
 The definition of the keys was achieved using the command:
 
\begin{docCommand}{keys_define:nn}{\marg{module}\marg{keyval list}}
The command parses the \meta{keyval list} and defines the keys associated there for \meta{module}. 
\end{docCommand}

The \meta{keyval list} should consist of one or more key names along with an associated
key \emph{property}. The properties of a key determine how it acts. The individual properties
are described in the following text; Note that the properties of the key begin from the dot (|.|) after the key name. The various properties available take no arguments or require one or more. All key definitions are local. 
 
 \begin{margoptionslist}
 \item [ .code:n] Stores the \meta{code} for execution when \meta{key} is used. 
 \item [.default:n] \meta{key} |.default:n| = \meta{default} This creates a \meta{default} value for \meta{key} if no value is given. This will be used if only the key name is given, but not if a blank \meta{value} is given. This behaviour is similar to the |pgfkeys| package.
 \item [.initial:n] \meta {key} |.initial:n| = \meta{value} Initialises the \meta{key} with the \meta{value}, equivalent to
|\keys_set:nn| \meta{module} \meta{key} = \meta{value}
 
 \item [.dim_set:N] \meta{key} |.dim_set:N| = \meta{dimension} Defines \meta{key} to set \meta{dimension} to \meta{value} (which must a dimension expression). If the variable does not exist, it will be created globally at the point that the key is set up.
 \end{margoptionslist}
 
%  \begin{texexample}{Key value}{ex:keyval1}
% \ExplSyntaxOn
% \dim_new:N \l_parskip
% \dim_new:N \l_parindent
% \keys_define:nn {scratch}
%   {
%      parindent .dim_set:N = \l_parindent,
%     % parindent .initial:n = 0pt,
%      parskip     .dim:n = \l_parskip,
%      %parskip     .initial:n = 1pt,
%      
%   }
%   
%\DeclareDocumentCommand \MyModuleSetup { m }
%     { \keys_set:nn { scratch } { #1 }  }
%     
%\DeclareDocumentCommand \MyModuleMacro { o }
%     {
%       \group_begin:
%         \dim_set_eq:NN \parindent \l_parindent
%         \dim_set_equal:NN \parskip \l_parskip
%         \keys_set:nn { scratch } { #1 }
%         % Main code for \MyModuleMacro
%         \lorem\par
%         \lorem\par
%       \group_end:
%     }
% \ExplSyntaxOff   
% 
% \MyModuleMacro [parindent=10pt, parskip=10pt]
% \end{texexample}

 
 \subsection{Choice keys}
 
 One of the most powerful features of modern key value packages is the ability to define and set keys for mutally exclusive values. In the |l3keys| module this can be achieved using the choice key.
 
 \begin{margoptionslist}
 \item [.choice] \meta{key} |.choice| This sets \meta{key} to act as a choice key. Each choice is then created, as discussed below:
 \end{margoptionslist}
 
 
 \begin{texexample}{Some choices}{}
 \ExplSyntaxOn
 \keys_define:nn { scratchi }
 {
    mycolor .choice:,
    mycolor/fire .code:n = {\color{red}},
    mycolor/sky .code:n = {\color{blue}},
    mycolor/orange .code:n = {\color{orange}},
    mycolor/lemon .code:n = {\color{yellow}},
    mycolor/grass .code:n = {\color{green}},
    mycolor .initial:n =sky,
    mycolor .default:n=orange,
    unknown .code:n={\color{red} ERROR},
 }

\keys_set:nn { scratchi } { mycolor=fire }  

\DeclareDocumentCommand \MyModuleSetup { m }
     { \keys_set:nn { scratchi } { #1 }  }

\DeclareDocumentCommand \MyModuleMacro { o +m}
     {
       \group_begin:
         \keys_set:nn { scratchi } { #1 }
         #2
         \group_end:
     }
     
 \ExplSyntaxOff
    
 \MyModuleSetup{mycolor=fire}

 \MyModuleMacro [mycolor=grass]{grass,} 
 
 \MyModuleMacro [mycolor]{default}
 
 \MyModuleMacro [apple]{}
 
 \MyModuleMacro [fire]{Fire}
\end{texexample} 
 
The |.choice|  key is a bit different from how it is used in the |xtemplate| package and |pgf| but probably easier to use and define. Of course our example was trivial and the colors should have been achieved with just one code key, capturing the value. It takes some practice to get used to all the types of keys available and to develop error free code easily, but by using a key value system, truly flexible, modern functions can be developed.
 

\subsection{Handling of unknown keys}
 
 Handling of unknown keys is similar to pgf where a key defined as |.unknown| is defined. 
 If a key has not previously been defined (is unknown), |\keys_set:nn| will look for a special
unknown key for the same module, and if this is not defined raises an error indicating that
the key name was unknown. This mechanism can be used for example to issue custom
error texts.

\begin{verbatim}
\keys_define:nn { mymodule }
{
unknown .code:n =
You~tried~to~set~key~’\l_keys_key_tl’~to~’#1’.
}
\end{verbatim}
 
 
 As for |pgf| there are many other key types and these are listed in the |l3keys| manual and are not listed here for brevity. 
 
 
\chapter{Sequence lists}

\epigraph{``Where did we get that (equation) from? Nowhere. It is not possible to derive it from anything you know. It came out of the mind of Schrödinger.''}{---Richard Feynman}

One very useful data type, which is incorporated in \latex3 is the ``sequence'' data type. This contains a list of entries which may contain any \meta{balanced text}. One of the most powerful features of lists is that t is possible to map functions to sequences such that the function is applied to every item in the sequence.

Sequences are also used to implement stack functions in \latex3. This is achieved using a number of dedicated stack functions.

\section{Creating sequences}

Like most of the modules new sequences are created using the prefix for the module and the word ``new''.

\begin{docCommand}{seq_new:N}{}
Creates a new \meta{sequence} or raises an error if the name is already taken. The declaration
is global. The \meta{sequence} will initially contain no items.
\end{docCommand}

First let us create and examine the meaning of a simple example of the use of sequences. 

\begin{texexample}{Creating sequences}{ex:seq}
\ExplSyntaxOn
\seq_new:N \g_scratch_seq 
\token_to_meaning:N \g_scratch_seq
\ExplSyntaxOff
\end{texexample}

Examining the meaning of the sequence we created with \refCom{seq_new:N} we observe that there is no magic involved, it is just another macro that holds two others. So let us add some material and see what happens next.

\begin{texexample}{Creating sequences}{ex:seq}
\ExplSyntaxOn
\gdef\tempa {AAA}
\gdef\tempb {BBB}

% Add some material left and right
\seq_gput_left:Nn \g_scratch_seq \tempa
\seq_gput_right:Nn \g_scratch_seq \tempb

% examine the meaning of the \scratch_seq:N
% and the marker at the beginning
\token_to_meaning:N \g_scratch_seq\\
\token_to_meaning:N \s__seq\\
\token_to_meaning:N \s__seq_item:n
\ExplSyntaxOff
\end{texexample}

We have used two more functions that by now you are familiar to put material both at the left and at the right of the sequence, and again examined its meaning. We also examined the meaning of |s__seq| which is the internal command at the start of the list. The concept is very similar to |\@elt| lists

\begin{texexample}{Creating sequences}{ex:seq}
\ExplSyntaxOn
% examine the meaning of the \g_scratch_seq:N
% and the marker at the beginning again
\token_to_meaning:N \g_scratch_seq\\
\token_to_meaning:N \s__seq\\
\token_to_meaning:N \s__seq_item:n\\

% pop the left of the sequence
% store in \l_tmpa
\seq_pop_left:NN \g_scratch_seq \l_tmpa_tl

% typeset contents of left cell
\tl_use:N \l_tmpa_tl

\ExplSyntaxOff
\end{texexample}

\begin{texexample}{Creating sequences}{ex:seq}
\ExplSyntaxOn
\def\urlctan   {\url{\http:ctan.org}}
\def\urlgithub {\url{http:github.org}}

% clearing the sequence
\seq_clear:N \g_scratch_seq 

\seq_gput_left:Nn \g_scratch_seq \urlctan
\seq_gput_left:Nn \g_scratch_seq \urlgithub
% typeset contents of left cell
% pop the left of the sequence
% store in \l_tmpa
\seq_gpop_left:NN \g_scratch_seq \l_tmpa_tl

% typeset contents of left cell
\tl_use:N \l_tmpa_tl

\ExplSyntaxOff
\end{texexample}

\begin{texexample}{An equation database}{ex:seq}
 % #1 name
 % #2 equation
 
\ExplSyntaxOn   
\cs_gset:Npn \addEquation #1#2{
  \expandafter\gdef\csname-#1\endcsname {{\bfseries#1}\begin{gather}#2\end{gather}}
  \seq_gput_right:Nn \g_scratch_seq {#1}
 }

\cs_gset:Npn \typesetEquations 
  {
    \seq_map_inline:Nn \g_scratch_seq 
      {
        \cs:w-##1\cs_end:
      }
  }
  
% clearing the sequence
\seq_clear:N \g_scratch_seq 

\ExplSyntaxOff

\addEquation {quadratic} 
  {
    ax^2 + bx + c =0
  }
\addEquation {linear}    
  {
    x = \frac{b}{a}
  }
\addEquation {cubic}    
  {
    x^3 + 2x^2 + 10x = 20
  }
    
\typesetEquations

\end{texexample}

By the way Leonardo de Pisa, also known as Fibonacci (1170–1250), was able to find the positive solution to the cubic equation \( x^3 + 2x^2 + 10x = 20\), using the Babylonian numerals. He gave the result as \(1,22,7,42,33,4,40\) (equivalent to \(1 + 22/60 + 7/602 + 42/603 + 33/604 + 4/605 + 40/606)\), which differs from the correct value by only about three trillionths.

But let us fill our little database with the quartic, quintic, sextic and septic functions,
so we can have a few more data in our sequence. Also I suggest you try and run some of the examples on your own to get used to the language, solving syntax errors for typos and the like.

\tcbset{texexmp/.style={ 
        colback = white,% background
        colframe=white, 
        %bottombox=ignored,   
        listing options={
          backgroundcolor=\color{white},
          keywordstyle=\color{black},
          breaklines=true,
          breakatwhitespace=true,
          commentstyle=\color{thelightgray},
          emph={addEquation, typesetEquations},
          emphstyle=\color{thegreen},
         },
        }}%
\begin{texexample}[texexmp]{An equation database}{ex:seq}
% add equations to db
\addEquation {quartic} 
  {
    f(x)=ax^4+bx^3+cx^2+dx+e
  }
  
\addEquation {quintic}    
  {
    g(x)=ax^5+bx^4+cx^3+dx^2+ex+f
  }
  
\addEquation {sextic}    
  {
    ax^6+bx^5+cx^4+dx^3+ex^2+fx+g=0
  }

\addEquation {septic}    
  {
    ax^7+bx^6+cx^5+dx^4+ex^3+fx^2+gx+h=0
  } 
  
\addEquation {BBGKY}
  {\scriptstyle  
   \frac{\partial f_s}{\partial t} + \sum_{i=1}^s \dot{\mathbf{q}}_i \frac{\partial f_s}{\partial \mathbf{q}_i} + \sum_{i=1}^s \left( - \frac{\partial \Phi_i^{ext}}{\partial \mathbf{q}_i} - \sum_{j=1}^s \frac{\partial \Phi_{ij}}{\partial \mathbf{q}_i} \right) \frac{\partial f_s}{\partial \mathbf{p}_i} = (N-s) \sum_{i=1}^s \frac{\partial}{\partial \mathbf{p}_i} \int \frac{\partial \Phi_{is+1}}{\partial \mathbf{q}_i}\cdot f_{s+1} \,d\mathbf{q}_{s+1} d\mathbf{p}_{s+1}.
  }
  
 \addEquation {Borda-Carnot}
  {
    \Delta E\, =\, \frac12\, \rho\, \left( v_3\, -\, v_2 \right)^2\,
           =\, \frac12\, \rho\, \left( \frac{1}{\mu}\, -\, 1 \right)^2\, v_2^2\,
           =\, \frac12\, \rho\, \left( \frac{1}{\mu}\, -\, 1 \right)^2\, \left( \frac{A_1}{A_2} \right)^2\, v_1^2.} %(*@\label{borda}@*)

% typeset all equations in db                      
\typesetEquations 
\end{texexample}

If you observe the last example we have hit a small problem, we had to reduce the size of the display
equation to fit it in. We would have been better to have displayed the equation in a \refEnv{multline}
or a brqew environment, as shown below. 

\begin{multline}
\frac{\partial f_s}{\partial t} + \sum_{i=1}^s \dot{\mathbf{q}}_i \frac{\partial f_s}{\partial \mathbf{q}_i} + \sum_{i=1}^s \left( - \frac{\partial \Phi_i^{ext}}{\partial \mathbf{q}_i} - \sum_{j=1}^s \frac{\partial \Phi_{ij}}{\partial \mathbf{q}_i} \right) \frac{\partial f_s}{\partial \mathbf{p}_i} =\\
 (N-s) \sum_{i=1}^s \frac{\partial}{\partial \mathbf{p}_i} \int \frac{\partial \Phi_{is+1}}{\partial \mathbf{q}_i}\cdot f_{s+1} \,d\mathbf{q}_{s+1} d\mathbf{p}_{s+1}.
\end{multline}

Recall how we defined |\addEquation|,

\begin{teXXX}
\cs_gset:Npn \addEquation #1#2{
  \expandafter\gdef\csname-#1\endcsname 
    {
      {\bfseries#1}\begin{gather}#2\end{gather}
    }
  \seq_gput_right:Nn \g_scratch_seq {#1}
 }
\end{teXXX}

We can change the function to accept an optional argument and a starred or unstarred version to allow the user to add a field to the input that can determine the output. This is fairly easy with \pkgname{xparse}.


\begin{texexample}{An equation database}{ex:seq}
\addEquation {quartic} 
  {
    f(x)=ax^4+bx^3+cx^2+dx+e
  }
\addEquation {quintic}    
  {
    g(x)=ax^5+bx^4+cx^3+dx^2+ex+f
  }
\addEquation {sextic}    
  {
    ax^6+bx^5+cx^4+dx^3+ex^2+fx+g=0
  }

\addEquation {septic}    
  {
    ax^7+bx^6+cx^5+dx^4+ex^3+fx^2+gx+h=0
  } 
\addEquation {BBGKY}
  {\scriptstyle  
   \frac{\partial f_s}{\partial t} + \sum_{i=1}^s \dot{\mathbf{q}}_i \frac{\partial f_s}{\partial \mathbf{q}_i} + \sum_{i=1}^s \left( - \frac{\partial \Phi_i^{ext}}{\partial \mathbf{q}_i} - \sum_{j=1}^s \frac{\partial \Phi_{ij}}{\partial \mathbf{q}_i} \right) \frac{\partial f_s}{\partial \mathbf{p}_i} = (N-s) \sum_{i=1}^s \frac{\partial}{\partial \mathbf{p}_i} \int \frac{\partial \Phi_{is+1}}{\partial \mathbf{q}_i}\cdot f_{s+1} \,d\mathbf{q}_{s+1} d\mathbf{p}_{s+1}.
  }
 \addEquation {Borda-Carnot}
  {
    \Delta E\, =\, \frac12\, \rho\, \left( v_3\, -\, v_2 \right)^2\,
           =\, \frac12\, \rho\, \left( \frac{1}{\mu}\, -\, 1 \right)^2\, v_2^2\,
           =\, \frac12\, \rho\, \left( \frac{1}{\mu}\, -\, 1 \right)^2\, \left( \frac{A_1}{A_2} \right)^2\, v_1^2.}
\typesetEquations 
\end{texexample}



\begin{texexample}{Sequence}{ex:sequence}
\ExplSyntaxOn
\def\exception{}
\NewDocumentCommand\SplitDemo { +m m } 
  {
    \my_seq_split:nn { #1 }{#2}
  }

\tl_new:N \l_first_word_tl

\cs_new_protected:Npn \my_seq_split:nn #1 #2
  { 
    
    \seq_set_split:Nnn \l_tmpa_seq { #2 } { #1 }
    \seq_use:Nn   \l_tmpa_seq {\par}
    \seq_get_left:NN \l_tmpa_seq \l_first_word_tl
    %\textcolor{blue} { \tl_use:N \l_first_word_tl  }
  }
      
\ExplSyntaxOff

\SplitDemo { This is one sentence. 
             This is a second one. 
             This is the third sentence. }{ . }\par
\SplitDemo { The \exception{A.B.C.} corporation. Another sentence. }{ . }
\SplitDemo { The \exception{A.B.C.} corporation. Another sentence. }{~}
\end{texexample}



Here we ha

Consider words that are not normally capitalized in a title.

\begin{texexample}{Sequence}{ex:sequence}
\ExplSyntaxOn
\clist_gset:Nn \title_words_not_capitalized_en 
 { 
 a,an,the,at,by,for,in,of,on,to,up,and,as,but,it,or, 
 nor,do,for,this,be,A,An,The,At,By,For,In,Of,On,To,Up, 
 And,As,But,It,Or,Nor,Do,For,This,Be, 
 abaft,aboard,about,above,absent,across,afore,after,against,along,
 alongside,amid,amidst,among,amongst,an,anenst,apropos,apud,around,
 as,aside,astride,at,athwart,atop,barring,before,behind,below,beneath,
 beside,besides,between,beyond,but,by,circa,concerning,despite,down,
 during,except,excluding,failing,following,for,forenenst,from,given,in,
 including,inside,into,lest,like,mid,midst,minus,modulo,near,next,
 notwithstanding,of,off,on,onto,opposite,out,outside,over,pace,past,
 per,plus,pro,qua,regarding,round,sans,save,since,than,through,
 throughout,till,times,to,toward,towards,under,underneath,unlike,
 until,unto,up,upon,Versus,versus,via,vice,with,within,without,worth
}

\clist_gset:Nn \abbreviations 
 {
  A.B.C.,iTunes
 }

\clist_gset:Nn \acronyms
  {
    NATO,UN,US
  }  						
    
\cs_new:Npn \ucfirst_aux:w #1#2 \q_stop { \tl_upper_case:n { #1 } #2 }

\cs_new:Npn \ucfirst #1 
	{
		\exp_after:wN \ucfirst_aux:w #1 \q_stop
	}

\cs_new:Npn \lowerfirst #1 
	{
		\tl_lower_case:n {#1}
 	}

\NewDocumentCommand\UppercaseTitle {s +m }
	{
		\IfBooleanTF { #1 } { {\bfseries {#2} } }
    		{     
       	\tex_hyphenpenalty:D = 10000
	       \seq_set_split:Nnn \g_tmpa_seq {~} {#2}
	       \seq_use:Nn   \g_tmpa_seq {~}\\
	        
	       \seq_pop_left:NN \g_tmpa_seq \l_tmpa_tl  
	       
	       {\bfseries\ucfirst \l_tmpa_tl \space} 
	       \seq_map_inline:Nn \g_tmpa_seq 
	       	{
	         	\clist_if_in:NnTF \title_words_not_capitalized_en { ##1 }
	           { {\bfseries \lowerfirst {##1}~}} { {\bfseries \ucfirst{##1}~ } }    
	            
	         }            
       }    
	} 
	
\ExplSyntaxOff    

 \UppercaseTitle {Top ten things to do in Paris}\\
 \UppercaseTitle {How to use {\LaTeX} sequence lists effectively}\\
 \UppercaseTitle {Senate Votes to Confirm Elena Kagan For U.S. Supreme Court}\\
 \UppercaseTitle {what would be a ``correct'' capitalization for the title of this question?}\\
 \UppercaseTitle* {How about {$e=mc^2$}? }\\
\end{texexample}
\ExplSyntaxOn
%\clist_gset:Nn \title_words_not_capitalized_en 
%    { a, an, the, at, by, for, in, of, on, to, up, and, as, but, it, or, nor, do, for, this, be,  
%    A, An, The, At, By, For, In, Of, On, To, Up, And, As, But, It, Or, Nor, Do, For, This, Be }
%    
%\cs_new:Npn \ucfirst_aux:w #1#2 \q_stop { \tl_upper_case:n { #1 } #2 }
%
%\cs_new:Npn \ucfirst #1 {
%     \exp_after:wN \ucfirst_aux:w #1 \q_stop
%}
%
%\cs_new:Npn \lowerfirst #1 {
%       \tl_lower_case:n {#1}
% }

\NewDocumentCommand\UppercaseTitle {s +m }
    {
      \IfBooleanTF { #1 } { {\bfseries {#2} } }
        {     \tex_hyphenpenalty:D = 10000
	        \seq_set_split:Nnn \g_tmpa_seq {~} {#2}
	        \seq_use:Nn   \g_tmpa_seq {~}\\
	        
	        \seq_pop_left:NN \g_tmpa_seq \l_tmpa_tl  
	       
	        {\bfseries\ucfirst \l_tmpa_tl \space} 
	        \seq_map_inline:Nn \g_tmpa_seq 
	           {
	              \clist_if_in:NnTF \title_words_not_capitalized_en { ##1 }
	              { {\bfseries \lowerfirst {##1}~}} { {\bfseries \ucfirst{##1}~ } }    
	            
	           }            
	       
       }    
    } 
\ExplSyntaxOff   
The rules for capitalization of titles varies from publication to publication and from department to department. A look at \href{http://arxiv.org/pdf/1505.04095v1.pdf}{arxiv} yielded a number of papers that do not follow the above rules. This will remain an unsolved problem, but at least we have moved forward. 

\begin{texexample}{Uppercase Title Issues}{}
\UppercaseTitle {Measuring Political Polarization: Twitter shows the two sides of Venezuela}\\
\UppercaseTitle {The Directed Dominating Set Problem: Generalized Leaf Removal and Belief Propagation}\\
\UppercaseTitle {Cities through the Prism of People's Spending Behavior}\\
\UppercaseTitle {On the p-th root of a p-adic number}\\
\UppercaseTitle {Planetary Formation Scenarios Revistied: Core-Accretion Versus Disk Instability}\\
\UppercaseTitle {de Haas-van Alphen effect versus Integer Quantum Hall effect}\\
\UppercaseTitle {A Simple Desultory Philippic (or How I Was Robert McNamara'd into Submission)}
\end{texexample}

The first title, shows a rule in many style manuals that words more than five characters should be capitalized, a rule broken by the third in the list above, although it is a preposition and many style books dictate that all prepositions be lowercase. It would make sense to add prepositions to our list. 

The last example is the Dutch and Afrikaans preposition \emph{de} meaning  \enquote{of} or \enquote{from}. This would make an exception on the first word of the sentence but not the last. The prefix von is not capitalised in German-speaking countries. The Duden dictionary recommends capitalizing the prefix von at the beginning of the sentence, but not in its abbreviated form, in order to avoid confusion with an abbreviated first name: \enquote{Von Humboldt kam später.} and \enquote{v. Humboldt kam später.} (Von Humboldt came later.) The Swiss Neue Zürcher Zeitung, however, recommends omitting the von completely at the beginning of the sentence: \enquote{Humboldt kam später.}

Just a sideline, the arxiv website has many papers in |.tex| format. If you want to peek at some papers it is an invaluable source of information.

\begin{description}
\item [First and last words] These are always capitalized. There is a general agreement for this one by all guides and editors.

\item [Prepositions] Do not capitalize English prepositions in the body of the title, but capitalize them if they are the first word.
\item [Foreign language prepositions] These have their own rules and are discussed later.
\end{description}

Let us now try and improve our code. In Example~\ref{ex:sequence}, we ensured that the first word is always capitalized by popping out the first word from the list and capitalizing it by using:

\begin{teXXX}
\seq_pop_left:NN \g_tmpa_seq \l_tmpa_tl 
\end{teXXX}

The first suggestion that comes to mind is to change is to add a pop left operation and perhaps to add both  to an auxiliary function so that we can later on add exceptions for foreign language names, as desribed in the specification earlier.

\begin{texexample}{Renew \textbackslash UppercaseTitle}{}
\ExplSyntaxOn
%\cs_new:Npn \ucfirst_aux:w #1#2 \q_stop { \tl_upper_case:n { #1 } #2 }
%
%\cs_new:Npn \ucfirst #1 {
%     \exp_after:wN \ucfirst_aux:w #1 \q_stop
%}
%
%\cs_new:Npn \lowerfirst #1 {
%       \tl_lower_case:n {#1}
% }

% Main command
\RenewDocumentCommand\UppercaseTitle {s +m }
    {
      \IfBooleanTF { #1 } { {\bfseries {#2} } }
        {     \tex_hyphenpenalty:D = 10000
	        \seq_set_split:Nnn \g_tmpa_seq {~} {#2}

	        % type splitted sequence	        
	        \seq_use:Nn   \g_tmpa_seq {~}\\
	        
	        % 
	        \pop_first:N  \g_tmpa_seq
	        \seq_pop_right:NN  \g_tmpa_seq \l_tmpb_tl
	        %
	        \seq_map_inline:Nn \g_tmpa_seq 
	           {
	              \clist_if_in:NnTF \title_words_not_capitalized_en { ##1 }
	              { {\bfseries \lowerfirst {##1}~}} { {\bfseries \ucfirst{##1}~ } }    
	            
	           }            
	    {{ \bfseries \ucfirst{\l_tmpb_tl} }}
       }    
    } 

 % Function to pop the first item and decorate it    
\cs_new_nopar:Npn \pop_first:N #1 {
 	        \seq_pop_left:NN \g_tmpa_seq\l_tmpa_tl
	        {\bfseries\ucfirst \l_tmpa_tl \space} 
  }
  
 % Function to pop last word and decorate it 
\cs_new_nopar:Npn \pop_last:N #1 {
	        \seq_pop_right:NN \g_tmpa_seq \l_tmpa_tl
	        {\bfseries\ucfirst  \l_tmpa_tl} 
}        
        
\ExplSyntaxOff    
\UppercaseTitle {What to do with Versus~}\\
\UppercaseTitle {What to do with versus?~}\\
\UppercaseTitle {What to do with versus: Versus or versus~?~}\\
\end{texexample}

What just happened is that we have also created two new auxiliaries one to pop the first word and another to pop the last word. We are now closer to a final solution, but the decoration of the words, needs to be taken care of as well. These are always better to be functions of their own and we can do it quite easily. By decoration we mean adding fonts colors and the like. We do not consider capitalization as decoration. 

\endinput

Another issue I would like to discuss is the usage of temporary variables in the examples. In my opinion it is not  good practice, as their syntax is sometimes confusing. 

\subsection{Final approach}

Given the rules above, some words can have three different ways of capitalization depending on their position in the sentence i.e, first, middle or end.

I posted some of the above code on |TEX.SX| and I had some amazing response from two of the developers of |expl3|. Under development there is a version of a |\title_case:n| command, which follows more or less the approach described in the example above.  I  have included the example in the chapter to demonstrate some of the aproaches to programming.

\robustify\url
\robustify\href
\robustify\textbf
\ExplSyntaxOn
\DeclareDocumentCommand \arxiv {g g}
{
  \IfNoValueTF {#1} {\href{http://arxiv.org}{{\color{blue}arxiv}}\xspace}
 {\href{http://arxiv.org/#1}{{\color{blue}#2}}\xspace}
}
\ExplSyntaxOff

The article \arxiv{abs/1505.05148}{ALMA maps the Star-Forming Regions in a Dense Gas Disk at z\char`\~3 } has also a problematic title.  Here the first word is an acronym and is left as is,  and the last word is an abbreviation also left as is. Exclusion list, abbreviation lists perhaps need to be build over time similar to hyphenation lists.  Another paper
\arxiv{abs/1505.05156}{Statistics of Measuring Neutron Star Radii: The Bayesian vs. The Frequentist Approach} shows some of the problems with abbreviations, such as \enquote{vs.}, so filtering through a list of exclusions before capitalization is unavoidable. 


\acrodef{QGP}{quark-gluon plasma}

The title  ``\arxiv{abs/1505.04994}{Viscosities of Gluon Dominated QGP Model within Relativistic Non-Abelian Hydrodynamics}'', will only be capitalized properly, except for the  \ac{QGP} acronym, which must be present in an exclusion list.

\begin{enumerate}
\item Write or ask for a specification. This can clarify requirements and avoid too many iterations of the code development. Have many examples of usage for testing.  Write tests for your code and always test against them. I understood the rules of title casing better from collecting titles from the \arxiv website.

\item Search for similar code and packages before you start developing.

\item Don't be afraid to ask the experts, most of the time they are more than willing to help.

\item Open source development is great. Consider contibuting to it. It is great for you and it is great for the community. The old concept of ``commons'' has more or less disappeared except in programming. Foster it and take care of it.
\end{enumerate}


\section{Summary}

In this chapter we reviewed the basics of the data type \enquote{Sequence lists} and have managed to produce some useful code in our final example. We have also reviewed some of the general concepts behind programming and have even managed to get two of the \latex3 developers to contribute code.

The code with some modifications is included in the \pkgname{phd} to provide title casing for headings and titles. The credit goes to Will Robertson and Joseph Wright. 

This chapter also brings us to the end of the list structures of |expl3|. Lua has its tables which are used to develop any data type and data structure required and similarly |expl3|'s lists can be used to develop and data structure you can imagine. One can think of link lists and tree data structures.

A link list can easily be developed as it consists of elements that point to the next element only.  

\begin{verbatim}
\expandafter\def \csname link_item_1_next\endcsname {end list marker}
\end{verbatim}

At creation the link list item will expand to a marker. When the second item is added, the previous item will point to the second element and so on. The advantages of a link list is that if we want to delete an item or insert, we do not have to iterate through the whole list. Of course, if we had to delete it we could just simply mark it as undefined.

All the lists and parsing described in this book depend on one amazing fact, which is what \tex does when scanning the argument specification of a macro (between |\def\acommand | and either the opening bracket |{|. Leverage this fact as much as you can in your parsers. 

During mapping this could be detected and not used. \tex like any language has its own paradigms and we need not   follow other language patterns, but is good to know that we truly have a highly flexible and Turing-complete language available (even if called a macro language). A macro language is still a language.



\chapter{LaTeX 3 clists module}

\epigraph{``I bet the human brain is a kludge.’’ }{---Marvin Minsky}

\precis{This chapter explores the expl3 comma delimited lists. It provides numerous working examples to demonstrate the use of the numerous available functions, provided by the module.}

\section{Introduction}

One of the most common data structure that computer languages provide are comma delimited lists.
 Comma lists contain ordered\footnote{Ordered does not mean sorted. It means they keep the order they were entered.} data where items can be added to the left
 or right end of the list. The resulting ordered list can then
 be mapped over using \docAuxCommand*{clist_map_function:NN}. Several items can
 be added at once, and spaces are removed from both sides of each item
 on input. Hence,
 \begin{verbatim}
   \clist_new:N \l_my_clist
   \clist_put_left:Nn \l_my_clist { ~ a ~ , ~ {b} ~ }
   \clist_put_right:Nn \l_my_clist { ~ { c ~ } , d }
 \end{verbatim}
 results in \docAuxCommand*{l_my_clist} containing |a,{b},{c~},d|.
 Comma lists cannot contain empty items, thus
 \begin{verbatim}
   \clist_clear_new:N \l_my_clist
   \clist_put_right:Nn \l_my_clist { , ~ , , }
   \clist_if_empty:NTF \l_my_clist { true } { false }
 \end{verbatim}
 will leave \texttt{true} in the input stream. To include an item
 which contains a comma, or starts or ends with a space,
 surround it with braces.  The sequence data type should be preferred
 to comma lists if items are to contain |{|, |}|, or |#| (assuming the
 usual \TeX{} category codes apply).

Implementation of list data structure normally provide the minimum following operations:

\begin{enumerate}
\item a constructor for creating an empty list;
\item an operation for testing whether or not a list is empty;
\item an operation for prepending an entity to a list
\item an operation for appending an entity to a list
\item an operation for determining the first component (or the "head") of a list
\item an operation for referring to the list consisting of all the components of a list except for its first (this is called the "tail" of the list.)
\end{enumerate}

 \section{Creating and initialising comma lists}

 \begin{docCommand}{clist_new:N}{ \meta{comma list}}
   Creates a new \meta{comma list} or raises an error if the name is
   already taken. The declaration is global. The \meta{comma list} will
   initially contain no items.
 \end{docCommand}


\begin{docCommand}{clist_const:Nn}{ \meta{clist~var} \marg{comma list}}
   Creates a new constant \meta{clist~var} or raises an error
   if the name is already taken. The value of the
   \meta{clist~var} will be set globally to the
   \meta{comma list}.
 \end{docCommand}

\begin{docCommand}{clist_clear:N}{ \meta{comma list}}
   Clears all items from the \meta{comma list}.
\end{docCommand}

 \section{Adding data to comma lists}

Adding data to a comma delimited list, is normally done through the use of helper functions and user commands.
If it is to be done once for example at the beginning of a document then it is a once off operation and we can use one of the \docAuxCommand*{clist_set} variants shown below. If the items are to be added programmatically or by the user in more than one place, then one of the functions \docAuxCommand*{clist_put_left} or \docAuxCommand*{clist_out_right} should be used. These prepend or append to the list, so goodbye \docAuxCommand*{@cdr}, \docAuxCommand*{@car} and their friends. 



% \begin{function}[added = 2011-09-06]
%   {
%     \clist_set:Nn,  \clist_set:NV,
%     \clist_set:No,  \clist_set:Nx,
%     \clist_set:cn,  \clist_set:cV,
%     \clist_set:co,  \clist_set:cx,
%     \clist_gset:Nn, \clist_gset:NV,
%     \clist_gset:No, \clist_gset:Nx,
%     \clist_gset:cn, \clist_gset:cV,
%     \clist_gset:co, \clist_gset:cx
%   }
  \begin{docCommand}{clist_set:Nn}{%
      \meta{comma list} \{
      \meta{item$_1$},
      \ldots,
      \meta{item$_n$} 
      \}
      }
   Sets \meta{comma list} to contain the \meta{items},
   removing any previous content from the variable.
   Spaces are removed from both sides of each item. Variants for : Nv,Nx,cV,cx,NV,cV exist.
 \end{docCommand}

\begin{texexample}{Creating a Comma delimited list}{ex:clists}
\ExplSyntaxOn
\clist_gset:Nn \phd_test_clist {one, two, three, four, five}
\phd_test_clist
\clist_gput_right:Nn \phd_test_clist{six, seven, eight}
\cs_new:Nn \nine: {9}
\clist_gput_right:Nn \phd_test_clist\nine:
\clist_gput_right:Nn \phd_test_clist\nine:
\par\phd_test_clist
\par\clist_if_in:NnTF\phd_test_clist {eight} {true} {false}
\ExplSyntaxOff 
\end{texexample}


%   \begin{syntax}
%     \docAuxCommand*{clist_put_left:Nn} \meta{comma list} \meta{item 1},\ldots{},\meta{item n}}
%   \end{syntax}
%   Appends the \meta{items} to the left of the \meta{comma list}.
%   Spaces are removed from both sides of each item.
% \end{function}
%
% \begin{function}[updated = 2011-09-05]
%   {
%     \clist_put_right:Nn,  \clist_put_right:NV,
%     \clist_put_right:No,  \clist_put_right:Nx,
%     \clist_put_right:cn,  \clist_put_right:cV,
%     \clist_put_right:co,  \clist_put_right:cx,
%     \clist_gput_right:Nn, \clist_gput_right:NV,
%     \clist_gput_right:No, \clist_gput_right:Nx,
%     \clist_gput_right:cn, \clist_gput_right:cV,
%     \clist_gput_right:co, \clist_gput_right:cx
%   }
 \begin{docCommand} {clist_put_right:Nn}  {\meta{comma list} \{\meta{item 1},\ldots{},\meta{item n}\}}
   Appends the \meta{items} to the right of the \meta{comma list}.
   Spaces are removed from both sides of each item.
\end{docCommand}

\begin{texexample}{Adding content to the list}{}
\ExplSyntaxOn
\clist_new:N \l_my_clist
\clist_put_right:Nn \l_my_clist{\square, \Diamond, \diamond, d, e, f}
\clist_put_left:Nn \l_my_clist{1,2,3,4,5,6,7,8,9,\hfill, 0}
\clist_use:Nn \l_my_clist{~}
\ExplSyntaxOff
\end{texexample}



\begin{texexample}{Adding content to the list}{}
\ExplSyntaxOn
\clist_put_right:Nn \l_my_clist {\alpha, \beta, \gamma, \delta, \epsilon}
\clist_put_left:Nn \l_my_clist {{\alpha\ldots}}
\[ \clist_use:Nnnn \l_my_clist {,} {,} {,} \]
\ExplSyntaxOff
\end{texexample}

\section{Mapping to comma lists}

\begin{texexample}{Mapping}{ex:longimages}
\ExplSyntaxOn
\clist_set:Nn \imgdb:n {fig145,fig161,fig162,fig163,fig164,fig165,fig166}
\clist_map_inline:Nn \imgdb:n {\includegraphics[width=1.5cm]{./images-01/#1}}
\ExplSyntaxOff
\end{texexample}

If the inline code is long, it might be preferable to use the function version of the map. This callback function should accept one parameter. Note that the mapping command format does not need the \#1 you only provide it with the function name.

\begin{texexample}{Mapping}{ex:longimages}
\ExplSyntaxOn
\cs_set:Npn \put_graphic:n #1 
   {
     \includegraphics[width=1.5cm]{./images-01/#1}
   }
\cs_set:Npn \put_graphic_with_space:n #1 
   {
     \put_graphic:n {#1}
     \hspace{5pt}
   }   
\clist_set:Nn \imgdb:n {fig145,fig161,fig162,fig163,fig164,fig165,fig166}

  \clist_map_function:NN \imgdb:n \put_graphic:n\par  
  \clist_map_function:NN \imgdb:n \put_graphic_with_space:n
\ExplSyntaxOff
\end{texexample}

The inline version is obviously a bit faster, as it does less work, but personally I prefer the callback style as it produces more readable code. Of course we could have used the |clist_if_in:NnTF| conditional. There are numerous conditionals and these are discussed later on.

The mapping function definitions are shown below,

\begin{docCommand}{clist_map_function:NN}{\meta{comma list} \meta{function}}
Applies a callback function to each item stored in the comma list. The function will receive one argument for each iteration. The items are returned from left to right. 
\end{docCommand}

\begin{docCommand}{clist_map_inline:NN}{\meta{comma list} \meta{inline function}}
Applies \meta{inline function} to every \meta{item} stored within the \meta{comma list}. The \meta{inline function} should consist of code which will receive the \meta{item} as \#1. One inline mapping can be nested inside another. The items are returned from left to right.
\end{docCommand}

There are is a third type of mapping function available where each entry in the list is passed to a variable which is then used in a function.

\begin{docCommand}{clist_map_variable:NNn}{\meta{comma list} \meta{tl. var} \marg{inline function}}
Stores each entry in the \meta{comma list} in turn in the \meta{tl var} and applies \meta{function} using \meta{tl var}. the function will usually consist of code making use of the \meta{t var}, but this is not enforced. One variable mapping can be nested inside another. the \meta{items} are returned from left to right.
\end{docCommand}

\subsection{Terminating mapping functions}

All lists in |expl3| can be terminated using a break function. A clist breaks by using the \docAuxCommand*{clist_map_break:n} function. 

\begin{texexample}{Mapping}{ex:longimages}
\ExplSyntaxOn
\cs_set:Npn \put_graphic:n # 1 
   { 
     \includegraphics[width=1.48cm]{./images-01/#1}
   }
\cs_set:Npn \put_graphic_with_space:n #1 
   {
      \parbox[b]{1.52cm}{\put_graphic:n {#1}\par\centering #1}
     \hspace{5pt}
   }   
   
\clist_set:Nn \imgdb:n {fig145,fig161,fig162,fig163,fig164,fig165,fig166}

 \clist_map_function:NN \imgdb:n \put_graphic_with_space:n\par
 \clist_map_inline:Nn \imgdb:n
     {
         \str_if_eq:nnTF {#1} {fig166}
         {\clist_map_break:n { \PASS~ \put_graphic:n {#1} ~#1} }
         {
          \FAIL #1
         }
     }
     
\ExplSyntaxOff
\end{texexample}

What just happened have added \docAuxCommand*{clist_map_inline:Nn} which iterates through all the elements in a list until a search string is found. As you can see as a search function it will be slow as it has to iterate through all the elements of a list. A more efficient way would have been to use \tex’s scanning mechanism of delimited functions to find the item. 

I have named the |clist| in the above examples as |imgdb| as one can easily extend the functions to store other information besides the filename. This can be done in many ways for example using the \meta{property} module of |expl3| or using |\csname|. In Chapter~\ref{ch:longfifgures} \nameref{ch:longfigures} we have used traditional techniques to typeset a lot of figures, in a similar fashion to a long table. Here we provide a similar example using |expliii|.

let us consider the simple case of a record for a person.

\begin{texexample}{Person record}{}
\ExplSyntaxOn
% create a new clist
\clist_new:N \personDB 

% auxiliary function to typeset an image
\cs_gset:Npn \put_graphic:n #1 
   {
     \includegraphics[height=3cm]{#1}
   }

% auxiliary function to enclose the image in a minipage      
\cs_gset:Npn \put_graphic_with_space:n #1 
   {
       \begin{minipage}[b]{3cm}
             \centering
             \put_graphic:n {#1}\par
             \csname#1_name\endcsname\\
             \csname#1_occupation\endcsname\\
      \end{minipage}\hspace{5pt}  
   }   

% helper function to add a person record to the clist (*@\label{lin:personrecord}@*)
\cs_gset:Npn \addtodb:nn #1#2#3
    {
        \cs_gset:cpn { #1_name } { #2 }
        \cs_gset:cpn { #1_occupation } { #3 }
        \clist_gput_left:Nn \personDB { #1 }
    }
    
%        
\addtodb:nn {turner}  {Ted~Turner}  {tycoon}
\addtodb:nn {britney} {Britney Spears} {actress}
\addtodb:nn {che} {Che ~Guevara} {revolutionary}
\clist_map_function:NN \personDB \put_graphic_with_space:n\par
\ExplSyntaxOff
\end{texexample}

The Line~\ref{lin:personrecord} creates to macros, one that will hold the name of the person and another that will hold the occupation. Note that the code would have normally used a |\csname| construction. Here |expl3| takes care of both the |expandafter| as well as the |\csname| construct, simply by using |:cpn| version of |gset|.

What is different with this example, we have added the \docAuxCommand*{addtodb:nn} to add the person names to the list. This still has to be done as an author interface, but as it is just an example, I want to keep the code short. 

\textbf{Automating the addition of fields and records} We have named our database |imageDB| and we have called it a database, but it is so far very unfriendly and all the fields are hard wired in the next Chapter we will create a more appropriate record database.

\textbf{Create a new data base} First we concern ourselves with creating a new database. This is the very first activity we need to define. We store the names of the databases in a master list which we have named |\g_DB_dbs_clist|. We will prefix all our functions and variables with |DB| and we will use this as our module name. 



\textbf{Specifying the database meta data} Our databases will be also records or objects if you want to use an inexactitude name and will also hold information, this is termed \emph{meta data}:

\begin{tabular}{ll}
  name   & \meta{database name} \\
  fields & \meta{list containing the fields as fieldnames}\\
  status & \meta{active or not active}\\
  number of records &\\
  tables & \meta{list}\\
  views  & \\
\end{tabular}

\def\paragraph#1{{\par\leavevmode\bfseries#1}}

\paragraph {Create the master database list} All databases that we will create will be stored
as meta data into another list. This is used only internally at this stage, so we give it an |expl3| sexy name \docAuxCommand{g_db_dbs_clist}.

\begin{texexample}{Creating a database package}{ex:master DB list}
\ExplSyntaxOn
% already defined no need to have it in the example
 \clist_new:N \g_db_dbs_clist
\ExplSyntaxOff
\end{texexample}

\paragraph{Constructor function} Next we create a function that is called when we
need to create a new database.

\begin{texexample}{Continued..}{}
\ExplSyntaxOn
% constructor function
\cs_gset:Npn \g_db_construct_clist:n #1
  { 
% create new DB
  \clist_new:c {#1} 
	% add to master
  \clist_put_left:Nn \g_db_dbs_clist { #1 }
% create meta table
  \g_construct_metatable:n { # 1}
  }
\ExplSyntaxOff
\end{texexample}
		
\paragraph{Creating tables} So far we have created the functions that we need to create a new database. Next we can start writing functions for creating tables for a database. In reality, I called them tables, but this is a misnomer as they hold other stuff as well. 
		
\begin{texexample}{...continued}{ex:db4}		
\ExplSyntaxOn
% persons metatable
% PERSONS-METANAME
% PERSONS-STATUS
% PERSONS-TABLES
\cs_gset:Npn \g_construct_metatable:n #1 
  {
    \cs_gset:cpn   {#1-METANAME  } {   #1    }
    \cs_gset:cpn   {#1-METASTATUS} {-NoValue-}
    \clist_gset:cn {#1-METATABLES} {-NoValue-}
  }		
  
% PERSONS-TABLE-TABLENAME 
% PERSONS-TABLE-TABLENAME-FIELDS (list)		

\cs_gset:Npn \g_construct_table:cc #1 #2 
  {
    \cs_gset:cpn   {#1-TABLE-#2-NAME      } {#2}
    \cs_gset:cpn   {#1-TABLE-#2-STATUS    } {}
    \clist_gset:cn {#1-TABLE-#2-FIELDNAMES} {}
    
    % index key as edef
    \tl_gset:cx  {#1#2-} {name}
    
    % data holding list
    \clist_gset:cn { #1 #2 } { } %(*@\label{lin:personsfamous}@*)
  }
\ExplSyntaxOff  
\end{texexample}

The interesting part is line~\ref{lin:personsfamous} which is a comma delimited list
that will hold all the index keys.

\begin{texexample}{Databases...continued}{ex:fields}
\ExplSyntaxOn
% adds a fieldname to fieldnames
% PERSONS-TABLE-TABLENAME-FIELDNAMES
\cs_gset:Npn \add_fieldname #1 #2 #3
  {
    \clist_gput_left:cx {#1-TABLE-#2-FIELDNAMES} {#3}

  }
%

\cs_gset:Npx \create_index_field #1#2#3#4
  {
    \clist_gput_left:cx {#1#2} {#4}
    
  }  
% create DB table FAMOUS 
\cs_gset:Npx \add_data_index #1#2#3#4
  {
    \clist_gput_left:cx {#1#2} {#4}
    
  } 
  
% add data if is index goes onto clist  
% PERSONS-FAMOUS-ID-SURNAME-VALUE
%   
\cs_gset:Npn \add_field_data #1#2#3#4#5 
  {
   \cs_gset:cpn {#1#2#3#4} 
    { #5    }
  } 
  



% read a field        
\cs_gset:Npn \get_field #1#2#3#4
  { 
    \cs:w #1#2#3#4\cs_end:  
  }
                                
 % create DB PERSONS   
\g_db_construct_clist:n {PERSONS}
\g_construct_table:cc {PERSONS}{FAMOUS}                                
%
\gdef\AddPerson#1#2#3#4{
	\add_data_index {PERSONS} {FAMOUS} {name} {#1}
	\add_field_data {PERSONS} {FAMOUS}{#1} {firstname   } {#1}
	\add_field_data {PERSONS} {FAMOUS}{#1} {surname   } {#2}
	\add_field_data {PERSONS} {FAMOUS}{#1} {occupation} {#3} 
	\add_field_data {PERSONS} {FAMOUS}{#1} {photo} {#4}
}
%
%\get_field {PERSONS} {FAMOUS} {Iggy}  {photo} 
\gdef\PrintImages#1#2{
  \centering 
  \clist_map_inline:cn {#1#2}
    {
      \includegraphics[height=3cm]
      {./martin-schoeller/
        \get_field {#1}{#2}{##1}{photo}
      }\hskip1sp
    }
}
\ExplSyntaxOff
\end{texexample}

What just happened is that we have created two lists one to hold DBs metadata as a simple list and a second |PERSONS|. We have also created the meta-data record. 
 
\begin{texexample}{Database ...continued}{ex:db2}
\AddPerson {Barack} {Obama} {Actor} {barack_obama_2004}
\AddPerson {Iggy} {Pop} {Actor} {iggy_pop_2001}
\AddPerson {Henry} {Kissinger} {Arsehole} {henry_kissinger_2007}
\AddPerson {Frankie} {Velilla} {Student} {frankie_velilla_2001}
\AddPerson {Cindy} {Sheman} {Queen} {cindy_sheman_2000}
\AddPerson {Joe} {Namath} {Tough} {joe_namath_2006}
\AddPerson {Christopher} {Walken} {Tough} {christopher_walken_2000}
\AddPerson {Xiakababoi} {Xiakababoi} {Tough} {xiakababoi_2005}
\AddPerson {Jack} {Nicholson} {Tough} {jack_nicholson_2002}
\AddPerson {Robert} {Deniro} {Actor} {robert_DeNiro_2006}
\PrintImages{PERSONS}{FAMOUS}
\end{texexample}


Now what happens if we decide that we want to add another field in 
our database of famous people, say their biography? we would need to add
another document level command |\AddPersonBio|

\begin{texexample}{adding a bio field}{ex:bio}
\ExplSyntaxOn
\long\gdef\AddPersonBio #1#2 {
   \add_field_data {PERSONS} {FAMOUS} {#1} {bio} {#2}
}
\ExplSyntaxOff
\end{texexample}

Let us add some data for some of the person records we have in our database.

\ExplSyntaxOn
\DeclareDocumentCommand \GetBio {m} {
  \get_field {PERSONS}{FAMOUS}{#1}{bio}
}
\DeclareDocumentCommand \GetPhoto {m} {
  \includegraphics[width=0.8\linewidth] {./martin-schoeller/
    \get_field {PERSONS} {FAMOUS} {#1} {photo}} 
  }
\DeclareDocumentCommand \GetFullName {m} {
    \get_field {PERSONS} {FAMOUS} {#1} {firstname}
    \space 
    \get_field {PERSONS} {FAMOUS} {#1} {surname}
} 
\ExplSyntaxOff


\begin{texexample}{add bio to some records}{ex:bio1}
\ExplSyntaxOn
\DeclareDocumentCommand \GetBio {m} {
  \get_field {PERSONS}{FAMOUS}{#1}{bio}
}
\DeclareDocumentCommand \GetPhoto {m} {
  \includegraphics[width=0.8\linewidth] {./martin-schoeller/
    \get_field {PERSONS} {FAMOUS} {#1} {photo}} 
  }
\DeclareDocumentCommand \GetFullName {m} {
    \get_field {PERSONS} {FAMOUS} {#1} {firstname}
    \space 
    \get_field {PERSONS} {FAMOUS} {#1} {surname}
}    
\ExplSyntaxOff
\end{texexample}
\begin{texexample}{add some more declarations}{ex:2}
\AddPersonBio{Robert}{
  Robert De Niro (/dəˈnɪroʊ/; born August 17, 1943) is an American actor and producer   who has starred in over 90 films. His first major film roles were in the sports drama Bang the Drum Slowly (1973) and Martin Scorsese's crime film Mean Streets (1973). In 1974, after being turned down for the role of Sonny Corleone in the crime film The Godfather (1972), he was cast as the young Vito Corleone in The Godfather Part II (1974), a role for which he won the Academy Award for Best Supporting Actor.

De Niro's longtime collaboration with Scorsese later earned him an Academy Award for Best Actor for his portrayal of Jake LaMotta in the 1980 film Raging Bull. He also earned nominations for the psychological thrillers Taxi Driver (1976) and Cape Fear (1991), both directed by Scorsese. De Niro received additional Academy Award nominations for Michael Cimino's Vietnam war drama The Deer Hunter (1978), Penny Marshall's drama Awakenings (1990), and David O. Russell's romantic comedy-drama Silver Linings Playbook (2012). His portrayal of gangster Jimmy Conway in Scorsese's crime film Goodfellas (1990) earned him a BAFTA nomination in 1990.[1] De Niro has earned four nominations for the Golden Globe Award for Best Actor – Motion Picture Musical or Comedy, for his work in the musical drama New York, New York (1977), opposite Liza Minnelli, the action comedy Midnight Run (1988), the gangster comedy Analyze This (1999), and the comedy Meet the Parents (2000). He has also simultaneously directed and starred in films such as the crime drama A Bronx Tale (1993) and the spy film The Good Shepherd (2006). De Niro has also received the AFI Life Achievement Award in 2003 and the Golden Globe Cecil B. DeMille Award in 2010.}

\AddPersonBio {Jack}{
John Joseph "Jack" Nicholson (born April 22, 1937) is an American actor and filmmaker. Throughout his career, Nicholson has portrayed unique and challenging roles, many of which include dark portrayals of excitable, neurotic and psychopathic characters. Nicholson's 12 Academy Award nominations make him the most nominated male actor in the Academy's history.

Nicholson has won the Academy Award for Best Actor twice, one for the drama One Flew Over the Cuckoo's Nest (1975) and the other for the romantic comedy As Good as It Gets (1997). He also won the Academy Award for Best Supporting Actor for the comedy-drama Terms of Endearment (1983). Nicholson is tied with Walter Brennan and Sir Daniel Day-Lewis as one of three male actors to win three Academy Awards. In 1988 Nicholson won a Grammy Award for Best Album for Children for The Elephant's Child. He is well known for playing Frank Costello in the Martin Scorsese-directed crime drama The Departed (2006), Jack Torrance in the Stanley Kubrick–directed psychological horror film The Shining and the Joker in Batman (1989).

Nicholson is one of only two actors to be nominated for an Academy Award for acting in every decade from the 1960s to the 2000s; the other was Michael Caine. He has won six Golden Globe Awards, and received the Kennedy Center Honor in 2001. In 1994, he became one of the youngest actors to be awarded the American Film Institute's Life Achievement Award. Other notable films in which he has starred include the road movie Easy Rider (1969), the drama Five Easy Pieces (1970), the comedy-drama film The Last Detail (1973), the neo-noir mystery film Chinatown (1974), the drama The Passenger (1975), the epic film Reds (1981), the romantic horror film Wolf (1994), the legal drama A Few Good Men (1992), the Sean Penn-directed mystery film The Pledge (2001), and the comedy-drama About Schmidt (2002).
}
\end{texexample}

Finally continuing our example we will now define a \docAuxCommand{PrintBio} that can be used
to finally extract the data and present typeset it.

  
\begin{texexample}{Printing the Bios}{ex:bio3}
\long\gdef\PrintBio#1{%
\par
 {\pagebreak
 \leavevmode 
 \Huge
 \bfseries
 \centerline{\GetFullName {#1}}}
 \par
 \vspace{20pt}

 {\centering
  \GetPhoto {#1}\par
  \vspace{20pt}}

 \parindent1em
 \GetBio {#1}
 \vfill
}

\PrintBio{Robert}

\PrintBio{Jack}
\end{texexample}

There are a lot of improvements that we can do to the code. Firstly we have not done any error checking. The idea of pre-packaged code is that the finer details can be handled. Error checking should be done for example to verify that an image is available on disk. Also not to hard wired any directories. Sorting is still an issue. Our indexing is also inadequate. What happens if we have Robert DeNiro and Robert Williams? We indexed on the Robert. We would have been better off to add an index key automatically or index by using both name and surname. All these are issues that need to be incorporated. 



\chapter{Queues}
\section{Queue Fundamentals}

A queue is an ordered list in which all insertions are made at one end, called the rear end, while all deletions are made at the other end, called the front end. Given a queue $Q=(a_1,a_2,\dots,a_n)$ with $a_1$ as the front element and $a_n$ as the rear element, we say that $a_{i+1}$ is behind $a_1$ $1 \leq i <n$.

\section{Operations on a Queue}

The operations which are carried on queue are similar to these which are carried on a stack, except their semantics are different. The operations are:

\begin{enumerate}
\item To create a queue
\item To insert an element into the queue
\item To delete an element from the queue
\item To check which element is in the front of  the queue
\item To check whether a queue is empty or not.
\end{enumerate}

\begin{figure}[htbp]
\centering
\includegraphics[width=0.5\textwidth]{queue}
\end{figure}

Since this is a book about typesetting, the next example will create a queue structure that will typeset the operations of a queue and provide diagrams to illustrate the algorithmic steps involved.


\begin{docCommand}{CreateQueue}{\meta{queue name}}
Creates an empty queue .
\end{docCommand}

\def\anitem{{\color{blue}\vrule height1.5cm width0.4cm}\thinspace}
\DeclareDocumentCommand\anitem{O{blue}}{%
{\color{#1}\vrule height1.5cm width0.4cm}\thinspace
}
\NewDocumentCommand\EnqueueString{s}{
  \IfBooleanTF #1
     {Enqueue $\rightarrow$}
     {Enqueue \phantom{$\rightarrow$}}
}

\NewDocumentCommand\DequeueString{s}{
  \IfBooleanTF #1
     {\phantom{$\rightarrow$} Dequeue $\rightarrow$ }
     {\phantom{$\rightarrow$} Dequeue \phantom{$\rightarrow$}}
}
\begin{enumerate}
\item The conventions we will use is that when an item is enqueued it will be typeset in red as shown below, when it enters the front end.

Enqueue $\rightarrow$ \anitem[red]\DequeueString* 

\item When another item is added the above procedure is repeated, but this time the elements not in the front are shown in blue.

\EnqueueString* \anitem[red]\anitem \phantom{$\rightarrow$} Dequeue

\item Enqueue one more item will change the diagram to the following:

\EnqueueString \anitem \anitem \anitem \anitem \anitem \phantom{$\rightarrow$} Dequeue \hfill \anitem[blue!30] 

\item Enqueue one more item will change the diagram to the following:

\EnqueueString \anitem \anitem \anitem \anitem  \DequeueString \hfill \anitem[blue!30] \anitem[blue!30]

\item To summarize the typeset diagram represents the three states of the queue, enqueue, status and dequeue. If a right arrow is shown it either enqueued or dequeued an element. If none is shown it represents the status of the system
\end{enumerate}

I have specifically made the example a bit more complicated, in order to reinforce some of the concepts discussed in other chapters.

The example requires that when a dequeuing command is entered it is indicated with a right arrow (|dequeue \rightarrow|) the arrow is not shown when the enqueuing operation takes place. To keep the length of the diagram spaced properly it requires that a phantom command is used for the enqueing operation.

\subsection{Coding auxiliary macros}

We will need two auxiliary macros to typese the enqueue and dequeue strings with or without arrows. We will use the \pkgname{xparse} package to create the commands. We will use the star version of the command as a toggle to show the \docAuxCommand*{rightarrow} or not. If the command is enetered with a star it will leave the right amount of space to the right of the string, so that all diagrams line nicely to the left. This is achieved using the  \docAuxCommand*{phantom} command that we have encountered earlier.

\emphasis{IfBooleanTF}
\begin{teXXX}
\NewDocumentCommand\EnqueueString{s}{
  \IfBooleanTF #1
     {Enqueue $\rightarrow$ }
     {Enqueue \phantom{$\rightarrow$}}
}
\NewDocumentCommand\DequeueString{s}{
  \IfBooleanTF #1
     {\phantom{$\rightarrow$} Dequeue $\rightarrow$ }
     {\phantom{$\rightarrow$} Dequeue \phantom{$\rightarrow$}}
}
\end{teXXX}

\subsection{Creating the Queue Macros}

In order to typeset the diagrams we will use two queues. One to store the main queue and a second one to store the dequeue items. Before we code the actual functions it will be nice to think of the  commands we want to offer our users. This will also dictate to an extend the code we require.

\begin{verbatim}
\DrawQueStatus
\Enque
\Deque
\DrawEnque
\DrawDeque
\end{verbatim}

\begin{texexample}{Adding content to the sequence}{}
\ExplSyntaxOn
\seq_new:N \g_qlisti
\seq_new:N \g_qlistii
\seq_gpush:Nn \g_qlisti{\anitem[blue]}
\seq_gpush:Nn \g_qlisti{\anitem[blue]}

\seq_push:Nn \g_qlistii{\anitem[blue!30]}

\DeclareDocumentCommand\Enque{O{red}}
   {
      \seq_gpush:Nn \g_qlisti{\anitem[#1]}
   }
   
 \DeclareDocumentCommand\Deque{O{blue!30}}
   {
      \seq_gpop_left:NN \g_qlisti \@tempa
      \seq_gpush:Nn\g_qlisti{\anitem[blue]}
      \seq_gpush:Nn \g_qlistii{\anitem[#1]}
   }  
   
\Enque\Enque\Enque\Enque

\EnqueueString\seq_use:Nn \g_qlisti {} 

\DequeueString*\hfill\seq_use:Nn \g_qlistii{}
\ExplSyntaxOff
%%%%
\end{texexample}
\ExplSyntaxOn
\DeclareDocumentCommand\Enque{O{red}}
   {
      \seq_gpush:Nn \g_qlisti{\anitem[#1]}
   }
   
 \DeclareDocumentCommand\Deque{O{blue!30}}
   {
      \seq_gpop_left:NN \g_qlisti \@tempa
      \seq_gpush:Nn\g_qlisti{\anitem[blue]}
      \seq_gpop_right:NN \g_qlisti \@tempa
      \seq_gpush:Nn \g_qlistii{\anitem[#1]}
   }  
\ExplSyntaxOff
   
One of the characteristics of the programming process is that it is like painting. Some programmers come up with  excellent code on their first attempt, whereas most of us will \emph{refactor} the code over several passes either to improve it, optimize it or catch possible errors.

A subtle issue with the above code is if we enqueue a number of items and then dequeue only the first item will change from red to blue the rest will be still in the que as red. What we will have to do is modify the \docAuxCommand*{Enque} to check if the list is not empty to remove the head item and replace it with a blue box, before effecting the enque operation. This will also give us a chance to use the sequence conditional functions for emptiness. We should also add the conditional in the \docAuxCommand*{Deque} function as well as the author typesetting commands \docAuxCommand*{DrawEnque} and \docAuxCommand*{DrawDeque}.

\begin{texexample}{The drawing functions}{ex:drgfunctions}
\ExplSyntaxOn
\DeclareDocumentCommand\DrawDeque{ O{blue!30} }
  { 
   \EnqueueString  \seq_use:Nn \g_qlisti {} 
   \DequeueString*  \hfill  \seq_use:Nn \g_qlistii{}    
  }
\Deque\Deque\Deque
\DrawDeque
\ExplSyntaxOff  
\end{texexample}

The \docAuxCommand*{DrawQues}, draws the two queues. This is very similar to the other two \docAuxCommand*{Draw}{\meta{deque}} or \meta{enque} functions. It just does not draw the arrows.

\begin{texexample}{The drawing functions}{ex:drgfunctions}
\ExplSyntaxOn
\DeclareDocumentCommand\DrawQues{ O{blue!30} }
  { 
   \EnqueueString  \seq_use:Nn \g_qlisti {} 
   \DequeueString  \hfill  \seq_use:Nn \g_qlistii{}    
  }
\Deque
\DrawQues
\ExplSyntaxOff  
\end{texexample}


 \chapter{Using Comma lists as stacks}
 
 In this chapter, we will look at one common Abstract Data Type (ADT), the stack. A stack is a \emph{collection}, meaning that it is a data structure that contains multiple elements. Other collections we have seen include dictionaries and lists. An ADT is defined by the operations that can be performed on it, which is called an interface. The interface for a stack consists of these operations:

\begin{description}
\item [init] Initialize a new empty stack.
\item [push]
Add a new item to the stack.
\item [pop]
Remove and return an item from the stack. The item that is returned is always the last one that was added.

\item [emptiness] Check whether the stack is empty.
\end{description}

A stack is sometimes called a “Last in, First out” or LIFO data structure, because the last item added is the first to be removed.

 Comma lists can be used as stacks, where data is pushed to and popped
 from the top of the comma list. (The left of a comma list is the top, for
 performance reasons.) The stack functions for comma lists are not
 intended to be mixed with the general ordered data functions detailed
 in the previous section: a comma list should either be used as an
 ordered data type or as a stack, but not in both ways.
 
 \begin{figure}[htbp]
 \hspace*{3cm}%optically center it
 \scalebox{0.7}{\begin{drawstack}
  \startframe
  \cell{First cell}
  \cell{Second cell}
  \finishframe{Some stack frame}
  \cell{Not interesting}
  \startframe
  \cell{Next stack frame}
  \cell{Next stack frame}
  \finishframe{Another stack frame}
\end{drawstack}}
\caption{A stack drawn with the \pkgname{drawstack} package. The package can be used to draw different stacks and their frames.}
\end{figure}

To construct a new empty stack, use the same functions as for a clist or sequence data structure. They are identical and calling them a stack is just syntactic sugar.

 \begin{docCommand}{clist_get:NN}{ \meta{comma list} \meta{token list variable}}
   Stores the left-most item from a \meta{comma list} in the
   \meta{token list variable} without removing it from the
   \meta{comma list}. The \meta{token list variable} is assigned locally.
   If the \meta{comma list} is empty the \meta{token list variable} will
   contain the marker value \docAuxCommand*{q_no_value}.
 \end{docCommand}
\makeatletter 
 \global\let\clistsort\lst@BubbleSort
\makeatother 
\begin{texexample}{Sequence}{ex:sequence}
\makeatletter
\ExplSyntaxOn
\clist_gset:Nn \title_words_not_capitalized_en 
 { 
  a,an,the,at,by,for,in,of,on,to,up,and,as,but,it,or, 
  nor,do,for,this,be,A,An,The,At,By,For,In,Of,On,To,Up, 
  And,As,But,It,Or,Nor,Do,For,This,Be, 
  abaft,aboard,about,above,absent,across,afore,after,against,along,
  alongside,amid,amidst,among,amongst,an,anenst,apropos,apud,around,
  as,aside,astride,at,athwart,atop,barring,before,behind,below,beneath,
  beside,besides,between,beyond,but,by,circa,concerning,despite,down,
  during,except,excluding,failing,following,for,forenenst,from,given,in,
  including,inside,into,lest,like,mid,midst,minus,modulo,near,next,
  notwithstanding,of,off,on,onto,opposite,out,outside,over,pace,past,
  per,plus,pro,qua,regarding,round,sans,save,since,than,through,
  throughout,till,times,to,toward,towards,under,underneath,unlike,
  until,unto,up,upon,Versus,versus,via,vice,with,within,without,worth
}
\clist_gset:Nn \abbreviations 
 {
   A.B.C.,iTunes
 }

\clist_gset:Nn \acronyms
 {
   NATO,UN,US,Scuba,Laser
 }  
\cs_new:Npn \addacronym #1 
 {
   \clist_put_left:Nn \acronyms {#1}
   \lst@BubbleSort\acronyms
   \clist_remove_duplicates:N\acronyms
 }  

\addacronym {EU}
\meaning\acronyms\\
\addacronym {AA}
\acronyms
\ExplSyntaxOff
\makeatother
\end{texexample}


\endinput
\end{document}

\section{Moving items from one list to another}{}

\begin{texexample}{Moving items from one stack to another.}{ex:stacks}
\ExplSyntaxOn
\clist_new:N \phd_stack_a
\clist_new:N \phd_stack_b
\token_to_meaning:N \phd_stack_a
\Expl_SyntaxOff
\end{texexample}

If we examine the meaning of the stacks at this stage, they are just empty macros, not holding any values.

\begin{texexample}{put something into the stacks}{}
\ExplSyntaxOn
\clist_gset:Nn \phd_stack_a {3,4,5,6,}
\clist_gpush:Nn\phd_stack_a {\ldots}

STACK a:~\phd_stack_a\par
\ExplSyntaxOff
\end{texexample}


Let us continue by popping and pushing some more values
\begin{texexample}{Continue}{}
\ExplSyntaxOn
\clist_gpop:NN\phd_stack_a\@tempa
\clist_gpush:Nx\phd_stack_b\@tempa
%% Pop a value
\clist_gpop:NN\phd_stack_a\@tempa
\clist_gpush:Nx\phd_stack_b\@tempa
\clist_gpop:NN\phd_stack_a\@tempa
\clist_gpush:Nx\phd_stack_b\@tempa
\clist_gpop:NN\phd_stack_a\@tempa
\clist_gpush:Nx\phd_stack_b\@tempa
stack a:~\phd_stack_a\par
stack b:~\phd_stack_b
\ExplSyntaxOff
%%%%%%%%%%%%%
\end{texexample}

The much promised freedom from having to deal with \tex expansion has not arrived---although we can save some frustration and typing. When moving items from |stacka| to |stack b| we have used the |:Nx| form of the command
so that the temporary token list variable is expanded. If we do not do that the second stack values will only store the last value of |@tempa|.

In practical applications the second stack is normally used as an array to just store the values. The symbol items before being popped are examined and if for example is a |+| sign the items will be summed up and placed again in the second stack to keep tally of our totals.

Our next step is to refactor the code in our example to recursively empty the first stack.

\begin{texexample}{Moving items from one stack to another}{ex:stacks}
\ExplSyntaxOn
\fboxsep=2pt
\fboxrule=0.4pt
\clist_gset:Nn \phd_stack_a {1,2,3,4,5,6,\ldots}
\clist_gset:Nn \phd_stack_b {}
original~stack a:~
\cs_gset:Nn\recurse:
 {
   \clist_gpop:NNTF\phd_stack_a\@tempa{\clist_gpush:Nx\phd_stack_b\@tempa
      \fbox{\@tempa}~
      \recurse:}{empty~stack\par}
 }  
\recurse: 
stack b:~\phd_stack_b
\ExplSyntaxOff
%%%%%%%%%%%%%
\end{texexample}

\begin{texexample}{Moving items from one stack to another}{ex:stacks}
\ExplSyntaxOn
\fboxsep=2pt
\fboxrule=0.4pt
\cs_gset:Nn\recurseb:
 {
   \clist_gpop:NNTF\phd_stack_b\@tempa{
      \if +\@tempa\relax\else\framebox[1.5em]{\strut\@tempa}\\ \fi
      \recurseb:}{empty~stack\par}
 }  
\recurseb: 
stack b:~\phd_stack_b
\ExplSyntaxOff
%%%%%%%%%%%%%
\end{texexample}

So far so good. We have managed to construct two stacks and to typeset their content in nice boxes. Hopefully, by now if you have been following the examples, you have the rudimentary skills to build our next, more complicate example that would parse a sequence of algebraic expressions and tokenize them. 



\begin{docCommand}{clist_get:NNTF}{ \meta{comma list} \meta{token list variable} \marg{true code} \marg{false code}}
 
   If the \meta{comma list} is empty, leaves the \meta{false code} in the
   input stream.  The value of the \meta{token list variable} is
   not defined in this case and should not be relied upon.  If the
   \meta{comma list} is non-empty, stores the top item from the
   \meta{comma list} in the \meta{token list variable} without removing it
   from the \meta{comma list}. The \meta{token list variable} is assigned
   locally.
 \end{docCommand}


  \begin{docCommand}{clist_pop:NN}{ \meta{comma list} \meta{token list variable}}
   Pops the left-most item from a \meta{comma list} into the
   \meta{token list variable}, \emph{i.e.}~removes the item from the
   comma list and stores it in the \meta{token list variable}.
   Both of the variables are assigned locally.
 \end{docCommand}


  \begin{docCommand}{clist_gpop:NN}{ \meta{comma list} \meta{token list variable}}
   Pops the left-most item from a \meta{comma list} into the
   \meta{token list variable}, \emph{i.e.}~removes the item from the
   comma list and stores it in the \meta{token list variable}.
   The \meta{comma list} is modified globally, while the assignment of
   the \meta{token list variable} is local. Also available as :cN
 \end{docCommand}

 \begin{docCommand}{clist_pop:NNTF}{ \meta{sequence} \meta{token list variable} \marg{true code} \marg{false code}}
   If the \meta{comma list} is empty, leaves the \meta{false code} in the
   input stream.  The value of the \meta{token list variable} is
   not defined in this case and should not be relied upon.  If the
   \meta{comma list} is non-empty, pops the top item from the
   \meta{comma list} in the \meta{token list variable}, \emph{i.e.}~removes
   the item from the \meta{comma list}. Both the \meta{comma list} and the
   \meta{token list variable} are assigned locally.
 \end{docCommand}


   \begin{docCommand}{clist_gpop:NNTF}{\meta{comma list} \meta{token list variable} \marg{true code} \marg{false code}}
     If the \meta{comma list} is empty, leaves the \meta{false code} in the
   input stream.  The value of the \meta{token list variable} is
   not defined in this case and should not be relied upon.  If the
   \meta{comma list} is non-empty, pops the top item from the
   \meta{comma list} in the \meta{token list variable}, \emph{i.e.}~removes
   the item from the \meta{comma list}. The \meta{comma list} is modified
   globally, while the \meta{token list variable} is assigned locally.
 \end{docCommand}

% \begin{function}
%   {
%     \clist_push:Nn,  \clist_push:NV,  \clist_push:No,  \clist_push:Nx,
%     \clist_push:cn,  \clist_push:cV,  \clist_push:co,  \clist_push:cx,
%     \clist_gpush:Nn, \clist_gpush:NV, \clist_gpush:No, \clist_gpush:Nx,
%     \clist_gpush:cn, \clist_gpush:cV, \clist_gpush:co, \clist_gpush:cx
%   }
 \begin{docCommand}{clist_push:Nn}{ \meta{comma list} \marg{items}}
   Adds the \marg{items} to the top of the \meta{comma list}.
   Spaces are removed from both sides of each item.
 \end{docCommand}
%
% \section{Using a single item}
%
% \begin{function}[added = 2014-07-17, EXP]
%   {\clist_item:Nn, \clist_item:cn, \clist_item:nn}
%   \begin{syntax}
%     \docAuxCommand*{clist_item:Nn} \meta{comma list} \Arg{integer expression}
%   \end{syntax}
%   Indexing items in the \meta{comma list} from~$1$ at the top (left), this
%   function will evaluate the \meta{integer expression} and leave the
%   appropriate item from the comma list in the input stream. If the
%   \meta{integer expression} is negative, indexing occurs from the
%   bottom (right) of the comma list. When the \meta{integer expression}
%   is larger than the number of items in the \meta{comma list} (as
%   calculated by \docAuxCommand*{clist_count:N}) then the function will expand to
%   nothing.
%   \begin{texnote}
%     The result is returned within the \tn{unexpanded}
%     primitive (\docAuxCommand*{exp_not:n}), which means that the \meta{item}
%     will not expand further when appearing in an \texttt{x}-type
%     argument expansion.
%   \end{texnote}
% \end{function}


\chapter{LaTeX3 quarks and recursion}
\label{ch:quarks}

\section{What are quarks?}
Quarks and recursion are central to the expl3 language. Quarks are a weird concept and is inherited from \tex’s way of scanning macro arguments.

But before we delve into the details of |expl3|'s quarks let us review \tex's delimited functions with an example. Consider the following example where we delimit the arguments of a macro |\test| with the control sequence |\texquark|. We do not need to define the |\texquark| and as we discussed in the section on macros it can even consist of the macro name itself. \tex will scan the input until the marker is found. It will also absorb the marker and do nothing about it.

\begin{texexample}{TeX quarks!}{}
\def\test#1\texquark{#1}
\test 123456\texquark \\
\def\test#1\test{#1}
\test 123456\test
\end{texexample}

In \latex2e macro delimiters are found all over the place, mostly in the form of \docAuxCommand{@nil}, \docAuxCommand{@nni}  or \docAuxCommand{@@}. See for example, how the \latex2e kernel defines lists.

 In \latex3 these have been termed \enquote{quarks} and \enquote{scan marks}. By convention all constants of type quark start out with |\q_| and scan marks start with |\s_|. Scan marks are reserved for internal use by the kernel and you should avoid using them in your code.\index{scan marks}\index{quarks}

They differ from the simple case above with the \tex example, in that they are used mostly indirectly. The \latex3 quarks, are defined so that they expand to themselves. As such they should never be executed directly in the code. This would cause and endless loop and cause either the program or even your computer to crash. The reason they hold a value, is that they can be tested, using |\ifx| which compares the meaning of two macros without expanding them. The equivalent construction in |expl3| is |\if_meaning:w|. We can use it at the next example. Note I gave used |\def| in th example to make it clearer, but one of course can use |\cs_set:Npn| or an equivalent function.

\begin{texexample}{Checking if is a quark}{ex:quarks}
\ExplSyntaxOn
\def\quark{\quark}
\cs_set_nopar:Npn \b {\quark}
\if_meaning:w  \quark\b
   \PASS
\else:
  \FAIL 
\fi:
\ExplSyntaxOff
\end{texexample}

This ingenious technique employed in Example~\ref{ex:quarks} depends on \tex’s ability to carry out comparisons without expanding the macros being compared. This way semantic definitions can be made for quarks and employed in generic recursive functions. 
Normally, you wouldn’t need to define your own quarks, as the ones made available by |expl3| are adequate for most tasks. If you have to create one, it can be created using:

\begin{docCommand}{quark_new:N}{ \meta{quark}}
Creates a new \meta{quark} which expands only to \meta{quark}. The \meta{quark} is defined globally, and an error message will be raised if the name was already taken.
\end{docCommand}

For example, the kernel defines two flavours of quarks to be used specifically for recursion and which we will use in the next section.

\begin{teXXX}
\quark_new:N \q_recursion_tail
\quark_new:N \q_recursion_stop
\end{teXXX}

Other flavours are lsited in the manual and summarized below:

\begin{docCommand}{q_stop}{ \meta{quark}}
Used as a marker for delimited arguments such as:
\begin{verbatim}
\cs_set:Npn \tmp:w #1#2 \q_stop {#1}
\end{verbatim}
\end{docCommand}





\section{Recursion}
 
One of the problem areas in programming recursion is to have a uniform interface to intercepting and terminating loops when one is doing recursion. \latex3 provides the building blocks.

First let us see an example:

\begin{texexample}{Recursion}{ex:l3recursion}
\ExplSyntaxOn

\cs_new:Npn \__my_decoration_fn:nn #1  {
  \str_if_eq:nnTF{e}{#1}
    {[{\bfseries\color{red}#1}]}
    {[#1]}
}

\cs_new:Npn \mymain #1 
{
      \__my_map:n #1 \q_recursion_tail\q_recursion_stop
}

\cs_new:Npn \__my_map:n #1 
  {
    \quark_if_recursion_tail_stop:n {#1}
    \__my_decoration_fn:nn  {#1} 
    \__my_map:n
  }
\ExplSyntaxOff
 
\mymain {abcdefgh}
\end{texexample}

The main function, will first call a mapping function leaving in the stream the following:
\medskip

\texttt{bcdefgh} {\hl{\textbackslash q\_recursion\_tail} \hl{\textbackslash q\_recursion\_stop}}
\medskip

On the second iteration the stream will be reduced by one token (b) and the remaing value will be:
\medskip

\texttt{cdefgh} {\hl{\textbackslash q\_recursion\_tail} \hl{\textbackslash q\_recursion\_stop}}
\medskip

This is repeated, until the quark is captured which causes the recursion to terminate. The termination is achieved by
the macro |\quark_if_recursion_tail_stop:n|. This will also absorb the |\_recursion_stop| quark. 

While the function is recursing we send the captured letter to a function to decorate and typeset it. This function can be programmed to do whatever you want to achieve. Note in the example it can only accept one argument.

Now what happens, if you wanted to capture two letters at a time or three letters at a time? The program would have to be modified as follows:

\begin{texexample}{Recursion}{ex:l3recursion}
\ExplSyntaxOn
\cs_new:Npn \__my_second_decoration_function:nn #1#2{
   {\color{red}
   [#1#2]}  
}
\cs_set:Npn \mymainother #1
{
 
   \__my_map_other:nn #1  \q_recursion_tail\q_recursion_tail\q_recursion_stop
}

\cs_new:Npn \__my_map_other:nn #1#2
  {
    \quark_if_recursion_tail_stop:n {#1}
    \quark_if_recursion_tail_stop:n {#2}
    \__my_second_decoration_function:nn  {#1}{#2} 
    \__my_map_other:nn
  }

\ExplSyntaxOff 
 
\mymainother {abcdefgh}
\end{texexample}

What just happened, we modified the custom function to accept two arguments, as well as |\_my_map_other:nn|. I also changed  their names to avoid clashes in this document.

In the next example we will iterate through two lists recursively. The first list will provide a string, which we will have to check if it consists of valid character. The valid characters are provided by the second argument of the main macro.


\begin{texexample}{Recursion}{ex:l3recursion}
\ExplSyntaxOn
\cs_new:Npn \ylcompare #1#2
  {
     \__yl_compare_auxi:nN {#2} #1 \q_recursion_tail \q_recursion_stop
  }
  
  
\cs_new:Npn \__yl_compare_auxi:nN #1#2
  {
    \quark_if_recursion_tail_stop:N #2
    \__yl_compare_auxii:nN {#1} #2
    \__yl_compare_auxi:nN {#1}
  }
 
 
\cs_new:Npn \__yl_compare_auxii:nN #1#2
  {
    \__yl_compare_auxiii:NN #2 #1 \q_recursion_tail \q_recursion_stop
  }
\cs_new:Npn \__yl_compare_auxiii:NN #1#2
  {
  % if found not found stop and print
    \quark_if_recursion_tail_stop_do:Nn #2 { \FAIL\  #1 }
  % if not the list end  
    \str_if_eq:nnT {#1} {#2}
      {
        \use_i_delimit_by_q_recursion_stop:nw { \PASS\  #1 }
      }
  % recurse     
    \__yl_compare_auxiii:NN #1
  }
\ExplSyntaxOff

\ylcompare{1234567890AAA}{-1234567890)(}
\ylcompare{text}{abcdefghijklmnopqrst} 
\end{texexample}

How would one modify the above to provide a boolean value if the string is made up only of valid characters? For example for a vowel or alphabet string. This is easy as we can define a boolean, so instead of printing the assertion we would set the boolean at false if it fails. For a number proving string, our method will fail, as we need to test for cases such as |-12345-567|, which is not a valid string also we need to think if we want to allow any spaces. This would probably have to be programmed as a special macro.

\section{Lower level functions}

As we have seen in the section for \tex iteration, one can build almost anything given patience and skills. Many examples can be found in the |expl3| package |fp|. Example~\ref{ex:fp1} is taken from the |fp| package and is a macro to trim leading zeros from a token representing a real number. All the |\@@_| are used in packages to add a prefix when processed through the doc/docstrip system, in this case it will add |fp_|.

\begin{texexample}{Weirds}{ex:fp1}
\makeatletter
\ExplSyntaxOn
 \cs_new:Npn \@@_trim_zeros:w #1 ;
  {
    \@@_trim_zeros_loop:w #1
      ; \@@_trim_zeros_loop:w 0; \@@_trim_zeros_dot:w .; \s__stop
  }
  
\cs_new:Npn \@@_trim_zeros_loop:w #1 0; #2 { #2 #1 ; #2 }

\cs_new:Npn \@@_trim_zeros_dot:w #1 .; { \@@_trim_zeros_end:w #1 ; }

\cs_new:Npn \@@_trim_zeros_end:w #1 ; #2 \s__stop { #1 }
 
 
\@@_trim_zeros:w  121200010.000; 
\ExplSyntaxOff
\end{texexample}


I have removed the |@@_| and replaced them with the |fp_| prefix to make the code more concise and readable.
The main function is delimited with a semi-colon |;| delimited function. Within the macro this is passed onto
|\fp_trim_zeros_loop:w| for further processing. 

\begin{texexample}{Weird}{ex:fp1}
\makeatletter
\ExplSyntaxOn

 \cs_new:Npn \fp_trim_zeros:w #1 ;
  {
    \fp_trim_zeros_loop:w #1;\fp_trim_zeros_loop:w 0; \fp_trim_zeros_dot:w .; \s__stop
  }
  
\cs_new:Npn \fp_trim_zeros_loop:w #1 0; #2 { #2 #1 ; #2 }

\cs_new:Npn \fp_trim_zeros_dot:w #1 .; { \fp_trim_zeros_end:w #1 ; }

\cs_new:Npn \fp_trim_zeros_end:w #1 ; #2 \s__stop { #1 }

 
\fp_trim_zeros:w  131.200010000 ; 
 \ExplSyntaxOff
\end{texexample}

This function is looking for two variables |#1 0; #2| It will scan until its end and then rescan again. The secon time it will absorb |\fp_trim_zeros_dot:w| as its second argument and then continue expanding this function.

\begin{teXXX}
\fp_trim_zeros_loop:w #1;\fp_trim_zeros_loop:w 0; {second macro}
\end{teXXX} 

Weird but wonderful functional programming.

\section{Summary}

This has brought us to almost the end of the |expl3| structures and language. There is much more to cover, but once you become proficient with the syntax and basic usage of its modules, you can pick up the rest through the documentation. 









\chapter{The LaTeX3 l3token package}
\label{ch:l3token}

The \tex concept of tokens is central to its operation. In earlier chapters we discussed extensively the use of category codes and other important aspects of \tex’s tokens. Rememeber a \tex token is either a single character or a control sequence such as a the control sequence |\test|.

\begin{texexample}{makeatletter}{}
\ExplSyntaxOn
\group_begin:
\char_set_catcode_letter:N @
\char_set_catcode_letter:N 1
\def\@store1a{AAAA}
\@store1a\\
\token_to_meaning:N @\\
\token_to_meaning:N 1\\
\char_set_catcode_other:N @
\char_set_catcode_other:N 1
\token_to_meaning:N @\\
\token_to_meaning:N 1\\
\group_end:
\ExplSyntaxOff
\end{texexample}

There are sixteen different commands to set the catcode to any of the predefined groups used by \tex. If you cannot remember the catcode number for a tab character, try and remember its command!

\begin{verbatim}
\char_set_catcode_escape:N 
\char_set_catcode_group_begin:N
\char_set_catcode_group_end:N
\char_set_catcode_math_toggle:N
\char_set_catcode_alignment:N
\char_set_catcode_end_line:N
\char_set_catcode_parameter:N
\char_set_catcode_math_superscript:N
\char_set_catcode_math_subscript:N
\char_set_catcode_ignore:N
\char_set_catcode_space:N
\char_set_catcode_letter:N
\char_set_catcode_other:N
\char_set_catcode_active:N
\char_set_catcode_comment:N
\char_set_catcode_invalid:N
\end{verbatim}

\section{Token predicate functions}

\begin{docCommand}{token_if_macro:NTF} { \meta{token} \marg{true code} \marg{false code}}
tests if the \meta{token} is a \tex macro.
\end{docCommand}

\begin{texexample}{Test if is a macro}{}
\ExplSyntaxOn
\csname sometest\endcsname
\expandafter\def\csname sometesti\endcsname{}
\token_if_macro:NTF \par { \PASS } { \FAIL }
\token_if_macro:NTF \minipage { \PASS } {\FAIL }
\token_if_macro:NTF \sometest { \PASS } {\FAIL }
\token_if_macro:NTF \sometesti { \PASS } {\FAIL }
\par
\meaning\sometest
\ExplSyntaxOff
\end{texexample}

 If it is a primitive we can find out, using yet another boolean construction \docAuxCommand*{token_if_primitive:NTF}  We can also check its meaning. It is interesting to note that \docAuxCommand*{par} is not a macro. Interestingly we can view what \tex does when we say |\csname somecs\endcsname|. It justs sets it equal to |\relax|. 
 
 Again this is important in parsing and in automating the generation of commands. For example  in the |phd| package, we allow for a key value to be entered either as a control sequence for example, |\Large| or simply as a |large|. A test could be provided before further processing such type of input.

\begin{texexample}{Test if is a macro}{ex:ifprimitive1}
\ExplSyntaxOn
\token_to_meaning:N \par\\
\token_to_meaning:N \toks
\token_if_primitive:NTF \par { \PASS } { \FAIL }\\
\ExplSyntaxOff
\end{texexample}

Example~\ref{ex:ifprimitive1} can be used to test if a primitive has been redefined (this can be important for your code and to restore its meaning if necessary or issue an error message.  Another test which is available is to check if a token is a macro. 

%% Dangerous??
\begin{texexample}{Test if is a macro}{ex:active}
\ExplSyntaxOn
\group_begin:
\token_if_cs:NTF \char_set_catcode_active:N  { \PASS } { \FAIL }
\group_end:
\ExplSyntaxOff
\end{texexample}

The next set of available commands are helper functions equivalent to the output of |\ifcat| 

\begin{docCommand} {token_if_group_begin:NTF} {\meta{token} \marg{true code} \marg{false code}}
Tests if \meta{token} has the category code of a begin group token (\{) when normal TEX
category codes are in force). Note that an explicit begin group token cannot be tested in
this way, as it is not a valid N-type argument. To test it you have to use |\c_group_begin_token|. This is mostly
used in conjuction with |futurelet| type constructions and or parsing.
\end{docCommand}


\begin{texexample} {Test if group begin} {ex:ifgroubbegin}
\ExplSyntaxOn
 \token_if_group_begin:NTF \c_group_begin_token { \PASS }{ \FAIL }
 \token_if_group_end:NTF    \c_group_end_token { \PASS }{ \FAIL }\par
 \the\catcode`{
\ExplSyntaxOff
\end{texexample}

Behind the scenes |expl3| uses the |\ifcat| primitive to test the token against the catcode values. Constructions for all categories are available and summarized in the test below rather than described.
\begin{texexample} {Test if group begin} {ex:ifgroubbegin}
\ExplSyntaxOn
 \token_if_group_begin:NTF \c_group_begin_token { \PASS }{ \FAIL }
 \token_if_group_end:NTF    \c_group_end_token { \PASS }{ \FAIL }\par
 \token_if_alignment:NTF     \c_alignment_token { \PASS }{ \FAIL }\par
 \token_if_parameter:NTF    \c_parameter_token { \PASS }{ \FAIL }\par
\ExplSyntaxOff
\end{texexample}


\section{LaTeX3 Futurelet type functions}

In Chapter Futurelet, we spend considerable effort to understand how \tex’s futurelet macro works. There is often a need to look ahead at the next token in the input stream while leaving
it in place. This is handled using the “peek” functions. The generic \docAuxCommand*{peek_after:Nw} is
provided along with a family of predefined tests for common cases. As peeking ahead does
not skip spaces the predefined tests include both a space-respecting and space-skipping
version.

\begin{texexample}{Peek ahead ignoring spaces} {}
\ExplSyntaxOn
\peek_catcode_remove_ignore_spaces:NTF =  
    { 
      \PASS  
      \token_if_letter:NTF
          {l_peek_token ~= ~\token_to_meaning:N \l_peek_token \\  } 
          {   }
    } 
    { \FAIL }  
 = abcde \\
\ExplSyntaxOff
\end{texexample}

Most applications would require to recursively pick up tokens from the input stream and only terminated once a special token is found. This is the most powerful method to parse input strings and create really powerful functions. 

You will understand better if we hide the code in a function.

\begin{texexample}{Peek ahead ignoring spaces} {ex}
\ExplSyntaxOn
\cs_new:Npn \checkletter #1 {
\peek_catcode_remove_ignore_spaces:NTF #1  
    { 
      \PASS  
      \token_if_letter:NTF
          {l_peek_token ~= ~\token_to_meaning:N \l_peek_token \\  } 
          {   }
    } 
    { \FAIL } }

\checkletter {=} =abcde \par
\checkletter {A} Abcde \par
\ExplSyntaxOff
\end{texexample}

\begin{texexample}{Peek ahead ignoring spaces} {}
\ExplSyntaxOn
\cs_set:Npn \check_letter_and_removeall #1 {
\peek_catcode_remove_ignore_spaces:NTF #1  
    { 
      \PASS  
      \removeallaux:w  
    } 
   { \FAIL } 
 }

\cs_set:Npn \removeallaux:w #1; { removed~#1~ }

\check_letter_and_removeall {W}  W 12pt; \par
\ExplSyntaxOff
\end{texexample}


\begin{texexample}{ex:recursivefl}  { }                            
\lorem 
\ExplSyntaxOn
\def\recurse {
   \peek_catcode_remove_ignore_spaces:NTF ; 
   {TRUE} 
   {
     \recurse
   } abcdouiop;
}
\ExplSyntaxOff
\end{texexample}
















\chapter{The xtemplate package of LaTeX3 and how to use it effectively}

Back in 1999 Frank Mittelbach together with David Carlisle and Chris Rowley published a paper in TUGboat describing their ideas of  \enquote{New Interfaces for \latex Class Design.} 
 
 \begin{latexquote}
 Traditional \latex class files typically implement one
fixed design via ad hoc, and often low-level, \latex
code. This style of implementation makes it much
harder than is either desirable or necessary to produce
classes that implement a specific visual design.
Moreover, the construction of such classes typically
involves a lot of work that is essentially programming
and thus does not live easily with the declarative
kind of design specification for a document (or
range of documents) that would be produced by a
professional typographic designer.
\end{latexquote}

The \emph{declarative kind} of design specification for a document, mentioned by the authors has been the holy grail of \latex for sometime. With the proliferation of key value packages it came closer to fruition and my own work in the |phd| package had this goal as one of its primary objectives. The \pkgname{xtemplate} is at a much lower level than the phd package and I have struggled in my head as to how to integrate the two, so far unsuccessfully. There are very few articles on |xtemplate| but a good introductory one is \emph{Some notes on templates} by  Lars Hellström’s and which was published in TUGboat. The \pkgname{xgalley} still under develpment makes use of templates extensively and is worth to have a good look at the code.

\section{Objects, templates and instances}

\subsection{Object types}

An \emph{object type} sometimes termed \enquote{object} is an abstract idea of a document element that has a fixed number of arguments corresponding to the information from the document author that it is representing.  A sectioning object, for example, might take three inputs: \enquote{title}, \enquote{short title}, and \enquote{label}.

\begin{docCommand} {DeclareObjectType} { \meta{object type} \meta{no of args}}
This function defines an \meta{object type} taking \meta{number of arguments}, where the \meta{object type} is an abstraction as discussed above. For example:
   \begin{verbatim}
     \DeclareObjectType{chapter}{3}
   \end{verbatim}
This would create an object type \enquote{sectioning}, where each use of that object type will need three arguments.   
\end{docCommand}

The object type doesn’t do much when it is declared. It just records the name and the number of arguments in a property store, as can be seen in the code below, extracted from the |xtemplate| package:

\begin{teXXX}
\cs_new_protected:Npn \@@_declare_object_type:nn #1#2
  {
    \int_set:Nn \l_@@_tmp_int {#2}
    \bool_if:nTF
      {
        \int_compare_p:nNn {#2} > \c_nine ||
        \int_compare_p:nNn {#2} < \c_zero
      }
      {
        \msg_error:nnxx { xtemplate } { bad-number-of-arguments }
          {#1} { \exp_not:V \l_@@_tmp_int }
      }
      {
        \msg_info:nnxx { xtemplate } { declare-object-type }
          {#1} {#2}
        \prop_gput:NnV \g_@@_object_type_prop {#1}
          \l_@@_tmp_int
      }
  }
\end{teXXX}

    
\subsection{Templates}

Once an object is created a \emph{template} can be used to generalize a design solution for representing the information of a specific object type. A template has a name and a parent object. There are two important parts to a template:

\begin{enumerate}
\item The parameters it takes to vary the design it is producing.
\item The implementation of the design.
\end{enumerate}

The template definition is split into two parts using \cs{DeclareTemplateInterface} and \cs{DeclareTemplateCode}.
We will first examine \docAuxCommand*{DeclareTemplateInterface}.

\begin{docCommand} {DeclareTemplateInterface} {\meta{object} \marg{key value list}}
The key value list is of the form:
\begin{verbatim}

    key1 : key type1,
    key2 : key type2,
    key3 : key type3  = default3,
    key4 : key type4  = default4,
\end{verbatim}

An important item to note is that spaces in key names are ignored so writing |my key| and |mykey| is one and the same. 

Essentially the |DeclareTemplateInterface| is a command that initializes the list of key values applicable to the object type. The key list must be the same as declared for |object|. I didn’t know the \latex guys were fans of Java. The code pattern here is very similar. You declare an object and then its interface. Once this is done then the code can be developed. The key types available are shown in Table~\ref{tab:key-types}, which has been extracted from the documentation.

\end{docCommand}

\begin{docKey}{boolean} { boolean type for template interface}{\meta{true or false}}
a true or false value
\end{docKey}

   \begin{table}
     \centering
     \begin{tabular}{>{\ttfamily}ll}
       \toprule
       \multicolumn{1}{l}{Key-type} & Description of input \\
       \midrule
       boolean    & \texttt{true} or \texttt{false}            \\
       choice\marg{choices}
         & A list of pre-defined \meta{choices} \\
       code
         & Generalised key type: use |#1| as the input to the key \\
       commalist  & A comma-separated list                        \\
       function\marg{$N$}
         & A function definition with $N$ arguments
          ($N$ from $0$ to $9$) \\
       instance\marg{name}
                      & An instance of type \meta{name} \\
       integer    & An integer or integer expression            \\
       length     & A fixed length                              \\
       muskip    & A math length with shrink and stretch components \\
       real         & A real (floating point) value               \\
       skip         & A length with shrink and stretch components \\
       tokenlist  & A token list: any text or commands          \\
       \bottomrule
     \end{tabular}
     \caption{Key-types for defining template interfaces with
       \cs{DeclareTemplateInterface}.}
     \label{tab:key-types}
   \end{table}
   
\begin{texexample}{xtemplate short example}{}
\DeclareObjectType{obj}{0}
\DeclareTemplateInterface{obj}{tmpt1}{0}
{
  section-name: tokenlist = section,
  section-numbering: tokenlist =Roman,
  section-color: tokenlist = blue,
}
\end{texexample}

The |\DeclareTemplateInterface| part of the code is just a macro, whose fourth argument is written in a funny way.
We could have just written it as:

\begin{teXXX}
\DeclareTemplateInterface{obj}{tmp1}{0}{section-numbering:tokenlist=arabic, ... }
\end{teXXX}

A confusing aspect of the |templates| package is how the code part is defined. Here for each
key declared in the \docAuxCommand{DeclareTemplateInterface} you will need to allocate it an appropriate
macro. This works like in normal \latex keys. 

\begin{texexample}{The template code}{}
\ExplSyntaxOn
\DeclareTemplateCode{obj}{tmpt1}{0}
{
  section-name         = \sectionname,
  section-numbering = \numberingtype, 
  section-color = \colorname,
  }
{
% the implementation part
\AssignTemplateKeys
  \sectionname\ ~ 
  {\cs:w\numberingtype\cs_end: {section}\scan_stop:}\\
}

\DeclareInstance {obj}{inst} {tmpt1}{section-numbering = roman}
\UseInstance{obj}{inst}
\DeclareInstance {obj}{inst2}{tmpt1}{section-name=SECTION,
                                                      section-numbering = arabic}
\UseInstance{obj}{inst2}
\ExplSyntaxOff
\end{texexample}

The implementation part is the part that starts with |\AssignTemplateKeys|. Here we can use the values stored in the key functions to do something useful. Again here, remember, we are using macros and |\AssignTemplateKeys| is a macro with five arguments. This is defined by the package as:

 \begin{verbatim}
   \@@_declare_template_code:nnnnn {#1} {#2} {#3} {#4} {#5}
\end{verbatim}

Looking back at our simple example, the formatting of the section number in |arabic| or |roman| did not make any particular checks for validity. This would have been better programmed as a |choice| key with all the choice words allowed hardcoded in the implementation part. 

\begin{teXXX}
 section-numbering  : choice { arabic, Roman, roman, words, Words, alph, Alph } = arabic
\end{teXXX}                                 

The |choice| key type implements multiple choice input. At the interface level only the list of valid choice is needed:

\begin{teXXX}
\DeclareTemplateInterface{ foo }{ bar }{ 0 }
    { key-name : choice { A, B, C } }
\end{teXXX}

Note that the choices are given in a comma delimited list (which must therefore be wrapped in braces). A default value can also be given:


\begin{teXXX}
\DeclareTemplateInterface{ foo }{ bar }{ 0 }
    { key-name : choice { A, B, C } = A }
\end{teXXX}

\begin{teXXX}
 section-numbering      =
      {
        roman =
          \cs_set_nopar:Npn \numberingtype:
            {
              ... code
            },
        roman  =
          \cs_set_nopar:Npn \numberingtype:
            {
              ... code 
            }
      },
\end{teXXX}

\begin{texexample}{The template code}{ex:unknownkey}
\ExplSyntaxOn
\DeclareTemplateInterface{obj}{section}{0}
{
  section-name: tokenlist = section,
  section-numbering: choice  {arabic, Roman, roman}=roman,
  section-color: tokenlist = blue,
}
\DeclareTemplateCode{obj}{section}{0}
{
  section-name         = \sectionname,
  section-numbering = 
     {
       roman     =   \cs_set_nopar:Npn  \numberingtypei: { \roman{section} \scan_stop: },
       Roman     =  \cs_set_nopar:Npn   \numberingtypei: { \Roman{section} \scan_stop: },
       arabic      =   \cs_set_nopar:Npn  \numberingtypei: { \arabic{section} \scan_stop: },
       unknown =   \cs_set_nopar:Npn  \numberingtypei: { ERROR~unknown~key }
     },  
  section-color = \colorname,
  }
{
% the implementation part

\AssignTemplateKeys
  \sectionname\ ~ 
  \numberingtypei: \par
}

\DeclareInstance {obj}{inst} {section}{section-numbering = roman}
\UseInstance{obj}{inst}
\DeclareInstance {obj}{inst2}{section}
    {
        section-name=SECTION,
        section-numbering = arabic
     }
     
\DeclareInstance {obj}{inst3}{section}
    {
        section-name=SECTION,
        section-numbering = Arabic
     }  
                                                          
\UseInstance{obj}{inst2}
\UseInstance{obj}{inst3}
\ExplSyntaxOff
\meaning\numberingtypei
\end{texexample} 

In Example~\ref{ex:unknownkey} we have introduced the |choice| type key. This also takes an option
\option{unknown}. If a value is given that has not been previously been defined, then it essentially acts as
an |else| branch to the code and executes the definition given, in our example just typesets an 
error message. The code in the example at this stage is very simplistic and it has not been abstracted properly. The example is simply here to demonstrate the various types of keys available. The |length| and the |skip| keys 
accept dimensions or skips and are simply coded. The |function| type of key can be very useful in many situations. 

In the next example we will add some skips before and after the section, as well introduce a boolean to choose betwen a block heading or an inline heading. 


\begin{texexample}{The template code}{ex:unknownkey}
\ExplSyntaxOn
\DeclareTemplateInterface{obj}{headings}{0}
{
  name: tokenlist = section,
  numbering: choice  {arabic, Roman, roman,none} = roman,
  color: tokenlist = blue,
  display: boolean = true,
  aboveskip: skip=10pt,
  belowskip: skip=10pt,
  }
 
\bool_new:N \l_display_bool

\DeclareTemplateCode{obj}{headings}{0}
{
  name         = \sectionname,
  numbering = 
     {
       roman     =  \cs_set_nopar:Npn \numberingtypei: {\roman{section}},
       Roman     = \cs_set_nopar:Npn \numberingtypei: {\Roman{section} },
       arabic      =  \cs_set_nopar:Npn \numberingtypei: {\arabic{section}},
       none       =  \cs_set_nopar:Npn \numberingtypei: {},
       unknown =  \cs_set_nopar:Npn\numberingtypei: {ERROR~unknown~key }
     },  
  color = \colorname,
  display = \l_display_bool,
  aboveskip = \l_tmpa_skip,
  belowskip = \l_tmpb_skip,
 }
 {
% the implementation part
  \AssignTemplateKeys
  \par\skip_vertical:N  \l_tmpa_skip
  \sectionname\ ~ 
  \numberingtypei: \par
}

\DeclareInstance {obj}{part} {headings}
  { name      = PART,
    numbering = Roman
  }
  
\DeclareInstance {obj}{section}{headings}
  { name      = SECTION,
    numbering = arabic
  }
  
\DeclareInstance {obj}{chapter}{headings}
  {
    name      = CHAPTER,
    numbering = Roman, 
    aboveskip = 5pt
  }    
                                                                                                                   
\UseInstance{obj}{part}
\UseInstance{obj}{section}
\UseInstance{obj}{chapter}

\ExplSyntaxOff

\end{texexample}   

Although the |xtemplate| manual recommends that booleans should be preferred over 
|choice| keys, but from a user interface point of view |choice| keys are more powerful. One can define
key variations such as (true, false, on, off, none) and other similar values. 

Another few notes for readers coming from \latexe. The  |\skip_vertical:N| is the
\tex |\vskip|.  There are also some questions arising from the approach, which can affect the
coding. The format and the flexibility of the final settings offered for the user. From a programmer’s
perspective the view is different.  We could view the three basic elements of a heading at a more
elementary level, consisting of a number, a label and a title. Consider the |HTML| element |<span>|,
how can we make an equivalent in \latex3? There are many approaches one could think of, but this time
having covered the basics of how to program templates, we will start from the Designer Level. The Designer
wishes to define commands that are normally used inline and are used for different type of purposes. Such functions can typeset words that are emphasized, others that represent computer code and are typeset verbatim, acronyms and abbreviations. These also can automatically add themselves to an index etc.

Of course we don’t want to offer the user a command called |\span| where he needs to type |\span[emph]|. What we want to offer the user is a series of commands. However at the Design Level, these can be created by means of templates.

\begin{texexample}{A template for spans}{ex:span}
\ExplSyntaxOn
\DeclareObjectType{inlineobj}{1}
\DeclareTemplateInterface{inlineobj}{span}{1}
{
  font-face: tokenlist,
  font-shape: choice {italic, slanted, normal},
  font-weight: choice {bold, normal},
  font-color: tokenlist,
  quote: function 1,
}
\cs_set_nopar:Npn \quote_format:n#1 {\enquote{#1}}
\cs_set_nopar:Npn \quote_format_none:n#1 {#1}

\DeclareTemplateCode{inlineobj}{span}{1}
{
  font-face         =  \l_font_tl,
  font-shape = {
     italic     = \cs_set_nopar:Nn \afontshape: {\itshape},
     slanted = \cs_set_nopar:Nn \afontshape: {\itshape},
     normal = \cs_set_nopar:Nn \afontshape: {\upshape}
  },
  font-weight = {
     bold    = \cs_set_nopar:Nn \afontseries: {\bfseries},
     normal =\cs_set_nopar:Nn \afontseries: {\mdseries}
   },
  font-color = \l_tmpa_tl,  
  quote = \quote_format:n,
}
{
% the implementation part
  \AssignTemplateKeys
  \group_begin:
  \color\l_tmpa_tl
   \cs:w \l_font_tl \cs_end: 
   \afontshape:
   \afontseries: 
       \quote_format:n{\detokenize{#1}} 
   \group_end:
 }
 
\ExplSyntaxOff

\DeclareInstance {inlineobj}{docFunction}{span}
    {
        font-face=arial,
        font-shape=normal, 
        font-weight=bold,
        font-color=green!40!black
     }

\DeclareDocumentCommand\docFunction{ m }{
   \IfInstanceExistTF {inlineobj}{docFunction} 
     {\UseInstance{inlineobj}{docFunction}{#1}}
     {ERROR                                                   }
}

\DeclareInstance {inlineobj}{tn}{span}
    {
        font-face=ttfamily,
        font-shape=normal, 
        font-weight=normal,
        font-color=green!40!black
     }

\DeclareDocumentCommand\tn{ m }{
   \IfInstanceExistTF {inlineobj}{tn} 
     {\UseInstance{inlineobj}{tn}{#1}}
     {ERROR                                                   }
} 
   
The function \docFunction {get_string ( )} is used throughout to get a string in LuaTeX, where macros in text paragraphs are shown as \docFunction\mymacro in green.
\end{texexample}
\ExplSyntaxOn
\DeclareObjectType{inlineobj}{1}
\DeclareTemplateInterface{inlineobj}{span}{1}
{
  font-face: tokenlist,
  font-shape: choice {italic, slanted, normal},
  font-weight: choice {bold, normal},
  font-color: tokenlist,
  quote: function 1,
}
\cs_set_nopar:Npn \quote_format:n#1 {\enquote{#1}}

\DeclareTemplateCode{inlineobj}{span}{1}
{
  font-face         =  \l_font_tl,
  font-shape = {
     italic     = \cs_set_nopar:Nn \afontshape: {\itshape},
     slanted = \cs_set_nopar:Nn \afontshape: {\itshape},
     normal = \cs_set_nopar:Nn \afontshape: {\upshape}
  },
  font-weight = {
     bold    = \cs_set_nopar:Nn \afontseries: {\bfseries},
     normal =\cs_set_nopar:Nn \afontseries: {\mdseries}
   },
  font-color = \l_tmpa_tl,  
  quote = \quote_format:n,
}
{
% the implementation part
  \AssignTemplateKeys
  \group_begin:
  \color\l_tmpa_tl
   \cs:w \l_font_tl \cs_end: 
   \afontshape:
   \afontseries: 
       \quote_format:n{\detokenize{#1}} 
   \group_end:
 }
 
\ExplSyntaxOff

\DeclareInstance {inlineobj}{docFunction}{span}
    {
        font-face=arial,
        font-shape=normal, 
        font-weight=bold,
        font-color=green!40!black
     }

\DeclareDocumentCommand\docFunction{ m }{
   \IfInstanceExistTF {inlineobj}{docFunction} 
     {\UseInstance{inlineobj}{docFunction}{#1}}
     {ERROR}
}

\DeclareInstance {inlineobj}{tn}{span}
    {
        font-face=ttfamily,
        font-shape=normal, 
        font-weight=normal,
        font-color=green!40!black
     }

\DeclareDocumentCommand\tn{ m }{
   \IfInstanceExistTF {inlineobj}{tn} 
     {\UseInstance{inlineobj}{tn}{#1}}
     {ERROR}
} 
   
With the last example I have introduced also the conditional \docAuxCommand*{IfInstanceTF} that provides a test if the template exist. In our case typesets |ERROR| if the instance does not exist.

\begin{docCommand}{IfInstanceExistTF}{\marg{object type} \marg{instance} \marg{true code} \marg{false code}}
Tests if the named \meta{instance} of an \meta{object type} exists, and then inserts the appropriate code into the input stream. 
\end{docCommand}


\section{Summary}

\latex3’s \pkgname{xtemplate} offers a flexible and robust way to enable  declarative
setting of typographical parameters for a document. For the package writer it has one major advantage. It can be used to expose an API through which users communicate with the package's important commands. I would go as far as to say that packages should only expose an API and no settings should occur during loading. This can reduce both errors during package loading with different key values, as well as perhaps stop the race at the |AtBeginDocument|. 

If you want to study a longer non-trivial example you can have a look at the \pkgname{xfrac} package. In this package Will Robertson used |xtemplate| extensively. He also used some of the more esoteric commands of the package and is worth studying the code, before you start using |xtemplate| in your package.

All the functionality made available by the package can easily be provided by |pgfkeys| and the creation of some custom commands. This will remain as a competitor to the package until some of the limitations of |xparse| are addressed. The main limitation currently from my point of view is the addition of custom types  in a similar fashion to pgfkeys \emph{handlers}, although the \tn{code} and the \tn{function} types can be used in this respect.




\input
\DocInput{\jobname.dtx}
\bibliography{phd} 
\printindex
 %
% 
\end{document}
%</driver>
% \fi
% 
%  \CheckSum{0}
%  \CharacterTable
%  {Upper-case    \A\B\C\D\E\F\G\H\I\J\K\L\M\N\O\P\Q\R\S\T\U\V\W\X\Y\Z
%   Lower-case    \a\b\c\d\e\f\g\h\i\j\k\l\m\n\o\p\q\r\s\t\u\v\w\x\y\z
%   Digits        \0\1\2\3\4\5\6\7\8\9
%   Exclamation   \!     Double quote  \"     Hash (number) \#
%   Dollar        \$     Percent       \%     Ampersand     \&
%   Acute accent  \'     Left paren    \(     Right paren   \)
%   Asterisk      \*     Plus          \+     Comma         \,
%   Minus         \-     Point         \.     Solidus       \/
%   Colon         \:     Semicolon     \;     Less than     \<
%   Equals        \=     Greater than  \>     Question mark \?
%   Commercial at \@     Left bracket  \[     Backslash     \\
%   Right bracket \]     Circumflex    \^     Underscore    \_
%   Grave accent  \`     Left brace    \{     Vertical bar  \|
%   Right brace   \}     Tilde         \~}
%
%
%
% \changes{1.0}{2013/01/26}{Converted to DTX file}
%
% \DoNotIndex{\newcommand,\newenvironment}
% \GetFileInfo{phd.dtx}
% 
%  \def\fileversion{v1.0}          
%  \def\filedate{2012/03/06}
% \title{The \textsf{phd} package.
% \thanks{This
%        file (\texttt{phd.dtx}) has version number \fileversion, last revised
%        \filedate.}
% }
% \author{Dr. Yiannis Lazarides \\ \url{yannislaz@gmail.com}}
% \date{\filedate}
% 
% ^^A\maketitle
% 
% ^^A\frontmatter
%  ^^A\coverpage{./images/hine02.jpg}{Book Design }{Camel Press}{}{}
%  \newpage
% ^^A\secondpage
%
% \raggedbottom
%  \OnlyDescription
%
%  ^^A\StopEventually{\printindex}

% \CodelineNumbered
% \pagestyle{headings}
% 
% 
% ^^A\part{IMPLEMENTATION AND FRIENDS}
% 
%
% \chapter{Counters Package Code Implementation Objectives and Strategy}
% 
% \epigraph{
% I was reflecting on the convoluted Java frameworks widely adopted at work. Those hefty frameworks brought coding structures and conventions to large engineering teams; meanwhile, they also sucked the fun of programming like a Pastafarian monster slurping all the tomato sauce on a plate of spaghetti.
%}{\href{http://blog.zmxv.com/2015/07/code-golf-at-google.html}{Zhen Wang}}
%
% We start by outlining what we are trying to achieve with this package:
%
% \begin{enumerate}
%
% \item To provide an expl3 interface to using integers as counters for
%       headings etc, without the need to resort to \latex2e counters.
%
% \item To provide a declarative interface to enable users to use counters
%       with headings and other book elements (interface to \pkgname{phd}).

% \item To provide an interface to packages handling localizations and internationalization.
%        This should provide facilities for representing integers as cardinals, ordinals
%        in a particular language.
%
% \item To provide a compatibility mode, where documents wishing to test the package
% can have an easy switch to switch in and out. This is also important for the testing of the package.
%
% \item To provide easy means of getting the total of standard or custom counters, 
%  such as last page, last section
%        etc.
% \item To provide a plug-in architecture for extensions.

% \end{enumerate}
% 
% \section{Terminology}
%
%  \begin{description}
%  \item [document] Any written item, as a book, article, or letter, especially 
%                  of a factual or informative nature.
%  \item [heading] A division of a document or document series. For a normal
%        book headings are chapters, sections etc. However we allow for
%        specifying a more complex document divided into books, volumes
%        parts etc. For example the Bible has Books, chapters and verses,
%        where a legal document might require divisions such as clauses.
%        In general these divisions are numbered. These document divisions
%        are stored in the comma list \refCom{phd_book_divisions_clist}.
%  \item [head] A typeset heading, such as chapter head, or section head.
%        This can include a counter, label and title for example, 
%        \emph{Chapter 1 Introduction}.
%  \item [dom] This is a programming interface that provides a structured
%        representation of the document (a tree) and it defines a way
%        that the structure can be accessed. Although \latexe does not
%        offer a standard way to build such a tree (mainly because
%        \tex does not require the marking of paragraphs, it is 
%        useful to think of the document as a tree structure. We also
%        allow for a semi-automated way to build such a tree (with the 
%        exception that paragraphs are not included).
% \item [element] A part of the document tree that can be styled on
%       its own. For example the chapter label, or the section number.
%
% \end{description}
%
% \section{Users}
%  We classify users according to the \LaTeX3 terminology as a) programmers b) template designers
%  and c) authors.
% \subsection{Author}
%  We assume that the author has an exising template which she is using but might want to do
%  some minor modifications, for example use an italic shape for the font of the mark, but an 
%  upright font for the page numbers. 
%
% {\obeylines 
%~~ |\cxset|
%~~~~~|{|
%~~~~~~~~\textit{chapter number color}~~|format          = apa,|
%~~~~~~~~\textit{section title font-size} |font-size   = Large,|
%~~~~~|}|
%}  
%
% We follow the idea of representing the basic elements of documents
% as elements, each one having a parent in order to specify
% the element we need to style as accurate as possible. One can think of
% this approach being congruent with objects in other languages.
% As a matter fact nothing stops us from defining a key value
% interface as shown below.
%
% {\obeylines 
%~~ |\cxset|
%~~~~~|{| 
%~~~~~~~~\textit{header.even.mark.font.size}   = |Large,|
%~~~~~~~~\textit{header.even.mark.font.family} = |serif,|
%~~~~~|}|
%}  
%
% This would pehaps make it easier for the template designer, but I have rejected
% the idea as my aim is to make it easy for the author, who can search the template
% and just enter a couple of new proerty values.
%
% \subsection{Template designer}
% \pagestyle{headings}
% The template designer in the example above would have selected the format style
% from a number of predefined formats (templates) or would have created a style
% called \textit{apa} from an existing template and modified it using declarative
% key style.
%
% \subsection{The programmer}
%
% The programmer in the example above could have created the basic format
% \textit{apa} by using both declarative as well as defining or using existing
% macros. To the programmer we offer an extension mechanism, where the contents
% of a |ps@| command are defined. For example the programmer can define a new
% style using \tikzname, but without having to worry about defining full |ps@|
% and their interface.
%
% \section{State of the art}
%
% \subsection{Resetting and adding or removing to reset}
%
% Counters can be reset to an initial value when another counter is reset. For
% example a section is reset when a chapter counter is incremented.
%
% \subsection{Determining the total of a counter}
%
% There are numerous packages that deal with counters one way or another on \ctan.
% Packages such as \pkgname{totpages} (\citep{totpages}), \pkgname{lastpage} \citep{lastpage}, \pkgname{pageslts} \citep{pageslts}, \pkgname{count1to} \citep{count1to} and \pkgname{TotCount} \citep{totcount}
% provide user level commands for accessing the last count of either special
% counters, such as |\page| or in the case of the \pkgname{TotCount} of creating
% counters that enable the computation and display of the last counter inside a
% document (usually the maximum value of the counter). Another package that can be
% used by programmers to access, such is information is Heiko Oberdiek's package
% \pkgname{zref} \citep{zref}. I have  a preference for the last one and its modules,
% as we are using it elsewhere as well and it is likely that if you are using 
% \pkgname{hyperref} this would have been loaded automatically. 

% Referring to the total 
% number of sections, pages, citations and similar can be difficult to achieve.
% The difficulty arises when the reference is before the \emph{definition} of all
% the elements that need to be counted. In this case it requires the saving
% of the counter in an auxiliary file, in much the same way that \latexe uses for 
% headings, table of contents and citations. For example this document's last page
% is last chapter is \zref[chapter]{LastPage}  \zref[thepage]{LastPage}. The absolute
% total pages are \ztotpages. 
%
% \subsection{Integer representation as Alph, Ordinals, Cardinals and Internationalization}
% 
% 
% \subsection{The expl3 l3int package}
%
%
% \section{Preliminaries}
%
%  Standard file identification. We first announce the package 
%	 and require that it be used with \LaTeX2e. 
% \iffalse
%<*COUNTERS>
% \fi
%  
%    \begin{macrocode}
\NeedsTeXFormat{LaTeX2e}[1994/12/01]%
\RequirePackage[2014/05/01]{latexrelease}
\ProvidesFile{phd-lists}[2015/1/13 v1.0 less preamble (YL)]%
%    \end{macrocode}
%
% We load the \pkgname{zref} with a number of options including the 
% |user| module. This means that all the user commands will be prefixed by |z|
% to avoid name clashes with macros of the same functionality. \zlabel{zref}
%
%    \begin{macrocode}
\RequirePackage[thepage,abspage,lastpage,totpages, user,savepos,xr,pagelayout]{zref}
%    \end{macrocode}
%
% The \pkgname{zref} is a programmer's toolbox for referencing, hence we
% need to add a number of properties and macros to set it up properly
% for what we need to achieve.See \zpageref{zref}, 
% \zpageref{abspage}
%
%    \begin{macrocode}
\newcounter{foo}
\renewcommand*{\thefoo}{\Alph{foo}}
\zref@newprop{thefoo}{\thefoo}
\zref@newprop{valuefoo}{\the\value{foo}}
\zref@newprop{chapter}{\thechapter}
\zref@newprop{thepage}{\thepage}
\zref@addprops{LastPage}{thepage,abspage,thefoo,valuefoo,chapter}
\newcommand*{\foo}{%
\stepcounter{foo}%
 [Current foo: \thefoo]%
}
%    \end{macrocode}
%
% Next we define error messages. We group them here for convenience
%    \begin{macrocode}
\ExplSyntaxOn
\msg_new:nnnn { phd-counters } {counter-too-large-for-ordinal}
  { #1~is~too~large~to~be~converted~to~ordinals }
  { I~am~leaving~it~as~a~numeral.\\ ~Press~enter~to~continue.}
\msg_new:nnnn { phd-counters } { unknown-language }
  { I do not know the language tag #1. }
  { I~am~leaving~it~as~a~numeral.\\ ~Press~enter~to~continue.} 
\msg_new:nnnn { phd-counters } { counter-exists }
  { I~cannot~create~the~counter~#1. }
  { As~it~already~exists.}    
  
\ExplSyntaxOff  
%    \end{macrocode}

% Unlike \latexe and in order to enable compatibility where required,
% define a counter prefix to be used with counters. When this is set
% to |c@| the counter will be fully compatible with \latexe but be
% able also to use commands from this package. Each counter has a list
% of associated counters that are reset, when it is changed. This is
% created with using the prefix |\counter_resets_prefix:|.
%
%    \begin{macrocode}
\ExplSyntaxOn
\cs_new:Npx  \ltx_counter_prefix_tl  { c@                       }
\cs_new:Npx  \phd_counter_prefix_tl  { c@                       }
\cs_new:Npx  \counter_resets_prefix: { __counter_resets_prefix_ }
%    \end{macrocode}
%
% \begin{docCmd}{phd_create_new_counter:n} {\marg{name}}
%    Internal version of the command to create a new counter. This
%    uses the expl3 internal allocation system to avoid conflicts
%    by creating a new integer.
% \end{docCmd}
%
%    \begin{macrocode}
\cs_gset:Nn  \phd_create_new_counter:n 
  {
    \int_if_exist:cTF {\phd_counter_prefix_tl#1}{ERROR~counter~exists}
      { 
        \int_gzero_new:c {\phd_counter_prefix_tl#1}
        \seq_new:c {\counter_resets_prefix:#1}
      }
  }
%    \end{macrocode}
% 
% Next define some functions around the counter in a similar fashion
% to \latexe. We provide |the|\meta{counter name}, but for consistency
% we also provide an uppercase variant, so if the counter is named
% |exam|, |\ExamValue| holds the integer value. 
%
% 
%    \begin{macrocode}
\cs_gset:Npn\phd_make_auxiliary_functions: #1
  {
    \phd_make_aux_commands#1\scan_stop:
  }

\cs_gset:Npn \phd_make_aux_commands #1 #2\scan_stop:
  {
    \uppercase{\exp_after:wN \cs_gset:Npn \cs:w #1}#2Value\cs_end:
       {\tex_the:D\cs:w\phd_counter_prefix_tl#1#2\cs_end:\relax}
       #1#2Value
       \cs_gset:cpn {the#2} 
         {
           \tex_the:D\cs:w\phd_counter_prefix_tl#1#2\cs_end:\relax 
         }
  }
%    \end{macrocode}  
%
% The \docAuxCmd{phd_add_to_reset:nn} adds a counter to the reset of another.
% It takes two arguments and store the resets in a sequence. 
%
%    \begin{macrocode}    
\cs_gset:Nn \phd_add_to_reset:nn 
  {
    \exp_args:Nf \seq_put_left:cn {\counter_resets_prefix:#1}{#2}
   % Added~ to~the~ #1 ~ resets~ #2.~The~resets~list~is~now~
   % \seq_use:cn {\counter_resets_prefix:#1}{,}
   }
%    \end{macrocode} 
%
% \begin{docCmd} {phd_reset_counter:c} { \marg{counter name}}
%   Resets the counter to zero.
% \end{docCmd}
%
%    \begin{macrocode} 
\cs_gset:Npn \resetcounter:c #1
  {
    \int_gset:cn {\phd_counter_prefix_tl #1}{0}
  }
%    \end{macrocode}
% 
% \begin{docCmd} {phd_step_counter:c} { \marg{counter name}}
%   Steps a counter by one and resets all counters in the resets list
%   to zero.
% \end{docCmd}
%
%    \begin{macrocode}  
\cs_gset:Npn \stepcounter:c #1
  {
    \int_gincr:c {\phd_counter_prefix_tl #1}
    \seq_set_eq:Nc \tempa {__counter_resets_prefix_#1}
    \seq_map_inline:Nn \tempa {\resetcounter:c{##1}}
  }      
%    \end{macrocode}
%
%    \begin{macrocode}
\cs_gset:Npn \setcounter:c #1
  {
    \int_gset:c {\phd_counter_prefix_tl #1}
  }
%    \end{macrocode}
%
%  \begin{docCmd} {NewCounter} { \oarg{reset} \marg{counter name} }
%    Creates a new counter. This is equivalent to \latexe's \docAuxCmd*{newcounter}.
%  \end{docCmd}
%
%    \begin{macrocode}
\DeclareDocumentCommand\newCounter { o m } 
  {
    \phd_create_new_counter:n {#2}
      \IfNoValueF {#1}
        {
          \int_if_exist:cT { \counter_resets_prefix:#1 } 
            { \phd_add_to_reset:nn {#1} {#2} } 
        }    
    \phd_make_auxiliary_functions: {#2}
  }
\ExplSyntaxOff
%    \end{macrocode}
%
%  \newCounter {exam}
%  \newCounter [exam]{problem}
%  \setcounter{exam}{135}  \ExamValue

% The next sections deal with cardinal and ordinal numbers. Ordinal numbers are
% words representing position or rank in a sequential order whereas cardinal numbers
% represent quantity, so cardinal is one, two etc whereas ordinal is first, second etc.
%
% These are incomplete but fast functions suitable for Chapter headings. For a 
% more comprehensive treatment the \pkgname{fmtcount} as well as \pkgname{Polyglossia}
% or \pkgname{Babel} can be used.
%
% Start with the mixed case, as it is easier and safer to convert it afterwards to
% uppercase or lowercase words.
%
% \begin{docCmd}{int_to_mixedcase_cardinal:nn} { \marg{language tag} \marg{integer} }
%  Converts an integer to its cardinal representation for a given language.
% \end{docCmd}
%    \begin{macrocode}
\ExplSyntaxOn
\cs_new:Npn \int_to_mixedcase_cardinal:nn #1 #2
  {
    \cs_if_exist:cTF {int_to_cardinal_aux_#1:n}
        { \cs:w int_to_cardinal_aux_#1:n\cs_end:{#2} }
        {  
          \msg_error:nnnn { phd-counters } { unknown-language } {#1} {} 
          \cs:w int_to_cardinal_aux_en:n\cs_end:{#2} 
        }   
  }
\ExplSyntaxOff  
%    \end{macrocode}
%
% The auxiliary function \docAuxCmd {int_to_cardinal_aux_en:n} is used to
% select the cardinal representation of the integer. There should be one
% for each of the supported languages. I still need to decide how best to
% integrate this with the rest of the code, either picking up files from
% Polyglossia or Babel. The auxiliary is always in mixed case.
%
%    \begin{macrocode}
\ExplSyntaxOn
\cs_new:Npn \hundred_en  #1 { Hundred~  }
\cs_new:Npn \thousand_en #1 { Thousand~ }
\cs_new:Npn \million_en  #1 { Million~  }
\cs_new:Npn \and_name_en #1 { ~and~    }
\ExplSyntaxOff
%    \end{macrocode}
% \begin{docCmd} {unit_strings_en:n } { \marg{integer}}
%   Converts integer to cardinal for integer 0--9.
% \end{docCmd}
%    \begin{macrocode}
\ExplSyntaxOn
\cs_new:Npn \unit_strings_en:n #1
  {
     \int_to_symbols:nnn {#1} {20}
       {
          {  1 } { Zero~~         }
          {  2 } { One~~          }
          {  3 } { Two~~          }
          {  4 } { Three~~        }
          {  5 } { Four~~         }
          {  6 } { Five~~         }
          {  7 } { Six~~          }
          {  8 } { Seven~~        }
          {  9 } { Eight~~        }
          {  10 } { Nine~~        }
          {  11 } { Ten~~         }
          {  12 } { Eleven~~         }
          {  13 } { Twelve~~         }
          {  14 } { Thirteen~~       }
          {  15 } { Forteen~~        }
          {  16 } { Fifteen~~        }
          {  17 } { Sixteen~~        }
          {  18 } { Seventeen~~      }
          {  19 } { Eighteen~~       }
          {  20 } { Nineteen~~       }
       }    
  }
\ExplSyntaxOff  
%    \end{macrocode}
%
% \begin{docCmd} {teens_strings_en:n } { \marg{integer}}
%   Converts 10+integer to cardinal.
% \end{docCmd}
%    \begin{macrocode}
\ExplSyntaxOn
\cs_new:Npn \teen_strings_en:n #1
  {
     \int_to_symbols:nnn {#1} {9}
       {
          
          {  1 } { Eleven         }
          {  2 } { Twelve         }
          {  3 } { Thirteen       }
          {  4 } { Forteen        }
          {  5 } { Fifteen        }
          {  6 } { Sixteen        }
          {  7 } { Seventeen      }
          {  8 } { Eighteen       }
          {  9 } { Nineteen       }
       }    
  }
\ExplSyntaxOff  
%    \end{macrocode}
%
% \begin{docCmd} {tens_strings_en:n } { \marg{integer}}
%   Converts 10*integer to cardinal.
% \end{docCmd}
%    \begin{macrocode}
\ExplSyntaxOn
\cs_new:Npn \tens_strings_en:n #1
  {
     \int_to_symbols:nnn {#1} {8}
       {
 
          {  1 } { Twenty~~      }
          {  2 } { Thirty~~      }
          {  3 } { Forty~        }
          {  4 } { Fifty~        }
          {  5 } { Sixty~        }
          {  6 } { Seventy~      }
          {  7 } { Eighty~       }
          {  8 } { Ninety~       }
       }    
  }
\ExplSyntaxOff  
%    \end{macrocode}
%
%    \begin{macrocode}
\ExplSyntaxOn
 \int_gzero_new:N \int_current_mod
 \int_gzero_new:c {int_current_val}
 \int_gzero_new:N \int_thousands
 \int_gzero_new:N \int_hundreds
 \int_gzero_new:N \int_tens
 \int_gzero_new:N \int_units
 \int_gzero_new:N \int_to_be_converted
 
\cs_new:Npn \int_to_cardinal_aux_en:n #1
  {
%    \end{macrocode}
%
% Start from higher numbers first. Take the number and divide by 100, then truncate
% the decimal part. We also store the original arabic number in \docAuxCmd{int_to_be_converted} in
% order to check for exceptions, where the algorithm fails such as zero for English, where we print
% it if is requested, but do not typeset it when it is a suffix or prefix.
%
%    \begin{macrocode}
   \int_gset:Nn \int_to_be_converted {#1} 
   \int_gset:Nn \int_thousands {\int_div_truncate:nn {#1}{1000}}
%    \end{macrocode}  
%
% We check to see if we have any thousands 
%    \begin{macrocode}
    \int_compare:nTF {\int_thousands > 0}
      {
        % \unit_strings_en:n {\int_hundreds +1 }
         \phd_hundreds {\int_thousands}
         \thousand_en\ \and_name_en\
         \exp_after:wN \int_gset:Nn \l_tmpa_int {#1-\int_thousands*1000}
         \phd_hundreds {\l_tmpa_int}
      } 
      { 
         \phd_hundreds {#1}
      }
  }
\ExplSyntaxOff  
%    \end{macrocode}
%
% \begin{docCmd}{l_phd_hundreds} { \marg{integer} }
% \end{docCmd}
%
%    \begin{macrocode}
\ExplSyntaxOn
\cs_set:Npn \phd_hundreds #1
  {
%    \begin{macrocode}
   \int_gset:Nn \int_hundreds {\int_div_truncate:nn {#1}{100}} 
%    \end{macrocode}
%
% If the number is greater than 0 we have hundreds 100/100 will give 1, so it
% also covers the case.
%
%    \begin{macrocode}   
    \int_compare:nNnTF {\int_hundreds} > {0}
      {
        % \unit_strings_en:n {\int_hundreds +1 }
         \phd_units:n {\int_hundreds+1}
         \hundred_en
         \exp_after:wN \int_gset:Nn \l_tmpa_int {#1-\int_hundreds*100}
         \l_phd_tens:n {\l_tmpa_int} 
      } 
      { 
         \l_phd_tens:n {#1}
      }
  }
\ExplSyntaxOff
%    \end{macrocode}
%
% \begin{docCmd} {l_phd_tens:n} { \marg{integer<100}}
%   Typesets an integer in the range 0--99.
% \end{docCmd}
%
%    \begin{macrocode}          
\ExplSyntaxOn
\cs_set:Npn \l_phd_tens:n #1 
  {
%    \end{macrocode}
%  Assuming 25 will give us 2 and 10 will give us 1 and 93 9  
%    \begin{macrocode}  
     \int_gset:Nn \int_tens {\int_div_truncate:nn {#1}{10}}
     [\int_use:N \int_tens]
  %    \end{macrocode}
% 
%  The first 19 numbers are hard-coded and hence we check for these 
%    \begin{macrocode}      
      \int_compare:nTF { #1 < 20 }
      {
        %\unit_strings_en:n {\int_eval:n {#1+1} }
        \phd_units:n {\int_eval:n {#1+1} }
      }
      {
        \tens_strings_en:n {\int_tens-1}
        % how many units?   
        \int_gset:Nn \int_current_val {#1+1-\int_tens*10 }
        \phd_units:n {\int_current_val}
     }
  }
\ExplSyntaxOff
%    \end{macrocode}
%
% \begin{docCmd} {phd_units:n} { \marg{integer<20} }
%    Typesets an integer in the range 0-19 as a cardinal.
% \end{docCmd}
%
% As these digits can be appended or prepended to constructions, we take care not to print
% the zero, such as One Hundred and Zero or Zero One hundred unless the original number to
% be converted was a zero. In retrospect maybe zero should have been handled as a special
% string, like hundred.
%
%    \begin{macrocode}
\ExplSyntaxOn
\cs_set:Npn \phd_units:n #1
  {
   \int_compare:nTF {#1 > 1 }
   {  \unit_strings_en:n {#1}  }
   {
     \int_compare:nT {\int_to_be_converted = 0}
       {
         \unit_strings_en:n {1} 
       }
   }
  }
\ExplSyntaxOff  
%    \end{macrocode}
%
% \begin{docCmd} {int_to_uppercase_cardinal:nn } { \marg{language tag} \marg{integer} }
%   Converts an integer to its uppercase cardinal representation for a given language.
% \end{docCmd}
%
%    \begin{macrocode}
\ExplSyntaxOn
\cs_new:Npn \int_to_uppercase_cardinal:nn #1 #2
  {
    \cs_if_exist:cTF {int_to_cardinal_aux_#1:n}
        { \exp_after:wN \MakeTextUppercase {\cs:w int_to_cardinal_aux_en:n\cs_end: {#2}} }
        {  
          \msg_error:nnnn { phd-counters } { unknown-language } {#1} {} 
          \cs:w int_to_cardinal_aux_en:n\cs_end:{#2} 
        }   
  }
\ExplSyntaxOff
%    \end{macrocode}
%
% \begin{docCmd} {int_to_lowercase_cardinal:nn } { \marg{language tag} \marg{integer} }
%  Converts an integer to its lowercase cardinal representation for a given language.
% \end{docCmd}
%    \begin{macrocode}
\ExplSyntaxOn
\cs_new:Npn \int_to_lowercase_cardinal:nn #1 #2
  {
    \cs_if_exist:cTF {int_to_cardinal_aux_#1:n}
        { \exp_after:wN \tl_lower_case:n {\cs:w int_to_cardinal_aux_#1:n\cs_end:{#2}} }
        {  
          \msg_error:nnnn { phd-counters } { unknown-language } {#1} {} 
          \cs:w int_to_cardinal_aux_en:n\cs_end:{#2} 
        }   
  }
\ExplSyntaxOff
%    \end{macrocode}
%
%
% \section{Ordinals (English)}
%
%  The package defaults to English and we provide conversion to
%  ordinals.
%  \begin{docCmd} {int_to_mixedcase_ordinal:nn} { \marg{short language code}  \marg{integer} }
%    Switch to typeset an ordinal given an integer. 
%  \end{docCmd}
%    \begin{macrocode}
\ExplSyntaxOn
\cs_new:Npn \int_to_mixedcase_ordinal:nn #1 #2
  {
    \cs_if_exist:cTF {int_to_ordinal_aux_#1:n}
        { \cs:w int_to_ordinal_aux_#1:n\cs_end:{#2} }
        {  
          \msg_error:nnnn { phd-counters } { unknown-language } {#1} {} 
          \cs:w int_to_ordinal_aux_en:n\cs_end:{#2} 
        }   
  }

\cs_new:Npn \int_to_ordinal_aux_en:n #1
  { 
    \int_compare:nNnTF  {#1} < {31} 
    { 
      \int_to_symbols:nnn {#1} {30}
       {
        {  0 } { Zeroeth      }
        {  1 } { First        }
        {  2 } { Second       }
        {  3 } { Third        }
        {  4 } { Fourth       }
        {  5 } { Fifth        }
        {  6 } { Sixth        }
        {  7 } { Seventh      }
        {  8 } { Eighth       }
        {  9 } { Ninth        }
        { 10 } { Tenth        }
        { 11 } { Eleventh     }
        { 12 } { twelfth      }
        { 13 } { Thirteenth   }
        { 14 } { Fourteenth   }
        { 15 } { Fifteenth    }
        { 16 } { Sixteenth    }
        { 17 } { Seventeenth  }
        { 18 } { Eighteenth   }
        { 19 } { Nineteenth   }
      }
    }{ \msg_error:nnnn { phd-counters } { counter-too-large-for-ordinal } {#1} {} }
  }
\ExplSyntaxOff
%    \end{macrocode}
%
%   \begin{docCmd} {int_to_lowercase_ordinal:nn} { \marg{short language code}  \marg{integer} }
%    Typesets a lowercase ordinal for the integer passed to its second argument. 
%  \end{docCmd}
%    \begin{macrocode}
\ExplSyntaxOn
\cs_new:Npn \int_to_lowercase_ordinal:nn #1 #2
  {
    \cs_if_exist:cTF {int_to_ordinal_aux_#1:n}
        { 
           \exp_after:wN 
           \MakeTextLowercase{ \cs:w int_to_ordinal_aux_#1:n\cs_end: {#2} } 
        }
        {  
          \msg_error:nnnn { phd-counters } { unknown-language } {#1} {} 
          \cs:w int_to_ordinal_aux_en:n \cs_end:{#2} 
        }   
  }
\ExplSyntaxOff  
%    \end{macrocode}
%
%   \begin{docCmd} {int_to_uppercase_ordinal:nn} { \marg{short language code}  \marg{integer} }
%    Typesets an uppercase ordinal for the integer passed to its second argument. 
%  \end{docCmd}
%    \begin{macrocode}
\ExplSyntaxOn
\cs_new:Npn \int_to_uppercase_ordinal:nn #1 #2
  {
    \cs_if_exist:cTF { int_to_ordinal_aux_#1:n }
        { 
           \exp_after:wN 
           \MakeTextUppercase{ \cs:w int_to_ordinal_aux_#1:n\cs_end: {#2} } 
        }
        {  
          \msg_error:nnnn { phd-counters } { unknown-language } {#1} {} 
          \cs:w int_to_ordinal_aux_en:n \cs_end:{#2} 
        }   
  }
\ExplSyntaxOff  
%    \end{macrocode}

% \begin{docCommand}{words@cx} {\marg{int}} Utility macro for translating a 
%   number from numbers to words.
% \end{docCommand}
%    \begin{macrocode}
\def\words@cx#1{%
  \ifcase#1 zero\or one\or two\or three\or four\or five\or six\or seven
\or eight\or nine\or ten\or eleven\or twelve\or thirteen\or
fourteen
\or fifteen\or sixteen\or seventeen\or eighteen\or nineteen \or
twenty
\or twenty one\or twenty two\or twenty three\or twenty four\or
twenty five
\or twenty six\or twenty seven \or twenty eight \or twenty
nine\or thirty
\or thirty one\or thirty two\or thirty three\or thirty four\or
thirty five
\or thirty six\or thirty seven\or thirty eight\or thirty nine\or
forty\or forty one
\or forty two \or forty three\or forty four\or forty five \or
forty six \or forty seven
\or forty eight \or forty nine\or fifty\or fifty on\or fifty
two\or fifty three
\or fifty four\or fifty five\or fifty six\or fifty seven\or
fifty eight\or fifty nine
  \or sixty \or sixty one \or sixty two
  \or sixty three \or sixty four \or sixty five
    \else
    #1
    %\@ctrerr
    \fi
}

\def\Words@cx#1{%
\ifcase#1 Zero\or One\or Two\or Three\or Four\or Five\or Six\or
Seven\or Eight\or Nine\or Ten\or
Eleven\or Twelve\or Thirteen\or Fourteen\or Fifteen\or
Sixteen\or Seventeen\or Eighteen\or Nineteen \or Twenty\or
Twenty One\or Twenty Two\or Twenty Three\or Twenty Four\or
Twenty Five\or Twenty Six\or Twenty Seven \or Twenty Eight \or
Twenty Nine\or Thirty\or Thirty One\or Thirty Two\or Thirty
Three\or Thirty Four\or Thirty Five\or Thirty Six\or Thirty
Seven\or Thirty Eight\or Thirty Nine\or Forty\or Forty One\or
Forty Two \or Forty Three\or Forty Four\or Forty Five \or Forty
Six \or Forty Seven\or Forty Eight \or Forty Nine\or Fifty\or
Fifty One\or Fifty Two\or Fifty Three\or Fifty four\or Fifty
Five\or Fifty Six\or Fifty Seven\or Fifty Eight\or Fifty Nine\or
Sixty \or Sixty One \or Sixty Two
\or Sixty Three \or Sixty Four \or Sixty Five \or SixtySix \or SixtySeven
\or Sixty Eight \or SixtyNine \or Seventy \or Seventy One \or Seventy Two
\else
#1
%\@ctrerr
\fi}

\def\WORDS@cx#1{%
\ifcase#1 ZERO\or ONE\or TWO\or THREE\or FOUR\or FIVE\or SIX\or
SEVEN\or EIGHT\or NINE\or TEN\or
ELEVEN\or TWELVE\or THIRTEEN\or FOURTEEN\or FIFTEEN\or
SIXTEEN\or SEVENTEEN\or EIGHTEEN\or NINETEEN \or TWENTY\or
TWENTY ONE\or TWENTY TWO\or TWENTY THREE\or TWENTY FOUR\or
TWENTY FIVE\or TWENTY SIX\or TWENTY SEVEN \or TWENTY EIGHT \or
TWENTY NINE\or THIRTY\or THIRTY ONE\or THIRTY TWO\or THIRTY
THREE\or THIRTY FOUR\or THIRTY FIVE\or THIRTY SIX\or THIRTY
SEVEN\or THIRTY EIGHT\or THIRTY NINE\or FORTY\or FORTY ONE\or
FORTY TWO \or FORTY THREE\or FORTY FOUR\or FORTY FIVE\or FORTY
SIX\or FORTY SEVEN\or FORTY EIGHT\or FORTY NINE\or FIFTY\or
FIFTY ONE\or FIFTY TWO\or FIFTY THREE\or FIFTY FOUR\or FIFTY
FIVE\or FIFTY SIX\or FIFTY SEVEN\or FIFTY EIGHT\or FIFTY NINE\or
SIXTY\or SIXTY ONE\or SIXTY TWO\or SIXTY THREE \or SIXTY FOUR\or
SIXTY FIVE\or SIXTY SIX\or SIXTY SEVEN\or SIXTY EIGHT\or SIXTY
NINE\or SEVENTY\or SEVENTY ONE\or SEVENTY TWO\or SEVENTY
THREE\or SEVENTY FOUR\or SEVENTY FIVE\or SEVENTY SIX\or SEVENTY
SEVEN\or SEVENTY EIGHT\or SEVENTY NINE\or EIGHTY\or EIGHTY
ONE\or EIGHTY TWO\or EIGHTY THREE\or EIGHTY FOUR\or EIGHTY
FIVE\or EIGHTY SIX\or EIGHTY SEVEN\or EIGHTY EIGHT\or EIGHTY
NINE\or NINETY \or NINETY ONE \or NINETY TWO \or NINETY THREE
\or NINETY FOUR \or NINETY FIVE
\else
#1
%\@ctrerr
\fi
}
   
\def\ORDINALS@cx#1{%
\ifcase#1 ZEROETH\or FIRST\or SECOND\or THIRD\or FOURTH\or
FIFTH\or SIXTH\or SEVENTH\or EIGHTTH\or NINTH\or TENTH\or
ELEVENTH\or TWELFTH\or THIRTEENTH\or FOURTEENTH\or FIFTEENTH\or
SIXTEENTH\or SEVENTEEN\or EIGHTEEN\or NINETEEN \or TWENTY\or
TWENTY ONE\or TWENTY TWO\or TWENTY THREE\or TWENTY FOUR\or
TWENTY FIVE\or TWENTY SIX\or TWENTY SEVEN \or TWENTY EIGHT \or
TWENTY NINE\or THIRTY\or THIRTY ONE\or THIRTY TWO\or THIRTY
THREE\or THIRTY FOUR\or THIRTY FIVE\or THIRTY SIX\or THIRTY
SEVEN\or THIRTY EIGHT\or THIRTY NINE\or FORTY\or FORTY ONE\or
FORTY TWO \or FORTY THREE\or FORTY FOUR\or FORTY FIVE\or FORTY
SIX\or FORTY SEVEN\or FORTY EIGHT\or FORTY NINE\or FIFTY\or
FIFTY ONE\or FIFTY TWO\or FIFTY THREE\or FIFTY FOUR\or FIFTY
FIVE\or FIFTY SIX\or FIFTY SEVEN\or FIFTY EIGHT\or FIFTY NINE\or
SIXTY\or SIXTY ONE\or SIXTY TWO\or SIXTY THREE \or SIXTY FOUR\or
SIXTY FIVE \or SIXTY SIX \or SIXTY SEVEN \or SIXTY EIGHT \or SIXTY NINE
\or SEVENTY \or SEVENTY ONE \or SEVENTY TWO \or SEVENTY THREE
\or SEVENTY FOUR \or SEVENTY FIVE \or SEVENTY SIX \or SEVENTY SEVEN
\or SEVENTY EIGHT \or SEVENTY NINE \or EIGHTY
\else
#1
%\@ctrerr
\fi}

\def\ordinals@cx#1{%
  \ifcase#1 Zeroeth\or First\or Second\or Third\or Fourth\or Fifth\or Sixth
  \or Seventh\or Eighth\or Ninth\or Tenth\or
 Eleventh\or Twelfth\or Thirteenth\or Fourteenth\or Fifteenth
\or SIXTEENTH\or SEVENTEEN\or EIGHTEEN\or NINETEEN \or TWENTY\or
TWENTY ONE\or TWENTY TWO\or TWENTY THREE\or TWENTY FOUR\or
TWENTY FIVE\or TWENTY SIX\or TWENTY SEVEN \or TWENTY EIGHT \or
TWENTY NINE\or THIRTY\or THIRTY ONE\or THIRTY TWO\or THIRTY
THREE\or THIRTY FOUR\or THIRTY FIVE\or THIRTY SIX\or THIRTY
SEVEN\or THIRTY EIGHT\or THIRTY NINE\or FORTY\or FORTY ONE\or
FORTY TWO \or FORTY THREE\or FORTY FOUR\or FORTY FIVE\or FORTY
SIX\or FORTY SEVEN\or FORTY EIGHT\or FORTY NINE\or FIFTY\or
FIFTY ONE\or FIFTY TWO\or FIFTY THREE\or FIFTY FOUR\or FIFTY
FIVE\or FIFTY SIX\or FIFTY SEVEN\or FIFTY EIGHT\or FIFTY NINE\or
SIXTY\or SIXTY ONE\or SIXTY TWO\or SIXTY THREE \or SIXTY FOUR\or
SIXTY FIVE\or SIXTY SIX \or SIXTY SEVEN \or \else
#1
%\@ctrerr
\fi
}
%    \end{macrocode}
%  \ExplSyntaxOn 
%
% \int_step_inline:nnnn {0} {101} {1030} 
%  {
%  (#1)~~\int_to_mixedcase_cardinal:nn {en} {#1}\\
%  }
%%    \int_to_mixedcase_cardinal:nn {en} {21} \\
%%    \int_to_uppercase_cardinal:nn {en} {22} \\
%%    \int_to_lowercase_cardinal:nn {en} {21}\\


%  \ExplSyntaxOff
% \iffalse
%</COUNTERS>
% \fi
\endinput

%%    \int_to_uppercase_cardinal:nn {en} {21} \\
%%    \int_to_lowercase_cardinal:nn {en} {21}\\
%%    \int_to_mixedcase_ordinal:nn {en} {21} \\
%%    \int_to_uppercase_ordinal:nn {en} {21} \\
%%    \int_to_lowercase_ordinal:nn {en} {21} \\
%%    \tens_strings_en:n {9}\\
%%    \int_use:N \int_current_mod \\
%%    \int_use:N \int_current_val
