% \iffalse meta-comment
%<*internal>
\iffalse
%</internal>
%<*readme>
----------------------------------------------------------------
phd-scriptsmanager --- a package to shorten preambles
E-mail: yannislaz@gmail.com
Released under the LaTeX Project Public License v1.3c or later
See http://www.latex-project.org/lppl.txt
----------------------------------------------------------------

%</readme>
%<*readmemd>
###The `phd-scriptsmanager` LaTeX2e package version 0.08.0

The `phd-scriptsmanager` latex package and the class 
with the same name provide
convenient methods to create new styles for books, reports
and articles. It also loads the most commonly used packages 
and resolves conflicts.

This work consists of the file  

     `phd-scriptsmanager.dtx`,
     
and the derived files   

     `phd-scriptsmanager.ins`,  
     `phd-scriptsmanager.pdf`, 
     
     and 
     
     `phd-scriptsmanager.sty`.

###Installation

The documentation of this package uses numerous fonts not available in a
normal `TeX` distribution. Before you regenerate it, make sure you install these
fonts first. All fonts are with an open source license. 

The fonts we require for the `phd` system to be fully functional and capable
to typeset almost _any_ script or language that existed or is still live are
the following:

- The [noto](https://www.google.com/get/noto/) fonts from Google. the fonts
  are licenced under the [Apache License Version 2.0](http://www.apache.org/licenses/LICENSE-2.0.html)  or the [SIL Open Font License, Version 1.1]. Download and
  install all the fonts.

- The ancient fonts provided by [George Douros](http://users.teilar.gr/~g1951d/). 

- The Tiresias PCfont font from  [Tiresias](http://www.tiresias.org/fonts/). This font
  is not actually used for scripts, but for experiments in readability. It looks
  very good for headings. The aim of this organization is to make Information
  Communication Technologies accessible to blind and partially sighted people. This
  I thought was a good opportunity to promote their work.
  
- code2000.ttf and code2001.ttf 

- Shonar Bangla font.

- Vrinda

If you have Windows

- Microsoft JhengHei and SimSun

There are many more fonts, I will revisit these docs to provide full documentation.

Once you ready then

run
     
      `phd-lua.bat` on windows
      
          

If you have any difficulties with the package come and join us at
http://tex.stackexchange.com and post a new question or
add a comment at http://tex.stackexchange.com/a/45023/963.
or send me a message at  yannislaz at gmail.com

### Documentation

The package was written using the `doc` and `docscript` packages,
so that it is self documented in a literary programming style. 
The .pdf is a fat document, providing over fifty book styles (the
equivalent of classes) plus there is a lot of write-up on the inner
workings of TeX and LaTeX2e. However, you don't need to know much
to use it.

      \usepackage{phd}
      \input{style13}

All choices, are made via an extended key-value interface. 
Although not a compliment, it resembles CSS and the keys are a bit verbose but
attributes are easy to change and have a consistent and easy to remember interface.

To set or add a key we only use the command `\cxset`:

      \cxset{chapter name font-size = Huge,
             chapter number font-size = HUGE} 

### Future Development

This is still an experimental version, but I will retain the
interface in future releases. There is a large amount of
work still to be carried out to improve the template styles
provided, to test it more thoroughly and to add a number of
improvements in the special designs. At present I estimate
that I have completed about 70% of the work that needs
to be done.

__The package as it stands is not production stable.__ 


%</readmemd>
%
%<*TODO>
## phd-scriptmanager
1.  Document fully all fonts not availabe at CTAN and provide links or folder for download.
2.  More selective de-activation of scripts via keys.
3.  Messaging and errors need to be extended.
%</TODO>
%<*internal>
\fi
\def\nameofplainTeX{plain}
\ifx\fmtname\nameofplainTeX\else
  \expandafter\begingroup
\fi
%</internal>
%<*install>
\input docstrip.tex
\keepsilent
\askforoverwritefalse
\preamble
----------------------------------------------------------------
phd-scriptsmanager --- A package to beautify documents.
E-mail: yannislaz@gmail.com
Released under the LaTeX Project Public License v1.3c or later
See http://www.latex-project.org/lppl.txt
----------------------------------------------------------------
\endpreamble

%\BaseDirectory{C:/users/admin/my documents/github/phd}
%\usedir{MWE}
\generate{\file{\jobname.sty}{
  \from{\jobname.dtx}{SCRIPTS}}
  }

%\nopreamble\nopostamble

%</install>

%<install>\endbatchfile
%<*internal>
%\usedir{tex/latex/phd}
\generate{
  \file{\jobname.ins}{\from{\jobname.dtx}{install}}
}
\nopreamble\nopostamble

\generate{
	\file{README.txt}{\from{\jobname.dtx}{readme}}
  }

\generate{
  \file{\jobname.md}{\from{\jobname.dtx}{readmemd}}
}
\generate{
  \file{\jobname-todo.tex}{\from{\jobname.dtx}{TODO}}
}

\ifx\fmtname\nameofplainTeX
  \expandafter\endbatchfile
\else
  \expandafter\endgroup
\fi
%</internal>
%<*driver>

%\listfiles
\documentclass[twoside,11pt,a4paper]{ltxdoc}
\usepackage[bottom=2cm]{geometry}
\savegeometry{std}
% \usepackage[style=mla]{biblatex}
\usepackage{phd}
\usepackage{phd-documentation}
\usepackage{phd-toc}
\usepackage{phd-runningheads}
\usepackage{phd-lowersections}
\usepackage{makeidx}
\usepackage{phd-lists}
\pagestyle{headings}
\sethyperref
\addbibresource{phd}
\cxset{palette bbc}
\makeindex
\begin{filecontents}{defaults-chapters}
%%    General Defaults for Chapters
\cxset{%    
    chapter title margin-top-width    =  0cm,
    chapter title margin-right-width  =  1cm,
    chapter title margin-bottom-width = 10pt,
    chapter title margin-left-width   = 0pt,
    chapter align                     = left,
    chapter title align               = left, %checked
    chapter name                      = hang,
    chapter format                    = fashion,
    chapter font-size                 = Huge,
    chapter font-weight               = bold,
    chapter font-family               = sffamily,
    chapter font-shape                = upshape,
    chapter color                     = black,
    chapter number prefix             = ,
    chapter number suffix             = ,
    chapter numbering                 = arabic,
    chapter indent                    = 0pt,
    chapter beforeskip                = -3cm,
    chapter afterskip                 = 30pt,
    chapter afterindent               = off,
    chapter number after              = ,
    chapter arc                       = 0mm,
    chapter background-color          = bgsexy,
    chapter afterindent               = off,
    chapter grow left                 = 0mm,
    chapter grow right                = 0mm, 
    chapter rounded corners           = northeast,
    chapter shadow                    = fuzzy halo,
    chapter border-left-width         = 0pt,
    chapter border-right-width        = 0pt,
    chapter border-top-width          = 0pt,
    chapter border-bottom-width       = 0pt,
    chapter padding-left-width        = 0pt,
    chapter padding-right-width       = 10pt,
    chapter padding-top-width         = 10pt,
    chapter padding-bottom-width      = 10pt,
    chapter number color              = white,
    chapter label color               = white,    
    }
 \cxset{    
    chapter number font-size        = huge,
    chapter number font-weight      = bfseries,
    chapter number font-family      = sffamily,
    chapter number font-shape       = upshape,
    chapter number align            = Centering,
    }
\cxset{%    
     chapter title font-size        = Huge,
     chapter title font-weight      = bold,
     chapter title font-family      = calligra,
     chapter title font-shape       = upshape,
     chapter title color            = black,
     }    
\end{filecontents}
%% LaTeX2e file `defaults-chapters'
%% generated by the `filecontents' environment
%% from source `phd-scriptsmanager' on 2015/08/25.
%%
%%    General Defaults for Chapters
\cxset{%
    chapter title margin-top-width    =  0cm,
    chapter title margin-right-width  =  1cm,
    chapter title margin-bottom-width = 10pt,
    chapter title margin-left-width   = 0pt,
    chapter align                     = left,
    chapter title align               = left, %checked
    chapter name                      = hang,
    chapter format                    = fashion,
    chapter font-size                 = Huge,
    chapter font-weight               = bold,
    chapter font-family               = sffamily,
    chapter font-shape                = upshape,
    chapter color                     = black,
    chapter number prefix             = ,
    chapter number suffix             = ,
    chapter numbering                 = arabic,
    chapter indent                    = 0pt,
    chapter beforeskip                = -3cm,
    chapter afterskip                 = 30pt,
    chapter afterindent               = off,
    chapter number after              = ,
    chapter arc                       = 0mm,
    chapter background-color          = bgsexy,
    chapter afterindent               = off,
    chapter grow left                 = 0mm,
    chapter grow right                = 0mm,
    chapter rounded corners           = northeast,
    chapter shadow                    = fuzzy halo,
    chapter border-left-width         = 0pt,
    chapter border-right-width        = 0pt,
    chapter border-top-width          = 0pt,
    chapter border-bottom-width       = 0pt,
    chapter padding-left-width        = 0pt,
    chapter padding-right-width       = 10pt,
    chapter padding-top-width         = 10pt,
    chapter padding-bottom-width      = 10pt,
    chapter number color              = white,
    chapter label color               = white,
    }
 \cxset{
    chapter number font-size        = huge,
    chapter number font-weight      = bfseries,
    chapter number font-family      = sffamily,
    chapter number font-shape       = upshape,
    chapter number align            = Centering,
    }
\cxset{%
     chapter title font-size        = Huge,
     chapter title font-weight      = bold,
     chapter title font-family      = calligra,
     chapter title font-shape       = upshape,
     chapter title color            = black,
     }
  
%\definecolor{bgsexy}{HTML}{FF6927}
%
%\definecolor{creamy}{HTML}{FDEBD7}
\cxset{chapter title color= creamy,
       chapter label color = creamy,
       chapter number color = creamy,
       chapter number font-size = Huge,
       subsection title color = creamy,
       chapter name = CHAPTER,
       chapter label case = upper,
       chapter number align=left,
       part format = traditional,
       part background-color=spot,
       part beforeskip                = -3cm,
       part afterskip                 = 30pt,
       }
\begin{document}
\parindent1em
\coverpage{asia}{Book Design Monographs}{Camel Press}{SCRIPTS}{DESIGN} 
\pagestyle{empty}
%\coverpage{habtoor-city}{Delay Claim}{HLS-DSE/JV}{HABTOOR CITY}{MEP CLAIM} 
\secondpage
\pagestyle{empty}
\clearpage

\tableofcontents

\pagestyle{empty}
\setcounter{secnumdepth}{6}
\parskip0pt plus.1ex minus.1ex
\mainmatter
\pagenumbering{arabic}
\pagestyle{headings} 
\hbadness=10001  
\hfuzz=50pt 
\vfuzz=50pt 
\vbadness=\maxdimen   
\makeatletter
%\@debugtrue

\makeatother
\newfontfamily\aegean{Aegean.ttf}
%\newfontfamily\lineara{Aegean.ttf}
%\newfontfamily\cypriote{Aegean.ttf}
\let\lineara\aegean
\let\cypriote\aegean
\newfontfamily\oldpersian{Noto Sans Old Persian}
\newfontfamily\inscriptionalpahlavi{Noto Sans Inscriptional Pahlavi}
\newfontfamily\imperialaramaic{NotoSansImperialAramaic-Regular.ttf}
\newfontfamily\emoji{Symbola}
\makeatletter
\@specialfalse
\makeatother

%\cxset{steward,
%  numbering=arabic,
%  custom=stewart,
%  offsety=0cm,
%  image={asia.jpg},
%  texti={An introduction to the use of font related commands. The chapter also gives a historical background to font selection using \tex and \latex. },
%  textii={Typesetting Middle Eastern scripts, Hebrew, Samaritan, Arabic, Thaana, Syriac
%  with \XeLaTeX. Selection and definition of related commands. Right to left writing systems. Image A fallah woman and her child. 1878 Elizabeth Jerichau Baumann.
% },
%}



\chapter{Those Other Languages}
\label{ch:languages}

\epigraph{New York\\
          1. Act making it a misdemeanor to make a speech or talk in public manner, in any language other
than English upon any subject relating to the form of a character of the government or the administration or enforcement of the laws of this state or the United States. }{\itshape Introduced in the Assembly by Mr Hamill, Feb.23, and referred to the Codes Committe (A.878.)}

\parindent1em



\section{The world's scripts and languages}


On May 23, 1918, Iowa Gov. William Harding banned the use of any foreign language in public: in schools, on the streets, in trains, even over the telephone.   \footcite{frese} published a detailed history of this even in American history during the Great War years and its effects, which can even today be seen in Iowa. Such events of course during stressful times in a history of a country are not unique to America and similar politics can be observed throughout all human history. Today most Americans' answer to the calling of such a law would probably be the Unicode Character \unicodenumber{U+1F4A9}\footnote{\protect\emoji\protect\char"1F4A9}. Getting the character to print as a footnote in a document is another story. As many of the world's languages are facing extinction and the inclusion of a section in the |phd| package to deal with different scripts and appropriate fonts has been done in this spirit.

 
Probably there are more users of \latexe whose mother tongue is not English than those who speak the language. \tex out of the box does not offer facilities for using non-latin based scripts easily; this presents numerous problems. The biggest problem---which has been solved to a large extent---was the entering of text without having to mark all the special
characters such as umlauts (\"o) with commands. The second issue and which has been addressed by packages such as Babel, is redefining the strings such as ``Chapter" to another language. In software this is called internationalization and a governing standard is |i18n|. None of the current packages take such an approach and none of them as yet offer a satisfactory solution for |LuaLaTeX|. 



Another issue with writing systems and scripts is finding and using appropriate fonts. Most writing systems that have ever existed are now extinct. Only minute vestiges of one of the most ancient---Egyptian hieroglyphs---live on, unrecognized, in the Latin alphabet in which English, among hundreds of other languages, is conveyed today. The latin \textit{m}, for example, ultimately derives from the Egyptian's cononantal n-sign, depicting waves. There may never be a font that includes all the unicode characters (code2000) came close. Good fonts with well over ten thousand characters, keyed to the Unicode system, are now readily available. 

Bringhurst in the Elements of Typographic Style \citeyearpar{Bringhurst2005} critisized the allotment of only 256 characters in the extended ASCII specification and other software and considered this practice by software developers as `typographically sectarian and culturally stunted’. 


Bringhurst comments were unfair to programmers as he was probably unaware of the difficulties. Many  scripts are widely different to the Latin script. Hanunó'o is written vertically from bottom to top, whereas \nameref{s:tibetan} and sometimes Chinese from top to bottom.  Middle Eastern scripts such as Hebrew and Arabic are written from right to left. Some of the scripts have other peculiarities as they take different forms when they are at the middle of a word or at the end. Ancient scripts such as hieroglyphics could be written from top to botton or from right to left or left to right or boustrostrephon. The glyphs of the latter could also face either left or right and the writing direction can be determined based on the direction the figures are ``seeing''. \index{boustrostrephon}

Note that  we will be using the word ``script" instead of a ``writing system". Many people associate the word ``script" with a small program which is normally used on the command line. Here ``script" means a collection of letters and other characters, meant for writing human languages in a systematic way.  We say that languages such as English, Dutch and Icelandic and Vietnam use the Latin \emph{script}, although they have different repertoires of characters. 


\section{TeX's support for different languages}

\tex's support for languages centered around hyphenation patterns.
Primitives such as \docAuxCommand{language}=\meta{number} can be used to store hyphenation patterns and exceptions for up to 256 different languages. 
This primitive is then used by \tex to apply an appropriate set of hyphenation rules for each paragraph or part of a paragraph in a document\footnote{\url{http://www.tug.org/utilities/plain/cseq.html language-rp}}. 

When \tex begins a ne paragraph it sets the \emph{current language} to \cmd{\language}. Just before it adds each new character to the paragraph in unrestricted horizontal mode, it compares the current language to \cmd{\language}. If they are different, TeX : 
\begin{enumerate}
\item changes the current language to \cmd{\language}; 

\item inserts a whatsit\index{whatsit>language} containing the new language and the values of |\lefthyphenmin| and |\righthyphenmin|; 

\item inserts the character. The |\setlanguage| command should be used to change languages in restricted horizontal mode (i.e., inside an |\hbox|). 
\end{enumerate}
If \meta{number} is less than 0 or greater than 255, 0 is used [455].
  Plain TeX has a \docAuxCommand{newlanguage} command which may be used to allocate numbers for languages [347]. Changes made to \refCom{language} are local to the group containing the change 
  
If you enter, for example, |\newlanguage\Catalan|, then to switch to the hyphenation patterns of the Catalan language, you need to write |\language = \Catalan|. Writing |\Catalan| by itsef is not sufficient. 
More about \tex's support for languages can be found in the \nameref{ch:hypenation}

\section{LaTeX language management}

\latexe follows the same route as \tex and Plain TeX and its only language support is for hyphenation.
In the source2e the File |lthyphen.dtx| describes the approach to loading the default file |hyphen.ltx| . If a file hyphen.cfg is found \latexe will load the appropriate hyphenation patterns. 

Traditionally language management was achieved using Johan 
Braams package \pkgname{Babel} which we describe in the next section. Numerous packages to assist in using different languages with \latex can be found at \url{http://www.ctan.org/tex-archive/language/}. 

\section{The Babel Package} 

The package \pkgname{Babel} debveloped by \citet{babel} was the first package to systematically offer foreign language
support for \latex. It has been updated for use with \XeTeX\ and \LuaTeX\ and provides an environment
in which documents can be typeset in a language
other than US English, or in more than one language.
However, no attempt has been done to
take full advantage of the features provided by the
latter, which would require a completely new core
(as for example polyglossia or as part of a future \latex3).

\subsection{Language files}
The package has a number of predefined language files with the extension |ldf|. Each \emph{language definition file} contains commands appropriate for setting strings and hyphenation patterns in the particular language, as well as
many ancillary macros to typeset dates and numbers in the typographical convention of the language. 


\begin{docCommand} {selectlanguage} {\marg{language}} {default none, initial US English}
When a user wants to switch from one language to another he can
do so using the macro |\selectlanguage|. This macro takes the
language, defined previously by a language definition file, as
its argument. It calls several macros that should be defined in
the language definition files to activate the special definitions
for the language chosen. For ``historical reasons'', a macro name is
converted to a language name without the leading |\|; in other words,
the two following declarations are equivalent:
\end{docCommand}
\begin{verbatim}
\selectlanguage{german}
\selectlanguage{\german}
\end{verbatim}

\begin{docCmd}{foreignlanguage}{ \marg{language} \marg{text}}
The command |\foreignlanguage| takes two arguments; the second
argument is a phrase to be typeset according to the rules of the
language named in its first argument. This command (1) only
switches the extra definitions and the hyphenation rules for the
language, \emph{not} the names and dates, (2) does not send
information about the language to auxiliary files (i.e., the
surrounding language is still in force), and (3) it works even if
the language has not been set as package option (but in such a
case it only sets the hyphenation patterns and a warning is shown).
\end{docCmd}

\begin{docCmd}{otherlanguage*} { \marg{language}{otherlanguage*}}
Same as |\foreignlanguage| but as environment. Spaces after the
environment are \textit{not} ignored.
\end{docCmd}


\section{The Polyglossia package}

The \pkgname{polyglossia} package has a lot of potential and has solved many issues
but its integration with large parts of the traditional |pdfLaTeX| world
is still under development and will probably take a while before one could
declare it easy to use and bug free \cite{polyglossia}. For example anything with the |bidi| package has issues with loading orders for a number of packages and least of which is with
the Ams packages. So if you are going to mix a number of languages in a \XeTeX\ document
you need to take extra care.

 Polyglossia is a package for facilitating multilingual typesetting with
 \XeLaTeX\ and (at an early stage) \LuaLaTeX.  Basically, it
 can be used as a replacement of \pkg{babel} for performing the following
 tasks automatically:
 
 \begin{enumerate}
 \item Loading the appropriate hyphenation patterns.
 \item Setting the script and language tags of the current font (if possible and
       available), via the package \pkg{fontspec}.
 \item Switching to a font assigned by the user to a particular script or language.
 \item Adjusting some typographical conventions according to the current language
       (such as afterindent, frenchindent, spaces before or after punctuation marks,
       etc.).
 \item Redefining all document strings (like chapter, ``figure'', ``bibliography'').
 \item Adapting the formatting of dates (for non-Gregorian calendars via external
       packages bundled with polyglossia: currently the Hebrew, Islamic and Farsi
       calendars are supported).
 \item For languages that have their own numbering system, modifying the formatting
       of numbers appropriately (this also includes redefining the alphabetic sequence
       for non-Latin alphabets).\footnote{ %
         For the Arabic script this is now done by the bundled package \pkg{arabicnumbers}.}
 \item Ensuring proper directionality if the document contains languages
       that are written from right to left (via the package \pkg{bidi},
       available separately).
 \end{enumerate}
 
 Several features of \pkg{babel} that do not make sense in the \XeTeX\/\luatex world (like font
 encodings, shorthands, etc.) are not supported supported by the package.
 
 Generally speaking, \pkg{polyglossia} aims to remain as compatible as possible
 with the fundamental features of \pkg{babel} while being cleaner, light-weight,
 and modern. The package \pkg{antomega} has been very beneficial in our attempt to
 reach this objective.


\section{Loading language definition files}

The recommended way of \pkg{polyglossia} to load language definition files
is given in the manual as:
 
\begin{docCmd}{setdefaultlanguage}{\oarg{options}\marg{lang}}
 (or equivalently \cmd\setmainlanguage).
\end{docCmd}
 
 Secondary languages can be loaded with

\begin{docCmd} {setotherlanguage}{\oarg{options}\marg{lang}}
\end{docCmd}
 These commands have the advantage of being explicit and of allowing you to set
 language-specific options.\footnote{ %
 More on language-specific options below.}
 It is also possible to load a series of secondary languages at once using

\begin{docCmd}{setotherlanguages} { \marg{lang1,lang2,lang3,\ldots}}
\end{docCmd}

 Language-specific options can be set or changed at any time by means of
\begin{docCmd}{setkeys} { \marg{lang}\marg{opt1=value1,opt2=value2,\ldots}}
\end{docCmd}

\subsection{Bidirectional languages}





\begin{comment}
\begin{Arabic}
ّ هو إذ الغاية؛ شريف الفوائد، جم المذهب، عزيز فنّ التاريخ فنّ أنّ اعلم
والملوك سيرهم، في والأنبياء أخلاقهم، في الأمم من الماضين أحوال على يوقفنا
ّ أحوال في يرومه لمن ذلك في الإقتداء فائدة تتم حتّى وسياستهم؛ دولهم في
والدنيا. الدين
\end{Arabic}
\end{comment}

The Greek language is represented both in modern Greek as well as its ancient variants.

\begin{verbatim}
\begin{greek}
\textbf{Η ελληνική γλώσσα} είναι μία από τις ινδοευρωπαϊκές γλώσσες, για την
οποία έχουμε γραπτά κείμενα από τον 15ο αιώνα π.Χ. μέχρι σήμερα. Αποτελεί το
μοναδικό μέλος ενός κλάδου της ινδοευρωπαϊκής οικογένειας γλωσσών. Ανήκει
επίσης στον βαλκανικό γλωσσικό δεσμό.\\	
\end{greek}
\end{verbatim}

\topline

\textbf{Η ελληνική γλώσσα} είναι μία από τις ινδοευρωπαϊκές γλώσσες, για την
οποία έχουμε γραπτά κείμενα από τον 15ο αιώνα π.Χ. μέχρι σήμερα. Αποτελεί το
μοναδικό μέλος ενός κλάδου της ινδοευρωπαϊκής οικογένειας γλωσσών. Ανήκει
επίσης στον βαλκανικό γλωσσικό δεσμό.\\	

\bottomline

\begin{verbatim}
\begin{russian}
\textbf{Русский язык} — один из восточнославянских языков, один из 
крупнейших языков мира, в том числе самый распространённый из славянских
языков и самый распространённый язык Европы, как географически, так и по
числу носителей языка как родного (хотя значительная, и географически бо́
льшая, часть русского языкового ареала находится в Азии).	\\

\end{russian}
\end{verbatim}



\textbf{Русский язык} — один из восточнославянских языков, один из крупнейших языков мира, в том числе самый распространённый из славянских языков и самый распространённый язык Европы, как географически, так и по числу носителей языка как родного (хотя значительная, и географически бо́льшая, часть русского языкового ареала находится в Азии).	\\





\section{The Translator package}

The \pkgname{translator} package was developed by Till Tantau \cite{translator}. It provides a flexible
mechanism for translating individual words into different languages.
For example, it can be used to translate a word like ``figure'' into,
say, the German word ``Abbildung''. Such a translation mechanism is
useful when the author of some package would like to localize the
package such that texts are correctly translated into the language
preferred by the user. The translator package is \emph{not} intended
to be used to automatically translate more than a few words. 

You may wonder whether the translator package is really necessary
since there is the (very nice) |babel| package available for
\LaTeX. This package already provides translations for words like
``figure''. Unfortunately, the architecture of the babel package was
designed in such a way that there is no way of adding translations of
new words to the (very short) list of translations directly build into
babel.

The translator package was specifically designed to allow an easy
extension of the vocabulary. It is both possible to add new words that
should be translated and translations of these words.

\subsection{Using the Translator Package}

  The \pkg{Translator} needs to be used with \pkgname{Babel} and I am not too sure yet 
  if it is ready  to be used with Polyglossia.

Once the package has loaded a language or a set of languages the optional argument to the
\cmd{\translate} can be used to translate a string. 

\begin{texexample}{Translating strings}{ex:translator}
  \translate[to=german]{rightpagename}
  \translate[to=dutch]{rightpagename}
\end{texexample}

Before you can provide the translations you need to provide your own dictionaries, where you require them. These need to be installed at a place where \tex can find them.

\begin{docCmd} {ProvidesDictionary} { \marg{dictionary file name} \marg{language} }
\end{docCmd}

The dictionary has to be saved in a specific format that relates to the \refCmd{ProvidesDictionary} command. The second argument of the command must be appended to the file name; for the example the file is saved as\footnote{This  example is from the translator package bundle and is under the folder \texttt{base}}:

|translator-basic-dictionary-German|

The concepts take a bit of time to sink in, but once you have everything set up, it is quite easy and straight forward to incorporate it, into your package. 

\begin{minted}{TeX}
\ProvidesDictionary{translator-basic-dictionary}{German}

\providetranslation{Abstract}{Zusammenfassung}
\providetranslation{Addresses}{Adressen}
\providetranslation{addresses}{Adressen}
\providetranslation{Address}{Adresse}
\providetranslation{address}{Adresse}
\providetranslation{and}{und}
\providetranslation{Appendix}{Anhang}
\providetranslation{Authors}{Autoren}
\providetranslation{authors}{Autoren}
\providetranslation{Author}{Autor}
\providetranslation{author}{Autor}
\end{minted} 

This is in contrast to Babel and Polyglossia that define
commands for each string to be translated such as,

\begin{minted}{TeX}
\def\captionsdutch{%
    \def\prefacename{Voorwoord}%
    \def\refname{Referenties}%
    \def\abstractname{Samenvatting}%
    \def\bibname{Bibliografie}%
    \def\chaptername{Hoofdstuk}%
    \def\appendixname{Bijlage}%
    ...
    \def\proofname{Bewijs}%
    \def\glossaryname{Verklarende woordenlijst}%
    \def\today{\number\day~\ifcase\month%
      \or januari\or februari\or maart\or april\or mei\or juni\or
      juli\or augustus\or september\or oktober\or november\or
      december\fi
      \space \number\year}}
\end{minted}

\begin{docCommand}{usedictionary}{\marg{kind}}
  This command tells the |translator| package, that at the beginning of
  the document it should load \textit{all} dictionaries of kind \meta{kind} for
  the languages used in the document. Note that the dictionaries are
  not loaded immediately, but only at the beginning of the document.

  If no dictionary of the given \emph{kind} exists for one of the
  language, nothing bad happens.

  Invocations of this command accumulate, that is, you can call it
  multiple times for different dictionaries.
\end{docCommand}

\begin{docCommand}{uselanguage}{\marg{list of languages}}
  This command tells the |translator| package that it should load the
  dictionaries for all languages in the \meta{list of languages}. The
  dictionaries are loaded at the beginning of the document.
\end{docCommand}

\section{The phd package Language facilities}

The \pkgname{phd} package provides facilities for language handling, but albeitly still at an experimental stage. Sectioning command strings can easily be set in one's language by just typing the key in the appropriate language.

\begin{texexample}{Example of changing language in headings}{ex:lheadings}
\cxset{chapter name=Kapital,
%      chapter opening=anywhere
      }
\chapter{Das German}
\cxset{chapter name = Chapter}          
\end{texexample}


\section{Fonts for all the world's scripts and languages}

Many commercial as well as open source fonts exist that can be used to typeset text the world's scripts and languages. The aim of this section of the documentation is to present an overview of the most common scripts represented in the Unicode~7.0 standard. All the examples require the use of the \XeTeX\ engine. In addition you need to have a copy of the font on your own system. If you do not have them, the font loading mechanism of \XeTeX\ will take some time to search all the directories and slows compilation tremendously. 

\subsection{Pan-Unicode Fonts}

Thousands of fonts exist on the market, but fewer than a dozen fonts—sometimes described as ``pan-Unicode" fonts—attempt to support the majority of Unicode's character repertoire. Instead, Unicode-based fonts typically focus on supporting only basic ASCII and particular scripts or sets of characters or symbols. Several reasons justify this approach: applications and documents rarely need to render characters from more than one or two writing systems; fonts tend to demand resources in computing environments; and operating systems and applications show increasing intelligence in regard to obtaining glyph information from separate font files as needed, i.e. font substitution. Furthermore, designing a consistent set of rendering instructions for tens of thousands of glyphs constitutes a monumental task; such a venture passes the point of diminishing returns for most typefaces.

The \texttt{NotoSerif} fonts from Google\footnote{\protect\url{http://www.google.com/get/noto/}} have good support for 96 language fonts and the list is growing. Since these are widely available most of the scripts that follow use these fonts. Follow the instructions at the website to install them. It is just a matter of dragging them into the fonts folder for most operating systems.

Another freeware pan-Unicode font is Titus
This is an extended version of this font is TITUS Cyberbit Unicode, includes 36,161 characters in v4.0.

On Windows systems |Arial Unicode MS| contains glyphs for all code points within the Unicode Standard version 2.1.  

The code2000 font provides 63546 glyphs and is the nearest font to a universal font to handle Unicode. Unfortunately development stopped in 2008. As a comparison Linux Libertine O, provides 2674 glyphs. \label{code2000}

CJK fonts naturally will have the most glyphs, \idxfont{MingLiU} 34046 glyphs and is a very good font for CJK typesetting. Google in conjunction with Adobe also provides a fee CJK font.

The \href{http://ftp.gnu.org/gnu/freefont/}{FreeFont Project} currently supports most of the useful set of free outline (i.e. OpenType) fonts covering as much as possible of the Unicode character set. The set consists of three typefaces: one monospaced and two proportional (one with uniform and one with modulated stroke). 

The idea of having lots of different writing systems into a single font at all? How good does such a font need to be?
There are two extreme views.  The first one is that glyphs in a font shold comprise a unified design entity. This in practice makes sense only within a single language script. Different script systems, such a Latin, Arabic and Devanagari, have different typesetting traditions and conventions.  A good discussion of the advantages and disadvantages can be found at the gnu website \footnote{\protect\url{https://www.gnu.org/software/freefont/articles/Why_Unicode_fonts.html}}. For TeX it is a better proposition in order to avoid switching of fonts that can distract the writer. At least one requires fonts that are inclusive of one's usage. 

\section{The \texttt{ucharclasses} package}

For multilingual texts font switching can become cumbersome. The use of a pan-Unicode font as the default can help. However, if the languages are distinct enough to use different Unicode blocks, which are not covered by the \pkgname{polyglossia} package Mike Kamermans' package \pkgname{ucharclasses} can be used. This package only works with \xelatex and does not work with LuaTeX. 

\begin{verbatim}
% and the font switching magic
\usepackage[CJK, Latin, Thai, 
           Sinhala, Malayalam, 
           DominoTiles, 
           MahjongTiles]{ucharclasses}
\usepackage{fontspec}

\ifxetex
% default transition uses the widest coverage font I know of
  \setDefaultTransitions{\fontspec{Code2000.ttf}}{}

% overrides on the default rules for specific informal groups
  \setTransitionsForLatin{\fontspec{Palatino Linotype}}{}
  \setTransitionsForCJK{\fontspec{code2000.ttf}}{}%HAN NOM A
  \setTransitionsForJapanese{\fontspec{code2000.ttf}}{}%Ume Mincho

% overrides on the default rules for specific unicode blocks
  \setTransitionTo{CJKUnifiedIdeographsExtensionB}{\fontspec{SimSun-ExtB}}
  \setTransitionTo{Thai}{\fontspec{IrisUPC}}
  \setTransitionTo{Sinhala}{\fontspec{Iskoola Pota}}
  \setTransitionTo{Malayalam}{\fontspec{Arial Unicode MS}}
\ifxetex
\end{verbatim}

{
\newfontfamily\mahjong{FreeSerif.ttf}
domino tiles, 🁇 🀼 🁐 🁋 🁚 🁝, and mahjong tiles: 🀑 🀑 🀑 🀒 🀒 🀒 🀕 🀕 🀕 🀗 🀗 🀗 🀅 🀅 (using FreeSerif)

}

The interaction between Polyglossia and Fontspec can result in infinite looping and memory leaks. I do not recommend that you use these commands as yet. The use of the charclasses will also slow down compilation possibly by a factor of 10.



\section{PhD Settings}

The \pkgname{phd} provides support both for scripts, as well as language settings. A script setting sets the system to use appropriate fonts and if the script is associated with a unique language it will automatically handle language settings. Alternatively for multi-language scripts such as the Latin script, the language key can be used. This will automatically setup the language and an appropriate default font. 

\begin{docKey}[phd]{script} { = \meta{script name}} {default none, initial US English}{}
\end{docKey}

\begin{docKey}{language}{ =\meta{language name}}  {default none, US English}
The key language sets the main language for the document. This language will be used for the sectioning commands and common string translations.

If the language is English Polyglossia or Babel are not loaded automatically. If the language is other than English we load either Babel or Polyglossia depending on the engine used.
\end{docKey}


\begin{docKey}{languages}{ = \meta{language1, language2, language3}}  {}
The key |languages|, determines all the other scripts available for typesetting. For each language default font commands are create automatically. The aim is to be able to run a fully multilingual system with the minimum of upfront settings. These we leave to customize in the style template files.
\end{docKey}

\begin{docKey}{greek font}{ = \meta{options}\meta{font file}}  {}
The package comes with numerous language and appropriate default fonts
for each operating system. 
\end{docKey}













\cxset{steward,
  chapter format   = stewart,
  chapter numbering=arabic,
  offsety=0cm,
  image={fellah-woman.jpg},
  texti={An introduction to the use of font related commands. The chapter also gives a historical background to font selection using \tex and \latex. },
  textii={In this chapter we discuss keys that are available through the \texttt{phd} package and give a background as to how fonts are used
in \latex.
 },
}

\pagestyle{headings}
%\pgfpagesuselayout{2 on 1}[a3paper,landscape,border shrink=0mm]

\chapter{Ancient and Historic Scripts}

\epigraph{Now that all the old women have died, grandmas, great grandmas and other
assorted persons of that ilk, they have managed to engender within me a heap of
profound perplexities about persons and things old and extiguished forever. As long
as they were alive, I don't know why, I practically never wanted to ask}{---\textit{Beginning of Ioannou 1974.}}

Writing was perhaps the most important human invention. \tex authors and developers either due to need or fascination developed macros and fonts for many archaic writing systems. Many of these packages are now outdated, as the Unicode standard and the newer engines opened up a fascinating world. My own fascination with writing systems prompted me to add support for such scripts in the \pkgname{phd} package. The development to an extend was frustrating as the overloading of numerous fonts caused compilation to be very slow. Finding the right font was also problematic in many cases, as we opted to identify Open Source fonts. The \tex engine of preference is \luatex. To avoid loading too many fonts, unless they are required, we provide the keys:

\def\loadscripts{}
\cxset{scripts/.store in = \loadscripts}

\begin{docKey}[phd]{scripts}{ = \meta{all, lineara, linearb, phaestos,\ldots}} {default none, initial=none}
 The scripts key takes a list of options to enable or disable the loading of fonts and the usage of the key is explained later on. You set it with our only command |\cxset|\meta{key value list}
\end{docKey}

Supplementary keys, exist for each individual script enabling the setting of specific fonts to a particular script. However, if all the recommended fonts have been installed is quicker to use the |scripts| key.

\def\olmecfontstore{}

\cxset{olmec font/.store in=\olmecfontstore}

\cxset{olmec font=epiolmec}

\begin{docKey}[phd]{olmec font}{ = \meta{font name}} {default none, initial=none}
\end{docKey}

The key |script| can be used on its own. It will then load the default fonts built-in, in the |phd| package.

\cxset{script/.store in = \scripttempt}

\begin{docKey}[phd]{script =}{ \meta{script name}}{}
\end{docKey}

\begin{figure}[b]
\centering
\includegraphics[width=0.6\textwidth]{./images/rongo.jpg}
\caption{Rongo rongo writing. Tablet B Aruku kurenga, verso. One of four texts which provided the Jaussen list, the first attempt at decipherment. Made of Pacific rosewood, mid-nineteenth century, Easter Island.
(Collection of the SS.CC., Rome)}
\end{figure}

The first attempt  at communication via writing was through ideographic or mnemonic symbols. Undoubtedly symbolic writing must have existed much earlier than the surviving artifacts, carved in woord or scribled on muddy walls. The earliest surviving symbolic writing are the Jiahu symbols. They were carved on tortoise shells in Jiahu, ca.~6600~BC. Jiahu was a neolithic Peligang culture site found in Henan, China. In Europe the Tărtăria tablets are three tablets, discovered in 1961 by archaeologist Nicolae Vlassa at a Neolithic site in the village of Tărtăria (about 30 km (19 mi) from Alba Iulia), in Romania.[1] The tablets, dated to around 5300 BC,[2] bear incised symbols - the Vinča symbols - and have been the subject of considerable controversy among archaeologists, some of whom claim that the symbols represent the earliest known form of writing in the world. The Indus script appeared ca. 3500 BC and the Nsibidi script of Nigeria, ca. before 500 AD. 

No type of writing system is superior or inferior to another, as the type is often dependent on the language they represent. For example, the syllabary works perfectly fine in Japanese because it can reproduce all Japanese words, but it wouldn't work with English because the English language has a lot of consonant clusters that a syllabary will have trouble to spell out. The pretense that the alphabet is more "efficient" is also flawed. Yes, the number of letters is smaller, but when you read a sentence in English, do you really spell individual letters to form a word? The answer is no. You scan the entire word as if it is a logogram.

And finally, writing system is not a marker of civilization. There are many major urban cultures in the world did not employ writing such as the Andean cultures (Moche, Chimu, Inca, etc), but that didn't prevent them from building impressive states and empires whose complexity rivals those in the Old World

Unicode encodes a number of ancient scripts, which have not been in normal use for a millennium or more, as well as historic scripts, whose usage ended in recent centuries. Although these scripts are no longer used to write living languages, documents and inscriptions using these languages exist, both for extinct languages and for precursors of modern languages. The primary user communities for these scripts are scholars, interested in studying the scripts and the languages written in them. A few, such as Coptic, also have contemporary liturgical or other special purposes. Some of the historic scripts are related to each other as well as to modern alphabets. The following are provides as of Unicode version~7.2.
\index{Ancient and Historic Scripts>Ogham}
\index{Ancient and Historic Scripts>Old Italic}
\index{Ancient and Historic Scripts>Runic}
\index{Ancient and Historic Scripts>Gothic}
\index{Ancient and Historic Scripts>Akkadian}
\index{Ancient and Historic Scripts>Old Turkic}
\index{Ancient and Historic Scripts>Hieroglyphs}
\index{Ancient and Historic Scripts>Linear B}
\index{Ancient and Historic Scripts>Linear A}
\index{Ancient and Historic Scripts>Phoenician}
\index{Ancient and Historic Scripts>Old South Arabian}
\index{Ancient and Historic Scripts>Mandaic}
\index{Ancient and Historic Scripts>Avestan}
\index{Ancient Anatolian Alphabets}
\index{Old South Arabian}
\index{Phoenician}
\index{Imperial Aramaic}
\begin{center}
\begin{tabular}{lll}
\nameref{s:ogham}           
&\nameref{s:anatolian}
&\nameref{s:avestan}\\

\nameref{s:olditalic}      
&\nameref{s:oldsoutharabian}          
&\nameref{s:ugaritic} \\

    \nameref{s:runic}
&\nameref{s:phoenician} 
&\nameref{s:oldpersian} \\
    \nameref{s:gothic}            
    
&\nameref{s:imperialaramaic}            
&\nameref{s:sumero} \\

    \nameref{s:oldturkic}   
& \nameref{s:mandaic} 
& \nameref{ch:hieroglyphics}\\

   \nameref{s:linearb} 
&\nameref{s:parthian}       
&\nameref{s:meroitic}\\

 \nameref{s:cypriot}
&\nameref{s:inscriptionalpahlavi}       
& \nameref{s:lineara}\\
\end{tabular}
\end{center}

The following scripts are also encoded but following the Unicode
convention are described in other sections

\begin{center}
\begin{tabular}{llllll}
Coptic &\nameref{s:glagolitic} &Phags-pa. &Kaithi &Kharoshi &\nameref{s:brahmi}.\\
\end{tabular}
\end{center}

Some scripts such as \nameref{s:olmec} are not described in the Unicode standard, but we provide support for them.

\section{Linear A}
\label{s:lineara}

\section{Aegean and Cypriote Syllabaries}

The Greeks had evidently already occupied the mainland and islands of the
Ægean, including Crete, by the middle of the third millennium
BC. Around 2000 BC, following their consolidation of power on
Crete, new wealth from trade with cosmopolitan Canaan
allowed the creation of a complex palace economy, with major
centres at Knossos, Phaistos and other Cretan sites – Europe’s
first high civilization, the Minoan. Trade with Canaan had evidently
also brought Greeks into contact with Byblos’ pictorial
syllabic writing, whose underlying principle the Minoans borrowed.
Now, Cretans could also write their Minoan Greek language
using a small corpus of syllabo-logographic signs
representing \textit{in-di-vi-du-al} syllables. The signs themselves and
their phonetic values – nearly all V (e) or CV (te) – were wholly
indigenous: what the rebus signs, all originating from the
Cretan world, depicted, one pronounced in Minoan Greek, not
in a Semitic language. (Minoan Greek appears to have been an
archaic sister tongue of the mainland’s Mycenæan Greek.\footnote{A History of Writing. })

Three separate but related forms of syllabo-logographic
writing emerged in the Ægean between c. 2000 and 1200 BC: the
Minoan Greeks’ ‘hieroglyphic’ script and Linear A, and the
later Mycenæan Greeks’ Linear B. Minoan Greeks apparently
also took their writing at an early date to Cyprus, where it experienced
two stages: Cypro-Minoan (evidently derived from
Linear A is one of two currently undeciphered writing systems used in ancient Greece. Cretan hieroglyphic is the other. Linear A was the primary script used in palace and religious writings of the Minoan civilization. It was discovered by archaeologist Arthur Evans. It is the origin of the Linear B script, which was later used by the Mycenaean civilization.

Linear A and its daughter Linear C, the ‘Cypriote Syllabic
Script’. All Ægean and Cypriote scripts are clearly syllabologographic,
as the objective identity of each rebus sign would
have been immediately recognizable to each learner and user. It
seems that determinatives were never employed in any of the
Ægean or Cypriote scripts; however, logograms additionally
depicted most spelt-out items on accounting tablets. All Ægean
and Cypriote scripts, but for these separate logograms, were
completely phonetic.
\medskip

\includegraphics[width=0.8\textwidth]{./images/cretan-hieroglyphs.png}

\medskip
Crete’s `hieroglyphic’ script is the patriarch of this robust
family, its inspiration perhaps derived from Byblos via Cyprus
around 2000 BC. As its name implies, this script used
pictorial signs to reproduce the syllabic inventory of the Minoan
Greek language, here used in rebus fashion as at Byblos. This
writing occurs on seal stones (and their clay impressions), baked
clay, and metal and stone objects, most of these discovered at
Knossos and dating from 2000– 1400 BC (the script was concurrent
with Linear A). There exist about 140 different signs in all –
that is, 70 to 80 syllabic signs and their alloglyphs (different signs
with the same sound value), as well as logograms: human figures,
parts of the body, flora, fauna, boats and geometrical shapes.
Writing direction was open: from left to right, from right to left,
with every other line reversed, even spiral. That this script also
included logograms and numerals suggests that it was initially
used for book-keeping, among other things, until its replacement
in this function with its simplification, Linear A. Thereafter, like
Anatolian hieroglyphs, the Cretan hieroglyphic script appears to
have assumed a ceremonial role in Minoan Greek society,
reserved for sacred inscriptions, dedications and royal proclamations
on round clay disks.

In the 1950s, Linear B was largely udeciphered and found to encode an early form of Greek. Although the two systems share many symbols, this did not lead to a subsequent decipherment of Linear A. Using the values associated with Linear B in Linear A mainly produces unintelligible words. If it uses the same or similar syllabic values as Linear B, then its underlying language appears unrelated to any known language. This has been dubbed the Minoan language.\footnote{\url{http://www.people.ku.edu/~jyounger/LinearA/LinAIdeograms/}}

\begin{scriptexample}[]{Linear A}
\unicodetable{lineara}{  
\number"10600,"10610,"10620,"10630,"10640,"10650,"10660,"10670,
"10680,"10690,"106A0,"106B0,"106C0,"106D0,"106E0,"106F0,"10710,"10720,"10730,"10740,"10750,"10760,"10770}
\end{scriptexample}

Many of the characters form group and specialists name them such as vases in transliterations.

\begin{scriptexample}[]{Vases}
\begin{center}
\scalebox{3}{{\lineara \char"106A6}}
\scalebox{3}{{\lineara \char"106A5}}
\scalebox{3}{{\lineara \char"106A7}}
\scalebox{3}{{\lineara \char"106A9}}
\end{center}
\end{scriptexample}

Linear A contains more than 90 signs (open vowels and consonants+vowels) in regular use and a host of
logograms, many of which are ligatured with syllabograms and/or fractions; about 80\% of these
logograms do not appear in Linear B. While many of Linear A’s signs are also found in Linear B, some
signs are unique to A (e.g., A *301 and following), while some signs found in Linear B are not yet found
in Linear A (e.g., B 12, 14-15, 18-19, 25, 32-33, 36, 42-43, 52, 62-64, 68, 71-72, 75, 83-84, 89-91).

The Unicode Linear A encoding is broadly based on the GORILA ([{\arial ɡɔɹɪˈlɑː}]) catalogue
(Godart and Olivier 1976–1985)\citep{gorila}, which is the basic set of characters used in decipherment efforts.However, “ligatures” which consist of simple horizontal juxtapositions are not uniquely encoded here, as
these may be composed of their constituent parts. On the other hand, “ligatures” which consist of stacked
or touching elements have been encoded.\footnote{An online resource for ancient writing systems in the mediterranean\protect\url{http://lila.sns.it/mnamon/index.php?page=Risorse&id=19&lang=en}. } 








\section{Linear B}
\label{s:linearb}

The Linear B script is a syllabic writing system that was used on the island of Crete and
parts of the nearby mainland to write the oldest recorded variety of the Greek language.

Linear B clay tablets predate Homeric Greek by some 700 years; the latest tablets date from
the mid- to late thirteenth century bce. Major archaeological sites include Knossos, first
uncovered about 1900 by Sir Arthur Evans, and a major site near Pylos. The majority of
currently known inscriptions are inventories of commodities and accounting records.

The first tablets bearing the scripts were discovered by Sir Arthur Evans (1851-1941) while he was excavating the Minoan palace at Knossos in Crete. 


\medskip

\begin{figure}[ht]
\centering
\begin{minipage}{5cm}
\includegraphics[width=5cm]{./images/iklaina.jpg}
\end{minipage}\hspace{2em}
\begin{minipage}{7cm}
\captionof{figure}{Recently discovered fragment with Linear B, inscription. Found in an olive grove in what's now the village of Iklaina, the tablet was created by a Greek-speaking Mycenaean scribe between 1450 and 1350 B.C. (See \protect\href{http://news.nationalgeographic.com/news/2011/03/110330-oldest-writing-europe-tablet-greece-science-mycenae-greek/}{National Geographic}).}
\end{minipage}

\end{figure}


Early attempts to decipher the script failed until Michael Ventris, an architect and amateur
decipherer, came to the realization that the language might be Greek and not, as previously
thought, a completely unknown language. Ventris worked together with John Chadwick,
and decipherment proceeded quickly. The two published a joint paper in 1953.\citep{ventrisa}


\newfontfamily\linearb{Aegean.ttf}

Linear B was added to the Unicode Standard in April, 2003 with the release of version 4.0.

The Linear B Syllabary block is \unicodenumber{U+10000–U+1007F}. The Linear B Ideograms block is {\smallcps U+10080–U+100FF}. The Unicode block for the related Aegean Numbers is U+10100–U+1013F.

\begin{scriptexample}[]{Linear B}
\unicodetable{linearb}{"10000,"10010,"10020,"10030,"10040,"10050,"10060,"10070}

\captionof{table}{Linear B Typeset with command \protect\string\linearb\ and the \texttt{Aegean} font.}
\end{scriptexample}

\begin{scriptexample}[]{Linear B}
\unicodetable{linearb}{"10080,"10090,"100A0,"100B0,"100C0,"100D0,"100E0,"100F0}
\captionof{table}{Linear B Ideograms. Typeset with command \protect\string\linearb\ and the \texttt{Aegean} font.}
\end{scriptexample}


\begin{scriptexample}[]{Aegean Numbers}
\unicodetable{linearb}{"10100,"10110,"10110,"10120,"10130}

\captionof{table}{Aegean Numbers}
\end{scriptexample}





\section{Phaestos Disc}

\begin{figure}[htp]
\centering

\includegraphics[width=0.67\textwidth]{./phaistosdiscs.jpg}
\caption{Phaistos discs.}
\end{figure}


\PrintUnicodeBlock{./languages/phaistos.txt}{\linearb}


The ideograms are symbols, not pictures of the objects in question, e.g. one tablet records a tripod with missing legs, but the ideogram used is of a tripod with three legs. In modern transcriptions of Linear B tablets, it is typically convenient to represent an ideogram by its Latin or English name or by an abbreviation of the Latin name. Ventris and Chadwick generally used English; Bennett, Latin. Neither the English nor the Latin can be relied upon as an accurate name of the object; in fact, the identification of some of the more obscure objects is a matter of exegesis.

\bgroup

\linearb

Vessels
\let\l\unicodenumber

\begin{tabular}{l>{\smallcps}l>{\smallcps}l>{\smallcps}l>{\smallcps}l}
𐃟	&U+100DF	&200	&\l{sartāgo}	&\l{Boiling Pan}\\
𐃠	&U+100E0	&201	&\l{tripūs}	&\l{Tripod Cauldron}\\
𐃡	&U+100E1	&202	&\l{pōculum}	&\l{Goblet}\\
𐃢	&U+100E2	&203	&\l{urceus}	&\l{Wine Jar?}\\
𐃣	&U+100E3	&204  &\l{Tahirnea}	&\l{Ewer}\\
𐃤	&U+100E4	&205  &\l{Tnhirnula}	&\l{Jug}\\
𐃥	&U+100E5	&206	&\l{hydria}	&Hydria\\
𐃦	&U+100E6	&207	&\l{TRIPOD}  &AMPHORA\\
𐃧	&\l{U+100E7}	&\l{208}	&\l{PAT patera}	&\l{BOWL}\\
𐃨	&U+100E8	&209	&AMPH amphora	&AMPHORA\\
𐃩	&U+100E9	&210	&STIRRIP &JAR\\
𐃪	&U+100EA	&211	&WATER &BOWL?\\
𐃫	&U+100EB	&212	&SIT situla	&WATER JAR?\\
𐃬	&U+100EC	&213	&LANX lanx	&COOKING BOWL\\
\end{tabular}
















\egroup











\section{Cypriot Syllabary}
\label{s:cypriot}
The Cypriot or Cypriote syllabary is a syllabic script used in Iron Age Cyprus, from ca. the 11th to the 4th centuries BCE, when it was replaced by the Greek alphabet. A pioneer of that change was king Evagoras of Salamis. It is descended from the Cypro-Minoan syllabary, in turn a variant or derivative of Linear A. Most texts using the script are in the Arcadocypriot dialect of Greek, but some bilingual (Greek and Eteocypriot) inscriptions were found in Amathus.

\begin{figure}[htb]
\centering
\begin{minipage}{7cm}
\includegraphics[width=7cm]{./images/idalion-tablet.jpg}
\end{minipage}\hspace{1.5em}
\begin{minipage}{6cm}
\captionof{figure}{The bronze Idalion Tablet, from Idalium, (Greek: Ιδάλιον), is from the 5th century BCE Cyprus. The tablet is inscribed on both sides.
The script of the tablet is in the Cypro-Minoan syllabary, and the inscription is in Greek. The tablet records a contract between "the king and the city":[1] the topic of the tablet rewards a family of physicians, of the city, for providing free health services to individuals fighting an invading force of Persians.}
\end{minipage}
\end{figure}


The characters are \textit{syllabic}. There is one character for each  vowel, \textit{a, e, i, o, u,} and perhaps one for \textit{o}. There is no distinction between long and short vowels. The other characters represent what are called \textit{open syllables}\footnote{ If a syllable ends with a consonant, it is called a closed syllable. If a syllable ends with a vowel, it is called an open syllable. }, i.e., beginning with a consonant and ending with a vowel. 

No distinction is made between smooth, middle and rough mutes. The same character stands for τά τ\'ασs, δα in Εδαλιον ανδ δα ιν Αθανα  κε, κη, γε, γη, χε, χη. This fact constitutes the greatest difficulty in reading Cypriote.  

The Cypriot syllabary was added to the Unicode Standard in April, 2003 with the release of version 4.0.
The Unicode block for Cypriot is \unicodenumber{U+10800–U+1083F}. The Unicode block for the related Aegean Numbers is \unicodenumber{U+10100–U+1013F}.



\begin{scriptexample}[]{Cypriot Syllabary}
\unicodetable{cypriote}{"10800,"10810,"10820,"10830}

\cypriote \symbol{"10803}
\end{scriptexample}


\printunicodeblock{./languages/cyprus.txt}{\cypriote}

\endinput

\section{Old Persian}
\label{s:oldpersian}


Old Persian, like Hittite an Indo-European language, was written in cuneiforms as of the first millenium BC, mostly between 550 and 350. King Darius’ monumental inscription at
Bisothum – in Old Persian, Elamite and Neo-Babylonian – furnished
the ‘key’ to cuneiform’s decipherment and the reconstruction
of these languages.28 Darius’ Old Persian scribes
effected the most drastic simplification of the borrowed Near
Eastern script (illus. 35). They reduced the cuneiform inventory
to only 41 signs of both syllabic (ka) and phonemic (/k/) values.
Thus, Old Persian cuneiform is ‘half syllabic, half letter writing’.
29 It appears to be on the fence between the Babylonians’
cuneiforms and the Levantines’ consonantal writing, a hybrid
solution using only four logograms and 36 syllabo-phonemic
signs written in wedges. Of particular significance is the fact
that Old Persian also conveys the individual long and short
vowels /a/ (pronounced AH), /i/ (EE) and /u/ (OO) that the
Ugaritic system had conveyed a thousand years earlier.

Old Persian cuneiform is a semi-alphabetic cuneiform script that was the primary script for the Old Persian language. Texts written in this cuneiform were found in Persepolis, Susa, Hamadan, Armenia, and along the Suez Canal.[1] They were mostly inscriptions from the time period of Darius the Great and his son Xerxes. Later kings down to Artaxerxes III used corrupted forms of the language classified as “pre-Middle Persian”.

\begin{scriptexample}[]{Old Persian}
\unicodetable{oldpersian}{"103A0,"103B0,"103C0,"103D0}
\end{scriptexample}

Scholars today mostly agree that the Old Persian script was invented by about 525 BC to provide monument inscriptions for the Achaemenid king Darius I, to be used at Behistun. While a few Old Persian texts seem to be inscribed during the reigns of Cyrus the Great (CMa, CMb, and CMc, all found at Pasargadae), the first Achaemenid emperor, or Arsames and Ariaramnes (AsH and AmH, both found at Hamadan), grandfather and great-grandfather of Darius I, all five, specially the later two, are generally agreed to have been later inscriptions.
Around the time period in which Old Persian was used, nearby languages included Elamite and Akkadian. One of the main differences between the writing systems of these languages is that Old Persian is a semi-alphabet while Elamite and Akkadian were syllabic. In addition, while Old Persian is written in a consistent semi-alphabetic system, Elamite and Akkadian used borrowings from other languages, creating mixed systems.
\medskip

{\leftskip-1.25cm
\includegraphics[width=\textwidth+2.5cm]{./images/naghshe.jpg}
\captionof{figure}{Panoramic view of the Naqsh-e Rustam. This site contains the tombs of four Achaemenid kings, including those of Darius I and Xerxes. (\textit{Wikimedia})}
}
\section{Inscriptional Pahlavi}
\label{s:inscriptionalpahlavi}

Pahlavi or Pahlevi denotes a particular and exclusively written form of various Middle Iranian languages. The essential characteristics of Pahlavi are[1]
the use of a specific Aramaic-derived script, the Pahlavi script;
the high incidence of Aramaic words used as heterograms (called hozwārishn, "archaisms").

Pahlavi compositions have been found for the dialects/ethnolects of Parthia, Parsa, Sogdiana, Scythia, and Khotan.[2] Independent of the variant for which the Pahlavi system was used, the written form of that language only qualifies as Pahlavi when it has the characteristics noted above.


Pahlavi is then an admixture of
written Imperial Aramaic, from which Pahlavi derives its script, logograms, and some of its vocabulary.

spoken Middle Iranian, from which Pahlavi derives its terminations, symbol rules, and most of its vocabulary.
Pahlavi may thus be defined as a system of writing applied to (but not unique for) a specific language group, but with critical features alien to that language group. It has the characteristics of a distinct language, but is not one. It is an exclusively written system, but much Pahlavi literature remains essentially an oral literature committed to writing and so retains many of the characteristics of oral composition.

\begin{scriptexample}[]{Pahlavi}
\unicodetable{inscriptionalpahlavi}{"10B60,"10B70}
\end{scriptexample}
\newacro{ANE}{Ancient Near East}
\section{Imperial Aramaic}
\label{s:imperialaramaic}

\subsection{History}

Aramaic is the best-attested and longest-attested
member of the NW Semitic subfamily of languages
(which also includes inter alia \nameref{s:hebrew}, \nameref{s:phoenician},
\nameref{s:ugaritic}, Moabite, Ammonite, and Edomite). The
relatively small proportion of the biblical text
preserved in an Aramaic original (Dan 2:4–7:28; Ezra
4:8–68 and 7:12–26; Jeremiah 10:11; Gen 31:47 [two
words] as well as isolated words and phrases in
Christian Scriptures) belies the importance of this
language for biblical studies and for religious studies
in general, for Aramaic was the primary international
language of literature and communication throughout
the Near East from ca. 600 B.C.E. to ca. 700 C.E. and
was the major spoken language of Palestine, Syria,
and Mesopotamia in the formative periods of
Christianity and rabbinic Judaism. 



Aramaic survived over a period of 3,000 years, during which time its grammar, vocabulary and usage experienced great changes. Aramaic scholars found it useful to divide the several Aramaic dialects into periods, groups and subgroups based both on the chronology as well as the geography.

\begin{enumerate}
\item Old Aramaic
\item Imperial Aramaic
\item  Middle Aramaic
\item Late Aramaic
\item Modern Aramaic
\end{enumerate}


\subsection{Old Aramaic (to ca. 612 BCE)}
This period
witnessed the rise of the Arameans as a major force
in \ac{ANE} history, the adoption of their language as an
international language of diplomacy in the latter days
of the Neo-Assyrian Empire, and the dispersal of
Aramaic-speaking peoples from Egypt to Lower
Mesopotamia as a result of the Assyrian policies of
deportation. The scattered and generally brief
remains of inscriptions on imperishable materials
preserved from these times are enough to
demonstrate that an international standard dialect had
not yet been developed. The extant texts may be
grouped into several dialects:

\subsection{Middle Aramaic (to ca. 250 C.E.)}
In the Hellenistic and Roman periods, Greek replaced
Aramaic as the administrative language of the Near
East, while in the various Aramaic-speaking regions
the dialects began to develop independently of one
another. Written Aramaic, however, as is the case
with most written languages, by providing a
somewhat artificial, cross-dialectal uniformity,
continued to serve as a vehicle of communication
within and among the various groups. For this
purpose, the literary standard developed in the
previous period, Standard Literary Aramaic, was
used, but lexical and grammatical differences based
on the language(s) and dialect(s) of the local
population are always evident. It is helpful to divide
the texts surviving from this period into two major
categories: epigraphic and canonical.

\subsection{Late Aramaic (to ca. 1200 C.E.)}
The bulk of
our evidence for Aramaic comes from the vast
literature and occasional inscriptions of this period.
During the early centuries of this period Aramaic
dialects were still widely spoken. During the second
half of this period, however, Arabic had already
displaced Aramaic as the spoken language of much
of the population. Consequently, many of our texts
were composed and/or transmitted by persons whose
Aramaic dialect was only a learned language.
Although the dialects of this period were previously
divided into two branches (Eastern and Western), it
now seems best to think rather of three: Palestinian,
Syrian, and Babylonian.

The Aramaic alphabet is adapted from the \nameref{s:phoenician} alphabet and became distinctive from it by the 8th century BCE.  The letters all represent consonants, some of which are \emph{matres lectionis}, which also indicate long vowels.

\subsection{Modern Aramaic (to the present day)}

These dialects can be divided into the same three
geographic groups.

\begin{description}

\item[a. Western]
Here Aramaic is still spoken only in
the town of Ma’lula (ca. 30 miles NNE of Damascus)
and surrounding villages. The vocabulary is heavily
Arabized.

\item[b. Syrian]
Western Syrian (Turoyo) is the language
of Jacobite Christians in the region of Tur-Abdin in
SE Turkey. This dialect is the descendant of
something very like classical Syriac. Eastern Syrian
is spoken in the Kurdistani regions of Iraq, Iran,
Turkey, and Azerbaijan by Christians and, formerly,
by Jews. Substantial communities of the former are
now found in North America. The Jewish speakers
have mostly settled in Israel. These dialects are
widely spoken by their respective communities and
have been studied extensively during the past
century. It has become clear that they are not the
descendants of any known literary Aramaic dialect.

\item[c. Babylonian] 

\nameref{s:mandaic} is still used, at least until
recently, by some Mandaeans in southernmost Iraq
and adjacent areas in Iran.

In addition, in recent years classical \pageref{s:syriac} has
undergone somewhat of a revival as a learned vehicle
of communication for Syriac Christians, both in the
Middle East and among immigrant communities in
Europe and North America.
\end{description}

\begin{figure}[htbp]
\centering
\includegraphics[width=0.6\textwidth]{./images/elephantine-papyrus.jpg}

\caption{The Elephantine papyri are ancient Jewish papyri dating to the 5th century BC, requesting the rebuilding of a Jewish temple. It also name three persons mentioned in Nehemiah: Darius II, Sanballat the Horonite and Johanan the high priest.}

\end{figure}


\subsection{Alphabet and typesetting}

The Aramaic alphabet is historically significant, since virtually all modern Middle Eastern writing systems can be traced back to it, as well as numerous non-Chinese writing systems of Central and East Asia. This is primarily due to the widespread usage of the Aramaic language as both a \emph{lingua franca} and the official language of the Neo-Assyrian Empire, and its successor, the Achaemenid Empire. Among the scripts in modern use, the Hebrew alphabet bears the closest relation to the Imperial Aramaic script of the 5th century BC, with an identical letter inventory and, for the most part, nearly identical letter shapes. 

Writing systems that indicate consonants but do not indicate most vowels (like the Aramaic one) or indicate them with added diacritical signs, have been called abjads by Peter T. Daniels to distinguish them from later alphabets, such as Greek, that represent vowels more systematically. This is to avoid the notion that a writing system that represents sounds must be either a syllabary or an alphabet, which implies that a system like Aramaic must be either a syllabary (as argued by Gelb) or an incomplete or deficient alphabet (as most other writers have said); rather, it is a different type.

The Imperial Aramaic alphabet was added to the Unicode Standard in October 2009 with the release of version 5.2.
The Unicode block for Imperial Aramaic is \unicodenumber{U+10840–U+1085F}.

\begin{scriptexample}[]{Aramaic}
\unicodetable{imperialaramaic}{"10840,"10850}
\end{scriptexample}




\PrintUnicodeBlock{./languages/imperial-aramaic.txt}{\imperialaramaic}
\input{./languages/ogham}
\section{Ancient Anatolian Alphabets}
\label{s:anatolian}
The Anatolian scripts described in this section all date from the first millenium BCE, and were used to write various ancient Indo-European languages of western and southwestern Anatolia (now Turkey). All are related to the Greek script and are probably adaptations of it. 



\section{Lycian}
\label{sec:lycian}
The Lycian alphabet was used to write the Lycian language. It was an extension of the Greek alphabet, with half a dozen additional letters for sounds not found in Greek. It was largely similar to the Lydian and the Phrygian alphabets.

 
\begin{scriptexample}[]{Lydian}
\unicodetable{lydian}{"10280,"10290}

Typeset with the \idxfont{Aegean.ttf} and the command \cmd{\lycian}
\end{scriptexample}

\begin{figure}[htb]

\begin{minipage}[b]{0.5\textwidth}
\includegraphics[width=1\linewidth]{./images/xanthian-obelisk.jpg}
\end{minipage}\hspace*{1em}
\begin{minipage}[b]{0.45\textwidth}
\captionof{figure}{Part of the Xanthian obelisk inscription. The Xanthian Obelisk, also known as the Xanthos or Xanthus Stele, the Xanthos or Xanthus Bilingual, the Inscribed Pillar of Xanthos or Xanthus, the Harpagus Stele, and the Columna Xanthiaca, is a stele bearing an inscription currently believed to be trilingual, found on the acropolis of the ancient Lycian city of Xanthos, or Xanthus, near the modern town of Kınık in southern Turkey. The three languages are Ancient Greek, Lycian and Milyan (the last two are Anatolian languages and were previously known as Lycian A and Lycian B respectively).}
\end{minipage}
\end{figure}


\printunicodeblock{./languages/lycian.txt}{\lycian}


\section{Lydian}
\label{sec:lydian}
 Lydian script was used to write the Lydian language. That the language preceded the script is indicated by names in Lydian, which must have existed before they were written. Like other scripts of Anatolia in the Iron Age, the Lydian alphabet is a modification of the East Greek alphabet, but it has unique features. The same Greek letters may not represent the same sounds in both languages or in any other Anatolian language (in some cases it may). Moreover, the Lydian script is alphabetic.



Early Lydian texts are written both from left to right and from right to left. Later texts are exclusively written from right to left. One text is boustrophedon. Spaces separate words except that one text uses dots. Lydian uniquely features a quotation mark in the shape of a right triangle.

The first codification was made by Roberto Gusmani in 1964 in a combined lexicon (vocabulary), grammar, and text collection.

\begin{scriptexample}[]{Lydian}
\unicodetable{lydianfont}{"10920,"10930}

\medskip

Typeset with the \idxfont{Aegean.ttf} and the command \cmd{\lydian}
\end{scriptexample}

Examples of words

\bgroup\lydian
𐤬𐤭𐤠  - Ora - "Month"

𐤬𐤳𐤦𐤭𐤲𐤬𐤩  - Laqrisa - "Wall"

𐤬𐤭𐤦𐤡  - "House, Home"

\egroup

Herodotus Hdt. 1.94 
Chapter on the Lydians is well known, but in order to evaluate it properly it will be
helpful to recall exactly what it says54:

\begin{latexquotation}
The Lydians have about the same customs as the Greeks, except that the
Lydians prostitute their female children. The Lydians are the first people
we know to have coined money of silver and gold, and they were the first to
be shopkeepers. The Lydians themselves also claim the invention of the
games that both they and the Greeks now play. They say that the invention
occurred at the same time that they colonized Tyrsenia. What they say
about these things goes like this (the following is in indirect discourse):
In the reign of Atys, son of Manes, there was a terrible famine
throughout Lydia. Although in hard straits, the Lydians persevered for
some time. But finally, when there was no let-up, they sought respite,
some trying one thing and others another. It was then that they invented
dice, and astragals, and ball, and all the other kinds of games, except for
draughts. For the Lydians don't claim to have invented draughts. After
their inventions, this is what they did about the famine. Every second
day they would play, all day, so as not to want food, and on the day
between they would eat, and not play. In this way they persevered for
eighteen years. Since the evil did not abate, but pressed them even
worse, their king divided them up into two parts, by lot: the one group
for staying on, the other group to emigrate from the country. And the
king himself was to be in charge of the group that remained, while in
charge of the departing group was the king's son, whose name was
Tyrsenos. The group whose lot it was to depart from the land went down
to Smyrna and built boats. They put everything they needed into the
boats and sailed away in search of life and land; passing by many
nations, they sailed until they reached the Ombrikians, where they built
cities for themselves and they still live there today. Instead of "Lydians",
they adopted a new name from the king's son, the man who led them.
Taking their eponym from him, they were called Tyrsenoi.

Well, then, the Lydians were enslaved by the Persians\ldots
\end{latexquotation}



\section{Carian}
\label{sec:carian}
The Carian alphabets are a number of regional scripts used to write the Carian language of western Anatolia. They consisted of some 30 alphabetic letters, with several geographic variants in Caria and a homogeneous variant attested from the Nile delta, where Carian mercenaries fought for the Egyptian pharaohs. They were written left-to-right in Caria (apart from the Carian–Lydian city of Tralleis) and right-to-left in Egypt. Carian was deciphered primarily through Egyptian–Carian bilingual tomb inscriptions, starting with John Ray in 1981; previously only a few sound values and the alphabetic nature of the script had been demonstrated. The readings of Ray and subsequent scholars were largely confirmed with a Carian–Greek bilingual inscription discovered in Kaunos in 1996, which for the first time verified personal names, but the identification of many letters remains provisional and debated, and a few are wholly unknown.

Carian was added to the Unicode Standard in April, 2008 with the release of version 5.1. It is encoded in Plane 1 (Supplementary Multilingual Plane).
The Unicode block for Carian is \unicodenumber{U+102A0–U+102DF}:

\begin{scriptexample}[]{Carian}
\unicodetable{carian}{"102A0,"102B0,"102C0,"102D0}
\end{scriptexample}

\newenvironment{carianscript}
{\carian
 \def\A{\char"102A0}
 \def\Uuu{\char"102A4} 
 \def\R{\char"102A5}
}
{}

\PrintUnicodeBlock{./languages/carian.txt}{\carian}

\begin{carianscript}
\A \Uuu \R
\end{carianscript}



\section{Phoenician}
\label{s:phoenician}
\arial

The Phoenician alphabet and its successors were widely used over a broad area surrounding the Mediterranean Sea.

\let\phoenician\lycian

\begin{scriptexample}[]{Phoenician}

\unicodetable{phoenician}{"10900,"10910}

\end{scriptexample}

The Phoenician alphabet, called by convention the Proto-Canaanite alphabet for inscriptions older than around 1200 BCE, is the oldest verified consonantal alphabet, or abjad.[1] It was used for the writing of Phoenician, a Northern Semitic language, used by the civilization of Phoenicia. It is classified as an abjad because it records only consonantal sounds (matres lectionis were used for some vowels in certain late varieties).

Phoenician became one of the most widely used writing systems, spread by Phoenician merchants across the Mediterranean world, where it evolved and was assimilated by many other cultures. The Aramaic alphabet, a modified form of Phoenician, was the ancestor of modern Arabic script, while Hebrew script is a stylistic variant of the Aramaic script. The Greek alphabet (and by extension its descendants such as the Latin, the Cyrillic, and the Coptic) was a direct successor of Phoenician, though certain letter values were changed to represent vowels.

\begin{figure}[ht]
\includegraphics[width=\textwidth]{./images/phoenician.jpg}
\captionof{figure}{
Phoenician votive inscription from Idalion (Cyprus), 390 BC. BM 125315 from The Early Alphabet by John F. Healy.}
\end{figure}

As the letters were originally incised with a stylus, most of the shapes are angular and straight, although more cursive versions are increasingly attested in later times, culminating in the Neo-Punic alphabet of Roman-era North Africa. Phoenician was usually written from right to left, although there are some texts written in boustrophedon.


\PrintUnicodeBlock{./languages/phoenician.txt}{\phoenician}


\input{./languages/palmyrene}



\input{./languages/mandaic}

\input{./languages/hieroglyphics}
\section{Meroitic}
\label{s:meroitic}

The Meroitic script is an alphabetic script, used to write the Meroitic language of the Kingdom of Meroë in Sudan. It was developed in the Napatan Period (about 700–300 BCE), and first appears in the 2nd century BCE. For a time, it was also possibly used to write the Nubian language of the successor Nubian kingdoms. Its use was described by the Greek historian Diodorus Siculus (c. 50 BCE).

Although the Meroitic alphabet did continue in use by the Nubian kingdoms that succeeded the Kingdom of Meroë, it was replaced by the Coptic alphabet with the Christianization of Nubia in the sixth century CE. The Nubian form of the Coptic alphabet retained three Meroitic letters.

The script was deciphered in 1909 by Francis Llewellyn Griffith, a British Egyptologist, based on the Meroitic spellings of Egyptian names. However, the Meroitic language itself has yet to be translated. In late 2008 the first complete royal dedication was found,[1] which may help confirm or refute some of the current hypotheses.

The longest inscription found is in the Museum of Fine Arts, Boston.

\newfontfamily\meroitic{Nilus.ttf}^^A

\unicodetable{meroitic}{"109A0,"109B0,"109C0,"109E0,"109F0}%


The examples here use the \idxfont{Nilus.ttf} font of George Douros\footnote{\url{http://users.teilar.gr/~g1951d/}}.

The name of the queen of Amenhotp III is rendered Teie, i.e. Teye, in the Armana 
tablets. The name of the city dedicated to her in Nubia was therefore pronounced 
Ha-Teye and appears in Meroitic as eyita (1916:119)









\input{./languages/ugaritic}
\input{./languages/sumero-akkadian}

\input{./languages/inscriptional-parthian}


\input{./languages/olditalic}
\section{Old South Arabian}
\label{s:oldsoutharabian}

\index{Ancient and Historic Scripts>Old South Arabian}
\index{Old South Arabian fonts>Noto Sans Old South Arabian}
\index{alphabets>Yemeni}

\newfontfamily\oldsoutharabian{NotoSansOldSouthArabian-Regular.ttf}

The ancient Yemeni alphabet (Old South Arabian ms3nd; modern Arabic: {\arabicfont المُسنَد‎}  musnad) branched from the Proto-Sinaitic alphabet in about the 9th century BC. It was used for writing the Old South Arabian languages of the Sabaic, Qatabanic, Hadramautic, Minaic (or Madhabic), Himyaritic, and proto-Ge'ez (or proto-Ethiosemitic) in Dʿmt. The earliest inscriptions in the alphabet date to the 9th century BC in Akkele Guzay, Eritrea[3] and in the 10th century BC in Yemen. There are no vowels, instead using the \emph{mater lectionis} to mark them.

Its mature form was reached around 500 BC, and its use continued until the 6th century AD, including Old North Arabian inscriptions in variants of the alphabet, when it was displaced by the Arabic alphabet.[4] In Ethiopia and Eritrea it evolved later into the Ge'ez alphabet,[1][2] which, with added symbols throughout the centuries, has been used to write Amharic, Tigrinya and Tigre, as well as other languages (including various Semitic, Cushitic, and Nilo-Saharan languages).

It is usually written from right to left but can also be written from left to right. When written from left to right the characters are flipped horizontally (see the photo).
The spacing or separation between words is done with a vertical bar mark (\textbar).
Letters in words are not connected together.

Old South Arabian script does not implement any diacritical marks (dots, etc.), differing in this respect from the modern Arabic alphabet.

\begin{scriptexample}[]{South Arabian}
\unicodetable{oldsoutharabian}{"10A60,"10A70}
\end{scriptexample}

Support in \latexe is provided via Peter Wilson's package \pkgname{sarabian}\citep{sarabian}. The package provides all the |metafont| sources as well as transliteration commands and other utilities \seedocs{\SARAB}. The package is based on fonts developed originally by Alan Stanier of Essex University.

The package provides the commands \docAuxCmd{sarabfamily} that selects the South Arabian font family. The family name is \texttt{sarab}. Another command \docAuxCmd{textsarab}\meta{text} typesets \meta{text} in the South Arabian font. The package provides two ways of accessing
glyphs: (a) by \texttt{ASCII} character commands, and (b) via commands. These are illustrated in
Table~\ref{sarabian1} which is a modified version of that provided in the Comprehensive Symbols.



\def\SAtdu{\oldsoutharabian\char"10A77}

A comparison between  the unicode and the rendering (scaled 5) \pkgname{sarabian} is shown below.

\centerline{\scalebox{3}{\SAtdu} \scalebox{3}{\textsarab{\SAtd}}}

There is no real advantage in using unicode fonts, if all you interested is to write some South Arabian text for inscriptions. 

\begin{symtable}[SARAB]{\SARAB\ South Arabian Letters}
\index{South Arabian alphabet}
\index{alphabets>South Arabian}
\label{sarabian1}
\begin{tabular}{*4{ll@{\qquad}}ll}
\K[\textsarab{\SAa}]\SAa   & \K[\textsarab{\SAz}]\SAz   & \K[\textsarab{\SAm}]\SAm   & \K[\textsarab{\SAsd}]\SAsd & \K[\textsarab{\SAdb}]\SAdb \\
\K[\textsarab{\SAb}]\SAb   & \K[\textsarab{\SAhd}]\SAhd & \K[\textsarab{\SAn}]\SAn   & \K[\textsarab{\SAq}]\SAq   & \K[\textsarab{\SAtb}]\SAtb \\
\K[\textsarab{\SAg}]\SAg   & \K[\textsarab{\SAtd}]\SAtd & \K[\textsarab{\SAs}]\SAs   & \K[\textsarab{\SAr}]\SAr   & \K[\textsarab{\SAga}]\SAga \\
\K[\textsarab{\SAd}]\SAd   & \K[\textsarab{\SAy}]\SAy   & \K[\textsarab{\SAf}]\SAf   & \K[\textsarab{\SAsv}]\SAsv & \K[\textsarab{\SAzd}]\SAzd \\
\K[\textsarab{\SAh}]\SAh   & \K[\textsarab{\SAk}]\SAk   & \K[\textsarab{\SAlq}]\SAlq & \K[\textsarab{\SAt}]\SAt   & \K[\textsarab{\SAsa}]\SAsa \\
\K[\textsarab{\SAw}]\SAw   & \K[\textsarab{\SAl}]\SAl   & \K[\textsarab{\SAo}]\SAo   & \K[\textsarab{\SAhu}]\SAhu & \K[\textsarab{\SAdd}]\SAdd \\
\end{tabular}

\bigskip
\begin{tablenote}
  \usefontcmdmessage{\textsarab}{\sarabfamily}.  Single-character
  shortcuts are also supported: Both
  ``\verb+\textsarab{\SAb\SAk\SAn}+'' and ``\verb+\textsarab{bkn}+''
  produce ``\textsarab{bkn}'', for example.  \seedocs{\SARAB}.
\end{tablenote}
\end{symtable}



\section{Avestan script}
\label{s:avestan}
The Avestan alphabet is a writing system developed during Iran's Sassanid era (AD 226–651) to render the Avestan language.
As a side effect of its development, the script was also used for Pazend, a method of writing Middle Persian that was used primarily for the Zend commentaries on the texts of the Avesta. In the texts of Zoroastrian tradition, the alphabet is referred to as \emph{din dabireh} or \emph{din dabiri}, Middle Persian for "the religion's script".

The Avestan alphabet was replaced by the Arabic alphabet after Persia converted to Islam during the 7th century CE. 


Notable Features

The alphabet is written from right to left, in the same way as Syriac, Arabic and Hebrew.
See more at: \url{http://www.iranchamber.com/scripts/avestan_alphabet.php#sthash.ZRu7AkEb.dpuf}


\begin{scriptexample}[]{Avestan}
\ifxetex\TeXXeTstate=1
\beginR\fi
\avestan\raggedleft
𐬄	
𐬅	
𐬆	
𐬇	
𐬈	
𐬉	
𐬊	
𐬋	
𐬌	
𐬍	
𐬎	
𐬏	
𐬐	
	
𐬒	
𐬓	
𐬔	
	
𐬖	
𐬗	
𐬘	
𐬙	
𐬚	
𐬛	
𐬜	
𐬝	
𐬞	
𐬟	
𐬠	
𐬡	
𐬢	
𐬣	
𐬤	
𐬥	
𐬦	
𐬧	
𐬨	
𐬩	
𐬪	
𐬫	
𐬬	
𐬭	
𐬮	
𐬯	
𐬰	
𐬱	
𐬲	
𐬳	
𐬴	
𐬵	
\ifxetex\endR
\TeXXeTstate=0\fi
\end{scriptexample}

The recent Google font \idxfont{NotoSansAvestan-Regular.ttf} can be used to typeset the Avestan script, but I am not sure if it is suitable for any serious study of the language.
\input{./languages/old-turkic}
\input{./languages/runic}
\input{./languages/glagolithic}

\ifscriptolmec
  \section{Epi-Olmec}
\label{s:olmec}
Epi-Olmec is an ancient Mesoamerican logosyllabic script which has been deciphered by Terrence Kaufman and John Justeson. A complete description of the script has been described by \cite{kaufman}. The most famous inscription is on the Tuxtla Statuette. The Tuxtla Statuette is a small 6.3 inch (16 cm) rounded greenstone figurine, carved to resemble a squat, bullet-shaped human with a duck-like bill and wings. Most researchers believe the statuette represents a shaman wearing a bird mask and bird cloak.[1] It is incised with 75 glyphs of the Epi-Olmec or Isthmian script, one of the few extant examples of this very early Mesoamerican writing system. The Tuxtla Statuette is particularly notable in that its glyphs include the Mesoamerican Long Count calendar date of March 162 CE, which in 1902 was the oldest Long Count date discovered. A product of the final century of the Epi-Olmec culture, the statuette is from the same region and period as La Mojarra Stela 1 and may refer to the same events or persons.[3] Similarities between the Tuxtla Statuette and Cerro de las Mesas Monument 5, a boulder carved to represent a semi-nude figure with a duckbill-like buccal mask, have also been noted.[4]

\begin{figure}[ht]
\centering
\includegraphics[height=0.35\textheight]{./images/tuxtla-statuette.png}\hspace{1em} 
\includegraphics[height=0.35\textheight]{./images/tuxtla-statuette-01.jpg}
\caption{Frontal view of the Tuxtla Statuette. Note the Mesoamerican Long Count calendar date of March 162 CE (8.6.2.4.17) down the front of the statuette. The left figure is from wikipedia and the right from the original \protect\href{http://www.readcube.com/articles/10.1525/aa.1907.9.4.02a00030}{Holmes} paper.\citep{holmes1907}}
\end{figure}

\subsection{The epiolmec package}

The script has not been as yet encoded as by the Unicode consortium. Syropoulos \citep{syropoulos} created a font for the script and also wrote an article for TUGboat. Interestingly the paper describes the procedure used to develop the font. The package \pkgname{epiolmec} which is available both in \TeX live and Mik\tex, provides commands to access the glyphs. It is also possibly easier to typeset the script using traditional \latexe techniques, as they provide transcription commands rather than using a unicode font with the glyphs allocated in the private area directly.

\begin{verbatim}
\documentclass{article}
\usepackage{epiolmec,multicol}
\begin{document}
  \begin{center}
      \begin{minipage}{80pt}
      \begin{multicols}{3}
         \EOku\\ \EOji\\  
         \EOtze\\ \EOstep \\
       \end{multicols}    
     \end{minipage}       
  \end{center}
\end{document}
\end{verbatim}

Since the Epi-Olmec script is a logosyllagraphy we
need some practical way to access the symbols of the
script. Originally Syropoulos used the Ω translation
process that mapped words and “syllables” to the
corresponding glyphs of the font. In this way one obtains
a natural way for typing in Epi-Olmec texts. In addition,
in order to avoid the problem mentioned above,
he used a wrapper that typesets the text vertically.
For short texts \cmd{\shortstacks} is adequate, while
for longer texts, he used a |multicols| environment
inside a relatively narrow minipage. 

\begin{scriptexample}{Epi-Olmec}
\bgroup
\HUGE
\centering
\EOpi   \EOofficerI \EOofficerII \EOofficerIII

\captionof{figure}{The output of \string\EOpi, \string\EOofficerI, \string\EOofficerII\ and \string\EOofficerIII\ commands. }
\egroup
\end{scriptexample}

\subsection{Numbering System}\index{Epi-Olmec>vigesimal system}

The Epi-Olmec people used the same numbering system  
 as the Maya. Their numbering system was a vigesimal system and
 the digits were written in a top-down fashion. Thus, we need a macro
 that will typeset numbers in this fashion when it is used with \LaTeX\
 (actually $\epsilon$-\LaTeX). In addition, we need a macro that will
 just output the vigesimal digits. Such a macro could be used with
 $\Lambda$ with the |LTL| text and paragraph directions. To recapitulate,
 we need to define two macros that will basically typeset vigesimal numbers
 in either horizontal or vertical mode.

 For the various calculations that are performed, we need at least three
 counter variables. The fourth is needed for the macro that typesets the
 vigesimal numbers vertically and its usage is explained below. 

\begin{scriptexample}{EpicOtmec}
\def\textb#1{\text{\makebox[6em]{\hss#1~~   \protect\string#1\hfill}}}
\begin{multicols}{3}
\bgroup
\parindent0pt
$\textb{\EOzero}=0$\\
$\textb{\EOi} = 1$\\
$\textb{\EOii} = 2$\\
$\textb{\EOiii} =3$\\
$\textb{\EOiv}  =4$\\
$\textb{\EOv}   =5$\\
$\textb{\EOvi}  =6$\\
$\textb{\EOvii} =7$\\
$\textb{\EOviii} =8$\\
$\textb{\EOix} =9$\\ 
$\textb{\EOx} =10$\\
$\textb{\EOxi} =11$\\
$\textb\EOxii =12$\\
$\textb{\EOxiii} =13$\\
$\textb{\EOxiv} =14$\\
$\textb{\EOxv} =15$\\
$\textb{\EOxvi} =16$\\
$\textb\EOxvii =17$\\
$\textb{\EOxviii} =18$\\
$\textb{\EOxix} =19$\\
$\textb{\EOxx} =20$\\
\egroup
\end{multicols}
\end{scriptexample}


%% TODO add to index all symbols

\begin{multicols}{4}
\bgroup
\def\K#1{\makebox[3em]{{\color{theunicodesymbolcolor}\hss#1\hfill}} \string#1}
\parindent0pt
\K\EOSpan\\ 
\K\EOJI \\
\K\EOvarji\\ 
\K\EOvarki \\
\K\EOpi \\
\K\EOpe \\
\K\EOpuu \\
\K\EOpa \\
\K\EOvarpa\\ 
\K\EOpu \\
\K\EOpo \\
\K\EOti \\
\K\EOte \\
\K\EOtuu \\
\K\EOta \\
\K\EOtu \\
\K\EOto \\
\K\EOtzi \\
\K\EOtze \\
\K\EOtzuu \\
\K\EOtza \\
\K\EOvartza\\ 
\K\EOtzu \\
\K\EOki \\
\K\EOke \\
\K\EOkuu \\
\K\EOvarkuu\\ 
\K\EOku\\ 
\K\EOko \\
\K\EOSi \\
\K\EOvarSi\\ 
\K\EOSuu \\
\K\EOSa \\
\K\EOSu \\
\K\EOSo \\
\K\EOsi \\
\K\EOvarsi\\ 
\K\EOsuu \\
\K\EOsa \\
\K\EOsu \\
\K\EOji \\
\K\EOje \\
\K\EOja \\
\K\EOvarja\\ 
\K\EOju \\
\K\EOjo \\
\K\EOmi \\
\K\EOme \\
\K\EOmuu \\
\K\EOma \\
\K\EOni \\
\K\EOvarni\\
\K\EOne \\
\K\EOnuu \\
\K\EOna \\
\K\EOnu \\
\K\EOwi \\
\K\EOwe \\
\K\EOwuu \\
\K\EOvarwuu\\
\K \EOwa\\
\K\EOwo \\
\K\EOye \\
\K\EOyuu \\
\K\EOya \\
\K\EOkak \\
\K\EOpak \\
\K\EOpuuk\\
\K\EOyaj \\
\K\EOScorpius\\
\K\EODealWith\\
\K\EOYear \\
\K\EOBeardMask \\
\K\EOBlood \\
\K\EOBundle \\
\K\EOChop \\
\K\EOCloth \\
\K\EOSaw \\
\K\EOGuise \\
\K\EOofficerI\\
\K\EOofficerII \\
\K\EOofficerIII \\
\K\EOofficerIV \\
\K\EOKing \\
\K\EOloinCloth \\
\K\EOlongLipII \\
\K\EOLose \\
\K\EOmexNew \\
\K\EOMiddle \\
\K\EOPlant \\
\K\EOPlay \\
\K\EOPrince \\
\K\EOSky \\
\K\EOskyPillar \\
\K\EOSprinkle \\
\K\EOstarWarrior\\
\K\EOTitleII \\
\K\EOtuki \\
\K\EOtzetze\\
\K\EOChronI \\
\K\EOPatron \\
\K\EOandThen\\
\K\EOAppear \\
\K\EODeer \\
\K\EOeat \\
\K\EOPatronII \\
\K\EOPierce \\
\K\EOkij \\
\K\EOstar  \\
\K\EOsnake \\
\K\EOtime \\
\K\EOtukpa  \\
\K\EOflint \\
\K\EOafter \\
\K\EOvarBeardMask \\
\K\EOBedeck \\
\K\EObrace \\
\K\EOflower  \\
\K\EOGod \\
\K\EOMountain \\
\K\EOgovernor \\
\K\EOHallow \\
\K\EOjaguar \\
\K\EOSini \\
\K\EOknottedCloth \\
\K\EOknottedClothStraps \\
\K\EOLord \\
\K\EOmacaw \\
\K\EOmonster \\
\K\EOmacawI \\
\K\EOskyAnimal\\
\K\EOnow \\
\K\EOTitleIV \\
\K\EOpenis \\
\K\EOpriest  \\
\K\EOstep\\
\K\EOsing \\
\K\EOskin \\
\K\EOStarWarrior \\
\K\EOsun \\
\K\EOthrone\\
\K\EOTime \\
\K\EOHallow \\
\K\EOTitle \\
\K\EOturtle \\
\K\EOundef \\
\K\EOGoUp \\
\K\EOLetBlood \\
\K\EORain \\
\K\EOset \\
\K\EOvarYear\\
\K\EOFold \\
\K\EOsacrifice \\
\K\EObuilding \\
\egroup
\end{multicols} 

\subsection{Technical}

The font is defined with the local encoding \texttt{LEO}. 

\begin{verbatim}
\DeclareFontEncoding{LEO}{}{}
\DeclareFontSubstitution{LEO}{cmr}{m}{n}
\DeclareFontFamily{LEO}{cmr}{\hyphenchar\font=-1}
\end{verbatim}

Note the |\hyphenchar\font=-1| that disables hyphenation in the |\DeclareFontFamily|  declaration. You cannot behead the \EOofficerII\ in order to hyphenate the text!



\fi

%\input{./languages/greek}
%\chapter{Middle Eastern Scripts}
\label{ch:middleeasternscripts}

The scripts in this section have a common origin in the ancient \nameref{s:phoenician} alphabet. They include:

\begin{center}
\begin{tabular}{ll}
\nameref{hebrew} & \nameref{s:samaritan}\\
Arabic & Thaana\\
\nameref{s:syriac} &\\
\end{tabular}
\end{center}

The Hebrew script is used in Israel and for languages of the Diaspora. The Arabic script is
used to write many languages throughout the Middle East, North Africa, and certain parts
of Asia. The Syriac script is used to write a number of Middle Eastern languages. These
three also function as major liturgical scripts, used worldwide by various religious groups.

The Samaritan script is used in small communities in Israel and the Palestinian Territories
to write the Samaritan Hebrew and Samaritan Aramaic languages. The Thaana script is
used to write Dhivehi, the language of the Republic of Maldives, an island nation in the
middle of the Indian Ocean. 

Text in these scripts is written from right to left. Arabic and Syriac are cursive scripts even when typeset, unlike Hebrew, Samaritan  and Thaana, where letters are unconnected. Most letters in Arabic and Syriac assume different forms depending on their position in a word. Shaping rules are not required for Hebrew because only five letters have position-dependent forms, and these forms are separately encoded.

Historically, Middle Eastern  scripts did not write short vowels. In modern scripts they are represented  by marks positioned above or below a consonantal letter. Vowels and other
marks of pronunciation (``vocalization’’) are encoded as combining characters, so support
for vocalized text necessitates use of composed character sequences. Yiddish, Syriac, and
Thaana are normally written with vocalization; Hebrew, Samaritan, and Arabic are usually written unvocalized. 


\input{./languages/samaritan}

\section{Hebrew}
\label{s:hebrew}



To properly typeset Hebrew texts you first need to choose an appropriate font and also set the directionality of the text. This
is done using the etex commands \docAuxCommand{beginL} and \docAuxCommand{beginR} 

For \XeTeX\ you also need to add near the top of your document |\TeXXeTstate=1|. The package \pkgname{bidi} can be used to set all parameters. Be warned that it redefines almost all of \latexe's commands, so for short mixed texts, I wouldn't recommend its usage. 



The Hebrew alphabet (Hebrew: אָלֶף־בֵּית עִבְרִי[a], alefbet ʿIvri ), known variously by scholars as the Jewish script, square script, block script, is used in the writing of the Hebrew language, as well as other Jewish languages, most notably Yiddish, Ladino, and Judeo-Arabic. There have been two script forms in use; the original old Hebrew script is known as the paleo-Hebrew script (which has been largely preserved, in an altered form, in the Samaritan script), while the present "square" form of the Hebrew alphabet is a stylized form of the Assyrian script. Various "styles" (in current terms, "fonts") of representation of the letters exist. There is also a cursive Hebrew script, which has also varied over time and place. On Windows you can use the \texttt{Miriam} font or \texttt{Arial Unicode MS} or \texttt{Miriam Fixed}.
\medskip

\topline
\bgroup
\ifxetex\TeXXeTstate=1\fi
\raggedleft\arial{}\beginR

הכתב הכנעני הקדום הלך והתפשט וסימניו היו מוכרים כל כך, עד כי המשתמשים בו התחילו "להתעצל" בהשלמת הציורים, והניחו כי הקורא יבין גם מתוך שרטוטים סכמתיים באיזו אות מדובר. כך, למשל, הפך הראש למשולש עם צוואר; כף היד מלאת האצבעות הפכה לשרטוט דל, ומהדג נותר רק הזנב. כשהעברים אמצו את הכתב הכנעני הם התקשו לזהות חלק מהציורים המקוריים והניחו למשל כי הסימן המתאר את המילה "זהה" הוא כלי נשק; שזנב הדג המשולש הוא דלת, ושדווקא הנחש הוא דג. כך נולדו שמותיהם העבריים של האותיות זי"ן, דל"ת ונו"ן (נון הוא דג, כמו אמנון, שפמנון וכו'). הציורים שהפכו לסימנים התגלגלו לכתבים נוספים, ואפילו ליוונית וללטינית. גם בכתב העברי המודרני ניתן לזהות המשך התפתחותי ברור מן הכתב הכנעני הקדום, והשתמרות שמות האותיות מקלה מאוד על פענוח המקור.


בתקופת בית שני, אומץ האלפבית הארמי לשימוש השפה העברית במקום האלפבית העברי העתיק, כאשר בזה האחרון נעשה שימוש מועט כגון כתיבת השמות הקדושים והטבעת מטבעות. עם הזמן, נעלם גם שימוש זה של הכתב העתיק. האלפבית העברי של ימינו הוא אפוא פיתוח של האלפבית הארמי ולא של הכתב העברי העתיק.	
{}

 לֹ֥א תִשָּׂ֛א

\endR


\egroup
\bottomline
\medskip

To make all paragraphs  RL use the \cmd{\everypar}\footnote{See discussions at \url{http://tex.stackexchange.com/questions/141867/minimal-bidi-for-typesetting-rl-text} and \url{http://www.tug.org/pipermail/xetex/2004-August/000697.html}}. 

\begin{verbatim}
\newbox\mybox \everypar{\setbox\mybox\lastbox\beginR\box\mybox}
\everypar={% at the start of each paragraph, do....
    \setbox0=\lastbox % save the paragraph indent, if any
    \beginR % set R-L direction
    \box0 % then re-insert the indent
	}
\end{verbatim}

The Hebrew alphabet has 22 letters, of which five have different forms when used at the end of a word. Hebrew is written from right to left. Originally, the alphabet was an abjad consisting only of consonants. Like other \textit{abjads}, such as the Arabic alphabet, means were later devised to indicate vowels by separate vowel points, known in Hebrew as niqqud. In rabbinic Hebrew, the letters א ה ו י are also used as matres lectionis to represent vowels. When used to write Yiddish, the writing system is a true alphabet (except for borrowed Hebrew words). In modern usage of the alphabet, as in the case of Yiddish (except that ע replaces ה) and to some extent modern Israeli Hebrew, vowels may be indicated. Today, the trend is toward full spelling with these letters acting as true vowels.


\input{./languages/syriac}

\section{Arabic}
\label{s:arabic}

The Arabic script is a writing system used for writing several languages of Asia and Africa, such as Arabic, Sorani and Luri Dialects of Kurdish language, Persian, Pashto and Urdu.[1] Even until the 16th century, it was used to write some texts in Spanish.[2] After the Latin script, Chinese characters, and Devanagari, it is the fourth-most widely used writing system in the world.[3]
The Arabic script is written from right to left in a cursive style. In most cases the letters transcribe consonants, or consonants and a few vowels, so most Arabic alphabets are abjads.

The Arabic script has its roots in the Aramaic language and the Nabataen Arabs who wrote in the Aramaic script between the first century \BC{} and third centuries \AD{}. The Nabataens were a gathering of nomadic Arab tribes living
in a region stretching from the Sinai Peninsula to northern
Arabia and eastern Jordan. In the Hellenistic era following
Alexander the Great’s conquests, they formed a kingdom that
lasted from around 150 BC until conquest by the Romans in 105
AD; their capital was the peerless rock city of Petra. Their
Nabatæn form of Aramaic writing became the immediate
parent of Arabic writing.   

The script was first used to write texts in Arabic, most notably the Qurʼān, the holy book of Islam. With the spread of Islam, it came to be used to write languages of many language families, leading to the addition of new letters and other symbols, with some versions, such as Kurdish, Uyghur, and old Bosnian being abugidas or true alphabets. It is also the basis for a rich tradition of Arabic calligraphy. Like Hebrew, Arabic is an important religious script whose
significance, longevity and expansion are owed to its veneration as a vehicle of faith. Once it was chosen to convey the Koran in the seventh
century, its hegemony in the region, and beyond, was assured.
Today, the Arabic consonantal alphabet is read and written on
the Arabian Peninsula, throughout the Near East, in western,
Central and South-East Asia, in parts of Africa and in all areas of
Europe influenced by Islam (illus. 66). 

The Arabic script has
been adapted to more languages belonging to more families
than any other Semitic script, including Berber, Somali, Swahili
(illus. 67), Urdu, Turkish, Uighur, Kazakh, Farsi (Persian),
Kashmiri, Malay, even Spanish and Slavonic in Europe.37 When
borrowed, Arabic letters were never dropped, but new or
derived letters frequently were added to reproduce sounds not
included in the Arabic inventory. Arabic facilitates this process
by distinguishing between some letters only by varying the
number of dots written with each; this function can then easily
be extended by foreign tongues needing new letters compatible
with Arabic’s fundamental appearance.38 Arabic is one of the
world’s great scripts, and will doubtless survive for many more
centuries.


\begin{Arabic}


ّ هو إذ الغاية؛ شريف الفوائد، جم المذهب، عزيز فنّ التاريخ فنّ أنّ اعلم
والملوك سيرهم، في والأنبياء أخلاقهم، في الأمم من الماضين أحوال على يوقفنا
ّ أحوال في يرومه لمن ذلك في الإقتداء فائدة تتم حتّى وسياستهم؛ دولهم في
والدنيا. الدين


\end{Arabic}

Like all Semitic scripts, Arabic uses a consonantal alphabet
commonly indicating word roots, but with a richer inventory of
28 basic letters and additional augmentations, some created by
adding a dot under existing letters (illus. 68). (A ‘29th’ letter is
the ligature of la¯m and ’a¯lif.) Arabic also inherited the long vowel
use of some consonants and the special diacritics to signal
other vowels. However, vowels in Arabic are consistently indicated
only in the Koran and in poetry. All other texts use only
consonantal writing, with diacritics assisting occasionally in
ambiguous readings. The use of ’a¯lif for long /a:/ is an Arabic
innovation. Short /a/, /i/ and /u/ make use of derived forms of
simplified consonants: for /a/, a horizontal bar over the consonant;
for /i/, a similar bar under the consonant; and for /u/, a
small hook over the consonant. If a tiny circle is written above a
consonant, this means no vowel accompanies the consonant. All
but six Arabic letters occur in four different shapes, each determined
by the letter’s position in a word: independent (the neutral
or standard shape), initial, medial or final (illus. 69).39

The oldest Islamic inscription was found in 1999 and described by ‘{}Ali ibn Ibrahim Ghamman in Zuhayr in 
Saudi Arabia and is dated \AD{644-645}\footnote{ 
The inscription of Zuhayr, the oldest Islamicinscription (24 AH/AD 644–645), the rise of theArabic script and the nature of the early Islamic state.} Hoyland\cite{hoyland2010} gives a good review of the development of Arabic as
a written language during the late Roman period in Palestine and Arabia. 




As of Unicode 7.0, the Arabic script is contained in the following blocks:
Arabic (0600—06FF, 255 characters)
Arabic Supplement (0750—077F, 48 characters)
Arabic Extended-A (08A0—08FF, 39 characters)
Arabic Presentation Forms-A (FB50—FDFF, 608 characters)
Arabic Presentation Forms-B (FE70—FEFF, 140 characters)
Rumi Numeral Symbols (10E60—10E7F, 31 characters)
Arabic Mathematical Alphabetic Symbols (1EE00—1EEFF, 143 characters)[1][2]

The basic Arabic range encodes the standard letters and diacritics, but does not encode contextual forms (U+0621–U+0652 being directly based on ISO 8859-6); and also includes the most common diacritics and Arabic-Indic digits. The Arabic Supplement range encodes letter variants mostly used for writing African (non-Arabic) languages. The Arabic Extended-A range encodes additional Qur'anic annotations and letter variants used for various non-Arabic languages. The Arabic Presentation Forms-A range encodes contextual forms and ligatures of letter variants needed for Persian, Urdu, Sindhi and Central Asian languages. The Arabic Presentation Forms-B range encodes spacing forms of Arabic diacritics, and more contextual letter forms. The presentation forms are present only for compatibility with older standards, and are not currently needed for coding text.[3] 

The Arabic Mathematical Alphabetical Symbols block encodes characters used in Arabic mathematical expressions.


Position in word:	Isolated	Final	Medial	Initial
Glyph form:\scalebox{3}[3]{ب}{ـب}‎	ـبـ‎	 \scalebox{3}{بـ}


\printunicodeblock[2]{./languages/arabic.txt}{\arabicfont}





\input{./languages/thaana}


\endinput











%\chapter{South East Asian Scripts}

\section{Introduction}

This section documents the facilities offered to typeset Southeast Asian Scripts. These scripts are used in most of Southeast Asia, Indonesia and the Philippines.


\begin{table}[htb]
\centering
\begin{tabular}{lll}
Thai. & Tai Tham &Balinese.\\
\nameref{s:lao}  &Tai Viet  &Javanese.\\
Myanmar &Kayah Li &Rejang\\
Khmer. &Cham &Batak\\
Tai Le &Philippine Scripts &\nameref{s:undanese}\\
\nameref{s:newtailue} & Buginese\\
\end{tabular}
\end{table}
\input{./languages/lao}
\input{./languages/thai}
\input{./languages/balinese}
\clearpage

%\input{./languages/javanese}


\section{Khmer}
\newfontfamily\normaltext{Arial}
\normaltext

\newfontfamily\khmer[Scale=1.05]{NotoSansKhmer-Regular.ttf}
\def\khmertext#1{{\khmer#1}}

\begin{docKey}[phd]{khmer font}{=\meta{font name} (Khmer  UI)} {}
Loads the font
command \cmd{\khmer}. When the command is used it typesets text in
khmer unicode. There is no need to load the language, unless it is the main document language. For windows the default font is \texttt{DaunPenh} this font is in general too small to read; a better font to use is Khmer UI.
\end{docKey}




The Khmer script (Khmer: {\Large\khmertext{អក្សរខ្មែរ}}; IPA: [ʔaʔsɑː kʰmaːe]) [2] is an \textit{abugida} (alphasyllabary) script used to write the Khmer language (the official language of Cambodia). It is also used to write Pali among the Buddhist liturgy of Cambodia and Thailand.

It was adapted from the Pallava script, a variant of Grantha alphabet descended from the Brahmi script of India, which was used in southern India and South East Asia during the 5th and 6th Centuries AD.[3] The oldest dated inscription in Khmer was found at Angkor Borei District in Takéo Province south of Phnom Penh and dates from 611.[4] The modern Khmer script differs somewhat from precedent forms seen on the inscriptions of the ruins of Angkor.

Not all Khmer consonants can appear in syllable-final position. The most common syllable-final consonants include {\khmer កងញតនបមល}. The pronunciation of the consonant in final position may differ from it's normal pronunciation.


\begin{tabular}{l l p{9cm}}
\khmertext{ំ}	&nĭkkôhĕt (\khmertext{និគ្គហិត})	&niggahita; nasalizes the inherent vowels and some of the dependent vowels, see anusvara, sometimes used to represent [aɲ] in Sanskrit loanwords\\

\khmertext{ះ}	&reăhmŭkh (\khmertext{រះមុខ})	&"shining face"; adds final aspiration to dependent or inherent vowels, usually omitted, corresponds to the visarga diacritic, it maybe included as dependent vowel symbol\\

\khmertext{ៈ}	&yŭkôleăkpĭntŭ (\khmertext{យុគលពិន្ទុ})	&yugalabindu ("pair of dots"); adds final glottalness to dependent or inherent vowels, usually omitted\\

\khmertext{៉}	 &musĕkâtônd (\khmertext{មូសិកទន្ត})	&mūsikadanta ("mouse teeth"); used to convert some o-series consonants (\khmertext{ង ញ ម យ រ វ}) to the a-series\\

\khmertext{៊}	&treisâpt (\khmertext{ត្រីសព្ទ})	&trīsabda; used to convert some a-series consonants (\khmertext{ស ហ ប អ}) to the o-series\\
\end{tabular}




ុ	kbiĕh kraôm (ក្បៀសក្រោម)	also known as bŏkcheung (បុកជើង); used in place of the diacritics treisâpt and musĕkâtônd when they would be impeded by superscript vowels
់	bântăk (បន្តក់)	used to shorten some vowels; the diacritic is placed on the last consonant of the syllable
៌	rôbat (របាទ)
répheăk (រេផៈ)	rapāda, repha; behave similarly to the tôndâkhéat, corresponds to the Devanagari diacritic repha, however it lost its original function which was to represent a vocalic r
 ៍	tôndâkhéat (ទណ្ឌឃាដ)	daṇḍaghāta; used to render some letters as unpronounced
៎	kakâbat (កាកបាទ)	kākapāda ("crow's foot"); more a punctuation mark than a diacritic; used in writing to indicate the rising intonation of an exclamation or interjection; often placed on particles such as /na/, /nɑː/, /nɛː/, /vəːj/, and the feminine response /cah/
៏	âsda (អស្តា)	denotes stressed intonation in some single-consonant words[5]
័	sanhyoŭk sannha (សំយោគសញ្ញា)	represents a short inherent vowel in Sanskrit and Pali words; usually omitted
៑	vĭréam (វិរាម)	a mostly obsolete diacritic, corresponds to the virāma
្	cheung (ជើង)	a.w. coeng; a sign developed for Unicode to input subscript consonants, appearance of this sign varies among fonts
\input{./languages/sundanese}
%\input{./languages/hanuno}
\section{New Tai Lue Script}
\label{s:newtailue}
\let\tailue\pan

New Tai Lue script, also known as Simplified Tai Lue, is an alphabet used to write the Tai Lü language. Developed in China in the 1950s, New Tai Lue is based on the traditional Tai Le alphabet developed ca. 1200 AD. The government of China promoted the alphabet for use as a replacement for the older script; teaching the script was not mandatory, however, and as a result many are illiterate in New Thai Lue. 

In addition, communities in Burma, Laos, Thailand and Vietnam still use the Tai Le alphabet. There are probably less than one million native speakers of the language who can be found in China, Burma, Laos, Thailand and Vietnam.

We use NotoSansNewTaiLue-Regular.ttf.

\begin{scriptexample}[]{Tai Lue}
{\centering\tailue \LARGE

ᦒ	ᦓ	ᦔ	ᦕ	ᦖ	ᦗ	ᦘ	ᦙ	ᦚ	ᦛ	ᦜ	ᦝ	ᦞ	

}
\end{scriptexample}

The New Tai Lue script was added to the Unicode Standard in March, 2005 with the release of version 4.1.

The Unicode block for New Tai Lue is |U+1980|–|U+19DF|:

\begin{scriptexample}[]{New Tai Lue}
\unicodetable{tailue}{"1980,"1990,"19A0,"19B0,"19C0,"19D0}
\end{scriptexample}
\input{./languages/myanmar}

%\input{./languages/oriya}

\subsection{Mongolian Script}



The classical Mongolian script (in Mongolian script: {\mongolian  ᠮᠣᠩᠭᠣᠯ ᠪᠢᠴᠢᠭ᠌} Mongγol bičig; in Mongolian Cyrillic: Монгол бичиг Mongol bichig), also known as Uyghurjin Mongol bichig, was the first writing system created specifically for the Mongolian language, and was the most successful until the introduction of Cyrillic in 1946. Derived from Uighur, Mongolian is a true alphabet, with separate letters for consonants and vowels. The Mongolian script has been adapted to write languages such as Oirat and Manchu. Alphabets based on this classical vertical script are used in Inner Mongolia and other parts of China to this day to write Mongolian, Sibe and, experimentally, Evenki.
\medskip

\bgroup\par
\noindent
\colorbox{thecodebackground}{\color{black}^^A
\begin{minipage}{\textwidth}^^A
\parindent1pt
\vskip10pt
\leftskip10pt \rightskip\leftskip
\mongolian
\large
ᠬᠦᠮᠦᠨ ᠪᠦᠷ ᠲᠥᠷᠥᠵᠦ ᠮᠡᠨᠳᠡᠯᠡᠬᠦ ᠡᠷᠬᠡ ᠴᠢᠯᠥᠭᠡ ᠲᠡᠢ᠂ ᠠᠳᠠᠯᠢᠬᠠᠨ ᠨᠡᠷ᠎ᠡ ᠲᠥᠷᠥ ᠲᠡᠢ᠂ ᠢᠵᠢᠯ ᠡᠷᠬᠡ ᠲᠡᠢ ᠪᠠᠢᠠᠭ᠃ ᠣᠶᠤᠨ ᠤᠬᠠᠭᠠᠨ᠂ ᠨᠠᠨᠳᠢᠨ ᠴᠢᠨᠠᠷ ᠵᠠᠶᠠᠭᠠᠰᠠᠨ ᠬᠦᠮᠦᠨ ᠬᠡᠭᠴᠢ ᠥᠭᠡᠷ᠎ᠡ ᠬᠣᠭᠣᠷᠣᠨᠳᠣ᠎ᠨ ᠠᠬᠠᠨ ᠳᠡᠭᠦᠦ ᠢᠨ ᠦᠵᠢᠯ ᠰᠠᠨᠠᠭᠠ ᠥᠠᠷ ᠬᠠᠷᠢᠴᠠᠬᠥ ᠤᠴᠢᠷ ᠲᠠᠢ᠃
\par
\vspace*{10pt}
\end{minipage}
}
\medskip
^^A\input{./languages/tibetan}
^^A
^^A
^^A

^^A\input{./languages/tamil}


^^A
^^A\input{./languages/kannada}

^^A
^^A\input{./languages/osmanian}

^^A

\cxset{steward,
  offsety=0cm,
  image={ethiopianbride.jpg},
  texti={An introduction to the use of font related commands. The chapter also gives a historical background to font selection using \tex and \latex. },
  textii={In this chapter we discuss keys that are available through the \texttt{phd} package and give a background as to how fonts are used
in \latex.
 },
 pagestyle = empty,
}




\cxset{steward,
  offsety=0cm,
  image={fellah-woman.jpg},
  texti={An introduction to the use of font related commands. The chapter also gives a historical background to font selection using \tex and \latex. },
  textii={In this chapter we discuss keys that are available through the \texttt{phd} package and give a background as to how fonts are used
in \latex.
 },
 pagestyle = empty
}

%\index{Katakana}\index{Hiragana}
\index{Bopomofo}\index{Hangul}\index{Yi}
\index{East Asian Scripts>Katakana}
\index{East Asian Scripts>Hiragana}
\index{East Asian Scripts>Hangul}
\index{East Asian Scripts>Bopomofo}
\index{East Asian Scripts>Yi}
\index{scripts>cjk}
\pagestyle{headings}
\index{Yi fonts>Microsoft Yi Baiti}
\chapter{East Asian Scripts}
\epigraph{

For writing is the foundation of the classics and the arts, the beginning of
royal government. It is the means by which people of the past reach posterity,
by which people of the future know the past. 

{\cjk 蓋文字者,經藝之本,王政之始。前人所以垂後,後人所以識古。}
}{ Xu Shen  in the ``Postface'' of the \emph{Shuowen}}

\bigskip

\noindent This chapter presents the most common scripts currently in use in East Asia. This includes Chinese, Japanese and Korean. It also discusses several scripts for minority languages spoken in southern China. The scripts discussed are as follows:


\begin{center}
\begin{tabular}{lll}
\nameref{s:han} &Hiragana &Hangul\\
\nameref{s:bopomofo} &Katakana &\nameref{s:yi}\\
\end{tabular}
\end{center}
\bigskip

\parindent1em

Settings for |cjk| languages and scripts follow:

\begin{docKey}[phd]{cjk font}{\meta{font name}}{default none, initial code2000.ttf}
This key when set produces all necessary command to set the font for cjk typesetting.
\end{docKey}

\parindent1em
\section{Han CJK Unified Ideographs}
\label{s:han}
\index{CJK}
The Chinese, Japanese and Korean (CJK) scripts share a common background. In the process called Han unification the common (shared) characters were identified, and named "CJK Unified Ideographs". Unicode defines a total of 74,617 CJK Unified Ideographs.[1]\footnote{\protect\url{http://shahon.org/wp-content/uploads/2010/02/Galambos-2006-Orthography-of-early-Chinese-writing.pdf}}

The terms ideographs or ideograms may be misleading, since the Chinese script is not strictly a picture writing system.
Historically, Vietnam used Chinese ideographs too, so sometimes the abbreviation "CJKV" is used. This system was replaced by the Latin-based Vietnamese alphabet in the 1920s.


\unicodetable{cjk}{"4E00,"4E10,"4E20,"4E30,"4E40,"4}


\input{./languages/bopomofo}


\section{Yi}
\label{s:yi}

The Yi script (Yi: {\yi ꆈꌠꁱꂷ} nuosu bburma [nɔ̄sū bū̠mā]; Chinese: {\cjk 彝文}; pinyin: Yí wén) is an umbrella term for two scripts used to write the Yi language; Classical Yi, an ideogram script, the later Yi Syllabary. The script is also historically known in Chinese as Cuan Wen (Chinese: {\cjk 爨文}; pinyin: Cuàn wén) or Wei Shu (simplified Chinese: {\cjk韪书}; traditional Chinese: {\cjk 違書}; pinyin: Wéi shū) and various other names ({\cjk夷字、倮語、倮倮文、毕摩文}), among them "tadpole writing" ({\cjk蝌蚪文}).[1]

This is to be distinguished from romanized Yi ({\yi 彝文罗马拼音} Yiwen Luoma pinyin) which was a system (or systems) invented by missionaries and intermittently used afterwards by some government institutions.[2][3] There was also a Yi abugida or alphasyllabary devised by Sam Pollard, the Pollard script for the Miao language, which he adapted into "Nasu" as well.[4][5] Present day traditional Yi writing can be sub-divided into five main varieties (Huáng Jiànmíng 1993); Nuosu (the prestige form of the Yi language centred on the Liangshan area), Nasu (including the Wusa), Nisu (Southern Yi), Sani (撒尼) and Azhe (阿哲).[6][7]

The Unicode block for Modern Yi is Yi syllables (U+A000 to U+A48C), and comprises 1,164 syllables (syllables with a diacritic mark are encoded separately, and are not decomposable into syllable plus combining diacritical mark) and one syllable iteration mark (U+A015, incorrectly named YI SYLLABLE WU). In addition, a set of 55 radicals for use in dictionary classification are encoded at U+A490 to U+A4C6 (Yi Radicals).[11] Yi syllables and Yi radicals were added as new blocks to Unicode Standard Version 3.0.[12]

Classical Yi - which is an ideographic script like the Chinese characters - has not yet been encoded in Unicode, but a proposal to encode 88,613 Classical Yi characters was made in 2007.[13]

\bgroup
\yi \char"A000: Yi Syllable It\\

\yi \char"A001: Yi Syllable Ix\\

\yi \char"A002: Yi Syllable I\\
\egroup

\begin{scriptexample}[]{Yi}
\unicodetable{yi}{"A000,"A010,"A020,"A030,"A040,"A050,"A060,"A070,"A080,"A090,"A0A0,"A0B0,"A0C0}
\end{scriptexample}



%\cxset{chapter format =fashion}

\bgroup
\arial


\chapter{South Asian Scripts}

The scripts of South Asia share so many characteristics that a side by side comparison of a few often reveal structural similarities even in the 
modern letterforms.
\medskip


\begin{center}
\begin{tabular}{lll}
\nameref{sec:Devanagari}. 
&Gujarati. &Telugu\\
\nameref{sec:bengali}
&Oriya &Kannada\\
Gurmukhi &Tamil.  
&\nameref{malayalam}\\
\nameref{sec:sinhala} 
&Kaithi  
&Meetei Mayek.\\
Tibetan 
&\nameref{sec:saurashtra} 
&Ol Chiki.\\
Lepcha  &Sharada &Sora Sompeng\\
Phags-pa &Takri &Kharoshthi\\
Limbu &Chakma. & Brahmi\\
Syloti Nagri &Mro. &\\
\end{tabular}
\end{center}

The sections that follow describe the scripts briefly and the |phd| settings
to activate the relevant commands and load appropriate fonts. 

\begin{figure}[htbp]
\includegraphics[width=\textwidth]{./images/indic-language-tree.jpg}
\caption{A family tree of a few of the most important Indic scriptsc scripts, (\textit{after Fischer})\protect\cite{writing}}
\end{figure}

\input{./languages/sinhala}
\input{./languages/meetei-mayek}
%\input{./languages/tibetan}
\input{./languages/oriya}
\input{./languages/mro}


\input{./languages/devanagari}
\section{Bengali}
\label{sec:bengali}
\idxlanguage{Bengali}
\index{Bengali fonts>Shonar Bangla}
\index{Bengali fonts>Vrinda}
\index{Bengali fonts>code2000}
\newfontfamily\bengali[Script=Bengali,Scale=1.3]{Shonar Bangla}

There are two Windows fonts that can be used with Windows \textit{Shonar Bangla} and \textit{Vrinda}. For open source fonts one can use, \texttt{code2000}.

\docAuxCommand{bengali} Once the key is set the command \cmd{\bengali} is available for use in typesetting Bengali text.

\bigskip

\bgroup



\bengali
\centering

  অ  আ ই  ঈ  উ  ঊ  ঋ  এ  ঐ\par

\fontspec[Script=Bengali,Scale=3.2]{Vrinda}

\centering

  অ  আ ই  ঈ  উ  ঊ  ঋ  এ  ঐ\par


\fontspec[Script=Bengali,Scale=3.2]{code2000.ttf}

\centering

  অ  আ ই  ঈ  উ  ঊ  ঋ  এ  ঐ\par

\captionof{table}{The consonant{\protect\bengal{} ক (kô)} along with the diacritic form of the vowels {\protect\bengal{} অ, আ, ই, ঈ, উ, ঊ, ঋ, এ, ঐ, ও and ঔ} \textit{from Wikipedia}.}
\egroup


Bengali is a Unicode block containing characters for the Bangla, Assamese, Bishnupriya Manipuri, Daphla, Garo, Hallam, Khasi, Mizo, Munda, Naga, Rian, and Santali languages. In its original incarnation, the code points U+0981..U+09CD were a direct copy of the Bengali characters A1-ED from the 1988 ISCII standard, as well as several Assamese ISCII characters in the U+09F0 column. The Devanagari, Gurmukhi, Gujarati, Oriya, Tamil, Telugu, Kannada, and Malayalam blocks were similarly all based on ISCII encodings.

\begin{scriptexample}[]{Bengal}
\unicodetable{bengal}{"0980,"0990,"09A0,"09B0,"09C0,"09D0,"09E0,"09F0}
\end{scriptexample}


\printunicodeblock{./languages/bengali.txt}{\bengal}



\bgroup
\bengali\LARGE
\char"0995 + \color{blue} \char"09BC + \color{red}\char"09AF  = \char"0995\char"09CD \char"09AF
\egroup



See also \url{http://www.nongnu.org/freebangfont/downloads.html} for additional fonts.










\section{Saurashtra}
\label{sec:saurashtra}
\idxlanguage{Saurashtra}\idxlanguage{Sourashtra}

\index{Saurashtra fonts>code2000}
\newfontfamily\saurashtra{code2000.ttf}
\def\test{}
\cxset{saurashtra font/.code=\test}
\cxset{saurashtra font=code2000.ttf}

\begin{docKey}[phd]{saurashtra font}{ = \meta{fontname}} {default none, initial = code2000}
  This key sets the saurashtra font.
\end{docKey}

Saurashtra or Sourashtra or {\saurashtra ꢱꣃꢬꢵꢰ꣄ꢜ꣄ꢬꢵ} or Palkar or Patkar (Sanskrit: सौराष्ट्र, Tamil: சௌராட்டிரம்) is an Indo-Aryan language[3] spoken by the Saurashtrian community native to Gujarat, who migrated and settled in Southern India. Madurai in Tamil Nadu has the highest number of people belonging to this community and also remains as their cultural center.

The language is largely only in spoken form even though the language has its own script. The lack of schools teaching Saurashtra script and the language is often cited as a reason for the very few number of people who actually know to read and write in Saurashtra script. Latin, Devanagari or Tamil script is used as alternative for Saurashtra Script by many Saurashtrians.

Census of India places the language under Gujarati. Official figures show the number of speakers as 185,420 (2001 census).[4]


\begin{scriptexample}[]{Saurashtra}
\unicodetable{saurashtra}{"A880,"A890,"A8A0,"A8B0,"A8C0,"A8D0}
\end{scriptexample}


\begin{scriptexample}[]{Saurashtra}
\bgroup
\saurashtra

ꢮꢶꢯ꣄ꢮ ꢱꣃꢬꢵꢰ꣄ꢜ꣄ꢬꢪ꣄ ꢦꢡ꣄ꢬꢶꢒꢾ ꢱꢵꢡ꣄ꢡꢒꢸ ꢂꢮꢬꢾ
ꢮꣁꢭꢱ꣄ꢢꢵꢥꢪꢸꢒ꣄(ꣀꢵꢮꢾꢔꢹ ꢂꢮ꣄ꢬꢶꢫꣁ


\arial

Text: Vishwa Sourashtram \url{http://www.sourashtra.info/ghEr.htm}
\egroup
\end{scriptexample}


\printunicodeblock{./languages/saurashtra.txt}{\saurashtra}

\input{./languages/gujarati}
\input{./languages/tamil}
\input{./languages/malayalam}
\input{./languages/syloti}

\input{./languages/limbu}

\input{./languages/cham}


\section{Ol Chiki script}
\arial

The Ol Chiki script, also known as Ol Cemetʼ (Santali: ol 'writing', cemet' 'learning'), Ol Ciki, Ol, and sometimes as the Santali alphabet, was created in 1925 by Raghunath Murmu for the Santali language.

Previously, Santali had been written with the Latin alphabet. But because Santali is not an Indo-Aryan language (like most other languages in the south of India), Indic scripts did not have letters for all of Santali's phonemes, especially its stop consonants and vowels, which made writing the language accurately in an unmodified Indic script difficult. The detailed analysis was given by Dr. Byomkes Chakrabarti in his 'Comparative Study of Santali and Bengali'. Missionaries (first of all Paul Olaf Bodding, a Norwegian) brought the Latin script, which is better at representing Santali stops, phonemes and nasal sounds with the use of diacritical marks and accents. Unlike most Indic scripts, which are derived from Brahmi, Ol Chiki is not an abugida, with vowels given equal representation with consonants. Additionally, it was designed specifically for the language, but one letter could not be assigned to each phoneme because the sixth vowel in Ol Chiki is still problematic.
Ol Chiki has 30 letters, the forms of which are intended to evoke natural shapes. Linguist Norman Zide said "The shapes of the letters are not arbitrary, but reflect the names for the letters, which are words, usually the names of objects or actions representing conventionalized form in the pictorial shape of the characters."[1] It is written from left to right.

\newfontfamily\olchiki{code2000.ttf}

\begin{scriptexample}[]{olchiki}
\bgroup
\olchiki
\obeylines

U+1C5x 	᱐	᱑	᱒	᱓	᱔	᱕	᱖	᱗	᱘	᱙	ᱚ	ᱛ	ᱜ	ᱝ	ᱞ	ᱟ
U+1C6x	   ᱠ	ᱡ	ᱢ	ᱣ	ᱤ	ᱥ	ᱦ	ᱧ	ᱨ	ᱩ	ᱪ	ᱫ	ᱬ	ᱭ	ᱮ	ᱯ
U+1C7x  	ᱰ	ᱱ	ᱲ	ᱳ	ᱴ	ᱵ	ᱶ	ᱷ	ᱸ	ᱹ	ᱺ	ᱻ	ᱼ	ᱽ	᱾	᱿
\egroup

\unicodetable{olchiki}{"1C50,"1C60,"1C70}
\end{scriptexample}

\section{Lepcha}
\newfontfamily\lepcha{Mingzat-R.ttf}

The Lepcha script, or Róng script is an abugida used by the Lepcha people to write the Lepcha language. Unusually for an abugida, syllable-final consonants are written as diacritics.

The Mingzat font is still under development by SIL so I am not too sure if the rendering is correct\footnote{\url{http://scripts.sil.org/cms/scripts/page.php?site_id=nrsi&id=Mingzat}}.

\begin{scriptexample}[]{Lepcha}
\bgroup
\lepcha
\obeylines
 	    0	1	2	3	4	5	6	7	8	9	A	B	C	D	E	F
U+1C0x	 ᰀ	ᰁ	ᰂ	ᰃ	ᰄ	ᰅ	ᰆ	ᰇ	ᰈ	ᰉ	ᰊ	ᰋ	ᰌ	ᰍ	ᰎ	ᰏ
U+1C1x	 ᰐ	ᰑ	ᰒ	ᰓ	ᰔ	ᰕ	ᰖ	ᰗ	ᰘ	ᰙ	ᰚ	ᰛ	ᰜ	ᰝ	ᰞ	ᰟ
U+1C2x	 ᰠ	ᰡ	ᰢ	ᰣ	ᰤ	ᰥ	ᰦ	ᰧ	ᰨ	ᰩ	ᰪ	ᰫ	ᰬ	ᰭ	ᰮ	ᰯ
U+1C3x	 ᰰ	ᰱ	ᰲ	ᰳ	ᰴ	ᰵ	ᰶ	᰷	x	x	x	᰻	᰼	᰽	᰾	᰿
U+1C4x	 ᱀	᱁	᱂	᱃	᱄	᱅	᱆	᱇	᱈	᱉	x	x	x	ᱍ	ᱎ	ᱏ

\egroup
\end{scriptexample}

\section{Sharada}

The Śāradā, or Sharada, script (शारदा) is an abugida writing system of the Brahmic family of scripts, developed around the 8th century. It was used for writing Sanskrit and Kashmiri. The Gurmukhī script was developed from Śāradā. Originally more widespread, its use became later restricted to Kashmir, and it is now rarely used except by the Kashmiri Pandit community for ceremonial purposes. Śāradā is another name for Saraswati, the goddess of learning.
Śāradā script was added to the Unicode Standard in January, 2012 with the release of version 6.1.

The Unicode block for Śāradā script, called Sharada, is U+11180–U+111DF: Unable to locate font in unicode.


\section{Sora Sompeng}

Sorang Sompeng script is used to write in Sora, a Munda language with 300,000 speakers in India. The script was created by Mangei Gomango in 1936 and is used in religious contexts.[1] He was familiar with Oriya, Telugu and English, so the parent systems of the script are Brahmi and Latin.[2]
The Sora language is also written in the Latin alphabet and the Telugu script.

Sorang Sompeng script was added to the Unicode Standard in January, 2012 with the release of version 6.1. Nirmala UI.ttf (Windows 8.1)
\newfontfamily\NirmalaU{Segoe UI Symbol}


\unicodetable{NirmalUI}{"110D0,"110E0,"110F0}
 	
This did not work with Windows 7, and the experiment failed. 



\section{Phags-pa}
\label{s:phagspa}
\newfontfamily\phagspa{code2000.ttf}
\arial 
The 'Phags-pa script,[1], (Mongolian: дөрвөлжин үсэг "Square script") was an alphabet designed by the Tibetan monk and vice-king Drogön Chögyal Phagpa for the Mongol Yuan emperor Kublai Khan as a unified script for the literary languages of the Yuan.

Widespread use was limited to about a hundred years during the Yuan Dynasty, and it fell out of use with the advent of the Ming dynasty. The documentation of its use provides clues about the changes in the varieties of Chinese, the Tibetic languages, Mongolian and other neighboring languages during the Yuan era.
\medskip


\includegraphics[width=1\linewidth]{./images/phags-pa.jpg}

credit \protect\url{http://turfan.bbaw.de/dta/monght/images/monght009_seite2.jpg}



\begin{scriptexample}[]{Phags-pa}
\bgroup
\unicodetable{phagspa}{"A840,"A850,"A860,"A870}

\arial
\hfill Typeset with \texttt{code2000.ttf} and \cmd{\phagspa}

\egroup
\end{scriptexample}
\medskip

Phags-pa is a historical script related to Tibetan that was created as the national script of
the Mongol empire. Even though Phags-pa was used mostly in Eastern and Central Asia for
writing text in the Mongolian and Chinese languages, it is discussed in this chapter because
of its close historical connection to the Tibetan script. The script has very limited modern use. It bears similarity to Tibetan and has no case distinctions. It is written vertically in columns running for left to right, like Mongolian. Units are often composed of several syllables and sometimes are separated by whitespace.


\printunicodeblock{./languages/phags-pa.txt}{\phagspa}

\cxset{script/.code={}}
\cxset{script=phags-pa}

\begin{docKey}[phd]{script}{ = \meta{phags-pa}} {}
The key |script| will activate the commands available for typesetting the phags-pa script.
\end{docKey}





\input{./languages/chakma}

\input{./languages/brahmi}

\arial
\begin{longtable}{lr}
Common	&6412\\
Latin	&1272\\
\nameref{s:greek}	&511\\
Cyrillic	&417\\
Armenian	&91\\
\nameref{s:hebrew}	 &133\\
Arabic	 &1234\\
\nameref{s:syriac}	 &77\\
Thaana	 &50\\
Devanagari	&151\\
Bengali	&92\\
Gurmukhi	&79\\
Gujarati	&84\\
Oriya	&90\\
Tamil	&72\\
Telugu	&93\\
Kannada	&86\\
Malayalam	&98\\
Sinhala	&80\\
Thai	&86\\
Lao	 &67\\
Tibetan	&207\\
Myanmar	&188\\
Georgian	&127\\
Hangul	   &11739\\
Ethiopic	&495\\
Cherokee	&85\\
Canadian Aboriginal	 &710\\
Ogham	&29\\
\nameref{s:runic}	&78\\
\nameref{s:khmer}	&146\\
Mongolian	&153\\
Hiragana	&91\\
Katakana	&300\\
Bopomofo	&70\\
Han	 &75963\\
\nameref{s:yi}	&1220\\
Old Italic	&35\\
\nameref{s:gothic}	 &27\\
Deseret	&80\\
Inherited	&524\\
Tagalog	&20\\
Hanunoo	&21\\
Buhid	&20\\
Tagbanwa	1&8\\
Limbu	&66\\
Tai Le	 &35\\
Linear B	&211\\
Ugaritic	&31\\
Shavian	&48\\
Osmanya	&40\\
Cypriot	&55\\
Braille	&256\\
Buginese	&30\\
Coptic	 &137\\
New Tai Lue	&83\\
Glagolitic	&94\\
Tifinagh	&59\\
Syloti Nagri	&44\\
Old Persian	&50\\
Kharoshthi	&65\\
Balinese	&121\\
Cuneiform	&982\\
Phoenician	&29\\
Phags Pa	&56\\
Nko	 &59\\
Sundanese	&72\\
Lepcha	 &74\\
Ol Chiki	&48\\
\nameref{s:vai}	&300\\
Saurashtra	&81\\
Kayah Li	&48\\
Rejang	 &37\\
Lycian	 &29\\
Carian	 &49\\
Lydian	 &27\\
Cham	 &83\\
Tai Tham	&127\\
Tai Viet	&72\\
Avestan	&61\\
Egyptian Hieroglyphs	&1071\\
Samaritan	 &61\\
Lisu	&48\\
Bamum	&657\\
Javanese	&91\\
Meetei Mayek	&79\\
Imperial Aramaic	&31\\
Old South Arabian	&32\\
Inscriptional Parthian	 &30\\
Inscriptional Pahlavi	&27\\
Old Turkic	&73\\
Kaithi	 &66\\
Batak	 &56\\
Brahmi	 &108\\
Mandaic	&29\\
Chakma	&67\\
Meroitic Cursive	&26\\
Meroitic Hieroglyphs	&32\\
Miao	&133\\
Sharada	&83\\
Sora Sompeng	&35\\
Takri	&66\\
	
	&110181\\
\end{longtable}

\egroup
%\input{./languages/modern-scripts}
%
 
\parindent1em
\cxset{section font-family=tiresias,
          section font-family=sffamily}

\chapter{Internationalization and Globalization}

\section{Introduction}

In this Chapter we discuss the requirements for localization of software and how this can be applied to \latex. In a way this chapter overlaps the one on languages. However, here we focus mostly on LuaTeX solutions. We also extend the discussion to calendric and solar calculations.

The development of routines for software internationalization and globalization has been an ongoing effort for many years. Currently the accepted method for building such software is the use of i18n. This is an abbreviation of the first letter and last letter of the word internationalization and the 18 is the number of characters in the word.

Internationalization based on i18n is not an easy task for \LaTeX. To an extend some of the issues have been removed with the use of Babel and Polyglossia that provide translation strings for many of the worlds scripts. The de facto standard for internationalization is the Unicode Consortium’s \href{http://cldr.unicode.org/}{CLDR} project.

\section{Locales}
\index{locale}

In computing, a \emph{locale} is a set of parameters that defines the user's language, country and any special variant preferences that the user wants to see in their user interface. Usually a locale identifier consists of at least a \textit{languag}e identifier and a \textit{region} identifier.

On POSIX platforms such as Unix, Linux and others, locale identifiers are defined similar to the BCP 47 definition of language tags, but the locale variant modifier is defined differently, and the character set is included as a part of the identifier. It is defined in this format: |[language[_territory][.codeset][@modifier]]|. (For example, Australian English using the UTF-8 encoding is en\_AU.UTF-8.)

For \latex these ``locales'' can be thought of as the settings of language keys through Babel and Polyglossia. These settings have served the community well for many years, but a litany of duct taping through other packages are a testimony to their limitations. Packages for dates, time and number formatting have been developed to assist. Here is my attempt to put the solution on a better footing and to start providing mechanisms via LuaTeX for a 'plugin'
architecture to find improve solutions. 

\section{Common Locale Data Repository}

The Common Locale Data Repository Project, is a project of the Unicode Consortium to provide locale data in the XML format for use in computer applications. CLDR contains locale specific information that an operating system will typically provide to applications. CLDR is written in LDML (Locale Data Markup Language). The information is currently used in International Components for Unicode, Apple's Mac OS X, OpenOffice.org, and IBM's AIX, among other applications and operating systems

\begin{enumerate}
\item Translations for language names.
\item Translations for territory and country names.
\item Translations for currency names, including singular/plural modifications.
\item Translations for weekday, month, era, period of day, in full and abbreviated forms.
\item Translations for timezones and example cities (or similar) for timezones.
\item Translations for calendar fields. This is useful especially in conjuction with PGF presentational forms.
\item Patterns for formatting/parsing dates or times of day.
\item Examplar sets of characters used for writing the language.
\item Patterns for formatting/parsing numbers.
\item Rules for language adapted collation. \label{collation}
\item Rules for formatting numbers in traditional numeral systems (like Roman numerals, Armenian numerals, ...).
\item Rules for spelling out numbers as words.
\item Rules for transliteration between scripts. A lot of it is based on BGN/PCGN romanization.
\item Rules for \emph{delimiters} such as quotations and question marks.
\end{enumerate}

Currently the consortium’s distribution make the data available in both json and xml formats.  These files hold data for a specific \emph{locale}. Sadly missing are any document sectioning information that would have enabled the incorporation of the above into LaTeX and overcoming some of the Babel and Polyglossia limitations.

We do not need many of the files provided by the CLDR unicode consortium and others we are missing. Take for example the |delimiters| file. 

\begin{verbatim}
  "main" = {
    "ff": {
      "identity": {
        "version": {
          "_cldrVersion": "26",
          "_number": "$Revision: 10739 $"
        },
        "generation": {
          "_date": "$Date: 2014-08-07 12:54:13 -0500 (Thu, 07 Aug 2014) $"
        },
        "language": "ff"
      },
      "delimiters": {
        "quotationStart": "„",
        "quotationEnd": "”",
        "alternateQuotationStart": "‚",
        "alternateQuotationEnd": "’"
      }
    }
  }
}
\end{verbatim}

Of course the |Json| format as it is, is not readable by Lua a format such as:

\begin{verbatim}
delimiters = {
        quotationStart = "«",
        quotationEnd = "»",
        alternateQuotationStart = "\"",
        alternateQuotationEnd = "\""
      }
\end{verbatim}

\begin{texexample}{i18n}{i18-1}
\begin{luacode*}
-- mock the delimiters from the json
-- file
greekname = 'el'
delimiters = {
        quotationStart = "«", 
        quotationEnd = "»",
        alternateQuotationStart = "\"",
        alternateQuotationEnd = "\""
      }
tex.sprint(delimiters.quotationStart .. 'test' .. delimiters.quotationEnd)
tex.print ([[\gdef\]] .. greekname .. [[quote#1{\directlua{tex.sprint(delimiters.quotationStart .. '#1' .. delimiters.quotationEnd)}}]])
\end{luacode*}

%\def\elquote#1{%
%  \directlua {tex.sprint(delimiters.quotationStart .. '#1' .. delimiters.quotationEnd)}
%}
\end{texexample}

This is of course a much more simplified way of what one needs to program for a full system. The advantage
of producing the \tex definition also through LuaTeX is that we can keep all the code in one place and econd, we can avoid |\csname| costructs.
\begin{texexample}{elquote}{}
\elquote{This is some longer text in Greek quotes.}
\end{texexample}

I have opted to incorporate these files in the |json| format and provide routines for interfacing via the \pkgname{phd} package.  The reason for opting for a json format, is my other attempts to interface the package with |couchdb|.  My preference for a Nosql type of database, is that  they are better suited in handling data that is commonly  found in documents and also many of the routines will be interchangeable for web applications. I am also hoping that the collation information (see \ref{collation}), will eventually lead to better indices, a subject left untouched in the current distribution.\index{json}

\section{The package phd approach}

The package |phd| packge takes an approach to use only json resource files for the provision of language dependent information, rather than TeX commands alone, as is done by Babel and Polyglossia. 

\section{Language and Region Tags}
\index{tags>regions}\index{tags>language}

Languages are represented by tags such as "en"  for English or "el" for Greek. Other languages have no significant variation and are represented by a language subtag such as "en-US".  The names are mostly intuitive, but in many case bear no relationship to their English names, for example Armenian is coded as \textbf{hy}. There is a useful utility at the SIL website for viewing these codes.\footnote{\protect\url{http://www-01.sil.org/iso639-3/codes.asp?order=reference_name&letter=\%25}.} Note that the CLDR database does not cover all the languages listed in the ISO-639.\index{ISO-639}

The language tags are based on the BGN which is mapped to languages based on ISO-639-1. 

We will describe the tables using the English language, which is normally the default and Greek as a second language, as the script is distinctive enough to demonstrate their use. We will also explain Lua routines available via the \pkgname{phd} that are provided as alternatives to Babel and Polyglossia.


{layout.lua}

{layout.orientation.characterOrder} = |left_to_right| or |right_to_left|

layout.orientation.lineOrder = |top_to_bottom|

Example \ref{i18-1} loads the Greek internationalization file |layout| and prints the two fields. Before we send it to
the TeX typesetter we sanitize the string underscores using |gsub|. For illustration purposes we have used |gsub| both as an object method and as a function.

\begin{texexample}{i18n}{i18-1}
\begin{luacode}
local c = require("i18n.el.layout")
local s1 = string.gsub(c.el.layout.orientation.characterOrder, '_', '\\textunderscore ')
local s2 = c.el.layout.orientation.lineOrder:gsub('_', '\\textunderscore ')
tex.print('typeof :', type(c))
tex.print(s1, '\\par', s2)
\end{luacode}
\end{texexample}

Of course for Greek the above information is hardly necessary, but at the level of Lua programming, if we are automating the switching of text direction Greek text might signal a change in direction. Let us have another try using the same code for arabic text. All we have to change is the \textbf{el} to \textbf{ar}.

\begin{texexample}{i18n}{i18-2}
\begin{luacode}
local c = require("i18n.ar.layout")
local s1 = string.gsub(c.ar.layout.orientation.characterOrder, '_', '\\textunderscore ')
local s2 = c.ar.layout.orientation.lineOrder:gsub('_', '\\textunderscore ')
tex.print('typeof :', type(c), '\\par')
tex.print(s1, '\\par', s2)
\end{luacode}
\end{texexample}

Based on the direction of the language we may then develop code to set the arabic text. This is discussed further
in the section for languages using the Arabic script.\footnote{The bidi package is useful for texts that are predominantly RTL.}

\section{Calendars and Dates}

\subsection{The year}
\indexmany[year]{solar,tropical}

The tropical or solar year, properly, and by way of eminence so-called, is the space of time in which the sun moves 
through the twelve signs of the zodiac. This, by observations of the best modern astronomers, contains \printtime[5]{365}{5}{48}{46.14912}. The quantity assumed by the authors of the Gregorian calendar was \printtime[0]{365}{ 5}{49}{0} which  corresponds exactly with the observations of Bianchini, and  de La Hire, in the next century. In the civil, or popular account, the year. 

The excess of the solar year over 365 days has been given by different astronomers as follows:---
 
 \def\daytime#1#2#3#4{%
 #1\textsuperscript{d}%
 #2\textsuperscript{h}%
 #3\textsuperscript{m}%
 #4\textsuperscript{s}%
  }
 
 \indexmany[calendric calculations]{Meton,Euctemon,%
   Hipparchos,Sosigenes,Albategnius,Copernicus,Tycho %  
   Brahe,Kepler,Halley,Lalande,Delambre,Laplace,Hind}

 
 \begin{longtable}{l r l}
Meton and Euctemon  &5th Century BC  &\printtimeinterval{6}{18}{57}{0}\\
Hipparchos          &2 Century BC        &\printtimeinterval{5}{55}{12}{0}\\
Sosigenes           &1 Century BC        &\printtimeinterval{6}{0}{0}{0}\\
Albategnius         &9th Century AD    &\printtimeinterval{5}{46}{24}{0}\\ 
Alphonsine Tables   &13th Century AD  &\printtimeinterval{5}{49}{16}{0}\\
Copernicus          &16th Century AD  &\printtimeinterval{5}{46}{6}{0}\\
Tycho Brahe     	  &16th Century AD  &\printtimeinterval{5}{48}{45.5}{0}\\
Kepler 				    &17th Century AD  &\printtimeinterval{5}{48}{57.65}{0}\\
Halley 				    &17th Century AD  &\printtimeinterval{5}{48}{54.691}\\
Lalande 			      &18th Century AD  &\PrintTimeInterval{5}{48}{35.5}\\
Delambre			      &18th Century AD  &\printtimeinterval {5}{48}{51.6}{0}\\ 
Laplace				    &18th Century AD  &\printtimeinterval {5}{48}{49.7}{0} \\
Hind, 1850			    &19th Century AD  &\printtimeinterval {5}{41}{46.2}{0} \\
\end{longtable} 


The oldest references to the Greek word τροπή [turn, soltice] are from Hesiod and Homer. Evidence
exists that from the earliest times the Chinese, the Hindus and the Greeks, and others did measure the
length of the tropical year, also called the seasonal year. According to Delabre this year is so called
`because the first astronomers did deduce it from the return of the Sun to the same tropic’.\footnote{Delabre, J., \textit{Histoire de l’Astronomie}, Paris, 1817.}

According to Ptolemy, Hipparchus wrote: `I composed a book about the length of the year, in which
I show that this is the time required for the Sun to travel from a tropic to the same tropic again, or from
an equinox to that same equinox, and that it is equal to 365.25 days minus 1/300 of a day-and-night, and not to a fourth of a day as the mathematicians believed’.\footnote{\textit{Almagest}, Book III. The citation has been reported by Ptolemy.}


The accepted current tropical year value on January 1, 2000 was 365.2421897 or \PrintTimeInterval{365}{5}{48}{45.19} . This changes slowly; an expression suitable for calculating the length in days for the distant past is

\begin{equation}
365.2421896698 - 6.15359×10-6T- 7.29\times10T2 + 2.64 \times10 -10 T3
\end{equation}


where $T$ is in Julian centuries of 36,525 days measured from noon January 1, 2000 TT (in negative numbers for dates in the past). (McCarthy \& Seidelmann, 2009, p. 18.; Laskar, 1986)

Modern astronomers define the tropical year as time for the Sun's mean longitude to increase by 360°. The process for finding an expression for the length of the tropical year is to first find an expression for the Sun's mean longitude (with respect to ♈), such as Newcomb's expression given above, or Laskar's expression (1986, p. 64). When viewed over a 1 year period, the mean longitude is very nearly a linear function of Terrestrial Time. To find the length of the tropical year, the mean longitude is differentiated, to give the angular speed of the Sun as a function of Terrestrial Time, and this angular speed is used to compute how long it would take for the Sun to move 360°. (Meeus \& Savoie, 1992, p. 42).

The length of the tropical year accordinng to Leverrier

\begin{equation}
365.24219647 - 0.00000624 T \text{days}
\end{equation}
while Newcomb's well known expression, derived from his solar theory, is

\begin{equation}
\PrintTimeInterval {365}{5}{46}{0} - \PrintTimeInterval {0}{0}{0}{.530}T
\end{equation}

In these two expressions, $T$ is the time in Julian centuries of 36525 days measured from 1900 January 0.5 Ephemeris time. 

\subsection{Month}
\index{month}\index{month, astronomical}

The next convenient measure for the division of time, which is marked by the revolution of the
celestial objects is the month. The astronomical month is the period of time in which the moon 
performs a complete revolution round the heavens, and is either \textit{periodical} or \textit{synodical}. The periodical
month is the time in which the moon moves from one point the same point again, and is equal to \printtime{27}{7}{ 43}{47}; and the synodical month, or lunation, as it is sometimes called, is that portion of time which
elapses between two successive new moons, or between two succesive conjuctions of the moon with the sun, and is equal to \printtime{29}{12}{44}{3.19}.  The solar month is that portion of time in which the sun moves through an
entire sign of the zodiac, the mean quantity of which is \printtime{30}{10}{29}{3.84576}, being the twelfth of the
solar year.


\subsection{Week}

The origins of the seven day week is thought to have originated with Sumeria, who gave the name
of one of the seven planets to each hour of the day, and deisgnated each day by the name of that planet, which corresponded with the first hour of the day. 

The Latins adopted these designations in their names of 
the days of the week. They are to be found in old law books 
and MSS. For a very interesting discussions as to how the seven
day week, survived and passed to us see The Economist.\footnote{\protect\url{http://www.economist.com/node/895542?fsrc=scn/fb/wl/ar/thepowerofseven}}

Occasionally, the signs only of the planets were used, for 
the sake of brevity, particularly in diaries and journals. This 
is notably the case in the original MS. field-book of Mason 
and Dixon's survey of the boundary line between Pennsylvania and Maryland, 1763 to 1768, 
in possession of the Historical 
Society of Pennsylvania. In this book the name of 
each day of the week is represented by the sign, in addition 
to the usual dates, for a period of over four years. See, also, 
" Minutes of the Provincial Council of Pennsylvania" (Colonial Records), vol. ii. pages 90 to 96, etc. etc. In the latter 
part of vol i. (same Records) the Latin names of the days 
were used. 

\begin{center}
\begin{tabular}{l c l l}
\toprule
Latin              &Signs                   &English      &Anglo-Saxon\\
\midrule
Dies Saturni   &{\pan\char"2644}  &Saturday     &Saetern-daeg\\
Dies Solis       &{\pan\char"2609}  &Sunday       &Sunnan-daeg\\
Dies Lunae     &{\pan\char"263D} &Monday      &Monan-daeg\\
Dies Martis     &{\pan\char"2642} &Tuesday       &Tiwes-daeg\\
Dies Mercurii  &{\pan\char"263F}  &Wednesday &Wodnes-daeg\\
Dies Jovis       &{\pan\char"2643}  &Thursday     &Thors-daeg\\
Dies Veneris   &{\pan\char"2640}  &Fiday           &Frigas-daeg\\
\bottomrule
\end{tabular}
\end{center}

The |IAU| discourages the use of the planetary symbols in articles and we only show them in the above table for historical reasons.  

The Aztecs had a ritual cycle of 260 days, known as Tonalpohuali, which was divided
into 20 weeks of 13 days known as Trecena. They also divided the solar year of 365 days, into 18
periods of 20 days and five nameless days known as Nemontemi. Although some consider this 20-day
grouping a month, it has no relation to lunation. It was divided into four ``weeks'' of five days.

The Soviet Union between 1929 and 1931 changed from the seven-day week to a five-day week. There
were 72 weeks and an additional five national holidays inserted within three of them totaling a year of
365 days.

\begin{figure}[ht]
\includegraphics[width=\textwidth]{./images/soviet-calendar.jpg}
\caption{The Soviet Calendar for 1930}
\end{figure}

The five day week was a social disaster and this was then changed to a six day week, which later was
abandoned by a decree issued on 27 June 1940.

\begin{figure}[ht]
\centering
\includegraphics[width=0.5\textwidth]{./images/soviet-calendar-1939.jpg}
\caption{The six day Soviet Calendar for 1939}
\end{figure}

From the summer of 1931 until 26 June 1940, each Gregorian month was usually divided into five six-day weeks, more and less (as shown by the 1933 and 1939 calendars displayed here).[2] The sixth day of each week was a uniform day off for all workers, that is days 6, 12, 18, 24 and 30 of each month. 

The last day of 31-day months was always an extra work day in factories, which, when combined with the first five days of the following month, made six successive work days. But some commercial and government offices treated the 31st day as an extra day off. To make up for the short fifth week of February, 1 March was a uniform day off followed by four successive work days in the first week of March (2–5). The partial last week of February had four work days in common years (25–28) and five work days in leap years (25–29). But some enterprises treated 1 March as a regular work day, producing nine or ten successive work days between 25 February and 5 March, inclusive. The dates of the five national holidays did not change, but they now converted five regular work days into holidays within three six-day weeks rather than splitting those weeks into two parts (none of these holidays was on a ``sixth day")


The Unicode |CLDR| specification dictates that for each language a set of files are provided for calendar related information. This for example enables the printing of calendars in a specific language such as Greek, but using an islamic calendar.  The internationalization tables do not provide any conversion or calculation routines. They just represent how the traslated string would look. 

\begin{texexample}{i18n}{i18-3}
\begin{luacode}
local c = require("i18n.el.caislamic")
local s1 = c.el.dates.calendars.islamic.months.format.abbreviated["1"]
local s2 = c.el.dates.calendars.islamic.months.format.abbreviated["2"]
tex.print('typeof :', type(c), '\\par')
tex.print(s1, '\\par', s2)
\end{luacode}
\end{texexample}

\subsection{Available calendar translations}

The available calendar translations for each language, they are provided as submodules to the |i18n| module. They are listed in Table    . The prefix is the 2-letter name of the language so for English it will be ca-buddhist.

\begin{table}[ht]

\begin{multicols}{2}
ca-buddhist\\
ca-chinese\\
ca-coptic\\
ca-dangi\\
ca-ethiopic\\
ca-ethiopic-amete-alem\\
ca-generic\\
ca-gregorian\\
ca-hebrew\\
ca-indian\\
ca-islamic\\
ca-islamic-civil\\
ca-islamic-rgsa\\
ca-islamic-tba\\
ca-islamic-umalqura\\
ca-japanese\\
ca-persian\\
ca-roc\\
\end{multicols}
\caption{Available calendars for each language.}
\end{table}

\subsection{Historical Roman Calendars}


\subsection{Julian Calendar}

Julius Caesar in 46 BC (708 AUC\footnote{A.U.C. is from the Latin ab urbe condita, which translates from the founding of the City (Rome).}) reformed the then current calendar.   It was the predominant calendar in the Roman world, most of Europe, and in European settlements in the Americas and elsewhere, until it was refined and superseded by the Gregorian calendar. The difference in the average length of the year between Julian (365.25 days) and Gregorian (365.2425 days) is 0.002%.

The Julian calendar has a regular year of 365 days divided into 12 months, as listed in Table of months. A leap day is added to February every four years. The Julian year is, therefore, on average 365.25 days long. It was intended to approximate the tropical (solar) year. Although Greek astronomers had known, at least since Hipparchus, a century before the Julian reform, that the tropical year was a few minutes shorter than 365.25 days, the calendar did not compensate for this difference. As a result, the calendar year gained about three days every four centuries compared to observed equinox times and the seasons. This discrepancy was corrected by the Gregorian reform of 1582. The Gregorian calendar has the same months and month lengths as the Julian calendar, but inserts leap days according to a different rule. Consequently, the Julian calendar is currently 13 days behind the Gregorian calendar; for instance, 1 January in the Julian calendar is 14 January in the Gregorian. Old Style (O.S.) and New Style (N.S.) are sometimes used with dates to indicate either whether the start of the Julian year has been adjusted to start on 1 January (N.S.) even though documents written at the time use a different start of year (O.S.), or whether a date conforms to the Julian calendar (O.S.) rather than the Gregorian (N.S.). Dual dating uses two consecutive years because of differences in the starting date of the year, or includes both the Julian and Gregorian dates.

The Julian calendar has been replaced as the civil calendar by the Gregorian calendar in all countries which formerly used it, although it continued to be the civil calendar of some countries into the 20th century. Among the last countries to convert to the Gregorian Calendar were Greece (in 1924), Turkey (in 1926) and Egypt (in 1928).[2] As of 1930, all countries that were using the Julian calendar had discontinued it. Most Christian denominations in the West and areas evangelized by Western churches have also replaced the Julian calendar with the Gregorian as the basis for their liturgical calendars. However, most branches of the Eastern Orthodox Church still use the Julian calendar for calculating the dates of moveable feasts, including Easter (Pascha). Some Orthodox churches have adopted the Revised Julian calendar for the observance of fixed feasts, while other Orthodox churches retain the Julian calendar for all purposes.[3] The Julian calendar is still used by the Berber people of North Africa, and on Mount Athos. In the form of the Alexandrian calendar, it is the basis for the Ethiopian calendar, which is the civil calendar of Ethiopia.

The ordinary years in the previous Roman calendar consisted of 12 months, for a total of 355 days. In addition a 27-day intercalary month, the \textit{Mensis Intercalaris}, was sometimes inserted between February and March. This extra month was formed by inserting 22 days after the first 23 or 24 days of February, which counted down toward the start of March, 
became the last five days of Intercalaris. The result was to add2 or 23 days to the year forming an intercalary year of 377 or 
378 days.

Caesar returned to Rome in 46 BC and, according to Plutarch, called in the best philosophers and mathematicians of his time to solve the problem of the calendar.[16] Pliny says that Caesar was aided in his reform by the astronomer Sosigenes of Alexandria[17] who is generally considered the principal designer of the reform. Sosigenes may also have been the author of the astronomical almanac published by Caesar to facilitate the reform.[18] Eventually, it was decided to establish a calendar that would be a combination between the old Roman months, the fixed length of the Egyptian calendar, and the 365¼ days of the Greek astronomy. According to Macrobius, Caesar was assisted in this by a certain Marcus Flavius.[19]

Since the Julian and Gregorian calendars were long used simultaneously, although in different places, calendar dates in the transition period are often ambiguous, unless it is specified which calendar was being used. In some circumstances, double dates might be used, one in each calendar. The notation ``Old Style" (O.S.) is sometimes used to indicate a date in the Julian calendar, as opposed to ``New Style" (N.S.), which either represents the Julian date with the start of the year as 1 January or a full mapping onto the Gregorian calendar. This notation is used to clarify dates from countries which continued to use the Julian calendar after the Gregorian reform, such as Great Britain, which did not switch to the reformed calendar until 1752, or Russia, which did not switch until 1918.

Throughout the long transition period, the Julian calendar has continued to diverge from the Gregorian. This has happened in whole-day steps, as leap days which were dropped in certain centennial years in the Gregorian calendar continued to be present in the Julian calendar. Thus, in the year 1700 the difference increased to 11 days after February 28 (Gregorian); in 1800, 12; and in 1900, 13. Since 2000 was a leap year according to both the Julian and Gregorian calendars, the difference of 13 days did not change in that year: 29 February 2000 (Gregorian) fell on 16 February 2000 (Julian). This difference will persist through the last day of February, 2100 (Gregorian), since 2100 is not a Gregorian leap year, but is a Julian leap year. Monday 1 March 2100 (Gregorian) falls on Monday 16 February 2100 (Julian).[81]


\subsection{Gregorian calendar}

The routines for calculating and displaying Gregorian Calendar dates are provided by 
the \pkgname{phd} as lua interfaced code. They are also provided (with limited functionality) for the other TeX engines.

The Gregorian calendar is the most widely calendar use today. The calendar was designed by a commission instructed by Pope~Gregory~XIII in the sixteenth century. This is strictly a solar calendar based on a 365-day common year divided into twelve months. On leap years of 366 days, one extra day is added to February.

For a computer implementation, the easiest way to reckon time is simply to count
days: Establish an arbitrary starting point as day 1 and specify a date by giving
a day number relative to that starting point;\footnote{See Leslie Lamport's \textit{On the Proof of Correctness of a Calendar Program}, Communications of ACM, Vol 22, Number 10, October 1979.} a single thirty-two bit integer allows
the representation of more than 11.7 million years. Such a reckoning of time is,
evidently, extremely awkward for human beings and is not in common use, except
among astronomers who use Julian day numbers to specify dates.

Thus a date is normally a triplet of three integers starting from an era. We specify a date such as (13, December, 2014) where we let "January", \ldots, "December" be the names for the integers $1,\ldots,12$. A calendar is the assignment of dates to days.  More precisely, let us define an era to be an infinite sequence of days. A calendar for that era is an assignment of a date to each day in the era. 

\[gregorian[n] = (day[n], month[n], year[n])\]
where the integer-valued functions $day, month$ and $year$ are defined inductively.

We will also need to define the range that we need to print calendar, as calendars are printed normally over a 42 day interval.

\begin{figure}[h]
\centering
\includegraphics[width=0.7\textwidth]{./images/calendar-01.jpg}
\end{figure}

The full page calendar is illustrated in Figure~\ref{girlcalendar}.
\begin{figure}[htb]
\centering
\includegraphics[width=0.4\textwidth]{./images/calendar-02.jpg}
\caption{The full page rendering of the makeCalendar command}
\label{girlcalendar}
\end{figure}

The calendar holds the images and styles in  a Lua table, which is easily configurable to use it, you need to
use the \cmd{\makeCalendar}\meta{year}, which creates a pdf. care must be used in selecting the right size images and with an aspect ratio that suits the paper dimensions. 

\clearpage

\input{./sections/coptic-calendar}

\subsection{Ethiopic calendar}

The Ethiopic calendar, is very similar to the Coptic. “The day starts with sunrise” is the conceptual basis for the clock in Ethiopia and many of its neighbors. Being near the equator this translates to roughly 6 AM each day with an even 12 hours of light and darkness with only a little seasonal drifting. A twelve hour clock is used that begins at “12 AM” with sunrise (aka 6 AM in the West), reaches “noon” at “6 AM”, followed by “12 PM” 6 hours later and “6 PM” at “midnight”. Think of it as a clock or watch with the “6” at the top and the “12” at the bottom.

The calendar in Ethiopia has 13 months and the year is 7 years 8 months and 11 days behind the Gregorian calendar (12 days when a leap year occurs). Which means that the year 2000 has only just occurred on September 12th of this year! But why? Many references will state that the Ethiopic calendar is based on the Julian, but this is only partly true. The Ethiopic calendar descends more directly from the Coptic which in turn is a reformation of the ancient Egyptian solar calendar with respect to the Julian scheme also known as the “Alexandrian Calendar”.

The ancient Egyptian solar calendar used a 365 day year with the year divided into 3 seasons of 120 days and each season into 4 months of 30 days. Five corrective, or epagomenal, days were added at the end of the year. The months were only numbered initially but later took on the corresponding month names from a second, lunar based calendar of Egypt. The month names under the lunar calendar derived their names from the major feast that would occur during the respective month. The problem with “calendar creep” was not addressed until the arrival of the Julian calendar in 46 BC with the introduction of an extra day for a “leap year”. The Coptic calendar applied the Julian leap year in 25 BC thus forever fixing the date synchronization between the two calendar systems.

However, the Coptic and Ethiopic calendars do not apply the leap year correction rule where leap year is skipped every 100 years, except every 400 years, except (maybe) every 16,000 years. So “calendar creep” will continue between the Coptic, Ethiopic and Gregorian calendars. Leap years however do not occur when the year is a multiple of 4, as with the Gregorian system, but will occur on the year prior. Thus 1999 was a leap year as will be 2003 and so on. The four year cycle is also enumerated with the names of the four Evangelists: Mateos, Markos, Luqas and Yohannes as they are known in Ethiopian Orthodox Church. Yohannes is the leap year and is considered the end of a four year cycle.

Where the Coptic and Ethiopic calendars differ will be in the month names, which are language specific even within Ethiopia, as are the days of the week and day divisions. The other major difference is that the year in the Coptic calendar is presently 1724 some 276 years behind the Ethiopic. The modern Coptic calendar’s origin, or epoch, is counted from the year 284 AD when many Coptic Christians were martyr under the rein of Roman Emperor Diocletian. So what accounts for the difference in the Ethiopic calendar versus the Gregorian? Popular legend has it that the 7 year, 8 month and 11 day difference is the time it took for the news of the Birth of Christ to reach Ethiopia. More likely the answer is that the Ethiopic calendar went through fewer reformations than did other calendar systems (notably skipping the reformation by Dionysius Exiguus in AD 525) which potentially makes it more in keeping with “actual time” since the birth of Christ. The truth is, we may never know.

\begin{luacode}
ethiopicMonths = {
"Meskerem",
"Tekemt",
"Hedar",
"Tahsas",
"Ter",
"Yekatit",
"Megabit",
"Miazia",
"Genbot",
"Sene",
"Hamle",
"Nehasse",
"Pagumen"
}

for i=1,13 do 
   tex.print(ethiopicMonths[i]..', ')
end
\end{luacode}


\section{The French Revolutionary Calendar}

The French Revolution, accompanied by a zeal to change all traditional things in France, induced the rulers to change their calendar along with their government.  It was decreed by the National Convention, in the autumn of 1793, that the old calendar was to be abolished and that the new French era should be reckoned from the foundation of the republic, September 22, 1792, of the Common Era, on the day of the true autumnal equinox; that each year should begin on the midnight of the day on which the autumnal equinox falls; and that the first year of the French Republic had begun immediately after 12 o’clock P.M. of the 21st of September, 1792, and had terminated on the midnight between the 21st and 22nd of September, 1793.

As the French months consisted of 30 days each, making in all 360 days, the remaining five days required to complete the year were called \textit{complementary days} and \textit{sans-culottides}. They were named as follows:

\begin{tabular}{llll}
1. Primedi  & Fe\^ete de la Vertu  & The Virtues & Sept. 17th \\
\end{tabular}

The intercalary day of every fourth year was called La  sans-culottide, and was to be the Festival of 
the Revolution,  to be dedicated to a grand solemnity, in which the French should celebrate the period of their enfranchisement, and the  institution of the Republic. The National oath, "To live  free or die," was to be renewed. 

Each day was divided according to the decimal system, 
into ten parts or hours, and these into ten others, and so on. 
Each month was divided into three decades, each consisting of ten days ; 
the names of which were taken from the 
Latin numerals. The first was called Primedi, 2nd Duodi, 3rd 
Iridi, 4th Quartidi, 5th Quintidi, 6th Sextidi, 7th Septidi, 8th 
Octidi, 9th Nonidi, and 10th Decadi. The last was the day 
of rest, and superseded the former Sunday. \index{Calendars>French Revolutionary}

\begin{figure}[htp]
\centering
\includegraphics[width=\textwidth]{./images/republican-calendar.jpg}
\caption{The months illustrated the Republican Calendar  
Author: Salvatore Tresca in 1794. 
Paris, Musée Carnavalet. }
\end{figure}

\subsection{Revolutionary Calendar to Gregorian Calendar}

The year can easily be reckoned by taking the Gregorian year and adding 1. 

\begin{texexample}{Revolutionary year example}{}
\begin{luacode}
  local year, revyear
  year = 1792
  revyear = 1
  tex.print((year-revyear)+1)
\end{luacode}
\end{texexample}


\section{Julian Day}








 
  
    



 
\DocInput{\jobname.dtx}

\nocite{*}
\printbibliography
\printindex
 %
% 
\end{document}
%</driver>
% \fi
% 
%  \CheckSum{0}
%  \CharacterTable
%  {Upper-case    \A\B\C\D\E\F\G\H\I\J\K\L\M\N\O\P\Q\R\S\T\U\V\W\X\Y\Z
%   Lower-case    \a\b\c\d\e\f\g\h\i\j\k\l\m\n\o\p\q\r\s\t\u\v\w\x\y\z
%   Digits        \0\1\2\3\4\5\6\7\8\9
%   Exclamation   \!     Double quote  \"     Hash (number) \#
%   Dollar        \$     Percent       \%     Ampersand     \&
%   Acute accent  \'     Left paren    \(     Right paren   \)
%   Asterisk      \*     Plus          \+     Comma         \,
%   Minus         \-     Point         \.     Solidus       \/
%   Colon         \:     Semicolon     \;     Less than     \<
%   Equals        \=     Greater than  \>     Question mark \?
%   Commercial at \@     Left bracket  \[     Backslash     \\
%   Right bracket \]     Circumflex    \^     Underscore    \_
%   Grave accent  \`     Left brace    \{     Vertical bar  \|
%   Right brace   \}     Tilde         \~}
%
%
%
% \changes{1.0}{2013/01/26}{Converted to DTX file}
%
% \DoNotIndex{\newcommand,\newenvironment}
% \GetFileInfo{phd.dtx}
% 
%  \def\fileversion{v1.0}          
%  \def\filedate{2012/03/06}
% \title{The \textsf{phd} package.
% \thanks{This
%        file (\texttt{phd.dtx}) has version number \fileversion, last revised
%        \filedate.}
% }
% \author{Dr. Yiannis Lazarides \\ \url{yannislaz@gmail.com}}
% \date{\filedate}
%
%
% 
% ^^A\maketitle
% 
% ^^A\frontmatter
%  ^^A\coverpage{./images/hine02.jpg}{Book Design }{Camel Press}{}{}
%  \newpage
% ^^A\secondpage
% \pagestyle{empty}
%
%
% 
%
%
% \pagestyle{headings}
% \raggedbottom
%  \OnlyDescription
%
%  ^^A\StopEventually{\printindex}

% \CodelineNumbered
% \pagestyle{headings}
% 
% 
% ^^A\part{IMPLEMENTATION AND FRIENDS}
% 
%
% \chapter{Scripts Package Code Implementation Objectives and Strategy}
% 
% \epigraph{
% I was reflecting on the convoluted Java frameworks widely adopted at work. Those hefty frameworks brought coding structures and conventions to large engineering teams; meanwhile, they also sucked the fun of programming like a Pastafarian monster slurping all the tomato sauce on a plate of spaghetti.
%}{\href{http://blog.zmxv.com/2015/07/code-golf-at-google.html}{Zhen Wang}}
%
% We start by outlining what we are trying to achieve with this package:
%
% \begin{enumerate}
% \item To provide a declarative interface to enable users to modify headings by
%       setting keys, rather than writing macros.  
% \item The interface must be able to manupulate properties of headings down to
%       the last detail.
% \item To provide a compatibility mode, where documents wishing to test the package
% can have an easy switch to switch in and out. This is also important for the testing of the package.
% \item To provide a number of templates that cover most of the typical use case.
% \item To provide means for a plug-in architecture for extensions.

% \end{enumerate}
% 
% \section{Terminology}
%
%  \begin{description}
%  \item [document] Any written item, as a book, article, or letter, especially 
%                  of a factual or informative nature.
%  \item [heading] A division of a document or document series. For a normal
%        book headings are chapters, sections etc. However we allow for
%        specifying a more complex document divided into books, volumes
%        parts etc. For example the Bible has Books, chapters and verses,
%        where a legal document might require divisions such as clauses.
%        In general these divisions are numbered. These document divisions
%        are stored in the comma list \refCom{phd_book_divisions_clist}.
%  \item [head] A typeset heading, such as chapter head, or section head.
%        This can include a counter, label and title for example, 
%        \emph{Chapter 1 Introduction}.
%  \item [dom] This is a programming interface that provides a structured
%        representation of the document (a tree) and it defines a way
%        that the structure can be accessed. Although \latexe does not
%        offer a standard way to build such a tree (mainly because
%        \tex does not require the marking of paragraphs, it is 
%        useful to think of the document as a tree structure. We also
%        allow for a semi-automated way to build such a tree (with the 
%        exception that paragraphs are not included).
% \item [element] A part of the document tree that can be styled on
%       its own. For example the chapter label, or the section number.
%
% \end{description}
%
% \section{Users}
%  We classify users according to the \LaTeX3 terminology as a) programmers b) template designers
%  and c) authors.
% \subsection{Author}
%  We assume that the author has an exising template which she is using but might want to do
%  some minor modifications, for example use an italic shape for the font of the mark, but an 
%  upright font for the page numbers. 
%
% {\obeylines 
%~~ |\cxset|
%~~~~~|{|
%~~~~~~~~\textit{chapter number color}~~|format          = apa,|
%~~~~~~~~\textit{section title font-size} |font-size   = Large,|
%~~~~~|}|
%}  
%
% We follow the idea of representing the basic elements of documents
% as elements, each one having a parent in order to specify
% the element we need to style as accurate as possible. One can think of
% this approach being congruent with objects in other languages.
% As a matter fact nothing stops us from defining a key value
% interface as shown below.
%
% {\obeylines 
%~~ |\cxset|
%~~~~~|{| 
%~~~~~~~~\textit{header.even.mark.font.size}   = |Large,|
%~~~~~~~~\textit{header.even.mark.font.family} = |serif,|
%~~~~~|}|
%}  
%
% This would pehaps make it easier for the template designer, but I have rejected
% the idea as my aim is to make it easy for the author, who can search the template
% and just enter a couple of new proerty values.
%
% \subsection{Template designer}
% \pagestyle{headings}
% The template designer in the example above would have selected the format style
% from a number of predefined formats (templates) or would have created a style
% called \textit{apa} from an existing template and modified it using declarative
% key style.
%
% \subsection{The programmer}
%
% The programmer in the example above could have created the basic format
% \textit{apa} by using both declarative as well as defining or using existing
% macros. To the programmer we offer an extension mechanism, where the contents
% of a |ps@| command are defined. For example the programmer can define a new
% style using \tikzname, but without having to worry about defining full |ps@|
% and their interface.
%
% \section{Preliminaries}
%
%  Standard file identification. We first announce the package 
%	 and require that it be used with \LaTeX2e. 
% \iffalse
%<*SCRIPTS>
% \fi
%  
%
%    \begin{macrocode}
\NeedsTeXFormat{LaTeX2e}[1994/12/01]%
\RequirePackage[2014/05/01]{latexrelease}
\ProvidesFile{phd-lists}[2015/1/13 v1.0 less preamble (YL)]%
%    \end{macrocode}
%
% \chapter{Scripts and Languages }
% 
%    \begin{macrocode} 
%
\ifengine
  {
    \PassOptionsToPackage{italian,english}{polyglossia}
    \RequirePackage{polyglossia}
    \setdefaultlanguage{english}}
  {%
   \expandafter\PassOptionsToPackage{italian,english}{babel}
    \RequirePackage[italian,english]{babel}
%    \RequirePackage{polyglossia}
%    \setdefaultlanguage{english}
  }
  {\PassOptionsToPackage{italian,english }{babel}
    \RequirePackage{babel}
  }
%
%    \end{macrocode}
%
%    \begin{macrocode}  
\RequirePackage[italian,english]{xlayouts}
%    \end{macrocode} 
% The clist \refCom{g_phd_scripts_clist} holds a list of all the scripts that have been loaded.
% Managing the user interface is problematic, we will have users that require
% only one script and users that might want all of them.
% There is also the issue between the blurring of alphabets, languages and scripts
% Since we will always specify a pan-unicode font, which we will make available
% with the |phd| package. We map all scripts to this font first.
%
% \begin{docCommand}{g_phd_scripts_clist} {\meta{clist}}
%   Holds a clist of all scripts loaded.
% \end{docCommand}
%
% 
%  Declare two global lists to hold all the scripts available.
% The |\script_prop| holds info for each script loaded
%
%    \begin{macrocode}
\ExplSyntaxOn
\clist_new:N \g_phd_scripts_clist
\clist_new:N \g_phd_noto_clist
\prop_new:N \script_prop
%\input{notolist.txt.tex}
%    \end{macrocode}
%
% \begin{docCommand}{g_phd_noto_clist}{clist}
% Holds a list of all noto fonts available.
% \end{docCommand} 
%
% \begin{docCommand}{printnotofontlist} {clist}
% It typesets a list in a two column environment with all the available Noto fonts.
% \end{docCommand}

% 
%    \begin{macrocode}
\cs_set:Npn \printnotofontlist 
  {
    \begin{multicols}{2}
      \clist_map_inline:Nn \g_phd_noto_clist
        {
          ##1\par 
		  }
    \end{multicols}  
  }
%    \end{macrocode}	
% 
% 
%    \begin{macrocode}	
\prop_put:Nnn \script_prop {name}{Armenian}
\prop_put:Nnn \script_prop {fonts}{NotoArmenian-Regular.ttf, Others}
\prop_get:NnN \script_prop {fonts}\l_tempa_tl
\prop_put:Nnn \script_prop {group}{Europe}
\prop_get:NnN \script_prop {group} \l_tempa_tl
%    \end{macrocode}
%
% \begin{docCommand}{SetPanUnicodeFont}{\marg{font name}}
%  Sets the pan-unicode font. This font is to be used as a default for all the scripts
%  The user can override it with another font.
% \end{docCommand}
%
%    \begin{macrocode}
\NewDocumentCommand\SetPanUnicodeFont { m }
  {
     \gdef\panunicodefontface{#1}
     \newfontfamily\panunicode[Scale=MatchUppercase]{#1}
  }
%    \end{macrocode}  
%
%  We set the \docAuxCommand{panunicode} to |code2000.ttf|.
%
%    \begin{macrocode}  
\SetPanUnicodeFont{code2000.ttf}    
%    \end{macrocode}

% \begin{docCmd} {makepanfontfamily} { \marg{script name} }
%    
% \end{docCmd}
%    \begin{macrocode}
\cs_gset:Npn \makepanfontfamily#1
  {
%  \newfontfamily\cs:w #1fontfamily\cs_end: { #2 }
  \cs_gset_eq:cN {#1fontfamily} \panunicode
  \cs_gset_eq:cc {#1} {#1fontfamily}
}
%    \end{macrocode}
% 
% \begin{docCmd} {add_a_script:n} { \marg{script name}}
%   Given a script name this function, adds it to the tracking list
%   creates an appropriate envrironment and also a |text<script>| command. 
%   This might overwrite similar commands defined by other
%   packages.
% \end{docCmd}
%    \begin{macrocode}
\cs_gset:Npn \add_a_script:n #1
 {
   \clist_gput_left:Nn \g_phd_scripts_clist {#1 }
   \createscriptenvironment {#1}
   \createtextscript {#1}
 }   
 
 % add a script
\NewDocumentCommand\addascript { m } 
  {
    \add_a_script:n {#1}
  }
  
% Mock an environment 
\gdef\createscriptenvironment #1{
   \exp_after:wN\gdef\csname #1script\endcsname{\group_begin:
      \csname #1fontfamily\endcsname}
   \exp_after:wN\gdef\cs:w end#1script\cs_end:{\group_end: }
}  
\ExplSyntaxOff
%    \end{macrocode}
%  
% \begin{docCommand}{createtextscript}{ \marg{script name}}
%   This creates a command of the form |\text|\meta{script name} i.e., for tibetan
%   it will produce |\texttibetan|
% \end{docCommand}
%    \begin{macrocode}
\ExplSyntaxOn
\cs_gset:Npn \createtextscript #1{
   \long\exp_after:wN\gdef\csname text#1\endcsname ##1
   {
      \group_begin: 
      \cs:w #1fontfamily\cs_end:
        ##1
     \group_end:
   }
}  
%
%
\cs_gset:Npn \makefontfamily#1#2 {
\if_meaning:w\panunicodefontface#2
  \else:
  \exp_after:wN
  \newfontfamily\cs:w #1fontfamily\cs_end: { #2 }
  \cs_gset_eq:cc {#1} {#1fontfamily}
\fi:  
}

\ExplSyntaxOff

\NewDocumentCommand\AddScript { m } {
    \cxset{script/.code=\addascript{##1}}
    \cxset{#1 font/.code=\makefontfamily{#1}{##1}}
    \cxset{script=#1}
    \cxset{#1 font=\panunicodefontface}
}
\cxset{add script/.code = \AddScript{#1}}

\ExplSyntaxOn
\clist_gset:Nn \g_phd_scripts_clist 
  {
      armenian,
      %hebrew,
      % arabic,
      syriac,
      thaana,
      devanagari,
      bamum,
      bengali,
      brahmi,
      buhid,
      bopomofo,
      cham,
      cherokee,
      cjk,
      coptic,
      cypriot,
      %e
      ethiopic,
      georgian,
      glagolitic,
      gurmukhi,
      gujarati,
      kayahli,
      lao,
      lisu,      
      kannada,
      malayalam,
      myanmar,
      ogham,
      oriya,
      oldturkic,
      phoenician, 
      runic,
      tamil,
      thai,
      tibetan,
      tifinagh, 
      telugu, 
      vai,
      rejang,
      saurashtra,
      sinhala,
      sylhetinagari,
      sundanese,%check this
      yi,%check
      meitei,%check
      mongolian,
}

\clist_map_inline:Nn\g_phd_scripts_clist 
  {
    \AddScript{#1}
    \makepanfontfamily {#1}
  }
\ExplSyntaxOff

\newfontfamily\arabicfont[Script=Arabic]{Amiri}
\newfontfamily\arabicfonttt[Script=Arabic,Scale=.75]{DejaVu   Sans Mono}
\newenvironment{Arabic}
   {\bgroup \arabicfont}
   {\egroup}
%    \end{macrocode}
%
% A small utility macro to typeset unicode tables
% examples can be see in the chapters for scripts.
%puts the unicode label (removes last char and adds x)
%
% \begin{docCommand} {putunicode@label} {\marg{unformatted string}} 
%  This macro receives a number in hexadecimal, removes the last
%  0 and replaces it with an x. It then prepends a U+ to fomat it
%  as a Unicode number e.g. U+0100x
% \end{docCommand}
% 
%    \begin{macrocode}
\newcounter{glyph@count}%counts glyphs
%    \end{macrocode}
%		
%		
%    \begin{macrocode}
\ExplSyntaxOn
\def\textU#1{{\unicodenumberfam #1}}
\ExplSyntaxOff
%    \end{macrocode}
%		
%    \begin{macrocode}
\def\putunicode@label#1#2;{%
%    \end{macrocode}
%    
%    \begin{macrocode}
\def\reformat@unicode@string##1{%
   \textU{U+}%
  \let\z\empty%
  \expandafter\@tfor\expandafter\i\expandafter:\expandafter=#2;\do{%
  \if\i;%
    \textU{x}%
  \else%
    \textU{\z}%
  \fi%
  \edef\z{\i}%
 }%
}%
  \makebox[5em]{\reformat@unicode@string{#2}\hfill}%
}
%    \end{macrocode}
% 
% \begin{docCommand} {putchar@cx} {\meta{char}}
% \end{docCommand}
% 
%    \begin{macrocode}
\def\putchar@cx#1{%
\stepcounter{glyph@count}
\let\oldactive@prefix\active@prefix
\let\active@prefix\relax
   \iffontchar\font\n
     \char\the\n$_{\pgfmathparse{Hex(\the\r@cx)}\text{\pgfmathresult}}$%
      %
   \else
    {\arial\graybox}
   \fi
\let\active@prefix\oldactive@prefix
 }
%    \end{macrocode}
%    
%  typesets one row of a unicode table
%    \begin{macrocode}    
\def\urow@cx#1{%
    \n=#1% 
    \r@cx=0%
    \expandafter\putunicode@label#1;%
    \loop%
        \ifnum\n<\numexpr#1+16\relax%
        \makebox[1.9em]{\expandafter\putchar@cx{#1}}%
        \advance\r@cx by1%  
        \ifnum\r@cx>16\r@cx=1\relax\else\fi
        \advance\n by1%
    \repeat
    \par
}

\def\typeseturows@cx#1{%
\@for\next:=#1\do{%
  \urow@cx\next\vskip3pt}%
}

\newcount\r@cx%
\newcount\n%
\newcommand\unicodetable[2]{%
\bgroup
  \par
  \leavevmode%
   \r@cx=0%
   {\hbox to 5em{\ignorespaces}}%
   \loop%
    \ifnum\r@cx<16\ignorespaces 
    \makebox[1.9em]{\pgfmathparse{Hex(\the\r@cx)}\pgfmathresult}%
    \advance\r@cx by\@ne%  
   \repeat
   \vskip3pt\par
   \@nameuse{#1}%
   \typeseturows@cx{#2}%
\egroup
}
%    \end{macrocode}
% \begin{docCommand} {unicodenumber} {\meta{string}}
% Typesets a string such as |x1020| in a typewriter font.
% \end{docCommand}
%    \begin{macrocode}    
\DeclareRobustCommand\unicodenumber[1]{{\ttfamily #1\xspace}}
%    \end{macrocode}
%    
%    \begin{macrocode}
\def\putdescription#1:{%
  {\parindent0pt 
  \begin{minipage}[t]{4cm}
  \bgroup\aegean
  \hangindent20pt
  #1\par
  \egroup
  \end{minipage} 
  }
}


\long\def\parsefields #1:#2\@@{%
    \ifx\par#1
    \else 
        {\small\aegean U+#1}%
         %%\iffontchar\font"#1 %
          \makebox[2.1em]{\color{theunicodesymbolcolor}\symbol{"#1}}% 
          \expandafter\putdescription#2\vskip3pt
        %%\else
          %%{\aegean \makebox[2.1em]{} Unallocated\par}%
        %%\fi
    \fi  
  }%
% Check if it can be saved
\newread\tempstream%s
%    \end{macrocode}
%
% \begin{docCommand}{printunicodeblock}{ \oarg{no columns} \marg{filename} \marg{fontcmd}}
%  The macro prints a unicode table from a file of definitions. This is
%   printed in a two column environment by default. 
% \end{docCommand}
% 
%    \begin{macrocode}
%\ExplSyntaxOn
\DeclareDocumentCommand{\printunicodeblock}{O{2} m m }
  {
    \bgroup
    \leavevmode\parindent0pt\par
    \begin{multicols}{#1}%
     #3
      \openin\@inputcheck=#2
      \loop\unless\ifeof\@inputcheck
      \read\@inputcheck to\fileline %
      \expandafter\parsefields \fileline:\@@ 
      \repeat
    \end{multicols}%
      \immediate\closein\@inputcheck
      \egroup
  }
\let\PrintUnicodeBlock\printunicodeblock
%\ExplSyntaxOff
%    \end{macrocode}
% 
%    \begin{macrocode}
\let\indicative\pan
\newfontfamily\brahmi{Noto Sans Brahmi}
\newfontfamily\bengal[Script=Bengali,Scale=1]{Shonar Bangla}
%    \end{macrocode}
%</SCRIPTS>
\endinput