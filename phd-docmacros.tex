%<*DOCS>
%\subsection{meta}\label{meta}
%
% This has been lifted from |doc|
% {meta}
%    The |\meta| macro is a bit tricky. We want to allow line
%    breaks at blanks in the argument but we don't want a break
%    in between. In the past this was done by defining |\meta| in a way that a
%    \verb*+ + is active when the argument is scanned. Words are then
%    scanned into |\hbox|es. The active \verb*+ + will end the
%    preceding |\hbox| add an ordinary space and open a new
%    |\hbox|. In this way breaks are only possible at spaces.  The
%    disadvantage of this method was that |\meta| was neither robust
%    nor could it be |\protect|ed. The new implementation  fixes this
%    problem by defining |\meta| in a radically different way: we
%    prevent hypenation by defining a |\language| which has no
%    patterns associated with it and use this to typeset the words
%    within the angle brackets. see \sref{meta}
% 
%    \begin{macrocode}
\ifx\l@nohyphenation\undefined
  \newlanguage\l@nohyphenation
\fi
%    \end{macrocode}
%    
%    \begin{macrocode}
\DeclareRobustCommand\meta[1]{%
%    \end{macrocode}
%    Since the old implementation of |\meta| could be used in math we
%    better ensure that this is possible with the new one as
%    well. So we use |\ensuremath| around |\langle| and
%    |\rangle|. However this is not enough: if |\meta@font@select|
%    below expands to |\itshape| it will fail if used in math
%    mode. For this reason we hide the whole thing inside an
%    |\nfss@text| box in that case.
%    \begin{macrocode}
     \ensuremath\langle
     \ifmmode \expandafter \nfss@text \fi
     {%
      \meta@font@select
%    \end{macrocode}
%    Need to keep track of what we changed just in case the user
%    changes font inside the argument so we store the font explicitly.
%    \begin{macrocode}
      \edef\meta@hyphen@restore
        {\hyphenchar\the\font\the\hyphenchar\font}%
      \hyphenchar\font\m@ne
      \language\l@nohyphenation
      #1\/%
      \meta@hyphen@restore
     }\ensuremath\rangle
}
%    \end{macrocode}
% 
%
%
% begin{macro}{meta@font@select} 
%  	We default the definition to upshape.
%    \begin{macrocode}
\def\meta@font@select{\upshape}
%    \end{macrocode}
% 
%
% begin{macro}{macro} 
%	The \cs{macro} environment is straight out of the
%	\pkg{doc} also. We redefine it here to allow usage in documents that have not 
%	preloaded the package.
%
%    \begin{macrocode}
\def\macro{\begingroup
   \catcode`\\12
   \MakePrivateLetters \m@cro@ \iftrue}
%    \end{macrocode}
% 
%
% begin{macro}{environment}
%    The ``environment'' envrironment will be implemented just like the
%    ``macro'' environment flagging any differences in the code by
%    passing |\iffalse| or |\iftrue| to the |\m@cro@| environment
%    doing the actual work.
%    \begin{macrocode}
\def\environment{\begingroup
   \catcode`\\12
   \MakePrivateLetters \m@cro@ \iffalse}
%    \end{macrocode}
% 
%
%    After scanning the argument we close the group to get the normal
%    |\catcode|$\,$s back. Then we assign a special value to
%    |\topsep| and start a \textsf{trivlist} environment. (Modified for normal indexing by YL)
%
%    \begin{macrocode}
\long\def\m@cro@#1#2{\index{\string#2}\endgroup \topsep\MacroTopsep \trivlist
%    \end{macrocode}
% We also save the name being described in |\saved@macroname| for
% 
%    \begin{macrocode}
   \edef\saved@macroname{\string#2}%
%    \end{macrocode}
%    Now there follows a variation of |\makelabel| which is used
%    should the environment not be nested, or should it lie between
%    two successive || instructions or explanatory
%    text.  One can recognize this with the switch |\if@inlabel|
%    which will be |true| in the case of successive |\item|
%    commands.
%
%    \begin{macrocode}
  \def\makelabel##1{\llap{##1}}%\llap
%    \end{macrocode}
%
%    If |@inlabel| is |true| and if $\verb=\macro@cnt= > 0$
%    then the above definition needs to be changed, because in this
%    case \LaTeX{} would otherwise put the labels all on the same line
%    and this would lead to them being overprinted on top of each
%    other.  Because of this |\makelabel| needs to be redefined
%    in this case.
%    \begin{macrocode}
  \if@inlabel
%    \end{macrocode}
%    If |\macro@cnt| has the value $1$, then we redefine
%    |\makelabel| so that the label will be positioned in the
%    second line of the margin.  As a result of this, two macro names
%    appear correctly, one under the other.  It's important whilst
%    doing this that the generated label box is not allowed to have
%    more depth than a normal line since otherwise the distance
%    between the first two text lines of \TeX{} will be incorrectly
%    calculated. The definition should then look like:
%\begin{verbatim}
%     \def\makelabel##1{\llap{\vtop to \baselineskip
%          {\hbox{\strut}\hbox{##1}\vss}}}
%\end{verbatim}
%    Completely analogous to this is the case where labels need to be
%    placed one under the other.  The lines above are only an example
%    typeset with the \textsf{verbatim} environment. To produce the real
%    definition we save the value of |\macro@cnt| in
%    |\count@| and empty the temp macro |\@tempa| for later
%    use.
%    \begin{macrocode}
    \let\@tempa\@empty \count@\macro@cnt
%    \end{macrocode}
%    In the following loop we append for every already typeset label
%    an |\hbox{\strut}| to the definition of |\@tempa|.
%    \begin{macrocode}
    \loop \ifnum\count@>\z@
      \edef\@tempa{\@tempa\hbox{\strut}}\advance\count@\m@ne \repeat
%    \end{macrocode}
%    Now be put the definition of |\makelabel| together.
%
%    \begin{macrocode}
    \edef\makelabel##1{\llap{\vtop to\baselineskip
                               {\@tempa\hbox{##1}\vss}}}%
%    \end{macrocode}
%    Next we increment the value of the nesting depth counter.  This
%    value inside the \textsf{macro} environment is always at least one
%    after this point, but its toplevel definition is zero. Provided
%    this environment has been used correctly, $|\macro@cnt|=0$
%    should not occur when |@inlabel|=\textsf{true}.  It is
%    however possible if this environment is used within other list
%    environments (but this would have little point).
%    \begin{macrocode}
    \advance \macro@cnt \@ne
%    \end{macrocode}
%    If |@inlabel| is false we reset |\macro@cnt| assuming
%    that there is enough room to print the macro name without
%    shifting.
%    \begin{macrocode}
  \else  \macro@cnt\@ne  \fi
%    \end{macrocode}
%    Now the label will be produced using |\item|. The following
%    line is only a hack saving the day until a better solution is
%    implemented.  We have to face two problems: the argument might be
%    a |\par| which is forbidden in the argument of other macros
%    if they are not defined as |\long|, or it is something like
%    |\iffalse| or |\else|, i.e.\ something which will be
%    misinterpreted when \TeX{} is skipping conditional text. In both
%    cases |\item| will bomb, so we protect the argument by using
%    |\string|.
%    \begin{macrocode}
  \edef\@tempa{\noexpand\item[%
%    \end{macrocode}
%    Depending on whether we are inside a ``macro'' or ``environment''
%    environment we use |\PrintMacroName| or |\PrintEnvName| to
%    display the name.
%    \begin{macrocode}
     #1%
       \noexpand\PrintMacroName
     \else
       \noexpand\PrintEnvName
     \fi
     {\string#2}]}%
  \@tempa
%    \end{macrocode}
%    At this point we also produce an index entry.  Because it is not
%    known which index sorting program will be used, we do not use the
%    command |\index|, but rather a command
%    |\SpecialMainIndex| after advancing the counter for indexing
%    by line number.  This may be redefined by the user in
%    order to generate an index entry which will be understood by the
%    index program in use (note the definition of
%    |\SpecialMainIndex| for our installation).
%
%    We advance the current codeline number and after producing an
%    index entry revert to the original value
%    \begin{macrocode}
  \global\advance\c@CodelineNo\@ne
%    \end{macrocode}
%    Again the macro to call depends on the environment we are
%    actually in.
%    \begin{macrocode}
   #1%
      \nobreak
      \DoNotIndex{#2}%
   \else
      \SpecialMainEnvIndex{#2}\nobreak
   \fi
  \global\advance\c@CodelineNo\m@ne
%    \end{macrocode}
%    The |\nobreak| is needed to prevent a page break after the
%    |\write| produced by the |\SpecialMainIndex| macro.  We
%    exclude the new macro in the cross-referencing feature, to
%    prevent spurious non-main entry references.  Regarding possibly
%    problematic arguments, the implementation takes
%    care of |\par| and the conditionals are uncritical.
%
%    Because the space symbol should be ignored between the
%    |{...}| and the following text we must take
%    care of this with |\ignorespaces|.
%    \begin{macrocode}
  \ignorespaces}
%    \end{macrocode}
% 
%	We now ready to define the code for the end of the environments.	
%	
% {endmacro}
% {endenvironment}
%     Older releases of this environment omit the |\endgroup| token,
%     when being nested. This was done to avoid unnessary stack usage.
%     However it does not work if \textsf{macro} and
%     \textsf{environment} environments are mixed, therefore we now
%     use a simpler approach.
%
%    \begin{macrocode}
\let\endmacro \endtrivlist
\let\endenvironment\endmacro
%    \end{macrocode}
%  
%  
%
% {MacroTopsep}
%    Here is the default value for the |\MacroTopsep| parameter
%    used above.
%    \begin{macrocode}
\newskip\MacroTopsep     \MacroTopsep = 7pt plus 2pt minus 2pt
%    \end{macrocode}
% 
%
%
% \subsection{Formatting the margin}
%
% The following three macros should be user definable.
% Therefore we define those macros only if they have not already
% been defined.
%
% begin{macro}{PrintMacroName}
% egin{macro}{PrintEnvName}
% begin{macro}{PrintDescribeMacro}
% begin{macro}{PrintDescribeEnv}
%    The formatting of the macro name in the left margin is done by
%    these macros. We first set a |\strut| to get the height and
%    depth of the normal lines. Then we change to the
%    |\MacroFont| using |\string| to |\catcode| the
%    argument to other (assuming that it is a macro name). Finally we
%    print a space.  The font change remains local since this macro
%    will be called inside an |\hbox|. NEED TO FIX
%    \begin{macrocode}
\@ifundefined{PrintMacroName}
   {\def\PrintMacroName#1{\strut \MarginMacroFonts \string #1\ }}{\def\PrintMacroName#1{\strut \MarginMacroFonts \string #1\ }}
%    \end{macrocode}
%    We use the same formatting conventions when describing a macro.
%    \begin{macrocode}
\@ifundefined{PrintDescribeMacro}
   {\def\PrintDescribeMacro#1{\strut \MacroFonts \string #1\ }}{\def\PrintDescribeMacro#1{\strut \MarginMacroFonts \string #1\ }}
%    \end{macrocode}
%    To format the name of a new environment there is no need to use
%    |\string|.
%    \begin{macrocode}
\@ifundefined{PrintDescribeEnv}
   {\def\PrintDescribeEnv#1{\strut \MacroFonts #1\ }}{\def\PrintDescribeEnv#1{\strut \MarginMacroFonts #1\ }}
\@ifundefined{PrintEnvName}
   {\def\PrintEnvName#1{\strut \MarginMacroFonts #1\ }}{\def\PrintEnvName#1{\strut \MarginMacroFonts #1\ }}
%    \end{macrocode}

%
% begin{macro}{MarginMacroFont} As we dont care for older versions of LaTeX we simplify the
%	code provide by \pkg{doc}. We also add a hook for color.
%	as we do not expect that the package will be used in places with no colour support.
%	The command used in the original \pkg{doc} is the same as the one used to
% 	typeset the code for |macrocode|, however we wish to have the option to color
%	the margin macros separately.
%	
%    \begin{macrocode}
\ifxetex
  \def\MarginMacroFonts{\color{spot!60}\ttfamily}
\else
  \ifluatex
    \def\MarginMacroFonts{\color{spot!60}\ttfamily}
  \else
    \def\MarginMacroFonts{%
                  \fontencoding\encodingdefault
                   \fontfamily\ttdefault
                   \fontseries\mddefault
                   \fontshape\updefault
                   \color{red}\small}%
  \fi
\fi
 \let\citep\cite
%    \end{macrocode}
% \section{Documentation Macros}

% This section defines commands for printing documentation
% such as this one. It draws inspiration and plagiarizes pgf,
% doc,symbols and many other packages for which I am grateful.
% First some macros for indexing commands.
% 
%    \begin{macrocode}
% Define a table environment that's similar to symtable except that it
% floats and it doesn't write an entry into the Table of Contents.  This
% is used for tables that contain something other than symbol lists.
\def\oarg#1{%
  \colOpt{{\ttfamily[}\meta{#1}{\ttfamily]}}}%
%  
\def\DescribeMacro{\leavevmode\@bsphack
   \begingroup\MakePrivateLetters\Describe@Macro}
\def\Describe@Macro#1{\endgroup
              {\raggedleft\PrintDescribeMacro{#1}}%
              \SpecialUsageIndex{#1}\@esphack\ignorespaces}


\def\DescribeEnv{\leavevmode\@bsphack\begingroup\MakePrivateLetters
  \Describe@Env}
\def\Describe@Env#1{\endgroup
              {\raggedleft\PrintDescribeEnv{#1}}{}%
              \SpecialEnvIndex{#1}\@esphack\ignorespaces}
\setlength\marginparpush{0pt}  



\newlength{\atemp}
 \def\PrintDescribeMacro#1{%
  \settowidth\atemp{\string #1} 
  \strut\MacroFont\color{thered}\normalsize\string#1}

\def\Describe#1{%
   \settowidth\atemp{\string #1}% 
  \par\penalty-500\vskip3ex\noindent
  \DescribeMacro{#1}\args}
\def\DescribeOther{\vskip-4ex\Describe}

\def\args#1{%
  \def\bbl@tempa{#1}%
  \ifx\bbl@tempa\@empty\else#1\vskip1ex\fi\ignorespaces}


\newenvironment{nonsymtable}[1]{%
  \begin{table}[htbp]
  \centering
  \caption{#1}\medskip
}{%
  \end{table}
}
%    \end{macrocode}
%

% Index "X Y" and "Y, X".  The "begin" and "end" variants are for page ranges.
%    \begin{macrocode}
\newcommand{\cmdI}[2][]{%
  \def\first@arg{#1}%
  \ifx\first@arg\@empty
    \texttt{\verbatimfont\string#2}\indexcommand[#2]{#2}%
  \else
    \texttt{\verbatimfont\string#2}\indexcommand[#1]{#2}%
  \fi
}


\newcommand{\cmdX}[1]{\cmdI[$\string#1$]{#1}}
\newcommand{\cmdW}[1]{\cmdI[$\string\blackacc{\string#1}$]{#1}}
\newcommand{\cmdIp}[1]{\texttt{\string#1}\indexpunct[$#1$]{#1}}
%    \end{macrocode}

% {CMDI}\oarg[symbol command]\marg{command}
% This macro \#1 symbol to be typeset next to
% \#2 in the index |\gothic (symbol)|
%
%    \begin{macrocode}
\DeclareRobustCommand\CMDI[1]{%
\bgroup%
\smallskip 
\noindent\texttt{\verbatimfont\string#1}%
\indexcommand{#1}%
\egroup%
}

\DeclareRobustCommand\luacmd[1]{%
  \bgroup
    \smallskip
    \noindent\color{black}\textbf{\string#1}%
    \indexcommand{#1}
 \egroup%
}

\DeclareRobustCommand\luafunction[1]{%
  \bgroup
    \smallskip
    \noindent\color{black}\textbf{\verbatimfont#1}%
    \indexcommand{#1}
 \egroup%
}

%    \end{macrocode}
% 
%</DOCS> 