\chapter{Go Web Services}

\section{Building our first route}

When we talk about routing in Go, we're talking about a multiplexer or |mux|. In this case
the multiplexer refers to taking a URLs or URL patterns and translating them into internal
functions.

One way to think about this, is that they are simple mappings from a request to a function (or handler).

At this stage one needs to think about the API design, for example:

\begin{teX}
/api/user  func apiUser
/api/message  func apiMessage
/api/status  func apiStatus
\end{teX} 

There are some limitations with the built-in |mux/router| provided by the |net/http| package. You cannot, for example, supply a wildcard or a regular expression to a route.

You might expect to be able to do something as discussed in the following code snippet:

However, this results in a parsing error.

If you've spent any serious time in any mature web API, you'll know that this won't do. We need to be able to react to dynamic and unpredictable requests. By this we mean that anticipating every numerical user is untenable as it relates to mapping to a function. We need to be able to accept and use patterns.

There are a few solutions for this problem. The first is to use a third-party platform that has this kind of robust routing built in. There are a few very good platforms to choose from, so we'll quickly look at these now.

\section{Gorilla}

Gorilla is an all-inclusive framework providing packages for web services. It also supplies some
other useful tools, such as JSON-RPC, secure cookies, and global session data.

\subsection{The \texttt{gorilla/mux} package}

Package |gorilla/mux| implements a request router and dispatcher.

The name mux stands for "HTTP request multiplexer". Like the standard http.ServeMux, mux.Router matches incoming requests against a list of registered routes and calls a handler for the route that matches the URL or other conditions. The main features are:

* Requests can be matched based on URL host, path, path prefix, schemes,
  header and query values, HTTP methods or using custom matchers.
  
* URL hosts and paths can have variables with an optional regular
  expression.
  
* Registered URLs can be built, or "reversed", which helps maintaining
  references to resources.
  
* Routes can be used as subrouters: nested routes are only tested if the
  parent route matches. This is useful to define groups of routes that
  share common conditions like a host, a path prefix or other repeated
  attributes. As a bonus, this optimizes request matching.
  
* It implements the http.Handler interface so it is compatible with the
  standard http.ServeMux.


Gorilla's mux package lets us use regular expressions, but it also has some shorthand expressions that let us define the kind of request string we expect without having to write out full expressions.

For example, if we have a request like /api/users/309, we can simple route it as follows in Gorilla:

\begin{teX}
gorillaRoute := mux.NewRouter()
gorillaRoute.HandleFunc("/api/{user}", UserHandler)
\end{teX}


However, there is a clear risk in doing so—by leaving this so open-ended, we have the potential to get some data validation issues. If this function accepts anything as a parameter and we expect digits or text only, this will cause problems in our underlying application.

So, \pkgname{Gorilla} allows us to clarify this with regular expressions, which are as follows:

\begin{lstlisting}
r := mux.NewRouter()
r.HandleFunc("/products/{user:\d+}", ProductHandler)
\end{lstlisting}





