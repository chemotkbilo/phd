% \iffalse meta-comment
%<*internal>
\iffalse
%</internal>
%<*readme>
----------------------------------------------------------------
phd-runningheads 
A package to manage running heads in LaTeX
E-mail: yannislaz@gmail.com
Released under the LaTeX Project Public License v1.3c or later
See http://www.latex-project.org/lppl.txt
----------------------------------------------------------------
This file provides a template for defining a class.
%</readme>
%<*todo>
Improve on User markup
%</todo>
%<*internal>
\fi
\def\nameofplainTeX{plain}
\ifx\fmtname\nameofplainTeX\else
  \expandafter\begingroup
\fi
%</internal>
%<*install>
\input docstrip.tex
\keepsilent
\askforoverwritefalse
\preamble
----------------------------------------------------------------
phd-runningheads 
A package to manage running heads in LaTeX
E-mail: yannislaz@gmail.com
Released under the LaTeX Project Public License v1.3c or later
See http://www.latex-project.org/lppl.txt
----------------------------------------------------------------
\endpreamble
\postamble
 Copyright (C) 2015 by Dr. Yiannis Lazarides <yannislaz@gmail.com>
\endpostamble
%\usedir{tex/latex/\jobname}
\generate{
  \file{\jobname.sty}{\from{\jobname.dtx}{RH}}
 }
%</install>
%<install>\endbatchfile

%<*internal>
%\usedir{source/latex/\jobname}
\generate{
  \file{\jobname.ins}{\from{\jobname.dtx}{install}}
}
\nopreamble\nopostamble
%\usedir{doc/latex/demopkg}
\generate{
  \file{README.txt}{\from{\jobname.dtx}{readme}}
}
%\generate{
%  \file{phd-testhead.tex}{\from{\jobname.dtx}{TEST}}
%}
%\generate{
%  \file{TODO.tex}{\from{\jobname.dtx}{TODO}}
%}
\ifx\fmtname\nameofplainTeX
  \expandafter\endbatchfile
\else
  \expandafter\endgroup
\fi
%</internal>
%<*driver>
\NeedsTeXFormat{LaTeX2e}
\ProvidesFile{phd-runningheads.drv}%
  [2013/01/13 v1.0 ]%
\documentclass[twoside,11pt,a4paper]{ltxdoc}
\usepackage[bottom=4cm,footskip=3cm,
            headsep=2cm]{geometry}
\savegeometry{std}
\usepackage{phd}
\usepackage{phd-lowersections}
\usepackage{phd-runningheads}
%% LaTeX2e file `defaults-chapters'
%% generated by the `filecontents' environment
%% from source `phd-scriptsmanager' on 2015/08/25.
%%
%%    General Defaults for Chapters
\cxset{%
    chapter title margin-top-width    =  0cm,
    chapter title margin-right-width  =  1cm,
    chapter title margin-bottom-width = 10pt,
    chapter title margin-left-width   = 0pt,
    chapter align                     = left,
    chapter title align               = left, %checked
    chapter name                      = hang,
    chapter format                    = fashion,
    chapter font-size                 = Huge,
    chapter font-weight               = bold,
    chapter font-family               = sffamily,
    chapter font-shape                = upshape,
    chapter color                     = black,
    chapter number prefix             = ,
    chapter number suffix             = ,
    chapter numbering                 = arabic,
    chapter indent                    = 0pt,
    chapter beforeskip                = -3cm,
    chapter afterskip                 = 30pt,
    chapter afterindent               = off,
    chapter number after              = ,
    chapter arc                       = 0mm,
    chapter background-color          = bgsexy,
    chapter afterindent               = off,
    chapter grow left                 = 0mm,
    chapter grow right                = 0mm,
    chapter rounded corners           = northeast,
    chapter shadow                    = fuzzy halo,
    chapter border-left-width         = 0pt,
    chapter border-right-width        = 0pt,
    chapter border-top-width          = 0pt,
    chapter border-bottom-width       = 0pt,
    chapter padding-left-width        = 0pt,
    chapter padding-right-width       = 10pt,
    chapter padding-top-width         = 10pt,
    chapter padding-bottom-width      = 10pt,
    chapter number color              = white,
    chapter label color               = white,
    }
 \cxset{
    chapter number font-size        = huge,
    chapter number font-weight      = bfseries,
    chapter number font-family      = sffamily,
    chapter number font-shape       = upshape,
    chapter number align            = Centering,
    }
\cxset{%
     chapter title font-size        = Huge,
     chapter title font-weight      = bold,
     chapter title font-family      = calligra,
     chapter title font-shape       = upshape,
     chapter title color            = black,
     }

\sethyperref
\EnableCrossrefs
\CodelineIndex
\RecordChanges
       
\begin{document}
  \let\bold\bfseries
 \coverpage{desert}{Book Design }{Camel Press}{RUNNING HEADS}{DESIGN} 
 \secondpage
 \newpage
 \tableofcontents
 \mainmatter
  \thispagestyle{headings}
% %% DESIGNING HEADERS AND FOOTERS  **************************


\chapter{Running Titles and Paging}

Early printed books had no running title or paging figures. The first attempt to satisfy this need of the reader was to repeat the number of the chapter at the head of each page.\footnote{De Vinne, pg 142.}  As books and styles evolved, if the words of the running title or chapter began appearing together with the page number. Practical considerations regarding the wearing of plates, school-books and all works that were printed frequently had running titles in capitals of light-faced antique. 

\begin{figure}[htp]
\includegraphics[width=\textwidth]{./images/beauty-and-art-spread.jpg}
\end{figure}

Almost every type of design has been adopted by typographers and book designers; sometimes the text is centered and in other cases it is set flush up to the inner or outer margin of the facing pages. The book chapter and the section of the book is sometimes specified in the running title, the chapter name on the left and the section on the right. When the running title consists of the name of the book, it was sometimes divided so that one half only of this name would appear on one page and the other half on the facing page. De Vrinde was highly critical of such practices and remarked `Nor is this a commendable fashion, for a line of many words can seldom be evenly divided; if it is not so divided, one heading will be longer than the other.’  Some modern books that follow in a similar fashion would place the chapter label and number at the left and the chapter title on the right. 

I am unsure if repeating the name of the book in its running title has any benefits to the reader, especially if the name of the book is well known to the reader. This title is most useful when it explains or to some extend defines the matter on the page, and this explanation should refer not to the first but to the last paragraph on that page.  Many authors prefer to not have sections in chapters and in such cases running the book name in the header rather than having left and right headers that just repeat the chapter name is preferable. An example of this is Tufte’s \textit{Beautiful Evidence}.  Tufte’s books do not have any footer material.  Many specialist scientific books have multi-authors, sometimes the running head includes the authors name (See figure from ). This particular illustration also shows the use of rules. Traditionally the rules were applied to protect the top of the block from mechanical wear during printing. 

\begin{figure}[hb]
\includegraphics[width=\textwidth]{./images/headers/header-humidification-odd.jpg}
\includegraphics[width=\textwidth]{./images/headers/header-humidification-even.jpg}
\end{figure}

As a rule,  paging with arabic figures begins with the text of the book. The matter before the text (as the title, preface, introduction, etc., which are printed last of all) is paged with roman lower-case numerals. Appendices, indices and all additions to the text take arabi figures for paging, but publisher’s advertisements at the end of the book should receive their special paging in a figure of a different face. Maps, portraits, and illustrations made on separate pages never receive printed paging, although they may be reckoned as pages in the table of contents or the index. 
\begin{figure}[htb]
\includegraphics[width=\textwidth]{./images/headers/architect.jpg}
\caption{The headers here, have a background shading.}
\end{figure}

\begin{figure}[htb]
\includegraphics[width=\textwidth]{logic.jpg}
\caption{The headers shown here include small dotted rules, running to the outer page end. This type of header can be build by adding properties and inheriting the properties of other headers.}
\end{figure}

\begin{figure}[htb]
\hskip-.1\textwidth\includegraphics[width=1.2\textwidth]{./images/headers/tulip-01.jpg}

\vspace*{1cm}

\hskip-.1\textwidth\hbox to 0pt{\includegraphics[width=1.2\textwidth]{./images/headers/tulip-02.jpg}}
\caption{The headers here, have a background shading.}
\end{figure}

\begin{figure}[htp]
\includegraphics[width=1\textwidth]{./images/headers/small-flash-01.jpg}

\vspace*{1cm}
\includegraphics[width=1\textwidth]{./images/headers/small-flash-02.jpg}

\caption{The headers here, have a background shading.}
\end{figure}


\begin{figure}[htp]
\includegraphics[width=1\textwidth]{./images/headers/power-and-politics-01}

\vspace*{1cm}
\includegraphics[width=1\textwidth]{./images/headers/power-and-politics-02}

\caption{The headers here, have a background shading.}
\end{figure}

\begin{figure}[htp]
\includegraphics[width=1\textwidth]{./images/headers/economic-warfare-01}

\vspace*{1cm}
\includegraphics[width=1\textwidth]{./images/headers/economic-warfare-02}

\caption{The headers here, have a background shading.}
\end{figure}


\section{The Requirements}

The brief discussion above and the examples from various publications can help in definind the final requirements of what we are about to program. The header or the footer for that matter as they are very similar needs to communicate with the page that is currently being processed to obtain the page number and any other marks that need to go into the running head.

\begin{tabular}{>{\raggedright}p{5cm}l}
Access to the page number & \\
Build up string from sections, chapters, titles or subtitles &\\
Distingusih between left and right numbers &\\
Add user data  &\\
Provide an intuitive user interface&\\
\end{tabular}

A more modern approach would be to offer a small templating language to deal with the headers and footers. This is for example, now common in web applications where variables are sent by the server to the web page being build and transformed in templates.

Another approach is to use a graphical language, such as metapost.

Since we have to deal with odd and even pages and a header and or a footer, the minimum variables needed to hold this information is four. 

A graphicablock can also happily contain the necessary information.

The algorithm is described below:

\begin{enumerate}
\item Set the variable headerleft and headerright to indicate one page or two page printing.
\item Define text block templates as macros to set the typesetting to a named style. Each header style
         will have its own name. Standardize parameters to enable easy redefinition of commands. As a 
         final fully flexible approach the key header = custom will provide full capabilities for any user
         defined design.
\item Distinguish how headers and footers will be typeset on title pages, chapter openings, bibliographies, 
         automatically generated pages, such as float pages etc.
\item Hook into LaTeX’s output routine to obtain information about the top and bottom inserts and other marks.         
\item Inherit properties, such as language and directionality.
\item Provide less intrusive ways to define different styles by the user.
\item define block commands to mark start of different headings for example |\mainmatter|. This will define
         the start of the main text of the publication and issue a command to process the pages that follow.
\end{enumerate}

A special type of header is something that will be repeated on every page, say a watermark of some sorts. These are dealt as backgrounds.
 
\section{Traditional LaTeX page style commands}
  
One of the first tasks of any \LaTeXe\ class is to redefine the headers and footers. The format of the running headers or footers in \LaTeX\ terminology is called the \textit{page style}. Each different format is given names like \textit{empty} or \textit{plain} to make it easier to select and remember. 

\begin{figure}[hbt]
\includegraphics[width=\textwidth]{./images/headers/Running-heads-lace.png}
\caption{This last example shows what kind of atmosphere you can create with running heads. Here a bit of lace texture has been softened and graduated, creating a kind of gentle, suggestive frame around these pages. I’ve also used line drawings, logos, and other graphic elements to dress up running heads like these. From the \protect\href{bookk  }{bookdesigner.com}}
\end{figure}


The LaTeX kernel\footnote{In File J file{ltpage.dtx}, page 311.} defines two commands for selecting the running heads:

\begin{lstlisting}
\pagestyle{<style>} : sets the page style of the current and succeeding pages to style
\thispagestyle{<style>} : sets the page style of the current page only to style.
\end{lstlisting}

\section{Traditional LaTeX page style definition}

To define a page style \textit{style}, you must define the \lstinline{\ps@style} to set the page parameters.

\subsection{How a page style makes running heads and feet}
The \lstinline{\ps@}. . . command defines the macros \lstinline{\@oddhead}, \lstinline{\@oddfoot}, \lstinline{\@evenhead},
and \lstinline{\@evenfoot} to define the running heads and feet. (See output routine.) As some headings contain information such as the chapter name or section number these
headings are based on the sectioning commands, which define them. The page style defines the commands




\verb!\chaptermark,\sectionmark!, etc., where

\verb+\chaptermark{<text>}+ is called by \verb+\chapter+ to set a mark. The  ...mark commands and the ...head
macros are defined with the help of the following macros.
%(All the \ ...mark commands should be initialized to no-ops.)



\subsection{marking conventions}

LaTeX produces two kinds of marks a `left' and a `right' mark using the following commands.

markboth

markright



\section{The low level page style interface}

The basic mechanics of defining page styles is provided in the \LaTeXe\ kernel and it  involves defining or redefining four macros:

\begin{marglist}
\item [\cs{oddhead}] For two-sided printing, it generates the header for the odd-numbered
pages; otherwise, it generates the header for all pages.

\item [\cs{oddfoot}] For two-sided printing, it generates the footer for the odd-numbered pages; otherwise, it generates the footer for all pages.

\item [\cs{evenhead}] For two-sided printing, it generates the header of the even-numbered
pages; it is ignored in one-sided printing.

\item [\cs{evenfoot}] For two-sided printing, it generates the footer of the even-numbered
pages; it is ignored in one-sided printing.

\end{marglist}
A named page style, involves the redefinition of these commands stored in a macro \cs{ps@<style>}.
The \cs{pagestyle}\marg{plain} is defined as:



%\begin{tcolorbox}
%\begin{lstlisting}
%\newcommand\ps@plain{%
%  \renewcommand\@oddhead{}%
%  \let\@evenhead\@oddhead
%  \renewcommand\@evenfoot{%
%  {\hfil\normalfont\textrm{\thepage}\hfil}}%
%  \let\@oddfoot\@evenfoot
%}
%\end{lstlisting}
%\end{tcolorbox}

Since the \textit{plain} style treats both the odd and even pages the same way, the \cs{@evenfoot} and \cs{@evenhead} are let to the \cs{@oddhead} and \cs{@oddfoot} commands. The style only prints a page number at the center of the footer.


\subsection{A longer example}

\index{watermark}\index{water mark!sample page style}
\thispagestyle{samplepage}
Consider the case, where we need to print on a page the words \textsc{sample page}, as you might have noticed in some places of this document and at the margin of this page. Sometimes this type of mark is called a \textit{watermark.}

We will call this type of page style \textit{samplepage} and we will activate it on a particular page by typing \cs{thispagestyle}\marg{samplepage}.




%\begin{tcolorbox}
%\begin{lstlisting}
%%% Some special styles
%\IfFileExists{rotating.sty}{\RequirePackage{rotating}}{}
%
%\def\even@samplepage{%
% \begin{picture}(0,0)
%   \put(\Xeven,\Yeven){\turnbox{90}{\Huge \textcolor{\watermark@textcolor}{\watermark@text}}}
%\end{picture}
%}
%
%\def\odd@samplepage{%
% \begin{picture}(0,0)
%   \put(\Xodd,\Yodd){\turnbox{90}{\Huge \textcolor{\watermark@textcolor}{\watermark@text}}}
% \end{picture}
%}
%
%\def\watermarktext#1{\gdef\watermark@text{\fontfamily{phv}\selectfont#1}}
%\def\watermarktextcolor#1{\gdef\watermark@textcolor{#1}}
%\watermarktext{SAMPLE PAGE}
%\watermarktextcolor{purple}
%
%\def\ps@samplepage{\let\@mkboth\@gobbletwo
% \let\@oddhead\odd@samplepage\def\@oddfoot{\reset@font\hfil\thepage}
% \let\@evenhead\even@samplepage\def\@evenfoot{\reset@font\thepage\hfil}}
%
%\def\Xodd{500}
%\def\Xeven{-70}\def\Yeven{-810}
%\def\Yeven{-\expandafter\strip@pt\textheight}
%\let\Yodd\Yeven
%\end{lstlisting}
%\end{tcolorbox}

If you study the code in the example, you will notice that we are using \LaTeXe's \env{picture} environment to
place the text exactly where we need it. This is one way of absolutely positioning text on a page, another way is to use |pgf|’s absolute positioning methods.




\subsection{The key value interface}

The key value interface provides a number of mechanisms to tap into the page styles, enabling consistency in the user interface.

\medskip

\keyval{header style}{\marg{text}}{Triggers a page style for one page only.} The following values can be used.

\begin{marglist}
\item [empty] Standard class empty headers.
\item [plain] Standard class plain headers.
\item [headings] Standard class headings.
\item [fancy] If you use the fancyhdr package any fancy header style.
\item [sample page] Prints sample at the edge of the paper.
\item [preprint] Prints preprint at the edge of the paper.
\item [watermark] Prints a watermark at predefined places.
\end{marglist}

\keyval{watermark}{\marg{true|false}}{Prints a watermark on all pages, defaults to false.}
\keyval{watermark text}{\marg{text}}{The watermark text.}
\keyval{watermark text left}{\marg{text}}{The watermark text on left pages.}
\keyval{watermark text right}{\marg{text}}{The watermark text on right pages.}
\keyval{watermark angle}{\marg{number}}{The rotation angle of the water mark}




%\cxset{ watermark text/.store in=\watermark@text,
%           watermark text color/.store in=\watermark@textcolor,
%           watermark font-size/.store in=\watermarkfontsize@cx,
%           watermark odd x/.store in=\watermarkoddx@cx,
%           watermark even x/.store in=\watermarkevenx@cx,
%           watermark even y/.store in=\watermarkeveny@cx}
%
%\cxset{watermark text= PRE-PRINT,
%          watermark text color=theblue,
%          watermark font-size=\huge,
%          watermark odd x=470,
%          watermark even y=700,
%          watermark even x=60}
%
%\def\Xodd{\watermarkoddx@cx}
%\def\Xeven{-\watermarkevenx@cx}
%\def\Yeven{-\watermarkeveny@cx}
%%\def\Yeven{-\expandafter\strip@pt\textheight}
%\let\Yodd\Yeven
%
%\def\even@samplepage{%
% \begin{picture}(0,0)
%   \put(\Xeven,\Yeven){\turnbox{60}{\watermarkfontsize@cx \textcolor{\watermark@textcolor}{\watermark@text}}}
%\end{picture}
%}
%
%\def\odd@samplepage{%
% \begin{picture}(0,0)
%   \put(\Xodd,\Yodd){\turnbox{90}{\watermarkfontsize@cx\textcolor{\watermark@textcolor}{\watermark@text}}}
% \end{picture}
%}






\subsection{Using the headings as hooks}

Since the headings are added to the page during processing of the output routine, they are sometimes used
to insert material on the page at places other than the head, through the use of a zero width box. For example we
can use this approach to add a watermark on a page. Other approaches to position material at absolute positions
on a page, is to hook at \emph{shipout}. Some packages such as TikZ can also be used through the |remember picture, overlay |  key settings. 

The |phd| package has a predefined style, named samplepage that can be used to typeset some text at the outer margin of a page. The text is configurable and you can set it for example to typeset “PRE-PRINT” rather than the “SAMPLE PAGE” string. 

\begin{tcolorbox}
\begin{lstlisting}
\cxset{
     watermark text= PRE-PRINT,
     watermark text color=theblue,
     watermark font-size=\huge
}
\end{lstlisting}
\end{tcolorbox}

\makeatletter
\cxset{watermark text/.code =\watermarktext }
\makeatother

\watermarktext{PRE-PRINT}
   
\pagestyle{samplepage}


\section{Adding marks}

Most books will have headers that include marks such as the chapter name and number and or other combinations together with section numbers.

The standard book class include two styles one called \textit{headings} and another called \textit{myheadings} that style such headers.




\subsection{Key value interface}
\makeatletter
\cxset{
   chaptermark name color/.store in=\chaptermarknamecolor@cx,
   sectionmark name color/.store in=\sectionmarkcolor@cx,
   sectionmark title font/.store in=\sectionmarktitlefont@cx,
   section title color/.store in=\sectiontitlecolor@cx,
}

\makeatother

\cxset{chaptermark name color=thered,
          sectionmark name color=thered}





\begin{tcolorbox}
\begin{lstlisting}
%% STYLE 57 QUANTUM FRONTIER
\cxset{headings style57/.style={
          headings titlestyle,
% Chaptermarks
          chaptermark name={\bfseries EVOLUTION OF THE INSECTS},
% Leftmarks
          leftmark before=\thepage\quad, %even pages
          leftmark after=\hfill\hfill,
% Right marks influenced by chapter name?
          rightmark before=\hfill\hfill, %odd pages
          rightmark after=\thepage,
% Section marks
          sectionmark name custom=\chaptertitle@cx,
          sectionmark after title=\quad,
%  rules we remove or inherit
          header top rule=false,
          header bottom rule=false,
          header offset even=0pt,
          header offset odd=0pt,
          }}
\end{lstlisting}
\end{tcolorbox}


%\if@twoside
%  \def\ps@headings{%
%      \let\@oddfoot\@empty
%      \def\@oddfoot{\rule{\textwidth}{0.4pt}}
%      \let\@evenfoot\@empty
%      \def\@evenhead{\parbox{\textwidth}{%
%                                   \leavevmode
%                                   \if@headertoprule\rule{\textwidth}{0.4pt}%
%                                       \vskip2pt plus1pt minus1pt\fi
%%typesetter
%                                     \hskip\headeroffseteven@cx\hbox to \textwidth{%
%                                           \leftmarkbefore@cx
%                                           \leftmark
%                                           \leftmarkafter@cx
%                                     }%
%                                     \if@headerbottomrule\vskip-7pt plus1pt minus1pt
%                                    \rule{\textwidth}{0.4pt}\fi%
%          }% end parbox
%       }%
%%% Defines the odd head
%      \def\@oddhead{
%         \parbox{\textwidth}{%
%                                   \leavevmode
%                                   \if@headertoprule\rule{\textwidth}{0.4pt}%
%                                       \vskip2pt plus1pt minus1pt\fi
%%typesetter
%                                     \hskip\headeroffsetodd@cx\hbox to \textwidth{%
%                                           \rightmarkbefore@cx
%                                           \rightmark
%                                           \rightmarkafter@cx
%                                     }%
%                                     \if@headerbottomrule\vskip-7pt plus1pt minus1pt
%                                    \rule{\textwidth}{0.4pt}\fi%
%          }% end parbox
%      }%
%      \let\@mkboth\markboth
% % chaptermark called by chapter and also by table of contents etc. This is essentially a
%%  leftmark
%\def\chaptermark##1{%
%     \gdef\chaptertitle@cx{##1}%
%      \markboth {%
%       \ifnum \c@secnumdepth >\m@ne
%          \if@mainmatter%
%              \color{\chaptermarknamecolor@cx}%
%              \MakeUppercase{\chaptermarkname@cx\ }%
%              \chaptermarknumber%
%              \chaptermarkafternumber@cx%
%          \fi
%        \fi
%        \color{\chaptermarktitlecolor@cx}%
%       % \hfill%
%        \MakeUppercase{\chaptermarktitlebefore@cx{##1}}}{}%
%}%end chaptermark
%% section
%  \def\sectionmark##1{%
%      \markright {%
%        \ifnum \c@secnumdepth >\z@
%           {\bfseries\textcolor{\sectionmarkcolor@cx}{\sectionmarkname@cx\sectionmarknumber@cx\sectionmarkafternumber@cx}%
%        } %
%  \fi
%         \color{\sectionmarktitlecolor@cx}\MakeUppercase{\normalfont\sffamily \sectionmarkbeforetitle@cx{##1}\sectionmarkaftertitle@cx}}}}%
%\else
%  \def\ps@headings{%
%    \let\@oddfoot\@empty
%    \def\@oddhead{{\slshape\rightmark}\hfil\thepage}%
%    \let\@mkboth\markboth
%    \def\chaptermark##1{%
%      \markright {%
%        \ifnum \c@secnumdepth >\m@ne
%          \if@mainmatter
%            \@chapapp\ \thechapter... \ %
%          \fi
%        \fi
%        ##1}}}
%\fi
%\def\ps@myheadings{%
%    \let\@oddfoot\@empty\let\@evenfoot\@empty
%    \def\@evenhead{\thepage\hfil\slshape\leftmark}%
%    \def\@oddhead{{\slshape\rightmark}\hfil\thepage}%
%    \let\@mkboth\@gobbletwo
%    \let\chaptermark\@gobble
%    \let\sectionmark\@gobble
% }

Note that the \cs{markboth} command takes two arguments the left mark and the right mark. It works reasonably well.



%\cxset{headings boxedpagenumber}
%\cxset{headings style58}
%\pagestyle{headings}

  % \catcode `\| =12
%  \DeleteShortVerb
  \makeatletter
  \@debugtrue
  \makeatother
  \DocInput{phd-runningheads.dtx}%
\end{document}
%</driver>
% \fi
%
% \CheckSum{553}
% \CharacterTable
%  {Upper-case    \A\B\C\D\E\F\G\H\I\J\K\L\M\N\O\P\Q\R\S\T\U\V\W\X\Y\Z
%   Lower-case    \a\b\c\d\e\f\g\h\i\j\k\l\m\n\o\p\q\r\s\t\u\v\w\x\y\z
%   Digits        \0\1\2\3\4\5\6\7\8\9
%   Exclamation   \!     Double quote  \"     Hash (number) \#
%   Dollar        \$     Percent       \%     Ampersand     \&
%   Acute accent  \'     Left paren    \(     Right paren   \)
%   Asterisk      \*     Plus          \+     Comma         \,
%   Minus         \-     Point         \.     Solidus       \/
%   Colon         \:     Semicolon     \;     Less than     \<
%   Equals        \=     Greater than  \>     Question mark \?
%   Commercial at \@     Left bracket  \[     Backslash     \\
%   Right bracket \]     Circumflex    \^     Underscore    \_
%   Grave accent  \`     Left brace    \{     Vertical bar  \|
%   Right brace   \}     Tilde         \~}
%
%
% \changes{1.0}{2011/05/03}{Converted to DTX file}
%
% \DoNotIndex{\newcommand,\newenvironment}
%
% \GetFileInfo{template.dtx}
% \providecommand*{\url}{\texttt}
%  \def\fileversion{v1.0}          
%  \def\filedate{2012/03/06}
% \title{The \textsf{\jobname} package.
% \author{Dr. Yiannis Lazarides}
% \thanks{This
%        file (\texttt{\jobname.dtx}) has version number 
%        \fileversion, last revised
%        \filedate.}
% }
% 
% \date{\filedate}
%
%% \newpage
% \maketitle
% 
% \begin{summary}
%   This package forms part of the \pkgname{phd} bundle and can be used to manage running heads
%   with a key-value interface. It comes with a number of predefined styles, that cover most of the
%  common use cases for books and journals. It is still in alpha stage. However it will not hopefully
%  break anything if used and is compatible with the fancyhdr package.
% \end{summary}
%
% 


% 
%
%
% ^^A\StopEventually{}
% 
% \chapter{Package Objectives and Implementation}
%  \pagestyle{headings}
% \begin{enumerate}
%  \item To simplify the user interface for specifying  headers and footers in \latexe.
% \item To make available a number of differently styled page headers to lessen
% the need for book designers to have to create new ones.
% \item To provide a key value interface to blend with the other keys provided by
%  the \pkgname{phd}  package.
% \end{enumerate}
%
% \section{Simplifying the terminology}
%
%  \begin{description}
%  \item [header] The top part of the page where a running head can be inserted.
%  \item [footer]  The bottom part of the page, where a running footer can be inserted.
%  \item [header odd] The header on an odd page.
%  \item [header even] The header on an even page.
%  \item [footer odd] The footer on an odd page.
%  \item [footer even] The footer on an even page.
% \end{description}
%
%  We classify users according to the \LaTeX3 terminology as a) programmers b) template designers
%  and c) authors.
%
% \subsection{Author}
%  We assume that the author has an exising template which she is using but might want to do
%  some minor modifications, for example use an italic shape for the font of the mark, but an 
%  upright font for the page numbers. 
%
% {\obeylines 
%~~ |\cxset|
%~~~~~|{|
%~~~~~~~~\textit{header even}~~|format          = apa,|
%~~~~~~~~\textit{header even mark} |font-size   = Large,|
%~~~~~~~~\textit{header even mark} |font-family = serif,|
%~~~~~~~~\textit{header even mark} |font-weight = normal,|
%~~~~~~~~\textit{header even mark} |font-shape  = italic,|  
%~~~~~|}|
%}  
%
% We follow the idea of representing the basic elements of documents
% as elements, each one having a parent in order to specify
% the element we need to style as accurate as possible. One can think of
% this approach being congruent with objects in other languages.
% As a matter fact nothing stops us from defining a key value
% interface as shown below.
%
% {\obeylines 
%~~ |\cxset|
%~~~~~|{| 
%~~~~~~~~\textit{header.even.mark.font.size}   = |Large,|
%~~~~~~~~\textit{header.even.mark.font.family} = |serif,|
%~~~~~|}|
%}  
%
% This would pehaps make it easier for the tempalte designer, but I have rejected
% the idea as my aim is to make it easy for the author, who can search the template
% and just enter a couple of new proerty values.
%
% \subsection{Template designer}
% The template designer in the example above would have selected the format style
% from a number of predefined formats (templates) or would have created a style
% called \textit{apa} from an existing template and modified it using declarative
% key style.
%
% \subsection{The programmer}
%
% The programmer in the example above could have created the basic format
% \textit{apa} by using both declarative as well as defining or using existing
% macros. To the programmer we offer an extension mechanism, where the contents
% of a |ps@| command are defined. For example teh programmer can define a new
% style using \tikzname, but without having to worry about defining full |ps@|
% and their interface.
%
% \section{Canned Templates}
%
% Luckily the available permutations for headers and footers are less than the
% variations one sees in headings. Our approah here is to create |ps@| like
% commands to cover most use cases. We are interested in defining these structurally
% and leave the font and colour styling to the the key value interface.
%
% \begin{enumerate}
%  \item Redefine \textit{plain,headings,myheadings} to add appropriate hooks
%        for styling and rules.
%  \item Provide in addition 5-10 other styles to cover all the needs of the
%        hundred or so designs we provide with the \pkgname{phd} package.
% \end{enumerate}
%
% \subsection{Mark composition}
%
% A mark can either be a \docAuxCommand{leftmark} or a \docAuxCommand{rightmark}. They
% are normally composed of the title of a heading and the page number styled in 
% numerous ways. We need to make allowances as to how to handle different type 
% of mark combinations. Sometimes they include dates, revision numbers, author name, title or short title, logos and even images or links. The latter offers a very user 
% friendly way to navigate through a book's sections.
%
% \section{Introduction}
%  This manual is typeset according to the conventions of the
% \LaTeX{} \textsc{docstrip} utility which enables the automatic
% extraction of the \LaTeX{} macro source files~\cite{GOOSSENS94}.
%    \begin{macrocode}
%<*RH>

\NeedsTeXFormat{LaTeX2e}
\ProvidesPackage{phd-runningheads}%
  [2015/13/06 v1.0 Running heads styling]%
%    \end{macrocode}
%
% \section{Marks and pagestyles}
% 
% We first define a number of commonly used headings and then we add some keys
% to make them more flexible. The package happily coexists with \pkgname{fancyhdr}
% so you can define your own if you want.
% 
%    \begin{macrocode}
\global\let\tikz@ensure@dollar@catcode=\relax
%    \end{macrocode}
%
% We define a handler to handle the choice keys for what goes in the center left or right
% of a header or footer.
% 
%    \begin{macrocode}
\ExplSyntaxOn
    \pgfkeys{/handlers/.mark/.code = 
    \pgfkeysalso
      {
        \pgfkeyscurrentpath/.code=
           \str_case_x:nnTF {##1}  
             {
               { none           } { \tl_gset:cn  {#1} { \empty}          } 
               { leftmark       } { \tl_gset:cn  {#1} { \leftmark     }  } 
               { rightmark      } { \tl_gset:cn  {#1} { \rightmark    }  } 
               { page           } { \tl_gset:cn  {#1} { \thepage      }  } 
               { today          } { \tl_gset:cn  {#1} { \today        }  } 
               { jobname        } { \tl_gset:cn  {#1} { \jobname      }  }
               { author         } { \tl_gset:cn  {#1} { \docauthoright }  }
             }
             {                         }
             {   \tl_gset:cn {#1} { } \tl_put_right:cn {#1} {##1}   }
     }   
   }   
% 
\dim_new:N \phd_headerwidth_dim
\dim_new:N \phd_footerwidth_dim
\ExplSyntaxOff
%
%
\newif\ifheadertoprule
\newif\ifheaderbottomrule
%    \end{macrocode}
%  As we are having three basic elements for every heading, this results in
%  12 sets of properties, i.e. if we target font weight, we need to allow for
%  odd pages, even pages and top and bottom. But we have left, center and 
%  right, hence  $4 \times 3 = 12$. An alternative way would be to target
%  the marks, however we are allowing for arbitrary text, fields such as jobname
%  and dates to be inserted, to cater for all the marks would be too tedious.
%
%  We also want to keep to the primary objectives of the phd package to provide
%  the user a set of keys that do not require macro mark-up. It ended up as tedious
%  and verbose, but by the time a full template is provided the user will see styles
%  rather than keys.
%  
%    \begin{macrocode}    
\ExplSyntaxOn
\cs_new:Npn \create_pagestyle_keys #1 
  {
    \cxset
      {
        #1~even.header.left/.mark   = #1_even_header_left,
        #1~even~header~center/.mark = #1_even_header_center,
        #1~even~header~right/.mark  = #1_even_header_right,
        #1~even~footer~left/.mark   = #1_even_footer_left,
        #1~even~footer~center/.mark = #1_even_footer_center,
        #1~even~footer~right/.mark  = #1_even_footer_right,
        #1~even~header~background~color/.code = 
          \cs_set:cpn {#1_even_header_background_color}{##1},
        #1~even~footer~background~color/.code = 
          \cs_set:cpn {#1_even_footer_background_color}{##1},
%          
        #1~even~header~left~font-family/.fontfamily   = #1_even_header_left_fontfamily, 
        #1~even~header~left~font-size/.fontsize       = #1_even_header_left_fontsize, 
        #1~even~header~left~font-weight/.fontweight   = #1_even_header_left_fontweight,  
        #1~even~header~left~font-shape/.fontstyle     = #1_even_header_left_fontshape,
%         
        #1~even~header~center~font-family/.fontfamily = #1_even_header_center_fontfamily, 
        #1~even~header~center~font-size/.fontsize     = #1_even_header_center_fontsize, 
        #1~even~header~center~font-weight/.fontweight = #1_even_header_center_fontweight,  
        #1~even~header~center~font-shape/.fontstyle   = #1_even_header_center_fontshape,
%          
        #1~even~header~right~font-family/.fontfamily  = #1_even_header_right_fontfamily, 
        #1~even~header~right~font-size/.fontsize      = #1_even_header_right_fontsize, 
        #1~even~header~right~font-weight/.fontweight  = #1_even_header_right_fontweight,  
        #1~even~header~right~font-shape/.fontstyle    = #1_even_header_right_fontshape,         
%        
        #1~odd~header~left/.mark    = #1_odd_header_left,
        #1~odd~header~center/.mark  = #1_odd_header_center,
        #1~odd~header~right/.mark   = #1_odd_header_right,
        #1~odd~footer~left/.mark    = #1_odd_footer_left,
        #1~odd~footer~center/.mark  = #1_odd_footer_center,
        #1~odd~footer~right/.mark   = #1_odd_footer_right,        
        #1~odd~header~background~color/.code = \cs_set:cpn {#1_odd_header_background_color}{##1},
        #1~odd~footer~background~color/.code = \cs_set:cpn {#1_odd_footer_background_color}{##1},     
%          
        #1~odd~header~left~font-family/.fontfamily   = #1_odd_header_left_fontfamily, 
        #1~odd~header~left~font-size/.fontsize       = #1_odd_header_left_fontsize, 
        #1~odd~header~left~font-weight/.fontweight   = #1_odd_header_left_fontweight,  
        #1~odd~header~left~font-shape/.fontstyle     = #1_odd_header_left_fontshape,
%         
        #1~odd~header~center~font-family/.fontfamily = #1_odd_header_center_fontfamily, 
        #1~odd~header~center~font-size/.fontsize     = #1_odd_header_center_fontsize, 
        #1~odd~header~center~font-weight/.fontweight = #1_odd_header_center_fontweight,  
        #1~odd~header~center~font-shape/.fontstyle   = #1_odd_header_center_fontshape,
%          
        #1~odd~header~right~font-family/.fontfamily  = #1_odd_header_right_fontfamily, 
        #1~odd~header~right~font-size/.fontsize      = #1_odd_header_right_fontsize, 
        #1~odd~header~right~font-weight/.fontweight  = #1_odd_header_right_fontweight,  
        #1~odd~header~right~font-shape/.fontstyle    = #1_odd_header_right_fontshape,               
  }  
}

\create_pagestyle_keys {plain}   

\cxset
  {
    plain~demo/.style={
    plain~even.header.left             = {plain~even~header~L},
    plain~even~header~center           = {plain~even~header~C},
    plain~even~header~right            = {plain~even~header~R}, 
    plain~even~footer~left             = {plain~even~footer~L},
    plain~even~footer~center           = {plain~even~footer~C},
    plain~even~footer~right            = {plain~even~footer~R},
    plain~odd~header~left              = {plain~odd~header~L},
    plain~odd~header~center            = {plain~odd~header~C},
    plain~odd~header~right             = {plain~odd~header~R}, 
    plain~odd~footer~left              = {plain~odd~footer~L},
    plain~odd~footer~center            = {plain~odd~footer~C},
    plain~odd~footer~right             = {plain~odd~footer~R},  
% bg colors    
    plain~odd~header~background~color  = spot!15,
    plain~odd~footer~background~color  = spot!15,
    plain~even~header~background~color = magenta!15,
    plain~even~footer~background~color = magenta!15,
   } 
  }  
  
\cxset
  { plain~colored/.style={
    plain~even.header.left             = page,
    plain~even~header~center           = none,
    plain~even~header~right            = rightmark, 
    plain~even~footer~left             = page,
    plain~even~footer~center           = none,
    plain~even~footer~right            = ,
    plain~odd~header~left              = leftmark,
    plain~odd~header~center            = none,
    plain~odd~header~right             = page, 
    plain~odd~footer~left              = today,
    plain~odd~footer~center            = none,
    plain~odd~footer~right             = none,  
% bg colors    
    plain~odd~header~background~color  = spot!15,
    plain~odd~footer~background~color  = spot!15,
    plain~even~header~background~color = magenta!15,
    plain~even~footer~background~color = magenta!15,
% fonts
    plain~even~header~left~font-family = sffamily, 
    plain~even~header~left~font-weight = bfseries, 
    plain~even~header~left~font-size   = small, 
    plain~even~header~left~font-shape  = italic,  
%
    plain~even~header~center~font-family = sffamily, 
    plain~even~header~center~font-weight = bfseries, 
    plain~even~header~center~font-size   = small, 
    plain~even~header~center~font-shape  = italic,   
%
    plain~even~header~right~font-family = sffamily, 
    plain~even~header~right~font-weight = bfseries, 
    plain~even~header~right~font-size   = small, 
    plain~even~header~right~font-shape  = italic, 
%
% fonts
    plain~odd~header~left~font-family   = sffamily, 
    plain~odd~header~left~font-weight   = bfseries, 
    plain~odd~header~left~font-size     = scriptsize, 
    plain~odd~header~left~font-shape    = italic,  
%
    plain~odd~header~center~font-family = sffamily, 
    plain~odd~header~center~font-weight = bfseries, 
    plain~odd~header~center~font-size   = small, 
    plain~odd~header~center~font-shape  = italic,   
%
    plain~odd~header~right~font-family = sffamily, 
    plain~odd~header~right~font-weight = bfseries, 
    plain~odd~header~right~font-size   = small, 
    plain~odd~header~right~font-shape  = italic,                 
   } 
  }   
   
\cxset{plain~colored} 
\cxset{plain~even~footer~background~color = green!90!black} 
%  \g@addto@macro\headerright{Chapter~\thechapter~Section~\thesection}
%  \g@addto@macro\headerright{Page ~\thepage}
\ExplSyntaxOff   


%    \begin{macrocode}
\cxset{pagestyle/.code=\pagestyle{#1}}
%cxset{pagestyle=ruled-01}
%

          
\newif\ifphd@multisty \phd@multistyfalse
\newcommand\copyrightline[1]{%
  \def\@copyrightline{#1}}

%\edef\@copyrightline{\copyright Y Lazarides\relax}
\edef\@copyrightline{}
\newcommand\c@pyrightline[1]{%
  \gdef\@c@pyrightline{#1}}

\gdef\@c@pyrightline{%
  \vbox to 5.5\p@{\noindent
  \parbox[t]{\textwidth}{\normalfont\footnotesize\baselineskip 9\p@
  \@copyrightline
  }%
  \vss}%
}
%    \end{macrocode}
%    
% \section{Formatters}
%  We wish to leave the original plain definition untouched. The command
%  clears the headers and set the footers to just print the page number
%  at the margins. We allow for rules.
% 
% \pagestyle{plain}
%    \begin{macrocode}
\ExplSyntaxOn
\let\ps@plainltx\ps@plain
\dim_new:N  \even_offset_l 
\dim_set:Nn \even_offset_l {0pt}
\dim_new:N  \even_offset_r
\dim_set:Nn \even_offset_r {0pt}
\dim_gzero_new:N \header_width_dim
\dim_set_eq:NN \header_width_dim \textwidth

\tl_new:N \leftglue
\tl_new:N \rightglue 
%
%    \end{macrocode}
% \begin{docCommand} {format_running_head_box} { \marg {prefix} \marg{} \marg{} \marg{} }
%   The general formatter for heads.
%
%   |#1| outer box style\\
%   |#2| page box style\\
%   |#3| left glue\\
%   |#4| right glue
%
% \end{docCommand}
%    \begin{macrocode}
\newtcbox{\headerbox}[1]%
  {
     nobeforeafter,
     size=minimal,
     width=\header_width_dim, 
     colback= \cs:w #1background_color \cs_end:, 
     colframe=white,
     box~align=base,
     arc=2mm,boxsep=3mm,
     rounded~corners=all,
     drop~shadow=black,
  }
  
\newtcbox{\headerboxleft}[1]%
  {
     nobeforeafter,
     size=minimal,
     width=\header_width_dim, 
     colback= \cs:w #1background_color \cs_end:, 
     colframe=white,
     box~align=base,
%     arc=2mm,boxsep=3mm,
%     rounded~corners=all,
%     drop~shadow=black,
  } 
\newtcbox{\headerboxcenter}[1]%
  {
     nobeforeafter,
     size=minimal,
     width=\header_width_dim, 
     colback= \cs:w #1background_color \cs_end:, 
     colframe=white,
     box~align=base,
%     arc=2mm,boxsep=3mm,
%     rounded~corners=all,
%     drop~shadow=black,
  }  
\newtcbox{\headerboxright}[1]%
  {
     nobeforeafter,
     size=minimal,
     width=\header_width_dim, 
     colback= \cs:w #1background_color \cs_end:, 
     colframe=black,
     box~align=base,
     %arc=2mm,
     %rounded~corners=all,
     %drop~shadow=black,
  }    
\cs_new:Npn \format_running_head_box #1 #2 #3 #4
  { 
    \dim_add:Nn \header_width_dim {\even_offset_l + \even_offset_l}
    \skip_horizontal:n {-\dim_use:N \even_offset_l}
      \headerbox{#1}
       {\hss\hbox_to_wd:nn \header_width_dim 
         { 
           \headerboxleft {#1}
             { 
               \cs:w #1left_fontfamily \cs_end:
               \cs:w #1left_fontweight \cs_end:
               \cs:w #1left_fontshape  \cs_end:
               \cs:w #1left_fontsize   \cs_end:
               \cs:w #1left \cs_end:
             }
             \hss
             \headerboxcenter {#1}
                {
                 \cs:w #1center \cs_end:
                }  
             \hss
             \headerboxright {#1}
                {
                  \cs:w #1right_fontfamily \cs_end:
                  \cs:w #1right_fontweight \cs_end:
                  \cs:w #1right_fontshape  \cs_end:
                  \cs:w #1right_fontsize   \cs_end:
                  %
                  \cs:w #1right \cs_end:
                }
                 
         }
      }
  }



\newcommand*{\ps_aux}[2]
  {
      
    \renewcommand*{\@evenhead}
      { 
        \cs:w format_running_head_#1 \cs_end: {plain_even_header_} {} {} {}
      } 
    
    \renewcommand*{\@oddhead}
      { 
        \cs:w format_running_head_#1\cs_end:  {plain_odd_header_} {} {} {}
      } 
      
    \renewcommand*{\@evenfoot}
      { 
        \cs:w format_running_head_#1\cs_end: {plain_even_footer_} {} {} {}
      } 
       
    \renewcommand*{\@oddfoot}
      { 
        \cs:w format_running_head_#1\cs_end: {plain_odd_footer_} {} {} {}
      }
%      
  \cs_set:Npn \sectionmark##1{##1}
  \def\sectionmark##1
    {
    \markright{\ifnum \c@secnumdepth >\z@
    \thesection\enskip\fi ##1}}%
 }
%    \end{macrocode}
% 
% \begin{docCommand} {make_ps} { \marg{style name} }
%  A generic function for creating new pagestyles. The function takes the pagestyle name as its
%  argument and calls an auxilairy function to create the |ps@| command. 
% \end{docCommand}
%
%    \begin{macrocode} 
\cs_gset:Npn \make_ps #1 #2
  {
     \cs_gset:cpn {ps@#1} 
       {
           \ps_aux{box}{#1}
       }
  }   
 
\make_ps {plain} {box}
\make_ps {headings} {box}
 
\ExplSyntaxOff
%    \end{macrocode} 
%    
%\section{A vertical rule heading}
%
% Provides a vertical rule (for style 22)
% 
%    \begin{macrocode}
\def\ps@verticalrule{\leftskip\z@\let\@mkboth\@gobbletwo\vfuzz=5\p@
    \def\@oddhead{}%
    \def\@evenhead{}%
    
\def\@oddfoot{\verbatimsize
    \parbox[t]{\textwidth}{
    \vspace{15pt}%
      \global\hoffset=0pc%
      \noindent\hbox to\textwidth{\hbox to 0pt{\rule{1pt}{\textheight}\color{blue}\thepage}}
      \makebox[\z@][l]{\@c@pyrightline}%
%     \noindent\hspace*{-9pc}\rule{37pc}{0.25pt}%
    }%
  }%
  
  \def\@evenfoot{\verbatimsize
    \vbox{\vspace{15pt}%
   % \global\hoffset=6pc%
    \noindent\hbox to\textwidth{{\color{blue}\rmfamily
    \thepage}\hfill\makebox[0pt][l]{\rule{1pt}{30pt}}}
    \makebox[\z@][l]{\@c@pyrightline}%
%   \noindent\rule{37pc}{0.25pt}%
    }%
  }%
  \def\sectionmark##1{}%
  \def\subsectionmark##1{}%
%
%
%  
 }
%    \end{macrocode}
%
%
%  \end{document}
% \section{The pagestyle headings}
% This is one of the most common headings. We provide a full |ps@| function for
% speed utilizing only some parameters. The \refCom{sectionmark} is inserted
% by \tex's inserts mechanism, during the section typesetting.
%    \begin{macrocode}
 \tcbset{thepage/.style = 
   {
    top=0pt,right=0pt,bottom=0pt,left=0pt,
    colframe=white,colback=spot!1,arc=1mm,
   }
 }
\ExplSyntaxOn
%
%\def\ps@headings{%
%  \let\@mkboth=\markboth
%  \def\@evenfoot{}
%  \def\@oddfoot {}
%  \cs_set:Npn \@evenhead 
%    {
%      
%      \vbox
%        {\verbatimsize
%        %  \global\hoffset=6pc\noindent
%          \makebox[\z@][l]{\rmfamily 
%          {\protect\tcbox[colframe=red]{-\thepage}}}%
%          \itshape\strut\hfill\leftmark\hbox{}%\par\vbox to 13pt{}%
%%     \noindent\rule{37pc}{0.25pt}%
%        }%
%    }%
%  \def\@oddhead
%    {
%       %\protect\tcbox[thepage]
%       {\vbox{ %replacetcbox vbox
%       \verbatimsize
%       \global\hoffset=0pc\noindent
%        \mbox{}\itshape \strut\rightmark\hfill\hbox{}
%        \makebox[\z@][r]{oo
%        \upshape\rmfamily
%        \thepage
%         }%\par\vbox to 13pt{}%
%%    \noindent\hspace*{-9pc}\rule{37pc}{0.25pt}%
%       }}%
%    }%
%  \def\chaptermark##1{\markboth{##1}{##1}}
%  \def\sectionmark##1
%    {
%    \markright{\ifnum \c@secnumdepth >\z@
%    \thesection\enskip\fi ##1}}%
%    
%   \def\chaptermark##1{\markboth{##1}{##1}}
%%   \def\sectionmark##1{\markright{\ifnum \c@secnumdepth >\z@
%%   \thesection\hspace{0.5em}\fi ##1}}%
%  
%}
%
\ExplSyntaxOff
%    \end{macrocode}
%    
% \section{Centered headings}
% 
%  \begin{docCommand}{ps@centerheadings } { \meta{void} }
%   Provides centered headings. Puts the section mark at the middle
%  of the header and the page at the right or left.
% \end{docCommand}
% 
%    \begin{macrocode}
\def\ps@centerheadings{%
 \let\@mkboth=\markboth
 \def\@evenfoot{}
  \def\@oddfoot {}
  
  \def\@evenhead{
   \verbatimsize
    \vbox{
     \tikzi[even center]
     \noindent
     \global\hoffset=0pc
      \mbox{\rm \thepage}%
      \it \strut
       \leftmark\hfill\hbox{}%\par\vbox to 13pt{}%
%    \noindent\rule{37pc}{0.25pt}%
    }%
  }%
  \def\@oddhead{
    \vbox{%
    \tikzi[odd center]
     \global\hoffset=0pc
    \noindent\hfill
    \mbox{}\it 
    \verbatimsize
    \rightmark\hfill\hbox{}
      \makebox[\z@][r]{\rm
      \thepage}\par\nointerlineskip%
      \vskip0pt
    \noindent\hspace*{0pc}\rule{37pc}{0.25pt}%
     }%
  }%
  
  \def\chaptermark##1{\markboth{##1}{##1}}
  
  \def\sectionmark##1{\markright{\ifnum \c@secnumdepth >\z@
    \thesection\enskip\fi ##1}}%
   \def\chaptermark##1{\markboth{##1}{##1}}
   \def\sectionmark##1{\markright{\ifnum \c@secnumdepth >\z@
      \thesection\hspace{0.5em}\fi ##1}}%
}
%    \end{macrocode}

% \section{Flush headings}
% 
%  \begin{docCommand}{ps@flush} { \meta{void} }
%    Provides centered headings. Puts the section mark at the middle
%    of the header and the page at the right or left.
%  \end{docCommand}
% 
%    \begin{macrocode}
\ExplSyntaxOn
\cs_set:Npn \ps@flush
  { 
     \let\@mkboth=\markboth
     \def\@evenfoot{}
     \def\@oddfoot {}
     \def\@evenhead{}
     \def\@oddhead {}
     \def\@evenhead{
         \verbatimsize
         \parbox[t]{\textwidth}{
         \tikzi[even flush]
          \noindent
           \mbox{
             \rm \thepage}%
             \it \strut
            \enskip\leftmark
            \hfill\hbox{}
             }%
      }%
  
\def\@oddhead
  {
     \hbox:n
       {
         \rlap{\parbox[t]{\textwidth}
          {
             \tikzi[odd flush]
             \noindent\hfill\hfill
              \mbox{} 
              \verbatimsize \rm
              \rightmark\hbox{}
              \mbox{
                 \rm
                 \space/\thinspace
                 \thepage
                }
%               \par\nointerlineskip%
%               \vskip0pt
%               \noindent\hspace*{0pc}\rule{\textwidth}{0pt}%
          }
       }  
    }%
  }
  
 \def\sectionmark##1{\markright{\ifnum \c@secnumdepth >\z@
        \thesection\enskip\fi ##1}}%
        
 %\def\chaptermark##1{\markboth{##1}{}}

 \def\sectionmark##1{\markright{\ifnum \c@secnumdepth >\z@
     \sectionname \enspace \thesection\hspace{0.5em}\fi ##1}}%
      
\ExplSyntaxOff      
}
%    \end{macrocode}

% \section{Chaperstyle heading}    
% 
%  This style provides a special heading to be used on chapter opening pages only.
%  \tcbdocmarginnote[width=2cm]{U 20-6-2015}
%  
%    \begin{macrocode}
\def\ps@chapterstyle{%
    \let\@oddfoot\@empty\let\@evenfoot\@empty
    \def\@evenhead{\thepage\hfil\slshape\leftmark}%
    \def\@oddhead{{\slshape\rightmark}\hfil\thepage}%
    \let\@mkboth\@gobbletwo
    \let\chaptermark\@gobble
    \let\sectionmark\@gobble}
%    \end{macrocode}
%
%
% 
% \section{Myheadings}
%
% \begin{docCommand}{ps@myheadings} {\meta{void}}
%   The classic myheadings macro, with some parameters!
% \end{docCommand}
%    \begin{macrocode}
\ExplSyntaxOn
\cs_set:Npn \ps@myheadings 
  {
    \let\@mkboth=\@gobbletwo
    \def\@evenfoot{Even page note 2}
    \def\@oddfoot {Odd page note 1}
    
    \def\@evenhead
    {
      \verbatimsize
      \vbox:n
        {  
          \global\hoffset=6pc\noindent
         \makebox[\z@][l]{\rm \thepage}%
         \upshape \strut\hfill\leftmark\hbox{}%\par\vbox to 13pt{}%
         \noindent\rule{37pc}{0.25pt}%
        }%
    }%
% for one side or both sides    
  \def\@oddhead{\verbatimsize
    \vbox:n 
      {
        \global\hoffset=0pc\noindent
        \tcbox[width=\textwidth]{\upshape\strut\rightmark\hfill\hbox{}\makebox[\z@][r]{\rm
        MHEADINGS  \thepage}%\par\vbox to 13pt{}%
        \vspace{1pt}
     \noindent\hspace*{-9pc}\rule{37pc}{0.25pt}%
      }%
  }%
%
    \def\chaptermark##1{}
    \def\sectionmark##1{}
    \def\subsectionmark##1{}
   }
 }
\ExplSyntaxOff 
%    \end{macrocode}
%
% \section{Watermarks}

% Some special styles
%    \begin{macrocode}
\IfFileExists{changepage.sty}{\RequirePackage{changepage}}{}
\IfFileExists{rotating.sty}{\RequirePackage{rotating}}{}
%    \end{macrocode}
%
% {even@samplepage}
% {odd@samplepage}
%    \begin{macrocode}
\def\even@samplepage{%
 \begin{picture}(0,0)
   \put(\Xeven,\Yeven){\turnbox{90}{\Huge \textcolor{\watermark@textcolor}{\watermark@text}}}
\end{picture}
}
%% Define a macro to print SAMPLE PAGE IN THE MARGIN
\def\odd@samplepage{%
 \begin{picture}(0,0)
   \put(\Xodd,\Yodd){\turnbox{90}{\Huge \textcolor{\watermark@textcolor}{\watermark@text}}}
 \end{picture}
}
%    \end{macrocode}
% 
% 

% \section{watermarktext}
%  Define the watermark words
%    \begin{macrocode}
\gdef\watermarktext#1{\gdef\watermark@text{\fontfamily{phv}\selectfont#1}}
\def\watermarktextcolor#1{\gdef\watermark@textcolor{#1}}
\watermarktext{SAMPLE PAGE}
\watermarktextcolor{black!50}
%    \end{macrocode}
% 
%    \begin{macrocode}
\def\ps@samplepage{\let\@mkboth\@gobbletwo
 \let\@oddhead\odd@samplepage\def\@oddfoot{\reset@font\hfil\thepage}
 \let\@evenhead\even@samplepage\def\@evenfoot{\reset@font\thepage\hfil}}
%
% We define two macros to position the watermark on the page
\def\Xodd{480}
\def\Xeven{-15}\def\Yeven{-810}
\def\Yeven{-\expandafter\strip@pt\textheight}
\let\Yodd\Yeven


\cxset{blank page text/.store in=\blankpagetext@cx{#1}}
\cxset{blank page text={}}

\def\cleardoublepage{\clearpage\if@twoside\ifodd\c@page\else
  \hbox{}
  \vspace*{\fill}
  \begin{center}
     \blankpagetext@cx      
  \end{center}
  \vspace{\fill}
  \thispagestyle{empty}
  \newpage
  \if@twocolumn\hbox{}\newpage\fi\fi\fi}
%    \end{macrocode}
%</RH>       
%
% \pagestyle{flush}
% \section{Test myheadings}
% \lipsum[1-3] 
% \section{my headings 1}
% \lipsum [4-7]
% \section{my headings 2}
% \lipsum [4-7]
% \section{Another one}
%  \makeatletter  
%    \@oddhead
%
%    \@evenhead
% \lipsum [4-5]
% \section{test}
%
%  \makeatletter  
%    \@oddhead
%
%    \@evenhead
% \makeatother
%
% \leftmark\\
% \rightmark\\
%
\endinput
% \bibliographystyle{alpha}
% \begingroup
% \raggedright
%
% \begin{thebibliography}{GMSN94A}
%
% \bibitem[GMS94]{GOOSSENS94}
% Michel Goossens, Frank Mittelbach, and Alexander Samarin.
% \newblock {\em The LaTeX Companion}.
% \newblock Addison-Wesley Publishing Company, 1994.
%

%
% \end{thebibliography}
% \endgroup
%
% \PrintIndex
% \Finale
\endinput


