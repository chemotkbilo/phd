% \iffalse meta-comment
%<*internal>
\iffalse
%</internal>
%<*readme>
----------------------------------------------------------------
phd-runningheads 
A package to manage running heads in LaTeX
E-mail: yannislaz@gmail.com
Released under the LaTeX Project Public License v1.3c or later
See http://www.latex-project.org/lppl.txt
----------------------------------------------------------------
This file provides a template for defining a class.
%</readme>
%<*todo>
Improve on User markup
%</todo>
%<*internal>
\fi
\def\nameofplainTeX{plain}
\ifx\fmtname\nameofplainTeX\else
  \expandafter\begingroup
\fi
%</internal>
%<*install>
\input docstrip.tex
\keepsilent
\askforoverwritefalse
\preamble
----------------------------------------------------------------
phd-runningheads 
A package to manage running heads in LaTeX
E-mail: yannislaz@gmail.com
Released under the LaTeX Project Public License v1.3c or later
See http://www.latex-project.org/lppl.txt
----------------------------------------------------------------
\endpreamble
\postamble
 Copyright (C) 2015 by Dr. Yiannis Lazarides <yannislaz@gmail.com>
\endpostamble
%\usedir{tex/latex/\jobname}
\generate{
  \file{\jobname.sty}{\from{\jobname.dtx}{TOC}}
 }
%</install>
%<install>\endbatchfile

%<*internal>
%\usedir{source/latex/\jobname}
\generate{
  \file{\jobname.ins}{\from{\jobname.dtx}{install}}
}
\nopreamble\nopostamble
%\usedir{doc/latex/demopkg}
\generate{
  \file{README.txt}{\from{\jobname.dtx}{readme}}
}
%\generate{
%  \file{phd-testhead.tex}{\from{\jobname.dtx}{TEST}}
%}
%\generate{
%  \file{TODO.tex}{\from{\jobname.dtx}{TODO}}
%}
\ifx\fmtname\nameofplainTeX
  \expandafter\endbatchfile
\else
  \expandafter\endgroup
\fi
%</internal>
%<*driver>
\NeedsTeXFormat{LaTeX2e}
\ProvidesFile{phd-toc.drv}%
  [2013/01/13 v1.0 ]%
\documentclass[oneside,11pt,a4paper]{ltxdoc}
\usepackage[bottom=4cm,footskip=3cm,
            headheight=15pt, headsep=2cm]{geometry}
\savegeometry{std}

\usepackage{phd}
\cxset{geometry units=mm}
\usepackage{phd-toc}
\usepackage{phd-lowersections}
\usepackage{phd-runningheads}
%% LaTeX2e file `defaults-chapters'
%% generated by the `filecontents' environment
%% from source `phd-scriptsmanager' on 2015/08/25.
%%
%%    General Defaults for Chapters
\cxset{%
    chapter title margin-top-width    =  0cm,
    chapter title margin-right-width  =  1cm,
    chapter title margin-bottom-width = 10pt,
    chapter title margin-left-width   = 0pt,
    chapter align                     = left,
    chapter title align               = left, %checked
    chapter name                      = hang,
    chapter format                    = fashion,
    chapter font-size                 = Huge,
    chapter font-weight               = bold,
    chapter font-family               = sffamily,
    chapter font-shape                = upshape,
    chapter color                     = black,
    chapter number prefix             = ,
    chapter number suffix             = ,
    chapter numbering                 = arabic,
    chapter indent                    = 0pt,
    chapter beforeskip                = -3cm,
    chapter afterskip                 = 30pt,
    chapter afterindent               = off,
    chapter number after              = ,
    chapter arc                       = 0mm,
    chapter background-color          = bgsexy,
    chapter afterindent               = off,
    chapter grow left                 = 0mm,
    chapter grow right                = 0mm,
    chapter rounded corners           = northeast,
    chapter shadow                    = fuzzy halo,
    chapter border-left-width         = 0pt,
    chapter border-right-width        = 0pt,
    chapter border-top-width          = 0pt,
    chapter border-bottom-width       = 0pt,
    chapter padding-left-width        = 0pt,
    chapter padding-right-width       = 10pt,
    chapter padding-top-width         = 10pt,
    chapter padding-bottom-width      = 10pt,
    chapter number color              = white,
    chapter label color               = white,
    }
 \cxset{
    chapter number font-size        = huge,
    chapter number font-weight      = bfseries,
    chapter number font-family      = sffamily,
    chapter number font-shape       = upshape,
    chapter number align            = Centering,
    }
\cxset{%
     chapter title font-size        = Huge,
     chapter title font-weight      = bold,
     chapter title font-family      = calligra,
     chapter title font-shape       = upshape,
     chapter title color            = black,
     }

\sethyperref
\EnableCrossrefs
\CodelineIndex
\RecordChanges
%\def\toctitle#1{\protect\color{red}#1}
%\def\addcontentsline#1#2#3{%
% \addtocontents{#1}{\protect\contentsline{#2}{\protect\toctitle{#3}}{\thepage}}} 
\definecolor{bgsexy}{HTML}{F09078}
\definecolor{creamy}{HTML}{D83078}
\cxset{chapter title color= creamy,
       chapter label color = creamy,
       chapter number color = creamy,
       chapter number font-size = Huge,
       subsection title color = creamy,
       chapter name = CHAPTER,
       chapter label case = upper,
       chapter align=left,
       title align=left}
\begin{document}
  \coverpage{desert}{Book Design Monographs}{Camel Press}{ON CONTENTS PAGES}{AND THEIR DESIGN} 
 \secondpage
 \newpage
 \tableofcontents
% \listoffigures
% \listoftables
 \mainmatter
 \thispagestyle{plain}
 %\makeatletter\@specialfalse\makeatother
%\parindent1em
%\cxset{section align=left}
%\cxset{toc image= ted_turner_2001,
%          subsection color=black,
%          section color=black,
%          subsubsection numbering=arabic,
%          subsubsection color=black,
%          subsubsection font-shape=upshape}
\chapter{TOC Styling}
\label{ch:toc}

\precis{In this chapter we outline a number of settings that have been defined to handle Table of Contents (ToC) formatting. We also review the technical issues facing the construction of Table of Contents and their variants. The relationship of the hyperref package to the ToC and the many issues arising out of it are also discussed.}
\addtocimage{-12pt}{-20pt}{../images/tocblock-rooster.jpg}



\section{Introduction}

The setting of parameters for the |ToC|, is a bit more complex than those for sectioning commands. It also requires that one has a good understanding of \latexe terminology. However, t set they are easy to change afterwards.
The \pkgname{phd} also provides a number of predefined ones.

An entry line to the |ToC| can be though of consisting of four components. The entry type (chapter, section etc.), number, title and page number. This presents a similar problem to that of styling sectioning commands to parametrize it in such a way as to provide a fully flexible system. The code provided by \latexe is complicated and difficult to parameterize it. Ideally we would like a fully flexible system that can be moulded into  


%\oldcontentsline{section}{heading title}{\thepage}
%
%\contentsline{section}{heading title}{\thepage}{chapter.1}
%
%\contentsline {section}{\numberline {}Introduction}{5}{section.2.1}


\section{Key values for Chapters}


\begin{docKey}[phd]{toc levels}{ = \meta{number} } {no default, initial 6}
 Equivalent parameterized key to |\toclevel|. It sets  the \texttt{tocdepth} counter. 
 If set to zero toc is defined. \latex uses this to determine how many levels of toc are included.
\end{docKey}

\begin{docKey}[phd]{toc number width}{ = \marg{dimension}}{}
The key sets the width of the numbers in the ToC. 
\end{docKey}

\subsection{Spacing Keys}
\begin{docKey}[phd]{toc chapter number color}{=\meta{color name}}{}
This key will set the color of the chapter number.
\end{docKey}

\begin{docKey}[phd]{toc chapter beforeskip}{=\meta{length}}{}
The amount of glue to insert before the toc chapter line.
\end{docKey}

\begin{docKey}[phd]{toc chapter afterskip}{ =\meta{length}}{}
The amount of glue to insert before the toc chapter line.
\end{docKey}

\begin{docKey}[phd]{toc chapter indent}{=\meta{length}}{}
The amount of glue to indent the line.
\end{docKey}
\subsection{Hooks}


\makeatletter
\def\options#1{
\@for\next:=#1\do{%
\textbar\next%
}%
\textbar%
}%
\options{test1,test2,test3}
\makeatother
\begin{docKey}[phd]{toc dots}{ = \textbar none\textbar false\textbar true}{}
\end{docKey}

\begin{marglist}
 \item[none] this sets the value of \cs{@dotsep} to 1000 thus eliminating the dots.
\item [false] alias for the above key.
\item [true] dot leaders are used, uses a default value for \cs{@dotsep} of 4.5,
\end{marglist}
\keyval{toc dotsep}{\marg{number}}{renewcommand  @dotsep}
\keyval{toc chapter name}{}{chaptername}
\keyval{toc chapter name color}{}{tocchapternamecolor@cx}
\keyval{toc title color}{}{toctitlecolor@cx}
\keyval{toc title font-weight}{}{toctitlefontweight@cx}
\keyval{toc title before}{}{toctitlebefore@cx}
\keyval{toc title after}{}{toctitleafter@cx}
\keyval{toc after pagenumber}{}{tocafterpagenumber@cx}
\keyval{toc right margin}{}{@tocrmarg}
\keyval{toc number after}{}{tocnumberafter@cx}
\keyval{toc image}{\marg{filename}}{If the TOC has images associated with chapters these are used in the typesetting. Used with custom templates.}
\keyval{toc custom}{\marg{cmd}}{Triggers the loading of alternative templates than those structured by LaTeX.}
\keyval{hypersetup linkcolor}{\marg{color}}{Changes the color of a hyperlink. This is experimental and better used at document level.}
\keyval{toc chapter precis}{\marg{true|false}}{This is experimental. If a chapter precis is specified in a chapter, then the precis is typeset in the ToC also\footnote{Such a command is provided by the \pkg{tocloft} package and in the memoir class}.}

\section{Styling the sections in the ToC}

\begin{docKey}[phd]{toc section beforeskip}{ = \meta{length}}{}
\end{docKey}
\begin{docKey}[phd]{toc section afterskip}{ = \meta{length}}{}
\end{docKey}
\begin{docKey}[phd]{toc section indent}{ = \meta{length}}{}
\end{docKey}

\begin{verbatim}
\cxset{toc section beforeskip/.store in=\tocsectionbeforeskip@cx,
       toc section indent/.store in=\tocsectionindent@cx,
%      fonts for title &num
       toc section font-size/.store in=\tocsectionfontsize@cx, 
       toc section font-family/.store in=\tocsectionfontfamily@cx, 
       toc section font-shape/.store in=\tocsectionfontshape@cx, 
       toc section font-weight/.store in=\tocsectionfontweight@cx, 
       toc section color/.store in=\tocsectioncolor@cx,
       toc section numwidth/.store in=\tocsectionnumwidth@cx,
%	  fonts etc for page number
       toc section page font-size/.store in=\tocsectionpagefontsize@cx,
       toc section page font-family/.store in=\tocsectionpagefontfamily@cx,
       toc section page font-shape/.store in=\tocsectionpagefontshape@cx,
       toc section page font-weight/.store in=\tocsectionpagefonteight@cx,
       toc section page color/.store in=\tocsectionpagecolor@cx,
%      leaders
	 toc section dotsep/.store in = \tocsecdotsep@cx,
%      before and after page number
       toc section page before/.store in=\tocsectionpagebefore@cx,
       toc section page after/.store in=\tocsectionpageafter@cx,
}
\end{verbatim}

Firstly we define the width of the box that the page number is set. Use ems so that it does not need to be redefined for every change in font size.
ToC entries are treated as rectangular areas where the text
and probably a filler will be written. Let's draw such an
area (of course, the lines themselves are not printed):



\section{Other Packages and Classes}
The package \pkg{tocloft} provides  provides handles for an author to change the design to meet the needs of the particular document, by providing a number of settings commands.


%\addtocontents{toc}{\colorbox{cyan}{\thesection this is some long command \thepage}}

\section{Technical Details}

In the standard classes the design of the Table of Contents (ToC) the List of Figures (LoF) and list of tables (LoT) is fixed and buried within the class definitions.

\begin{macro}{\addcontentsline}
\begin{macro}{\addcontents}
To understand the way \LaTeX\ formats the ToC, one has to understand that the ToC entries are generated and typeset in different operations. Firstly when the document is processed, every time a sectioning command such as \cs{chapter} or \cs{section} is activated it calls on either the macro \cs{addcontentsline} or \cs{addcontents}, which in turn will initiate the process of writing the entry onto a file.
\end{macro}
\end{macro}


\subsection{Reading from the ToC file}

\begin{macro}{\tableofcontents}
The second operation happens when \latexe sees a \cs{tableofcontents} command. This initiates the read operation, where the information that has been stored in the ToC file is read and typeset.
\end{macro}


\subsubsection{Initiating reading of the ToC, the \texttt{\textbackslash tableofcontents} command}

We should start dissecting the algorithm by first viewing the \cs{tableofcontents}. This command is provided in
the standard classes (see \pageref{tableofcontents}) and it does not take any parameters.

\begin{teXX}
\setcounter{tocdepth}{2}
\newcommand\tableofcontents{%
    \if@twocolumn
      \@restonecoltrue\onecolumn
    \else
      \@restonecolfalse
    \fi
    \chapter*{\contentsname
        \@mkboth{%
           \MakeUppercase\contentsname}{\MakeUppercase\contentsname}}%
          \@starttoc{toc} (*@\label{starttoc}@*)
    \if@restonecol\twocolumn\fi
    }
\end{teXX}


The important thing to notice here is that the words contents are typeset by calling the star version and hence the contents name is not added to the toc. If we needed to add it we need to explicitly add to the toc as well as format it, if necessary. The other notable item is the \cs{@starttoc}\marg{toc} macro. This command opens the |toc| file to read or write.

\subsubsection{The Contents heading}
 
 The \pkgname{phd}  package redefines the \cmd{\tableofcontents}  to call a specific macro to typeset
 the contents name and provides hooks to provide flexible styling. 
              
%\begin{texexample}{Contents heading and hooks}{}
%\begingroup
%\cxset{toc name=Inhalt,
%       toc name color=black,
%       toc name align=left,
%       toc name indent=,
%       toc name before=\topline\par,
%       toc name after=\par\bottomline\par}
%
%\cxset{toc name=Contents,
%          toc name case=none,
%          toc name color=black,
%          toc name align=none,
%          toc name indent=\hspace*{4.2cm},
%          toc name before=\hspace*{4.2cm}\rule{\textwidth-4.2cm}{1pt}\vskip1.5pt,
%          toc name after=\par\vskip30pt\bottomline\par,
%          toc name font-size=Huge,
%          toc name font-family=rmfamily}
%\endgroup 
%\end{texexample}

The last example is from an Oxford University Press publication \textit{Portraiture}. It would never pass my mind to design such a Contents page style, but it does look and is a good test for our code  (Figure~\ref{tocsample}). Our spacing looks odd, as the geometry of the page is different as well as the selection of font. This design is found in the full Oxford History of Art series.

\begin{figure}[htbp]
\centering
\includegraphics[width=0.8\textwidth]{oxford-toc}
\caption{Spread with ToC starting page from \textit{Portraiture}.}
\label{tocsample}
\end{figure}

While we at it, let us go over one more example. This time the template is shown in Figure~\ref{fig:reinvent}. This is a ruled example. The top rule will be at the beginning of the ToC and the bottom will be at the beginning of the Part. We will come back for the latter later.

\begin{figure}[htbp]
\centering
\includegraphics[width=0.5\textwidth]{reinvent}
\caption{Spread with ToC starting page from \textit{Reinvent Yourself}.}
\label{fig:reinvent}
\end{figure}

%\begin{texexample}{Contents heading and hooks}{}
%\bgroup
%\def\doublerule{\hrule width\textwidth height1pt depth0pt\vskip1pt
%     \hrule width\textwidth height3.5pt depth0pt\vskip24pt\relax}
%\cxset{toc name=Contents,
%          toc name case=none,
%          toc name color=black,
%          toc name align=center,
%          toc name indent=, 
%          toc name before= doublerule,
%          toc name font-size=Huge,
%          toc name font-family=rmfamily}
% \sampletoctitle
%\egroup 
%\end{texexample}

What we have just done we defined a double rule macro and just inserted it as the argument before the name hook. The ``contents'' was typeset as is and centered align. 



\subsubsection{The \textbackslash @starttoc macro}

Remember when the \cmd{\tableofcontents} was intitiated it called the \cmd{\@starttoc} command in line \ref{starttoc}. This is defined in the LaTeX kernel and not in the class files. The \cmd{@starttoc}\meta{ext} command is used with the commands:
\cs{tableofcontents}, \cs{listoffigures}, etc. \footnote{See a more detailed explanation on ltsect.dtx, page 288.}

For example: \cs{@starttoc}{lof} is used in listoffigures. This command
reads the |.ext| and typesets it if available and sets up to write the new file. The reading operation always takes place (Line \ref{readtoc}). The |\@nobreakfalse| is globally set to allow breaks. 

\begin{teX}
\def\@starttoc#1{%
\begingroup
  \makeatletter
  \@input{\jobname.#1}% (*@\label{readtoc}@*)
  \if@filesw
    \expandafter\newwrite\csname tf@#1\endcsname
    \immediate\openout \csname tf@#1\endcsname \jobname.#1\relax
   \fi
   \@nobreakfalse
\endgroup}
\end{teX}

We can modify the command slightly in a group and load |phd.tst| a file I have created to see how it is working. For the time being let us ignore how we wrote to the file.

\begin{texexample}{Test ToC}{}
\makeatletter
\cxset{toc section color=black} 
\begingroup
\def\@starttoc#1{
    \@input{\jobname.#1}
    \@nobreakfalse}
\@starttoc{tst}  
\endgroup
\makeatother
\end{texexample}

The reason we have defined the command in a group---as well as called it---was to avoid redefining it and breaking this document and also not to waste another file as they are limited in TeX. This command it breaks with good programming practice, where functions should be defined to do one thing at a time it would have been preferable to have had |\@readtoc| and  |\@writetoc| commands. More about this later.

The file we have just read is as follows:

\begin{verbatim}
\select@language {english}
\contentsline {chapter}{\chapternumberline {1}{Introduction}{}}{1}{section*.2}
\addtocimage@cx {-12pt}{-20pt}{../images/tocblock-fish.jpg}
\contentsline {section}{\numberline {1}The key value concept}{2}{section.1.1}
\contentsline {subsection}{\numberline {1.1}Settings}{3}{subsection.1.1.1}
\contentsline {subsection}{\numberline {1.2}Cascading}{3}{subsection.1.1.2}
\end{verbatim}

The \cs{contentsline} definition triggers the calling of macros that start with \verb+l@+ and for the sectioning commands have typical formats such as \lstinline{\l@chapter, \l@section etc.} So in reality when we read the
file the comand:
\begin{teX}
\def\contentsline#1{\csname l@#1\endcsname}
\end{teX}
was expanded and absorbed the first parameter, for example |{section}| which then continued expanding to |\l@section| to absorb the balance parameters.

\begin{texexample}{Expansion of l@section}{}
\makeatletter
\l@section{\numberline {1}The key value concept}{2}{section.1.1}
\l@section{\numberline {1}The key value concept}{2}{}
\makeatother
\end{texexample}

\subsubsection{Hyperref}

The example did not work as advertized for the simple reason that hyperref interferes to pick the last parameter for its own purpose. As Peter Wilson says the \pkg{hyperref} package dislikes authors using
\cs{addcontentsline}. To get it to work properly with \pkg{hyperref}  you normally have to put \cs{phantomsection} (a macro defined within  the \pkg{hyperref} package) immediately  before \cs{addcontentsline}. This gave me considerable headaches when redefining these commands for special |ToC|s.
When we use hyperef we get an additional parameter |chapter.1| on line one and if we combine the example above, we will get errors of run away arguments.

\begin{texexample}{ToC example with hyperref}{ex:toc2}
\contentsline {chapter}{\numberline {1}First Chapter}{3}{chapter.1}
\contentsline {section}{\numberline {1.1}Introduction}{3}{section.1.1}
\contentsline {subsection}{\numberline {1.1.1}The difficulties}{3}{subsection.1.1.1}
\contentsline {chapter}{\numberline {2}Second Chapter}{5}{chapter.2}%
\end{texexample}

Predictably |hyperref| redefines the |\contentsline| command. We see from the documentation that 
the \cmd{\contentsline} now takes four parameters and uses a case statement to handle the options.
\startlineat{7901}
\begin{teX}
\def\contentsline#1#2#3#4{%
  \ifx\\#4\\%
    	\csname l@#1\endcsname{#2}{#3}%
  \else
 	\ifcase\Hy@linktoc % none
 		\csname l@#1\endcsname{#2}{#3}%
 	\or % section
 		\csname l@#1\endcsname{%
  	   \hyper@linkstart{link}{#4}{#2}\hyper@linkend
    	}{#3}%
  	 \or % page
		\csname l@#1\endcsname{{#2}}{%
    	\hyper@linkstart{link}{#4}{#3}\hyper@linkend
    	}%
 	\else % all
 		\csname l@#1\endcsname{%
 	\hyper@linkstart{link}{#4}{#2}\hyper@linkend
 	}{%
 	\hyper@linkstart{link}{#4}{#3}\hyper@linkend
 	}%
 	\fi
 \fi
}
\end{teX}

\begin{texexample}{Hyperref Test}{ex:testhyper}
\bgroup
\makeatletter
\edef\one{section}
\edef\two{ {\numberline {2}Second Chapter}}
\edef\three{5}
\edef\four{chapter.2}

\l@section{\numberline {2}Second Chapter}{%
    	\hyper@linkstart{link}{chapter.2}{5}\hyper@linkend}{}
\makeatother
\egroup
\end{texexample}

Moral of a long story beware of redefinitions by other packages. But at least hyperref did not redefine |l@section|,
which was left for us to use as a hook for typesetting? 

As we get many variations as to how the links are styled by hyperref
Here is a minimal how to set it.

\emphasis{linktocpage}
\begin{teX}
\documentclass{book}
\usepackage{xcolor}
\usepackage{hyperref}
\hypersetup{pdftex,
  bookmarks,
  raiselinks,
  pageanchor,
  hyperindex=true,
  colorlinks,
  allcolors=theblue, 
  hyperfootnotes=true,
  breaklinks=true,
  anchorcolor= blue,
  filecolor=blue,
  urlcolor= blue,
  linkcolor= blue,
  pdftitle={My Title},
  pdfauthor={Yiannis Lazarides},
  pdfsubject={The phd LaTeX package},
  pdfkeywords={LaTeX package management, document design}
 }
\hypersetup{linktocpage}
\begin{document}
\tableofcontents
\chapter{First Chapter}
\section{Introduction}
\subsection{The difficulties}
\chapter{Second Chapter}
test
\end{document}
\end{teX}


\begin{verbatim}
\hypersetup{linktocpage=true}
\end{verbatim}

\subsection{Taking a look at the \textbackslash l@ commands}

\begin{teX}
\renewcommand*\l@chapter[2]{%
  %#1 number and title  #2 page number
  \ifnum \c@tocdepth >\m@ne
    \addpenalty{-\@highpenalty}%
    \vskip 1.0em \@plus\p@
    \setlength\@tempdima{1.5em}%
    \begingroup
      \parindent \z@ \rightskip \@pnumwidth
      \parfillskip -\@pnumwidth
      \leavevmode \bfseries \color{thegray}
      \advance\leftskip\@tempdima
      \hskip -\leftskip
      (#1)\nobreak\hfil \nobreak\hb@xt@\@pnumwidth{\hss#2\hspace*{3cm}}\par
      \penalty\@highpenalty
    \endgroup
  \fi}

\renewcommand*\l@chapter[2]{%
  %#1 number and title  #2 page number
  \ifnum \c@tocdepth >\m@ne
    \addpenalty{-\@highpenalty}%
    \vskip 1.0em \@plus\p@
    \setlength\@tempdima{1.5em}%
    \begingroup
      \parindent \z@ \rightskip \@pnumwidth
      \parfillskip -\@pnumwidth
      \leavevmode \bfseries \color{thegray}
      \advance\leftskip\@tempdima
      \hskip -\leftskip
      \colorbox{red}{\hbox to 10cm{\color{white}#1\hss #2}}\par
      \penalty\@highpenalty
    \endgroup
  \fi}
\l@chapter{1 test}{12}
\end{teX}


So far we have simplistically examined the formatting. Life can gets more complex if we want to have a layout as shown in Figure \ref{fig:toc}. The |TOC| has different color formatting for sections and chapters, there are no leaders and and the layout is set in a two column format as compared to the one column format of the standard classes |TOC|.

\begin{figure}[tp]
\centering
\fbox{\includegraphics[width=0.5\textwidth]{contents01.png}}
\caption{Complex table of contents layout.}
\label{fig:toc}
\end{figure}


\begin{lstlisting}
\renewcommand\l@chapter[3]{%
  %#1 number and title  #2 page number
  \ifnum \c@tocdepth >\m@ne
    \addpenalty{-\@highpenalty}%
    \vskip 1.0em \@plus\p@
    \setlength\@tempdima{1.5em}%
    \begingroup
      \parindent \z@ \rightskip \@pnumwidth
      \parfillskip -\@pnumwidth
      \leavevmode
      \advance\leftskip\@tempdima
      \hskip -\leftskip
      \vbox{\raggedright#1\vskip1pt%
      \hrule width3cm height0.4pt}\par
      #2
      \penalty\@highpenalty
    \endgroup
  \fi}
\end{lstlisting}

\begin{lstlisting}
% define three parameters the chapter number, title separate
\renewcommand\l@chapter[3]{%
  %#1 number and title  #2 page number
  \ifnum \c@tocdepth >\m@ne
    \addpenalty{-\@highpenalty}%
    \vskip 1.0em \@plus\p@
    \setlength\@tempdima{1.5em}%
    \begingroup
      \parindent \z@ \rightskip \@pnumwidth
      \parfillskip -\@pnumwidth
      \leavevmode
      \advance\leftskip\@tempdima
      \hskip -\leftskip
      \vbox{\raggedright#1\vskip1pt%
      \hrule width3cm height0.4pt}\par
      #2
      \penalty\@highpenalty
    \endgroup
  \fi}
\cxset{chapter color=thegreen}
\l@chapter{\color{\chaptercolor@cx}
\bfseries\chaptername\hskip1em\thechapter}{A Chapter Title}{12}
\end{lstlisting}


\subsection{Images in the TOC}
Figure \ref{fig:tocsteward} shows another |TOC| this time from a mathematics textbook. This is a much more complicated layout and includes images.

If we take a similarly flexible approach of redefining l@chapter we can try and format the toc shown in Figure \ref{fig:tocsteward}.

\begin{figure}[tp]
\centering

\includegraphics[width=0.8\textwidth]{contents02.png}
\caption{Complex table of contents layout.}
\label{fig:tocsteward}
\end{figure}



\begin{lstlisting}
\renewcommand\l@chapter[3]{%
  %#1 number and title  #2 page number
  \ifnum \c@tocdepth >\m@ne
    \addpenalty{-\@highpenalty}%
    \vskip 1.0em \@plus\p@
    \begingroup
      \parindent \z@
      \leavevmode
      \vbox{\raggedright\colorbox{blue}{\color{white}\bfseries\sffamily#1} #2\qquad  #3\vskip0pt%
      \color{blue}\hrule width0.7\textwidth height0.4pt}\par
      \penalty\@highpenalty
    \endgroup
  \fi}
\end{lstlisting}




The above method of redefinition is a bit more flexible and can be extended to cover all cases that do not fall broadly with the standard class provisions. Any layout is possible.


\section{Writing the Table of Contents Entries}

So far we have examined the reading of the ToC file and next we need to understand the writing operation
to the ToC. If we have to ensure that information survives we need to write it to the file. This is done via two macros.


\begin{macro}{\addcontentsline}
The \cs{addcontentsline}\marg{table}\marg{type}\marg{entry} command allows the user to add an entry to a table of contents. The command adds the entry
\cs{contentsline}\marg{type}\marg{entry}\marg{page} to the \marg{.lot} file.
\end{macro}

\begin{teXXX}
 \addcontentsline{toc}{chapter}%
       {\protect\numberline{\thechapter}#1}
\end{teXXX}

\startlineat{144}
\begin{teX}       
 \long\def\addtocontents#1#2{%
 \protected@write\@auxout
 {\let\label\@gobble \let\index\@gobble \let\glossary\@gobble}%
 {\string\@writefile{#1}{#2}}}
\end{teX}

\begin{teX}
\def\addcontentsline#1#2#3{%
  \addtocontents{#1}{\protect\contentsline{#2}{#3}{\thepage}}}

\def\contentsline#1{\csname l@#1\endcsname}
\end{teX}




  % \catcode `\| =12
%  \DeleteShortVerb
  \makeatletter
  \@debugtrue
  \makeatother
  \cxset{headings odd header background color = white,
         headings even header background color = white,
  }
  \pagestyle{headings}
  \DocInput{phd-toc.dtx}%
\end{document}
%</driver>
% \fi
% \CheckSum{553}
% \CharacterTable
%  {Upper-case    \A\B\C\D\E\F\G\H\I\J\K\L\M\N\O\P\Q\R\S\T\U\V\W\X\Y\Z
%   Lower-case    \a\b\c\d\e\f\g\h\i\j\k\l\m\n\o\p\q\r\s\t\u\v\w\x\y\z
%   Digits        \0\1\2\3\4\5\6\7\8\9
%   Exclamation   \!     Double quote  \"     Hash (number) \#
%   Dollar        \$     Percent       \%     Ampersand     \&
%   Acute accent  \'     Left paren    \(     Right paren   \)
%   Asterisk      \*     Plus          \+     Comma         \,
%   Minus         \-     Point         \.     Solidus       \/
%   Colon         \:     Semicolon     \;     Less than     \<
%   Equals        \=     Greater than  \>     Question mark \?
%   Commercial at \@     Left bracket  \[     Backslash     \\
%   Right bracket \]     Circumflex    \^     Underscore    \_
%   Grave accent  \`     Left brace    \{     Vertical bar  \|
%   Right brace   \}     Tilde         \~}
%
%
% \changes{1.0}{2015/05/03}{Converted to DTX file}
%
% \DoNotIndex{\newcommand,\newenvironment}
%
% \GetFileInfo{template.dtx}
% \providecommand*{\url}{\texttt}
%  \def\fileversion{v1.0}          
%  \def\filedate{2012/03/06}
% \title{The \textsf{\jobname} package.
% \author{Dr. Yiannis Lazarides}
% \thanks{This
%        file (\texttt{\jobname.dtx}) has version number 
%        \fileversion, last revised
%        \filedate.}
% }
% 
% \date{\filedate}
%
% \newpage
% \maketitle
% 
% \begin{summary}
%   This package forms part of the \pkgname{phd} bundle and can be used to manage running heads
%   with a key-value interface. It comes with a number of predefined styles, that cover most of the
%  common use cases for books and journals. It is still in alpha stage. However it will not hopefully
%  break anything if used and is compatible with the fancyhdr package.
% \end{summary}
%
%
%
% ^^A\StopEventually{}
%<*TOC> 
% \part{Package Implementation}
% 
% \chapter{Objectives and Design Philosophy}
% ^^A\def\headhookleft{\agrid}
%  
% Before we start with the coding is importnat to outline the  package objectives:
%
% \begin{enumerate}
%  \item To simplify the user interface for specifying  contents pages in \latexe. This
%        is to be implemented as a key value interface.
% \item To make available a number of differently styled contents pages
%       so that the book designer can adjust properties, rather than creating new
%       layouts.
% \item To provide a key value interface to blend with the othes keys provided by
%  the \pkgname{phd}  package.
% \end{enumerate}

% \section{Terminology}
% 
% \begin{description}
%   \item [Table of Contents] A list of all document headings and the page reference.
%   \item [List of Figures] A list with all figures appearing in the document and
%                           the page where they are referenced. 
%   \item [List of Tables] 
%   \item [List of \meta{list name}] Other lists that can be found, such as list
%         of algorithms, listings etc.
%   \item [Element] An element is a particular component of a document that can 
%         be typeset individually, for example the title of a heading, 
%         for example in this section it is \emph{terminology}. This is an 
%         important concept used by the |phd| package following that of |html|. These
%         elements have |properties| which can be set using key value mark-up.
%         
%         The element name can be composed from different words to describe 
%         a particular document element. For example:
%
%         \begin{tabular}{lp{7cm}}
%           \texttt{toc}      & table of contents. This refers to ToC as a to level.\\
%           \texttt{toc name}           & The element name \\
%           \texttt{toc name} \meta{font-size} & The font size used to typeset the toc name.\\
%          \texttt{toc chapter page} \meta{font-size} & This refers to the font size of the page number of a chapter toc entry. \\  
%         \end{tabular} 
% \end{description}
%
% \section{Conventions}
%  The coding follows a strict set of conventions to enable automatic generation
%  of both code and keys.
%
% \paragraph{Key capitalization} All keys are defined using lower case letters. 
% This saves keystrokes and is easier to remember.
% 
% \paragraph{Spacing and hyphens} All element names use a space to divide the 
%       individual words. Property attributes that can be found in |CSS| are
%       written the same way as |css| with a hyphen, hence |font-size| and not
%       |font size|.
%
% \paragraph {Control sequences} We use the expl3 conventions for most of
%       control sequences. Keys stored in control sequences follow strictly
%       the naming convention of the key, but omitting any hyphens and replacing spaces with underscores. For example:
% 
% |toc chapter page font-size = \phd_toc_chapter_page_fontsize|
% 
%      I have only used the |_tl| su suffix only where the cs is manipulated. For the rest
% I have followed expl3 conventions regarding suffixes.
%      Macros are prefixed with |phd_| where appropriate.
%
% \section{General notes on the coding part}

%	Most of the macros here are a re-write the LaTeX macros in a way that 
%	we can add appropriate hooks for styling. In writing this section
%	we had inspiration and used liberally code from Peter Wilson's 
%	\pkg{tocloft}., including the code for the image.
%
% \newcommand{\maxx}{120}       ^^A picture width
% \newcommand{\maxxm}{118}      ^^A \maxx - 2\
% \newcommand{\maxy}{55}        ^^A picture height
% \newcommand{\maxym}{53}       ^^A \maxy - 2
% \newcommand{\findent}{20}     ^^A indent
% \newcommand{\findentp}{22}    ^^A \findent + 2
% \newcommand{\fnumwidth}{10}   ^^A numwidth
% \newcommand{\ftocrmarg}{30}   ^^A \@tocrmarg
% \newcommand{\fpnumwidth}{20}  ^^A \@pnumwidth
% \newcommand{\fipn}{30}        ^^A \findent + \fnumwidth
% \newcommand{\frmarg}{90}      ^^A \maxx - \ftocrmarg
% \newcommand{\frnum}{100}      ^^A \maxx - \fpnumwidth
% \newcommand{\fyi}{10}         ^^A 1st y height
% \newcommand{\fyim}{8}         ^^A \fyi - 2
% \newcommand{\fyii}{20}        ^^A 2nd y height
% \newcommand{\fyiii}{25}       ^^A 3rd y height
% \newcommand{\fyiv}{30}        ^^A 4th y height
% \newcommand{\fyv}{40}         ^^A 5th y height
% \newcommand{\fyvp}{42}        ^^A \fyv + 2
% \newcommand{\flin}{4}         ^^A length of leader lines
% \newcommand{\frmargm}{89}     ^^A \frmarg (90) - a little bit
% 
% \providecommand{\bs}{\textbackslash}
% \begin{figure}[bp]
% \centering
% \setlength{\unitlength}{1mm}
% \begin{picture}(\maxx,\maxy)
%     ^^A side lines and linewidth
%   \put(0,0){\line(0,1){\maxy}}
%   \put(\maxx,0){\line(0,1){\maxy}}
%   \put(0,\maxy){\vector(1,0){\maxx}}
%   \put(2,\maxym){\makebox(0,0)[tl]{\texttt{\bs linewidth}}}
%     ^^A \@pnumwidth
%   \put(\maxx,\fyi){\vector(-1,0){\fpnumwidth}}
%   \put(\maxxm,\fyim){\makebox(0,0)[tr]{\texttt{\bs @pnumwidth}}}
%   \put(\frnum,\fyi){\line(0,1){\flin}}
%     ^^A \@tocrmarg
%   \put(\maxx,\fyv){\vector(-1,0){\ftocrmarg}}
%   \put(\maxxm,\fyvp){\makebox(0,0)[br]{\texttt{\bs @tocrmarg}}}
%   \put(\frmarg,\fyv){\line(0,-1){\flin}}
%     ^^A indent
%   \put(0,\fyv){\vector(1,0){\findent}}
%   \put(2,\fyvp){\makebox(0,0)[bl]{\textit{toc margin-left}}}
%   \put(\findent,\fyv){\line(0,-1){\flin}}
%     ^^A numwidth
%   \put(\findent,\fyv){\vector(1,0){\fnumwidth}}
%   \put(\findentp,\fyvp){\makebox(0,0)[bl]{\textit{numwidth}}}
%   \put(\fipn,\fyv){\line(0,-1){\flin}}
%     ^^A last title line
%   \put(\maxx,\fyii){\makebox(0,0)[br]{487}}
%   \put(\fipn,\fyii){title end}
%     ^^A second title line
%   \put(\fipn,\fyiii){continue\ldots}
%   \put(\frmarg,\fyiii){\makebox(0,0)[br]{\ldots title}}
%     ^^A first title line
%   \put(\findent,\fyiv){\textbf{3.5}}
%   \put(\fipn,\fyiv){Heading\ldots}
%   \put(\frmarg,\fyiv){\makebox(0,0)[br]{\ldots title}}
%     ^^A dotted leader
%   \multiput(\frmargm,\fyii)(-\flin,0){12}{.}
%   \multiput(\frmarg,\fyi)(-\flin,0){2}{\line(0,1){\flin}}
%   \put(\frmarg,\fyi){\vector(-1,0){\flin}}
%   \put(\frmarg,\fyi){\vector(1,0){0}}
%   \put(\frmarg,\fyim){\makebox(0,0)[tr]{\texttt{\bs @dotsep}}}
% 
% \end{picture}
% \setlength{\unitlength}{1pt}
% \caption{Standard Layout of a ToC (LoF, LoT) entry} \label{fig:ltoc}
% \end{figure}
%
%    \begin{macrocode}
%
\NeedsTeXFormat{LaTeX2e}
\ProvidesPackage{phd-toc}%
  [2015/13/06 v1.0 ToC styling]%
%    \end{macrocode}

% 
% 
% We will be using either chapter or section type headings for the ToC, etc.,
% so we need to know which of these the document class supports.
%
%    \begin{macrocode}
\ExplSyntaxOn
\newif\if@haschapter@cx\@haschapter@cxtrue
\int_new:N \toc_depth
\int_gset:Nn \toc_depth {\c@tocdepth}
%
\bool_new:N \haschapter_bool \bool_gset_true:N \has_chapter_bool
\bool_new:N \haspart_bool \bool_gset_true:N \haspart_bool
%
\cs_if_exist:cTF {part} 
   { \bool_gset_true:N \haspart_bool   } 
   { \bool_gset_false:N \haspart_bool  }
%   
\ExplSyntaxOff
%    \end{macrocode}
% 
% 
% {if@koma@cx}
% The \pkg{koma} classes have different defaults than the standard classes,
% so we need to know if a \pkg{koma} class has been loaded.
%    \begin{macrocode}
\newif\if@koma@cx  \@koma@cxfalse
\@ifclassloaded{scrartcl}{\@koma@cxtrue}{}
\@ifclassloaded{scrreprt}{\@koma@cxtrue}{}
\@ifclassloaded{scrbook}{\@koma@cxtrue}{}
%    \end{macrocode}
% 
%
% {if@memoir@cx}
%    \begin{macrocode}
\newif\if@memoir@cx  \@memoir@cxfalse
\@ifclassloaded{memoir}{\@memoir@cxtrue}{}
%    \end{macrocode}
% 
%
% Issue a warning if there are no recognised sectional divisions 
% and then skip the rest of the package code.
%    \begin{macrocode}
\@ifundefined{chapter}{%
  \@haschapter@cxfalse
  \@ifundefined{section}{%
    \PackageWarning{phd}%
      {I don't recognize any sectional divisions so I'll do very little 
      and many things can break}
    \renewcommand{\quit@cx}{\endinput}
    }{\PackageInfo{phd}{The document has section divisions}}
  }{\@haschapter@cxtrue
    \PackageInfo{phd}{The document has chapter divisions}}
%    \end{macrocode}
% bailing out or continue.
%
% 
%	We define a user macro and to be used in keys
%   a pagestyle for the first page of the ToC.
%   The default is the |plain| pagestyle. CHECK THIS.
%    \begin{macrocode}
\newcommand{\settocpagestyle}[1]{%
  \def\tocpagestyle@cx{\thispagestyle{empty}}} %CHANGED
 
%    \end{macrocode}
% 
% 
%
% {tocparskip@cx}
% The |\parskip| local to the ToC, etc., is set to the length |\tocparskip@cx|.
%
%    \begin{macrocode}
\newlength{\tocparskip@cx}
\setlength{\tocparskip@cx}{1pt}
%    \end{macrocode}
% 
%
% \section{General Formatters}
%
% In order to provide maximum flexibility and to re-use code, we provide general
% formatting code. These in general start with the prefix \emph{format}. They take
% as input the basic parameters, required to format. The decoration parameters come
% from the key value interface.
%  
% \begin{docCommand}{format_toc_name:n} { \marg{prefix} }
%   Formats and typesets the conents name, in a ToC. 
% \end{docCommand}
%
%    \begin{macrocode}
\ExplSyntaxOn

\newtcolorbox {float_box} [1]
  { 
    %size            = minimal,  
    enhanced,
    colback         = \cs_if_exist_use:cTF {#1_background_color}{}{blue},
    colframe        = \cs_if_exist_use:cTF {#1_frame_color}{}{blue},
    rounded~corners = all,
    arc=3mm,
    auto~outer~arc,
    boxsep          = 3pt,
    \cs_if_exist_use:cTF {#1_shadow}{}{no~shadow},
  } 
  
\cs_new:Npn \set_font_aux:n #1
 {
   \cs_if_exist_use:cTF { #1_fontfamily }{}{#1~Family}
   \cs_if_exist_use:cTF { #1_fontweight }{}{#1~W}
   \cs_if_exist_use:cTF { #1_fontshape  }{}{#1~Sh}
   \cs_if_exist_use:cTF { #1_fontsize   }{}{#1~Siz}
 }  
  
\cs_gset:Npn \format_toc_name:n #1 
  {
 \begin{float_box}{toc_name}
      \leavevmode
      \cs_if_exist_use:cTF {#1_align}{}{F~#1}
      
      %\toc_name_indent 
      \tcbox[size            = minimal,
             nobeforeafter,
             colback         = \cs_if_exist_use:cTF {#1_background_color}{}{blue},
             colframe        = \cs_if_exist_use:cTF {#1_frame_color}{}{blue},
             ]
        {
               \toc_name_before
               \set_font_aux:n {toc_name}
               %\color{\toc_name_color}
               \cs_if_exist_use:cTF {#1_case}{}{F #1}
               {
                 \cs_if_exist_use:cTF {#1_contentsname}{}{F~#1}
                }
        }
       
      
     \end{float_box} 
    \toc_name_after%
 }
\ExplSyntaxOff
%    \end{macrocode}
%  \begin{docCommand} {phd_toc_start} { \meta{void}}
%    Typesets any material before the toc, for example a rule or image. This
%    can also be used to typeset a two column or three column toc.
%  \end{docCommand}
%
%  \begin{docCommand}{phd_toc_finish} { \meta{void}}
%    Typesets any material after the toc, for example a rule or image. This
%    can also be used to typeset end a two column or three column layout.
%  \end{docCommand} 
%
%    \begin{macrocode}
\let\ltxtableofcontents\tableofcontents
%
\ExplSyntaxOn
\cs_new:Npn \phd_toc_start:  { }
\cs_new:Npn \phd_toc_finish: { } 
\ExplSyntaxOff
%    \end{macrocode} 
% 
% \begin{docCommand}{tableofcontents} {\meta{void}}
%  This is a parameterised version of the default |\tableofcontents| command.
%  Each class has its own definition, but we have to cater for all classes
%  in one definition, hence some of the checks. The definition is
%  modified after all packages have been loaded. The normal LaTeX way is to use
%  the chapter to set it in the book class and the section in others. Here we opted to
%  leave it up to the user.
%	 Consider more checks here
% \end{docCommand}
%
%    \begin{macrocode}
\ExplSyntaxOn
 \renewcommand{\tableofcontents}{%
    \phd_toc_start:
    \bgroup
   % \hypersetup{linkcolor=spot}
%    \end{macrocode}
% Ensure that any previous paratgraph has been finished. 
%	within a group set
% the local paragraphing style and typeset the title. \label{code:tableofcontents}
%    \begin{macrocode}
    \par
    \begingroup
      \parindent\z@ 
      \parskip\tocparskip@cx
      \phd_make_toc_title:n {toc_name}
%    \end{macrocode}
%
% Finally, start reading the \docfile{.toc} file and finish up.
%    \begin{macrocode}
    \start_toc:n {toc}%
    \endgroup
    \egroup
    \phd_toc_finish:
}%
\ExplSyntaxOff
%    \end{macrocode}
%
%  \begin{docCommand} {start_toc} { \meta {void}}
%    Reads the file |.toc|. Write to the file conditionally. This was
%    originally provided in the source2e class |lsect|, which we redefine.\FIRE
%  \end{docCommand}
%
%    \begin{macrocode}
\ExplSyntaxOn
 \cs_new:Npn \start_toc:n #1 
   {
     \group_begin:
     \makeatletter
     \@input{\jobname.#1}%
     \if@filesw
       \expandafter\newwrite\csname tf@#1\endcsname
       \immediate\openout \csname tf@#1\endcsname \jobname.#1\relax
     \fi
     \@nobreakfalse
     \group_end:
  }
\ExplSyntaxOff 
%    \end{macrocode}
% 
% \begin{docCommand} {numberline} { \meta {the number} }
%  The purpose of the |\numberline{|\meta{secnum}|}| command is to typeset
%  \meta{secnum} left justified in a box of width |\@tempdima|. I redefine
%  it to add three additional parameters, namely |\toc_number_before|, 
%  |\toc_number_after| and |\toc_number_after_box| 
%  (see \docfile{ltsect.dtx} for the original definition).\FIRE
%
% \begin{verbatim}
%   \contentsline {section}
%      {\numberline {4}Language Manager}
%      {10}{section.1.4}
% \end{verbatim} 
% \end{docCommand}
%
%    \begin{macrocode}
\ExplSyntaxOn
\cs_set:Npn \toc_number_before_box {}
\cs_set:Npn \toc_number_before {}
\cs_set:Npn \toc_number_after {}
\cs_set:Npn \toc_number_after_box {}
%    \end{macrocode}
%
% This is the most important part of all. It is saved in the .toc
% and when we use l@\meta{chapter} etc it is used for formatting. No
% separation of concerns here.
%
%    \begin{macrocode}  
\dim_new:N \numberlineboxwidth  
\cs_gset:Npn \numberline #1
  {
   \toc_number_before_box
   \hbox_to_wd:nn \numberlineboxwidth                   %\numberlineboxwidth  
     { 
       \toc_number_before\relax #1 \toc_number_after \hfil 
     }
   \toc_number_after_box
  }
\ExplSyntaxOff
%    \end{macrocode}
% 
% \section{ToC Name Parameters}
%
%  The ToC name keys are intended for styling the top part of the ToC.
%
%  \begin{tcolorbox}[colframe=black,colback=white]
%  \hfill \tcbox[size=minimal,colback=white,
%               nobeforeafter]{\bfseries\LARGE Contents}
%  \end{tcolorbox}
%    \begin{macrocode}
\ExplSyntaxOn
\pgfkeys
 {/handlers/.case/.code = 
    \pgfkeysalso
      {\pgfkeyscurrentpath/.code=
         \str_case_x:nnTF {##1}  
             {
               { none       } { \cs_gset:cpn {#1} { \empty             } } 
               { lower      } { \cs_gset:cpn {#1} { \MakeTextLowercase } } 
               { lowercase  } { \cs_gset:cpn {#1} { \MakeTextLowercase } } 
               { upper      } { \cs_gset:cpn {#1} { \MakeTextUppercase } } 
               { uppercase  } { \cs_gset:cpn {#1} { \MakeTextUppercase } } 
               { upper~case } { \cs_gset:cpn {#1} { \MakeTextUppercase } } 
             }
             {                         }
             { \cs_gset:cpn #1 {\MakeTextLowercase} }
      }
  }
\ExplSyntaxOff  
%    \end{macrocode}
% \section{Shadows}
% 
% Many components can be rendered with shadows. This can be done through the
% tcolorbox shadowing commands or directly through \tikzname. The way parameters
% are specified in both cases results in multi-argument keys, which is generally
% against the philosophy of the mark-up semantics of the |phd| package. Most
% of these keys required color specification and or size specification.
% 
% Since colors are linked to palettes, I decided that the color part would belong to
% the palette settings rather than the shadow keys. Also any sizing of shadows has been
% delegated to default macros. This simplifies the user interface tremendously. 
% Should a template designer wish to provide a more complicated shadow, this can be
% achieved through the style property of the element.
%
% Shadows are only available when the box rendering engine depend on tcolorbox. 
%
%    \begin{macrocode}  
\tcbset{halostyle/.style={fuzzy halo=2mm with magenta!5}}
\ExplSyntaxOn
 \pgfkeys
 {/handlers/.shadow/.code = 
    \pgfkeysalso
      {\pgfkeyscurrentpath/.code=
         \str_case_x:nnTF {##1}  
             {
               { none       }   { \cs_gset:cpn {#1} { {no~shadow}       } } 
               { drop~shadow  } { \cs_gset:cpn {#1} { drop~shadow       } }
               { drop~lifted~shadow  } { \cs_gset:cpn {#1} {{##1         }} } 
               { fuzzy~halo}           { \cs_gset:cpn {#1} {halostyle } }
             }
             {                         }
             { \cs_gset:cpn #1 {##1} }
      }
  } 

 \pgfkeys
 {/handlers/.store/.code = 
    \pgfkeysalso
      {\pgfkeyscurrentpath/.code=
         \cs_gset:cpn {#1} {##1}
      }
  }   
%    \end{macrocode}
%  \begin{docCommand}{ new_toc_keys:nn } {}
%    ToC, LoF, ToF etc, follow the conventions of the |phd| package in declaring
%    basic elements using prefixes, normally with two names, for example
%    |toc name|\meta{field} defines properties for the name used on top of
%    contents pages.
%  \end{docCommand}
%    \begin{macrocode}
\cs_new:Npn \new_toc_keys #1 #2
{           
\cxset 
  {
    #1/.store                                   = #2_contentsname,
    #1~before/.store                            = #2_before,
    #1~after/.store                             = #2_after,
%    
    #1~font-size/.fontsize                      = #2_fontsize,
    #1~font-weight/.fontweight                  = #2_fontweight,
    #1~font-family/.fontfamily                  = #2_fontfamily,
    #1~font-shape/.fontstyle                    = #2_fontshape,
%    
    #1~color/.store                             = #2_color,
    #1~background-color/.store                  = #2_background_color,
    #1~frame-color/.store                       = #2_frame_color,
%    
    #1~shadow/.shadow                           = #2_shadow,
%    
    #1~case/.case                               = #2_case,
%    
    #1~afterskip/.store                         = #2_after_skip,
    #1~align/.textalign                         = #2_align,
    
    #1~indent/.store                            = #2_indent,
%    
    toc~pagestyle/.code                               =
      \gdef\contentspagestyle@cx{\thispagestyle{empty}},%
}
 
}

\ExplSyntaxOff     
%    \end{macrocode}      
%
% The contents page is enabled to have its own pagestyle. We default this later on
% to plain.
% This needs also a bit of a thought, if we require to enable it further down the line.
%
%    \begin{macrocode}
\ExplSyntaxOn
\cs_set:Npn \toc_set_key_defaults #1 #2
  {
    \cxset{
       #1                   = #2,
       #1~before            =,
       #1~after             =, 
       #1~color             = black,
       #1~background-color  = magenta!15,
       #1~frame-color       = white,
       #1~shadow            = drop shadow,       
       #1~font-weight       = normal,
       #1~font-family       = sffamily,
       #1~font-shape        = upshape,
       #1~font-size         = LARGE,
       #1~afterskip         = 10pt, %set as 40pt in LaTeX
       #1~after        = ,
       #1~align        = left,
       #1~indent       = ,
       #1~case         = none,
       toc~pagestyle         = empty,
  }%
}  
\ExplSyntaxOff
%    \end{macrocode}
%  Next we create key sets for all three default lists, ToC, LoF and LoT.
%  
%    \begin{macrocode}
\ExplSyntaxOn
\new_toc_keys {toc~name}{toc_name}
\toc_set_key_defaults {toc~name}{Table~of~Contents}

% list of figures
\new_toc_keys {lof~name}{lof_name}
\toc_set_key_defaults {lof~name}{List~of~Figures}

% list of tables
\new_toc_keys {lot~name}{lot_name}
\toc_set_key_defaults {lot~name}{List~of~Tables}
\ExplSyntaxOff
%    \end{macrocode}
%
% \begin{docCommand}{phd_make_toc_title:n } { \meta{prefix} }
%	Typesets the heading that goes on top of the Contents page.
%	The prefix is the code prefix i.e, |toc_name| or |lof_name| etc. 
% \end{docCommand}
%
%    \begin{macrocode}
\ExplSyntaxOn

\cs_new:Npn \phd_make_toc_title:n #1 
{
  \addpenalty\@secpenalty
  \if@haschapter@cx
    \vspace*{10pt}
    \pdfbookmark[0]{\contentsname}{toc}
  \else
    \vspace{10pt}
  \fi
  \markboth{\contentsname}{\contentsname}%
  %\contentspagestyle@cx CHECK THIS
  \interlinepenalty\@M
%    \end{macrocode}
%  Next we call the appropriate renderer.
%    \begin{macrocode}  
  \format_toc_name:n {#1}
%    \end{macrocode}
%    \begin{macrocode}  
    \par\nobreak
    \vskip\toc_name_after_skip\relax
    \@afterheading
 }
\ExplSyntaxOff 
%    \end{macrocode}
% Next we create some demonstration code to see that is all well.
%    \begin{macrocode}     
\ExplSyntaxOn
  \let\sampletoctitle\phd_make_toc_title:n
\ExplSyntaxOff 
%    \end{macrocode}
% 
%  \ExplSyntaxOn
%
%  \sampletoctitle {toc_name}
%  \sampletoctitle {lof_name}
%  \sampletoctitle {lot_name}
%
% \ExplSyntaxOff 
%
% {setpnumwidth@cx}
% {setocmarg@cx}
%  Users commands for setting |\@pnumwidth| and |\@tocrmarg|.
%    \begin{macrocode}
\newcommand{\setpnumwidth@cx}[1]{\renewcommand{\@pnumwidth}{#1}}
\newcommand{\settocmarg@cx}[1]{\renewcommand{\@tocrmarg}{#1}}
\setpnumwidth@cx{25pt}
\settocmarg@cx{20pt}
%    \end{macrocode}
% 
% 
%
% \section{Styling the dot leaders}
%  	Here we will allow the user to either have dotfills and
%    be	able to specify the type and spacing of the dots.
%	We also provide a key to disable dotfills.
%
% \begin{docCommand} {dot@cx} { \meta{void}}
%   Stores the leaders pattern. In the standard classes this
%   is normally a dot.
% \end{docCommand}
%
% \begin{docCommand} {dotfill@cx} { \meta{void}}
%   Typesets the leaders based on the pattern stored in \#1
% \end{docCommand}
%
%   In the default |ToC|, a dotted line can be used to provide a leader between
%   a title and the page number. As Peter Wilson wrote and I found at my
%   distress the definition of the leader is buried
%   in the \cs{@dottedtocline} command. The 
%	\cs{dotfill@cx}\marg{sep}
%   command provides a parameterised version of the leader code, where
%   \marg{sep} is the seperation between the dots in mu units.
%   The symbol used for the `dots' in the leader is given by the 
%   value  of |\dot@cx|. 
% 
%    \begin{macrocode}
\ExplSyntaxOn
\cs_gset:Npn \dot@cx { - }
\cs_gset:Npn \dot_fill #1 
  {
    \leaders\hbox{$\m@th\mkern #1 mu\hbox{\dot@cx}\mkern #1 mu$}\hfill
  }
\ExplSyntaxOff  
%    \end{macrocode}
% 
% 
%\parskip=1pt plus0.2pt minus0.2pt
%    \begin{macrocode}
\def\nodotfill@cx{}
\cxset{toc dotfill/.is choice,
       toc dotfill/none/.code = \nodotfill@cx,
       toc dotfill symbol/.code= \renewcommand{\dot@cx}{#1},
       toc dotfill sep/.store in=\dotfillsep@cx,
}
\cxset{toc dotfill symbol=.,
       toc dotfill sep=4.5}
%    \end{macrocode}
%
% 
% The |\l@kind| commands modify (locally) the value of |\parfillskip|.
% |\parfillskip@CX| is a copy of the default \texbook\ 
% |\parfillskip| definition.
%    \begin{macrocode}
\ExplSyntaxOn
\newcommand{\parfillskip@CX}
  {
  \parfillskip=0pt plus1fil
  }
\ExplSyntaxOff  
%    \end{macrocode}
%
% 
%    \begin{macrocode}
\ExplSyntaxOn
\cs_set:Npn \format_toc_entry:nn #1 #2 #3
  {
  
  %     \cs_set:Npn \tocindent 
%     {
         %\cs_if_exist_use:cTF {#1_indent}{TI}{0pt} 
         
%     }
 \begingroup  
 \hypersetup{linkcolor=magenta,
             }
 \begin{tcolorbox}[size=minimal,
                   colback=white!10,
                   width=\linewidth-1em]   
     \expandafter\leftskip \cs:w #1_indent\cs_end:
     \renewcommand\@tocrmarg{3em}
     \dim_set_eq:NN \tex_rightskip:D \@tocrmarg
     \parfillskip -\rightskip
     \dim_set_eq:NN \parindent {1em} %\tocindent
%    
     \interlinepenalty\@M
     \leavevmode
     \numberlineboxwidth 4.2em %\cs:w #1_number_width \cs_end:\relax
     \let\toc_number_before \cftsecpresnum
     \advance\leftskip \numberlineboxwidth
     \null\nobreak\hskip -\leftskip
    
%       \tcbox[size=minimal,
%         nobeforeafter,
%         before=\hspace{0em},
%         colback=white,
%         box~align=base,
%        ]
         {
            \set_font_aux:n  {#1}
           % \cs_if_exist_use:cTF {#1_color}{T #1}{F #1}
            \expandafter\color{\cs:w #1_color\cs_end:}
            \cs_if_exist_use:cTF {#1_case}  {#2}{#2}
         }
       \cs_if_exist_use:cTF {toc_section_leader}{} {}
%    \end{macrocode}
% Next we take care of the page box. This is again a box within a 
% box arrangement as in most case we need to float the page numbers to the right.
% Think page 1 and page 999.  
%    \begin{macrocode}       
       \begin{tcolorbox}
       [
          size=minimal,
          nobeforeafter,
          colback=\toc_section_page_background_color,
          width=\toc_section_number_width-0.5em,
          halign=center,
          box~align = base,
          boxsep    = 3pt, 
       ]
       \tcbox
        [
          size=minimal,
          nobeforeafter,
          width=\toc_section_number_width-0.5em,
          colback=\toc_section_page_background_color,
          box~align = base,
        ]
         {
         %\mbox{
         \set_font_aux:n  {#1_page}
         \toc_section_page_before #3 \toc_section_page_after
         }
       \end{tcolorbox}  
     \par
    \end{tcolorbox} 
    \par  
    \group_end:
     \@afterindentfalse
     \nobreak
 }
  \ExplSyntaxOff
%    \end{macrocode}
%
% \section{Automating key generation}
%
%     \begin{macrocode}
\ExplSyntaxOn 
\cs_new:Npn \make_new_toc_entry_keys #1 #2
{
 \cxset
  {
    #1~beforeskip/.store            = #2_beforeskip,
    #1~afterskip/.store             = #2_afterskip,    
    #1~indent/.store                = #2_indent,
% fonts 
    #1~font-size/.fontsize          = #2_fontsize, 
    #1~font-family/.fontfamily      = #2_fontfamily, 
    #1~font-shape/.fontstyle        = #2_fontshape, 
    #1~font-weight/.fontweight      = #2_fontweight, 
    #1~color/.store                 = #2_color,
    #1~numwidth/.store              = #2_number_width,
    #1~case/.case                   = #2_case,
    #1~page~font-size/.fontsize     = #2_page_fontsize,
    #1~page~font-family/.fontfamily = #2_page_fontfamily,
    #1~page~font-shape/.fontstyle   = #2_page_fontshape,
    #1~page~font-weight/.fontweight = #2_page_fontweight,
    #1~page~color/.store            = #2_page_color,
    #1~page~background-color/.store = #2_page_background_color,    
%   leaders template only
	  #1~dotsep/.store                = #2_dotsep,
%   before and after page number
    #1~page~before/.store           = #2_page_before,
    #1~page~after/.store            = #2_page_after,
  }
}
%    \end{macrocode}
%
% Next we provide a function that sets keys to preset defaults. This makes
% it less cumbersome to modify later.
%
%    \begin{macrocode}
\cs_new:Npn \make_new_toc_entry_key_defaults #1 
  {
    \cxset
    {%
      #1~beforeskip  =\z@ \@plus.2\p@,
      #1~afterskip   =0pt,      
      #1~indent=0em,
      #1~font-family= sffamily,
      #1~font-weight = bfseries,
      #1~font-shape = upshape,
      #1~color= spot,
      #1~case = none,
      #1~font-size= normal,
      #1~numwidth = 3em,
      #1~dotsep = 2.7,
%    \end{macrocode}
%
% The page number is a child of the |<toc><element>| and is provided to
% as page, it has both box as well as textual properties. Settings
% are what I thought were typical of documents.
%
%    \begin{macrocode}      
%page parameters      
      #1~page~font-size          = normal,
      #1~page~font-shape         = upshape,  
      #1~page~font-weight        =,  
      #1~page~font-family        = sffamily,
      #1~page~color              = black, 
      #1~page~background-color   = spot!15,       
      #1~page~before             =,
      #1~page~after              =,
    }
  }
\ExplSyntaxOff  
%    \end{macrocode} 
% 
% \subsection{Styling Part in the Toc}
%  |\l@part{|\meta{title}|}{|\meta{page}|}| typesets the ToC entry for
% a |part| heading. It is a parameterised copy of the default |\l@part|
% (see \docfile{classes.dtx} for the original definition and the code
%  below for |\l@part| for an explanation of most of this
%  code). 
%
% By default, Parts
% (and Chapters) do not have dotted leaders. This package provides
% for all entries to have the ability to have dotted leaders, as some styles treat the part in a similar manner.
%
% In article class, Part level is 0 not -1 and hence the conditional below.
%	
%	We start by defining a number of keys and macros to store parameters.
%	An entry to the ToC consists always of a number, the title and 
%	a page number. For each part there are different styling keys.
%
%	{tocpartindent@cx}	 
%    \begin{macrocode}
\ExplSyntaxOn
\make_new_toc_entry_keys {toc~part}{toc_part}  
\make_new_toc_entry_key_defaults {toc~part}
\ExplSyntaxOff
\cxset{toc part font-size=LARGE,
       toc part color=spot,
       toc part beforeskip=1em}
%    \end{macrocode}
%
%
% \begin{docCommand} {l@part} { \meta{title} \meta{pagenumber} }
% We now renew the command, in order to allow for hooks. 
% This might be cloberred by hyperref if too many changes
% are carried out. It takes two parameters (one for the caption and another for the title if different).
% \end{docCommand}
%
% We need first to define conditionals to switch from
% printing the part or not.
%
%  We allow for any document type to have parts, as long as the control sequence |\part|
%  is defined.
%  The part |toc_level = -1|
%    \begin{macrocode}
\ExplSyntaxOn
\newif\if@dopart@cx
\newif\if@haspart@cx
  \@ifundefined{part}{\@haspart@cxfalse}{\@haspart@cxtrue}
\if@haspart@cx
%    \end{macrocode}
% I am not too sure about the need of all the penalties, but will review
% this part of the code before the final distribution.
%
%    \begin{macrocode}
\cs_gset:Npn \l@part #1 #2
  {
    \bool_if:NT \haspart_bool
      {
        \bool_if:NTF \has_chapter_bool 
          {
            \addpenalty{-\@highpenalty}
          }
          { 
            \addpenalty\@secpenalty 
          }
          \addvspace{\toc_part_beforeskip}%
          \format_toc_entry:nn {toc_part} {#1} {#2}
           \nobreak
           \bool_if:NT \has_chapter_bool
             {
               \global\@nobreaktrue
               \tex_everypar:D { \global\@nobreakfalse\tex_everypar:D {} }%
 	           }
           %\group_end:
    }
   % \format_toc_entry:nn {toc_part} {#1} {#2}
  } %end l@part

\ExplSyntaxOff
%    \end{macrocode}
%  These are the user commands to control the typesetting of Part entries.
%  They are initialised to give the standard appearance.
%    \begin{macrocode}
\ExplSyntaxOn
  \newcommand{\partpresnum@cx}{}
  \newcommand{\partaftersnum@cx}{..}
  \def\cftnodots{2.7}
  \newcommand{\partdotsep@cx}{\cftnodots}
  \newcommand{\toc_part_leader}{\large\bfseries\dot_fill{\partdotsep@cx}}
\ExplSyntaxOff       
%    \end{macrocode}
% \section{Handling of chapters in ToC.}
%
%  These are the user commands to control the typesetting of Chapter entries.
%  They are initialised to give the standard appearance.
%    \begin{macrocode}
%\if@haschapter@cx
  \newlength{\beforetocchapterskip@cx}
  \setlength{\beforetocchapterskip@cx}{1.0em \@plus\p@}
  \newlength{\cftchapindent}
  \setlength{\cftchapindent}{0em}
  \newlength{\cftchapnumwidth}\setlength{\cftchapnumwidth}{1.5em}
  \newcommand{\cftchapfont}{\bfseries}
  \newcommand{\cftchappresnum}{}
  \newcommand{\cftchapaftersnum}{}
  \newcommand{\cftchapaftersnumb}{}
  \newcommand{\cftchapleader}{\dot_fill{\cftchapdotsep}}
%    \end{macrocode}
%
%	The following code determines the spacing of the dots.
%    \begin{macrocode}
  \newcommand{\cftchapdotsep}{\toc_chapter_dot_sep} 
  \newcommand{\cftchappagefont}{\sffamily\bfseries\color{teal}}
  \newcommand{\cftchapafterpnum}{}
%
%    \end{macrocode}
%
% \subsection{l@chapter}
%
% \begin{docCommand} {l@chapter} { }
%  \cs{l@chapter}\marg{title}\marg{page} typesets the ToC entry for
% a |chapter| heading. It is a parameterised copy of the default |\l@chapter|
%  (see \docfile{classes.dtx} for the original definition). This only applies
%  to chaptered documents.
% \end{docCommand}
%
%    \begin{macrocode}
\ExplSyntaxOn
  \make_new_toc_entry_keys {toc~chapter}{toc_chapter}  
  \make_new_toc_entry_key_defaults {toc~chapter}
\ExplSyntaxOff
\cxset{toc chapter font-size=LARGE,
       toc chapter color=spot,
       toc chapter beforeskip=1em,
       toc chapter afterskip = 1em plus0.2pt minus .2pt}
%    \end{macrocode}
%
%    \begin{macrocode}
\ExplSyntaxOn
  \newcommand{\toc_chapter_leader}{\normalfont\dot_fill{\toc_section_dotsep}}
\ExplSyntaxOff
%    \end{macrocode}  
%    \begin{macrocode}  
\ExplSyntaxOn
  \renewcommand*{\l@chapter}[2]{%
     \ifnum \c@tocdepth >\m@ne
       \addpenalty{-\@highpenalty}%
       \vskip \toc_chapter_beforeskip\relax 
        {
         \format_toc_entry:nn {toc_chapter}{#1}{#2}
        }%
       \vskip \toc_chapter_afterskip 
    \fi
 }%
\ExplSyntaxOff 
%    \end{macrocode}
%
% We define a macro for mocking sample toc chapters for the documentation
% 
% \begin{docCommand} {sampletocchapter} {\meta{void}}
%   Typesets  a sample based on l@chapter
% \end{docCommand}
%    \begin{macrocode}
\let\sampletocchapter\l@chapter
%    \end{macrocode}
% 
%
% \section{ToC section styling}
%  
%     \begin{macrocode}
\ExplSyntaxOn 
  \make_new_toc_entry_keys {toc~section}{toc_section}  
  \make_new_toc_entry_key_defaults {toc~section}
\ExplSyntaxOff
%    \end{macrocode}
%
% \begin{docCommand} { l@section } { \meta{title} } { page number }
% 	 This macro is called when the \cs{tableofcontents}
%	 is read from the |.toc| file and it typesets
%	 the title and the page number. It is called in a |csname| by |\contentsline|
% 
%    \#1 section title\\
%    \#2 page number\\
%    \#3 added by Hyperref
%      
% \end{docCommand}
%    \begin{macrocode}
\ExplSyntaxOn
\cs_gset:Npn \l@section #1 #2
  {
  \ifnum \c@tocdepth >\z@
    \if@haschapter@cx
      \skip_vertical:n \toc_section_beforeskip
    \else
      \addpenalty \@secpenalty
      \addvspace{\toc_section_beforeskip}%
    \fi
    \format_toc_entry:nn {toc_section} {#1} {#2}
  \fi
  }
\ExplSyntaxOff  
%    \end{macrocode}
% 
%
%  These are the user commands to control the typesetting 
%	 of Section entries.
%    They are initialised to give the standard appearance.
%	 These are hooks to \cs{numberline}.
%    \begin{macrocode}
\ExplSyntaxOn
\newcommand{\cftsecpresnum}{}
\newcommand{\cftsecaftersnum}{}
\newcommand{\cftsecaftersnumb}{}
%
%
\newcommand{\toc_section_leader}  {\normalfont\dot_fill{\toc_section_dotsep}}
%^^A\newcommand{\cftsecdotsep}{\cftdotsep}
%    \end{macrocode}
%    We can now define the command \cmd{\tocsectionfillnum@cx}. 
%    will print the 
%	leaders if any and the page number \#1. 
%    \begin{macrocode}
%\newcommand{\tocsectionfillnum@cx}[1]{%
%  {\tocsectionleader@cx}\nobreak
%  \hb@xt@\@pnumwidth{\hfil\tocsectionpagefont@cx
%   \toc_section_page_before #1}%
%   \tocsectionpageafter@cx\par}%
\ExplSyntaxOff
%    \end{macrocode}
%
%
% \section{Toc subsection styling}
%
% 
%  \refCom{l@subsection} typesets the ToC entry for
% a |section| heading. It is similar to   \refCom{l@section}
% (see \docfile{classes.dtx} for the original definition). 
% 	We start by defining all our parameters and variables.
%
%    \begin{macrocode}
\ExplSyntaxOn
\make_new_toc_entry_keys {toc~subsection}{toc_subsection}  
\make_new_toc_entry_key_defaults {toc~subsection}
\ExplSyntaxOff 
%    \end{macrocode}
% Settings
%
% \begin{docCommand}{l@subsection} { \meta{title} \meta{page number} }
%  Similar to \refCom{l@section} function.
% \end{docCommand}
%
%    \begin{macrocode}        
\ExplSyntaxOn
\cs_gset:Npn \l@subsection #1 #2 
  {%
    \ifnum \c@tocdepth >\z@
      \if@haschapter@cx
        \skip_vertical:N \toc_subsection_beforeskip
      \else
        \addpenalty \@secpenalty
        \addvspace{\toc_subsection_beforeskip}%
    \fi
      \format_toc_entry:nn {toc_subsection} {#1} {#2}
  \fi
}
 \ExplSyntaxOff 
%    \end{macrocode}
% 
%

%    \begin{macrocode}
\ExplSyntaxOn
\cs_new:Npn \toc_subsection_leader 
  {
  \normalfont\dot_fill{\toc_subsection_dot_sep}
  }
\ExplSyntaxOff    
%    \end{macrocode}
%
% \section{Toc subsubsection styling}
%  Next the toc subsubsection properties. 
%    \begin{macrocode}
\ExplSyntaxOn
\make_new_toc_entry_keys {toc~subsubsection}{toc_subsubsection}  
\make_new_toc_entry_key_defaults {toc~subsubsection}
\ExplSyntaxOff
%
%    \end{macrocode}
%
%    For convenience we define font setting commands for
%    the page number. We use \cs{setfont@cx}, which we have
%	defined earlier. Note this might be clobbered if 
%  hyperref is to provide a page link.
%    
%    \begin{macrocode}
\newcommand\tocsubsubsectionpagefont@cx{%
	\setfont@cx{\toc_subsubsection_page_fontweight}%
       {\toc_subsubsection_page_fontfamily}{\toc_subsubsection_page_font_size}%
       {\toc_subsubsection_page_fontshape}\color{\toc_subsubsection_page_color}
}%
%    \end{macrocode}
%
% \begin{docCommand} {l@subsubsection} { \meta{title} \meta{page number}} 
%  typesets the ToC entry for
% a |subsubsection| heading. It is a parameterised copy of the default |\l@subsubsection|
%	We start by defining all our parameters and variables.
% \end{docCommand} 
%       
%    \begin{macrocode}
\ExplSyntaxOn
\cs_gset:Npn \l@subsubsection #1 #2
{%
  \ifnum \c@tocdepth >\z@
    \if@haschapter@cx
      \vskip \toc_subsubsection_beforeskip
    \else
      \addpenalty \@secpenalty
      \addvspace{\toc_subsubsection_beforeskip}%
    \fi
%    {\leftskip \toc_subsection_indent\relax
%     \rightskip \@tocrmarg
%     \parfillskip -\rightskip
%     \parindent \toc_subsection_indent\relax\@afterindenttrue
%     \interlinepenalty\@M
%     \leavevmode
%     \@tempdima \tocsubsubsectionnumwidth@cx\relax
%     \let\toc_number_before \cftsecpresnum
%     \let\toc_number_after \cftsecaftersnum
%     \let\toc_number_after_box \cftsecaftersnumb
%     \advance\leftskip \@tempdima \null\nobreak\hskip -\leftskip

  \format_toc_entry:nn {toc_subsubsection} {#1} {#2}
  \fi
}
\ExplSyntaxOff
%    \end{macrocode}
%
% 
%    \begin{macrocode}
\ExplSyntaxOn
 \cs_new:Npn \toc_subsubsection_leader 
   {
     \normalfont\dot_fill{\toc_subsubsection_dot_sep}
   }
\ExplSyntaxOff
%\newcommand{\cftsecdotsep}{\cftdotsep}
%    \end{macrocode}
%    We can now define the command \cmd{\tocsectionfillnum}. It will print the 
%	leaders if any and the page number \#1. TODO IS par necessary??
%    \begin{macrocode}
\newcommand{\tocsubsubsectionfillnum}[1]{%
  {\tocsubsubsectionleader}\nobreak
  \hb@xt@\@pnumwidth{\hfil\toc_subsubsection_page_font
   \toc_subsubsection_page_before #1}%
   \toc_subsubsection_page_after\par}%
%    \end{macrocode}
%
% \section{Toc paragraph styling}
%
% 
% Similarly to the higher headings \refCom{l@subsubsection} typesets the ToC entry for
% a \emph{paragraph} heading.	We start by defining  parameters and variables.
%
%    \begin{macrocode}
\ExplSyntaxOn
\make_new_toc_entry_keys {toc~paragraph}{toc_paragraph}  
\make_new_toc_entry_key_defaults {toc~paragraph}

\ExplSyntaxOff
%    \end{macrocode}
% We set default values.
%    For convenience we define font setting commands for
%    the page number. We use \cs{setfont@cx}, which we have
%	defined earlier.
%    
% \begin{docCommand} {l@paragraph} { \meta{title} \meta{page number}}
% \end{docCommand}
%
%    \begin{macrocode}
\ExplSyntaxOn    
\renewcommand*{\l@paragraph}[2]{%
  \ifnum \c@tocdepth >\z@
    \bool_if:NTF \has_chapter_bool
      {
        \skip_vertical:N \toc_paragraph_beforeskip
      }
      {
        \addpenalty \@secpenalty
        \addvspace{\toc_paragraph_beforeskip}%
      }
     \format_toc_entry:nn {toc_paragraph} {#1} {#2}
  \fi
  }
\ExplSyntaxOff  
%    \end{macrocode}
%
%
%    \begin{macrocode}
\ExplSyntaxOn
\newcommand{\toc_paragraph_leader}{\normalfont\dot_fill{\tocparagraphdotsep@cx}}
\ExplSyntaxOff
%\newcommand{\cftsecdotsep}{\cftdotsep}
%    \end{macrocode}
%
% \section{Toc subparagraph styling}
%
% \tcbdocmarginnote{U 20-6-2015}
% 
% Similarly to the higher headings \refCom{l@subsubsection} typesets the ToC entry for
% a \emph{subparagraph} heading.	
%
%    \begin{macrocode}
\ExplSyntaxOn
  \make_new_toc_entry_keys {toc~subparagraph}{toc_subparagraph}  
  \make_new_toc_entry_key_defaults {toc~subparagraph}
\ExplSyntaxOff 
%    \end{macrocode}
% Next we set the keys

% \begin{docCommand} {l@subparagraph} { \meta{title} \meta{page number}}
% \end{docCommand}
%
% ^^A\paragraph{Testing paragraph}
% ^^A\subparagraph{Testing subparagraph}
%
%    \begin{macrocode} 
\ExplSyntaxOn   
\renewcommand*{\l@subparagraph}[2]{%
  \ifnum \c@tocdepth >\z@
    \if@haschapter@cx
      \vskip \toc_subparagraph_beforeskip
    \else
      \addpenalty \@secpenalty
      \addvspace{\toc_subparagraph_beforeskip}%
    \fi
    \format_toc_entry:nn {toc_subparagraph} {#1} {#2}
  \fi
  }
\ExplSyntaxOff  
%    \end{macrocode}
%
%
%    \begin{macrocode}
\ExplSyntaxOn
\cs_new:Npn \toc_subparagraph_leader
  {
    \normalfont\dot_fill {\toc_subparagraph_dotsep}
  }
\ExplSyntaxOff
%    \end{macrocode}
%
% This brings us, dear reader to a long and arduous 
% path. Surely there must be an easier way. we have 
% added parameters in all sectioning commands, down to 
% paragraph level and we can even lower if you want
% for the legal guys and for construction specs. But
% we also need to do the other lists, list of figure
% and list of tables and maybe others.
%
% \section{List of Figures}
% 

% The standard list of figures follows the same patterns
% for the ToC. We need to redefine the standard macros
% with adequate hooks for parameters. The parameters are
% simpler than the ToC, as we do not have to care
% about different heading levels.\tcbdocmarginnote{U 30-06-2015}
%
% \begin{docCommand}{format_lof_name:n} { \marg{name} }
%   Formats and typesets the contents name part, in a LoF. 
% \end{docCommand}
%
% \begin{docCommand}{listoffigures} { \meta{void} }
%  Start by redefining the list of figures.
%  This will call its own function to format the heading
%  of the LoF and then either write to the file or read a
%  a file using \refCom{start_toc:n}
% \end{docCommand}
%
%    \begin{macrocode}
\ExplSyntaxOn
\renewcommand\listoffigures{%
    \if@twocolumn
      \@restonecoltrue\onecolumn
    \else
      \@restonecolfalse
    \fi
% This is common with the toc    
    \phd_make_toc_title:n {lof_name}
    \start_toc:n {lof}%
    \if@restonecol
      \twocolumn
    \fi
    }
\ExplSyntaxOff  
%    \end{macrocode}

% 
% \subsection{Keys for LoF}
% Next we define all the properties we need to add for the LoF heading. In
% the standard classes it just uses |\chapter*|, but many books have a totally
% different style for this.
% 
%    \begin{macrocode}
\cxset{lof name= Illustrations,}%
%    \end{macrocode}    
%
%    \begin{macrocode}    
\ExplSyntaxOn
\def\phd_lof_start{}
\def\phd_lof_end {}
\cs_new:Npn \make_lof_title
  {
    \newpage
    \phd_lof_start
    \format_toc_name:n {lof_name}%
    \@mkboth{\MakeUppercase\listfigurename}%
            {\MakeUppercase\listfigurename}%
    \phd_lof_end         
  }        
\ExplSyntaxOff    
%    \end{macrocode}    
%
% The |l@figure| is a much simpler operation and it only
% needs to adjust a much smaller set of parameters. The \emph{entry}
% refers to the whole block of a LoF entry. The \emph{page} only
% at the page number of the entry. 
% 

%    \begin{macrocode}
\ExplSyntaxOn 
\make_new_toc_entry_keys {lof~entry}{lof_entry}  
\make_new_toc_entry_key_defaults {lof~entry}
\ExplSyntaxOff
%    \end{macrocode}
% 
% \begin{docCommand}{l@figure} { \marg{number and title} } { \marg {page number} }
%  As for the other lists the |l@figure| has been defined in one of the classes.
%  We re-write it to add parameters. As the style can vary considerably from book
%  to book we also introduce special formatters.
% \end{docCommand}
%    \begin{macrocode}
\ExplSyntaxOn
\renewcommand*{\l@figure}[2]
  {
    \vskip \lof_entry_beforeskip
    {
      \format_toc_entry:nn {lof_entry} {#1} {#2}
    }
 }
\ExplSyntaxOff  
%    \end{macrocode}
%
% \section{List of Tables}
%
% The standard list of tables is similar to that of the
% for the LoF. As a matter of fact in the book class they are
% let to equal. We need to redefine the standard macros
% with adequate hooks for parameters. The parameters are
% simpler than the ToC, as we do not have to care
% about different heading levels.
%
%
% \begin{docCommand}{listoftables} { \meta{void} }
% \end{docCommand}
%
%    \begin{macrocode}
\ExplSyntaxOn
\make_new_toc_entry_keys {lot~entry}{lot_entry}  
\make_new_toc_entry_key_defaults {lot~entry}
\renewcommand\listoftables 
  {
    \if@twocolumn
      \@restonecoltrue\onecolumn
    \else
      \@restonecolfalse
    \fi
    \phd_make_toc_title:n {lot_name}
    \start_toc:n {lot}%
    \if@restonecol
      \twocolumn
    \fi
   }
\ExplSyntaxOff
\endinput    
%    \end{macrocode}
% \drawlistdiagram
% 
% \subsection{Keys for Lists of Tables}

% Next we define all the properties we need to add for the LoF heading. In
% the standard classes it just uses |\chapter*|, but many books have a totally
% different style for this.
% 
%    \begin{macrocode}    
\ExplSyntaxOn
\def\phd_lot_start{}
\def\phd_lot_end {}
    
\ExplSyntaxOff    
%    \end{macrocode}    
%
% The |l@figure| is a much simpler operation and it only
% needs to adjust a much smaller set of parameters. The \emph{entry}
% refers to the whole block of a LoF entry. The \emph{page} only
% at the page number of the entry. 
% 
% 
% \begin{docCommand} {l@figure} { \marg{number and title}  \marg {page number} }
%  As for the other lists the |l@figure| has been defined in one of the classes.
%  We re-write it to add parameters. As the style can vary considerably from book
%  to book we also introduce special formatters.
% \end{docCommand}
%    \begin{macrocode}
\ExplSyntaxOn
\renewcommand*{\l@table}[2]
 {
   \vskip \lot_entry_beforeskip
   \format_toc_entry:nn {lot_entry} {#1} {#2}
  }

\ExplSyntaxOff  
%    \end{macrocode}
% \begin{table}
% \caption{This is the first table}
% \end{table}
% \begin{table}
% \caption{This is the second table}
% \end{table}
% \captionof{table}{This is with captionof}
% \captionof{figure}{This is figure with caption of}
%
% \section{This is a very long heading to see.This is a very long heading to see.This is a very long heading to see.This is a very long heading to see.This is a very long heading to see.This is a very long heading to see.This is a very long heading to see. }

% \meaning\contentsline
%
%
% \meaning\numberline
% \def\tst#1{\tcbox[nobeforeafter,box align=base,size=minimal]{#1}}
%\contentsline {section}{\numberline {1\relax .1\relax }\tst{Introduction} }{1}{section.1.1}
%\contentsline {section}{\numberline {1\relax .2\relax } \tst{Key values for Chapters}}{1}{section.1.2}
%
%
%\makeatletter
%\hyper@linkstart{link}{section.1.2}{12}\hyper@linkend
%\makeatother


%</TOC>      
\endinput       
 %    \begin{macrocode}
\ExplSyntaxOn
\cs_new:Npn \format_toc_chapter_leaders:nn #1 #2
  {
   \leftskip\toc_chapter_indent\relax
   \rightskip \@tocrmarg
   \parfillskip -\rightskip
   \parindent \toc_chapter_indent\relax%
   \@afterindenttrue
   \interlinepenalty\@M
   \leavevmode
   \begin{tcolorbox}[colback=spot!30,arc=3mm,colframe=white]
   \numberlineboxwidth\toc_chapter_number_width\relax
        %\let\toc_number_before \cftchappresnum
        %\let\toc_number_after \cftchapaftersnum
        %\let\toc_number_after_box \cftchapaftersnumb
    \advance\leftskip\numberlineboxwidth
    \null\nobreak\hskip -\leftskip
    {
      \tocchapterfont@cx 
      \exp_after:wN \cs:w toc_chapter_case \cs_end:
      {#1}
    }
    {\lotleader}\nobreak 
     \toc_chapter_page_before\makebox[\toc_chapter_page_number_width][r]
    {
        \exp_after:wN \cs:w \exp_after:wN \lot_page_font_size \cs_end:
        \exp_after:wN \cs:w \exp_after:wN \lot_page_fontweight \cs_end: 
        \exp_after:wN \cs:w \exp_after:wN \lot_page_fontfamily \cs_end:
        \exp_after:wN \cs:w \exp_after:wN \lot_page_fontshape \cs_end:
       \hss#2
    }\toc_chapter_page_after
   \end{tcolorbox}
          
  }          
\ExplSyntaxOff  
%    \end{macrocode}       
%
\newcommand{\lof_entry_leader}{\normalfont\dot_fill{\lof_entry_dotsep}}

%\cs_new:Npn \format_lof_entry_leaders_type #1 #2 
%  {
%   \leftskip \lof_indent\relax
%   \rightskip \lof_rmargin
%   \parfillskip -\rightskip
%   \parindent \lof_indent\relax\@afterindenttrue
%   \interlinepenalty\@M
%   \leavevmode
%    % set parameters for number
%    \numberlineboxwidth \lof_number_width\relax
%    % set numberline hooks if any
%    \advance\leftskip \numberlineboxwidth \null\nobreak\hskip -\leftskip
%    #1 
%    {\lofleader}\nobreak
%    % set hooks for page??
%     \lof_page_before\makebox[\lof_page_number_width][r]
%      {
%        \set_font_aux:n {lof_page}
%       \hss#2
%      }\lof_page_after
%     
%    \par
%  }            
%
\

\cs_new:Npn \format_lot_entry_leaders_type #1 #2 
  {
   \leftskip \lot_indent\relax
   \rightskip \lot_rmargin
   \parfillskip -\rightskip
   \parindent \lot_indent\relax\@afterindenttrue
   \interlinepenalty\@M
   \leavevmode
    % set parameters for number
    \numberlineboxwidth \lot_number_width\relax
    % set numberline hooks if any
    \advance\leftskip \numberlineboxwidth
     \null\nobreak\hskip -\leftskip
    #1 
    {\lotleader}\nobreak
    % set hooks for page??
     \lot_page_before\makebox[\lot_page_number_width][r]
      {
        \exp_after:wN \cs:w \exp_after:wN \lot_page_font_size \cs_end:
        \exp_after:wN \cs:w \exp_after:wN \lot_page_fontweight \cs_end: 
        \exp_after:wN \cs:w \exp_after:wN \lot_page_fontfamily \cs_end:
        \exp_after:wN \cs:w \exp_after:wN \lot_page_fontshape \cs_end:
       \hss#2
      }\lot_page_after
     
    \par
  }  