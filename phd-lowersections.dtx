% \iffalse meta-comment
%<*internal>
\iffalse
%</internal>
%<*readme>
----------------------------------------------------------------
phd-pkgmanager --- a package to shorten preambles
E-mail: yannislaz@gmail.com
Released under the LaTeX Project Public License v1.3c or later
See http://www.latex-project.org/lppl.txt
----------------------------------------------------------------
This file provides a phd for defining a class.
%</readme>
%<*readmemd>
###The `phd` LaTeX2e package

The `phd` latex package and the class with the same name provide
convenient methods to create new styles for books, reports
and articles. It also loads the most commonly used packages 
and resolves conflicts.

This work consists of the file  `phd.dtx`,
and the derived files   `phd.ins`,  `phd.pdf`, and `phd.sty`.

###Installation

run
          phd-lua.bat on windows
           pdflatex phd.dtx
           makeindex -s gind.ist -g phd 

If you have any difficulties with the package come and join us at
http://tex.stackexchange.com and post a new question or
add a comment at http://tex.stackexchange.com/a/45023/963.
or send me a message at  yannislaz at gmail.com

### Documentation

The package was written using the `doc` and `docscript` packages,
so that it is self documented in a literary programming style. 
The .pdf is a fat document, providing over fifty book styles (the
equivalent of classes) plus there is a lot of write-up on the inner
workings of TeX and LaTeX2e. However, you don't need to know much
to use it.

      \usepackage{phd}
      \input{style13}

All choices, are made via an extended key-value interface. 
Although not a compliment, it resembles CSS and the keys are a bit verbose but
attributes are easy to change and have a consistent and easy to remember interface.

To set or add a key we only use one command:

      \cxset{chapter name font-size = Huge,
             chapter number font-size = HUGE} 

### Future Development

This is still an experimental version, but I will retain the
interface in future releases. There is a large amount of
work still to be carried out to improve the template styles
provided, to test it more thoroughly and to add a number of
improvements in the special designs. At present I estimate
that I have completed about 70% of the work that needs
to be done.

__The package as it stands is not production stable.__ 


%</readmemd>
%
%<*TODO>
1. On final round add pkg options. This was left as last in order not to solve problems by adding
    options. Too many options are not a good User Interface.
2.  Finish symbol management, both text and math. Math already 60% incorporated.
3.  Better integration of indexing commands.   
4.  Revisit layout manager for Chapters. Broke again in tests.
5.  Docs. Add all references.
6.  Incorporate phd class for more flexibility.
7. Improve package manager.
8. Group script loading for better font management.
9. General font management to relook it again.
10. Add all style sections (about 100 already prepared). Once they
     are all working issue beta version.
%</TODO>
%<*internal>
\fi
\def\nameofplainTeX{plain}
\ifx\fmtname\nameofplainTeX\else
  \expandafter\begingroup
\fi
%</internal>
%<*install>
\input docstrip.tex
\keepsilent
\askforoverwritefalse
\preamble
----------------------------------------------------------------
phd --- A package to beautify documents.
E-mail: yannislaz@gmail.com
Released under the LaTeX Project Public License v1.3c or later
See http://www.latex-project.org/lppl.txt
----------------------------------------------------------------
\endpreamble

%\BaseDirectory{C:/users/admin/my documents/github/phd}
%\usedir{MWE}
\generate{\file{\jobname.sty}{
  \from{\jobname.dtx}{LSECT}}
  }

%\nopreamble\nopostamble

%</install>

%<install>\endbatchfile
%<*internal>
%\usedir{tex/latex/phd}
\generate{
  \file{\jobname.ins}{\from{\jobname.dtx}{install}}
}
\nopreamble\nopostamble

\generate{
	\file{README.txt}{\from{\jobname.dtx}{readme}}
  }

\generate{
  \file{README.md}{\from{\jobname.dtx}{readmemd}}
}
\generate{
  \file{TODO.tex}{\from{\jobname.dtx}{TODO}}
}

\ifx\fmtname\nameofplainTeX
  \expandafter\endbatchfile
\else
  \expandafter\endgroup
\fi
%</internal>
%<*driver>

%\listfiles
%gdef\@onlypreamble{} % TO BE REMOVED NEEDED FOR TUTS
\documentclass[oneside,11pt,a4paper]{ltxdoc}
\usepackage[bottom=2cm]{geometry}
\savegeometry{std}
% \usepackage[style=mla]{biblatex}
\usepackage{phd}
\usepackage{phd-lowersections}
%\usepackage{pkgindoc}             %%% danger
\sethyperref

\begin{filecontents}{defaults-chapters}
%\cxset{chapter shadow = drop shadow,
%       chapter font-size = Huge,
%       section shadow = no shadow,
%       chapter background-color=white,
%       chapter title font-size=Huge}
 \cxset{    
    chapter number font-size       = huge,
    chapter number font-weight     = bfseries,
    chapter number font-family     = sffamily,
    chapter number font-shape      = upshape,
    chapter number align           = Centering,
    }
\cxset{%    
     chapter title font-size        = huge,
     chapter title font-weight      = bold,
     chapter title font-family      = sffamily,
     chapter title font-shape       = upshape,
     }   
\cxset{%    
    chapter title margin-top       = 0cm,
    chapter title margin-right     = 1cm,
    chapter align                  = RaggedLeft,
    chapter title align            = Centering, %checked
    chapter name                   = Section,
    chapter format                 = block,
    chapter font-size              = huge,
    chapter font-weight            = bold,
    chapter font-family            = sffamily,
    chapter font-shape             = upshape,
    chapter color                  = spot!30,
    chapter number prefix          = ,
    chapter number suffix          = ,
    chapter numbering              = arabic,
    chapter indent                 = 0pt,
    chapter beforeskip             = -1sp,
    chapter afterskip              = 30pt,
    chapter afterindent            = off,
    chapter number after           = ,
    chapter arc                    = 3pt,
    chapter background-color       = spot!30,
    chapter afterindent            = off,
    chapter grow left              = 0mm,
    chapter grow right             = 0mm,
    chapter rounded corners        = northeast,
    chapter shadow                 = drop shadow,
    chapter border-left-width      = 0pt,
    chapter border-right-width     = 0pt,
    chapter border-top-width       = 2pt,
    chapter border-bottom-width    = 2pt,
    chapter padding-left-width     = 0pt,
    chapter padding-right-width    = 10pt,
    chapter padding-top-width      = 10pt,
    chapter padding-bottom-width   = 10pt,
    chapter number color           = spot!30    
    chapter title color            = spot!30}
\end{filecontents}
%% LaTeX2e file `defaults-chapters'
%% generated by the `filecontents' environment
%% from source `phd-scriptsmanager' on 2015/08/25.
%%
%%    General Defaults for Chapters
\cxset{%
    chapter title margin-top-width    =  0cm,
    chapter title margin-right-width  =  1cm,
    chapter title margin-bottom-width = 10pt,
    chapter title margin-left-width   = 0pt,
    chapter align                     = left,
    chapter title align               = left, %checked
    chapter name                      = hang,
    chapter format                    = fashion,
    chapter font-size                 = Huge,
    chapter font-weight               = bold,
    chapter font-family               = sffamily,
    chapter font-shape                = upshape,
    chapter color                     = black,
    chapter number prefix             = ,
    chapter number suffix             = ,
    chapter numbering                 = arabic,
    chapter indent                    = 0pt,
    chapter beforeskip                = -3cm,
    chapter afterskip                 = 30pt,
    chapter afterindent               = off,
    chapter number after              = ,
    chapter arc                       = 0mm,
    chapter background-color          = bgsexy,
    chapter afterindent               = off,
    chapter grow left                 = 0mm,
    chapter grow right                = 0mm,
    chapter rounded corners           = northeast,
    chapter shadow                    = fuzzy halo,
    chapter border-left-width         = 0pt,
    chapter border-right-width        = 0pt,
    chapter border-top-width          = 0pt,
    chapter border-bottom-width       = 0pt,
    chapter padding-left-width        = 0pt,
    chapter padding-right-width       = 10pt,
    chapter padding-top-width         = 10pt,
    chapter padding-bottom-width      = 10pt,
    chapter number color              = white,
    chapter label color               = white,
    }
 \cxset{
    chapter number font-size        = huge,
    chapter number font-weight      = bfseries,
    chapter number font-family      = sffamily,
    chapter number font-shape       = upshape,
    chapter number align            = Centering,
    }
\cxset{%
     chapter title font-size        = Huge,
     chapter title font-weight      = bold,
     chapter title font-family      = calligra,
     chapter title font-shape       = upshape,
     chapter title color            = black,
     }
  
\begin{document}
%\coverpage{asia}{Book Design }{Camel Press}{HEADINGS}{DESIGN} 
\pagestyle{empty}
\coverpage{habtoor-city}{Delay Claim}{HLS-DSE/JV}{HABTOOR CITY}{MEP CLAIM} 
\secondpage
\pagestyle{empty}
\clearpage

\tableofcontents
\pagestyle{empty}
\setcounter{secnumdepth}{2}
\parskip0pt
\mainmatter
\pagenumbering{arabic}
\pagestyle{myheadings}        
%       
%%\def\chaptername{Chapter}
%\input{./styles/style87a}
\cxset{ 
           %chapter toc=true,
           chapter numbering=arabic,
           chapter number color=black,
           chapter number font-shape=upshape,
           subsubsection numbering=none,
           subsubsection font-family=itshape,
           subsubsection color=black,
           subsection number after=\quad,
          section number after=\quad,
          section color=black,
    }
\def\thesubsubsection{}         
%\pagenumbering{gobble}
\def\JV{HLS DSE-JV\xspace}
\def\letter#1{\texttt{HLSDSEJV/HC/L/YL/#1}\xspace}
\def\KA{K\&A}
\def\DT#1{HLG Transmittal Ref. No.: \texttt{HLG-626-DT-HLS-#1}\xspace}
\def\idxbusbar#1{\index{Busbar Delays>#1}}
\def\idxwestin#1{\index{Westin Delays>#1}}
\def\idxstregis#1{\index{St. Regis Delays>#1}}
\def\idxahu#1{\index{Air Handling Unit Delays>#1}}
\let\idxahus\idxahu
\def\CAR#1{\index{Cost Adjustment Requests>CAR-#1}{\texttt{CAR-#1}}\xspace}
\def\idxbasement#1{\index{Basement delays>#1}}
\let\basement\idxbasement
\def\idxdewa#1{\index{Dewa Approvals>#1}}


\mainmatter
\pagestyle{plain}
\cxset{chapter name=,
          chapter numbering=none}
\chapter{Executive Summary}
\thispagestyle{empty}

This short report provides background information related to  the Habtoor City Project MEP works and the steps taken by the \JV to accelerate the works, under the instructions of the Client, Engineer and Main Contractor.  We mobilized to the Project late August 2013. At the time construction was on-going, with the basements structures mostly completed. On mobilization the only K\&A MEP designs available were those provided with the tender package---which was issued in March~2013. Besides procurement and some engineering activities, the \JV  construction activities were mainly focused on builder's works and remaining underground services until March 2014. 

We started receiving design drawings in March and April 2014. The design was issued piecemeal and in out of sequence fashion for the works to progress as planned and according to the agreed Baseline Program . This enabled us to proceed with works only in the Car Parking Areas of the Basements.  The first partially workable set of design drawings received to enable construction in other areas were the drawings received in September 2014 (Mechanical) and December 2014 (Electrical).


\medskip
		
\paragraph{Delayed Incomplete and Unworkable MEP Designs} The general issue of drawings in September~14, provided general design concepts without concerns for physical plant and ceiling constraints. The Plant rooms at T1 and PD6, as designed were not constructible, as the allocated headroom and space was inadequate. We assisted the Engineer by providing 3D and other drawings to at least fit the equipment in the available space. Fans had to be relocated in ceilings at Podium 1, and ducting was re-routed over the same ceiling void. This delayed finalization of Shop Drawings for essentially all the public areas.


The K\&A \enquote{design} mechanical design for St. Regis was only partially completed in September 2014. This design was deficient in many respects, especially in areas such the Technical floors, and as it stood the design was not constructible. This design was inadequate to close equipment orders for long delivery plant, such as AHU, fans and pumps, as calculations for static pressures could not progress. However, we took the initiative to finalize orders based on estimates and released orders before design finalization. We also assisted the Engineer with solving many of the design issues in order to progress with the works.  In addition the Electrical works suffered because of the designs issued in September 2014, as they have not been co-ordinated with the requirements of the Mechanical plant, Kitchen Contractors etc. \par

The delays  to the completion of the final Project requirements are still on-going with many areas of the Hotels still under design development and without related subcontractors appointed on time.

\begin{table}[ht]
\centering

\begin{tabular}{l l p{3cm}  l l}
\toprule
        &Area         &\raggedright Design required as per baseline program & Design Issued & Delay\\
\midrule        
\inc  &First Floor &16 Apr 14  &5 Jan 15  &  8 months\\
\inc  &Attic Floor & 8 Apr 14  &5 Jan 15  & 8 months \\
\inc  &Podium 6  &1 Apr 14   &6 Sep 14  & 5 months \\
\inc  &Podium 5  &29 Mar 14 &6 Sep 14  & 5 months\\
\inc &Podium 4   &20 Mar 14 &6 Sep 14  & 5.5 months\\
\inc &Podium 3  &12 Mar 14  &6 Sep 14  & 5.5 months\\
\inc &Technical 1 &6 Feb 14   &6Sep 14    &7 months\\
\inc &Mezzanine &26 Dec 13  &6 Sep 14  &9 months\\ 
\bottomrule
\end{tabular}
\caption{Design delays for St. Regis}

\end{table}

The MEP Good for Engineering Designs as received from K\&A enabled part of the Engineering and Procurement activities to start bu the design as it stood was  proceed, they are not sufficient to install MEP services. Drawings from ID Consultants, Lighting Consultants, Kitchen Consultant, ELV Consultants and subcontractor Shop Drawings for the same are necessary. These were mostly unavailable.

\paragraph{Instruction to accelerate the works}
Under this background we received the instruction to  accelerate the works (July 2014). We wrote to to the Main Contractor, requesting that a plan be first agreed as to how program recovery could be achieved and \emph{then} agree to a plan to accelerate the works further, so as to bring the Contract Completion dates forward. The request was to accelerate the St. Regis Hotel first with a Target Completion date of 30 March 2014.

At the time approximately 40\% of the slabs  for St. Regis were incomplete. This included critical plant areas at the two technical floors. Not only the structure had to be completed, but also the technical floor, had to have floating floors casted. The T1 floor was partially handed over to us end October and the PD6 floor in January 2015. As is also evident from the subsequently issued Design MEP Drawings, ID Drawings, Lighting Consultant and ELV Consultant drawings issued, the Professional Team was not ready with their Designs. 

As MEP works are closely interlinked with other trades it is important to note that the Structure Cabling, Kitchen Subcontractors, AV and CCTV Subcontractors were not appointed. 
\medskip

 
   


\label{acceleration}
\index{acceleration>manpower}\index{manpower>acceleration}
\paragraph{JV actions taken to accelerate the works.} Once the information started flowing, we reinforced our Engineering and Site Teams. We also added technicians as areas opened to us for work.
\medskip

\noindent\textit{Workforce}
\medskip

\noindent The \JV upon receipt of the instructions to accelerate, and under the impression that designs and appointments of other subcontractors would be accelerated as well, doubled the workforce in July~2014 and subsequently added technicians and other staff until it is at its current level of approximately 3000 personnel. The deployment of personnel is shown in the table below.

\begin{table}[hbp]
\begin{tabular}{c c c c c c c c c}
\toprule
Item &Sep 13 &Feb 14 &Mar 14 & Jul-14 & Aug-14 &Oct-14 & Jan-15 & Mar-15\\
\midrule
 Site Labour   & 48      &610      & 634     & 1212   &  1300     & 1845   &2 781   & 2 731 \\
\bottomrule
\end{tabular}
\end{table}

Although issues prohibited us from fully handing over areas and ceiling closures, the quantum of the work achieved in this short time can be gauged from the gross claimed amount of close to AED~280,000,000.00. (April~14-April~15). 
\medskip

\noindent\textit{Air-freighting of equipment}
\medskip

\noindent In addition to adding personnel we proceeded to air-freight the following equipment, without which the program recovery would have failed:

\begin{enumerate}
\item Chilled water pumps. The chilled water pumps were necessary to be delivered as early as possible in order to enable piping to be connected and for providing wild air as possible. The first submittal for pumps was made on the 25 February 2014. This was returned on the 26 March 2014. The pumps were again resubmitted in 23 April 2014, after revisions to match changes in equipment. They were returned after 40 days, despite the fact that at the time the Engineer was asking us to accelerate the works. Third and fourth submittals followed and the pumps finally approved on 3 July 2014. Pump heads were reverified to meet new layouts and the order place in August, after opening LCs and finalizing prices with Supplier. Cost AED 50,000.00. 
\index{airfreight>chilled water pumps}
\index{chilled water pumps>air freight costs}

\item First fan coil units deliveries for St Regis. These were subjected to similar delays and 388 fan coil units were air-freighted from Thailand at a cost of 196,539.60~AED. 

\item Air Separator for the St Regis Plantroom was air-freighted at a cost of AED~14,185.00.
\item All fans for St Regis. Many of these fans were to be installed in ceiling voids. These were air-freighted at a cost of AED~221,772.00. This also included air-freighting charges for fire rated motors to be air-freighted from Brazil to the Nuaire Wales factory.

\item The above secured the St Regis Hotel plant room areas.

\item Air-freighting of ECUs and Basement fans was stopped after Client Representative wrote us a letter that they would not consider paying for the above costs. \index{Ecology Units>air freight} \index{air freight costs}

These were sea-freighted, with a consequent further compression in the program of works and delaying completion of the following areas:

\begin{enumerate}
\item Basement areas
\item PD6 St Regis Plantroom
\item Kitchens
\end{enumerate}
\end{enumerate}


These are also expected to delay commissioning of kitchen areas in the basement and the Car Park Ventilation System.

\setcounter{chapter}{0}





%\chapter{Summary MEP Progress Report for St Regis Hotel, Habtoor City}
%\pagenumbering{arabic}
%\thispagestyle{plain}
%\section{Current Status}
%
%We have started flushing of the Chilled Water system on the 7 April 2015, as planned and we anticipate to be in a position to progressively provide \emph{wild air} before the end of April, ahead of the scheduled date of the 7 May 2015. In the Basements and in the Guest rooms we have started final fix works, where possible. The Main Plantrooms at Technical Floors 1 and Podium 6, are in the main completed, except final ductwork connections where they impede access. BMS DDC Panels are expected to arrive by the 22 April 2015 and installation expected to be completed within 25-30 days to ensure that by end May we can provide controlled conditions.
%
%Delays have been experienced in the receipt of Electrical panels, such as DBs (delayed due to late deliveries of components by Legrand) and others that were subjected to numerous changes, as described later on.  Other delays were due to late instructions as briefly detailed in Section~\ref{delays}. 
%
%The current outstanding works for the St Regis Hotel are as follows:
%
%\subsection{St Regis Basement}
%
%\begin{description}
%\item[Kitchen Corridors] Some kitchen corridors cable pulling is still under progress. Expected to complete by 30 Apr 2015.
%\item[Main Electrical Room] Delays experienced due to the failure of cable trays during cable pulling and also due to the some of the MDBs being returned to the factory for modifications, as they failed QA/QC Inspections.
%\item[Fan Rooms] Fans scheduled to be delivered 23 Apr 2015.
%\item[BMS] DDC Panels still to be delivered.
%\item[Sump Pumps] Expected to be delivered by 10 May 2015. 
%\item[Others] There are still closure related works, for areas currently inaccessible, such as the new ramp areas, store and office areas. 
%\end{description}
%
%\subsection{Ground Floor}
%\begin{description}
%\item[Ballroom] This area is still under scaffolding being used by the Main Contractor to erect walk-ways in the ceiling. Once the scaffolding is dropped and we are given access to the lower level, we have to install another layer of services, give ceiling grid clearances and upon construction of the ceiling grid we can then install final sprinkler droppers and give clearances for final boarding.
%\item[Banquet Hall] This area has been delayed due to the Iridium Spa delays in Design and appointment of subcontractors. As this area is above the Banquet Hall, coring for drainage pipes delayed the works. This coring is now complete and we expect to ask the Main Contractor to lower the scaffolding and start with the rest of the services.
%\end{description}
%\subsection{Mezzanine}
%\begin{description}
%\item[Festival Dining Restaurant] Currently this area is under nomination, there is no ID Design and final details are still awaited. 
%\item[Security Room] The design for this room has recently changed. The room as shown in the new designs is different from what has been constructed on site and has no space for CCUs. 
%\item[AV Room] Expected to be completed 30 Apr 2015.
%\item[Furniture Store] Expected to be completed 25 Apr 2015.
%\item[Balance Corridors] Expected to be completed 25 Apr 2015.
%\end{description}
%
%\subsection{Podium 1}
%
%\begin{description}
%\item[Banquet A/V Technician] We have no access. This is currently being used as a store.
%\item[Service Corridor] Plan to release for ceiling grid on 23 Apr 2015.
%\item[St Regis Main Kitchen and Corridor] Plan to release on 30 Apr 2015.
%\item[Property Store] Currently no access. If access provided we can release by 30 Apr 2015.
%\item[Steak House Kitchen] Plan to release by 30 Apr 2014.
%\end{description}
%
%\subsection{Podium 2}
%\begin{description}
%\item[Iridium Spa and related areas] We are currently working in the area, which was delayed by late appointment of Finishing Contractor. Still some ID Shop Drawings not available. We expect to catch-up with delays by end May 2015. We plan to complete final fix by 10 June 2015 and Testing and Commissioning by 20 Jul 2015.
%\item[Other Areas] All other areas will be released for closure by 26 Apr 2015.
%\end{description}
%
%\subsection{Podium 3-6}
%
%All guestrooms have been handed over for ceiling closures with the exception of some of the suites, where information and access was provided late. These are the following:
%
%\begin{description}
%\item[Ambassador Suite] Co-ordination ongoing. Expect resolution and final clearances 25 May 2015.
%\item[Bentley Suite] Incomplete information. Completion targets uncertain at this stage.
%\item[Royal Suite] Co-ordination on-going. Expect resolution and final clearances 25 May 2015.
%\end{description}
%
%\subsection{Floor 1}
%
%\begin{description}
%\item[Kitchen 4 and Kitchen 6] Works for walls are progressing, insufficient detail information. Can complete by 15 May 2015, provided all Kitchen Subcontactor’s drawings become available and unimpeded access.
%\end{description}
%
%\section{Delays in Target Dates}
%\label{delays}
%This is a brief summary of recent selected instructions for additional works that have impacted  MEP Progress. 
%In addition to these additional works another critical factor that affected progress was the congestion of services and the numerous RFIs and responses we had to raise in order to resolve them.
%
%\begin{itemize}
%\item Relocation of Kitchen Extract ducting Ground Floor, Mezzanine and Podium BOH areas.
%\item  Additional AV points in all public areas.
%\item  Additional telephone, data and CCTV points in all Public Areas.
%\item  Motorized curtains Meeting Rooms.
%\item Lighting Control System. 
%\item Emergency Lighting System. (see details Chapter~\ref{emergencylights})
%\item Changes to Electrical DBs, SMDBs due to late receipt of DEWA approved drawings. (See Chapter~\ref{electrical})
%\end{itemize}
%
%We have reacted as fast as possible to all instructions and as soon they were received we have added resources to mitigate delays. Where days slipped these are only by a few days and we are confident that by end of this month all physical installations will be completed with the exception of the English Pub, Banquet and Royal Suite. 
%
%\subsection{Back of the House Areas}
%
%All back of the House Areas experienced delays, due to the lack of primary co-ordination at design stage. This caused delays until solutions were found enabling us to install the services. 
%
%The allowable ceiling height in this area was impossible to be achieved and the kitchen extract duct eventually was split in two sections and distributed through two different routes in order to avoid passing it through the corridors which could not accomodate it.
%
%In addition a new roller shutter window was introduced, that made it impossible to install the fresh air ducts feeding the kitchen. After several attempts by |K&A| to find an acceptable solution the roller shutter  window was abandoned as per the instructions of the Client Representative. 
%
%\subsection{Basement Kitchen and Related Areas at B1}
%
%Please note that these areas (with the exception of the corridor) have been cleared for ceiling grid closures in most areas and the balances are as per target to close by the 15 April 2015, including additional works. The additional works were mostly for additional ELV points on walls and for which we have received drawings on the 29 March 2015. We have instituted overtime and added additional crews to complete the works as fast as possible. Most rooms in the area have been affected. 

\begin{comment}
\chapter{Busbar System}

As per the approved Baseline Program we expected to place the busbar order for all three hotels on 27 February 2014. However, HLS DSE-JV were unable to place any orders due to the events that are outlined below, with finality on all busbars only achieved in April 2015. 

\begin{enumerate}
\item On the 23 December 2013 we were requested to change the specification for some busbars via HLG transmittal Ref. No. HLG-626-DT-HLS-0628 dated 23 Decemeber 2013 \textit{Fire Resistance Bus Bar Specification}.

\item On the 25 February 2014 we were issued revised designs via tranmittal Ref. No. HLG-626-DT-HLS-0873 \textit{Revised Electrical Drawings}.

\end{enumerate}


\chapter{Generators}

\section{Generator Ventilation}

\subsection{Background}

The original tender drawings indicated the Generator Ventilation to be by means of Louvres. When such an approach is taken normally the ventilation openings are dictated by the size of the generators.


HLS DSE-JV have submitted as early as 2014 RFIs outlining concerns regarding the adequacy of the ventilation openings and sizing of Generator rooms in the basements.

On the 25 March 2015, we were instructed to proceed with the purchase of additional fans from Systemaire. We issued the order request on the ..... and the order placed on the ......  without formal approval of the amounts in order to speed up the purchase. This affected the commissioning of the generators.

\chapter{Transformer Room Ventilation}

\subsection{Background}

\subsection{Design Errors}
\end{comment}


\cxset{chapter name=Section,
          chapter numbering=arabic}
\chapter{Emergency Lighting System}
\label{emergencylights}
The Emergency Lighting System was finalized on the 22 February 2015. This is impacting on the final fix and commissioning of the Hotel’s Central Battery and Emergency Lighting System. 

\begin{enumerate}
\item As per the approved Baseline Program, we were planning to submit the Material Submission of the Emergency Lighting System by the 25 Feb 2014.
\item On the 25 Nov 2013, we raised RFI \texttt{HLS-DSE/142 JV-RFI-MEP-E028} requesting full details of the Emergency Lights as well as the capacity of the central battery system in order to proceed with Technical Submittals, design of containment system and procurement of equipment.
\item On the 12 Dec 2013 we received an insufficient reply to the above mentioned RFI. We have notified you that the repsonse was insufficient via letter \texttt{HLS DSE/JV/HLG/YL1181} dated 14 Jan 2014, clearly stating that we were unable to proceed further with the submission of the Central Battery System, until the requested information was provided. In our letter we had requested that all details such as diffuser details, base type, IP rating and lamp characteristics are provided. We have also provided details as to Civil Defence requirements.
\item The above concerns were forwarded to the Engineer by the Main Contractor on the 20 Jan 2014. The Engineer instructed us to follow the current design dawings until the completion of the Lighting Consultant’s works.

\item On 10 Feb 2014, we had responded via letter \texttt{HLS-DSE/JVHLG/YL/1227} stating that the information provided by the Engineer, as response to RFI HLS-DSE/142 MEP-E028 was inadequate to produce Shop Drawings and to proceed with material procurement or calculations.
\item On 19 March 2014, once again we responded via letter HLS DSE/JV/626/2.05/YE/nd/2609/14 dated 4 Mar 2014 stating that the inforamtion was inadequate.
\item On 16 April 2014 we sent a clear notification that the lack of information was expected to delay the works via letter \texttt{HLS DSE/JV/HC/L/YL/1322} stating that we were unable to proceed with this portion of the works.

\item On the 20 August 2014 we received via an email instructions to proceed based on a generalized scheme.
\item We raised RFI-MEP-E249 dated 21 Sep 2014, requesting more details on locations and quantities of Emergency Light Fittings. The RFI response was received on 13 Oct 2014 with the response to follow the latest issued Guest Room drawings. 
\item Engineer’s letter \texttt{DU1211/DU/L20054/14} dated 15 Sep 2014, confirmed that due to several ID Design issues the above details were no longer applicable.
\item On 30 Sep 2014 we served notices regarding additional works due to revisions of the Emergency Lighting System for all three hotels.
\item On 15 Nov 2014, we raised concerns due to late finalization of the Central Battery System for W and Westin Hotels. 
\item On the 20 Dec 2014 the we received instructions from the Engineer and Client requesting us to revert back to the original K\&A designs.
\item On the 22 Feb 2015, the Engineer instructed us to procure and install all the Front of House exit lights. We confirmed receipt of the instruction via letter \texttt{YL/1935} dated 24 Mar 2014 once all final details and samples were finalized.
\end{enumerate}


















  
%\chapter{Dewa Approval}
\label{ch:dewa}

As per the approved Baseline Program, we were expected to receive Dewa approved drawings on the 28th November 2013. However, HLS-DSE JV received the LV approved drawings on 15th July 2014, as per HLG transmittal reference No. HLG-626-DT-HLS-1397 dated 15th July 2014. This delayed finalization of orders and progress on site.

 In particular:
 
 \begin{enumerate}
 \item Cables cannot be ordered until such time as approved single line diagrams are available. Once these become available  Shop drawings are prepared and main panels can also be finalized.
 \item MDBs and SMDBs can be finalized and ordered.
 \item Completion of Generator Rooms.
 \item Completion of Transformer and LV Rooms.
  \end{enumerate} 
  
\section{Action by the HLS DSE-JV}
  
Given the enormous task at hand and the instructions received to accelerate the works, we added an Electrical Engineering Manager to assist the Team with the task at hand. We also added additional CAD Operators.

\section{Design Deficiencies}

The Dewa drawings were out of step with the latest revisions of other drawings in terms of architectural, HVAC, Kitchen requirements and other equipment. They also underestimated both the main power required by 2.5MW, as well as the stand-by power required, leading to revisions to the Generator Plant. The Generator Plant is handled under delays of Electrical equipment.

Normally once drawings are submitted and approved by Dewa, the design can be considered complete, however, many areas remained incomplete.

\begin{enumerate}
\item On 20th August 2014, we requested by letter HLSDSEJV/HC/L/YL/1502 to be issued officially a number of revisions we received via email correspondence for the St Regis Hotel. 

\item On 21st August 2014 we confirmed receipt of revised Electrical Drawings from Ground to First Floor via letter ref. no. HLSDSEJV/HC/L/YL/1524. (\CAR{0076})\idxdewa{21 August revisions}

\item On 1 September 2014, we issued delay notice for revised electrical drawings, received by email for St. Regis via letter ref. no.: HLSDSEJV/HC/L/YL/1533 (CAR 83).\CAR{0083}

\item On 11 September, 2014 we confirmed via letter 

\item On 8 September 2014 we received further changes to Electrical Drawings for Westin via HLG Transmittal Ref. No. HLG-626-DT-HLS-1671 dated 8 September 2014.

\item As there was uncertainty over which drawings were to be used, HLG issued us a letter from the Engineer dated 11 September 2004, confirming the following:

      \begin{enumerate}
    	\item  St. Regis Hotel - Electrical Design drawings to be followed as per 1 September 2014 issue drawings (DU/L/18451/14).
    	\item Westin Hotel - Electrical Design drawings to be followed as per the 4 September issued drawings (DU/L/18896/14).
    	\item W Hotel - Electrical Design drawings will be issued after incorporating new Restaurant and ID drawings.
	  \end{enumerate}
\end{enumerate}

\section{30 December 2014 Dewa approved drawings issue}
\label{electrical}

On the 31 December 2014 final Dewa revised drawings were issued. This incorporated further revisons to electrical panels, additional SMDBs, changes to cable sizes, breaker sizes etc. Delay notice was served via letter Ref: YL/1796 date 19/1/2015. Changes affected all areas, including basements, St Regis, Westin and W Hotels. We wrote to the Engineer with suggestions to minimize the impact via letter Ref 25 January 2015 and recording the changes. For the W \& Westin Hotel we did the same via letter ref YL/1843 dated 3 February 2015 (\CAR {0136}).\CAR{0126}

The letters remained unanswered and we issued reminder letter related to these changes via letter ref YL/1907. We also confirmed that the works ere put on hold until such time as we had received confirmation from the Engineer.

Additional works as per letter \texttt{HLG/626/2.05/YE/es/7312/15} dated 6 April 2015. These changes relate to late approval of DEWA drawings. These changes affected all the hotels.

These revisions to the electrical design obstructed us from finalizing and ordering the Electrical Panels including MDBs, MCC, SMDB and electrical cables. The final impact of these changes is described below.

\section{St. Regis}
The following changes were instructed via the above letter and were based on drawing number |EM3300|.
\begin{description}
\item[SMDB-H1-1PLBPR] The works adds outgoing cables feeding |ADD-SS-01| and for |DBP-H1-1PLBPR1|  the cable size was changed from 4c:10mm2 XLPE to 4c:16 mm2 XLPE. The breaker size was changed to 60A MCCB.

\item[SMDB-H1-1TEFCWF] The instruction requests the changing of 15A breaker to 20A for eight CP-H1-1-TEWF/05 T.C.L.-1kW and one CP-H1-1TEWF/09 T.C.L.-1kW.

\item[SMDB-H1-2PL] The instruction requests the following changes:
   \begin{enumerate}
      \item DBP-H1-2PL MCCB 60A change to 80A and cable size 4c:16mm2 XLPE change to 4c:25mm2 XLPE.
      \item BPN-PN-16 and 18 30mA ELCB added.
   \end{enumerate}

\item[SMDB-H1-2PSPA] The instruction requests the following changes:
    \begin{enumerate}
      \item Male and female Jacuzzi bath MCCB 15A change to 20A TCL-3kW.
      \item Female steam room cable size changed (4c:10mm2 XLPE to 4c:16mm2 XLPE).
    \end{enumerate}


\item[SMDB-H1-6PL] The instruction requests the following changes:
   \begin{enumerate}
      \item Additional outgoing feeders for EC-01B, EC-02B, WET-PN-011, WET-PN-017 and WET-PN-020.
      \item 40ATP MCCB removed for FP-H1-1FL
   \end{enumerate}

\end{description}

The following changes were due to drawing No:EM3301

\begin{description}
\item [MDB-H1-B1R1] The instruction requests the following changes:
    \begin{enumerate}
       \item SMDB-H1-GR2 MCCB 200A change to 225A and cable size 4c:95mm2. XLPE change to 4c: 120mm2 XLPE (TCL 116.8kW).
       \item UPS MCCB 60A change to 80A.
    \end{enumerate}
\item[MDB-H1-GR2] The following changes were instructed:
    \begin{enumerate}
       \item Incomer MCCB 200A TP change to 225A TP.
       \item Additional outgoing for WPN-PA-012, WPN-PA-032.
    \end{enumerate}
\end{description}

The following changes were due to drawing No:EM3302

\begin{description}
\item[SMDB-H1-GLSTBR] The following changes were requested:
   \begin{enumerate}
      \item Additional outgping for St Regis, Special Event, St Regis BR.
      \item DBP-H1-GLSTBR MCCB 60A change to 80A.
   \end{enumerate}
\item[SMDB-H1-1PLMK] The following changes were requested:
      \begin{enumerate}
        \item Additional space.
        \item 60A TP MCCB removed.
      \end{enumerate}  
\item[SMDB-H1-2PGSC] The following changes were requested:
     \begin{enumerate}
        \item CAF-SS-01 cable and MCCB size changed from 4c:70mm2 XLPE and 150A TP to 4c:XLPE and 30A TP (TCL-6.5kW).
     \end{enumerate}
\end{description}

The following changes were detailed on drawing No:EM3303

\begin{description}
\item[MDB-H1-B1LR2] SMDB-H1-GL MCCB80A change to 100A.
\item[SMDB-H1-GL] DBP-H1-GLPFA MCCB 60A change 80A and cable size 4c:16mm2 XLPE change 4c:25mm2 XLPE.
\item[SMDB-H1-GLBP1] Additional outgoing for BOQ-KIT-016.
\end{description}

The following changes were detailed on drawing No:EM3304.

\begin{description}
\item[EMDB-H1-B1]  The following changes were requested:
   \begin{enumerate}
      \item ESMDP-H1-GR2 MCCB 80A change to 150A and cable size 4c:35mm2 XLPE change to 4c:70mm2 XLPE.
      \item ESMDB-H1-6PMS1 MCCB 400ATP change to 500ATP.
   \end{enumerate}
\item[EMDB-H1-6PMS1] The following changes were requested:
    \begin{enumerate}
       \item Incomer MCCB 400ATP change to 500ATP.
       \item Additional outgoing for EC-01A,B and future load.
       \item ESMDB-H1-RS cable size changed from 4c:70mm2 XLP (125A TP to 4c:95mm2 XLPE (TCL-55kW).
    \end{enumerate}
\item[ESMDB-H1-GL]
\item[ESMDB-H1-6PMS2]  The incomer to MCCB was changed from 700A TP to 800A TP.
\item[ESMDB-H1-6PL] An additional outgoing cable was requested for EC-02A. For LIFT-H1-SL05 and LIFT-H1-SL06 the cable size was requested to be changed to 4c:35mm2 MGT/XLPE.
\item[ESMDB-H1-2PL] CAF-SK-012, EC-01B MCCB and cable size changed from 30A SP 2c:16mm2 XLPE to 20ASP and 2c:4mm2 PVC (T.C.L.-2.6kW and 0.8kW).
\end{description}

\subsection{St Regis Basement Areas}
The following changes were detailed on drawing No:EM3200.\idxdewa{basements}\idxbasement{SMDB revisions}
\begin{description}
\item[SMDB-BP-1BS1] Additional outgoing circuits were requested for DB-LS-SR2, DB-LS-SR3.
\item[SMDB-BP-1BS3]  An additional outgoing circuit was instructed for DB-LS-SR5.
\item[SMDB-BP-1BS5] An additional outgoing circuit was requested for DB-LS-SR6.
\end{description}

The following changes were detailed on drawing No:EM3201.

\begin{description}
\item[EMDB-BP-1B3] ESMDB-BP-1BS7 MCCB 40A change to 80A and cable size 4c:10mm2 XLPE change 4c:16mm2 XLPE (TCL-17.8kW).\idxbasement{EMDB revisions}\idxbasement{ESMDB revisions}
\item[ESMDB-BP-1BS9] cable size 4c:35mm2 XLPE change to 4c:70mm2 XLPE (TCL-44.4kW).
\item[ESMDB-BP-1B3]  The following changes affected this panel:
     \begin{enumerate}
        \item ESMDB-BP-1BS7 MCCB 40A change to 80A and cable size 4c:10mm2 XLPE change to 4c:16mm2 XLPE    (TCL-17.8kW). 
        \item ESMDB-BP-1BS9 cable size 4c:35mm2 XLPE change to 4c:70mm2 XLPE (TCL-44.4kW). 
        \item ESMDB-BP1BS10 cable size 4c:240mm2 MGT change to 4c:300mm2 MGT(TCL-120kW).
     \end{enumerate}
\item[ESMDB-BP-1BS2]  3 Nos CP-BP-1BE/F1 cable size 4c:16mm2 MGT change to 4c:25mm2 MGT (TCL-17kW).
\item[ESMDB-BP-1BS3]  20ATP, pulse meter, 10mm2 MGT removed for SPCP-BP-1B12.
\item[ESMDB-BP-1BBPA] Incomer MCCB 80A TP change to 100A TP.
\item[ESMDB-BP-1BCOM1] Additional outgoing for COM-IC-001, COM-IC-002, COM-IC-003, COM-IC-006.
\item[USMDB-BP-1BS] UDB-BP-1BS4 and UDB-BP-1BS5 cable size 4c:10mm2 XLPE change to 4c:16mm2 XLPE (TCL-8.8kW and TCL-7.6kw).
\item[ESMDB-BP-1BS1] Incomer MCCB 200A TP change to 250A TP.
\item[ESMDB-BP-1B] ESMDB-BP-1BBPA MCCB 80A change to 100A and cable size 4c:50mm2 XLPE change to 4c:70mm2 XLPE(TCL-44.8kW).
\end{description}

The following changes were due to additional works detailed on drg No: EM3204.

\begin{description}
\item[MDB-BP-2BMEC]
   \begin{enumerate}
     \item Incomer MCCB 80A TP change 100A TP.
     \item FPCP-H1-2B2 cable size 4c:10mm2 XLPE change to 4c:16mm2 XLPE.
     \item FPCP-H1-2B1 cable size 4c:6mm2 XLPE change to 4c:6mm2 XLPE change to 4c:10mm2 XLPE (TCL-5.5kW).
   \end{enumerate}
\item[SMDB-FB-2BMEC]
\end{description}

The following changes were due to drawing No: EM3206.
\begin{description}
\item[MDB-BP-1B2] 
    \begin{enumerate}
       \item MDB-BP-1BCOM Additional outgoings for COM-MP-041.
       \item SMDB-BP-1BS6 MCCB 400A change to 500A and cable size 2x4c:120mm2 XLPE change to 2x4c:150mm2 XLPE (TCL-221kW).
       \item 400A TP+2x4c:120mm2 XLPE removed for FFP-3.
    \end{enumerate}
\item[SMDB-BP-1BS6] Additional outgoing for DB-LS-SR4.
\item[SMDB-BP-1BS10] Additional works were requested as follows:
    \begin{enumerate}
      \item DB-H3-1BSS2 cable size change to 2c:10mm2 XLPE change to 2c:16mm2 XLPE (TCL-1.2kW).
      \item CP-BP-1BTE/F4 cable size change to 4c:16mm2 XLPE change to 4c:25mm2 XLPE MCCB 40A TP Change to 60A TP (TCL-25kW).
      \item CP-BP-1BTF/F2 and CP-BP-1BTE/F2 MCCB 60A TP change to 80A TP (TCL-37kW).
      \item CP-BP-1BTF/F3 and CP-BP-1BTE/F3 MCCB 40A TP change to 60A TP (TCL-22kW and 25kW).
    \end{enumerate}
\end{description}







 
%
\chapter{Delays in Finalizing Requirements for the Busbar System}

As per the approved Baseline Program\footnote{Issued 4 Jan 14 and approved 9 Jan 14, as per HLG letter Ref: HLG/626/2.5/SO/nd/1862/14},  we were planning to order the Busbar on 27 February 2014. The \JV was unable to finalize the Bus Bar Material Submittal due to the numerous revisions issued and the lack of Dewa approved drawings.

\begin{enumerate}
\item On 23 December 2013 we received HLG transmittal ref: no. HLG-626-DT-HLS-0628 dated 23 December 2013 ``Fire Resistance Bus Bar Specification'', instructing us to change some of the busbars to fire rated busbars.
\idxbusbar{change in specification}\idxbusbar{fire rated}
\label{fireratedbusbar}

\item HLG transmittal Ref: No.: HLG-626-DT-HLS-0797 dated 10 February 2014 titled ``Electrical Updated Coordinated Drawings for Basements". (\CAR{0004}).

\item HLG Transmittal Ref. No.: HLG-626-DT-HLS-0873 dated 25 February 2014 ``Revised Electrical Drawings''.

\item On 18 and 20 March 2014 via \DT{0930\&939} we were issued updated drawings for three Hotel (\CAR{0036}).  

\item \DT {10127} dated 10 April 2014 ``Revised Electrical Drawings''. 

\item On 16 June 2014 we were instructed to stop any works on Westin and W Hotel Bus bars due to ``significant comments on Dewa LV approval''. This instruction was received via \KA letter ref. no. DU1211/DU/L/13086. At the time we had completed LV Schematic drawings and had the bus bar isometrics completed. 

\item The approved Dewa LV drawings were received on 15 July 2014 as per \DT {1397} dated 15 July 2014. One month later than the instruction to hold the orders and the works.

\item On 20 August 2014, we requested that we be issued officially the St Regis Hotel revised drawings that we were being send piecemeal by email. \letter{1515}.

\item On 1 September 2014 we had responded to HLG letter ref. no. HLG/626/1.12.AMM/es/4109/14 expressing concerns as to delays in receiving receiving workable design drawings (\letter{1533} (\CAR{0083}).

\item We received revised Electrical Drawings for Westin via \DT {1671} dated 8 September 2014.

\item On 17 September 2014 we received Electrical Drawings for W Hotel via \DT{1728}. 

\item On 27 September we served notice for additional costs and time due to revised electrical drawings related to Mechanical Equipment. The additional loads were received as response to RFI/MEP/E295 for all three Hotels (\CAR{0083}, \CAR{0086}\& \CAR{0087}).

\item On 2 October 2014 we served notices for additional works due to revised Electrical Drawings for Food and Beverage areas. (\CAR{0093}) {\CAR{0093}}

\item We served notice for drawings issued to us for W Hotel from 24 to 27 floor (\CAR{0112}).

\item On 19 February 2015 we issued estimated costs at the Engineer's request for the fire rated bus bars. On 25 March 2015 we were instructed to change all fire rated bus bars to normal bus bars. Accordingly the procurement of this particular bus bars took from 23 December 2013 until 19 February 2015 to be concluded and delayed the works.  (See \ref{fireratedbusbar} referred to when the first instruction was received. )


\end{enumerate}
%

\chapter{Delays in Engineering, Procurement and Construction due to Frequent  Design Changes}

As per the approved Baseline programme the HLS-DSE JV were expecting to receive Good for Construction (GFC) drawings on 14 September 2013 for all areas. This would have ensured that the Contract dates could have been met. GFC drawings are an industry practice by which the Engineer signals the completion of the design and the avoidance of errors omissions and delays by using drawings such as those marked as GFE (Good for Engineering). 

The Consultant has been unable to finalize the design on time and drawings and designs were provided mostly reactively to requests and notices by the Contractor. This has subsequently caused disruption to the \JV Engineering, Submittals, Procurement and Work Progress activities.

\section{Design Changes}

Many design changes were as a response to the \JV RFIs. As of today more than 1300 RFIs have been issued. The events described below are more or less in reverse chronological order from the more recent to the earliest.

Design changes can in general be grouped in the following categories:

\begin{enumerate}
\item As responses to RFIs to resolve, space constraints. The Engineer's design was not coordinated with the basic architectural and structural design. This was most acute in the St Regis Hotel, where large beams and inadequate floor to ceiling height resulted in congested  areas. This was not evident during the tender process and has resulted in additional costs to the Contractor.

\item As responses to RFIs to provide further information due to lack of completeness of the design.

\item Errors and omissions, which either the Engineer corrected or as responses to RFIs.
\end{enumerate}

\begin{enumerate}
\item Variation notification for HVAC works was served under \letter{YL/1922 \& 1931} dated 15 March 2015 and 23 March 2015 (CAR 085). This variations to the works related to:
  \begin{enumerate}
	\item HVAC provisions for Electrical and Telecommunication rooms.
	\item Modifications to duct sizes in St. Regis Mezzanine Floor.
	\item Wow suite toilet exhaust requirements.
	\item Fan air flow changes.
	\item Updates AHUs serving F35 Westin Hotel.
	\item Modifications to duct sizes (W Hotel)
	\item Additional exhaust fan.
  \end{enumerate}
  
 \item Updated electrical drawings (38) were issued on 7 Mar 2015. We served notification as to the time impact in studying the changes (as there were no cloud revisions) and to advice impacts. Costs were reserved under CAR 146. Notifications were served via \letter{1913} and \letter{1920} dated 11 Mar 2015 and 15 Mar 2015.
 
 \item We raised concerns for missing design information for Kitchen equipment revised SLD/power drawings for W Hotel areas in order to proceed with Shop drawings preparation \letter{YL/1856} dated 5 Feb 2015.
 
 \item We raised variation notification due to additional HVAC works \letter{YL/1850} dated 5 Feb 2015.
       \begin{enumerate}
          \item BTU meters for the chilled water system.
          \item Seasonal Taste air distribution details.
          \item Relocation of AHUs at Basement-2
          \item AHU orientation, piping connection and duct re-arrangement.
          \item Additional motorized smoke dampers for standby smoke fans.
       \end{enumerate}

    
\item We requested Specialist subcontractor drawing for Data Centre and Security rooms  to proceed with MEP Shop drawings (affected Mezzanine St. Regis) \letter{YL/1813} dated 20 Jan 2015.\footnote{Security room finalization still pending as of 20 May 2015}. 

\item On 16 December 2014, we advised that MEP delays to the accelerated program were attributed to design delays, civil work delays in the casting of slabs, lack of primary co-ordination by \KA\ delays in appointing subcontractors affecting MEP works (kitchens, ELV works, IT and Audio-visual) and failure of ID designers \letter{YL/1741} dated 16 November 2014.
  
\item We received instructions to provide adaptors and sanitaryware accessories for all Hotels via \letter{Du/26882/14}  dated 16 Nov 2014. 

\item We recorded our concerns due to revisions to the design we received via sketches showing cross-contamination between fresh air intakes and exhaust from kitchens via letter ref: \letter{YL/1760} dated 24 December 2014.

\item Additional fire-fighting works for all hotels were confirmed via letter ref: \letter{YL/1765} dated 28 December 2014.

\item Revised RCP drawings were issued to the \JV which lead to abortive works at St. Regis Ground Floor to Level-2. This was confirmed by \letter{YL/1732} dated 4 December 2014.

\item We received instructions to install earthing system for all structured cabling Telecommunications rooms for all hotels (CAR 130) and \letter{YL/1728} dated 3 Dec 2014.

\item Additional UPS was added to serve the St Regis Hotel Data Center. This was confirmed by letter on 30 October 2014. (CAR 111).

\item Additional floor clean-outs were requested by the Engineer during inspections (CAR-116).

\item FCU type changed from decorative to ducted in W-hotel (CAR-115).

\item Changes to drainage services at the St Regis Hotel due to floor sinks passing over post tension slabs. (CAR 114).

\item Additional Control Panels in St. Regis Hotel Areas. (CAR-113).

\item Revisions to air outlets (CAR-117)

\item Revised Electrical works at St. Regis Hotel (CAR 83) submitted on 4/11/2014.

\item Additional Fans/AHUs that were instructed via RFI responses (CAR 85).

\item  Revised HVAC works  in St Regis Hotel (CAR 116).

\item Additional Electrical Works to St Regis Hotel Data Center (CAR 118)

\item Revised CCTV layouts for W \& Westin Hotel Areas (CAR 087)
  
\item Delays in the finalization of the Central Battery System for W \& Westin Hotels. We raised concerns via letter \letter{YL/1688}  dated 15 October 2014.

\item On 25 September 2014 we confirmed additional works for Drainage Service for one additional toilet at basement. (CAR 92).

\item On 25 September 2014, we confirmed additional works for water supply supply points for pantry at P4 in the St Regis Hotel.

\item On 25 September 2014, we confirmed additional works due to revisions to equipment schedules received via \letter{RF/MEP/M307}  (CAR 85).

\item On 25 September 2014 we confirmed additional works due to revised electrical drawings for Westin Hotel. (CAR 86).

\item On 1 October 2014 we confirmed instructions for additional electrical works due to missing power feeders at the Banquet Hall. received via RFI/MEP306 \& 309 for St Regis Hotel. (CAR 83).

\item On 2 October 2014, we confirmed additional works due to revisions of Mechanical Electrical loads. This was received as a reply to RFI/MEP/E295 for all three Hotels (CAR 83, 86 \&87). 

\item On 2 October 2014 we confirmed additional works due to revised Electrical drawings for Food and Beverage areas (CAR 93).

\item On 2 October 2014, we confirmed additional works due to Revised Mechanical Equipment schedules. This was received via RFI/MEP/M309 for all three Hotels (CAR 85).

\item On 8 October 2014, we were instructed by letter HLG/626/2.05/YE/nd/4327/14 attaching \KA letter Ref. No.: DU1211/DU/L/20844/14 dated 7 September 2004 confirming additional containment for the Access Control System.

\item On 13 October 2014, we confirmed additional works due to the addition of Electrical Heaters to some of the AHUs. This was received via RFI/MEP/M309 for all three hotels (CAR 94).

\item On 6 September 2014, we expressed concerns as to the constructibility and maintainability of the St Regis Technical 1 Plant room and issued proposals to minimize impacts. \letter{1538}. In our letter we requested \KA to resolve the issues latest within 7 days in order to minimize the impact on the accelerated target dates. Further updates were issued via letter \letter{1545} dated 9 September 2014 and \letter{1566} dated 17 September 2014 and \letter{1729} dated 4 December 2014.


\item On 7 September 2014, we requested details for the Access Control System via \letter{1542}.

\item On 9 September 2014, we expressed concerns as to changes to the Architectural design of the St. Regis Fire Pump Room which remained incomplete via \letter{1550} impeding installation of Fire Pumps.

\item On 10 September 2014, we confirmed receipt of revised HVAC drawing via \letter{1549} (CAR 85).

\item On 11 September 2014, we confirmed receipt of revised Electrical drawings for St. Regis Hotel via letter \letter{1555} (CAR 83).

\item On 22 September 2014, we issued notification for additional works related to the Domestic Cold Water system via \letter{1572} (CAR 89).

\item We received revised electrical drawings for Westin via HLG Transmittal Ref. No: HLG-626-DT-HLS-1671 dated 8 September 2014. (CAR 86)

\item We received revised electrical drawings for W Hotel via HLG Transmittal Ref. No.: HLG-626-DT-HLS-1728 dated 17 September 2014.
\end{enumerate}

The above do not record fully the method and lack of detail in issuing design information to the \JV. The general drawing issued to us did not contain adequate information to develop Shop Drawings. Clarifications and proposals were sent to the Engineer for missing information  via RFIs. 




%\makeatletter
\long\def\hlshadi#1{\hl{#1}}
\cxset{enumerate numberingi/.is choice,
  enumerate numberingi/.code={\renewcommand\theenumi {\csname#1\endcsname{enumi}}},
  enumerate numberingii/.code={\renewcommand\theenumii {\csname#1\endcsname{enumii}}},
  enumerate numberingiii/.code={\renewcommand\theenumiii {\csname#1\endcsname{enumiii}}},
  enumerate numberingiv/.code={\renewcommand\theenumiv {\csname#1\endcsname{enumiv}}},
  enumerate labeli punctuation/.store in=\enumeratepunctuationi@cx,
  enumerate labeli/.is choice,
  enumerate labeli/brackets/.code={\renewcommand\labelenumi{(\theenumi\enumeratepunctuationi@cx)}},
  enumerate labeli/square brackets/.code={\renewcommand\labelenumi{[\theenumi\enumeratepunctuationi@cx]}},
  enumerate labeli/right bracket/.code={\renewcommand\labelenumi{\theenumi\enumeratepunctuationi@cx)}},
  enumerate label left/.store in=\enumeratelabelleft@cx,
  enumerate label right/.code=\renewcommand\labelenumi{\enumeratelabelleft@cx\theenumi\enumeratepunctuationi@cx#1},
  enumerate leftmargini/.code={\setlength\leftmargini{#1}},
  enumerate leftmarginii/.code={\setlength\leftmarginii{#1}},
  enumerate leftmarginiii/.code={\setlength\leftmarginiii{#1}},
  enumerate leftmarginiv/.code={\setlength\leftmarginiv{#1}},
  listi topsep/.store in=\listitopsep@cx,
  listi partopsep/.store in=\listipartopsep@cx,
  listi itemsep/.store in=\listiitemsep@cx,
  listi parsep/.store in=\listiparsep@cx,
  listii topsep/.store in=\listiitopsep@cx,
  listii partopsep/.store in=\listiipartopsep@cx,
  listii itemsep/.store in=\listiiitemsep@cx,
  listii parsep/.store in=\listiiparsep@cx,
  listiii topsep/.store in=\listiiitopsep@cx,
  listiii partopsep/.store in=\listiiipartopsep@cx,
  listiii itemsep/.store in=\listiiiitemsep@cx,
  listiii parsep/.store in=\listiiiparsep@cx,
}
\cxset{compact1/.style={%
  enumerate numberingi=arabic,
  enumerate numberingii=alph,
  enumerate numberingiii=alph,
  enumerate numberingiv=roman,
  enumerate labeli punctuation=.,
  enumerate label left=,
  enumerate label right=,
  enumerate leftmargini=2.2em,
  enumerate leftmarginii=2.1em,
  enumerate leftmarginiii=1.5em,
  enumerate leftmarginiv=2em,
  listi topsep=8\p@ \@plus2\p@ \@minus\p@,
  listi itemsep=0\p@ \@plus2\p@ \@minus\p@,
  listi parsep=0\p@ \@plus2\p@ \@minus\p@,
  listii topsep=0\p@ \@plus2\p@ \@minus\p@,
  listii itemsep=0\p@ \@plus2\p@ \@minus\p@,
  listii parsep=0\p@ \@plus2\p@ \@minus\p@,
  listiii topsep=0\p@ \@plus2\p@ \@minus\p@,
  listiii itemsep=0\p@ \@plus2\p@ \@minus\p@,
  listiii parsep=0\p@ \@plus2\p@ \@minus\p@,
}}
\cxset{compact2/.style={%
  enumerate numberingi=alph,
  enumerate numberingii=roman,
  enumerate numberingiii=alph,
  enumerate numberingiv=roman,
  enumerate labeli punctuation=,
  enumerate label left=(,
  enumerate label right=),
  enumerate leftmargini=2.2em,
  enumerate leftmarginii=2.1em,
  enumerate leftmarginiii=1.5em,
  enumerate leftmarginiv=2em,
  listi topsep   = 8\p@ \@plus2\p@ \@minus\p@,
  listi itemsep = 0\p@ \@plus2\p@ \@minus\p@,
  listi parsep   = 0\p@ \@plus2\p@ \@minus\p@,
  listii topsep  = 0\p@ \@plus2\p@ \@minus\p@,
  listii itemsep= 0\p@ \@plus2\p@ \@minus\p@,
  listii parsep  = 0\p@ \@plus2\p@ \@minus\p@,
  listiii topsep = 0\p@ \@plus2\p@ \@minus\p@,
  listiii itemsep= 0\p@ \@plus2\p@ \@minus\p@,
  listiii parsep  = 0\p@ \@plus2\p@ \@minus\p@,
}}

\ExplSyntaxOn
\def\setenumerate#1{
\cxset{#1}
\def\@listi{%
           \leftmargin\leftmargini
            \parsep\listiparsep@cx
            \topsep\listitopsep@cx\relax
            \itemsep\listiitemsep@cx}
            
\def\@listii{\leftmargin\leftmarginii
            \parsep\listiiparsep@cx
            \topsep\listiitopsep@cx\relax
            \itemsep\listiiitemsep@cx}
            
\def\@listiii{\leftmargin\leftmarginiii
            \parsep\listiiiparsep@cx
            \topsep\listiiitopsep@cx\relax
            \itemsep\listiiiitemsep@cx}
}
\ExplSyntaxOff

\setenumerate{compact1}
\makeatother
\def\delay{\textcolor{red}{\Fire}}

\ExplSyntaxOn
% Holds master fields for all ODBs
\clist_new:c {DBSMASTER}

% New DBS
\clist_new:c {DBS}

%% Create the DB. The DB can have any name
%% 
\NewDocumentCommand {\CreateDB} { m }
  {
    \clist_new:c {#1}
    \clist_gput_left:cn {DBS} {#1}
  }   

% add only the number, and this only at the right,
% expecting the user to type it in ascending order and thus make sorting 
% easier
% Note the elements are stores as 1,3,4,68,112 etc.
% #1 DB name
% #2 field index - integer
% 
\NewDocumentCommand \addtoDB { m m  }
  {
    \clist_gput_right:cx { #1 } { #2 }
  }
 
%% Generate some variants
%%

\cs_generate_variant:Nn \clist_sort:Nn {cn}   

%% generalized sort DB
\NewDocumentCommand \SortDB { m }
{
  % remove any duplicates before sorting out
   \clist_remove_duplicates:c { #1 }
   % make variant here
   \clist_sort:cn {#1}
     {
       \int_compare:nNnTF { ##1 } < { ##2 }
        { \sort_reversed: }
        { \sort_ordered: }
    }
}

% #1 DB #2 Suffix
%
 \NewDocumentCommand \printRFI { m m }
  {
  % sort the list in numerical order
    \SortDB { #1 }
     
 % map and print only for category (one to many also possible here)   
     \clist_map_inline:cn { #1 }
      {  
       \cs_if_exist:cT {RFI##1-#2}
          {
            \cs:w RFI##1-#2\cs_end:
            \index {RFI~Mechanical>RFI-M-##1-#2}
          }
       
      }
  }
\ExplSyntaxOff    


 %#1 Number
%#2 Hotel
%#3 Impact
%#4 Description
%#5 Area

\newenvironment {RFI} [3] {%
  \vspace*{12pt}
  \parindent0pt
  \mbox{\bfseries\color{red!80!black}\textsf{RFI-M-#1}}
        \textbf{#2} \textbf{#3} \par
  \begin{enumerate}}
 {\end{enumerate}}

% Optional DB letters W - W hotel
% WE - Westing
% SR  - St Regis

\NewDocumentCommand {\addRFI} { O{SR} +m +m +m +m +g  +g } 
{%
   \expandafter\gdef\csname RFI#2-#1\endcsname
   {%
      \begin{RFI}{#2} {#3} {#4} 
      
        #5
      
      \end{RFI}
       
      \IfNoValueTF{ #6 }{ #6 }{ } 
       
      \IfNoValueTF{ #7 }{ #7 }{ }
    } 
    \addtoDB {MRFI} {#2}
   \par 
}

%% Create DB 
\CreateDB {MRFI}   




\chapter{Mechanical RFIs Related to St Regis}



\addRFI {497} {7 Dec 14 received 19 Jan 2015} {St Regis, Podium 1, Main Kitchen}
{
\item The RFI referred to previous response to RFI, where the Engineer directed the Contractor 
         to their comments on Shop Drawing AHC-HLS-SRH-SDM-AC-P1-0. Contractor re-iterated
         that achievable ceiling in Kitchen area could only be 2300mm and at the kitchen extract hood 2m.
\item This area was almost impossible to complete given the congestion to achieve this ceiling height.         
}


\addRFI {537} {18 Jan 2015 received 22 Jan 2015} {St Regis, Basement 1}%
 {
   \item Engineer issued instruction via drgs DU-1211-DU-00564 on 12 Jan 2014. 
    Engineer issued revised layout for AC-4020.
   \item Suggested re-arrangement blocked access to shaft 35-36/D-F.
   \item No chilled water piping and valving was not shown.
   \item Length of AHU-B2-02 is in excess of 5 meters plus a 500 mm plenum. 
 }



\addRFI{657}
{20-Apr-2015, responded 27 April 2015} {St Regis, Mezzanine, Security Room} 
{
\index{CCU>Security Room}\index{Primary co-ordination>CCTV}
\index{St Regis>Mezzanine>Security Room}

\item The layouts were not matching with latest Architectural drawings.
\item Layout not matching with HVAC Design.
        \begin{enumerate}
        \item The new layouts did not allow space for the CCUs.
        \item The new layouts did not show any access flooring.
        \end{enumerate}
\item Location of CCU 
\item Clear height not achievable.
\item Kitchen extract passing through security room, clashing with Monitor/LED display.
\item Uncertainty if there is access floor or not.
\item Engineer responded to swap store room with security room. The response did not adequately cover all the 
queries. The kitchen headroom or the access flooring was not responded to.
\item The response was followed by workshop meetings. As of May 20 2015, the Main Contractor did not carry the architectural changes required. \delay\delay\delay
}


\addRFI{662} {12-Apr-2014, responded 13-Apr-2014} {St Regis Podium 1 Winter Garden} 
{
\item This RFI dealt with incomplete issues regarding ceilings in Podium 1 Winter Garden.
\item The RFI requested details of return air grilles.
}


\addRFI {0646}{23 April 2015 responded 5 April 2015}{St Regis, Spa Area Podium 2}%
 {
 \index{St Regis>Spa>RFI-0646}
 \item Contractor requested clarifications for discrepancies between BARR \& WRAY requirement drawings and MEP design layout.
 \item Engineer responded to all queries and confirmed requirements.
 \item Area affected Spa. \delay \hl{Shadi to report on nature of delay}
 }{
 The Spa area subcontractor was appointed late. The area at Podium 2, was one of the first areas to be
 completed. The public areas around it, as well as the French Courtyard were designed very late.
 
 The Spa area subcontractor was appointed late. The area at Podium 2, was one of the first areas to be
 completed. The public areas around it, as well as the French Courtyard were designed very late.
}{
 Appointment took place only in January 15.
}

\addRFI [B] {0222}{10 Jun 2014 received 17 Jun 2014 }{Air and Dirt Separators, Boiler Room}
{
\item The Contractor requested clarification on additional air separators in the Boiler Room.
\item Relevant drawings Ref: DU1211-WS2173 Westin Hotel \& DU1211-WS2024 for Basement-1
   \begin{enumerate}
      \item Air and dirt separators are shown in B-1 Boiler plant room for the central water heating. This was not shown on Tender drawings.
      \item Air and Dirt separator is shown in Westin Hotel TE-3 plant room.
      \item Contractor requested verification and noted that the additional equipment shown would lead to additional costs to the Client.
      \item Engineer responded that they should be provided as shown on the revised drawings.
   \end{enumerate}
\item   Later Contractor made arrangements for submittals and for air-freighting later on these equipment. 
}{}{}

\addRFI [B] {0215}{2 Jun 2014 replied 12 Jun 2014} {Basement Duct clashes with Water Feature Plant Room}
{
\item Issues arose due to co-ordination of HV cables tray and ducting.
\item Engineer instructed that the duct be re-sized and re-routed.
}{}{}

\addRFI [WE] {0247} {25 June 2014  returned 5 July 2014} {Westin, Ground Floor, Insufficient ceiling void}
{
   \item Contractor advised that the ID drawings show a gap between the ceiling and the duct of 60mm and that this was going to impede return air.
   \item Please refer to attached modified ceiling levels.
}

\addRFI [B] {0655} {16 Apr 15 received 4 May 15 } {Basement 1, MEP Services in loading Bay}
{
  \item Contractor requested confirmation to Engineer's reply RFI: No. HLG-626-RFI-ME-0618 to proceed with capacity as mentioned in the RFI with one set of fans, one duty and one standby (total 2 fans). 
  \item Engineer responded to refer to RFI reply dated 21 April 2015.
}

\addRFI [W] {0661} {12 Apr 2015 received 22 Apr 2015} {W Mezzanine Floor, Male and Female Accessible Toilets RCP}
  {
    \item Contractor requested co-ordinated RCP layout with diffuser size type and location for male/female and accessible toilet to finalize Shop Drawing.
   \item Engineer responded by providing CAD and PDF files. 
   
   \hlshadi{Shadi to confirm if the drawings received were adequate. (Drawings were received by HLG Feb 15 2015.
     They seem to me they were for a Tender package. We should request Subcontractor Shop Drawings from
     Main Contractor.
    } 
  }
  
\addRFI[B] {0664}  {21 Apr 2015 received 28 Apr 2015} {Basement 1, Loading Bay Area}
 {
   \item Contractor requested clarifications and made proposals for additional fans.
   \item Engineer responded with detailed reply, including tag-number. 
 }
 
\addRFI {0666} {28 April 2015 received 14 May 2015 } {St Regis, Festival Dining Kitchen Podium~1 }
 {
   \item Contractor highlighted that the drainage for this area are not available and would cause problems in finalizing Mezzanine below.
   \item This was responded by CKP Who advised that designs have still not been received from AHG (FNB Division). These are AHG spaces (Specialty Restaurants). \delay\delay\delay
 }


\addRFI [WE] {0676} {7 May 2015 received 18 May 2015} {No access to shaft MR1 at Technical~2 Westin Hotel}
{
  \item Westin Hotel Technical 2 core wall penetration Builder's Works drawings (rev0 and rev1) we proposed 1700 x 950   opening. This was approved by the Engineer. This size has been revised due to the comment, changes were then incorporated in rev 1 layout. At Site the available opening is 2440 x 600. Contractor included an Annexure with proposals.
  \item Engineer responded with proposal to reduce space further between pipes. \hl{Shadi to confirm what eventually was followed.}
}

\printRFI {MRFI} {SR} 

%% Typeset all Chapters
%% 
\chapter {Mechanical RFIs Related to Westin Hotel}
\printRFI {MRFI} {WE}


\chapter{Mechanical RFIs Related to `W'~Hotel}
\printRFI {MRFI} {W}

\chapter{Mechanical RFIs Related to `Basements'  }
\printRFI {MRFI} {B}
  



%
\def\omar#1{\hl{#1}}

\ExplSyntaxOn
 
\edef\aprefix{ERFI}
% #1 DB #2 Suffix
%
 \NewDocumentCommand \printERFI { m m }
  {
  % sort the list in numerical order
    \SortDB { #1 }
     
 % map and print only for category (one to many also possible here)   
     \clist_map_inline:cn { #1 }
      {  
       \cs_if_exist:cT {\aprefix##1-#2}
          {
      %     \PASS RFI-E##1-#2
           \cs:w \aprefix##1-#2\cs_end:
           \index {RFI~Electrical>RFI-E-##1-#2}
           }
         % {\FAIL ##1-#2}\par
      }
  }
  
\ExplSyntaxOff    


 %#1 Number
%#2 Hotel
%#3 Impact
%#4 Description
%#5 Area

\newenvironment {ERFI} [3] {%
  \vspace*{12pt}
  \parindent0pt
  \mbox{\bfseries\color{red!80!black}\textsf{RFI-E-#1}}
        \textbf{#2} \textbf{#3} \par
  \begin{enumerate}}
  {\end{enumerate}}

% Optional DB letters W - W hotel
% WE - Westing
% SR  - St Regis

\NewDocumentCommand {\addERFI} { O{SR} +m +m +m +m +g  +g } 
{%
   \expandafter\gdef\csname \aprefix#2-#1\endcsname
   {%
      \begin{ERFI}{#2} {#3} {#4} 
        #5
     \end{ERFI}
       
      \IfNoValueTF{ #6 }{ #6 }{ } 
       
      \IfNoValueTF{ #7 }{ #7 }{ }
    } 
    \addtoDB {ERFIDB} {#2}
   \par 
}

%% Create DB 
\CreateDB {ERFIDB}   
  
\addERFI[W] {703} {10 May 2015 received 24 May 2015}{Westin and W Guestroom Floors}
{\index{W RFI>RFI-E-703}
   \item Contractor issued query based on HLG letter Ref: HLG/626/2.05/AMM/es/7764/15 dated 4 May 2015
      to request how the lighting in guestroom corridors would be controlled. The HLG letter referred to the Client instructing that lighting control in corridors be cancelled. 
   \item Engineer answered, 'grid switch to be provided in each electrical room for each floor, no of gangs to be as per circuits requirement and to proceed futher as per site conditions.  
   \hl{Omar and PMs to report time implications. Does this involve redrafting of any drawings? If yes please specify}   
}{}{}

\addERFI[WE] {703} {10 May 2015 received 24 May 2015}{Westin and W Guestroom Floors}
{
   \item Contractor issued query based on HLG letter Ref: HLG/626/2.05/AMM/es/7764/15 dated 4 May 2015
      to request how the lighting in guestroom corridors would be controlled. The HLG letter referred to the Client instructing that lighting control in corridors be cancelled. 
   \item Engineer answered, 'grid switch to be provided in each electrical room for each floor, no of gangs to be as per circuits requirement and to proceed futher as per site conditions.  
   \omar{Omar and PMs to report time implications. Does this involve redrafting of any drawings? If yes please specify}   
}{}{}


\addERFI [B] {702} {18 May 2015 received 22 Jan 2015} {St Regis, Basements 1, 2 and 3, Confirmation for lighting circuits for Executive Lift Lobby (B1) and Banquet suite Lift Lobby (B2 and B3) }%
 {
  \item Contractor observed that no lighting design was available for the Executive Lift Lobby (B1) and Banquet suite Lift Lobbies (B2 \& B3). Contractor added circuit references to the light points shown on RCP drawings which were received via \DT{2885}{} dated 5 May 2015 and \DT{2896}{} dated 10 May 2015.
  \item Co-ordinate light location/distribution with ID/Fit Out drawings. Submit RCP for review and approval. Update load schedule based on similar loads.
 }{}{}

\addERFI [W] {701} {12 May 2015 received 23 May 2015} {W Hotel Podium 2, Kitchen, missing circuit references}
 {
   \item Contractor advised of missing DB references and circuits for kitchen areas.
   \item \KA responded to be connected to nearest light DB, advisable DB-FB-2PR2 which has an available spare. \KA also requested for the load schedule to be submitted for approval.
   \omar{Omar I thought we have submitted load schedules, for W?}
 }
 {}{}

\addERFI [B] {696} {7 May 2015 received 19 May 2015} {Basement 1 Janitor Room and Staff Toilets }
{
  \item Contractor noted that the Janitor Room and Staff Toilet lighting and power design is not available.
  \item \KA responded to update the drawing and to be resubmitted for approval!
   \omar{Omar Did we submit?}   
}

\addERFI [SR] {695} {7 May 2015 received 12 May 2015 } {St Regis, PD3, PD4, PD5}
  {
    \item Contractor attached drawing, showing power supply to door requiring not shown in the Electrical design. Contractor requested feeder details for the power socket.
    \item \KA advised to connect to nearby motorized damper.
    \hl{Kyriacos, Omar, how does this affected site. This is clearly a disruptive activity.}
  }

\addERFI [W] {693} {27 Apr 2014 received 30 Apr 2014} {Westin Hotel, discrepancy between Fino RCP drawings and approved shop drawing, Podium 2, Meeting Rooms}
 {
    \item Contractor highlighted conflict between Contractor's approved Shop Drawing and Fino's latest issued RCP drawings. 
    \item \KA respondedn to follow RCP from Fino Int'l for location and number of downlights.
    \omar{Does this mean we need to go back and modify works on site?}
 }

\addERFI {692} {26 Apr 2015 received 2 May 2015 } {All levels having Sofia drawings}
  {
    \item  Contractor referred to nmerous RFI replies   and updated SLD attached.
    \item \KA responded to submit the SLD, for review assesment considering voltage drop, available NOC load etc. \omar{Did we submit? We are not responsible for Sofia loads and hence KA should not be asking us to stay with submitted Dewa loads. If these were exceeded please advise them}
  }

\addERFI [B] {691} {26 April 2015 received 28 April 2015 }{ Basement Loading and unloading area. Missing height of the isolators for scissor lifts.}
 {
   \item Contractor requested information as to the height of required scissor lift isolators.
   \item \KA instructed HLG to co-ordinate with Otis and advise.
   \omar{There are cables in the air in that vicinity, has this been resolved?}  
 }

\addERFI [W] {690} {7 May 2015 received 23 May 2015} {W Hotel, Mezzanine, discrepancy between FA design and architectural drawing.}
 {
 	\item Contractor advised conflicts between architectural drawings and FA design drawings.
 	\item \KA advised to update fire alarm drawings in co-ordination with FA Supplier and UAE code. Shop Drawing to be submitted for approval.
 }
 
\addERFI [SR] {689} {7 May 2015 received 13 May 2015}{St Regis, Podium 1 and TEchnical 1, relocation of amplifier. }
 {
	\item Contractor requested confirmation of location of amplifier rack from Podium-1 to Technical-1.
	\item \KA confirmed and noted routing of containment to be co-ordinated with other services.
	\omar{What amplifier is this? Was containment missed earlier, otherwise this is a disruptive activity.}
 } 

\addERFI [W] {688} {3 May 15 returned 18 May 2015} {W Hotel Ground Floor }
 {
	\item Contractor advised conflicts between current architectural drawings and Approved Fire Alarm Shop drawings.
	\item \KA replied to revise Shop Drawings according to UAE Fire regulations and resubmit.
	\omar {Did we resubmit?}
 }
 
 \addERFI [B] {687} {} {Basement 1 Clashing location of telephone and power sockets with lockers.}
  {
    \item We noticed that the location of power socket and telephone is clashing with locker location and requested clarifications.
    \item \KA provided Mediatech's response to move to nearest available free wall space.
    \omar {Why did we query this? Was it oicked up during an inspection?}
  }

 \addERFI [B] {686} {} {Basement 1 Fire Command and BMS Requirement}
  {
    \item Engineer made comments on \JV Shop drawing. Contractor requested clarifications.
    \item Provision for monitoring of BMS to be provided in the Fire Command Room \JV \& BMS Specialist to confirm that 1 data point and 1 socket is required only.
  }
  
\addERFI [B] {685} {28 Apr 2015 received 6 May 2015} {Basement 1 Fire Command Center - Elevator Panel Details} 
 {
 	\item Contractor requested information for the following:
 	\item \KA replied HLG shall lead the co-ordination with all subcontractors and conclude the requested information as all the submittals are approved. 
 	
 	\delay\delay\delay
 	
 	\omar{Has this now been received?}
 }
 
\addERFI [B] {684} {28 Apr 2015 received 6 May 2015} {Basement 1 Fire Command Center} 
 {
	\item Contractor requested the actual furniture layout for fire conrol room, as commented by Engineer on Shop drawings.
	\item \KA\ instructed HLG to co-ordinate with Furniture Supplier to submit proposal for Operator Approval.
	
	\omar {Zeljko, Rabih/Omar/Rahul we need to finish and get out, request furniture officially from HLG to install computers. } 
	\delay\delay\delay
 }

\addERFI [B] {683}{}{BMS}
 {
 	\item BMS DDC Panel relocation, became necessary due to space constraints.
 	\item \KA responded negatively.
 }

\addERFI [B] {682} { } {Du}
 {
 	\item \lorem
 }


 %%
\chapter{Electrical RFIs St~Regis Hotel}
\printERFI {ERFIDB} {SR}
%
\chapter {Electrical Basements}
\printERFI {ERFIDB} {B}
%
\chapter {Westin Hotel}
\printERFI {ERFIDB} {WE}
%
\chapter {W Hotel}
\printERFI {ERFIDB} {W}











%%internal

\section{Strategy Westin}

\section{Obtain Prices}

Following areas are provisional sums, requrire quotations from subcontractors

\subsection{Westdin}

\begin{enumerate}
\item Westin Spa
\item Round ducting
\item Level 29 and up

\item St Regis Crown and Festival RCP, flooring, ID elevation shop drawings, kitchen shop drawings.
\item Wood deck bar  
\item Royal Suite RCP \& ID elevation shop drawings data \& wifi Shop Drawings AV? Electronics? RMU?
\item Kids party area All MEP design dawings except fire fighting, RCP, Partition, flooring \& ID elevation shop drawings, access control (by Zio).

\item Earthing to be cleared for installation
\item Investigation into costs (masks)
\item Exposed conduits rather than in slabs. (Too costly a difference, also slowed the program).

\item Third fix  St Regis

\item generators

\item Did we achieve recovery? At what cost?

\item RFIs

\item Transformer rooms (difficulties with ventilation requirements)

\item Possible appointment subcontractor.

\item Felxible drops, pull rope to be costed in elev containments. Time consuming and expensive.

\item lift lobbies basements - release of ceilings

\item Containment costs

\item Disruption and Cumulative impact disclaimer at the bottom of all claims. Missing changes to equipment after 1/12/2014 in AHU section.

\item generator ventilation

\item commissioning

\item Stress analysis
\item End June

\item Lifting of equipment
\end{enumerate}


%\def\hot{{\color{red}\scalebox{1.5}{\Fire}} delayed works. }
\def\ghot{{\color{green!80!black}\raggedright\scalebox{1.5}{\Fire}} delayed works, but completed. }
\def\phot{{\color{green!80!black}\raggedright\scalebox{1.5}{\Fire}} delayed works. partially completed. }

\def\check{{$\color{green!80!black}\check$}}
\def\partiald {50\% complete}
\def\unavailable{\hot Design unavailable}
\cxset{subsection color=black}

\chapter{Late Finalization of Provisional Sum Works}

In April 2014, we wrote to the Main Contractor informing them of cut-off dates for providing information related to provisional sums and cut-off dates for their release based on the Baseline Program. The dates varied with the latest one being middle May 2014. Some basic designs were available that enabled some embedded first fix activities to start. However, none of the information was issued on time and according to the baseline program of works. 

The state of the design upon our appointment can be gauged by the value of the provisional works which was in the vicity of 60,000,000.00~AED. This represented approximately 20\% of the base contract value. The sections that follow detail the delays that occured in receiving information.\footnote{A \textcolor{red}{\Fire} indicates the designs arrived late and a \ghot}

\section{Mechanical}

The Total Mechanical Provisionals 19,600,000.00 AED.  Supply, installation, connection, testing and commissioning of all mechanical services including Kitchen hood-make up unit, hood extract ecology unit, AHU, ducting, insulation, air outlets, grills \& diffusers, cold water \& hot water piping, valves, drainage piping, floor drains, gas piping, firefighting piping, sprinklers, fire extinguishers etc (hood fire suppression system by others) to the approval of the Engineer for food services facilities for
\medskip

\bgroup
\raggedright\small
\setcounter{step}{0}
\begin{longtable}{lp{3.4cm}rllp{3.5cm}}
\toprule
\textbf{Item}	& \textbf{Description}	 &\textbf{Amount}&\textbf{Remarks}	&\textbf{1st Fix Start}	&\textbf{K\&A Design}\\
\midrule
1 & Food Service Facilities, Main Receiving Dock and Waste Control \& Centralized Commissary and Stores
& 1,100,000.00 
& B1-SR
&1-Mar-14
& \ghot\\
\midrule
2&Food Service Facilities for St. Regis Hotel	 &1,700,000.00 &&&\\
&&&B2	&30-Mar-14	& \ghot\\
&&&B1	&30-Mar-14	& \ghot\\
&&&GF	&30-Mar-14	& \hot partially still unavailable \\
&&&P1	&10-Apr-14	& \ghot\\
&&&P2	&10-Apr-14	& \ghot\\
\midrule
3 &Food Service Facilities for W. Hotel	 &1,500,000.00 &&&\\
&&&GF	&30-Mar-14	&\hot \\
&&&P1	&30-Mar-14	&\hot \\
&&&P2	&30-Mar-14	&\hot \\
&&&P4	&30-Mar-14	&\hot \\
&&&1st Flr	&30-Mar-14	&\hot \\
&&&24th Flr	&10-May-14	&\hot \\
&&&25th Flr	&10-May-14	&\hot \\
&&&26th Flr	&10-May-14	&\hot \\
\midrule
4&Food Service Facilities for Westin Hotel	 &1,300,000.00 &&&\\
&&&GF	&30-Mar-14	&\hot \partiald  \\
&&&P1	&30-Mar-14	&\hot \\
&&&P2	&30-Mar-14	&\hot \\
&&&1st Flr	&30-Mar-14	&\hot \\
&&&35th Flr	&30-Mar-14	&\hot\\
\midrule
5&Food Service Facilities for Client's Areas	 &1,000,000.00&&&\ghot\\ 
\midrule
6&Laudry Facilities& 1,500,000.00 	&B1-SR		&& \ghot \\
 \bottomrule
\end{longtable}
\egroup

Supply, installation, connection, testing and commissioning of all mechanical services including ducting, insulation, air outlets, grills \& diffusers, chilled water piping, cold water \& hot water piping, valves, drainage piping, floor drains, etc. to the approval of the Engineer for:				
\medskip

\bgroup
\raggedright\small
\setcounter{step}{0}
\begin{longtable}{lp{3.4cm}rlll}
\toprule
\textbf{Item}	& \textbf{Description}	 &\textbf{Amount}&\textbf{Remarks}	&\textbf{1st Fix Start}	&\textbf{K\&A Design}\\
\midrule
7	&Final Fix-Mechanical	 &3,000,000.00 		&&&\hot \\

\multicolumn{2}{c}{\textbf{Total Pro. Sums - Mechanical}}	 &\textbf{19,600,000.00} &&&			\\

\end{longtable}
\egroup

\section{Electrical}

\raggedright\small
\setcounter{step}{0}
\begin{longtable}{cp{3.4cm}rllp{3.5cm}}
\toprule
\textbf{Item}	& \textbf{Description}	 &\textbf{Amount}&\textbf{Remarks}	&\textbf{1st Fix Start}	&\textbf{K\&A Design}\\
\midrule
1	&\hcancel{Supply and fix of ID works for guest elevator cars}	 &\hcancel{2,335,000.00} 	&All	&	& \\
2	&\hcancel[red]{Obstruction lighting}	 &\hcancel[red]{280,000.00} 	&\hcancel{All}	&30-Apr-14	&Cancelled\\
3	&Containment - structured cabling system	 &1,350,000.00 	&All	&15-Apr-14	& \phot \\ 
4	&Containment - CCTV system	 &600,000.00 	&All	&15-Apr-14	& \phot \\
5	&Containment - access control system	 &450,000.00 &All	&15-Apr-14	& \phot \\
6	&Containment - A/V system	 &1,300,000.00 	&All	&15-Apr-14	& \phot \\
7	&Containment - Room Management System	 &1,100,000.00 	&All	&15-Apr-14	& \phot \\
8	&Lighting Control System	 &5,000,000.00 	&All	&30-Apr-14	& \phot\par "BOH partially completed.\\ 
9	&Façade Lighting 	 &1,100,000.00 	&All	&30-Apr-14	& \phot \\
10	&landscape lighting 	& 540,000.00 	&All	&30-Apr-14	& \ghot \\
\midrule
	&\textbf{Total for P.S. Electrical}	 &14,055,000.00 &&&\\

\bottomrule
\end{longtable}


Provisional Sums - Electrical
				
Supply, installation and connection of electrical works including lighting outlets, lighting switches, lighting control, socket outlets, telephone outlet, TV outlet, speakers, internal conduits and wiring, and all necessary accessories				

\newenvironment{pstable}{
\raggedright\small
\setcounter{step}{0}
\begin{longtable}{c>{\raggedright}p{3.4cm}rl p{2cm} p{3.5cm}}
\toprule
\textbf{Item}	& \textbf{Description}	 &\textbf{Amount}&\textbf{Remarks}	&\textbf{Design Cut-off date}
             	&\textbf{K\&A}\\
\midrule}{\bottomrule
\end{longtable}}

\begin{pstable}
a) &Basement/Common Facilities Valets Lifts Lobby 	 &7,865.00 	&B3		& & \ghot\\
b) &Common Facilities: Public Lifts Lobby 2nd Basement Level	 &26,483.00 	&B2	&	& \ghot \\
c) &Common Facilities: Parking Lifts Lobby 2nd Basement Level	 &8,712.00 	&B2		&& \ghot \\
d)	&Common Facilities: Banquet Lifts Lobby 2nd Basement Level	 &16,781.00 	&B2 &		& \ghot \\
e)	&Common Facilities: Staff Cafeteria 2nd Basement Level	 &732,325.00 	&B2	 &	& \ghot\\
   & \textbf{St Regis}                                              &              &     & &\\
\end{pstable}

\begin{pstable}   
f)	&Common Facilities: Valets Lifts Lobby 1st Basement Level	 &7,260.00 	&B1  && \ghot \\
g)	&Common Facilities: Parking Lifts Lobby 1st Basement Level	 &10,868.00 	&B1  && \ghot\\
h)	&Common Facilities: Parking Lifts Lobby 1st Basement Level	 &8,712.00 	&B1  && \ghot \\
i)	&Common Facilities: Royal Suite Lift Lobby 1st Basement Level	 &26,208.00 	&B1 && \ghot \\
j)	&Common Facilities: Exec. Suite Lift Lobby 1st Basement Level	 &8,833.00 	&B1 && \ghot \\
k)	&Common Facilities: Valets Lifts Lobby 1st Basement Level	 &9,504.00 	&B1&& \ghot \\
l)	&Common Facilities: St. Regis Housekeeping 1st Basement Level	 &71,525.00 	&B1 &	1-Apr-14&\ghot \\
m)	&Common Facilities: Maintenance/Engineering Workshop 1st Basement Level	 &267,025.00 	&B1	&1-Apr-14 &\hot \\
n)	&Common Facilities: Laundry 1st Basement Level	 &483,460.00 	&B1	 &1-Apr-14	& \ghot \\
o)	&Common Facilities: Kitchen 1st Basement Level	 &263,495.00 	&B1	 &1-Apr-14	&\ghot \\
p)	&Common Facilities: Chillers/Freezers 1st Basement Level	 &232,220.00 	&B1 &1-Apr-14	& \ghot \\
q)	&Common Facilities: Kitchen Stores 1st Basement Level	 &134,500.00 	&B1 &1-Apr-14	&\ghot \\
r)	&Common Facilities: Office 1st Basement Level	 &13,629.00 	&B1	 &1-Apr-14	&\ghot \\
s)	&Common Facilities: Chillers 1st Basement	 &31,573.00 	&B1	 &1-Apr-14	&\ghot \\
t)	&Common Facilities: Cold Rooms Compressor Room 1st Basement	 &23,216.00 	&B1	 &1-Apr-14	&\ghot \\
u)	&Common Facilities: Toilet 1st Basement Level	 &9,567.00 	&B1	 &1-Apr-14	&\ghot \\
v)	&Common Facilities: Waste Handling Area 1st Basement Level	 &259,105.00 	&B1 &1-Apr-14	&\ghot \\
\midrule
	   &\textbf{Total  P Sums - Electrical}	 &2,652,866.00 &&\\
\end{pstable}

\bigskip

\subsection{Electrical St.Regis Hotel}

Supply, installation and connection of electrical works including lighting outlets, lighting switches, lighting control, socket outlets, telephone outlet, TV outlet, speakers, internal conduits and wiring, and all necessary accessories

\begin{pstable}
1	&St. Regis Hotel Meeting Room 1 Ground Level	 &49,753.00 	&GF	  &27-Apr-14	&\ghot \\
2	&St. Regis Hotel Meeting Room 2 Ground Level	 &26,158.00 	&GF	  &27-Apr-14	&\ghot \\
3	&St. Regis Hotel: Prefunction Area Ground Level	 &326,150.00 	&GF	 &27-Apr-14	&\ghot \\
4	&St. Regis Hotel: Male/Female Toilets Ground Level	 &39,758.00 	&GF	 &27-Apr-14	&\ghot \\
5	&St. Regis Hotel: Boardroom Ground Level	 &43,758.00 	&GF	       &27-Apr-14	&\ghot \\
\end{pstable}

\begin{pstable}
6	&St. Regis Hotel: Lift's Lobby Ground Level	 &42,669.00 	&GF	    &27-Apr-14	&\ghot \\
7	&St. Regis Hotel: Lobby/Sculpture/Reception Ground Level	 &500,429.00 &GF	&27-Apr-14 &\ghot \\
8	&St. Regis Hotel: Pantry Ground Level	 &12,078.00 	&GF	    &27-Apr-14	&\ghot \\
9	&St. Regis Hotel: Lift's Lobby Ground Level	 &47,163.00 	&GF	  &27-Apr-14	&\ghot \\
10	&St. Regis Hotel: Shop Ground Level	 &9,851.00 	&GF	 &27-Apr-14	& \ghot \\
11	&St. Regis Hotel: Valet Parking Cor. Ground Level	 &9,636.00 	&GF	  &27-Apr-14	&\hot \\
12	&St. Regis Hotel: Suite Lift's Lobby Ground Level	 &5,357.00 	&GF	  &27-Apr-14	&\ghot \\
13	&St. Regis Hotel: St. Regis Restaurant Ground Level	 &89,029.00 	&GF	  &27-Apr-14	&\ghot \\
14	&St. Regis Hotel: Cigar Room Ground Level	 &44,798.00 	&GF	       &27-Apr-14	&\ghot \\
15	&St. Regis Hotel: Female Toilet Ground Level	 &16,929.00 	&GF	      &27-Apr-14	&\ghot \\
16	&St. Regis Hotel: Male Toilet Ground Level	 &13,928.00 	&GF	      &27-Apr-14	&\ghot \\
17	&St. Regis Hotel: Corridor Ground Level	 &84,909.00 	&GF	         &27-Apr-14	   &\ghot \\
18	&St. Regis Hotel: Male Toilet Ground Level	 &19,679.00 	&GF      &27-Apr-14   &\ghot \\
19	&St. Regis Hotel: Prefunction Hall Ground Level	 &379,049.00 	&GF	 &27-Apr-14	&\ghot \\
20	&St. Regis Hotel: Pub Kitchen Ground Level	 &40,855.00 	&GF	   &27-Apr-14	&\ghot \\
21	&St. Regis Hotel: St. Regis Ballroom Ground Level	 &455,637.00 	&GF	      &28-Apr-14	&\hot \\
22	&St. Regis Hotel: Banquet Hall Ground Level	 &851,532.00 	&GF	   &24-May-14	&\hot \\
23	&St. Regis Hotel: Banquet Pantry Ground Level	 &160,287.00 &GF	&15-May-14	&\ghot \\

24	&St. Regis Hotel: St. Regis Restaurant Podium 1 Level	 &308,165.00 	&P1	 &15-May-14	&\ghot \\
25	&St. Regis Hotel: Café/Lounge Podium 1 Level	 &417,373.00 	&P1	 &15-May-14	&\ghot \\

26	&St. Regis Hotel: Male/Female Toilets Podium 1 Level	 &43,457.00 	&P1	 &15-May-14	&\ghot \\
27	&St. Regis Hotel: Champagne Bar Podium 1 Level	 &90,418.00 	&P1	 &15-May-14	&\ghot \\

28	&St. Regis Hotel: Pantry Podium 1 Level	 &23,672.00 	&P1	 &15-May-14	&\ghot \\

29	&St. Regis Hotel: Lift's Lobbies Podium 1 Level	 &180,950.00 	&P1	 &15-May-14	&\ghot \\

30	&St. Regis Hotel: St. Regis Main Kitchen Podium 1 Level	 &169,054.00 	&P1	 &15-May-14	&\ghot \\

31	&St. Regis Hotel: Italian Restaurant Podium 1 Level	 &215,254.00 	&P1 &15-May-14	&\ghot \\

32	&St. Regis Hotel: Kitchen Podium 1 Level	 &42,335.00 	&P1	 &15-May-14	&\ghot \\

33	&St. Regis Hotel: Boardroom Podium 1 Level	 &40,798.00 	&P1	 &1-May-14	&\ghot \\

34	&St. Regis Hotel: Break-Out Area Podium 1 Level	 &96,443.00 	&P1  &1-May-14&\ghot \\
35	&St. Regis Hotel: Meeting Room Podium 1 Level	 &25,815.00 	&P1	 & 1-May-14&\ghot \\
36	&St. Regis Hotel: Meeting Room Podium 1 Level	 &52,712.00 	&P1	 &1-May-14&\ghot \\
37	&St. Regis Hotel: Meeting Room Podium 1 Level	 &52,712.00 	&P1	 &1-May-14&\ghot \\
38	&St. Regis Hotel: Meeting Room Podium 1 Level	 &36,452.00 	&P1	 &1-May-14&\ghot \\
39	&St. Regis Hotel: Male Toilet Podium 1 Level	    &14,756.00 	&P1	 &1-May-14&\ghot \\
40	&St. Regis Hotel: Female Toilet Podium 1 Level	 &12,794.00 	&P1	 &1-May-14&\ghot \\

41	&St. Regis Hotel: Gourmet Shop/Cafe Podium 2 Level	 &267,790.00 	&P2	 &1-May-14&\ghot \\
\end{pstable}

\begin{pstable}
42	&St. Regis Hotel: Spa Podium 2 Level	 &541,352.00 	&P2	 &24-Apr-14	&\ghot \\
43	&St. Regis Hotel: Business Centre Podium 2 Level	 &197,690.00 	&P2	 &24-Apr-14	&\ghot \\

44	&St. Regis Hotel: Deluxe Suite Podium 2 Level	 &114,930.00 	&P2 &24-Apr-14	&\ghot \\
45	&St. Regis Hotel: Junior Suites Podium 2 Level	 &100,752.00 	&P2 &24-Apr-14	&\ghot \\

46	&St. Regis Hotel: Junior Suites Podium 2 Level	 &144,084.00 	&P2	 &24-Apr-14	&\hot \\

47	&St. Regis Hotel: Ambassador Suite Podium 2 Level	 &175,242.00 	&P2 &24-Apr-14	&\hot \\
48	&St. Regis Hotel: Executive Suite Podium 2 Level	 &203,292.00 	&P2 &24-Apr-14	&\hot \\

49	&St. Regis Hotel: Water Fountain Podium 2 Level	 &409,475.00 	&P2	 &8-May-14	&\hot \\
50	&St. Regis Hotel: Corridor Podium 2 Level	 &349,470.00 	&P2   &8-May-14	&\hot \\

51	&St. Regis Hotel: Guest Corridor Podium 2 Level	 &447,436.00 	&P2	 &8-May-14	&\hot \\
52	&St. Regis Hotel: Lift's Lobby Podium 2 Level	 &25,342.00 	&P2	 &8-May-14	&\hot \\

53	&St. Regis Hotel: Deluxe Suite Podium 3 Level	 &114,930.00 	&P3	 &4-May-14	&\hot \\
54	&St. Regis Hotel: Ambassador Suite Podium 3 Level	 &175,242.00 	&P3 &4-May-14	&\hot \\

55	&St. Regis Hotel: Junior Suite 1 Podium 3 Level	 &50,340.00 	&P3 &4-May-14	&\hot \\
56	&St. Regis Hotel: Junior Suite 2 Podium 3 Level	 &53,682.00 	&P3	 &4-May-14	&\hot \\

57	&St. Regis Hotel: Junior Suite 3 Podium 3 Level	 &72,042.00 	&P3	 &4-May-14	&\hot \\
58	&St. Regis Hotel: Junior Suite 4 Podium 3 Level	 &50,376.00 	&P3	 &4-May-14	&\hot \\
59	&St. Regis Hotel: Executive Suite Podium 3 Level	 &135,528.00 	&P3	 &4-May-14	&\hot \\

60	&St. Regis Hotel: Executive Suite 1 Podium 3 Level	 &132,060.00 	&P3	 &4-May-14	&\hot  \\

61	&St. Regis Hotel: Corridor Podium 3 Level	 &412,704.00 	&P3	 &18-May-14	&\hot \\
62	&St. Regis Hotel: Deluxe Suite Podium 4 Level	 &114,930.00 	&P4	 &15-May-14	&\hot \\

63	&St. Regis Hotel: Designer Suite Podium 4 Level	 &175,242.00 	&P4	 &15-May-14	&\hot \\
64	&St. Regis Hotel: Junior Suite 1 Podium 4 Level	 &50,340.00 	&P4	 &15-May-14	&\ghot \\

65	&St. Regis Hotel: Junior Suite 2 Podium 4 Level	 &53,682.00 	&P4	 &15-May-14	&\ghot \\

66	&St. Regis Hotel: Junior Suite 3 Podium 4 Level	 &72,042.00 	&P4	 &15-May-14	&\ghot \\

67	&St. Regis Hotel: Junior Suite 4 Podium 4 Level	 &50,376.00 	&P4	 &15-May-14	&\ghot \\

68	&St. Regis Hotel: Executive Suite 2 Podium 4 Level	 &264,120.00 	&P4	 &15-May-14	&\ghot \\
69	&St. Regis Hotel: Corridor Podium 4 Level	 &412,704.00 	&P4	 &30-May-14	&\ghot \\
70	&St. Regis Hotel: Presidential Suite Podium 5 Level	 &145,536.00 	&P5 &22-May-14	&\hot \\
71	&St. Regis Hotel: Designer Suite Podium 5 Level	 &174,918.00 	&P5	 &22-May-14	&\hot \\

72	&St. Regis Hotel: Executive Suite Podium 5 Level	 &271,056.00 	&P5	 &22-May-14	&\hot \\
73	&St. Regis Hotel: Junior Suite 1 Podium 5 Level	 &50,340.00 	&P5	 &22-May-14	&\ghot \\
74	&St. Regis Hotel: Junior Suite 2 Podium 5 Level	 &53,682.00 	&P5	 &22-May-14	&\ghot \\
75	&St. Regis Hotel: Junior Suite 3 Podium 5 Level	 &72,042.00 	&P5	 &22-May-14	&\ghot \\
76	&St. Regis Hotel: Junior Suite 4 Podium 5 Level	 &50,376.00 	&P5	 &22-May-14	&\ghot \\
77	&St. Regis Hotel: Corridor Podium 5 Level	 &461,461.00 	&P5	   &10-May-14	&\ghot \\
78	&St. Regis Hotel: Presidential Suite Podium 6 Level	 &145,536.00 	&P6	 &3-May-14	&\ghot \\
79	&St. Regis Hotel: Royal Suite Podium 6 Level	 &521,004.00 	&P6	 &3-May-14	&\hot \\
80	&St. Regis Hotel: Junior Suite 1 Podium 6 Level	 &72,042.00 	&P6	 &3-May-14	&\ghot \\
81	&St. Regis Hotel: Junior Suite 2 Podium 6 Level	 &50,376.00 	&P6	 &3-May-14	&\ghot \\
82	&St. Regis Hotel: Corridor Podium 6 Level	 &166,150.00 	&P6	 &26-May-14	&\ghot \\
83	&St. Regis Hotel: Special Cabanas F01 Level	 &42,878.00 	&F01	&9-May-14	&\hot \\
84	&St. Regis Hotel: Male \& Female Changing Room F01 Level	 &25,553.00 	&F01	&9-May-14	&\hot \\
85	&St. Regis Hotel: Royal Private Swimming Pool F01 Level	 &105,096.00 	&F01	&9-May-14	&\hot \\
86	&St. Regis Hotel: Pantry F01 Level	 &18,216.00 	                 &F01	&9-May-14	&\hot \\
87	&St. Regis Hotel: Special Cabanas F01 Level	 &42,895.00 	&F01	&9-May-14	&\hot \\
88	&St. Regis Hotel: Kids Play Room F01 Level	 &148,500.00 	&F01	&9-May-14	&\hot \\
89	&St. Regis Hotel: Male \& Female Changing Room F01 	 &52,008.00 	&F01	&9-May-14	&\hot \\
90	&St. Regis Hotel: Pantry F01 Level	           &25,889.00 	&F01	&9-May-14	&\hot \\
91	&St. Regis Hotel: Bar Pagoda F01 Level	 &69,603.00 	&F01	 &9-May-14	&\hot \\
92	&St. Regis Hotel: Lift's Lobbies F01 Level	 &41,250.00 	&F01	&2-May-14	&\hot \\
\midrule
   &\textbf{Total for PS - Electrical}	 &\textbf{13,266,154.00} &&&\\

\end{pstable}

\subsection{Electrical Westin Hotel}

Supply, installation and connection of electrical works including lighting outlets, lighting switches, lighting control, socket outlets, telephone outlet, TV outlet, speakers, internal conduits and wiring, and all necessary accessories

\begin{pstable}
1	&W. Hotel: W. Hotel Entrance Lobby Ground Level	 &191,100.00 	&GF	 &17-Apr-14	&\hot \\
2	&W. Hotel: Theatre Entrance Lobby Mezzanine Level	 &224,437.00 	&MZ	 &17-Apr-14	&\hot \\

3	&W. Hotel: Public Lifts Mezzanine Level	 &35,959.00 	&MZ	 &17-Apr-14	&\hot \\

4	&W. Hotel: Corridor Mezzanine Level	 &10,368.00 	&MZ	 &17-Apr-14	&\hot \\

5	&W. Hotel: Male/Female Toilets Mezzanine Level	 &61,785.00 	&MZ	 &17-Apr-14	&\hot \\


6	&W. Hotel: Kitchen Podium 1 Level	 &62,743.00 	&P1	 &19-Apr-14	& \hot \\

7	&W. Hotel: Kitchen Podium 1 Level	 &44,945.00 	&P1	 &19-Apr-14	& \hot \\

8	&W. Hotel: Kitchen Tech 1 Level	    &29,695.00 	&T1	 &19-Apr-14	&\hot \\

9	&W. Hotel: Festival Dining Restaurant 3 Area 276m2 PL1	 &133,480.00 	&P1	 &30-Apr-14	&\hot \\

10	&W. Hotel: Public Lifts Podium 1 Level	 &146,898.00 	&P1	 &30-Apr-14	& \hot \\

11	&W. Hotel: Male/Female Toilets Tech 1 Level	 &19,929.00 	&T1	 &30-Apr-14	& \hot \\

12	&W. Hotel: Lifts Lobby Tech 1 Level	 &191,363.00 	&T1	 &30-Apr-14	& \hot \\

13	&W. Hotel: KitchenPodium 2 Level	 &38,590.00 	&P2 &22-Apr-14	& \hot \\

14	&W. Hotel: Jazz Room Podium 2 Level	 &306,873.00 	&P2	 &3-Apr-14	& \hot \\

15	&W. Hotel: Public Lift Lobby Podium 2 Level	 &229,092.00 	&P2	 &3-Apr-14	& \hot \\

16	&W. Hotel: Female WC Podium 2 Level	 &8,140.00 	&P2	 &3-Apr-14	& \hot \\

17	&W. Hotel: Male WC Podium 2 Level	 &13,140.00 	&P2	 &3-Apr-14	&\hot \\

18	&W. Hotel: W-Great Room Podium 4 Level	 &434,605.00 	&P4	 &13-Apr-14	&\hot \\

19	&W. Hotel: Great Room Pantry Podium 4 Level	 &83,182.00 	&P4	 &13-Apr-14	&\hot \\

20	&W. Hotel: Business Centre Podium 4 Level	 &50,479.00 	&P4	 &13-Apr-14	&\hot \\

21	&W. Hotel: Male \& Female WC Podium 4 Level	 &37,752.00 	&P4	 &13-Apr-14	&\hot \\

22	&W. Hotel: Lift Lobby Podium 4 Level	 &22,517.00 	&P4	 &13-Apr-14	& \hot \\

23	&W. Hotel: Pre Function Hall Podium 4 Level	 &157,487.00 	&P4	 &13-Apr-14	& \hot \\

24	&W. Hotel: W-Studio 1 Podium 5 Level	 &59,004.00 	&P5	 &23-Apr-14	& \hot \\

25	&W. Hotel: Pre Function Podium 5 Level	 &101,954.00 	&P5	 &23-Apr-14	&\hot \\

26	&W. Hotel: Lift Lobby Podium 5 Level	 &17,226.00 	&P5	 &23-Apr-14	&\hot \\

27	&W. Hotel: W-Studio 4 Podium 5 Level	 &22,154.00 	&P5	 &23-Apr-14	& \hot \\

28	&W. Hotel: W-Studio 5 Podium 5 Level	 &19,399.00 	&P5 &23-Apr-14	& \hot \\

29	&W. Hotel: Male \& Female WC Podium 5 Level	 &40,018.00 	&P5	 &23-Apr-14	& \hot \\

30	&W. Hotel: W-Studio 3 Podium 5 Level	 &28,666.00 	&P5	 &23-Apr-14	& \hot \\

31	&W. Hotel: W-Studio 2 Podium 5	 &28,160.00 	&P5	 &23-Apr-14	& \hot \\

32	&W. Hotel: W Specialty Restaurant F01 Level	 &597,564.00 	&F01	&12-Apr-14	& \hot \\

33	&W. Hotel: WOW Lifts Lobby F01 Level	 &42,884.00 	&F01	&12-Apr-14	& \hot \\

34	&W. Hotel: Fantastic Suite F02 Level	 &51,126.00 	&F02	&8-Apr-14	& \hot \\

35	&W. Hotel: WOW/Parlour Suite F02 Level	 &53,598.00 	&F02	&8-Apr-14	& \hot \\

36	&W. Hotel: Studio Suite F02 Level	 &41,382.00 	&F02	&8-Apr-14	& \hot \\
37	&W. Hotel: Fabulous Room F02 Level	 &29,233.00 	&F02	&8-Apr-14	& \hot \\

38	&W. Hotel: Party Room F02 Level	 &27,005.00 	&F02	&8-Apr-14	& \hot \\

39	&W. Hotel: Corridor F02 Level	 &100,007.00 	&F02	&8-Apr-14	&\hot \\

40	&W. Hotel: Fantastic Suite TP1 Level	 &64,542.00 	&TP1	&15-Apr-14	& \hot \\

41	&W. Hotel: WOW/Parlour Suite TP1 Level	 &67,938.00 	&TP1	&15-Apr-14	& \hot \\

42	&W. Hotel: Studio Suite TP1 Level	 &42,306.00 	&TP1	&15-Apr-14	& \hot \\

43	&W. Hotel: Fabulous Room TP1 Level	 &29,233.00 	&TP1	&15-Apr-14	&\hot \\

44	&W. Hotel: Party Room TP1 Level	 &27,005.00 	&TP1	&15-Apr-14	& \hot \\
45	&W. Hotel Corridor TP1 Level	 &95,348.00 	&TP1	&15-Apr-14	& \hot \\

46	&W. Hotel: Fantastic Suite TP2 Level	 &64,884.00 	&TP2	&12-Apr-14	& \hot \\

47	&W. Hotel: WOW/Parlour Suite TP2 Level	 &69,612.00 	&TP2	&12-Apr-14	& \hot \\

48	&W. Hotel: Studio Suite TP2 Level	 &42,678.00 	&TP2	&12-Apr-14	& \hot \\

49	&W. Hotel: Fabulous Room TP2 Level	 &29,233.00 	&TP2	&12-Apr-14	& \hot \\

50	&W. Hotel: Party Room TP2 Level	 &27,005.00 	&TP2	&12-Apr-14	& \hot \\

51	&W. Hotel: Corridor TP2 Level	 &95,348.00 	&TP2	&12-Apr-14	& \hot \\

52	&W. Hotel: WOW Suite F18 Level	 &124,062.00 	&F18	&12-May-14	&\hot \\

53	&W. Hotel: WOW Suite F18 Level	 &98,436.00 	&F18	&12-May-14	& \hot \\

54	&W. Hotel: Studio Suite F18 Level	 &45,396.00 	&F18	&12-May-14	& \hot \\

55	&W. Hotel: Fabulous Suite F18 Level	 &39,204.00 	&F18	&12-May-14	& \hot \\
56	&W. Hotel: Emergency Lifts Lobby F18 Level	 &19,173.00 	&F18	&12-May-14	& \hot \\
57	&W. Hotel: Corridor F18 Level	 &93,918.00 	&F18	&12-May-14	& \hot \\

58	&W. Hotel: WOW Suite F19 Level	 &124,062.00 	&F19	&25-May-14	& \hot \\

59	&W. Hotel: WOW Suite 2 F19 Level	 &98,436.00 	&F19	&25-May-14	& \hot \\

60	&W. Hotel: Studio Suite F19 Level	 &45,396.00 	&F19	&25-May-14	& \hot \\

61	&W. Hotel: Fabulous Suite F19 Level	 &31,890.00 	&F19	&25-May-14	& \hot \\

62	&W. Hotel: Party Room F19 Level	 &27,005.00 	&F19	&25-May-14	& \hot \\

63	&W. Hotel: Corridor F19 Level	 &93,918.00 	&F19	&25-May-14	& \hot \\

64	&W. Hotel: Extreme WOW Suite F20 Level	 &198,558.00 	&F20	&4-May-14	& \hot \\

65	&W. Hotel: Studio Suite F20 Level	 &45,396.00 	&F20	&4-May-14	 & \hot \\

66	&W. Hotel: Fabulous Suite F20 Level	 &39,204.00 	&F20	&4-May-14	 & \hot \\

67	&W. Hotel: Emergency Lifts Lobby F20 Level	 &19,173.00 	&F20	&4-May-14	& \hot \\

68	&W. Hotel: Corridor F20 Level	 &102,707.00 	&F20	&4-May-14	& \hot \\

69	&W. Hotel: Extreme WOW Suite TP3 Level	 &139,116.00 	&TP3	&15-May-14	& \hot \\
70	&W. Hotel: Studio Suite TP3 Level	 &45,396.00 	&TP3	&15-May-14	& \hot \\

71	&W. Hotel: Fabulous Suite TP3 Level	 &31,890.00 	&TP3	&15-May-14	& \hot \\

72	&W. Hotel: Party Room TP3 Level	 &27,005.00 	&TP3	&15-May-14	& \hot \\

73	&W. Hotel: Corridor TP3 Level	 &91,933.00 	&TP3	&15-May-14	&\hot \\
74	&W. Hotel: Extreme WOW Suite F22 Level	 &198,558.00 	&F22	&25-May-14	&\hot \\
75	&W. Hotel: Studio Suite F22 Level	 &45,396.00 	&F22	&25-May-14	&\hot \\
76	&W. Hotel: Fabulous Suite F22 Level	 &31,890.00 	&F22	&25-May-14	& \hot \\
77	&W. Hotel: Party Room F22 Level	 &27,005.00 	&F22	&25-May-14	& \hot \\
78	&W. Hotel: Corridor F22 Level	 &91,933.00 	&F22	&25-May-14	& \hot \\
79	&W. Hotel: Back of House Pantry F24 (F25) Level	 &32,570.00 	&F25	&12-May-14	&\hot \\
80	&W. Hotel: W-Living Room F24 (F25) Level	 &519,261.00 	&F25	&25-May-14	&\hot \\
81	&W. Hotel: Male \& Female WC F24 (F25) Level	 &52,707.00 	&F25	&25-May-14	&\hot \\
82	&W. Hotel: Lifts Lobby F24 (F25) Level	 &22,077.00 	&F25	&25-May-14	&\hot \\
83	&W. Hotel: Deputy GM F24 (F25) Level	 &25,823.00 	&F25	&25-May-14	&\hot \\
84	&W. Hotel: Lifts Lobby 2 F24 (F25) Level	 &32,791.00 	&F25	&25-May-14	&\hot \\
85	&W. Hotel: Back of House Kitchen F25L (F26) Level	 &24,165.00 	&F26	&16-May-14	&\hot \\
86	&W. Hotel: Restaurant F25L (F26) Level	 &430,694.00 	&F26	&29-May-14	&\hot \\
87	&W. Hotel: Male \& Female WC F25L (F26) Level	 &52,707.00 	&F26	&29-May-14	&\hot \\
88	&W. Hotel: Lifts Lobby 1 F25L (F26) Level	 &21,978.00 	&F26	&29-May-14	&\hot \\
89	&W. Hotel: Lifts Lobby 2 F25L (F26) Level	 &13,211.00 	&F26	&29-May-14	&\hot \\
90	&W. Hotel: Destination Restaurant F26L (F28) Level	 &568,524.00 	&F28	&15-May-14	&\hot \\
91	&W. Hotel: BOH Pantry F26L (F28) Level	 &15,862.00 	&F28	&15-May-14	&\hot \\
92	&W. Hotel: Lift Lobby F26L (F28) Level	 &22,077.00 	&F28	&15-May-14	&\hot \\
93	&W. Hotel: Male \& Female WC F26L (F28) Level	 &36,872.00 	&F28	&15-May-14	&\hot \\
94	&W. Hotel: Entrance/WOW Lift Lobby F26L (F28) Level	 &22,471.00 	&F28	&15-May-14	&\hot \\
95	&W. Hotel: VIP's Deck F27 (F30) Level	 &175,373.00 	&F30	&15-May-14	&\hot \\
\midrule
	&\textbf{Total for Provisional Sums - Electrical}	 &8,494,290.00 &&&\\			

\end{pstable}
\bigskip

\section{Westin Electrical PS}

Supply, installation and connection of electrical works including lighting outlets, lighting switches, lighting control, socket outlets, telephone outlet, TV outlet, speakers, internal conduits and wiring, and all necessary accessories

\begin{pstable}
1	&Westin Hotel: Westin Bar Ground Level	 &139,700.00 &GF	&3-Mar-14	&\hot  \\
2	&Westin Hotel: Westin Bar Kitchen Ground Level	 &26,500.00 	&GF	 &3-Mar-14 &\hot \\
3	&Westin Hotel: Lobby/Reception/Entrance Ground Level	 &676,000.00 	&GF	 &3-Mar-14 &\hot \\
4	&Westin Hotel: Wash Rooms Ground Level	 &41,250.00 	&GF &3-Mar-14	&\hot \\

5	&Westin Hotel: Retail Ground Level	 &13,750.00 	&GF &3-Mar-14 &\hot \\
6	&Westin Hotel: Cocktail Room Ground Level	 &123,200.00 	&Gf	 &3-Mar-14	&\hot \\

7	&Westin Hotel: Concierge Ground Level	 &11,550.00 	&GF	 &3-Mar-14 &\hot \\
8	&Westin Hotel: Pantry Ground Level	 &24,750.00 	&GF	 &3-Mar-14	&\hot \\

9	&Westin Hotel: Westin Bar Upper Level Mezzanine Level	 &112,750.00 	&MZ &3-Mar-14 &\hot \\

10	&Westin Hotel: Food Stations Podium 1 Level	 &853,600.00 	&P1	 &22-Apr-14	& \unavailable \\

11	&Westin Hotel: Coffee Pantry Podium 1 Level	 &57,200.00 	&P1	 &22-Apr-14	& \hot \\
12	&Westin Hotel: Toilets Podium 1 Level	 &42,750.00 	&P1 &22-Apr-14	&\hot \\

13	&Westin Hotel: Kitchen Podium 1 Level	 &148,000.00 	&P1	 &10-Apr-14	&\hot \\
14	&Westin Hotel: Corridor Podium 2 Level	 &71,500.00 	&P2	 &11-Apr-14	& \hot \\

15	&Westin Hotel: Meeting Rooms Podium 2 Level	 &521,400.00 	&P2 &11-Apr-14	&\ghot \\
16	&Westin Hotel: Corridor of Meeting Rooms Podium 2 Level	 &386,100.00 	&P2 &11-Apr-14	&\ghot \\

17	&Westin Hotel: Toilets Podium 2 Level	 &47,250.00 	&P2 &11-Apr-14	&\ghot \\
18	&Westin Hotel: Kitchen Podium 2 Level	 &52,500.00 	&P2	 &29-Apr-14	&\hot \\

19	&Westin Hotel: Administration Offices Podium 4 Level	 &1,037,300.00 	&P4	 &23-Apr-14	&\ghot \\
20	&Westin Hotel: Toilets Podium 4 Level	 &34,200.00 	&P4	 &23-Apr-14	&\ghot \\

21	&Westin Hotel: W/Westin Gym/Spa Lower Level Podium 6 Level &1,125,850.00 	&P6	 &12-Apr-14	&\unavailable \\
22	&Westin Hotel: Corridor \& Lobby F01 Level	 &289,250.00 &F01	&19-Apr-14	&\unavailable \\

23	&Westin Hotel: W/Westin Gym/Spa Upper Level F01 Level	 &676,500.00 &F01	&19-Apr-14 &\unavailable \\
24	&Westin Hotel: W-Sweat Fitness Room F01 Level	 &112,750.00 	&F01	&19-Apr-14	&\unavailable \\

25	&Westin Hotel: Corridor \& Lobby for each floor in F02 to F31 Level	 &5,206,500.00 	&F31	&15-Apr-14& Completed up to 22 Floor as of 20 may 2015.	\\

26	&Westin Hotel: Presidential Suite F32 Level	 &241,800.00 	&F32	&3-May-14	 &\unavailable \\
27	&Westini Hotel: Corridor \& Lobby F32 Level	 &192,400.00 	&F32	&3-May-14	 & \unavailable\\
28	&Westin Hotel: Executive Lounge F33 Level	 &391,600.00 	&F33	&18-May-14	&\unavailable \\

29	&Westin Hotel: Corridor \& Lobby F33 Level	 &189,150.00 	&F33	&18-May-14	&\unavailable \\
30	&Westin Hotel: Royal Suite F34 Level	 &267,600.00 	&F34	&26-May-14	&\unavailable  \\

31	&Westin Hotel: Corridor \& Lobby F34 Level	 &228,150.00 	&F34	&26-May-14	&\unavailable \\
32	&Westin Hotel: Lounge and Restaurant F35 Level	 &482,350.00 	&F35	&25-May-14	&\unavailable \\
33	&Westin Hotel: Outdoor Deck F35 Level	 &102,850.00 	&F35	&25-May-14	&\unavailable \\
34	&Westin Hotel: Specialty Restaurant F35 Level	 &517,000.00 	&F35	&25-May-14	&\unavailable \\
35	&Westin Hotel: Outdoor Deck F35 Level	 &102,850.00 	&F35	&25-May-14	&\unavailable\\
\midrule
	&Total for Provisional Sums - Electrical	 &14,547,850.00 	&&&\\		
\end{pstable}




%\parindent1em
%\cxset{style13}
%\cxset{title margin bottom=10pt,
%          title beforeskip=1pt}

\chapter{Introduction}
\addtocimage{-12pt}{-20pt}{../images/tocblock-fish.jpg}


\epigraph{``Begin at the beginning,'' the king said
"and then go on till you come to the end, then stop."}{
---Lewis Carroll, Alice in Wonderland}


\noindent This package and its documentation attempts to eliminate some common 
problems encountered when using \LaTeX2e. The first one is the loading of 
recommended packages for a large and perhaps complicated document and 
the second is the re-designing of styles for a document.

 \LaTeX2e, does not provide a standard library, but comes equipped with
 a package mechanism that allows code extensions to be loaded as required.
 This has created a strong vibrant community, hundreds of packages and a 
 headache to both new and seasoned users. What packages are available, when
 to use them and in which order is a common theme for many questions on
 lists and |TX.SE|.

 It is quite common during the writing of a thesis or book
 for the author to keep on adding macros and packages
 at the preamble of the document. In most cases this can
 be satisfactory but in many others it leads to
 incompatibilities and errors. This package aims at
 minimizing one's preamble, by prefetching a number of
 commonly used packages. It also aims at loading them
 in the right order and providing patches for conflicts.
 
 I am hoping that using this package, will lead to less
 frustrations with the intricacies of \LaTeX2e\ packages.

The package code is complicated, but its usage is simple. You first load the package and then
you use one of the available templates:

 \begin{commands}[]{}
 \begin{verbatim}
 \usepackage{phd}
 \usetemplate{style13}
 \end{verbatim}
 \end{commands}

This is what you need to typeset a good looking book or thesis. The rest of this book is a footnote and you can skip them if you want. 

It will be better for the longer projects to just fork the
 package and adapt it to your needs. In this respect, I have
 uploaded the package to |github|.\footnote{\url{https://github.com/yannisl/phd}}

 My goal in selecting the packages and adding a number of 
 commands for the authors was to be able to typeset a 
 document for most common use cases, without the need of
 additional packages. The packages I selected are biased
 towards academic publications, although they can find use
 in almost any fields. The package provides a mechanism via
 PGF keys to provide a settings file. 
 
 Most of the documentation can be found in the implementation part.

Browse any books in a library or bookshop and the striking thing is that their design is very individualistic. They might have similarities but their main features vary. In many respects they resemble people's faces where minor differences have striking effects.

This package arose out of a question at stackexchange. How to redefine chapter heads. Having seen the popularity of the |pgf| package \cite{pkg-pgf} I realized that \latex users prefer this method of styling rather the traditional \latex method.

The user interface can be extended to basically all major packages. The principle is to keep to a minimum changes that can affect the LaTeX core commands. If there are any additions a key setting is provided to be able to revert back to normal LaTeX.

The workflow can be simplified. In addition I want to believe that the interface can provide a useful addition to the open source community and that other people will contribute style libraries, which will be simpler to write. It is also possible
to device an easy and uncomplicated web interface to handle
such a great number of variables.


Most people when they get started with \LaTeX\ will either use one of the standard classes such as the \docfile{book.cls} or one of the generic classes notably koma-script or memoir. Most students will be forced to use on of the many thesis classes available.

\section{The key value concept}

The key-value concept that originated with \LaTeX\ has been extended many times, the last and most serious implementation of it by Tantau in the PGF package. What essentially Tantau developed is a scripting language to script TeX code. The \tikzname and pgfplots packages are two major packaged that use keys effectively. Their popularity is growing and what this package does is to offer a user interface that has been modelled to be similar to that of \texttt{css} (cascade style sheets). 
\smallskip

\begin{scriptexample}{}{}
\textit{number} font-size = Large,\\
\textit{chapter} color = theblue
\end{scriptexample}
\smallskip

The main idea behind the package, is that you are configuring a document style by means of \emph{settings} rather than writing macros. In the example above the \emph{number, chapter} can be thought of as class or id names in css style sheets and the |font-size, color| as property settings that apply to the particular element. 


\subsection{Settings}

Settings are activated either by using the command |\cxset|  or by loading a full style sheet. In most cases you will probably import a style sheet and then modify some of the properties using |cxset|.  For example this heading has a dot after the subsection number. This was accomplished by setting,

We can de-activate it for the next and subsequent subsection headings with the setting:

\begin{scriptexample}{}{}
\begin{verbatim}
\cxset{subsection number after=\quad}
\end{verbatim}
\end{scriptexample}


\cxset{subsection number after=\quad,
          section number after=\quad,
          title margin-bottom=10pt}

\subsection{Cascading}

Most values once set for a higher section will be seen in a cascade by all subsectioning commands in a similar fashion similar to CSS. These include properties such as color, font families and alignment. Best though to specify all of them for maximum flexibility to your users.

\section{On typography}

This package hopefully will assist in improving the typography of books set with \latexe. Any typographical comments on the various styles are just my own ramblingss and not necessarily absolute truths. Like fashion and art typography has opinions rather than absolute truths. In many styles the design is slightly adapted to blend a bit better with this manual. Also I did not select fonts as per the samples but this is left on you the user to decide.



\section{Packages and Fonts}

This manual has been typeset with numerous fonts in order to enable the typsetting of almost all the scripts provided by the Unicode standard. In order to process it from the |.dtx| file, these fonts must be available in your system, otherwise \XeLaTeX\ will have a problem finding the fonts and it will take an awful long time to process. This is especially true for the scripts section, where virtually all the Unicode defined scripts are discussed. You will need a fast computer and a fast hard disk to process the document within a reasonable time. When using \pkgname{fontspec} always define your fonts with the \cmd{\newfontfamily} this will speed up processing by an order of magnitude. Compiling from the command prompt will speed up compilation. Average speed 2-3 pages per second.

Many of \tex's parameters are stretched to the limit with a complicated document such as this manual. You will require a full distribution otherwise expect some errors. Important packages is \pkgname{morefloats} and \pkgname{morewrites}. The package will also expect that you have |e-tex| installed. Ubuntu users are normally one year behind in updates, so you might wish to update manually. It will take upwards of 5 minutes to compile fully on an old laptop and a couple of minutes on a state of the art computer.

The |dtx| should be processed best with its own make file provided for Windows only |phd.bat|. The make file will process the documentation using \lualatex. You can also process the document with \xelatex but is prone to produce errors. Using \latexe the sections on scripts etc will not be printed and a much shorter version of the manual is provided. 

\section{Scripts and Languages}

The package and the documentation offer a full repertoire of font selection keys for different scripts and languages. It hasn't been possible, however hard I tried to compile this section of the documentation with \xelatex, as it kept giving errors of too many files open. This was also not possible even with the \pkgname{morewrites} package loaded. With \lualatex the document compiled with no major problems other than the font rendering being of a lower quality to that of XeLaTeX om windows, other than disabling incompatible packages and a number of commands that were redefined. 

Some good news for multi-script typesetting is the Noto fonts from Google. These fonts named Noto from "No Tofu" meaning you do not see any little square blocks for undefined glyphs, are fast to load. Disantvantage you need to switch between font commands fairly often.

\section{This book}

When developing the templates, I started using \emph{lorem ipsum} text as samples. Half-way through this
became a jumble mass of uninteresting pages interspersed with code. Headings and the contents of the book
determine both the structure and the selection of fonts, so I went back and wrote narratives  to accompany
the headings. Many of the narratives are semi-autobiographical in nature; others are clustered around books I read and my own interests. Some I stumbled on them accidentally and are mostly there to demonstrate some code.

Besides the templates and the code there is another narrative which is based on notes I kept on \tex and its friends over the years and are offered as a more advanced introduction to coding \latexe and \tex. The whole manual was typeset in a |ltxdoc| class, slightly modified to turn into a book class.

The implementation code is also available and it was mostly for my own benefit. The whole manual with the exception of the |\cxset| introduction, is just a test document. The notes and the “dissection” of the standard \latexe and the standard classes are there to explain the background to the many coding decisions that I took while I was developing the package.

PhD students are notorious for going in all directions and exploring many adjacent fields before they sit down and write their theses. Some become life-time students. To all these new men and women of the Renaissance that slave away to inch knowledge one thesis at a time, I dedicate this book and the name of the package.

\subsection{The TeX hacking sections}

To start programming \tex you need to have a knowldge of \tex basic commands and approach. \latex2015 is a format build on top of \tex to provide a more structured approach. To program \latexe packages you need to understand \latexe concepts, code organization and conventions. To program in \latex3, you need to learn a whole new language and you still need to understand \tex, \latexe and the expl3 language and conventions. To program using LuaTeX, other than the Lua language you need to understand \tex very well.
None of these can be found in one place.  I have gathered a lot of material and put it together. This is not a language you can master easily or quickly, but can teach you a lot about typesetting, computer science and many other interesting topics.


 \section{Version control with Git and Github}
 
 If you are involved with code or a publication that will have frequent changes, you should consider
 some type of version control system. My own recommendation is to use |git| and an online repository such
 as |github|. The latter is currently very fashionable and makes sharing code easier. Note that the |github|
 offers both public as well as private repositories. The general recommendation is that for unpublished work
 such as a thesis or code under development, it is preferable to go for a private repository. 
 

 \section{Ordering of Packages}
 
One package that normally leads to errors is the 
\pkgname{hyperref}. The package which is an outstanding example of software engineering and supported single handledy by Heiko Oberdiek \citeyearpar{hyperref} redefines a a lot of internal commands of the kernel. As a lot of other packages do the same it has to be loaded at the end of the preable with the exception of some packages! 
 
 This manual is typeset according to the conventions of the
 \LaTeX \textsc{docstrip} utility which enables the automatic
 extraction of the \LaTeX{} macro source files~\cite{GOOSSENS94}.

 
 \href{http://tex.stackexchange.com/questions/96350/problem-with-algorithmic-and-hyperref}{problem with algorithmic and hyperref}

 \begin{verbatim}
\usepackage{float}  % load float package first!

\usepackage{hyperref} % let hyperref patch the float package stuff
.
 \usepackage{algorithm} % let algorithm use the patched version of the float package
 \end{verbatim}
 

\section{Known problems}

Perhaps the biggest issue with the package is the speed of
compilation with \XeLaTeX\ or \LuaTeX. This is to be expected, as both engines spend a lot of resources in font management. On demand loading of packages is something I have in the back of my mind. This should be done via document styles i.e., if a book is for the humanities, perhaps only a rudimentary amount of maths packages should be loaded.

\section{Future Directions}

\latexe and \tex usage appears to be increasing. This is mostly by programs that export results with \latexe code rather than authors writing books.  The method adopted here is easier to automate all sorts of reports and automated texts. I would like too develop a web interface for processing such templates and at the same time export into html instead of just producing pdfs. I have already a prototype.   

%\ClockFramefalse\ClockStyle=0\clock{13}{10}
%\ClockFramefalse\ClockStyle=1\clock{14}{22}
%\ClockFramefalse\ClockStyle=2\clock{15}{48}
%\ClockFramefalse\ClockStyle=3\clock{7}{50}
%
%\ClockFrametrue\ClockStyle=0\clock{11}{32}
%\ClockFrametrue\ClockStyle=1\clock{12}{0}
%\ClockFrametrue\ClockStyle=2\clock{8}{9}
%\ClockFrametrue\ClockStyle=3\clock{1}{15}

%{\HHHUGE\showclock{0}{45}}










%
%\makeatletter\@specialfalse\makeatother

\cxset{
 %toc image =,
 chapter format = block,,
 chapter name=Chapter,
 chapter numbering=arabic,
% number font-size= LARGE,
% number font-family= rmfamily,
% number font-weight= bfseries,
% number before=,
% number dot=,
% number after=\par\offinterlineskip,
% number position=leftname,
% chapter font-family= sffamily,
% chapter font-weight= normalfont,
% chapter font-size= Large,
% chapter before={\vspace*{15pt}\par},
% chapter after=,
% number color=black!90,
 %
 chapter title margin-top=30pt,
% title margin bottom=20pt,
 chapter align=left,
 chapter title align=RaggedRight,
 chapter title width=\textwidth,
%
% chapter title before={},
 chapter title font-family= sffamily,
 chapter title color= black,
 chapter title font-weight= bfseries,
 chapter title font-size= LARGE,
% title spaceout=none,
 header style= plain,
 section font-size=Large,
 section color=black,
 section number prefix=\thechapter.,
 section indent=0pt,
 }


\chapter{Introduction to Style One}
\addcontentsline{toc}{section}{Template 1 (style01)}
\cxset{headings ruled-01}

\begin{summary}
This design is simple and its distinguishing characteristic is a short summary at the beginning of the chapter. This is almost like an abstract typeset in italic font without setting the margins in. We provide a \lstinline{summary} environment for convenience. Note the very simple line in the running head to the left of the page number.
\end{summary}

\medskip
\begin{figure}[ht]
\centering
\includegraphics[width=\textwidth]{./chapters/chapter01.jpg}
\end{figure}

The summary after the chapter head can easily be incorporated using \textit{summary}. You
can also use \string\begin\meta{abstract}. The latter will produce a heading with the word, abstract.
Both the summary as well as the abstract can take parameters to be set, for internationalization and to typeset
the words abstract, or summary. If you use \textit{precis}, the summary will be added into the Table of Contents as
well.

I originally picked this style, as a boringly easy style to develop, but it proved a hard nut to crack when it came to sections. Both the section numbering as well as the caption of figures, proved to be difficult to style using
the build-in \latexe commands. 

\section{Images and Figures}

Images and figures are using traditional captions with the exception they are restricted in a certain portion of the textwidth. The word Fig. is abbreviated to Fig. and uses a dot.

\captionsetup[figure]{format = plain,
                                 width=.67\textwidth,
                                 justification=justified,
                                 singlelinecheck=false,
                                 name=Fig.,
                                 labelsep=space,
                                 oneside,
                                 margin=0pt
                                 }

\begin{figure}[ht]
\includegraphics[width=\textwidth]{./images/elementary-images.jpg}
\caption{Note the abbreviation and the restriction of the caption to\\
 a minipage. This combined with the width option manages\\
  the typesetting well.}
\end{figure}

I had to make a special style to capture this. It is unbelievable that a piece of textblock can get so complicated and this particular style, let me to re-think some of the concepts in the phd design. the complication that arises here
is that with most images measuring the image is necessary. The narrower image in the figure would of course not work on the same settings, and the caption is at the full width of the figure, as shown below.













\makeatother

%\clearpage
\makeatletter\@debugtrue\makeatother
\cxset{
% chapter toc=true,
 name=CHAPTER,
 chapter numbering=ORDINALS,
% number font-size=Large,
% number font-family=rmfamily,
% number font-weight=bfseries,
% number before=\kern0.5em,
% number dot=,
% number after=\hfill\hfill\par,
% number position=rightname,
% number color=black,
 chapter font-family=rmfamily,
 chapter font-weight=bold,
 chapter font-size=Large,
 chapter before={\vspace*{20pt}\par\hfill},
 chapter after=,
 chapter color=black,
 %
 title margin top=10pt,
% title before=\par\nointerlineskip\hfill,
% title after=\hfill\hfill\par\nointerlineskip,
 chapter title font-family=rmfamily,
 chapter title font-color= black,
 chapter title font-weight=bfseries,
 chapter title font-size=LARGE,
 chapter title width=0.8\textwidth,
 chapter title align=centering,
% title margin-left=0pt,
 author block=false}

\debugtitle
\debugchapter
\chapter[Template 2]{Mondino, the Restorer of Anatomy}

The archive.org is an extraordinary hunting ground  for typographical surprises. On a recent excursion to find some books on Versalius I stubled on a book titled \emph{Andreas Vesalius, the reformer of anatomy} by  Ball, James Moores. It is an old book published in 1910 and has a couple of unusual features. Check the figure below and see if you can identify the challenging feature.

\begin{figure}[ht]
\centering
\includegraphics[width=0.8\textwidth]{versalius}
\caption{J.B. Moore’s \emph{Andreas Versalius, the Reformer of Anatomy} has many unusual features, including chapter numbers using ordinals. }
\end{figure}

\cxset{chapter toc=true,
          chapter opening=anywhere}
          
\chapter{The Template}          
The template is called \emph{Versalius} and is stored under style02. It can be loaded in the normal way using:
\begin{verbatim}
\usepackage[style02]{phd}
\end{verbatim}

I have not reproduced the full extend of the book’s requirements, as some details are quite cumbersome to be automated through \tex. These though can easily be incorporated in a manual way. More about this later.


\section{The Table of Contents}
Another interesting aspect of this book, which is common with many books of its period is the ToC. The ToC shows the full range of the chapter pages, i.e., it is marked as Page 1-16 rather than the common practice nowdays that indicates only the starting page of the chapter. It also has “TABLE OF CONTENTS”  as a heading and not just contents as you would expect from today’s books.

\begin{figure}[ht]
\centering
\includegraphics[width=0.8\textwidth]{versalius-01}
\caption{J.B. Moore’s \emph{Andreas Versalius, the Reformer of Anatomy} has many unusual features, including chapter numbers using ordinals. }
\end{figure}

\section{List of Illustrations}

\begin{figure}[ht]
\centering
\includegraphics[width=0.8\textwidth]{versalius-02}
\caption{J.B. Moore’s \emph{Andreas Versalius, the Reformer of Anatomy} has many unusual features, including chapter numbers using ordinals. }
\end{figure}

\section{The Frontmatter}
As a foreward there is an unumbered chapter called ``Introduction’’. The chapter heading also has a head band.
\begin{figure}[ht]
\centering
\includegraphics[width=0.8\textwidth]{versalius-03}
\caption{J.B. Moore’s \emph{Andreas Versalius, the Reformer of Anatomy} has many unusual features, including chapter numbers using ordinals. }
\label{lettrine}
\end{figure}

\bgroup
\centering
\includegraphics[width=0.7\textwidth]{versalius-headband}

\LARGE\bfseries INTRODUCTION\par
\egroup
\def\dropcapversalius{%
\vbox to 0pt{\vskip6pt\leavevmode\noindent\includegraphics[width=2.39cm]{versalius-dropcap}%
}%
}
\parindent0pt

\hangindent2.6cm \hangafter0
\dropcapversalius \textsc{he dropcap will have to be inserted}, either using the lettrine package or do be achieved via a parshape command and manual entry. You can also write your own macro command using the details we provide under the Paragraphs chapter. On this page I have manually inserted it, as I used an image from the book for the dropcap. If you were to use the template for a full book, it will be then preferable to use
the lettrine package to set the dropcaps. If you observe Figure~\ref{lettrine} carefully, you will notice the first line of theopening paragraph is in small caps. As \tex typesets the full paragraph this is almost an impossible task to achieve through normal \tex commands and in order not to overcomplicate the discussion it can be achieved manually via trial and error. 

\section{Figures}

Most of the figures are wrapped illustrations. A couple are full page figures and bear no caption numbering. One such illustration is shown on page~\pageref{fig:vesalius}. Do note that the List of Illustrations does have the illustrations listed with additional information to that shown in the captions. 

\begin{figure}[p]
\centering
\includegraphics[width=\textwidth]{vesalius}
\centering
ANDREAS VESALIUS\par
(From an old copperplate engraving)\par
\label{fig:vesalius}
\end{figure}







%\input{./styles/style03}
%\input{./styles/style04}
%
\cxset{style05/.style={
 chapter opening=any,
% chapter toc=true,
 chapter name=Chapter,
 chapter color=black!90,
 chapter numbering=arabic,
% number font-size=Large,
% number font-family=rmfamily,
% number font-weight=\normalfont\itshape,
% number color= black,
% number before=\kern0.5em,
% number dot=,
% number after=\hfill\hfill\par,
 number position= rightname,
 chapter shape=none,
 chapter display=block,
 chapter float=center,
 chapter font-family=rmfamily,
 chapter font-weight= itshape,
 chapter font-size=Large,
 chapter before=\hrule width \columnwidth \kern12.6pt \par\hfill,
 chapter after=,
 chapter color=black!90,
 chapter spaceout=none,
% handle margins and padding
% title margin top=10pt,
% title margin bottom=10pt,
% title before=,
% title after=\vskip12.6pt\hrule width \columnwidth,
% title font-family=rmfamily,
% title font-color=black!90,
% title font-weight=bfseries,
% title font-size=huge,
 chapter title align=centering,
 title font-shape = normal,
 header style= headings}}

\cxset{style05}
\chapter{Introduction to Style Five}\index{styles!style5}

\tcbset{width=\textwidth}
I think this style can be improved with a bit of color. You can experiment with it quite easily. The spacing on top of this style can also be adjusted to suit your typographical taste.
\medskip
\begin{figure}[ht]
\centering
\includegraphics[width=0.6\textwidth]{./chapters/chapter05.png}
\end{figure}

%\section{General notes on rules}

LaTeX's default rules would normally give problems. Best is to use TeX's primitives to built them.

\index{rules!example color}

\begin{texexample}{}{}
\makeatletter
\hrule width 5cm \kern2.6\p@
AAAAAAAAAAAAAAAAAAAAA
\vskip2.6pt\hrule width 5cm
\medskip

Problem with LaTeX rules.

\rule{5cm}{0.4pt}\par
AAAAAAAAAAAAAAAAAAAAA\par%
\rule[6.5pt]{5cm}{0.4pt}

\def\rule{\@ifnextchar[\@rule{\@rule[\z@]}}
\def\@rule[#1]#2#3{%
 \leavevmode
 \hbox{%
 \setlength\@tempdima{#1}%
 \setlength\@tempdimb{#2}%
 \setlength\@tempdimc{#3}%
 \advance\@tempdimc\@tempdima%
 \vrule\@width\@tempdimb\@height\@tempdimc\@depth-\@tempdima}}

\def\thickrule{\leavevmode \leaders \hrule height 3pt \hfill \kern \z@}

{\color{teal}\hrule width 10.5cm height3pt \kern2.6\p@
    {{\color{black!80}\HUGE CHAPTER TITLE}}\vskip3pt
\hrule width 10.5cm height3pt}
\makeatother
\end{texexample}


\section{Images}

\begin{figure}[htbp]
\centering
\includegraphics[width=0.8\textwidth]{telegraphy}
\caption{Spread from the \textit{History of Telegraphy}, the caption is set left and the image is centered.}
\end{figure}

%\makeatletter\@specialtrue\makeatother
%\cxset{custom = stewart}
%\cxset{steward,
%  numbering=arabic,
%  custom=stewart,
%  offsety=0cm,
%  image={./images/hine03.jpg},
%  texti={When Lamport designed the original \LaTeX\ sectioning commands, limitations of computer power forced him to restrict the abstraction of complicated chapter layouts. With current tools available improvements are much easier to program.},
%  textii={In this chapter we discuss a method that allows the production of fancy section headings and formatting, based on a set of key values. Central  to this process is the separation of content from presentation.
%We also discuss the basic formatting tools that are available and how one can modify them to mould new book designs.
% }
% }



\chapter{Lower Level Headings}


\section{Introduction}

Good book design dictates that sectioning styles follow that the general book design and theme. An academic publication for example might have chapters and section numbered in arabic numerals, whereas a high school textbook might have sections marked in colored boxes. Most traditional books had very humble headings,
set in black ink and the reason was economics. Nowdays most publications will be read online and the use
of color can be useful.

Similarly to the chapter key value interface, the package offers a key value interface to adjust sectioning command parameters.



\cxset{section afterskip={10pt}}

\section{Section styling}

In a similar fashion to the chapter commands the following keys are provided.

\subsection{Fonts and numerals}

Font and numeral keys are shown below.
\medskip
\begin{docKey}[phd]{section font-size}{ = \marg{sizing commands}} {no default, intial=Large}
The font-size command takes arguments
of the  type |Large|, |large| both as commands or without the backslash, which is the recommended way
of setting styles with the |phd| package. 
\end{docKey}

\begin{docKey}[phd] {section font size} {= \marg{sizing commands}} {normal size} 
All the font commands, come in two flavours,
with a hyphen or without, in order to present a user interface that is similar to |pgf/TikZ| conventions for that
are familiar with \latex and another for those used to |CSS| conventions.
\end{docKey}

\begin{docKey}{/phd/section font-family}{= \marg{sizing commands}}{no default, initial value normal} The font-family key, accepts normal LateX values
related to families, but if LuaTeX or XeLaTeX are present it can also accept commands created with |\newfontfamily| 
command of the |fontspec| package, which is loaded automatically by the |phd| package. The package has a database of a number of human friendly names for fonts and commands. If one of these are detected the
family is created at run-time to avoid overloading too many fonts at start-up. 
\begin{verbatim}
\cxset{section font-family = Arial}
\cxset{section font-family = sffamily}
\cxset{section font-family = ttfamily}
\end{verbatim}
The family command family name (if undeined by the user), defaults to the human friendly version name but without the spaces. 
\end{docKey}

%
%  \keyval{section font-weight}{\marg{cmd}}{Font weight command such as \cs{bfseries.}}
%  \keyval{section font-family}{\marg{cmd}}{Font family command such as \cs{sffamily.}}
%  \keyval{section font-shape}{\marg{cmd}}{Font shape command such as \cs{itshape}}
%  \keyval{section color}{\marg{color}}{Color of section.}
%  \keyval{section numbering}{\marg{arabic|roman|Roman|alph|Alph|words|WORDS}}{Section number style.}
  \begin{marglist}
  \item [arabic] Typesers the section number in arabic numerals.
  \item [roman] Typesets the section number in lowercase roman numerals.
  \item [Roman] Typesets the section number in uppercase roman numerals.
  \item [alph] Typesets the section number in lowercase alphabetic numbering.
  \item [Alph] Typesets the section number in uppercase alphabetic numerals.
  \item [words] Typesets the numbers in words (lowercase).
  \item [WORDS] Typesets the number in words (uppercase).
  \end{marglist}

\subsection{Skip and indentation commands}

The keys for indentation and above and below skips are shown below.
\medskip

\keyval{section beforeskip}{}{}
\keyval{section afterskip}{}{}
\keyval{section indent}{\marg{dim}}{Indentation from margin as per standard LaTeX class definitions.}
\keyval{section spaceout}{}{}
\begin{marglist}
 \item[soul]
 \item[none]
\end{marglist}



\subsection{align}

\keyval{section align}{\marg{cmd}}{One of the alignment commands centering, ragged right, raggedleft}

\subsection{Hooks}

Hooks for adding material are shown in the following sketch.
\medskip

\fbox{aboveskip}

\fbox{indent} \fbox{number}\fbox{hook}\fbox{title}

\fbox{belowskip}


\section{Example usage}

In our first example we will use a predefined style for the chapter headings, so we do not need to clutter the example with the chapter commands that we have previously discussed. Our first example will number the section in lower roman, enclosed in brackets and center it.


\makeatletter\@specialfalse
\cxset{
% chapter toc=false,
% chapter  name=CHAPTER,
% numbering=arabic,
% number font-size=huge,
% number font-family=sffamily,
% number font-weight=bfseries,
% number before=,
% number dot=,
% number after=\hspace{1em},
% number position=rightname,
% chapter opening=anywhere,
% chapter font-family=sffamily,
% chapter font-weight=bfseries,
% chapter font-size=huge,
% chapter before={\vspace*{0.1\textheight}\hfill},
% chapter after={\hfill\hfill\vskip0pt\thinrule\par},
% chapter color=black!90,
% number color= black!90,
% title beforeskip={\vspace*{30pt}},
% title afterskip={\vspace*{30pt}\par},
% title before={\hfill},
% title after={\hfill\hfill},
% title font-family=\sffamily,
% title font-color= black!90,
% title font-weight=bfseries,
% title font-size=huge,
 section font-size= LARGE,
 section font-weight= bold,
 section font-family= sffamily,
 section align= centering,
 section numbering=arabic,
 section indent=0em,
 section align= centering,
 section beforeskip=20pt,
 section afterskip=10pt,
 section font-shape= itshape,
}

\cxset{book/.style={
 section numbering=arabic,
 section font-size=Large,
 section font-weight=bfseries,
 section font-family=rmfamily,
 section font-shape=normalfont,
 section align=\raggedright,
 subsection font-size=\large
 section indent=0em,
 section beforeskip=-3.5ex \@plus -1ex\@minus -0.2ex,
 section afterskip=2.3ex\@plus.2ex,
 subsection beforeskip=-3.5ex \@plus -1ex\@minus -0.2ex,
 subsection afterskip= 1.5ex \@plus .2ex,
}}
\makeatother


\begin{texexample}{Adjusting section parameters}{ex:sec}
\cxset{ section font-size= LARGE,
 section font-weight= bold,
 section font-family= sffamily,
 section font-shape=upshape,
 section numbering=(roman), 
 section indent=0em,
 section align= centering,
 section beforeskip=20pt,
 section afterskip=10pt,,
 section align=right}
\chapter{A First Look at the Sectioning Keys}
\section{First section}
\lorem
  % adjust counter number so it does not affect the
  % rest of the document
\addtocounter{section}{-1}
\end{texexample}


The keys are mostly self-explanatory. We have used a |beforeskip| and |afterskip| without any glue. The numbering is just a continuation of the document sections. 

One notable thing to keep in mind is that the numbering of the chapter is independent of that for the section, so if you need to have strange combinations rather define a section numbering custom.\index{section formatting>vertical space}


\cxset{section numbering=arabic}
\subsection{Adjusting vertical spaces}

Perhaps the most important issues we need to consider is the adjusting of vertical spaces; example~\ref{ex:latex}, that follows illustrates settings from the Octavo class and compare them with those of standard the \LaTeXe\ book class. The Octavo class through settings that are based on baselineskip fractions and multiples endeavours to achieve a grid layout. The class also tones down the `loudness' of some of the headings compared to those of the book class.

\makeatletter
\cxset{octavo/.style={
 section font-size=large,
 section font-weight=,
 section font-family=rmfamily,
 section font-shape=scshape,
 section indent=0em,
 section align=\centering,
 section beforeskip=-1.666\baselineskip\@minus -2\p@,
 section afterskip=0.835\baselineskip \@minus 2\p@,
 section after indent = false,
 subsection numbering=none,
 subsection font-family= rmfamily,
 subsection font-size=,
 subsection font-shape=scshape,
 subsection font-weight=,
 subsection indent=1em,
 subsection align=RaggedRight,
 subsection beforeskip=-0.666\baselineskip\@minus -2\p@,
 subsection afterskip=0.333\baselineskip \@minus 2\p@,
 subsection color=spot!50,
 subsubsection color=spot!50,
 }}


\cxset{book/.style={
 section numbering=arabic,
 section font-size= Large,
 section font-weight= bfseries,
 section font-family= rmfamily,
 section font-shape= upshape,
 section align= RaggedRight,
 subsection font-size= large,
 section indent=0em,
 section beforeskip=-3.5ex plus -1ex minus -0.2ex,
 section afterskip=2.3ex plus 0.2ex,
 subsection font-size= large,
 subsection font-weight= bfseries,
 subsection numbering=arabic,
 subsection indent=0pt,
 subsection beforeskip=-3.5ex \@plus -1ex\@minus -0.2ex,
 subsection afterskip= 1.5ex \@plus .2ex,
}}

\cxset{octavo headings/.style={
 section numbering=none,
 section font-size=Large,
 section font-weight=,
 section font-family=rmfamily, section font-shape= scshape,
 section indent=0em, 
 section align=centering, 
 section afterindent=off,
 section beforeskip=-1.666\baselineskip\@minus -2\p@,
 section afterskip=0.835\baselineskip \@minus 2\p@, 
 %
 subsection numbering=none,
 subsection font-family=\rmfamily, 
 subsection font-size=, subsection font-shape=scshape,
 subsection font-weight=, subsection indent=1em, 
 subsection align= RaggedRight,
 subsection beforeskip=-0.666\baselineskip\@minus -2\p@,
 subsection afterskip=0.333\baselineskip \@minus 2\p@,
 subsubsection numbering=none,
 subsubsection font-family= rmfamily,
 subsubsection font-size=,
 subsubsection font-shape= itshape,
 subsubsection font-weight=,
 subsubsection indent = 0em,
 subsubsection align= raggedright,
 subsubsection beforeskip=-0.666\baselineskip\@minus -2\p@,
 subsubsection afterskip=0.333\baselineskip \@minus 2\p@,
 subsubsection color=spot!50,
 paragraph numbering=none,
 paragraph font-family= rmfamily,
 paragraph font-size=,
 paragraph font-shape=itfamily,
 paragraph font-weight=,
 paragraph color = spot!50,
 paragraph indent=0em,
 paragraph align= RaggedRight,
 paragraph beforeskip=10pt,
 paragraph afterskip=1em,
}}
\makeatother

\cxset{octavo headings}


%\begin{texexample}{Octavo class headings, settings}{}
%\cxset{octavo headings/.style={
% section numbering=none,section font-size=large,
%section font-weight=,
% section font-family=rmfamily, section font-shape=scshape,
% section indent=0em, 
% paragraph numbering=none,
% paragraph font-family=rmfamily,
% paragraph font-size=,
% paragraph font-shape=,
% paragraph font-weight=,
% paragraph indent=-1em,
% paragraph align=raggedright,
% paragraph beforeskip= 0pt,
% paragraph afterskip=0pt,
%}}
%
%\cxset{octavo headings}
%\renewsection\renewsubsection\renewsubsubsection
%\section{Octavo Class Heading}
%\lorem
%\subsection{Octavo subsection}
%This is some text short text\par
%\subsubsection{Octavo sub-subsection}
%\lorem
%\paragraph{paragraph heading} This is some short text.
%\makeatother
%\end{texexample}

\begin{comment}
The following example was set using the |style| |\cxset{Octavo headings}| with some minor adaptations to enable us to show it inline with the rest of the material on this page\footnote{We set it using \cs{cxset}\marg{chapter opening = anywhere}}. We kept the use of a typical colour throughout the text, whereas the Octavo class, does not allow the use of color.

\cxset{chapter opening = anywhere,
          chapter color = spot!50,
          title font-color = spot!50,
          chapter name={},
          chapter numbering = none,
          chapter before = \addvspace{\baselineskip},
          chapter after = ,
          title spaceout=soul,
          title before =,
          title afterskip=\bigskip\bigskip,
          number before=,
          number after=,
          }
          
\bgroup
\parindent=0pt
\par

\chapter{Octavo Chapter Heading}
\lorem

\section{Octavo Class Heading (Section) }
\lorem

\subsection{Octavo subsection}
\lorem

\subsubsection{Octavo sub-subsection}
\lorem

\paragraph{Paragraph heading} This is some short text.
\lorem

\paragraph{paragraph heading} This is some short text.
\lorem

\egroup
\end{comment}

\begin{texexample}{\LaTeXe\ book class headings settings}{ex:latex}
\makeatletter
\cxset{book/.style={
 section numbering prefix = \thechapter.,
 section numbering=arabic,
 section number after=,
 section font-size= Large,
 section font-weight=bfseries,
 section font-family=rmfamily,
 section font-shape=upshape,
 section align=RaggedRight,
 section beforeskip=10pt,
 section spaceout = none,
 section color  = red,
 subsection font-size=large,
 section indent=0em,
 section beforeskip=-3.5ex plus1ex minus0.2ex,
 section afterskip=2.3ex\@plus.2ex,
 subsection color = blue,
 subsection font-size=large,
 subsection font-shape=upshape,
 subsection font-weight=bfseries,
 subsection numbering prefix=\thesection.,
 subsection numbering = arabic,
 subsection beforeskip=-3.5ex \@plus -1ex\@minus -0.2ex,
 subsection indent= 0pt,
 subsection afterskip= 1.5ex \@plus .2ex,
}}

\cxset{book}


\section{LaTeX Book  Class Heading}
\lorem
\subsection{A subsection}
\lorem
\makeatother
\end{texexample}



\section{Grid example}

One problem sometimes is that the sectioning commands create problems with grid layouts. Example~\ref{ex:grid} shows example settings.

\begin{texexample}{Section styles from the grid package}{ex:grid}
\makeatletter
\cxset{grid/.style={
 section numbering=arabic,
 section font-size=,
 section font-weight=bfseries,
 section font-family=rmfamily,
 section font-shape=upshape,
 section beforeskip=-.999\baselineskip,
 section afterskip=0.001\baselineskip,
 section align= RaggedRight,
 subsection font-size=,
 section indent=0em,
 subsection font-shape=,
 subsection font-weight=bfseries,
 subsection numbering=arabic,
 subsection indent=0pt,
 subsection beforeskip=1\baselineskip,
 subsection afterskip= -.35\baselineskip,
 subsubsection font-shape=itshape,
 subsubsection font-weight=bfseries,
 subsubsection numbering= none,
 subsubsection indent=0pt,
 subsubsection beforeskip=1\baselineskip,
 subsubsection afterskip= -.35\baselineskip,
}}
\cxset{grid}




\begin{multicols}{2}
\section{Grid  Class Heading}
\lorem
\subsection{Grid  subsection.}
\lorem
\subsubsection{A subsection grid.}
\lorem
\subsubsection{Another subsection grid.}
\lorem
\end{multicols}
\makeatother
\end{texexample}



The key \option{\bfseries section numbering custom}=\marg{code} is quite powerfull and can be used to define any type of section number style. Just remember that the numbering so far depends on two counters, the c@chapter and c@section. What the section numbering does, it redefines the macro \cs{thesection} to the new definition provided as argument for the key.

Although the temptation to define a lot of key combinations one would rather define new styles as a more user friendly approach.

\cxset{section numbering=arabic, section align= RaggedRight, section font-shape=upshape, section font-family=rmfamily}
\section{Handling Other Section Levels}

Other sectioning commands such as \cs{subsubsection}, \cs{paragraph} and \cs{subparagraph} have equivalent keys. Examples can be found in the chapters that follow for specific styles.

\section{Technical discussion}

The standard LaTeX classes, book report and article have sections showing dot leaders, whereas in the article class the sections are shown without the dotted lines, as the |\l@section| macro is redefined for articles. With the \pkgname{phd} the distinction is unecessary and style files can do the trick that is, either load style article or book or for that matter any other style that has the relevant settings.

\index{macros!\textbackslash @seccntformat}

\subsection{Lower Section Headings}

\LaTeXe\ offers two pathways in redefining section commands, the first one is \refCom{@startsection} and the second is \refCom{@seccntformat} \index{sectioning macros}. It also uses the macro \cs{secdef} to create the starred and unstarred versions of the sectioning commands.

 In the article document class the entry in the table of contents
 for sections looks much like the chapter entries for the report
 and book document classes.
\begin{tcolorbox}{}
\begin{lstlisting}
% \begin{macro}{\l@section}

%
%    First we make sure that if a pagebreak should occur, it occurs
%    \emph{before} this entry. Also a little whitespace is added and a
\newcommand*\l@section[2]{%
  \ifnum \c@tocdepth >\z@
    \addpenalty\@secpenalty
    \addvspace{1.0em \@plus\p@}%
%    \end{macrocode}
%
%    The macro |\numberline| requires that the width of the box that
%    holds the part number is stored in \LaTeX's scratch register
%    |\@tempdima|. Therefore we put it there. We begin a group, and
%    change some of the paragraph parameters (see also the remark at
%    \cs{l@part} regarding \cs{rightskip}).
%    \begin{macrocode}
    \setlength\@tempdima{1.5em}%
    \begingroup
      \parindent \z@ \rightskip \@pnumwidth
      \parfillskip -\@pnumwidth
%    \end{macrocode}
%    Then we leave vertical mode and switch to a bold font.
%    \begin{macrocode}
      \leavevmode \bfseries
%    \end{macrocode}
%    Because we do not use |\numberline| here, we have do some fine
%    tuning `by hand', before we can set the entry. We discourage but
%    not disallow a pagebreak immediately after a section entry.
%    \begin{macrocode}
      \advance\leftskip\@tempdima
      \hskip -\leftskip
      #1\nobreak\hfil \nobreak\hb@xt@\@pnumwidth{\hss #2}\par
    \endgroup
  \fi}
%</article>
\end{lstlisting}
\end{tcolorbox}



As you can see the dot leaders are not present in the above definition. Although we can get rid of dot leaders in other section by redefining them, it is not as easy to add them back.

As our aim is to be able to have all the classes used a common denominator we can define a command as follows (using book as a base)

\begin{tcolorbox}{}
\begin{lstlisting}
\def\articlesection{
\newcommand*\l@section[2]{%
  \ifnum \c@tocdepth >\z@
    \addpenalty\@secpenalty
    \addvspace{1.0em \@plus\p@}%
    \setlength\@tempdima{1.5em}%
    \begingroup
      \parindent \z@ \rightskip \@pnumwidth
      \parfillskip -\@pnumwidth
      \leavevmode \bfseries
      \advance\leftskip\@tempdima
      \hskip -\leftskip
      #1\nobreak\hfil \nobreak\hb@xt@\@pnumwidth{\hss #2}\par
    \endgroup
  \fi}
}
\end{lstlisting}
\end{tcolorbox}


\begin{docCommand}{@startsection}{}
The \cs{@startdsection} macro is one of those locomotive type of commands. It takes 7 required arguments and 2 optional ones and hidden within it are two booleans. The full set looks like this:

\cs{@startsection} \marg{name} \marg{level} \marg{indent} \marg{beforeskip} \marg{afterskip} \marg{style}[*]
  [\marg{altheading}]\marg{heading}.
\end{docCommand}

\begin{marglist}
\item[name] The name of the level command.
\item [level] A number denoting the depth of the section, chapter=1, section=2, etc. A section number will be printed only if \marg{level} is equal or smaller than the value of \textit{secnumdepth}
\item[indent] The indentation of the heading from the left margin.
\item[beforeskip]  The absolute value of this argument is the skip to leave above the heading. If it is negative, then the paragraph indent of the text following the heading is suppressed.
\item [afterskip] If positive, it is the skip to leave below the heading, else it is the skip to the right of a run-in heading.
\item [style] Sets the style of the heading.
\item[\textup{[*]}] When this is missing the heading is numbered and the corresponding counter is incremented.
\item[\textup{[\textit{altheading}]}] Gives an alternative heading to use in the table of contents and in the running heads. This should be present when the * form is used.
\item[heading] The heading of the new section.
\end{marglist}

%\begin{texexample}{Example formatting run-in section}{}
%\makeatletter
%\bgroup
%\renewcommand\section{
%    \@startsection{section}
%    {1}
%    {0em}
%    {-0.8em}
%    {-0.5em}
%    {\large\normalfont\scshape}}
%\makeatother
%\section[]{test}
%\lorem
%\egroup
%\end{texexample}



Note we run the example in a group so that we will not influence the formatting of this document.

As mentioned earlier there is an additional way to introduce formatting for sections and this is using the command \cs{@seccntformat}, which is responsible for typesetting the counter part of a section title. The default definition of the command typesets the \cs{the} representation of the section counter.

%\begin{texexample}{}{}
%\bgroup
%\renewcommand\section{%
%    \@startsection{section}%
%    {1}%
%    {0em}%
%    {-0.8em}%
%    {-0.5em}%
%    {\large\normalfont\scshape}}
%\renewcommand\@seccntformat[1]{\fbox
%{\csname the#1\endcsname}\hspace{0.5em}}
%\makeatother
%\section[]{test}\label{sec:ok}
%\lorem
%
%See section \ref{sec:ok}.
%\egroup
%\end{texexample}



\cxset{section color=spot!50,
          subsection color = spot!50 }
          
\section{Custom headings}

\begin{docCommand*}{@secdef}{}
So far we have used the |phd|’s keys to set keys that are affecting the standard commands used by
\latexe to set headings. Another way to achieve this,  is to use the macro
 \cs{@secdef}. Therefore, if you wish to use different definitions of \cs{@seccntformat}
for different headings, you must put the appropriate code into every heading
definition.
\end{docCommand*}



\begin{phdverbatim}
\newcommand\part{\secdef\starcmd\unstarcmd}
\end{phdverbatim}

The |part| and |chapter| and sometimes |appendix| are defined this way, but nothing stops us from doing the same for other sectioning commands. What the \cs{secdef} command does it will produce the definitions required for a star or unstarred version of the sectioning command, such as |\section|.\footnote{See \ttfamily File F: ltsect.dtx Date: 2014/09/29 Version v1.0z 360} 

\begin{texexample}{}{}
\bgroup
\makeatletter
\renewcommand\section[2] [?]{%
    \refstepcounter{section}
    \addcontentsline{toc}{section}
    {\protect\numberline{section-\thesection}#1}
    {\raggedright\large\bfseries SECTION-\thesection\par \centering#2\par}
    \sectionmark{#1}
    \@afterheading 
   \addvspace{\baselineskip}
 }%
\section[test]{Section Heading}
\lorem
\makeatother
\egroup
\end{texexample}

Many other strategies can also be implemented that are perhaps easier to grasp.

\begin{teX}
\def\@seccntformat##1{\csname the##1\endcsname{}}
\end{teX}

\begin{comment}
\begin{texexample}{}{}
\makeatletter
\bgroup
\def\strut{\vrule height12pt depth1pt width0pt}
  \renewcommand\section[2] []{% % Complex form:
  \refstepcounter{section}% % step counter/ set label
  \addcontentsline{toc}{section}% % generate toc entry
  {\protect\numberline{\thesection} }%
  {\raggedright\large\bfseries\scshape %
  \parbox[b]{\dimexpr(\linewidth-0.5\columnsep)}{\colorbox{brown!80}%
  {{\vbox{\strut\raise2pt\hbox{#2}}}}}}\vskip0pt% % and number
  \sectionmark{#1}% % add to running header
  \@afterheading % prepare indentation handling
  \vspace{\dimexpr\baselineskip+6pt}%must have a parameter
}
\chapter{Fossil Insects}
\begin{multicols*}{2}\raggedcolumns
\section[Insect Fossilization]{\raggedright \thinspace Insect Fossilization}
\lipsum[1]
\end{multicols*}
\egroup
\makeatother
\end{texexample}
\end{comment}

Of course some work is needed to center the text properly in the middle of the colour box. For all practical purposes it is lining up as per the sample.

In Chapter we discussed a forward, but this may not apply if there are no chapters or we need to treat these as sections, the example \ref{ex:forwardsection} shows such a method.


\begin{texexample}{Defining a Foreward Section}{ex:forwardsection}
\makeatletter
\newcommand\prematter@sp[1]{
\addcontentsline{toc}{section}
{\protect\numberline{}#1}
\sectionmark{#1}
{\LARGE\centering\normalfont\sffamily\colorbox{brown!80}{ \textsc{#1}}\par}%
\@afterheading
\addvspace{\baselineskip}
\@afterindentfalse
}

\newenvironment{prematter}[1]{%
   \prematter@sp{#1}}
{}
\begin{multicols}{2}
\label{theok}
\begin{prematter}{Foreward}
\lipsum[1]
\end{prematter}\ref{theok}
\end{multicols}
\makeatother
\end{texexample}


\section{underlining}

I am aware that some people have no choice but have some sections underlined as dictated by archaic regulations in some establishments for thesis submission. If nobody is forcing you to underline it is best to avoid it. We use Donald Arsenau's ulem package to achieve underlining. \footnote{\protect\url{http://tex.stackexchange.com/questions/52998/change-title-to-small-caps-but-not-in-toc}}
\endinput

\makeatletter
\gdef\sectionopen{}
\def\@sectionsuffix{}
\def\@sectionprefix{\sectionname\space}
\newif\if@sectioncase \@sectioncasefalse

\cxset{
  section special/.code =\def\specialsection@cx{#1},
  section xcolor/.store in = \sectionxcolor@cx,
  section opening/.is choice,
  section opening/openany/.code=\gdef\sectionopen{\clearpage},
  section opening/right/.code = \gdef\sectionopen{\cleardoublepage},
  section opening/none/.code = \gdef\sectionopen{},
  section top rule/.is choice, 
  section top rule/true/.code =\DeclareRobustCommand\sectiontoprule{%
        \leavevmode\par\noindent\rule{\textwidth}{1pt}\vskip3.5pt},
  section top rule/true/.code=\def\sectiontoprule{\leavevmode\par\noindent\tikzrule},      
  section top rule/false/.code=\gdef\sectiontoprule{},
  % bottom rule
  section bottom rule/.is choice, 
  section bottom rule/true/.code =\DeclareRobustCommand\sectionbottomrule{%
        \leavevmode\par\noindent\rule{\textwidth}{1pt}\vskip.5pt},
  section bottom rule/true/.code=\def\sectionbottomrule{\vskip-0.5\baselineskip\rlap{\tikzrule}},      
  section bottom rule/false/.code=\gdef\sectionbottomrule{},
  % upper and lower case - TODO in lua
  section case/.is choice,
  section case/lower/.code=\def\sectioncase@cx{\@sectioncasetrue
                             \if@sectioncase\expandafter\MakeTextLowercase\fi},
  section  case/upper/.code=\def\sectioncase@cx{\@sectioncasefalse
                    \if@sectioncase\else\expandafter\MakeTextUppercase \fi},
  section  case/none/.code=\def\sectioncase@cx{\@empty},
}
\cxset{
          section special = sectionspecialruled@cx,
          section xcolor=spot!50,
          section afterindent=false,
          section opening=right,
          section top rule=true,
          section bottom rule=true,
          section afterskip=20pt,
          section case=lower,
          section font-family=aegean
          }


%\def\specialsection@cx{sectionspecialruled@cx}
\def\secdef#1#2{\@ifstar{\@dblarg{#2}}{\@dblarg{#1}}}
%
\newcommand\sectionx{%
  \par  
  \sectionopen   %determines if it is to be treated like a chapter
  \addpenalty\@secpenalty\nobreak
  \secdef\sectionspecialruled@cx\@ssection
   } 
  

% The macro sectionspecial@cx is a more generic macro that typesets the block of tex
% for the section heading.
% 
\def\sectionspecialruled@cx[#1]#2{%
   \sectiontoprule
  \ifnum\c@secnumdepth>0\relax
     \refstepcounter{section}%
     \addcontentsline{toc}{section}{%
      \@sectionprefix\thesection\@sectionsuffix
       \texorpdfstring{\quad}{ }#1}%
  \else
     \addcontentsline{toc}{section}{#1}%
  \fi
  {% start the title
    \color{\sectionxcolor@cx}%
    \noindent\centering\interlinepenalty\@M
   \setfont@cx{\sectionfontweight@cx}%
       {\sectionfontfamily@cx}{\sectionfontsize@cx}{\sectionfontshape@cx}%
     \ifnum\c@secnumdepth>0\relax
        \@sectionprefix\thesection\@sectionsuffix
        \quad\sectioncase@cx{#2}%
    \else %
       \sectioncase@cx{#2}
      % \luadirect{tex.print(string.upper(#2))}%
   \fi%
   \sectionbottomrule
   %\expandafter\addvspace\sectionafterskip@cx\relax%
%   \tikzrule 
   %\rule{\textwidth}{3pt}%
   \afterindent@cx
   \nobreak\par}}


\def\@ssection[#1]#2{%
  \phantomsection
  \addcontentsline{toc}{section}{#1}%
  {\noindent\centering\interlinepenalty\@M
   \color{\sectioncolor@cx}
     \setfont@cx{\sectionfontweight@cx}%
       {\sectionfontfamily@cx}{\sectionfontsize@cx}{\sectionfontshape@cx}%
       \sectiontoprule
       
        \sectioncase@cx{#2}%
        \sectionbottomrule
       %\expandafter \addvspace\sectionafterskip@cx \relax
      \afterindent@cx
   \nobreak\par}}
\makeatother

\let\section\sectionx

\section{Special Sections}

When we described the usage of the chapter setting keys, we extended the system to describe commands
for specially constructed chapter heads that do not follow the normal style of \latexe.

This section describes how to design and program, sectioning styles that go a little bit more than those that
can be defined so far and that they will require you to have a bit more knowledge of \tex and \latexe programming skills.

For example, the heading of this section started on a new page and has rules above and below the title and section number. In addition the title was capitalized automatically, despite having been typed as:

\begin{verbatim}
\section{Special Sections}
\end{verbatim}

By setting the key and calling the section again, we can typeset it on the same page

\begin{verbatim}
   \cxset{section opening=none}
   \section{Another example}
\end{verbatim}

\cxset{section opening=none,
          section case=upper,
          section top rule=false,
          section bottom rule=true,
          section afterindent=false}
          
\section{Another example}

Special sections have their own user provided macros, that have been pre-defined by the user and are invoked using the key |section special|. In the example below we have predefined a macro |\sectionsspecialruled@cx|.
Do not use a command in the value just the literal name of the command as shown below,

\begin{verbatim}
\cxset{section special = ruled,}
\end{verbatim}

\cxset{section opening=none,
          section case=none,
          section top rule=true,
          section bottom rule=false}
          
The star section of the command omits the section number from the heading. It will still insert an entry into the toc. If it is provided with an optional argument it will insert the optional text into the toc.

Check the Table of Contents to see the rendering.

\begin{verbatim}
\section*{No number test}
\section*[Short Title]{No number test}
\end{verbatim}

\section*{No number test}
\lorem

\cxset{section bottom rule=true,
         section afterindent=false,
         section font-family=agean}

\section*[Short Title]{No number test}

\lorem

\cxset{chapter opening=any,
          chapter toc=true,
          chapter numbering=arabic}

One can extend these \emph{specials} to much more complicated sections (which can resemble) chapter openings.
\makeatletter 
\newif\if@debug \@debugtrue
\bgroup
\leftskip-3cm \rightskip2cm
\def\hook{\node[right=5pt, yshift=-12pt] at (0,-3) {\HUGE\color{purple} This is the  Title}; }
\def\hook{}

\cxset{chapter name = CHAPTER}
%\expandafter\ifnum\thechapter=0\stepcounter{chapter}\else\fi

\hspace*{-2cm}\begin{tikzpicture}
\if@debug\draw [help lines] (0,0) grid (18,-13);\else\fi
\draw[fill=red]  (0,0) circle (1.5pt) ;
\node[rectangle,draw, right, baseline] (x) at (0,1) {\LARGE\color{black!30}{before}\relax};
\draw[fill=red]  (0,1) circle (1.5pt) ;
\node[rectangle,draw, right=1sp] at (0,0) {\LARGE\color{black!20} \so\chaptername\relax};

\node[rectangle,draw, color=white, below right, fill=blue!50, text=white] at +(\textwidth,0) {\scalebox{2}{\HUGE \thechapter}};
\draw[fill=red]  (0,-3) circle (1.5pt) ;
% The title of the block
\node[rectangle, draw, text width=9cm,below right, yshift=-1pt] at (0,-3) {%
         \sffamily
         \HUGE Title Format\vskip1sp \medskip\Large Blue colors in jeans, dresses skirts\\ and hats.\\
         How to dress in stylish blues. \\Getting your partner to get\\ into LaTeX. }; 
\node at (12.5,-9) {\includegraphics[width=7cm]{./images/fashion.jpg}};
\hook
\end{tikzpicture}
\makeatother
\tikzrule 
\egroup 

For such complex layouts, it is always best to start from a piece of paper where you roughly outline
the design of the template. I call such layouts templates, because we will insert a number of variables
to parameterize them. All the typesetting commands will need to be inserted in a macro, which you
should give it a unique name. We will name the above template \emph{fashion} and we will later on define
a macro \cmd{\fashion}. The sectioning mechanism provided by the \pkgname{phd} will enable the
setting of such layouts to be carried out as:

\begin{verbatim}
\cxset{section custom = fashion}
\end{verbatim}

Everytime we call the above in our document settings, in the preamble or elswehere or subsequent sections will
be typeset using this format. 

Also before you get into too much detail in programming you should define the \emph{new} parameters
that may have to be introduced. In the example above most of the fields are already defined either
using the |phd|  key value interface or by LaTeX itself. What is new here is only the introduction of an image
and perhaps some rules as to its exact location. For example you can establish a rule that if half the width of
the image is less than the right margin then it should be centered at the right side of the textblock, alternatively it should be lined at the end of the page. We will see how to achieve this a bit later on.

It is also best to start with a MWE and to first achieve the layout you want without any parameters being introduced. We assume that we will be using TikZ to position the text and the image exactly where we 
want them, although nothing stops us from using either plain TeX boxes or the picture environment.
Since we are loading the TikZ package it is best though to use it for the graphical layout.

Introduce a |debug| boolean to help you with switching grid lines on and off. Depending on what you are trying to accomplish you may want to also add some hooks into the definitions. Start from the layout first.

\begin{verbatim}
\begin{tikzpicture}
\if@debug
   \draw [help lines] (0,0) grid (18,-13);
\else
\fi
...
\fashionposthook
\end{tikzpicture}
\end{verbatim}

We draw a grid of $18\times13$ cells which just happens to suit this particular layout well; The command 
\cmd{\fashionposthook} was just added to provide any further tikz instructions at runtime.

We then draw the layout first as best as we can and without too much consideration for parameterizing the layout at this stage.

\emphasis{if@debug,else,fi}
\begin{scriptexample}{}{}
\begin{teX}
\begin{tikzpicture}
\if@debug
  \draw [help lines] (0,0) grid (18,-13);
  \draw[fill=red]  (0,0) circle (1.5pt) ;
  \draw[fill=red]  (0,-3) circle (1.5pt) ;
\else
\fi
% draw debug rectangles
\node[rectangle,draw, right, baseline] (x) at (0,1) {\LARGE\color{black!30}{before}\relax};
\draw[fill=red]  (0,1) circle (1.5pt) ;
\node[rectangle,draw, right=1sp] at (0,0) {\LARGE\color{black!20} \so\chaptername\relax};

\node[rectangle,draw, color=white, below right, fill=blue!50, text=white] at +(\textwidth,0) {\scalebox{2}{\HUGE \thechapter}};

% The title of the block
\node[rectangle, draw, text width=9cm,below right, yshift=-1pt] at (0,-3) {%
         \sffamily
         \HUGE Title Format\vskip1sp \medskip\Large Blue colors in jeans, dresses skirts\\ and hats.\\
         How to dress in stylish blues. \\Getting your partner to get\\ into LaTeX. }; 
   \IfFileExists{\fashionimage@cx}%   
         {\node at (12.5,-9) {\includegraphics[width=7cm]{fashion}};}
         { \node at (12.5,-9) {\includegraphics[width=7cm]{fashion}};}
\hook
\end{tikzpicture}
\end{teX}
\end{scriptexample}

As I mentioned earlier, adding parameters increases the complexity of the layout and it might onfuse you
at first, but we do need to go back and iterate to improve the template.

\begin{description}
\item [odd or even pages]  Most opening layouts such as this one, might be redrawn differently for left or right pages. We need to check for this.
\item [fonts] You should never restrict your template to fixed size fonts or families. Here we can use all the |phd|
keys that are available.
\item [fine tuning positioning] This can be done by defining new keys.
\item [image] Some form of key for the image is required as well as checking, if the image is available or not. If the user forgot to type it in, we will just show a message  and typeset our standard template image.
\makeatletter

\begin{teXX}
\cxset{fashion image/.store in = \fashionimage@cx} (*@\label{fashionimage}@*)
\cxset{fashion image = {./images/fashion.jpg}}
\IfFileExists{\fashionimage@cx}{Found image file code}{Image File not found code}
\end{teXX}



%\IfFileExists{\fashionimage@cx}{image found code}{image not found code}


The line \ref{fashionimage} simply stores the image path and filename in the \cmd{\fashionimage@cx}. We then immediately set it to a default value, to ensure that it is always available. We could just also use a draft
key when we load the image. We will revisit this, once we get ready to test the template. Make sure that you add the \% at the end of the curly brackets when you testing, otherwise you may get weird errors. This is due to the TiKz’s parser. 

\end{description}
\makeatletter
\cxset{fashion image/.code = \gdef\fashionimage@cx{#1}}
\cxset{fashion image = shock.jpg}

\cxset{subtitle font-color/.store in=\subtitlefontcolor@cx}
\cxset{subtitle font-color=black!35}
%default value for the image width
\def\imagewidth@cx{5cm}
\def\fashionnumberbg@cx{gray!30}
\if@debug
   \tikzset{fashion/.style = rectangle, draw}
\else   
\fi
\@debugfalse
\long\gdef\fashion{%
\begin{tikzpicture}

\if@debug
  \draw [help lines] (0,0) grid (18,-13);
  \draw[fill=red]  (0,0) circle (1.5pt) ;
  \draw[fill=red]  (0,-3) circle (1.5pt) ;
\else
\fi
% draw debug rectangles
\node[fashion, right, baseline] (x) at (0,1) {\LARGE\color{black!30}{before}\relax};
\draw[fill=red]  (0,1) circle (1.5pt) ;
\node[fashion, right=1sp] at (0,0) {\LARGE\color{black!20} \so\chaptername\relax};

\node[rectangle,draw, color=white, below right, fill=\fashionnumberbg@cx, text=white] at +(13,0) {\scalebox{2}{\HUGE \thechapter}};

% The title of the block
\node[fashion, text width=9cm,below right, yshift=-1pt] at (0,-3) {%
         \sffamily
         \Huge\color{\titlefontcolor@cx}Title Format\vskip1sp \medskip\Large% 
         \color{\subtitlefontcolor@cx}Blue colors in jeans, dresses skirts\\ and hats.\\
         How to dress in stylish blues. \\Getting your partner to get\\ into LaTeX. }; 
        \IfFileExists{\fashionimage@cx}%   
           {\node at (12.5,-9) {\includegraphics[width=\imagewidth@cx]{\fashionimage@cx}};}%
           { \node at (12.5,-9) {\includegraphics[width=7cm]{shock.jpg}};}%
\end{tikzpicture}
}

At this point let us try the new code and see the small improvements we have done.

\cxset{title font-color=spot!50}
\cxset{subtitle font-color/.store in=\subtitlefontcolor@cx}
\cxset{subtitle font-color=black!35}
\cxset{fashion image=shock.jpg}

% Image needs debugging, something is not capturing it.
\fashion

We have also used a different image and as you can observe with shock, our layout has lost its appeal, will
probably offend some people and the color scheme seems messed up. What we will probably have to do
is add a few more parameters, as well as measure the image’s dimension and implement different rules for
different aspect ratios. Try at this stage and use your own code to modify the layout.

\long\def\storyi{
         In antiquity men and women saw each other as different; 
         accordingly, they developed
        complex taxonomies (philosophical explanations) 
        for understanding anatomical,
        physiological, emotional, and rational differences. \par

Some of these differences seem
profoundly odd to us moderns. Modern discussions about erotic art have often concerned the place of women: to what
extent are they objects of social manipulation, to what extent can they be subjects?
}
\long\gdef\fashion#1{%
\begin{tikzpicture}

\if@debug
  \draw [help lines] (0,0) grid (18,-13);
  \draw[fill=red]  (0,0) circle (1.5pt) ;
  \draw[fill=red]  (0,-3) circle (1.5pt) ;
\else
\fi
% draw debug rectangles
\node[fashion, right, baseline] (x) at (0,1) {\LARGE\color{black!30}{before}\relax};
\draw[fill=red]  (0,1) circle (1.5pt) ;
\node[fashion, right=1sp] at (0,0) {\LARGE\color{black!20} \so\chaptername\relax};

\node[rectangle,draw, color=white, below right, fill=\fashionnumberbg@cx, text=white] at +(12,0) {\scalebox{2}{\HUGE \thechapter}};

% The title of the block
\node[fashion, text width=9cm,below right, yshift=-1pt] at (0,-3) {%
        { \sffamily\raggedleft
        \Huge\bfseries\color{\titlefontcolor@cx}#1\par}
         \bigskip
         \Large% 
         \centering
         \color{\subtitlefontcolor@cx}%
         \raggedleft
        \storyi\par}; 
        \IfFileExists{\fashionimage@cx}%   
           {\node at (12.5,-9) {\includegraphics[width=\imagewidth@cx]{\fashionimage@cx}};}%
           { \node at (12.5,-9) {\includegraphics[width=7cm]{shock.jpg}};}%
\end{tikzpicture}
}

\fashion{SEXUALITY IN ANCIENT GREECE}
\makeatother
\bigskip

Using your document as a User Interface is  programming in a hostile environment. As mentioned
earlier, try pen and paper, it is the quickest way to get a layout right. Adding and removing text, in layouts such
as the one we have been developing is an essential part in getting the layout to get the layout aesthetics right.
Of course other people might have different taste than you and what you like would probably be distateful to other persons.
This is a common lamentation of Graphic Designers, who complain about the value systems of their Clients.

\subsection{Hooking onto LaTeX}

I think the layout is now much better and it has evolved to transform itself from a modern and colorful template to a more serious one, perhaps more appropriate for scientific work.

We have now won half the battle, the next battle is to hook into the |\section| or |\chapter| command using |\secdef|. As you might have noticed, the chapter number has not been incremented. We will need to also
add it to the Table of Contents and also get the indentation after the heading to work correctly. We do not want our users to have to worry about this and adding |\noindent|’s all over the place. At this point we will also 
add functions to add the chapter number and title to the Table of Contents. 

\makeatother

%\makeatletter\@specialfalse\makeatother
%\input{./sections/more-on-boxes}
%\cxset{style87/.style={
 chapter opening=any,
 chapter name=none,
 % positioning and float - inline is 0
 %  float right is 2
 number display=block,
 number float=right,
 number shape=starburst,
 numbering=arabic,
 number spaceout=none,
 number font-size=huge,
 number font-weight=bold,
 number font-family=rmfamily,
 number font-shape=normal,
 number before=,
 number display=inline,
 number float=none,
% 
 number border-top-width=0pt,
 number border-right-width=0pt,
 number border-bottom-width=0pt,
 number border-left-width=0pt,
 number border-width=1pt,
%  
 number padding-left=0em,
 number padding-right=0.5em,
 number padding-top=0em,
 number padding-bottom=0pt,
  %number margin-top=, to do
 %number margin-left=0pt,  to create
 %
 number after=\par,
 number dot=,
 number position=rightname,
 number color=sweet,
 number background-color=white,
 %chapter name
 chapter display=block,
 chapter float=left,
 chapter shape=ellipse,
 chapter color=black,
 chapter background-color=sweet,
 chapter font-size= Huge,
 chapter font-weight=bfseries,
 chapter font-family=itshape,
 chapter before=,
 chapter spaceout=none,
 chapter after=,
 chapter margin-left=0cm,
 chapter margin-top=0pt,
 %
 chapter border-width=2pt,
 chapter border-top-width=1pt,
 chapter border-right-width=1pt,
 chapter border-bottom-width=1pt,
 chapter border-left-width=4pt,
% 
 chapter padding-left=20pt,
 chapter padding-right=20pt,
 chapter padding-top=20pt,
 chapter padding-bottom=10pt,
  %chapter title
 title font-family=rmfamily,
 title font-color=spot!80,                    %CHANGED
 title font-weight=bfseries,
 title font-size=huge,
 chapter title align=none,
 title margin-left=1cm,
 title margin-bottom=1.3cm,
 title margin-top=30pt,
 % title borders
 title border-width=0pt,
 title padding=0pt,
 title border-color=black!80,
 title border-top-color=spot!50,
 title border-top-width=2pt,
 title border-left-color=black!80,
 title border-left-width=2pt,
 title border-color=black!80,
 title padding-top=0pt,
 title padding-bottom=0pt,
 title padding-left=0pt,
 title padding-right=0pt,
 title border-right-color=spot!50,
 title border-right-width=2pt,
 title border-bottom-color=spot!50,
 title border-bottom-width=2pt,
 %
 chapter title align=left,
 chapter title text-align=left,
 chapter title width=0.8\textwidth,
 title before=,
 title after=,
 title display=block,
 title beforeskip=12pt,
 title afterskip=12pt,
 author block=false,
 section font-family=rmfamily,
 section font-size=LARGE,
 section font-weight=bfseries,
 section indent=0pt,
  section font-weight=mdseries,
 section align=left,
 subsubsection font-family=tiresias,
 subsubsection font-shape=upshape,
 subsubsection font-weight=mdseries,
 subsubsection align=flushleft,
 epigraph width=\dimexpr(\textwidth-2cm)\relax,
 epigraph align=center,
 epigraph text align=center,
 epigraph rule width=0pt,
 header style=plain,
 toc name = List of Illustrations}}
 
\cxset{style87}
\renewsection\renewsubsection\renewsubsubsection
\ExplSyntaxOff
\makeatother
\endinput

\makeatletter
\cxset{enumerate numberingi/.is choice,
  enumerate numberingi/.code={\renewcommand\theenumi {\csname#1\endcsname{enumi}}},
  enumerate numberingii/.code={\renewcommand\theenumii {\csname#1\endcsname{enumii}}},
  enumerate numberingiii/.code={\renewcommand\theenumiii {\csname#1\endcsname{enumiii}}},
  enumerate numberingiv/.code={\renewcommand\theenumiv {\csname#1\endcsname{enumiv}}},
  enumerate labeli punctuation/.store in=\enumeratepunctuationi@cx,
  enumerate labeli/.is choice,
  enumerate labeli/brackets/.code={\renewcommand\labelenumi{(\theenumi\enumeratepunctuationi@cx)}},
  enumerate labeli/square brackets/.code={\renewcommand\labelenumi{[\theenumi\enumeratepunctuationi@cx]}},
  enumerate labeli/right bracket/.code={\renewcommand\labelenumi{\theenumi\enumeratepunctuationi@cx)}},
  enumerate label left/.store in=\enumeratelabelleft@cx,
  enumerate label right/.code=\renewcommand\labelenumi{\enumeratelabelleft@cx\theenumi\enumeratepunctuationi@cx#1},
  enumerate leftmargini/.code={\setlength\leftmargini{#1}},
  enumerate leftmarginii/.code={\setlength\leftmarginii{#1}},
  enumerate leftmarginiii/.code={\setlength\leftmarginiii{#1}},
  enumerate leftmarginiv/.code={\setlength\leftmarginiv{#1}},
  listi topsep/.store in=\listitopsep@cx,
  listi partopsep/.store in=\listipartopsep@cx,
  listi itemsep/.store in=\listiitemsep@cx,
  listi parsep/.store in=\listiparsep@cx,
  listii topsep/.store in=\listiitopsep@cx,
  listii partopsep/.store in=\listiipartopsep@cx,
  listii itemsep/.store in=\listiiitemsep@cx,
  listii parsep/.store in=\listiiparsep@cx,
  listiii topsep/.store in=\listiiitopsep@cx,
  listiii partopsep/.store in=\listiiipartopsep@cx,
  listiii itemsep/.store in=\listiiiitemsep@cx,
  listiii parsep/.store in=\listiiiparsep@cx,
}
\cxset{compact1/.style={%
  enumerate numberingi=arabic,
  enumerate numberingii=alph,
  enumerate numberingiii=alph,
  enumerate numberingiv=roman,
  enumerate labeli punctuation=.,
  enumerate label left=,
  enumerate label right=,
  enumerate leftmargini=2.2em,
  enumerate leftmarginii=2.1em,
  enumerate leftmarginiii=1.5em,
  enumerate leftmarginiv=2em,
  listi topsep=8\p@ \@plus2\p@ \@minus\p@,
  listi itemsep=0\p@ \@plus2\p@ \@minus\p@,
  listi parsep=0\p@ \@plus2\p@ \@minus\p@,
  listii topsep=0\p@ \@plus2\p@ \@minus\p@,
  listii itemsep=0\p@ \@plus2\p@ \@minus\p@,
  listii parsep=0\p@ \@plus2\p@ \@minus\p@,
  listiii topsep=0\p@ \@plus2\p@ \@minus\p@,
  listiii itemsep=0\p@ \@plus2\p@ \@minus\p@,
  listiii parsep=0\p@ \@plus2\p@ \@minus\p@,
}}
\cxset{compact2/.style={%
  enumerate numberingi=alph,
  enumerate numberingii=roman,
  enumerate numberingiii=alph,
  enumerate numberingiv=roman,
  enumerate labeli punctuation=,
  enumerate label left=(,
  enumerate label right=),
  enumerate leftmargini=2.2em,
  enumerate leftmarginii=2.1em,
  enumerate leftmarginiii=1.5em,
  enumerate leftmarginiv=2em,
  listi topsep   = 8\p@ \@plus2\p@ \@minus\p@,
  listi itemsep = 0\p@ \@plus2\p@ \@minus\p@,
  listi parsep   = 0\p@ \@plus2\p@ \@minus\p@,
  listii topsep  = 0\p@ \@plus2\p@ \@minus\p@,
  listii itemsep= 0\p@ \@plus2\p@ \@minus\p@,
  listii parsep  = 0\p@ \@plus2\p@ \@minus\p@,
  listiii topsep = 0\p@ \@plus2\p@ \@minus\p@,
  listiii itemsep= 0\p@ \@plus2\p@ \@minus\p@,
  listiii parsep  = 0\p@ \@plus2\p@ \@minus\p@,
}}

\ExplSyntaxOn
\def\setenumerate#1{
\cxset{#1}
\def\@listi{%
           \leftmargin\leftmargini
            \parsep\listiparsep@cx
            \topsep\listitopsep@cx\relax
            \itemsep\listiitemsep@cx}
            
\def\@listii{\leftmargin\leftmarginii
            \parsep\listiiparsep@cx
            \topsep\listiitopsep@cx\relax
            \itemsep\listiiitemsep@cx}
            
\def\@listiii{\leftmargin\leftmarginiii
            \parsep\listiiiparsep@cx
            \topsep\listiiitopsep@cx\relax
            \itemsep\listiiiitemsep@cx}
}


\setenumerate{compact1}

%\cxset{section align=left}
%\cxset{section font-weight=bold}
%\cxset{section font-family=sffamily} 
%\cxset{section top rule=false,
%          section bottom rule = false,
%}
          
          
          
\makeatletter
\@debugtrue
\makeatother
\DocInput{\jobname.dtx}
\chapter{All Tests}

 \testsections

 \lorem

 \section{Sections}

 \lorem
 \subsection{Subsection}

 \lorem

 \subsubsection{Subsubsection}

 \lorem

 
 \paragraph{Block paragraph} 
 
 \lorem
 
 \subparagraph{Subparagraph} 
 
 \lorem

 \lorem

 \subparagraph{Subparagraph} 
\lorem

 \cxset{section format=display}

 \section{This is a displayed section}
 \lorem

 \paragraph{This is a paragraph}


 \cxset{section format=display}
 \lorem\lorem

 \section{This is a displayed section}

 \subparagraph{This is a subparagraph}

 \cxset{hang/.style = {
        section format=hang,
        subsection format=hang,
^^A     subsubsection format=hang,
      }}
 \cxset{hang}
 \section{This is a hanged section \lorem}
 \lorem\lorem

 \subsection{This is a hanged subsection \lorem}
 \lorem\lorem

 \subsubsection{This is a hanged subsubsection}
 \lorem\lorem

 \paragraph{This is a hanged paragraph.}
 \ExplSyntaxOn
 \lorem
 \clist_new:N \test_numbering_clist 
 \clist_gset:Nn \test_numbering_clist {arabic,roman,Roman,words,Words,WORDS,}
 \clist_map_inline:Nn \test_numbering_clist
  {
    \cxset{section~numbering=#1,
           section~format=block,
           section~number~prefix=,
           section~align=center,
           section~background-color=white}
    \section{Section~set~in~#1}
    \lorem\par
  }  
 \ExplSyntaxOff

 \cxset{chapter title margin-top=1cm,
           chapter title width=0.8\textwidth}
 \chapter{Subsubsection~Numbering Tests}
  \ExplSyntaxOn
 \clist_map_inline:Nn \test_numbering_clist
  {
    \cxset{subsubsection~numbering=#1,
           subsubsection~format=block,
           subsubsection~number~prefix=(,
           subsubsection~number~suffix =),
           subsubsection~align=left}
    \subsubsection{Subsubsection~set~in~#1}
    \lorem\par
  }  
 \ExplSyntaxOff
  
 \chapter{Paragraph~Tests}
  \ExplSyntaxOn
 \clist_map_inline:Nn \test_numbering_clist
  {
    \cxset{paragraph~format=inline,
           paragraph~numbering=#1,
           paragraph~number~prefix=(,
           paragraph~number~suffix=),
           }
    \paragraph{Paragraph~set~in~#1}
    \lorem\par
  }  
\ExplSyntaxOff

 \chapter{Subparagraph~Tests}
 
 \ExplSyntaxOn
 \clist_map_inline:Nn \test_numbering_clist
  {
    \cxset{subparagraph~numbering=#1,
           subparagraph~number~prefix=(,
           subparagraph~number~suffix=),
           }
    \subparagraph{Subparagraph~set~in~#1}
    \lorem\par
  }  
 \ExplSyntaxOff

 \chapter {Test Formatters}

 \cxset{section format = block,
        section font-size=small,
        section font-weight=normal,
        section font-shape=italic,
        section numbering = arabic}

 \section{Block section} 

\lorem\lorem


 \section{inline section} \lorem

 \cxset{section arc=5mm,
        section format=block,
        section beforeskip=-3.25ex,
        section afterskip=3.25ex,
        section background-color=spot!15,
        section title margin-top=0pt,
 }

 \section{block section}\lorem
 \cxset{
        section arc                = 5mm,
        section format=block,
        section beforeskip         = -3.25ex,
        section afterskip          = 2em,
        section background-color   = white,
        section title margin-top   = 2pt,
        section title width        = 8cm,
        section border-left-width  = 2pt,
        section afterindent        = on,
        section name               = Section,
        section background-color   = spot!30,
        section font-size          = LARGE,
        section font-weight        = bfseries,  
        section font-shape         = normal,
        section font-family        = rmfamily,
        section border-right-width = 2pt,
        chapter shadow             = drop shadow,
        section shadow             = drop shadow,
        section rounded corners    = northeast,
        chapter title margin-top=20pt, 
 }

 \section{display section}\lorem
 \lorem\lorem

 
 \chapter {Shadow Tests}

 \section {Some other tests}
 
\printproperties\\
% \printindex
 %
% 
\end{document}
%</driver>
% \fi
% 
%  \CheckSum{0}
%  \CharacterTable
%  {Upper-case    \A\B\C\D\E\F\G\H\I\J\K\L\M\N\O\P\Q\R\S\T\U\V\W\X\Y\Z
%   Lower-case    \a\b\c\d\e\f\g\h\i\j\k\l\m\n\o\p\q\r\s\t\u\v\w\x\y\z
%   Digits        \0\1\2\3\4\5\6\7\8\9
%   Exclamation   \!     Double quote  \"     Hash (number) \#
%   Dollar        \$     Percent       \%     Ampersand     \&
%   Acute accent  \'     Left paren    \(     Right paren   \)
%   Asterisk      \*     Plus          \+     Comma         \,
%   Minus         \-     Point         \.     Solidus       \/
%   Colon         \:     Semicolon     \;     Less than     \<
%   Equals        \=     Greater than  \>     Question mark \?
%   Commercial at \@     Left bracket  \[     Backslash     \\
%   Right bracket \]     Circumflex    \^     Underscore    \_
%   Grave accent  \`     Left brace    \{     Vertical bar  \|
%   Right brace   \}     Tilde         \~}
%
%
%
% \changes{1.0}{2013/01/26}{Converted to DTX file}
%
% \DoNotIndex{\newcommand,\newenvironment}
% \GetFileInfo{phd.dtx}
% 
%  \def\fileversion{v1.0}          
%  \def\filedate{2012/03/06}
% \title{The \textsf{phd} package.
% \thanks{This
%        file (\texttt{phd.dtx}) has version number \fileversion, last revised
%        \filedate.}
% }
% \author{Dr. Yiannis Lazarides \\ \url{yannislaz@gmail.com}}
% \date{\filedate}
%
%
% 
% ^^A\maketitle
% 
% ^^A\frontmatter
%  ^^A\coverpage{./images/hine02.jpg}{Book Design }{Camel Press}{}{}
%  \newpage
% ^^A\secondpage
% \pagestyle{empty}
%
%
% 
%
%
% \pagestyle{headings}
% \raggedbottom
%  \OnlyDescription
%
%  \StopEventually{\printindex}

% \CodelineNumbered
% \pagestyle{headings}
% 
%<*LSECT>
% \part{IMPLEMENTATION}
% 
% \cxset{chapter title margin-top=12pt,
%        chapter beforeskip=-1sp,
%        chapter title color=thegray,
%        chapter number color=blue }
% \chapter{Code Implementation Objectives and Strategy}
% 
% 
%
% The original sectioning routines of the \latexe source are somewhat complicated by the
% way optional commands were build. With expl3's xparse module these type of multi-switch commands can be simplified and the code made clearer.
% 
% The aim of this package is to provide:
% \begin{enumerate}
% \item An extended key value interface to lower level sectioning functions.

% \item To provide a compatibility mode, where documents wishing to test the package
% can have an easy switch to switch in and out. This is also important for the testing of the package.

% \item To provide a number of pre-canned templates that cover most of the typical use case.

% \item To provide means for a plug-in architecture for extensions.
% \end{enumerate}
% 
%
% \section{Preliminaries}
%
%  Standard file identification. We first announce the package 
%	 and require that it be used with \LaTeX2e. 
%
%  \lorem
%
%    \begin{macrocode}
\NeedsTeXFormat{LaTeX2e}[1994/12/01]%
\ProvidesFile{phd-lowersections}[2015/1/13 v1.0 less preamble (YL)]%
%    \end{macrocode}
%
% \cxset{section format=hang}
% \section{Source2e Interface}
% 
% I am not very fond of mixing expl3 control sequences with source2e commands. Here
% we provide an interface for all these commands we might use. 
% This section can be revisited once expl3 code becomes available.
%
%    \begin{macrocode}
\let\ltxtoday\today
\newif\if@ltxcompat \@ltxcompatfalse
%
\newcommand\tikzi[1][after heading] {
   %\if@debug
    \tikz[remember picture,overlay] 
    \draw[<->] (0,0)--(0,.2)--++(-.5,0) node[left,fill=blue!15,text=black]%
       {{\ttfamily\footnotesize #1}};%\space%
   %\fi    
  }
%    \end{macrocode}
%
% \section{Key Management}
%
% This part of the code is a bit verbose. We care to provide keys for all
% parameters in order to allow flexibility and easy extensions.
%
% \subsection{Sections}
%    \begin{macrocode}
\ExplSyntaxOn
\def\l_phd_subsection_number_prefix_tl {}
\def\l_phd_subsection_number_suffix_tl {}
\def\l_phd_subsubsection_number_prefix_tl{}
\def\l_phd_subsubsection_number_suffix_tl{}
\def\l_phd_paragraph_number_prefix_tl{}
\def\l_phd_paragraph_number_suffix_tl{}
\def\l_phd_subparagraph_number_prefix_tl{}
\def\l_phd_subparagraph_number_suffix_tl{}
\clist_new:N \phd_book_divisions_clist
\clist_gset:Nn \phd_book_divisions_clist
  {
    chapter,section,subsection,subsubsection,
    paragraph,subparagraph
  }

\dim_gzero_new:N \l_phd_section_title_border_left_width_dim

%    \end{macrocode}
% \begin{docCommand} {phd_create_new_primary_element:nn} { \marg{label}}
%  Creates new dims for primary elements. The label is one of chapter, section
%  etc. 
% \end{docCommand}
%    \begin{macrocode}
\cs_new:Npn \phd_create_new_element:nn #1 #2 #3
  {
    \dim_gzero_new:c {l_phd_#1#2_#3_top_width_dim} %margin
    \dim_gzero_new:c {l_phd_#1#2_#3_right_width_dim}
    \dim_gzero_new:c {l_phd_#1#2_#3_bottom_width_dim}
    \dim_gzero_new:c {l_phd_#1#2_#3_left_width_dim}
 }

\cs_new:Npn \phd_create_new_element_auxiliary:nn #1 
  {
     \phd_create_new_element:nn {#1} {} {margin}
     \phd_create_new_element:nn {#1} {} {padding}
     \phd_create_new_element:nn {#1} {} {border}
     \phd_create_new_element:nn {#1} {title} {margin}
     \phd_create_new_element:nn {#1} {title} {padding}
     \phd_create_new_element:nn {#1} {title} {border}
  }
  
\clist_map_inline:Nn \phd_book_divisions_clist
  {
     \phd_create_new_element:nn {#1} {} {margin}
     \phd_create_new_element:nn {#1} {} {padding}
     \phd_create_new_element:nn {#1} {} {border}
     \phd_create_new_element:nn {#1} {_title} {margin}
     \phd_create_new_element:nn {#1} {_title} {padding}
     \phd_create_new_element:nn {#1} {_title} {border}
  }
 
\cs_new:Npn \printproperties
  {
    \ExplSyntaxOn
\docAuxCommand{ l_phd_chapter_margin_top_width_dim} = \dim_use:N \l_phd_chapter_title_margin_top_width_dim,\\
\docAuxCommand{l_phd_chapter_title_margin_right_width_dim}= \dim_use:N \l_phd_chapter_title_margin_right_width_dim,\\
\docAuxCommand{l_phd_chapter_margin_bottom_width_dim} \dim_use:N \l_phd_chapter_title_margin_bottom_width_dim,\\
\docAuxCommand{l_phd_chapter_title_margin_top_width_dim} \dim_use:N \l_phd_chapter_title_margin_left_width_dim,\\
\endinput
|\dim_use:N \l_phd_chapter_margin_left_width_dim\\|
|\dim_use:N \l_phd_chapter_padding_top_width_dim\\|
|\dim_use:N \l_phd_chapter_padding_right_width_dim\\|e
|\dim_use:N \l_phd_chapter_padding_bottom_width_dim\\|
|\dim_use:N \l_phd_chapter_padding_left_width_dim\\|
|\dim_use:N \l_phd_chapter_border_top_width_dim\\|
|\dim_use:N \l_phd_chapter_border_right_width_dim\\|
|\dim_use:N \l_phd_chapter_border_bottom_width_dim\\|
|\dim_use:N \l_phd_chapter_border_left_width_dim\\|
|\dim_use:N \l_phd_chapter_title_margin_top_width_dim\\|
|\dim_use:N \l_phd_chapter_title_margin_right_width_dim\\|
|\dim_use:N \l_phd_chapter_title_margin_bottom_width_dim\\|
|\dim_use:N \l_phd_chapter_title_margin_left_width_dim\\|
|\dim_use:N \l_phd_chapter_title_padding_top_width_dim\\|
|\dim_use:N \l_phd_chapter_title_padding_right_width_dim\\|
|\dim_use:N \l_phd_chapter_title_padding_bottom_width_dim\\|
|\dim_use:N \l_phd_chapter_title_padding_left_width_dim\\|
|\dim_use:N \l_phd_chapter_title_border_top_width_dim\\|
|\dim_use:N \l_phd_chapter_title_border_right_width_dim\\|
|\dim_use:N \l_phd_chapter_title_border_bottom_width_dim\\|
|\dim_use:N \l_phd_chapter_title_border_left_width_dim\\|
|\dim_use:N \l_phd_section_margin_top_width_dim\\|
|\dim_use:N \l_phd_section_margin_right_width_dim\\|
|\dim_use:N \l_phd_section_margin_bottom_width_dim\\|
|\dim_use:N \l_phd_section_margin_left_width_dim\\|
|\dim_use:N \l_phd_section_padding_top_width_dim\\|
|\dim_use:N \l_phd_section_padding_right_width_dim\\|
|\dim_use:N \l_phd_section_padding_bottom_width_dim\\|
|\dim_use:N \l_phd_section_padding_left_width_dim\\|
|\dim_use:N \l_phd_section_border_top_width_dim\\|
|\dim_use:N \l_phd_section_border_right_width_dim\\|
|\dim_use:N \l_phd_section_border_bottom_width_dim\\|
|\dim_use:N \l_phd_section_border_left_width_dim\\|
|\dim_use:N \l_phd_section_title_margin_top_width_dim\\|
|\dim_use:N \l_phd_section_title_margin_right_width_dim\\|
|\dim_use:N \l_phd_section_title_margin_bottom_width_dim\\|
|\dim_use:N \l_phd_section_title_margin_left_width_dim\\|
|\dim_use:N \l_phd_section_title_padding_top_width_dim\\|
|\dim_use:N \l_phd_section_title_padding_right_width_dim\\|
|\dim_use:N \l_phd_section_title_padding_bottom_width_dim\\|
|\dim_use:N \l_phd_section_title_padding_left_width_dim\\|
|\dim_use:N \l_phd_section_title_border_top_width_dim\\|
|\dim_use:N \l_phd_section_title_border_right_width_dim\\|
|\dim_use:N \l_phd_section_title_border_bottom_width_dim\\|
|\dim_use:N \l_phd_subsection_margin_top_width_dim\\|
|\dim_use:N \l_phd_subsection_margin_right_width_dim\\|
|\dim_use:N \l_phd_subsection_margin_bottom_width_dim\\|
|\dim_use:N \l_phd_subsection_margin_left_width_dim\\|
|\dim_use:N \l_phd_subsection_padding_top_width_dim\\|
|\dim_use:N \l_phd_subsection_padding_right_width_dim\\|
|\dim_use:N \l_phd_subsection_padding_bottom_width_dim\\|
|\dim_use:N \l_phd_subsection_padding_left_width_dim\\|
|\dim_use:N \l_phd_subsection_border_top_width_dim\\|
|\dim_use:N \l_phd_subsection_border_right_width_dim\\|
|\dim_use:N \l_phd_subsection_border_bottom_width_dim\\|
|\dim_use:N \l_phd_subsection_border_left_width_dim\\|
|\dim_use:N \l_phd_subsection_title_margin_top_width_dim\\|
|\dim_use:N \l_phd_subsection_title_margin_right_width_dim\\|
|\dim_use:N \l_phd_subsection_title_margin_bottom_width_dim\\|
|\dim_use:N \l_phd_subsection_title_margin_left_width_dim\\|
|\dim_use:N \l_phd_subsection_title_padding_top_width_dim\\|
|\dim_use:N \l_phd_subsection_title_padding_right_width_dim\\|
|\dim_use:N \l_phd_subsection_title_padding_bottom_width_dim\\|
|\dim_use:N \l_phd_subsection_title_padding_left_width_dim\\|
|\dim_use:N \l_phd_subsection_title_border_top_width_dim\\|
|\dim_use:N \l_phd_subsection_title_border_right_width_dim\\|
|\dim_use:N \l_phd_subsection_title_border_bottom_width_dim\\|
|\dim_use:N \l_phd_subsection_title_border_left_width_dim\\|
|\dim_use:N \l_phd_subsubsection_margin_top_width_dim\\|
|\dim_use:N \l_phd_subsubsection_margin_right_width_dim\\|
|\dim_use:N \l_phd_subsubsection_margin_bottom_width_dim\\|
|\dim_use:N \l_phd_subsubsection_margin_left_width_dim\\|
|\dim_use:N \l_phd_subsubsection_padding_top_width_dim\\|
|\dim_use:N \l_phd_subsubsection_padding_right_width_dim\\|
|\dim_use:N \l_phd_subsubsection_padding_bottom_width_dim\\|
|\dim_use:N \l_phd_subsubsection_padding_left_width_dim\\|
|\dim_use:N \l_phd_subsubsection_border_top_width_dim\\|
|\dim_use:N \l_phd_subsubsection_border_right_width_dim\\|
|\dim_use:N \l_phd_subsubsection_border_bottom_width_dim\\|
|\dim_use:N \l_phd_subsubsection_border_left_width_dim\\|
|\dim_use:N \l_phd_subsubsection_title_margin_top_width_dim\\|
|\dim_use:N \l_phd_subsubsection_title_margin_right_width_dim\\|
|\dim_use:N \l_phd_subsubsection_title_margin_bottom_width_dim\\|
|\dim_use:N \l_phd_subsubsection_title_margin_left_width_dim\\|
|\dim_use:N \l_phd_subsubsection_title_padding_top_width_dim\\|
|\dim_use:N \l_phd_subsubsection_title_padding_right_width_dim\\|
|\dim_use:N \l_phd_subsubsection_title_padding_bottom_width_dim\\|
|\dim_use:N \l_phd_subsubsection_title_padding_left_width_dim\\|
|\dim_use:N \l_phd_subsubsection_title_border_top_width_dim\\|
|\dim_use:N \l_phd_subsubsection_title_border_right_width_dim\\|
|\dim_use:N \l_phd_subsubsection_title_border_bottom_width_dim\\|
|\dim_use:N \l_phd_subsubsection_title_border_left_width_dim\\|
|\dim_use:N \l_phd_paragraph_margin_top_width_dim\\|
|\dim_use:N \l_phd_paragraph_margin_right_width_dim\\|
|\dim_use:N \l_phd_paragraph_margin_bottom_width_dim\\|
|\dim_use:N \l_phd_paragraph_margin_left_width_dim\\|
|\dim_use:N \l_phd_paragraph_padding_top_width_dim\\|
|\dim_use:N \l_phd_paragraph_padding_right_width_dim\\|
|\dim_use:N \l_phd_paragraph_padding_bottom_width_dim\\|
|\dim_use:N \l_phd_paragraph_padding_left_width_dim\\|
|\dim_use:N \l_phd_paragraph_border_top_width_dim\\|
|\dim_use:N \l_phd_paragraph_border_right_width_dim\\|
|\dim_use:N \l_phd_paragraph_border_bottom_width_dim\\|
|\dim_use:N \l_phd_paragraph_border_left_width_dim\\|
|\dim_use:N \l_phd_paragraph_title_margin_top_width_dim\\|
|\dim_use:N \l_phd_paragraph_title_margin_right_width_dim\\|
|\dim_use:N \l_phd_paragraph_title_margin_bottom_width_dim\\|
|\dim_use:N \l_phd_paragraph_title_margin_left_width_dim\\|
|\dim_use:N \l_phd_paragraph_title_padding_top_width_dim\\|
|\dim_use:N \l_phd_paragraph_title_padding_right_width_dim\\|
|\dim_use:N \l_phd_paragraph_title_padding_bottom_width_dim\\|
|\dim_use:N \l_phd_paragraph_title_padding_left_width_dim\\|
|\dim_use:N \l_phd_paragraph_title_border_top_width_dim\\|
|\dim_use:N \l_phd_paragraph_title_border_right_width_dim\\|
|\dim_use:N \l_phd_paragraph_title_border_bottom_width_dim\\|
|\dim_use:N \l_phd_paragraph_title_border_left_width_dim\\|
|\dim_use:N \l_phd_subparagraph_margin_top_width_dim\\|
|\dim_use:N \l_phd_subparagraph_margin_right_width_dim\\|
|\dim_use:N \l_phd_subparagraph_margin_bottom_width_dim\\|
|\dim_use:N \l_phd_subparagraph_margin_left_width_dim\\|
|\dim_use:N \l_phd_subparagraph_padding_top_width_dim\\|
|\dim_use:N \l_phd_subparagraph_padding_right_width_dim\\|
|\dim_use:N \l_phd_subparagraph_padding_bottom_width_dim\\|
|\dim_use:N \l_phd_subparagraph_padding_left_width_dim\\|
|\dim_use:N \l_phd_subparagraph_border_top_width_dim\\|
|\dim_use:N \l_phd_subparagraph_border_right_width_dim\\|
|\dim_use:N \l_phd_subparagraph_border_bottom_width_dim\\|
|\dim_use:N \l_phd_subparagraph_border_left_width_dim\\|
|\dim_use:N \l_phd_subparagraph_title_margin_top_width_dim\\|
|\dim_use:N \l_phd_subparagraph_title_margin_right_width_dim\\|
|\dim_use:N \l_phd_subparagraph_title_margin_bottom_width_dim\\|
|\dim_use:N \l_phd_subparagraph_title_margin_left_width_dim\\|
|\dim_use:N \l_phd_subparagraph_title_padding_top_width_dim\\|
|\dim_use:N \l_phd_subparagraph_title_padding_right_width_dim\\|
|\dim_use:N \l_phd_subparagraph_title_padding_bottom_width_dim\\|
|\dim_use:N \l_phd_subparagraph_title_padding_left_width_dim\\|
|\dim_use:N \l_phd_subparagraph_title_border_top_width_dim\\|
|\dim_use:N \l_phd_subparagraph_title_border_right_width_dim\\|
|\dim_use:N \l_phd_subparagraph_title_border_bottom_width_dim\\|
|\dim_use:N \l_phd_subparagraph_title_border_left_width_dim\\|
\ExplSyntaxOff  
  }    
\ExplSyntaxOff  
%    \endmacrocode}
% 
% 
%    \begin{macrocode}
\ExplSyntaxOn
\def\makekeys#1
 {
  \cxset
    {
% Main elements names etc
     #1~name/.code                 = \cs_gset:cpn {#1name} {##1}, 
     #1~beforeskip/.code           = \cs_gset:cpn {l_phd_#1_before_skip_tl}{##1},
     #1~afterskip/.code            = \cs_gset:cpn {l_phd_#1_after_skip_tl}{##1},
     #1~indent/.code               = \cs_gset:cpn {l_phd_#1_indent_tl}{##1},
% fonts
     #1~font-size/.fontsize        = l_phd_#1_fontsize_tl,
     #1~font-weight/.fontweight    = l_phd_#1_fontweight_tl, 
     #1~font-shape/.fontstyle      = l_phd_#1_fontshape_tl,  
     #1~font-family/.fontfamily    = l_phd_#1_fontfamily_tl,   
% Format
     #1~format/.format~in          = l_phd_#1_format_tl,   
% Colors
     #1~background-color/.colorin  =  l_phd_#1_background_color_tl,
     #1~color/.colorin             =  l_phd_#1_color_tl, 
     #1~shadow/.code               = \cs_gset:cpn {l_phd_#1_shadow_tl}{drop~shadow},
% Main text alignment
     #1~align/.textalign           = l_phd_#1_align_tl,   
% Rounded boxes
     #1~arc/.code                  = \cs_gset:cpn {l_phd_#1_arc_tl} {##1},
     #1~grow~left/.code            = \cs_gset:cpn {l_phd_#1_grow_left_dim}{##1},
     #1~grow~right/.code           = \cs_gset:cpn {l_phd_#1_grow_right_dim}{##1},
     #1~rounded~corners/.code      = \cs_gset:cpn {l_phd_#1_rounded_corners_tl}{##1},
% afterindent     
     #1~afterindent/.onoff         = {afterindent@cx},           
% Main element borders
     #1~border-top-width/.code     = \dim_set:cn {l_phd_#1_border_top_width_dim}    {##1},  
     #1~border-right-width/.code   = \dim_set:cn {l_phd_#1_border_right_width_dim}  {##1}, 
     #1~border-bottom-width/.code  = \dim_set:cn {l_phd_#1_border_bottom_width_dim} {##1},  
     #1~border-left-width/.code    = \dim_set:cn {l_phd_#1_border_left_width_dim}   {##1}, 
% 
     #1~padding-top-width/.code    = \dim_set:cn {l_phd_#1_padding_top_width_dim}    {##1},  
     #1~padding-right-width/.code  = \dim_set:cn {l_phd_#1_padding_right_width_dim}  {##1}, 
     #1~padding-bottom-width/.code = \dim_set:cn {l_phd_#1_padding_bottom_width_dim} {##1},  
     #1~padding-left-width/.code   = \dim_set:cn {l_phd_#1_padding_left_width_dim}   {##1}, 
%         
     #1~margin-top-width/.code     = \dim_set:cn {l_phd_#1_margin_top_width_dim}    {##1},  
     #1~margin-right-width/.code   = \dim_set:cn {l_phd_#1_margin_right_width_dim}  {##1}, 
     #1~margin-bottom-width/.code  = \dim_set:cn {l_phd_#1_margin_bottom_width_dim} {##1},  
     #1~margin-left-width/.code    = \dim_set:cn {l_phd_#1_margin_left_width_dim}   {##1}, 
% 
}
}
\def\makenumberingkeys #1 {     
% Numbering Wow!
\cxset{
     #1~number~prefix/.code        = \cs_gset:cpn {l_phd_#1_number_prefix_tl} {##1},
     #1~number~suffix/.code        = \cs_gset:cpn {l_phd_#1_number_suffix_tl} {##1},
     
     #1~number~after/.code         = \cs_gset:cpn {l_phd_#1_number_after_tl},
     #1~numbering/.is~choice,
     #1~numbering/roman/.code          =
       \cs_gset:cpn {the#1}
         {
           \cs:w l_phd_#1_number_prefix_tl \cs_end: 
             \@roman\cs:w c@#1\cs_end:\relax
           \cs:w l_phd_#1_number_suffix_tl \cs_end:   
         },
     #1~numbering/Roman/.code          =
      \cs_gset:cpn {the#1}
        {
          \cs:w l_phd_#1_number_prefix_tl \cs_end: 
          \expandafter\@Roman{\cs:w c@#1\cs_end:\relax} 
          \cs:w l_phd_#1_number_suffix_tl \cs_end:  
        },
    #1~numbering/(roman)/.code          =
    \cs_gset:cpn {the#1}
       {
       \cs:w l_phd_#1_number_prefix_tl \cs_end: 
         (\@roman\cs:w c@#1\cs_end:\relax)
       \cs:w l_phd_#1_number_suffix_tl \cs_end:    
       },
    #1~numbering/(Roman)/.code          =
    \cs_gset:cpn {the#1}
      {
        \cs:w l_phd_#1_number_prefix_tl \cs_end: 
        (\@Roman \cs:w c@#1\cs_end:\relax)
        \cs:w l_phd_#1_number_suffix_tl \cs_end:  
      },
   #1~numbering/arabic/.code           =
    \cs_gset:cpn {the#1}
      {
        \cs:w l_phd_#1_number_prefix_tl \cs_end: 
        \@arabic\cs:w c@#1\cs_end: \relax
        \cs:w l_phd_#1_number_suffix_tl \cs_end: 
      },
   #1~numbering/numeric/.code          =
    \cs_gset:cpn {the#1}
      {
        \cs:w l_phd_#1_number_prefix_tl \cs_end: 
        \@arabic\cs:w c@#1\cs_end:\relax
        \cs:w l_phd_#1_number_suffix_tl \cs_end: 
      },
   #1~numbering/none/.code             = 
    \cs_gset:cpn {the#1} {},
   #1~numbering/alpha/.code            = 
    \cs_gset:cpn {the#1}
      {
        \cs:w l_phd_#1_number_prefix_tl \cs_end: 
        \exp_after:wN \alphalph {\cs:w c@#1\cs_end:\relax}
        \cs:w l_phd_#1_number_suffix_tl \cs_end:   
      },
   #1~numbering/Alpha/.code            = 
    \cs_gset:cpn {the#1}
      {
        \cs:w l_phd_#1_number_prefix_tl \cs_end: 
          \exp_after:wN \AlphAlph{\cs:w c@#1\cs_end:\relax}
        \cs:w l_phd_#1_number_suffix_tl \cs_end:  
      },
   #1~numbering/words/.code            = 
    \cs_gset:cpn {the#1}
      {
       \cs:w l_phd_#1_number_prefix_tl \cs_end: 
       \words@cx{\@arabic\cs:w c@#1\cs_end:\relax}
       \cs:w l_phd_#1_number_suffix_tl \cs_end:  
      },
   #1~numbering/Words/.code            =
    \cs_gset:cpn {the#1}
      {
        \cs:w l_phd_#1_number_prefix_tl \cs_end: 
        \Words@cx{\@arabic\cs:w c@#1\cs_end:\relax }
        \cs:w l_phd_#1_number_suffix_tl \cs_end:  
      },
   #1~numbering/WORDS/.code            =
    \cs_gset:cpn {the#1}
      {
       \cs:w l_phd_#1_number_prefix_tl \cs_end:   
       \WORDS@cx{\@arabic\cs:w c@#1\cs_end:\relax}
       \cs:w l_phd_#1_number_suffix_tl \cs_end:  
      },
   #1~numbering~custom/.code           = 
    \cs_gset:cpn {the#1} {##1},        
  }   
 }   
%  
%



\clist_map_inline:Nn \phd_book_divisions_clist
  {
    \makekeys{#1}
  }
  
\clist_map_inline:Nn \phd_book_divisions_clist
  {
    \makenumberingkeys{#1}
  } 
 
\ExplSyntaxOff
%    \end{macrocode}
% \subsection{Creating keys for the auxiliary elements}
%  This is now getting complex. We need to create the other element keys. 
%  chapter title text-align etc..
%  
%  So we should have been passing double arguments to our previously defined 
%  function. This screws up the idea completely of fully automating the key
%  lists. Need to review carefully before I go any further to avoid a lot of duplication!.
%    \begin{macrocode}
\ExplSyntaxOn
\gdef\makeotherelements #1 #2 {
  \cxset{
  %     
     #1~#2~width/.code          = 
        \expandafter\def\cs:w l_phd_#1_#2_width_dim\cs_end:{##1},
% title margins        
     #1~#2~margin-top/.code       = \dim_gset:cn {l_phd_#1_#2_margin_top_width_dim} {##1}\relax,
     #1~#2~margin-right/.code     = \dim_gset:cn {l_phd_#1_#2_margin_right_width_dim} {##1}\relax,
     #1~#2~margin-bottom/.code    = \dim_gset:cn {l_phd_#1_#2_margin_bottom_width_dim} {##1}\relax,
     #1~#2~margin-left/.code      = \dim_gset:cn {l_phd_#1_#2_margin_left_width_dim} {##1}\relax,
     #1~#2~align/.textalign           = l_phd_#1_#2_align_tl, 
% fonts
     #1~#2~font-size/.fontsize        = l_phd_#1_#2_fontsize_tl,
     #1~#2~font-weight/.fontweight    = l_phd_#1_#2_fontweight_tl, 
     #1~#2~font-shape/.fontstyle      = l_phd_#1_#2_fontshape_tl,  
     #1~#2~font-family/.fontfamily    = l_phd_#1_#2_fontfamily_tl,
% color     
     #1~#2~color/.code                = \cs_gset:cpn {l_phd_#1_#2_color_tl} {\color{##1}},       
   }
 }
\def\makeotherelementsaux #1 
{
     \makeotherelements {#1}{title}
     \makeotherelements {#1}{label}   % chapter_label is Chapter
     \makeotherelements {#1}{number}  % chapter_number refers to the number
     \makeotherelements {#1}{before}
     \makeotherelements {#1}{after}
}

\clist_map_inline:Nn \phd_book_divisions_clist 
{
  \makeotherelementsaux {#1} 
}  

     
  
\ExplSyntaxOff 
%    \end{macrocode}
%    \begin{macrocode}
\cxset
  { 
    chapter title margin-top       = 0cm, 
    chapter title margin-right     = 1cm,
    chapter align                  = Centering,
    chapter title align            = Centering, %checked
    chapter name                   = Section,
    chapter format                 = block,
    chapter font-size              = HUGE,
    chapter font-weight            = bold,
    chapter font-family            = sffamily,
    chapter font-shape             = upshape,
    chapter color                  = spot!30,
    chapter number prefix          = ,
    chapter number suffix          = ,
    chapter numbering              = arabic,
    chapter indent                 = 0pt,
    chapter beforeskip             = 10pt,
    chapter afterskip              = -3ex,
    chapter afterindent            = off,
    chapter number after           = \quad,
%    
    chapter arc                    = 3pt,
    chapter background-color       = spot!30,
    chapter afterindent            = off, 
    chapter grow left              = 0mm,
    chapter grow right             = 0mm,
    chapter rounded corners        = northeast,
    chapter shadow                 = drop shadow,
%    
    chapter border-left-width      = 0pt,
    chapter border-right-width     = 0pt,
    chapter border-top-width       = 2pt,
    chapter border-bottom-width    = 2pt,
    chapter padding-left-width     = 0pt,
    chapter padding-right-width    = 10pt,
    chapter padding-top-width      = 10pt,
    chapter padding-bottom-width   = 10pt,
    %
    chapter number font-size        = Huge,
    chapter number font-weight      = bfseries,
    chapter number font-family      = sffamily,
    chapter number font-shape       = upshape, 
    chapter number color            = spot!30,  
%
    chapter title font-size        = Huge,
    chapter title font-weight      = bold,
    chapter title font-family      = sffamily,
    chapter title font-shape       = upshape, 
    chapter title color            = spot!30,   
}  
%    \end{macrocode}
%
%    \begin{macrocode} 
\ExplSyntaxOn 

\cxset 
  { 
    section~spaceout/.is~choice,
    section~spaceout/soul/.code            = \@sectionspaceouttrue,
    section~spaceout/none/.code            = \@sectionspaceoutfalse,
 }  
 
\ExplSyntaxOff 
%    \end{macrocode}
% As described earlier boxed headings have numerous elements, each of which can be styled
% on its own. 
% 
%
%
% \chapter{Formatters}

% We first define some formatters to tie up with using |format| in sectioning commands.
%
%    \begin{macrocode}
\ExplSyntaxOn
\cs_set:Npn \format_inmargin:nnn #1#2#3
  {
     \tcbdocmarginnote
     { 
       
       \hbox{Section~\@svsec}
       #3
     } 
  }    
\ExplSyntaxOff  
%    \end{macrocode}
%
%    \begin{macrocode}
\ExplSyntaxOn
% 1 skip from left
% 2 
\cs_set:Npn \format_hang:nn #1#2#3 
  {{ \@hangfrom{#2\relax\@svsec%
      \interlinepenalty \@M #3\@@par}%
  }}
\ExplSyntaxOff
%    \end{macrocode}
%
% \subsection{Block format}
%
% Before we start setting boxes within boxes, we need a function to rename
% parameters to enable automatic creation of styles for all the boxed
% elements. The following elements in block formats are contained in
% their own boxes. Each box has its own style.
%  
%    \begin{macrocode}

\ExplSyntaxOn
\cs_gset:Npn \phd_set_box_parameters:nn #1 #2
{
  % background color  
    \cs_if_exist:cTF {l_phd_#1_background_color_tl}
      {
        \cs_set:cpn {#1tcbbgcolor} { \cs:w l_phd_#1_background_color_tl\cs_end: }
      }
      {
        \cs_set:cpn {#1tcbbgcolor} {white} 
      }
% 
   \cs_if_exist:cTF {l_phd_#1_arc_tl}
     {
       \cs_set:cpn {tcbarc_#2}
              {\cs:w l_phd_#1_arc_tl \cs_end: }
     }
     {
        \cs_set:cpn {tcbarc_#2} {0pt}
     }
% alignment \FIRE
%\cs_if_exist:cTF {l_phd_#1_align_tl}
%     {
%        \cs_gset:cpn {chaptertcbalign} {\cs:w l_#1_align_tl\cs_end:}
%     }
%     {
%        \cs_gset:cpn {chaptertcbalign} {undefined}
%     }
%  \l_phd_section_grow_left_dim         
\cs_if_exist:cTF {l_phd_#1_grow_left_dim}
    {
      \def\tcbgrowleft{\csname l_phd_#1_grow_left_dim \endcsname} 
    }
    {
      \def\tcbgrowleft{0pt}
    }
\cs_if_exist:cTF {l_phd_#1_grow_right_dim}
    {
      \def\tcbgrowright{\cs:w l_phd_#1_grow_right_dim \cs_end:} 
    }
    {
      \def\tcbgrowright{0pt}
    } 
    
\cs_if_exist:cTF {l_ phd_#1_shadow_tl}
    {
      \gdef\tcbshadow
       {
        \cs:w l_phd_#1_shadow_tl \cs_end:
       } 
    }
    {
      \gdef\tcbshadow{no~shadow}
    } 

\cs_if_exist:cTF {l_phd_#1_rounded_corners_tl}
    {
      \def\tcbroundedcorners{\cs:w l_phd_#1_rounded_corners_tl \cs_end:} 
    }
    {
      \def\tcbroundedcorners{all}
    } 
 % dimensional constants
  \dim_if_exist:cTF { l_phd_#1_title_margin_top_width_dim }
    {
      \gdef\tcbtitlevspace{\dim_use:c { l_phd_#1_title_margin_top_width_dim} } 
    }
    {
      \gdef\tcbtitlevspace{0pt}
    } 
  \dim_if_exist:cTF {l_phd_#1_border_top_width_dim }
    {
      \def\tcbtoprulewidth{\dim_use:c {l_phd_#1_border_top_width_dim} }
    }
    {
      \def\tcbtoprulewidth{0pt}
    }    
 \dim_if_exist:cTF {l_phd_#1_border_left_width_dim }
    {
      \def\tcbleftrulewidth{\dim_use:c {l_phd_#1_border_left_width_dim} }
    }
    {
      \def\tcbleftrulewidth{0pt}
    } 
 
 \dim_if_exist:cTF {l_phd_#1_border_bottom_width_dim }
    {
      \def\tcbbottomrulewidth{\dim_use:c {l_phd_#1_border_bottom_width_dim} }
    }
    {
      \def\tcbbottomrulewidth{0pt}
    }           
 \dim_if_exist:cTF {l_phd_#1_border_right_width_dim }
    {
      \def\tcbrightrulewidth {\dim_use:c {l_phd_#1_border_right_width_dim} }
    }
    {
      \def\tcbrightrulewidth{0pt}
    } 
    
    
% Padding       
\dim_if_exist:cTF {l_phd_#1_padding_top_width_dim }
    {
      \cs_set:cpn {#1tcbtopsepwidth} {\dim_use:c {l_phd_#1_padding_top_width_dim} }
    }
    {
      \cs_set:cpn {#1tcbtopsepwidth} {0pt}
    }    
 \dim_if_exist:cTF {l_phd_#1_padding_left_width_dim }
    {
      \cs_set:cpn {#1tcbleftsepwidth} {\dim_use:c {l_phd_#1_padding_left_width_dim} }
    }
    {
      \cs_set:cpn {#1tcbleftsepwidth} {0pt}
    } 
 
 \dim_if_exist:cTF {l_phd_#1_padding_bottom_width_dim }
    {
      \cs_set:cpn {#1tcbbottomsepwidth} {\dim_use:c {l_phd_#1_padding_bottom_width_dim} }
    }
    {
      \cs_set:cpn {#1tcbbottomsepwidth} {0pt}
    }           
 \dim_if_exist:cTF {l_phd_#1_padding_right_width_dim }
    {
      \cs_set:cpn {#1tcbrightsepwidth} {\dim_use:c {l_phd_#1_padding_right_width_dim} }
    }
    {
      \cs_set:cpn {#1tcbrightsepwidth} {0pt}
    }        
    
}         
%    \end{macrocode}
% 
% \begin{docCommand}{make_box_style:n}{ \meta{label name} \meta{style second name} }
%   Every block formatted heading is contained in an outer box. This has
%   a named style such as |outerbox section|. There is a lot of name swapping
%   to generally abstract any element.
% \end{docCommand}
%
%    \begin{macrocode} 
\cs_set:Npn \make_box_style:n #1 #2
  {
    \phd_set_box_parameters:nn {#1}{#2} %section outer
    \tcbset
      {
        #1~#2/.style=
          {
            size               = minimal, %resets
            enhanced,
            colback            = white, %fix\cs:w #1tcbbgcolor \cs_end:, 
            colframe           = black!80, 
            sharpish~corners,
            sharp~corners      = all,
            %arc                = 6mm,%\tcbarc_outer,
            auto~outer~arc,                   
            rounded~corners    = east,%\tcbroundedcorners,
            %\tcbshadow,   
            fuzzy~shadow       = {2mm}{-1mm}{0mm}{0.1mm}{black!50!white},
% padding           
%            left               = \csname #1tcbleftsepwidth \endcsname,
            right              = 10mm,%\cs:w #1tcbrightsepwidth  \cs_end:,
            bottom             = 3mm,%\cs:w #1tcbbottomsepwidth \cs_end:,
            top                = 3mm,%\cs:w #1tcbtopsepwidth    \cs_end:,
% frame rules           
%            toprule            = \tcbtoprulewidth,
%            leftrule           = \tcbleftrulewidth,
%            rightrule          = \tcbrightrulewidth,
%            bottomrule         = \tcbbottomrulewidth,
% border rules           
%            grow~to~right~by   = 1cm,% \tcbgrowright,
%            grow~to~left~by    = 1cm, %\tcbgrowleft,
%            boxsep             = 10pt,
%            valign~lower       = center,%not necessary?
%           interior~style     = {left~color=blue!20!white,
%                                 right~color=blue!30!white},
%\if@debug
%            borderline~north   = {0pt}{-1pt}{red},  
%            borderline~south   = {0pt}{-1pt}{red},   
%            borderline~east    = {0pt}{0pt}{red},
%            borderline~west    = {0pt}{0pt}{red},                             
%\fi%makestyle          
       }
    }   
  }   
\tcbset{
  chapter/.style = {
%            borderline~north   = {0pt}{0pt}{orange},  
%            borderline~south   = {0pt}{0pt}{orange},   
%            borderline~east    = {0pt}{0pt}{orange},
%            borderline~west    = {0pt}{0pt}{orange},   
      }
  }
\tcbset{
  _title/.style = {
%            borderline~north   = {0pt}{0pt}{yellow},  
%            borderline~south   = {0pt}{0pt}{yellow},   
%            borderline~east    = {0pt}{0pt}{yellow},
%            borderline~west    = {0pt}{0pt}{yellow},   
      }
  }    
\tcbset{
  _number/.style = {
%            borderline~north   = {0pt}{0pt}{blue},  
%            borderline~south   = {0pt}{0pt}{blue},   
%            borderline~east    = {0pt}{0pt}{blue},
%            borderline~west    = {0pt}{0pt}{blue},   
      }
  }   

 \tcbset{section/.style={chapter}} 
 \tcbset{subsection/.style={chapter}} 
 \tcbset{subsubsection/.style={chapter}}   
%    \end{macrocode}
%
% \begin{docCommand}{format_block:nnnn} {\meta{}\meta{}\marg{}\marg{}}
%   This function is the main function for block and fancy headings. It
%   is also used for Chapter and part formatting. Depending on the 
%   settings it contains the following boxes:\\
%   1.0 Label box e.g., section\\
%   2.0 Number box e.g., 1.12.24\\
%   3.0 Title\\
%   It also handles the before and after boxes that handle rules and 
%   ornamentation if any. 
%
% \end{docCommand}
% 
%    \begin{macrocode}    
\bool_new:N \combo_if_bool \bool_gset_true:N \combo_if_bool

\cs_set:Npn \format_block:nnnn #1#2#3#4
{
 \bgroup
 \make_box_style:n {#1} {outer} 
 \make_box_style:n {#1} {inner} 
 \leftskip-1cm
 \begin{tcolorbox}[
                   #1~outer, 
                   width=\linewidth+2cm,
                   arc=3mm,
                   %drop~shadow=black,
                   %rounded~corners=all,
                   colback=spot!15
                  ] 
%\phd_float_box:nnn {#1}{}
%  {
%         \tcbox{\cs:w #1name\cs_end: }
%  } 
\bool_if:NTF \combo_if_bool 
  {
    \phd_float_box:nnn {#1}{_number}
        { 
          \tcbox[size=minimal,
                 nobeforeafter,
                 colback=spot!15,
           fontlower={
             \l_phd_chapter_number_fontweight_tl
             \l_phd_chapter_number_fontfamily_tl
             \l_phd_chapter_number_fontshape_tl
             \l_phd_chapter_number_fontsize_tl
              },
          ] 
          { 
            \cs:w #1name\cs_end:\space 
            \cs:w the#1\cs_end:
           }
        } 
   }
   {         
     \phd_float_box:nnn {#1}{}
       {
         \tcbox{\@svsec}
  
       }  
   }
%  
    \dim_compare:nNnTF {\tcbtitlevspace} > {0sp}
    {
      \skip_vertical:N \tcbtitlevspace
    }
    {}
% above title box {empty}    
   \phd_float_box:nnn {#1}{}{}     
% title float box    
   \phd_float_box:nnn {#1}{_title}{#4}
   \par
   \end{tcolorbox}
 \egroup  
 \par\nobreak\nointerlineskip
}    
%    \end{macrocode}
%
%
%  \begin{docCommand} {phd_float_box} { \marg{} \marg{} \marg{}}
%    |#1|  The section label\\
%    |#2|  The secondary identifier i.e, title\\
%    |#3|  The text of the box\\
%  \end{docCommand}
%
%  This box contains an outer and an inner box, permitting the second box to float freely into
%  the first box. The widths are constrained based on user inputs or automatic calculations.
%
%    \begin{macrocode}
\cs_set:Npn \phd_float_box:nnn #1 #2 #3
{
  \begin{tcolorbox}
    [#1~outer,
      size=minimal,no~shadow,colback=spot!15, 
      #1
    ]
    \cs:w l_phd_#1#2_align_tl \cs_end:
    \begin{tcolorbox}
      [
        #1~outer,#2,
        size=minimal, no~shadow,colback=spot!15,
        %drop~shadow, 
        width=0.7\textwidth, 
       ]
          \language-1\relax
      \cs:w  l_phd_#1#2_fontweight_tl  \cs_end:
      \cs:w  l_phd_#1#2_fontfamily_tl  \cs_end:
      \cs:w  l_phd_#1#2_fontsize_tl    \cs_end:
      \cs:w  l_phd_#1#2_fontshape_tl   \cs_end:
     % \cs:w  l_phd_#1#2_color_tl       \cs_end:
      \cs:w  l_phd_#1#2_align_tl       \cs_end: 
      #3
      \par
  \end{tcolorbox}
   \par
  \end{tcolorbox}
}

\ExplSyntaxOff
%    \end{macrocode}
% 
% \subsection{Display format}
% The display format typesets a heading in a similar fashion to 
% traditional chapters.
%
% \begin{docCommand}{format_display:nn} {\marg{section name}} { \marg{skip after number} \marg {} }
%   Displays a section similar to Chapters
% \end{docCommand}
%  \#1 Section name \\
%  \#2 indent       \\
%  \#3 format para        \\
%  \#4 Title text\\
%  svsec number
%    \begin{macrocode}
\ExplSyntaxOn
\cs_set:Npn \format_display:nnnn #1 #2 #3 #4
{
  \cxset{#1~title~margin-top=30pt}
  \format_block:nnnn {#1}{#2}{#3}{#4}
}
 \ExplSyntaxOff
%    \end{macrocode}
%
% \#1 name
% \#2 indent
% \#3 title
%  
%    \begin{macrocode}
\ExplSyntaxOn
\cs_set:Npn \format_inline:nnn #1 #2 #3
  {
   {\bfseries\normalfont
    \theparagraph #3}
   }    
\ExplSyntaxOff  
%    \end{macrocode}
%
% \chapter{Layout Engine Code}
%
% The standard kernel factory commands, they are real locomotives. To hook into them
% we need to dig deep.
%
% We also define \cs{@startsection} as somehow there are problems
% with after indent false. valid |\section*{title}|, |\section[toc-entry]|, 
% |\section [toc-entry] {title}|
%
%  |#1| name i.e, section
%  |#2| level number 2 section
%  |#3| indent
%  |#4| beforeskip
%  |#5| afterskip
%  |#6|  styling command
%
%    \begin{verbatim}
% \def\@startsection#1#2#3#4#5#6{%
%    \if@noskipsec \leavevmode \fi
%    \par
%    \@tempskipa #4\relax 
%    \@afterindenttrue
%    \ifdim \@tempskipa <\z@
%        \@tempskipa -\@tempskipa\@afterindentfalse
%    \fi
%    \if@nobreak
%    \everypar{}%
%    \else
%      \addpenalty\@secpenalty\addvspace\@tempskipa
%    \fi
%   \@ifstar
%   {\@ssect{#3}{#4}{#5}{#6}}%defined in the kernel
%   {\@dblarg{\@sect{#1}{#2}{#3}{#4}{#5}{#6}}}}
%    \end{verbatim}
%   
%    \begin{macrocode}
\ExplSyntaxOn
\DeclareDocumentCommand \start_section:nnnnnnnnn {m m m m m m s o m}      
  {
    \if@noskipsec \leavevmode \fi
    \par
%    check for before skip    
    \l_tmpa_skip #4\relax 
    \@afterindenttrue
    \if_dim:w \l_tmpa_skip <\z@
% make it positive    
      \skip_gset:Nn\l_tmpa_skip {-\l_tmpa_skip}  
      \@afterindentfalse
    \fi:
%    
    \if@nobreak
      \everypar{\tikzi[start\\sect]}
    \else
      \addpenalty \@secpenalty
      \addvspace\l_tmpa_skip
    \fi
%
% redirect depending on star or option
%     
    \IfBooleanTF {#7}
      {\@ssect {#3} {#4} {#5} {#6} {#9} }
      {
        \IfValueTF {#8} {\@sect:  {#1} {#2} {#3} {#4} {#5} {#6} [{#8}] {#9} }
                        {\@sect:  {#1} {#2} {#3} {#4} {#5} {#6} [{#8}] {#9} } %sends TF we get it later
      }   
  }
\@ltxcompattrue
\if@ltxcompat 
  \else 
  \cs_gset_eq:NN \@startsection \start_section:nnnnnnnnn
\fi

\ExplSyntaxOff  
%    \end{macrocode}
%

%    \begin{macrocode}
\ExplSyntaxOn
\cs_set:Npn \@sect: #1 #2 #3 #4 #5 #6 [#7] #8 
  {

%  Decide if we need to add the section in the toc.
    \int_compare:nTF {#2>\c@secnumdepth} 
      {
         \let\@svsec\@empty
      }
      {
        \refstepcounter{#1}
        \protected@edef\@svsec
          {
            \@seccntformat{#1}\relax
          }
        % add short title or long title  
        \IfValueTF{#7}  
          { 
             \cs:w #1mark\cs_end: {#7} 
             \addcontentsline{toc}{#1}{
                \protect\numberline{\csname the#1\endcsname}#7}
          }
          { 
             \cs:w #1mark\cs_end: {#8} 
              \addcontentsline{toc}{#1}{ 
                \protect\numberline{\csname the#1\endcsname}#8}
          }
      } 
     
% 
   \@tempskipa #5\relax
   \gdef\@svsechd{
   %\@seccntformat{#1}#6{\hskip #3\relax #8}
   #8
   }%
%  \ifdim \@tempskipa>\z@
%    \end{macrocode}

%    \begin{macrocode}
    \str_case_x:nnTF {\cs:w l_phd_#1_format_tl \cs_end:}  
      {
          { display } {\format_display:nnnn { #1 } { #3 } {#6} { #8 } 
                        \xsect:n {#5}   } 
          { block   } { \format_block:nnnn  {#1 } { #3 } {#6} { {#8}       } 
                        \xsect:n {#5} 
                      } 
          { plain   } { \format_hang:nn    {#1} { #3 } { #8 }         } 
          { hangs    } { \format_hang:nn    { #1 } { #3 } {{#6#8}} \xsect:n {#5}   }
          { inline   } {  \xsect:n {-3.5ex} }%#5
          { inmargin } {\format_inmargin:nnn {#1} {#3} {#6#8} }
      }
      {
      %{\if@debug~\tiny\csname#1format@cx\endcsname \fi }
      } %true code
      { 
        { 
        %\if@debug~\tiny\csname#1format@cx\endcsname\fi
        }
        \@hangfrom{#6\relax\@svsec}%
        \interlinepenalty\@M {#6#8}\par\xsect:n{#5} 
      } %false code 
  %
  }
\ExplSyntaxOff  
%    \end{macrocode}
%
% \begin{docCommand}{@ssect} { {\meta{indent}} {\meta{beforeskip}} {\meta{afterskip}} \meta{styling commands} \meta{arg1} }
% This is the star verson of the command.  What it means is that we want a heading with
% no numbers and not in the toc. Also it does not add it as a mark! This is very limiting
% as originally programmed in the kernel; probably the thinking was to use it to create
% same style headings, that one would use for purposes other than sectioning. In reality
% many books have unnumbered sections and one might want them to go on the headings.
% We modify it to be able to do both based on a settings command.
% So to summarize star section means unumbered. Will use choices as
% to what must be done with it.
%
%  
%  \#1 indent\\
%  \#2 beforeskip\\
%  \#3 afterskip\\
%  \#4 styling command\\
%  \#5 arg1 follows\\
%
% \end{docCommand} 
%    \begin{macrocode}  
\ExplSyntaxOn  
%  
\cs_set:Npn \@ssect #1 #2 #3 #4 #5 {%
  \@tempskipa #3\relax
  \ifdim \@tempskipa>\z@
  \begingroup
    #4{
    \@hangfrom{\hskip #1}%
    \interlinepenalty \@M (#5)\@@par}%
    \endgroup
  \else
  \def\@svsechd{#4{\hskip #1\relax #5}}%
  \fi
% |\xsect:n{afterskip}| then sets the afteskipping as well as the afterindent.   
  \xsect:n{#3} ONLY NEEDED FOR HANG PARA
}
%   
\ExplSyntaxOff   
   
%    \end{macrocode}
%
% \begin{docCommand}{@xsect:n} {\marg{afterskip}}
%  This command sets handles indentation after a sectioning command. It also handles
%  the printing of the title for inline sections (it is saved as |\@svsechd| earlier. It is common
%  for both the star and unstarred versions of |\section|.
% \end{docCommand}
%
%|\@noskipsec| A switch set true by a sectioning command when it is creating an
%in-text heading with |\everypar|.
%    \begin{macrocode}
\ExplSyntaxOn
\cs_set:Npn \xsect:n #1 
  { 
  \l_tmpa_skip #1\relax
  \if_dim:w \l_tmpa_skip>0pt %WATCH better boolean
    \par \nobreak
    \vskip\l_tmpa_skip
    \@afterheading 
   \else:
    \@nobreakfalse
    \global\@noskipsectrue
    \tex_everypar:D 
      {\if@noskipsec
          \global\@noskipsecfalse%resets switch
          {\setbox\z@\lastbox}
          \tex_clubpenalty:D\@M
          \group_begin:
            \parindent0pt\tcbox[size=minimal,
                    nobeforeafter,
                    box~align=base]{\bfseries \@svsechd }%} 
          \group_end:
          \tex_unskip:D
          \l_tmpa_skip #1\relax
          \hskip -\l_tmpa_skip
        \else
          \tex_clubpenalty:D \@clubpenalty
          \tex_everypar:D {\tikzi[every]}
        \fi
      }
  \fi:
  \tex_ignorespaces:D
  }

\cs_set:Npn \after_block:n #1 
 {
   \par \nobreak
    \vskip\l_tmpa_skip
    \@afterheading
 }    
  
  
\ExplSyntaxOff
 \def\@afterheading{%
 \@nobreaktrue
 \everypar{%
   \if@nobreak
     \@nobreakfalse
     \clubpenalty \@M
     \if@afterindent 
     \else
      {\setbox\z@\lastbox}%
     %\tikzi[everypar]
     \fi
 \else
 \clubpenalty \@clubpenalty
 \everypar{
   %\tikzi[cleared]
 }%
 \fi}
 }  
%    \end{macrocode}
%
% 
% When LaTeX is typesetting the section number it calls |\@seccntformat|
% to use it when typsetting a section heading number. This is common for
% all the subsectioning commands. We modify it based on code from \pkgname{sectsty} in order
% to generalize it.
% 
% We first check if \meta{section}|@cntformat| is defined and then we redirect
% to specific section level command.
%
% \begin{docCommand} {@seccntformat} {\marg{section name}}
%  This is a \latexe kernel factory command that produces |thesection| etc.
%  In the kernel it only takes a generic value, where we have 
%  \refCom{section_number_after_tl}.
%  We modify to enable adjustable values for all sectioning commands. 
% \end{docCommand}
%
%    \begin{macrocode}
\ExplSyntaxOn
 \cs_gset:Npn \@seccntformat #1
 {
  \@ifundefined{#1@cntformat}%
  {\csname the#1\endcsname\section_number_after_tl}% default
  {\csname #1_cntformat\endcsname}% individual control
 }
%    \end{macrocode}
%
% \begin{docCommand}{section_number_after_tl} { \meta{void}}
%  This function and its siblings are auxiliary functions.
% \end{docCommand}
%
%    \begin{macrocode}
\tl_set:Nn  \section_number_after_tl{\quad}%default value only space
\tl_set:Nn  \subsection_number_after_tl{\quad}%default value only space
\tl_set:Nn  \subsubsection_number_after_tl{\quad}%default value only space
\tl_set:Nn  \l_phd_paragraph_number_after_tl{\quad}%default value only space
\tl_set:Nn  \subparagraph_number_after_tl{\quad}%default value only space
%
\cs_set:Npn \section_cntformat{\thesection\section_number_after_tl}
\cs_set:Npn \subsection_cntformat{\thesubsection\subsection_number_after_tl}
\cs_set:Npn \subsubsection_cntformat{\thesubsubsection\subsubsection_number_after_tl}
\cs_set:Npn \paragraph_cntformat {\theparagraph\l_phd_paragraph_number_after_tl }
\cs_set:Npn \subparagraph_cntformat {\thesubparagraph\subparagraph_number_after_tl }
\ExplSyntaxOff
%    \end{macrocode}
%
% \begin{docCommand}{chapter} { \meta{*} \oarg{arg1} \marg{arg2} }
%   The standard \latexe chapter is renewed next. We avoid all the complications of
%   the kernel with different longish sections for chapters and parts.
% \end{docCommand}  
%
%    \begin{macrocode}
\ExplSyntaxOn

 \renewcommand\chapter {%
   \newpage\null
    \start_section:nnnnnnnnn{chapter}%
      {0}  %level check this conflicts with source2e
      
      {\l_phd_chapter_indent_tl} %indent#2
      
      {\l_phd_chapter_before_skip_tl}%before skip#3
      
      {\l_phd_chapter_after_skip_tl}% after skip#4
      
      { 
%        \setfont@cx
%        {\l_phd_chapter_fontweight_tl}%
%        {\l_phd_chapter_fontfamily_tl}
%        {\l_phd_chapter_fontsize_tl}
%        {\l_phd_chapter_fontshape_tl}%
          %\expandafter\setfontparam@cx\l_phd_chapter_align_tl;%
          %\color{\l_phd_chapter_color_tl}%5
      }
 }%

%    \end{macrocode} 
%
% \begin{docCommand}{section} { \meta{*} \oarg{arg1} \marg{arg2} }
%   The standard \latexe subsection is renewed next. 
% \end{docCommand} 
%    \begin{macrocode}

%\if@ltxcompat
%\else
%  \renewcommand\section{%
%    \@startsection{section}%
%      {1}%level check this conflicts with source2e
%      
%      {\l_phd_section_indent_tl}%indent#2
%      
%      {\l_phd_section_before_skip_tl}%before skip#3
%      
%      {\l_phd_section_after_skip_tl}% after skip#4
%      
%      {% 
%        \setfont@cx
%        {\l_phd_section_fontweight_tl}%
%        {\l_phd_section_fontfamily_tl}
%        {\l_phd_section_fontsize_tl}
%        {\l_phd_section_fontshape_tl}%
%          \expandafter\setfontparam@cx\l_phd_section_align_tl;%
%        %  \expandafter\color{\l_phd_section_color_tl}%5
%      }
% }%
%}
%\fi
\ExplSyntaxOff
%    \end{macrocode}
% 
% \begin{docCommand}{subsection} { \meta{*} \oarg{arg1} \marg{arg2} }
%   The standard \latexe subsection is renewed next. 
% \end{docCommand}  
%     \begin{macrocode}
\ExplSyntaxOn
%\if@ltxcompat
%
%  \renewcommand\subsection
%    {
%      \@startsection{subsection}
%        {1}
%        {\l_phd_subsection_indent_tl}%indent
%        {\l_phd_subsection_before_skip_tl}%before skip#3
%        {\l_phd_subsection_after_skip_tl}% after skip#4
%      
%        { 
%        \setfont@cx
%          {\l_phd_subsection_fontweight_tl}
%          {\l_phd_subsection_fontfamily_tl}
%          {\l_phd_subsection_fontsize_tl}
%          {\l_phd_subsection_fontshape_tl}
%         % \expandafter\setfontparam@cx\l_phd_subsection_align_tl;%
%          %\color{\l_phd_subsection_color_tl}%5
%        }
%   }
%\fi
\ExplSyntaxOff
%\renewcommand\subsection{\@startsection{subsection}{2}{\z@}%
%                                     {-3.25ex\@plus -1ex \@minus -.2ex}%
%                                     {1.5ex \@plus .2ex}%
%                                     {\normalfont\large\bfseries\raggedright}}
%\fi
%
%    \end{macrocode}
%
%  
% \begin{docCommand}{subsubsection} { \meta{*} \oarg{arg1} \marg{arg2} }
%   The standard \latexe subsection is renewed next. 
% \end{docCommand}
%   \begin{macrocode}
\ExplSyntaxOn
%\renewcommand{\subsubsection}
%{
%  \@startsection{subsubsection}%
%    {3}%level
%    {\l_phd_subsubsection_indent_tl}%indent
%    {\l_phd_subsubsection_before_skip_tl}
%    {\l_phd_subsubsection_after_skip_tl}
%    {
%      %\setfont@cx
%%    {\l_phd_subsubsection_fontweight_tl }
%%    {\l_phd_subsubsection_fontfamily_tl}
%%    {\l_phd_subsubsection_fontsize_tl}
%%    {\l_phd_subsubsection_fontshape_tl}
%%      \expandafter\setfontparam@cx
%      %\l_phd_subsubsection_align_tl; not needed here
%     % \color{\l_phd_subsubsection_color_tl}
%    }
%}       
\ExplSyntaxOff
%    \end{macrocode}
% 

%
% \subsection{Paragraphs}
%
%  We now deal with paragraphs and subparagraphs, normally termed `runin’ heads, as they produce
%  headings that are inlined with the text that follows. We add hooks, so that later the key mechanism
%  can be used to pick-up values. Although they are termed runins, there is no issue to display
%  them as block.
% 
% \begin{docCommand}{paragraph} { \meta{*} \oarg{arg1} \marg{arg2} }
%  There is a feature in the standard \latexe classes that a subparagraph is indented
%  by the value of \cs{parindent}. This also features in memoir but is absent in the
%  KOMA classes. In our defaults we follow the European norm.
% \end{docCommand}
% 
%    \begin{macrocode}
\ExplSyntaxOn
%\if@ltxcompat
%\renewcommand\paragraph{\@startsection{paragraph}{4}{\z@}%
%                                    {3.25ex \@plus1ex \@minus.2ex}%
%                                    {-1em}%
%                                    {\normalfont\normalsize\bfseries}}
%\else
%\if@ltxcompat
%\else
%  \renewcommand\paragraph{%
%     \@startsection{paragraph}%
%     {4}%level
%     {\l_phd_paragraph_indent_tl}%indent
%     {\l_phd_paragraph_before_skip_tl}%
%     {\l_phd_paragraph_after_skip_tl}%
%     {
%%     \setfont@cx
%%      {\l_phd_paragraph_fontweight_tl}%
%%     {\l_phd_paragraph_fontfamily_tl}
%%     {\l_phd_paragraph_fontsize_tl}
%%     {\l_phd_paragraph_fontshape_tl}%
%%     \expandafter\setfontparam@cx\l_phd_paragraph_align_tl;%
%         \expandafter\color{\l_ph_paragraph_color_tl}%
%     }%
% }
%\fi
\ExplSyntaxOff
%    \end{macrocode}
%
% \begin{docCommand}{subparagraph} { \meta{*} \oarg{arg1} \marg{arg2} }
%  There is a feature in the standard \latexe classes that a subparagraph is indented
%  by the value of \cs{parindent}. This also features in memoir but is absent in the
%  KOMA classes. In our defaults we follow the European norm.
% \end{docCommand}
%    \begin{macrocode}
\ExplSyntaxOn  
\if@ltxcompat
\renewcommand\subparagraph{\@startsection{subparagraph}{5}{0pt}%
                                       {2ex}%
                                       {-1em}%
                                      {\normalfont\normalsize\bfseries}}
\renewcommand\subparagraph
      {
         \@startsection{subparagraph}
         {5}%level
         {\l_phd_subparagraph_indent_tl}%indent
         {\l_phd_subparagraph_before_skip_tl}
         {\l_phd_subparagraph_after_skip_tl}
         {
           \setfont@cx
           {\l_phd_subparagraph_fontweight_tl}
           {\l_phd_subparagraph_fontfamily_tl}
           {\l_phd_subparagraph_fontsize_tl}
           {\l_phd_subparagraph_fontshape_tl}%
           \expandafter\setfontparam@cx
             \l_phd_subparagraph_align_tl;
           \color{\l_phd_subparagraph_color_tl}
         }
       } 

\fi
\ExplSyntaxOff
%    \end{macrocode}
%

% 
% \cxset{section numbering=arabic}
% 
%
% \chapter{Default Settings}
%
% Setting default values
%
%
%    \begin{macrocode}
\cxset { section name              = Section,
    section format                 = block,
    section align                  = Centering,
    section title align            = Centering, %checked
%    
    section font-size              = Large,
    section font-weight            = bfseries,
    section font-family            = serif,
    section font-shape             = upshape,
%    
    section number font-size       = Large,
    section number font-weight     = bfseries,
    section number font-family     = serif,
    section number font-shape      = upshape,    
%
    section title font-size        = Large,
    section title font-weight      = bfseries,
    section title font-family      = serif,
    section title font-shape       = upshape,
%    
    section color                  = spot,
    section number prefix          = \thechapter.,
    section number suffix          =,
    section numbering              = arabic,
    section indent                 = 0pt,
    section beforeskip             = -3ex,
    section afterskip              = 10pt,
    section afterindent            = off,
    section number after           = \quad,
%    
    section arc                    = 3pt,
    section background-color       = white,
    section afterindent            = off, 
    section grow left              = 0mm,
    section grow right             = 0mm,
    section rounded corners        = northeast,
%    
    section border-left-width      = 0pt,
    section border-right-width     = 0pt,
    section border-top-width       = 2pt,
    section border-bottom-width    = 2pt,
%
    section padding-left-width     = 0pt,
    section padding-right-width    = 10pt,
    section padding-top-width      = 2pt,
    section padding-bottom-width   = 2pt,
%
    section title margin-top       = 2pt, 
    section title color            = spot,    
    section shadow                 = no shadow,  
  }   
%    \end{macrocode} 
%    \begin{macrocode}
\cxset
  { 
    subsection name                   = Subsection,
    subsection format                 = block, 
%    
    subsection font-size              = large,  
    subsection font-weight            = bfseries,
    subsection font-family            = rmfamily,
    subsection font-shape             = upshape,
%
    subsection number font-size       = large,  
    subsection number font-weight     = bfseries,
    subsection number font-family     = rmfamily,
    subsection number font-shape      = upshape,
%    
    subsection title font-size        = Large,
    subsection title font-weight      = bfseries,
    subsection title font-family      = serif,
    subsection title font-shape       = upshape,
    subsection title color            = spot,    
%        
    subsection color                  = spot,
    subsection numbering              = arabic,
    subsection align                  = Centering, %checked
    subsection title align            = Centering, %checked
    subsection beforeskip             = -3.25ex\@plus -1ex \@minus -.2ex,
    subsection afterskip              = 1.5ex \@plus .2ex,
    subsection number prefix          = \thesection.,
    subsection indent                 = 0pt,
    subsection number after           = 0pt,
    subsection background-color       = white,
%    
    subsection border-left-width      = 0pt,
    subsection border-right-width     = 0pt,
    subsection border-top-width       = 5pt,
    subsection border-bottom-width    = 5pt,
%
    subsection padding-left-width     = 0pt,
    subsection padding-right-width    = 0pt,
    subsection padding-top-width      = 20pt,
    subsection padding-bottom-width   = 20pt,                
    subsection shadow                 = drop shadow,
  }
%    \end{macrocode}
%    \begin{macrocode}
\cxset
  { 
    subsubsection name                    = Subsubsection,
    subsubsection format                  = block,  
    subsubsection background-color        = white!30, %checked
%    
    subsubsection font-family             = rmfamily, 
    subsubsection font-size               = large,
    subsubsection font-weight             = bfseries,
    subsubsection font-family             = tiresias,
    subsubsection font-shape              = upshape,
%    
    subsubsection font-family             = rmfamily, 
    subsubsection font-size               = large,
    subsubsection font-weight             = bfseries,
    subsubsection font-family             = tiresias,
    subsubsection font-shape              = upshape,
%    
    subsubsection color                   = spot,
    subsubsection number prefix           = \thesubsection,
    subsubsection number suffix           = ,
    subsubsection numbering               = arabic,
    subsubsection indent                  = 0pt,
    subsubsection beforeskip              = -3.25ex\@plus -1ex \@minus -.2ex,
    subsubsection afterskip               = 1.5ex \@plus .2ex,
    subsubsection align                   = center,
    subsubsection title align             = center,
    subsubsection number after     =,
%    
    subsubsection border-left-width       = 0pt,
    subsubsection border-right-width      = 0pt,
    subsubsection border-top-width        = 2pt,
    subsubsection border-bottom-width     = 0pt,
%
    subsubsection padding-left-width      = 0pt,
    subsubsection padding-right-width     = 0pt,
    subsubsection padding-top-width       = 20pt,
    subsubsection padding-bottom-width    = 20pt, 
    subsubsection shadow                  = no shadow,  
%
    subsubsection title font-size         = large,
    subsubsection title font-weight       = bfseries,
    subsubsection title font-family       = serif,
    subsubsection title font-shape        = upshape,
    subsubsection title color             = spot,          
  }
%    \end{macrocode}
%
%    \begin{macrocode}
% paragraph
\cxset
  {
    paragraph name                = paragraph,
    paragraph format              = inline, 
    paragraph name                = paragraph,
    paragraph font-size           = large,
    paragraph font-weight         = bfseries,
    paragraph font-family         = rmfamily,
    paragraph font-shape          = upshape,
    paragraph numbering           = alpha,
    paragraph align               = flushleft,
    paragraph beforeskip          = 3.25ex plus1ex minus.2ex,
    paragraph afterskip           = -1em,
    paragraph indent              = 0pt,
    paragraph number after        = \quad,
    paragraph color               = spot,
    paragraph background-color    = white,
    paragraph shadow              = no shadow,
  }
%    \end{macrocode}
%    \begin{macrocode}    
\cxset
  {
    subparagraph name             = subparagraph,
    subparagraph format           = inline,
    subparagraph name             = subparagraph, 
    subparagraph font-size        = large,
    subparagraph font-weight      = bfseries,
    subparagraph font-family      = rmfamily,
    subparagraph font-shape       = upshape,
    subparagraph color            = spot!30,
    subparagraph background-color = sweet!50
    subparagraph numbering        = none,
    subparagraph align            = flushleft,
    subparagraph beforeskip       = 3.25ex plus1ex minus .2ex,
    subparagraph afterskip        = -1em,
    subparagraph indent           = 0pt,
    subparagraph number after     = ,
    subparagraph shadow           = off, 
  }
%    \end{macrocode}
%
% \section{Styles}
%    \begin{macrocode}  
\cxset{chapter title style/.style= {
       chapter title align = Centering,}
 } 
\cxset{section title style/.style= {
       section title align = Centering,}
 } 
\cxset{section title style/.style= {
       section title align = Centering,}
 }
 \cxset{subsection title style/.style= {
       subsection title align = Centering,}
 }
 \cxset{subsubsection title style/.style= 
   {
        subsubsection align       = #1,
        subsubsection title align = #1,
   }
 }
 \cxset{subsubsection title style= raggedright}
%    \end{macrocode}
% \cxset{chapter afterskip=60pt}
% \chapter[test]{Tests}
%
%  We prepare a number of tests to verify that all settings work as advertized.
%
% \begin{docCommand}{testsections} {\meta{void}}
%  In honor of Barbara Beeton all testing commands are in lowercase, but we also provide
%  them in mixed case for the rest of the crowd.
% \end{docCommand}
%    \begin{macrocode}
\ExplSyntaxOn
\cs_set:Npn \testsections 
  {
    \section{Sections}
    \lorem\par
    \subsection{Subsections}
    \lorem\par
    \subsubsection{Subsubsections}
    \lorem\par
    \paragraph {Paragraph}
  }
\cs_set_eq:NN \TestSections\testsections  
\ExplSyntaxOff  
%    \end{macrocode}
% \testsections
%  \cxset{section format = hang,
%         section font-size=large,
%         section font-weight=bold,
%         subsection format=hang,
%         subsection font-family=sffamily,
%         chapter title font-size=Huge,
%         chapter number font-size=Huge,
%         chapter number font-family =sffamily ,
%         chapter align  = centering,
%}
% 
% \chapter{Another Test} 
% \section{\lorem}
% \subsection{\lorem}

%</LSECT>
\endinput
