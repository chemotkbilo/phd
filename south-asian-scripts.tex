\cxset{chapter format =fashion}

\bgroup
\arial


\chapter{South Asian Scripts}

The scripts of South Asia share so many characteristics that a side by side comparison of a few often reveal structural similarities even in the 
modern letterforms.
\medskip


\begin{center}
\begin{tabular}{lll}
\nameref{sec:Devanagari}. 
&Gujarati. &Telugu\\
\nameref{sec:bengali}
&Oriya &Kannada\\
Gurmukhi &Tamil.  
&\nameref{malayalam}\\
\nameref{sec:sinhala} 
&Kaithi  
&Meetei Mayek.\\
Tibetan 
&\nameref{sec:saurashtra} 
&Ol Chiki.\\
Lepcha  &Sharada &Sora Sompeng\\
Phags-pa &Takri &Kharoshthi\\
Limbu &Chakma. & Brahmi\\
Syloti Nagri &Mro. &\\
\end{tabular}
\end{center}

The sections that follow describe the scripts briefly and the |phd| settings
to activate the relevant commands and load appropriate fonts. 

\begin{figure}[htbp]
\includegraphics[width=\textwidth]{./images/indic-language-tree.jpg}
\caption{A family tree of a few of the most important Indic scriptsc scripts, (\textit{after Fischer})\protect\cite{writing}}
\end{figure}

\input{./languages/sinhala}
\input{./languages/meetei-mayek}
%\input{./languages/tibetan}
\input{./languages/oriya}
\input{./languages/mro}


\input{./languages/devanagari}
\section{Bengali}
\label{sec:bengali}
\idxlanguage{Bengali}
\index{Bengali fonts>Shonar Bangla}
\index{Bengali fonts>Vrinda}
\index{Bengali fonts>code2000}
\newfontfamily\bengali[Script=Bengali,Scale=1.3]{Shonar Bangla}

There are two Windows fonts that can be used with Windows \textit{Shonar Bangla} and \textit{Vrinda}. For open source fonts one can use, \texttt{code2000}.

\docAuxCommand{bengali} Once the key is set the command \cmd{\bengali} is available for use in typesetting Bengali text.

\bigskip

\bgroup



\bengali
\centering

  অ  আ ই  ঈ  উ  ঊ  ঋ  এ  ঐ\par

\fontspec[Script=Bengali,Scale=3.2]{Vrinda}

\centering

  অ  আ ই  ঈ  উ  ঊ  ঋ  এ  ঐ\par


\fontspec[Script=Bengali,Scale=3.2]{code2000.ttf}

\centering

  অ  আ ই  ঈ  উ  ঊ  ঋ  এ  ঐ\par

\captionof{table}{The consonant{\protect\bengal{} ক (kô)} along with the diacritic form of the vowels {\protect\bengal{} অ, আ, ই, ঈ, উ, ঊ, ঋ, এ, ঐ, ও and ঔ} \textit{from Wikipedia}.}
\egroup


Bengali is a Unicode block containing characters for the Bangla, Assamese, Bishnupriya Manipuri, Daphla, Garo, Hallam, Khasi, Mizo, Munda, Naga, Rian, and Santali languages. In its original incarnation, the code points U+0981..U+09CD were a direct copy of the Bengali characters A1-ED from the 1988 ISCII standard, as well as several Assamese ISCII characters in the U+09F0 column. The Devanagari, Gurmukhi, Gujarati, Oriya, Tamil, Telugu, Kannada, and Malayalam blocks were similarly all based on ISCII encodings.

\begin{scriptexample}[]{Bengal}
\unicodetable{bengal}{"0980,"0990,"09A0,"09B0,"09C0,"09D0,"09E0,"09F0}
\end{scriptexample}


\printunicodeblock{./languages/bengali.txt}{\bengal}



\bgroup
\bengali\LARGE
\char"0995 + \color{blue} \char"09BC + \color{red}\char"09AF  = \char"0995\char"09CD \char"09AF
\egroup



See also \url{http://www.nongnu.org/freebangfont/downloads.html} for additional fonts.










\section{Saurashtra}
\label{sec:saurashtra}
\idxlanguage{Saurashtra}\idxlanguage{Sourashtra}

\index{Saurashtra fonts>code2000}
\newfontfamily\saurashtra{code2000.ttf}
\def\test{}
\cxset{saurashtra font/.code=\test}
\cxset{saurashtra font=code2000.ttf}

\begin{docKey}[phd]{saurashtra font}{ = \meta{fontname}} {default none, initial = code2000}
  This key sets the saurashtra font.
\end{docKey}

Saurashtra or Sourashtra or {\saurashtra ꢱꣃꢬꢵꢰ꣄ꢜ꣄ꢬꢵ} or Palkar or Patkar (Sanskrit: सौराष्ट्र, Tamil: சௌராட்டிரம்) is an Indo-Aryan language[3] spoken by the Saurashtrian community native to Gujarat, who migrated and settled in Southern India. Madurai in Tamil Nadu has the highest number of people belonging to this community and also remains as their cultural center.

The language is largely only in spoken form even though the language has its own script. The lack of schools teaching Saurashtra script and the language is often cited as a reason for the very few number of people who actually know to read and write in Saurashtra script. Latin, Devanagari or Tamil script is used as alternative for Saurashtra Script by many Saurashtrians.

Census of India places the language under Gujarati. Official figures show the number of speakers as 185,420 (2001 census).[4]


\begin{scriptexample}[]{Saurashtra}
\unicodetable{saurashtra}{"A880,"A890,"A8A0,"A8B0,"A8C0,"A8D0}
\end{scriptexample}


\begin{scriptexample}[]{Saurashtra}
\bgroup
\saurashtra

ꢮꢶꢯ꣄ꢮ ꢱꣃꢬꢵꢰ꣄ꢜ꣄ꢬꢪ꣄ ꢦꢡ꣄ꢬꢶꢒꢾ ꢱꢵꢡ꣄ꢡꢒꢸ ꢂꢮꢬꢾ
ꢮꣁꢭꢱ꣄ꢢꢵꢥꢪꢸꢒ꣄(ꣀꢵꢮꢾꢔꢹ ꢂꢮ꣄ꢬꢶꢫꣁ


\arial

Text: Vishwa Sourashtram \url{http://www.sourashtra.info/ghEr.htm}
\egroup
\end{scriptexample}


\printunicodeblock{./languages/saurashtra.txt}{\saurashtra}

\input{./languages/gujarati}
\input{./languages/tamil}
\input{./languages/malayalam}
\input{./languages/syloti}

\input{./languages/limbu}

\input{./languages/cham}


\section{Ol Chiki script}
\arial

The Ol Chiki script, also known as Ol Cemetʼ (Santali: ol 'writing', cemet' 'learning'), Ol Ciki, Ol, and sometimes as the Santali alphabet, was created in 1925 by Raghunath Murmu for the Santali language.

Previously, Santali had been written with the Latin alphabet. But because Santali is not an Indo-Aryan language (like most other languages in the south of India), Indic scripts did not have letters for all of Santali's phonemes, especially its stop consonants and vowels, which made writing the language accurately in an unmodified Indic script difficult. The detailed analysis was given by Dr. Byomkes Chakrabarti in his 'Comparative Study of Santali and Bengali'. Missionaries (first of all Paul Olaf Bodding, a Norwegian) brought the Latin script, which is better at representing Santali stops, phonemes and nasal sounds with the use of diacritical marks and accents. Unlike most Indic scripts, which are derived from Brahmi, Ol Chiki is not an abugida, with vowels given equal representation with consonants. Additionally, it was designed specifically for the language, but one letter could not be assigned to each phoneme because the sixth vowel in Ol Chiki is still problematic.
Ol Chiki has 30 letters, the forms of which are intended to evoke natural shapes. Linguist Norman Zide said "The shapes of the letters are not arbitrary, but reflect the names for the letters, which are words, usually the names of objects or actions representing conventionalized form in the pictorial shape of the characters."[1] It is written from left to right.

\newfontfamily\olchiki{code2000.ttf}

\begin{scriptexample}[]{olchiki}
\bgroup
\olchiki
\obeylines

U+1C5x 	᱐	᱑	᱒	᱓	᱔	᱕	᱖	᱗	᱘	᱙	ᱚ	ᱛ	ᱜ	ᱝ	ᱞ	ᱟ
U+1C6x	   ᱠ	ᱡ	ᱢ	ᱣ	ᱤ	ᱥ	ᱦ	ᱧ	ᱨ	ᱩ	ᱪ	ᱫ	ᱬ	ᱭ	ᱮ	ᱯ
U+1C7x  	ᱰ	ᱱ	ᱲ	ᱳ	ᱴ	ᱵ	ᱶ	ᱷ	ᱸ	ᱹ	ᱺ	ᱻ	ᱼ	ᱽ	᱾	᱿
\egroup

\unicodetable{olchiki}{"1C50,"1C60,"1C70}
\end{scriptexample}

\section{Lepcha}
\newfontfamily\lepcha{Mingzat-R.ttf}

The Lepcha script, or Róng script is an abugida used by the Lepcha people to write the Lepcha language. Unusually for an abugida, syllable-final consonants are written as diacritics.

The Mingzat font is still under development by SIL so I am not too sure if the rendering is correct\footnote{\url{http://scripts.sil.org/cms/scripts/page.php?site_id=nrsi&id=Mingzat}}.

\begin{scriptexample}[]{Lepcha}
\bgroup
\lepcha
\obeylines
 	    0	1	2	3	4	5	6	7	8	9	A	B	C	D	E	F
U+1C0x	 ᰀ	ᰁ	ᰂ	ᰃ	ᰄ	ᰅ	ᰆ	ᰇ	ᰈ	ᰉ	ᰊ	ᰋ	ᰌ	ᰍ	ᰎ	ᰏ
U+1C1x	 ᰐ	ᰑ	ᰒ	ᰓ	ᰔ	ᰕ	ᰖ	ᰗ	ᰘ	ᰙ	ᰚ	ᰛ	ᰜ	ᰝ	ᰞ	ᰟ
U+1C2x	 ᰠ	ᰡ	ᰢ	ᰣ	ᰤ	ᰥ	ᰦ	ᰧ	ᰨ	ᰩ	ᰪ	ᰫ	ᰬ	ᰭ	ᰮ	ᰯ
U+1C3x	 ᰰ	ᰱ	ᰲ	ᰳ	ᰴ	ᰵ	ᰶ	᰷	x	x	x	᰻	᰼	᰽	᰾	᰿
U+1C4x	 ᱀	᱁	᱂	᱃	᱄	᱅	᱆	᱇	᱈	᱉	x	x	x	ᱍ	ᱎ	ᱏ

\egroup
\end{scriptexample}

\section{Sharada}

The Śāradā, or Sharada, script (शारदा) is an abugida writing system of the Brahmic family of scripts, developed around the 8th century. It was used for writing Sanskrit and Kashmiri. The Gurmukhī script was developed from Śāradā. Originally more widespread, its use became later restricted to Kashmir, and it is now rarely used except by the Kashmiri Pandit community for ceremonial purposes. Śāradā is another name for Saraswati, the goddess of learning.
Śāradā script was added to the Unicode Standard in January, 2012 with the release of version 6.1.

The Unicode block for Śāradā script, called Sharada, is U+11180–U+111DF: Unable to locate font in unicode.


\section{Sora Sompeng}

Sorang Sompeng script is used to write in Sora, a Munda language with 300,000 speakers in India. The script was created by Mangei Gomango in 1936 and is used in religious contexts.[1] He was familiar with Oriya, Telugu and English, so the parent systems of the script are Brahmi and Latin.[2]
The Sora language is also written in the Latin alphabet and the Telugu script.

Sorang Sompeng script was added to the Unicode Standard in January, 2012 with the release of version 6.1. Nirmala UI.ttf (Windows 8.1)
\newfontfamily\NirmalaU{Segoe UI Symbol}


\unicodetable{NirmalUI}{"110D0,"110E0,"110F0}
 	
This did not work with Windows 7, and the experiment failed. 



\section{Phags-pa}
\label{s:phagspa}
\newfontfamily\phagspa{code2000.ttf}
\arial 
The 'Phags-pa script,[1], (Mongolian: дөрвөлжин үсэг "Square script") was an alphabet designed by the Tibetan monk and vice-king Drogön Chögyal Phagpa for the Mongol Yuan emperor Kublai Khan as a unified script for the literary languages of the Yuan.

Widespread use was limited to about a hundred years during the Yuan Dynasty, and it fell out of use with the advent of the Ming dynasty. The documentation of its use provides clues about the changes in the varieties of Chinese, the Tibetic languages, Mongolian and other neighboring languages during the Yuan era.
\medskip


\includegraphics[width=1\linewidth]{./images/phags-pa.jpg}

credit \protect\url{http://turfan.bbaw.de/dta/monght/images/monght009_seite2.jpg}



\begin{scriptexample}[]{Phags-pa}
\bgroup
\unicodetable{phagspa}{"A840,"A850,"A860,"A870}

\arial
\hfill Typeset with \texttt{code2000.ttf} and \cmd{\phagspa}

\egroup
\end{scriptexample}
\medskip

Phags-pa is a historical script related to Tibetan that was created as the national script of
the Mongol empire. Even though Phags-pa was used mostly in Eastern and Central Asia for
writing text in the Mongolian and Chinese languages, it is discussed in this chapter because
of its close historical connection to the Tibetan script. The script has very limited modern use. It bears similarity to Tibetan and has no case distinctions. It is written vertically in columns running for left to right, like Mongolian. Units are often composed of several syllables and sometimes are separated by whitespace.


\printunicodeblock{./languages/phags-pa.txt}{\phagspa}

\cxset{script/.code={}}
\cxset{script=phags-pa}

\begin{docKey}[phd]{script}{ = \meta{phags-pa}} {}
The key |script| will activate the commands available for typesetting the phags-pa script.
\end{docKey}





\input{./languages/chakma}

\input{./languages/brahmi}

\arial
\begin{longtable}{lr}
Common	&6412\\
Latin	&1272\\
\nameref{s:greek}	&511\\
Cyrillic	&417\\
Armenian	&91\\
\nameref{s:hebrew}	 &133\\
Arabic	 &1234\\
\nameref{s:syriac}	 &77\\
Thaana	 &50\\
Devanagari	&151\\
Bengali	&92\\
Gurmukhi	&79\\
Gujarati	&84\\
Oriya	&90\\
Tamil	&72\\
Telugu	&93\\
Kannada	&86\\
Malayalam	&98\\
Sinhala	&80\\
Thai	&86\\
Lao	 &67\\
Tibetan	&207\\
Myanmar	&188\\
Georgian	&127\\
Hangul	   &11739\\
Ethiopic	&495\\
Cherokee	&85\\
Canadian Aboriginal	 &710\\
Ogham	&29\\
\nameref{s:runic}	&78\\
\nameref{s:khmer}	&146\\
Mongolian	&153\\
Hiragana	&91\\
Katakana	&300\\
Bopomofo	&70\\
Han	 &75963\\
\nameref{s:yi}	&1220\\
Old Italic	&35\\
\nameref{s:gothic}	 &27\\
Deseret	&80\\
Inherited	&524\\
Tagalog	&20\\
Hanunoo	&21\\
Buhid	&20\\
Tagbanwa	1&8\\
Limbu	&66\\
Tai Le	 &35\\
Linear B	&211\\
Ugaritic	&31\\
Shavian	&48\\
Osmanya	&40\\
Cypriot	&55\\
Braille	&256\\
Buginese	&30\\
Coptic	 &137\\
New Tai Lue	&83\\
Glagolitic	&94\\
Tifinagh	&59\\
Syloti Nagri	&44\\
Old Persian	&50\\
Kharoshthi	&65\\
Balinese	&121\\
Cuneiform	&982\\
Phoenician	&29\\
Phags Pa	&56\\
Nko	 &59\\
Sundanese	&72\\
Lepcha	 &74\\
Ol Chiki	&48\\
\nameref{s:vai}	&300\\
Saurashtra	&81\\
Kayah Li	&48\\
Rejang	 &37\\
Lycian	 &29\\
Carian	 &49\\
Lydian	 &27\\
Cham	 &83\\
Tai Tham	&127\\
Tai Viet	&72\\
Avestan	&61\\
Egyptian Hieroglyphs	&1071\\
Samaritan	 &61\\
Lisu	&48\\
Bamum	&657\\
Javanese	&91\\
Meetei Mayek	&79\\
Imperial Aramaic	&31\\
Old South Arabian	&32\\
Inscriptional Parthian	 &30\\
Inscriptional Pahlavi	&27\\
Old Turkic	&73\\
Kaithi	 &66\\
Batak	 &56\\
Brahmi	 &108\\
Mandaic	&29\\
Chakma	&67\\
Meroitic Cursive	&26\\
Meroitic Hieroglyphs	&32\\
Miao	&133\\
Sharada	&83\\
Sora Sompeng	&35\\
Takri	&66\\
	
	&110181\\
\end{longtable}

\egroup