%\def\chaptername{Chapter}
%\input{./styles/style87a}
\cxset{ 
           %chapter toc=true,
           chapter numbering=arabic,
           chapter number color=black,
           chapter number font-shape=upshape,
           subsubsection numbering=none,
           subsubsection font-family=itshape,
           subsubsection color=black,
           subsection number after=\quad,
          section number after=\quad,
          section color=black,
    }
\def\thesubsubsection{}         
%\pagenumbering{gobble}
\def\JV{HLS DSE-JV\xspace}
\def\letter#1{\texttt{HLSDSEJV/HC/L/YL/#1}\xspace}
\def\KA{K\&A}
\def\DT#1{HLG Transmittal Ref. No.: \texttt{HLG-626-DT-HLS-#1}\xspace}
\def\idxbusbar#1{\index{Busbar Delays>#1}}
\def\idxwestin#1{\index{Westin Delays>#1}}
\def\idxstregis#1{\index{St. Regis Delays>#1}}
\def\idxahu#1{\index{Air Handling Unit Delays>#1}}
\let\idxahus\idxahu
\def\CAR#1{\index{Cost Adjustment Requests>CAR-#1}{\texttt{CAR-#1}}\xspace}
\def\idxbasement#1{\index{Basement delays>#1}}
\let\basement\idxbasement
\def\idxdewa#1{\index{Dewa Approvals>#1}}


\mainmatter
\pagestyle{plain}
\cxset{chapter name=,
          chapter numbering=none}
\chapter{Executive Summary}
\thispagestyle{empty}

This short report provides background information related to  the Habtoor City Project MEP works and the steps taken by the \JV to accelerate the works, under the instructions of the Client, Engineer and Main Contractor.  We mobilized to the Project late August 2013. At the time construction was on-going, with the basements structures mostly completed. On mobilization the only K\&A MEP designs available were those provided with the tender package---which was issued in March~2013. Besides procurement and some engineering activities, the \JV  construction activities were mainly focused on builder's works and remaining underground services until March 2014. 

We started receiving design drawings in March and April 2014. The design was issued piecemeal and in out of sequence fashion for the works to progress as planned and according to the agreed Baseline Program . This enabled us to proceed with works only in the Car Parking Areas of the Basements.  The first partially workable set of design drawings received to enable construction in other areas were the drawings received in September 2014 (Mechanical) and December 2014 (Electrical).


\medskip
		
\paragraph{Delayed Incomplete and Unworkable MEP Designs} The general issue of drawings in September~14, provided general design concepts without concerns for physical plant and ceiling constraints. The Plant rooms at T1 and PD6, as designed were not constructible, as the allocated headroom and space was inadequate. We assisted the Engineer by providing 3D and other drawings to at least fit the equipment in the available space. Fans had to be relocated in ceilings at Podium 1, and ducting was re-routed over the same ceiling void. This delayed finalization of Shop Drawings for essentially all the public areas.


The K\&A \enquote{design} mechanical design for St. Regis was only partially completed in September 2014. This design was deficient in many respects, especially in areas such the Technical floors, and as it stood the design was not constructible. This design was inadequate to close equipment orders for long delivery plant, such as AHU, fans and pumps, as calculations for static pressures could not progress. However, we took the initiative to finalize orders based on estimates and released orders before design finalization. We also assisted the Engineer with solving many of the design issues in order to progress with the works.  In addition the Electrical works suffered because of the designs issued in September 2014, as they have not been co-ordinated with the requirements of the Mechanical plant, Kitchen Contractors etc. \par

The delays  to the completion of the final Project requirements are still on-going with many areas of the Hotels still under design development and without related subcontractors appointed on time.

\begin{table}[ht]
\centering

\begin{tabular}{l l p{3cm}  l l}
\toprule
        &Area         &\raggedright Design required as per baseline program & Design Issued & Delay\\
\midrule        
\inc  &First Floor &16 Apr 14  &5 Jan 15  &  8 months\\
\inc  &Attic Floor & 8 Apr 14  &5 Jan 15  & 8 months \\
\inc  &Podium 6  &1 Apr 14   &6 Sep 14  & 5 months \\
\inc  &Podium 5  &29 Mar 14 &6 Sep 14  & 5 months\\
\inc &Podium 4   &20 Mar 14 &6 Sep 14  & 5.5 months\\
\inc &Podium 3  &12 Mar 14  &6 Sep 14  & 5.5 months\\
\inc &Technical 1 &6 Feb 14   &6Sep 14    &7 months\\
\inc &Mezzanine &26 Dec 13  &6 Sep 14  &9 months\\ 
\bottomrule
\end{tabular}
\caption{Design delays for St. Regis}

\end{table}

The MEP Good for Engineering Designs as received from K\&A enabled part of the Engineering and Procurement activities to start bu the design as it stood was  proceed, they are not sufficient to install MEP services. Drawings from ID Consultants, Lighting Consultants, Kitchen Consultant, ELV Consultants and subcontractor Shop Drawings for the same are necessary. These were mostly unavailable.

\paragraph{Instruction to accelerate the works}
Under this background we received the instruction to  accelerate the works (July 2014). We wrote to to the Main Contractor, requesting that a plan be first agreed as to how program recovery could be achieved and \emph{then} agree to a plan to accelerate the works further, so as to bring the Contract Completion dates forward. The request was to accelerate the St. Regis Hotel first with a Target Completion date of 30 March 2014.

At the time approximately 40\% of the slabs  for St. Regis were incomplete. This included critical plant areas at the two technical floors. Not only the structure had to be completed, but also the technical floor, had to have floating floors casted. The T1 floor was partially handed over to us end October and the PD6 floor in January 2015. As is also evident from the subsequently issued Design MEP Drawings, ID Drawings, Lighting Consultant and ELV Consultant drawings issued, the Professional Team was not ready with their Designs. 

As MEP works are closely interlinked with other trades it is important to note that the Structure Cabling, Kitchen Subcontractors, AV and CCTV Subcontractors were not appointed. 
\medskip

 
   


\label{acceleration}
\index{acceleration>manpower}\index{manpower>acceleration}
\paragraph{JV actions taken to accelerate the works.} Once the information started flowing, we reinforced our Engineering and Site Teams. We also added technicians as areas opened to us for work.
\medskip

\noindent\textit{Workforce}
\medskip

\noindent The \JV upon receipt of the instructions to accelerate, and under the impression that designs and appointments of other subcontractors would be accelerated as well, doubled the workforce in July~2014 and subsequently added technicians and other staff until it is at its current level of approximately 3000 personnel. The deployment of personnel is shown in the table below.

\begin{table}[hbp]
\begin{tabular}{c c c c c c c c c}
\toprule
Item &Sep 13 &Feb 14 &Mar 14 & Jul-14 & Aug-14 &Oct-14 & Jan-15 & Mar-15\\
\midrule
 Site Labour   & 48      &610      & 634     & 1212   &  1300     & 1845   &2 781   & 2 731 \\
\bottomrule
\end{tabular}
\end{table}

Although issues prohibited us from fully handing over areas and ceiling closures, the quantum of the work achieved in this short time can be gauged from the gross claimed amount of close to AED~280,000,000.00. (April~14-April~15). 
\medskip

\noindent\textit{Air-freighting of equipment}
\medskip

\noindent In addition to adding personnel we proceeded to air-freight the following equipment, without which the program recovery would have failed:

\begin{enumerate}
\item Chilled water pumps. The chilled water pumps were necessary to be delivered as early as possible in order to enable piping to be connected and for providing wild air as possible. The first submittal for pumps was made on the 25 February 2014. This was returned on the 26 March 2014. The pumps were again resubmitted in 23 April 2014, after revisions to match changes in equipment. They were returned after 40 days, despite the fact that at the time the Engineer was asking us to accelerate the works. Third and fourth submittals followed and the pumps finally approved on 3 July 2014. Pump heads were reverified to meet new layouts and the order place in August, after opening LCs and finalizing prices with Supplier. Cost AED 50,000.00. 
\index{airfreight>chilled water pumps}
\index{chilled water pumps>air freight costs}

\item First fan coil units deliveries for St Regis. These were subjected to similar delays and 388 fan coil units were air-freighted from Thailand at a cost of 196,539.60~AED. 

\item Air Separator for the St Regis Plantroom was air-freighted at a cost of AED~14,185.00.
\item All fans for St Regis. Many of these fans were to be installed in ceiling voids. These were air-freighted at a cost of AED~221,772.00. This also included air-freighting charges for fire rated motors to be air-freighted from Brazil to the Nuaire Wales factory.

\item The above secured the St Regis Hotel plant room areas.

\item Air-freighting of ECUs and Basement fans was stopped after Client Representative wrote us a letter that they would not consider paying for the above costs. \index{Ecology Units>air freight} \index{air freight costs}

These were sea-freighted, with a consequent further compression in the program of works and delaying completion of the following areas:

\begin{enumerate}
\item Basement areas
\item PD6 St Regis Plantroom
\item Kitchens
\end{enumerate}
\end{enumerate}


These are also expected to delay commissioning of kitchen areas in the basement and the Car Park Ventilation System.

\paragraph{Focus of the claim}

The claim should focus on the following:

\begin{enumerate}
\item Establishing the time extension claim. This should not be too difficult given the delays in design information. Also the casting on all buildings had considerable delays (recorded in the weekly  reports). The biggest delays in casting occured in the "W" hotel. Considerable delays in the issue of provisional sums information was another source of delay. These are listed in the last section of this report. Engineering has all the details as to when the information was released to us. I have recorded the dates we required the information in order not to incur delays.

\item Establishing disruption. Unless this element can be argued successfully, the monetary claim will be insignificant. We should at least try and recover 20\% on the labour component.

\item Establishing acceleration. The Site Team needs to provide the Claim Consultant's an accurate status of the Project and details as to what is still incomplete. Please give attention to teh fact that there are two  non-binding contractual milestones. the completion of basement works and teh completion of the St. Regis Hotel by July 2015. These milestones need to be achieved in order to validate the acceleration part of the claim.
\end{enumerate}

Between the above three components we should be able to Claim in excess of  AED 30 million, which hopefully would recover the higher labour costs incurred.

The Sections that follow are general outlines and an incomplete list of what can be claimed. 
Full information is available with the Engineering Team and the Commercial Team.

\setcounter{chapter}{0}





%\chapter{Summary MEP Progress Report for St Regis Hotel, Habtoor City}
%\pagenumbering{arabic}
%\thispagestyle{plain}
%\section{Current Status}
%
%We have started flushing of the Chilled Water system on the 7 April 2015, as planned and we anticipate to be in a position to progressively provide \emph{wild air} before the end of April, ahead of the scheduled date of the 7 May 2015. In the Basements and in the Guest rooms we have started final fix works, where possible. The Main Plantrooms at Technical Floors 1 and Podium 6, are in the main completed, except final ductwork connections where they impede access. BMS DDC Panels are expected to arrive by the 22 April 2015 and installation expected to be completed within 25-30 days to ensure that by end May we can provide controlled conditions.
%
%Delays have been experienced in the receipt of Electrical panels, such as DBs (delayed due to late deliveries of components by Legrand) and others that were subjected to numerous changes, as described later on.  Other delays were due to late instructions as briefly detailed in Section~\ref{delays}. 
%
%The current outstanding works for the St Regis Hotel are as follows:
%
%\subsection{St Regis Basement}
%
%\begin{description}
%\item[Kitchen Corridors] Some kitchen corridors cable pulling is still under progress. Expected to complete by 30 Apr 2015.
%\item[Main Electrical Room] Delays experienced due to the failure of cable trays during cable pulling and also due to the some of the MDBs being returned to the factory for modifications, as they failed QA/QC Inspections.
%\item[Fan Rooms] Fans scheduled to be delivered 23 Apr 2015.
%\item[BMS] DDC Panels still to be delivered.
%\item[Sump Pumps] Expected to be delivered by 10 May 2015. 
%\item[Others] There are still closure related works, for areas currently inaccessible, such as the new ramp areas, store and office areas. 
%\end{description}
%
%\subsection{Ground Floor}
%\begin{description}
%\item[Ballroom] This area is still under scaffolding being used by the Main Contractor to erect walk-ways in the ceiling. Once the scaffolding is dropped and we are given access to the lower level, we have to install another layer of services, give ceiling grid clearances and upon construction of the ceiling grid we can then install final sprinkler droppers and give clearances for final boarding.
%\item[Banquet Hall] This area has been delayed due to the Iridium Spa delays in Design and appointment of subcontractors. As this area is above the Banquet Hall, coring for drainage pipes delayed the works. This coring is now complete and we expect to ask the Main Contractor to lower the scaffolding and start with the rest of the services.
%\end{description}
%\subsection{Mezzanine}
%\begin{description}
%\item[Festival Dining Restaurant] Currently this area is under nomination, there is no ID Design and final details are still awaited. 
%\item[Security Room] The design for this room has recently changed. The room as shown in the new designs is different from what has been constructed on site and has no space for CCUs. 
%\item[AV Room] Expected to be completed 30 Apr 2015.
%\item[Furniture Store] Expected to be completed 25 Apr 2015.
%\item[Balance Corridors] Expected to be completed 25 Apr 2015.
%\end{description}
%
%\subsection{Podium 1}
%
%\begin{description}
%\item[Banquet A/V Technician] We have no access. This is currently being used as a store.
%\item[Service Corridor] Plan to release for ceiling grid on 23 Apr 2015.
%\item[St Regis Main Kitchen and Corridor] Plan to release on 30 Apr 2015.
%\item[Property Store] Currently no access. If access provided we can release by 30 Apr 2015.
%\item[Steak House Kitchen] Plan to release by 30 Apr 2014.
%\end{description}
%
%\subsection{Podium 2}
%\begin{description}
%\item[Iridium Spa and related areas] We are currently working in the area, which was delayed by late appointment of Finishing Contractor. Still some ID Shop Drawings not available. We expect to catch-up with delays by end May 2015. We plan to complete final fix by 10 June 2015 and Testing and Commissioning by 20 Jul 2015.
%\item[Other Areas] All other areas will be released for closure by 26 Apr 2015.
%\end{description}
%
%\subsection{Podium 3-6}
%
%All guestrooms have been handed over for ceiling closures with the exception of some of the suites, where information and access was provided late. These are the following:
%
%\begin{description}
%\item[Ambassador Suite] Co-ordination ongoing. Expect resolution and final clearances 25 May 2015.
%\item[Bentley Suite] Incomplete information. Completion targets uncertain at this stage.
%\item[Royal Suite] Co-ordination on-going. Expect resolution and final clearances 25 May 2015.
%\end{description}
%
%\subsection{Floor 1}
%
%\begin{description}
%\item[Kitchen 4 and Kitchen 6] Works for walls are progressing, insufficient detail information. Can complete by 15 May 2015, provided all Kitchen Subcontactor’s drawings become available and unimpeded access.
%\end{description}
%
%\section{Delays in Target Dates}
%\label{delays}
%This is a brief summary of recent selected instructions for additional works that have impacted  MEP Progress. 
%In addition to these additional works another critical factor that affected progress was the congestion of services and the numerous RFIs and responses we had to raise in order to resolve them.
%
%\begin{itemize}
%\item Relocation of Kitchen Extract ducting Ground Floor, Mezzanine and Podium BOH areas.
%\item  Additional AV points in all public areas.
%\item  Additional telephone, data and CCTV points in all Public Areas.
%\item  Motorized curtains Meeting Rooms.
%\item Lighting Control System. 
%\item Emergency Lighting System. (see details Chapter~\ref{emergencylights})
%\item Changes to Electrical DBs, SMDBs due to late receipt of DEWA approved drawings. (See Chapter~\ref{electrical})
%\end{itemize}
%
%We have reacted as fast as possible to all instructions and as soon they were received we have added resources to mitigate delays. Where days slipped these are only by a few days and we are confident that by end of this month all physical installations will be completed with the exception of the English Pub, Banquet and Royal Suite. 
%
%\subsection{Back of the House Areas}
%
%All back of the House Areas experienced delays, due to the lack of primary co-ordination at design stage. This caused delays until solutions were found enabling us to install the services. 
%
%The allowable ceiling height in this area was impossible to be achieved and the kitchen extract duct eventually was split in two sections and distributed through two different routes in order to avoid passing it through the corridors which could not accomodate it.
%
%In addition a new roller shutter window was introduced, that made it impossible to install the fresh air ducts feeding the kitchen. After several attempts by |K&A| to find an acceptable solution the roller shutter  window was abandoned as per the instructions of the Client Representative. 
%
%\subsection{Basement Kitchen and Related Areas at B1}
%
%Please note that these areas (with the exception of the corridor) have been cleared for ceiling grid closures in most areas and the balances are as per target to close by the 15 April 2015, including additional works. The additional works were mostly for additional ELV points on walls and for which we have received drawings on the 29 March 2015. We have instituted overtime and added additional crews to complete the works as fast as possible. Most rooms in the area have been affected. 

\begin{comment}
\chapter{Busbar System}

As per the approved Baseline Program we expected to place the busbar order for all three hotels on 27 February 2014. However, HLS DSE-JV were unable to place any orders due to the events that are outlined below, with finality on all busbars only achieved in April 2015. 

\begin{enumerate}
\item On the 23 December 2013 we were requested to change the specification for some busbars via HLG transmittal Ref. No. HLG-626-DT-HLS-0628 dated 23 Decemeber 2013 \textit{Fire Resistance Bus Bar Specification}.

\item On the 25 February 2014 we were issued revised designs via tranmittal Ref. No. HLG-626-DT-HLS-0873 \textit{Revised Electrical Drawings}.

\end{enumerate}


\chapter{Generators}

\section{Generator Ventilation}

\subsection{Background}

The original tender drawings indicated the Generator Ventilation to be by means of Louvres. When such an approach is taken normally the ventilation openings are dictated by the size of the generators.


HLS DSE-JV have submitted as early as 2014 RFIs outlining concerns regarding the adequacy of the ventilation openings and sizing of Generator rooms in the basements.

On the 25 March 2015, we were instructed to proceed with the purchase of additional fans from Systemaire. We issued the order request on the ..... and the order placed on the ......  without formal approval of the amounts in order to speed up the purchase. This affected the commissioning of the generators.

\chapter{Transformer Room Ventilation}

\subsection{Background}

\subsection{Design Errors}
\end{comment}


\cxset{chapter name=Section,
          chapter numbering=arabic}
\chapter{Emergency Lighting System}
\label{emergencylights}
The Emergency Lighting System was finalized on the 22 February 2015. This is impacting on the final fix and commissioning of the Hotel’s Central Battery and Emergency Lighting System. 

\begin{enumerate}
\item As per the approved Baseline Program, we were planning to submit the Material Submission of the Emergency Lighting System by the 25 Feb 2014.
\item On the 25 Nov 2013, we raised RFI \texttt{HLS-DSE/142 JV-RFI-MEP-E028} requesting full details of the Emergency Lights as well as the capacity of the central battery system in order to proceed with Technical Submittals, design of containment system and procurement of equipment.
\item On the 12 Dec 2013 we received an insufficient reply to the above mentioned RFI. We have notified you that the repsonse was insufficient via letter \texttt{HLS DSE/JV/HLG/YL1181} dated 14 Jan 2014, clearly stating that we were unable to proceed further with the submission of the Central Battery System, until the requested information was provided. In our letter we had requested that all details such as diffuser details, base type, IP rating and lamp characteristics are provided. We have also provided details as to Civil Defence requirements.
\item The above concerns were forwarded to the Engineer by the Main Contractor on the 20 Jan 2014. The Engineer instructed us to follow the current design dawings until the completion of the Lighting Consultant’s works.

\item On 10 Feb 2014, we had responded via letter \texttt{HLS-DSE/JVHLG/YL/1227} stating that the information provided by the Engineer, as response to RFI HLS-DSE/142 MEP-E028 was inadequate to produce Shop Drawings and to proceed with material procurement or calculations.
\item On 19 March 2014, once again we responded via letter HLS DSE/JV/626/2.05/YE/nd/2609/14 dated 4 Mar 2014 stating that the inforamtion was inadequate.
\item On 16 April 2014 we sent a clear notification that the lack of information was expected to delay the works via letter \texttt{HLS DSE/JV/HC/L/YL/1322} stating that we were unable to proceed with this portion of the works.

\item On the 20 August 2014 we received via an email instructions to proceed based on a generalized scheme.
\item We raised RFI-MEP-E249 dated 21 Sep 2014, requesting more details on locations and quantities of Emergency Light Fittings. The RFI response was received on 13 Oct 2014 with the response to follow the latest issued Guest Room drawings. 
\item Engineer’s letter \texttt{DU1211/DU/L20054/14} dated 15 Sep 2014, confirmed that due to several ID Design issues the above details were no longer applicable.
\item On 30 Sep 2014 we served notices regarding additional works due to revisions of the Emergency Lighting System for all three hotels.
\item On 15 Nov 2014, we raised concerns due to late finalization of the Central Battery System for W and Westin Hotels. 
\item On the 20 Dec 2014 the we received instructions from the Engineer and Client requesting us to revert back to the original K\&A designs.
\item On the 22 Feb 2015, the Engineer instructed us to procure and install all the Front of House exit lights. We confirmed receipt of the instruction via letter \texttt{YL/1935} dated 24 Mar 2014 once all final details and samples were finalized.
\end{enumerate}


















