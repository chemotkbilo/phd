\chapter{Rust}


\section{Rust Project Tree}

Once we timestamp the compilation our Project tree has now some more files.

\includegraphics[width=\textwidth]{rust-project-tree}


\begin{minted}[fontsize=\footnotesize]{rust}
// from Chapter 3/code/ifelse.rs
fn main() {
  let dead = false;
  let health = 48;
  if dead {
    println!("Game over!");
    return;
  }
  if dead {
    println!("Game over!");
    return;
  } else {
    println!("You still have a chance to win!");
  }
  if health >= 50 {
    println!("Continue to fight!");
  } else if health >= 20  {
    println!("Stop the battle and gain strength!");
  } else {
    println!("Hide and try to recover!");
  }
}
\end{minted}


\section{Functions}

Every rust program starts with the |fn main()|. This can be further be subdivided into separate
functions to reuse code or for better organization. 


\section{Macros}

\section{Windows issues}

http://msys2.github.io/


