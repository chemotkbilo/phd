% \iffalse meta-comment
%<*internal>
\iffalse
%</internal>
%<*readme>
----------------------------------------------------------------
phd-runningheads 
A package to manage epigraphs in LaTeX documents
E-mail: yannislaz@gmail.com
Released under the LaTeX Project Public License v1.3c or later
See http://www.latex-project.org/lppl.txt
----------------------------------------------------------------
%</readme>
%<*readmemd>
###The `phd` LaTeX2e package

The `phd-epigraphs` latex package is part of the `phd` package
buddle and the class with the same name. The `phd` package provides
convenient methods to create new styles for books, reports
and articles. It also loads the most commonly used packages 
and resolves conflicts. The `phd-fontmanager` provides functions
and key values for managing font usage in documents.

This work consists of the file  `phd-epigraphs.dtx`,
and the derived files   `phd-epigraphs.ins`, `phd-epigraphs.pdf`, 
and `phd-epigraphs.sty`.

This is version 0.10.0

###Installation

run
          phd-lua.bat on windows
          pdflatex phd.dtx
          makeindex -s phd.ist -g phd.idx 

If you have any difficulties with the package come and join us at
http://tex.stackexchange.com and post a new question
or send me a message at  yannislaz at gmail.com. Please note
that at this stage the package is not production stable but close
to completion. It will be released as a bundle with the `phd` package.
The `phd` package loads `phd-fontmanager` automatically.

### Documentation

The package was written using the `doc` and `docscript` packages,
so that it is self documented in a literary programming style. 
The .pdf is a fat document, providing over fifty book styles (the
equivalent of classes) plus there is a lot of write-up on the inner
workings of TeX and LaTeX2e. However, you don't need to know much
to use it.

      \usepackage{phd}
      \input{style13}

All choices, are made via an extended key-value interface. 
Although not a compliment, it resembles CSS and the keys are a bit verbose but
attributes are easy to change and have a consistent and easy to remember interface.

To set or add a key we only use one command:

      \cxset{chapter name font-size = Huge,
             chapter number font-size = HUGE} 

### Future Development

This is still an experimental version, but I will retain the
interface in future releases. There is a large amount of
work still to be carried out to improve the template styles
provided, to test it more thoroughly and to add a number of
improvements in the special designs. At present I estimate
that I have completed about 70% of the work that needs
to be done.

__The package as it stands is not production stable.__ 


%</readmemd>
%<*todo>
Improve on User markup
%</todo>
%<*internal>
\fi
\def\nameofplainTeX{plain}
\ifx\fmtname\nameofplainTeX\else
  \expandafter\begingroup
\fi
%</internal>
%<*install>
\input docstrip.tex
\keepsilent
\askforoverwritefalse
\preamble
----------------------------------------------------------------
phd-epigraphs
A package to manage epigraphs in LaTeX
E-mail: yannislaz@gmail.com
Released under the LaTeX Project Public License v1.3c or later
See http://www.latex-project.org/lppl.txt
----------------------------------------------------------------
\endpreamble
\postamble
 Copyright (C) 2015 by Dr. Yiannis Lazarides <yannislaz@gmail.com>
\endpostamble
%\usedir{tex/latex/\jobname}
\generate{
  \file{\jobname.sty}{\from{\jobname.dtx}{EPI}}
 }
%</install>
%<install>\endbatchfile

%<*internal>
%\usedir{source/latex/\jobname}
\generate{
  \file{\jobname.ins}{\from{\jobname.dtx}{install}}
}
\nopreamble\nopostamble
%\usedir{doc/latex/demopkg}
\generate{
  \file{README.txt}{\from{\jobname.dtx}{readme}}
}
%\generate{
%  \file{TODO.tex}{\from{\jobname.dtx}{TODO}}
%}
\generate{
  \file{phd-epigraphs.md}{\from{\jobname.dtx}{readmemd}}
}
\ifx\fmtname\nameofplainTeX
  \expandafter\endbatchfile
\else
  \expandafter\endgroup
\fi
%</internal>
%<*driver>
\NeedsTeXFormat{LaTeX2e}
\ProvidesFile{phd-epigraphs.ins}%
  [2013/01/13 v1.0 ]%
\documentclass[oneside,11pt,a4paper]{ltxdoc}
\usepackage[bottom=2cm]{geometry}
\savegeometry{std}
\usepackage{phd}
\sethyperref
\EnableCrossrefs
\CodelineIndex
\RecordChanges
\begin{document}
  \let\bold\bfseries
 \coverpage{asia}{Book Design }{Camel Press}{EPIGRAPH}{DESIGN} 
 \secondpage
  
  \newpage
  
  \tableofcontents
  \mainmatter
  \pagestyle{myheadings}
%  \parindent1em
%\cxset{style13}
%\cxset{title margin bottom=10pt,
%          title beforeskip=1pt}

\chapter{Introduction}
\addtocimage{-12pt}{-20pt}{../images/tocblock-fish.jpg}


\epigraph{``Begin at the beginning,'' the king said
"and then go on till you come to the end, then stop."}{
---Lewis Carroll, Alice in Wonderland}


\noindent This package and its documentation attempts to eliminate some common 
problems encountered when using \LaTeX2e. The first one is the loading of 
recommended packages for a large and perhaps complicated document and 
the second is the re-designing of styles for a document.

 \LaTeX2e, does not provide a standard library, but comes equipped with
 a package mechanism that allows code extensions to be loaded as required.
 This has created a strong vibrant community, hundreds of packages and a 
 headache to both new and seasoned users. What packages are available, when
 to use them and in which order is a common theme for many questions on
 lists and |TX.SE|.

 It is quite common during the writing of a thesis or book
 for the author to keep on adding macros and packages
 at the preamble of the document. In most cases this can
 be satisfactory but in many others it leads to
 incompatibilities and errors. This package aims at
 minimizing one's preamble, by prefetching a number of
 commonly used packages. It also aims at loading them
 in the right order and providing patches for conflicts.
 
 I am hoping that using this package, will lead to less
 frustrations with the intricacies of \LaTeX2e\ packages.

The package code is complicated, but its usage is simple. You first load the package and then
you use one of the available templates:

 \begin{commands}[]{}
 \begin{verbatim}
 \usepackage{phd}
 \usetemplate{style13}
 \end{verbatim}
 \end{commands}

This is what you need to typeset a good looking book or thesis. The rest of this book is a footnote and you can skip them if you want. 

It will be better for the longer projects to just fork the
 package and adapt it to your needs. In this respect, I have
 uploaded the package to |github|.\footnote{\url{https://github.com/yannisl/phd}}

 My goal in selecting the packages and adding a number of 
 commands for the authors was to be able to typeset a 
 document for most common use cases, without the need of
 additional packages. The packages I selected are biased
 towards academic publications, although they can find use
 in almost any fields. The package provides a mechanism via
 PGF keys to provide a settings file. 
 
 Most of the documentation can be found in the implementation part.

Browse any books in a library or bookshop and the striking thing is that their design is very individualistic. They might have similarities but their main features vary. In many respects they resemble people's faces where minor differences have striking effects.

This package arose out of a question at stackexchange. How to redefine chapter heads. Having seen the popularity of the |pgf| package \cite{pkg-pgf} I realized that \latex users prefer this method of styling rather the traditional \latex method.

The user interface can be extended to basically all major packages. The principle is to keep to a minimum changes that can affect the LaTeX core commands. If there are any additions a key setting is provided to be able to revert back to normal LaTeX.

The workflow can be simplified. In addition I want to believe that the interface can provide a useful addition to the open source community and that other people will contribute style libraries, which will be simpler to write. It is also possible
to device an easy and uncomplicated web interface to handle
such a great number of variables.


Most people when they get started with \LaTeX\ will either use one of the standard classes such as the \docfile{book.cls} or one of the generic classes notably koma-script or memoir. Most students will be forced to use on of the many thesis classes available.

\section{The key value concept}

The key-value concept that originated with \LaTeX\ has been extended many times, the last and most serious implementation of it by Tantau in the PGF package. What essentially Tantau developed is a scripting language to script TeX code. The \tikzname and pgfplots packages are two major packaged that use keys effectively. Their popularity is growing and what this package does is to offer a user interface that has been modelled to be similar to that of \texttt{css} (cascade style sheets). 
\smallskip

\begin{scriptexample}{}{}
\textit{number} font-size = Large,\\
\textit{chapter} color = theblue
\end{scriptexample}
\smallskip

The main idea behind the package, is that you are configuring a document style by means of \emph{settings} rather than writing macros. In the example above the \emph{number, chapter} can be thought of as class or id names in css style sheets and the |font-size, color| as property settings that apply to the particular element. 


\subsection{Settings}

Settings are activated either by using the command |\cxset|  or by loading a full style sheet. In most cases you will probably import a style sheet and then modify some of the properties using |cxset|.  For example this heading has a dot after the subsection number. This was accomplished by setting,

We can de-activate it for the next and subsequent subsection headings with the setting:

\begin{scriptexample}{}{}
\begin{verbatim}
\cxset{subsection number after=\quad}
\end{verbatim}
\end{scriptexample}


\cxset{subsection number after=\quad,
          section number after=\quad,
          title margin-bottom=10pt}

\subsection{Cascading}

Most values once set for a higher section will be seen in a cascade by all subsectioning commands in a similar fashion similar to CSS. These include properties such as color, font families and alignment. Best though to specify all of them for maximum flexibility to your users.

\section{On typography}

This package hopefully will assist in improving the typography of books set with \latexe. Any typographical comments on the various styles are just my own ramblingss and not necessarily absolute truths. Like fashion and art typography has opinions rather than absolute truths. In many styles the design is slightly adapted to blend a bit better with this manual. Also I did not select fonts as per the samples but this is left on you the user to decide.



\section{Packages and Fonts}

This manual has been typeset with numerous fonts in order to enable the typsetting of almost all the scripts provided by the Unicode standard. In order to process it from the |.dtx| file, these fonts must be available in your system, otherwise \XeLaTeX\ will have a problem finding the fonts and it will take an awful long time to process. This is especially true for the scripts section, where virtually all the Unicode defined scripts are discussed. You will need a fast computer and a fast hard disk to process the document within a reasonable time. When using \pkgname{fontspec} always define your fonts with the \cmd{\newfontfamily} this will speed up processing by an order of magnitude. Compiling from the command prompt will speed up compilation. Average speed 2-3 pages per second.

Many of \tex's parameters are stretched to the limit with a complicated document such as this manual. You will require a full distribution otherwise expect some errors. Important packages is \pkgname{morefloats} and \pkgname{morewrites}. The package will also expect that you have |e-tex| installed. Ubuntu users are normally one year behind in updates, so you might wish to update manually. It will take upwards of 5 minutes to compile fully on an old laptop and a couple of minutes on a state of the art computer.

The |dtx| should be processed best with its own make file provided for Windows only |phd.bat|. The make file will process the documentation using \lualatex. You can also process the document with \xelatex but is prone to produce errors. Using \latexe the sections on scripts etc will not be printed and a much shorter version of the manual is provided. 

\section{Scripts and Languages}

The package and the documentation offer a full repertoire of font selection keys for different scripts and languages. It hasn't been possible, however hard I tried to compile this section of the documentation with \xelatex, as it kept giving errors of too many files open. This was also not possible even with the \pkgname{morewrites} package loaded. With \lualatex the document compiled with no major problems other than the font rendering being of a lower quality to that of XeLaTeX om windows, other than disabling incompatible packages and a number of commands that were redefined. 

Some good news for multi-script typesetting is the Noto fonts from Google. These fonts named Noto from "No Tofu" meaning you do not see any little square blocks for undefined glyphs, are fast to load. Disantvantage you need to switch between font commands fairly often.

\section{This book}

When developing the templates, I started using \emph{lorem ipsum} text as samples. Half-way through this
became a jumble mass of uninteresting pages interspersed with code. Headings and the contents of the book
determine both the structure and the selection of fonts, so I went back and wrote narratives  to accompany
the headings. Many of the narratives are semi-autobiographical in nature; others are clustered around books I read and my own interests. Some I stumbled on them accidentally and are mostly there to demonstrate some code.

Besides the templates and the code there is another narrative which is based on notes I kept on \tex and its friends over the years and are offered as a more advanced introduction to coding \latexe and \tex. The whole manual was typeset in a |ltxdoc| class, slightly modified to turn into a book class.

The implementation code is also available and it was mostly for my own benefit. The whole manual with the exception of the |\cxset| introduction, is just a test document. The notes and the “dissection” of the standard \latexe and the standard classes are there to explain the background to the many coding decisions that I took while I was developing the package.

PhD students are notorious for going in all directions and exploring many adjacent fields before they sit down and write their theses. Some become life-time students. To all these new men and women of the Renaissance that slave away to inch knowledge one thesis at a time, I dedicate this book and the name of the package.

\subsection{The TeX hacking sections}

To start programming \tex you need to have a knowldge of \tex basic commands and approach. \latex2015 is a format build on top of \tex to provide a more structured approach. To program \latexe packages you need to understand \latexe concepts, code organization and conventions. To program in \latex3, you need to learn a whole new language and you still need to understand \tex, \latexe and the expl3 language and conventions. To program using LuaTeX, other than the Lua language you need to understand \tex very well.
None of these can be found in one place.  I have gathered a lot of material and put it together. This is not a language you can master easily or quickly, but can teach you a lot about typesetting, computer science and many other interesting topics.


 \section{Version control with Git and Github}
 
 If you are involved with code or a publication that will have frequent changes, you should consider
 some type of version control system. My own recommendation is to use |git| and an online repository such
 as |github|. The latter is currently very fashionable and makes sharing code easier. Note that the |github|
 offers both public as well as private repositories. The general recommendation is that for unpublished work
 such as a thesis or code under development, it is preferable to go for a private repository. 
 

 \section{Ordering of Packages}
 
One package that normally leads to errors is the 
\pkgname{hyperref}. The package which is an outstanding example of software engineering and supported single handledy by Heiko Oberdiek \citeyearpar{hyperref} redefines a a lot of internal commands of the kernel. As a lot of other packages do the same it has to be loaded at the end of the preable with the exception of some packages! 
 
 This manual is typeset according to the conventions of the
 \LaTeX \textsc{docstrip} utility which enables the automatic
 extraction of the \LaTeX{} macro source files~\cite{GOOSSENS94}.

 
 \href{http://tex.stackexchange.com/questions/96350/problem-with-algorithmic-and-hyperref}{problem with algorithmic and hyperref}

 \begin{verbatim}
\usepackage{float}  % load float package first!

\usepackage{hyperref} % let hyperref patch the float package stuff
.
 \usepackage{algorithm} % let algorithm use the patched version of the float package
 \end{verbatim}
 

\section{Known problems}

Perhaps the biggest issue with the package is the speed of
compilation with \XeLaTeX\ or \LuaTeX. This is to be expected, as both engines spend a lot of resources in font management. On demand loading of packages is something I have in the back of my mind. This should be done via document styles i.e., if a book is for the humanities, perhaps only a rudimentary amount of maths packages should be loaded.

\section{Future Directions}

\latexe and \tex usage appears to be increasing. This is mostly by programs that export results with \latexe code rather than authors writing books.  The method adopted here is easier to automate all sorts of reports and automated texts. I would like too develop a web interface for processing such templates and at the same time export into html instead of just producing pdfs. I have already a prototype.   

%\ClockFramefalse\ClockStyle=0\clock{13}{10}
%\ClockFramefalse\ClockStyle=1\clock{14}{22}
%\ClockFramefalse\ClockStyle=2\clock{15}{48}
%\ClockFramefalse\ClockStyle=3\clock{7}{50}
%
%\ClockFrametrue\ClockStyle=0\clock{11}{32}
%\ClockFrametrue\ClockStyle=1\clock{12}{0}
%\ClockFrametrue\ClockStyle=2\clock{8}{9}
%\ClockFrametrue\ClockStyle=3\clock{1}{15}

%{\HHHUGE\showclock{0}{45}}










%  
\cxset{epigraph rule color=spot,epigraph width=0.7\textwidth}

\chapter{Epigraphs}\index{epigraphs}
\label{c:epigraphs}

\epigraph{Please give examples of good use of epigraphs in fiction.

I mean them quoted dealies they sometimes put at the start of chapters.

What counts as ``good use" is whatever you think counts. Part of my goal is to understand what people like about these things.
.}{\href{http://ask.metafilter.com/207423/Good-use-of-epigraphs-in-fiction}{Stebulus}}



\section{Introduction}

Epigraphs or quotations before or after chapters are quite common in books. Peter Wilson's epigraph package \citep{epigraph}, 
does a good job and we have adapted it where necessary to allow for a key value interface. The command:

\cs{epigraph}\marg{text}\marg{source}. By default the epigraph is placed at the right
hand side of the textblock, and the \marg{source} is typeset at the bottom right of the \marg{text}. 
Numerous settings allow for manipulating the width of the epigraph, the location and other 
variables. If the package is available we use it otherwise we use other internal commands.

All key values for epigraphs, start with the keyword \emph{epigraph}. You can think of the epigraph of a block of text that can go anywhere on a page and has some formatting rules that are set 

\section{Key-value interface}
The key value interface provided by the package is shown below. It mostly follows the 
naming conventions of the epigraph package to make the transition easier for experienced users. Use any dimension or a dimension expression.
\medskip

\begin{docKey}[phd]{epigraph width}{ = \marg{dim} }{no default, initial=0.6\cs{textwidth}}
  Sets the width of the epigraph block. 
\end{docKey}



\begin{docKey}[phd]{epigraph align}{ = \marg{left\textbar center\textbar right}}{no default, initial=right}
 A font-size command such as \cs{footnotesize}, 
\cs{small} and other similar commands. This will align the full block containing the epigraph, left right or center according to the setting of the key. Most epigraphs are aligned right.
\end{docKey}

\begin{texexample}{Setting epigraph widths}{ep:align}
% set properties
 \cxset{epigraph align=left, 
           epigraph width=300pt}
% write the epigraph           
 \epigraph{Example is the school of mankind, and they
   will learn at no other.}{unknown}
\end{texexample}

\begin{docKey}[phd]{epigraph rule width}{ = \marg{dim}} {no default, initial=0.4pt }
 The width of the rule separating the epigraph from the source. Set to 0pt,if you do not want a rule.
\end{docKey}



\begin{docKey}[phd]{ epigraph font-size}{ = \marg{font sizing cmd}} {no default,}
Use a font sizing command such as \cmd{\footnotesize}
\end{docKey}

\begin{docKey}[phd]{ epigraph beforeskip}{ = \marg{dim}}{no default, }
Space before the epigraph.
\end{docKey}

\begin{docKey}[phd]{ epigraph afterskip}{ = \marg{dim}}{no default, }
Space after the epigraph.
\end{docKey}

\subsection{Styling the source part}

\begin{docKey}[phd]{ epigraph source align =}{ \marg{left\textbar center\textbar right}}{no default, }
Align the source text to the right, left or center.
\end{docKey}

\begin{docKey}[phd]{ epigraph source}{ font-size=\marg{dim}}{no default, }
Align the source text to the right, left or center.
\end{docKey}

\begin{docKey}[phd]{ epigraph source font-shape}{ = \marg{dim}}{no default, }
Align the source text to the right, left or center.
\end{docKey}

\begin{docKey}[phd]{ epigraph source font-family}{ = \marg{dim}}{no default, }
Align the source text to the right, left or center.
\end{docKey}


\begin{docKey}[phd]{epigraph source font-weight}{ = \marg{bold,normal}}{no default, }
Align the source text to the right, left or center.
\end{docKey}


Usage examples can be found in relevant style examples (See Chapter~\ref{ch:41}) for a rather 
nice example with non-traditional alignment.

\section{Epigraphs on empty pages}

When a chapter open on an odd page sometimes the  previous page is left empty. Some book designers 
add the words ``this page left intentionally blank'' and other might add a quote. To add such a quote use:

%\begin{verbatim}
%\cxset{blank page text=\epigraph{The great tragedy of science is 
%                                   the slaying of a beautiful theory
%                                  by an ugly fact.}{Thomas Huxley}}
%\end{verbatim}

\endinput


 
  \DocInput{\jobname.dtx}%
  \nocite{*}
  \printbibliography
  \printindex
\end{document}
%</driver>
% \fi
%
% \CheckSum{53}
% \CharacterTable
%  {Upper-case    \A\B\C\D\E\F\G\H\I\J\K\L\M\N\O\P\Q\R\S\T\U\V\W\X\Y\Z
%   Lower-case    \a\b\c\d\e\f\g\h\i\j\k\l\m\n\o\p\q\r\s\t\u\v\w\x\y\z
%   Digits        \0\1\2\3\4\5\6\7\8\9
%   Exclamation   \!     Double quote  \"     Hash (number) \#
%   Dollar        \$     Percent       \%     Ampersand     \&
%   Acute accent  \'     Left paren    \(     Right paren   \)
%   Asterisk      \*     Plus          \+     Comma         \,
%   Minus         \-     Point         \.     Solidus       \/
%   Colon         \:     Semicolon     \;     Less than     \<
%   Equals        \=     Greater than  \>     Question mark \?
%   Commercial at \@     Left bracket  \[     Backslash     \\
%   Right bracket \]     Circumflex    \^     Underscore    \_
%   Grave accent  \`     Left brace    \{     Vertical bar  \|
%   Right brace   \}     Tilde         \~}
%
%
% \changes{1.0}{2011/05/03}{Converted to DTX file}
%
% \DoNotIndex{\newcommand,\newenvironment}
%
% \GetFileInfo{phd-epigraphs.dtx}
%  \def\fileversion{v1.0}          
%  \def\filedate{Typeset \today}
% \title{The \textsf{\jobname} package.
% \author{Dr. Yiannis Lazarides \\ \url{yannislaz@gmail.com}}
% \thanks{This
%        file (\texttt{\jobname.dtx}) has version number 
%        \fileversion, last revised
%        \filedate.}
% }
% 
% \date{\filedate}
%
%
% \maketitle
% 
%
% \chapter{Implementation}
%  This manual is typeset according to the conventions of the
% \LaTeX{} \textsc{docstrip} utility which enables the automatic
% extraction of the \LaTeX{} macro source files~\cite{GOOSSENS94}.
% \pagestyle{headings}
% 
%  ^^A\OnlyDescription
%
% ^^A\StopEventually{}
%
%<*EPI>
% \section{Preliminary}
%
%    \begin{macrocode}
\NeedsTeXFormat{LaTeX2e}
\ProvidesPackage{phd-epigraphs}%
  [2015/13/06 v1.0 epigraph styling]%
%    \end{macrocode}
%
% \section{Epigraphs}
%
% This section deals with epigraphs.\index{epigraph}\index{epigraph>rule}
% We first get the \pkgname{epigraph}. As the memoir class defines the epigraph 
% we first check if it has been defined and bale out of the package.
%    \begin{macrocode}

\@ifundefined{epigraph}
  {%
    \RequirePackage{epigraph}
   %% Set up the epigraph to be a bit wider
    \setlength{\epigraphwidth}{8cm} 
    \setlength{\epigraphrule}{0pt}
    \newcommand{\theepigraph}[2]{\epigraphhead[30]{\epigraph{#1}{\textit{#2}}}}
  }
  {%for memoir
   \setlength{\epigraphwidth}{8cm} 
   \setlength{\epigraphrule}{0pt}
   \newcommand{\theepigraph}[2]{\epigraphhead[30]{\epigraph{#1}{\textit{#2}}}}%
  }

%    \end{macrocode}
%
%    \begin{macrocode}
\cxset{
  epigraph width/.code               = {\setlength\epigraphwidth{#1}},
  epigraph font-size/.code           = {\renewcommand{\epigraphsize}{#1}},
  epigraph beforeskip/.code          = {\setlength\beforeepigraphskip{#1}},
  epigraph afterskip/.code           = {\setlength\afterepigraphskip{#1}},
  epigraph align/.is choice,
  epigraph align/center/.code        = {\renewcommand{\epigraphflush}{center}},
  epigraph align/left/.code          = {\renewcommand{\epigraphflush}{flushleft}},
  epigraph align/right/.code         = {\renewcommand{\epigraphflush}{flushright}},
  epigraph source align/.is choice,
  epigraph source align/left/.code   = {\renewcommand{\sourceflush}{flushleft}},
  epigraph source align/right/.code  = {\renewcommand{\sourceflush}{flushright}},
  epigraph source align/center/.code = {\renewcommand{\sourceflush}{center}},
  epigraph text align/.is choice,
  epigraph text align/left/.code     = {\renewcommand{\textflush}{flushleft}},
  epigraph text align/right/.code    = {\renewcommand{\textflush}{flushright}},
  epigraph text align/center/.code   = {\renewcommand{\textflush}{center}},
  epigraph rule width/.code          = {\setlength\epigraphrule{#1}},
  epigraph rule color/.store in      = \epigraphrulecolor@cx,
  epigraph rule/.code={
 \renewcommand{\@epirule}{
 \color{\epigraphrulecolor@cx}\rule[.5ex]{\epigraphwidth}{\epigraphrule}}
},
}

\cxset{epigraph width=0.7\linewidth,
    epigraph font-size=\small,
    epigraph rule width=0.4pt,
    epigraph align=right,
    epigraph source align=right,
    epigraph text align=right,
    epigraph rule color=black,
    epigraph rule}
%    \end{macrocode}
% 
% \Finale
%</EPI>
%
%
%
\endinput
% \bibliographystyle{alpha}
% \begingroup
% \raggedright
%
% \begin{thebibliography}{GMSN94A}
%
% \bibitem[GMS94]{GOOSSENS94}
% Michel Goossens, Frank Mittelbach, and Alexander Samarin.
% \newblock {\em The LaTeX Companion}.
% \newblock Addison-Wesley Publishing Company, 1994.
%\endgroup
% \PrintIndex
%
% \end{thebibliography}

